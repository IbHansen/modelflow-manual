\documentclass{article}
\usepackage{booktabs}
\usepackage{caption} % Include the caption package
\captionsetup{justification=raggedright,singlelinecheck=false}
\usepackage{graphicx}
\usepackage{pgf}
\usepackage{lscape}
\begin{document}


{\setlength{\parskip}{1em}
 \setlength{\parindent}{0pt}
  
# A Demography Model - Detailed Explanation

This model describes the evolution of a population over time, divided by age groups and sex.
It captures the fundamental demographic dynamics - births, deaths, and migration - which together determine 
how the total population changes from one period to the next.

---

## 1. List Definitions

>list ages = ages : age_0 * age_11  
>list sexes = sexes   : female male /  
>             fertile :     1     0  

These lists define the model’s structural dimensions:

- "ages" enumerates age groups, from age_0 (newborns) up to age_11 (the oldest group).  
- "sexes" distinguishes between female and male.  
- The "fertile" flag marks whether a sex is fertile (1 for females, 0 for males), which determines participation in the fertility equations.

---

## 2. Population Dynamics

The population is tracked at various stages within each period.

### a. Primo Population

>doable [ages ages_nostart]  pop_primo__a = pop__ab(-1)

For all age groups except the youngest (`ages_nostart` excludes the start group),
the beginning-of-period (primo) population in an age group is equal to the end-of-period (ultimo) population of the previous age group in 
the previous period.
This represents aging: people move from one age group to the next as time progresses.

### b. Population Before Death

For the youngest age group:

>doable [ages ages_start]   pop_before_death__a   = birth__total__{sexes} + migration__a

The youngest group’s population before death depends on births of that sex in the current period and net migration for that age group.
This reflects that new members of the population come from births and migration, since there is no prior age group for newborns.

For all other age groups:

>doable [ages ages_nostart] pop_before_death__a   = pop_primo__a + migration__a

Older groups’ populations before death are determined by the primo population (those who aged into the group) and migration inflows or outflows.

---

## 3. Mortality and Deaths

>doable  [ages ages_end ]  death_rate__a =1.0  
>doable  <sum=total> death__a = pop_before_death__a  * death_rate__a

For the last age group, the death rate is set to 1.0, implying that everyone in this group dies within the period - 
a closure condition preventing population overflow beyond the oldest cohort.
Deaths in all groups are computed as the product of the population before death and the age-specific death rate.

## 4. End-of-Period (Ultimo) Population

>doable  <sum=total> pop__a  = pop_before_death__a  - death__a

The end-of-period population equals the population before death minus deaths.
This is the surviving population that will age into the next group in the following period.


## 5. Births and Fertility

>doable <sum=total> [ages ages_nostart,  sexes fertile] birth__a = pop_primo__a * fertility__a

Births are generated by fertile females in each age group.
Fertility rates (fertility__a) vary by age, and are in the dataframe set to zero outside reproductive ages.
The total number of births is thus the sum across fertile age groups.

---

## 6. Sex Ratio at Birth

>birth__total__female = FRAC_BIRTH__female * birth_total  
>birth__total__male = (1 - FRAC_BIRTH__female) * birth_total

The total births are split into female and male newborns using the fraction `FRAC_BIRTH__female`.
This ensures a biologically realistic sex distribution at birth.

---

## 7. Totals and Consistency Check

>migration__total = sum(ages,sum(sexes,migration__a))  
migration__total aggregates migration flows over all ages and sexes.

>pop__total_check  = (pop__total(-1) + birth__total + migration__total - death__total) - pop__total

pop__total_check serves as a consistency test for the model:
The change in total population should equal births + migration – deaths.
Any deviation indicates potential modeling or data inconsistencies.

---

## Summary

This demographic model captures the fundamental accounting identities of population dynamics:

Population<sub>t</sub><sup>age</sup> = Population<sub>t-1</sub><sup>age-1</sup> + Births<sub>t</sub><sup>age</sup> + Migration<sub>t</sub><sup>age</sup> - Deaths<sub>t</sub><sup>age</sup>

For the youngest cohort (age = 0), there is no inflow from age − 1,  
so the equation simplifies to:

Population<sub>t</sub><sup>0</sup> = Births<sub>t</sub> + Migration<sub>t</sub><sup>0</sup> - Deaths<sub>t</sub><sup>0</sup>

It distinguishes individuals by age and sex, allowing for realistic aging, fertility, and mortality processes.
The model is modular and extendable - one could add features such as:
- Age-dependent migration patterns
- Variable mortality shocks (for example, pandemics)
- Policy impacts on fertility rates
- Regional or socioeconomic subpopulations

   }
\end{document}
        