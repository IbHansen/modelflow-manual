%% Generated by Sphinx.
\def\sphinxdocclass{jupyterBook}
\documentclass[letterpaper,10pt,english]{jupyterBook}
\ifdefined\pdfpxdimen
   \let\sphinxpxdimen\pdfpxdimen\else\newdimen\sphinxpxdimen
\fi \sphinxpxdimen=.75bp\relax
\ifdefined\pdfimageresolution
    \pdfimageresolution= \numexpr \dimexpr1in\relax/\sphinxpxdimen\relax
\fi
%% let collapsible pdf bookmarks panel have high depth per default
\PassOptionsToPackage{bookmarksdepth=5}{hyperref}
%% turn off hyperref patch of \index as sphinx.xdy xindy module takes care of
%% suitable \hyperpage mark-up, working around hyperref-xindy incompatibility
\PassOptionsToPackage{hyperindex=false}{hyperref}
%% memoir class requires extra handling
\makeatletter\@ifclassloaded{memoir}
{\ifdefined\memhyperindexfalse\memhyperindexfalse\fi}{}\makeatother

\PassOptionsToPackage{warn}{textcomp}

\catcode`^^^^00a0\active\protected\def^^^^00a0{\leavevmode\nobreak\ }
\usepackage{cmap}
\usepackage{fontspec}
\defaultfontfeatures[\rmfamily,\sffamily,\ttfamily]{}
\usepackage{amsmath,amssymb,amstext}
\usepackage{polyglossia}
\setmainlanguage{english}



\setmainfont{FreeSerif}[
  Extension      = .otf,
  UprightFont    = *,
  ItalicFont     = *Italic,
  BoldFont       = *Bold,
  BoldItalicFont = *BoldItalic
]
\setsansfont{FreeSans}[
  Extension      = .otf,
  UprightFont    = *,
  ItalicFont     = *Oblique,
  BoldFont       = *Bold,
  BoldItalicFont = *BoldOblique,
]
\setmonofont{FreeMono}[
  Extension      = .otf,
  UprightFont    = *,
  ItalicFont     = *Oblique,
  BoldFont       = *Bold,
  BoldItalicFont = *BoldOblique,
]



\usepackage[Bjarne]{fncychap}
\usepackage[,numfigreset=1,mathnumfig]{sphinx}

\fvset{fontsize=\small}
\usepackage{geometry}


% Include hyperref last.
\usepackage{hyperref}
% Fix anchor placement for figures with captions.
\usepackage{hypcap}% it must be loaded after hyperref.
% Set up styles of URL: it should be placed after hyperref.
\urlstyle{same}

\addto\captionsenglish{\renewcommand{\contentsname}{Introduction}}

\usepackage{sphinxmessages}



        % Start of preamble defined in sphinx-jupyterbook-latex %
         \usepackage[Latin,Greek]{ucharclasses}
        \usepackage{unicode-math}
        % fixing title of the toc
        \addto\captionsenglish{\renewcommand{\contentsname}{Contents}}
        \hypersetup{
            pdfencoding=auto,
            psdextra
        }
        % End of preamble defined in sphinx-jupyterbook-latex %
        

\title{MFMod models in Python with ModelFlow}
\date{Apr 02, 2023}
\release{}
\author{Andrew Burns and Ib Hansen}
\newcommand{\sphinxlogo}{\vbox{}}
\renewcommand{\releasename}{}
\makeindex
\begin{document}

\pagestyle{empty}
\sphinxmaketitle
\pagestyle{plain}
\sphinxtableofcontents
\pagestyle{normal}
\phantomsection\label{\detokenize{introduction::doc}}


\sphinxAtStartPar
Andrew Burns and Ib hansen

\sphinxstepscope


\part{Introduction}

\sphinxstepscope


\chapter{Introduction}
\label{\detokenize{content/01_Introduction/Introduction:introduction}}\label{\detokenize{content/01_Introduction/Introduction::doc}}
\begin{sphinxadmonition}{warning}{Warning:}
\sphinxAtStartPar
This Jupyter Book is work in progress.
\end{sphinxadmonition}

\sphinxAtStartPar
This paper describes the implementation of the World Bank’s MacroFiscalModel (MFMod) {[}\hyperlink{cite.content/litterature:id15}{BCJ+19}{]} in the open source solution program ModelFlow (\sphinxhref{https://ibhansen.github.io/doc/index}{Hansen, 2023}).


\section{Background}
\label{\detokenize{content/01_Introduction/Introduction:background}}
\sphinxAtStartPar
The impetus for this paper and the work that it summarizes was to make available to a wider constituency the work that the Bank has done over the past several decades to disseminate Macro\sphinxhyphen{}structural models%
\begin{footnote}[1]\sphinxAtStartFootnote
Economic modelling has a long tradition at the World Bank.  The Bank has had a long\sphinxhyphen{}standing involvement in CGE modeling is the World Bank {[}\hyperlink{cite.content/litterature:id19}{DJ13}{]}, indeed the popular mathematics package GAMS, which is widely used to solve CGE and Linear Programming models, \sphinxhref{https://www.gams.com/about/company/}{started out} as a project begun at the World Bank in the 1970s.
%
\end{footnote} – notably those that form part of its MFMod (MacroFiscalModel) framework.

\sphinxAtStartPar
MFMod is the World Bank’s work\sphinxhyphen{}horse macro\sphinxhyphen{}structural economic modelling framework. It exists as a linked system of 184 country specific models that can be solved either independently or as a larger system. The MFMod system replaced the Bank’s RMSIM\sphinxhyphen{}X model \{cite\}p:\sphinxcode{\sphinxupquote{addison\_world\_2019}} \sphinxstylestrong{not in the bib file}  and evolved from earlier macro\sphinxhyphen{}structural models developed by the Bank during the 2000s to strengthen the basis for the forecasts produced by the World Bank.

\sphinxAtStartPar
Some examples of these models were released on the World Bank’s isimulate platform early in 2010 along with several CGE models dating from this period.

\sphinxAtStartPar
Beginning in 2015, the core model that was developed for the isimulate platform was developed and extended substantially into the main MFMod (MacroFiscalModel) model that is used today for the World Bank’s twice annual forecasting exercise \sphinxhref{https://www.worldbank.org/en/publication/macro-poverty-outlook}{The Macro Poverty Outlook}.  This model continues to evolve and be used as the workhorse forecasting and policy simulation model of the World Bank.


\section{Early steps to bring the MFMod system to the broader economics community}
\label{\detokenize{content/01_Introduction/Introduction:early-steps-to-bring-the-mfmod-system-to-the-broader-economics-community}}
\sphinxAtStartPar
Bank staff were quick to recognize that the models built for its own needs could be of use to the broader economics community. An initial project \sphinxcode{\sphinxupquote{isimulate}} in 2007 made several versions of this earlier model available for simulation on the \sphinxhref{https://isimulate.worldbank.org}{isimulate platform}, and these models continue to be available there.  The isimulate platform housed (and continues to house) public access to earlier versions of the MFMod system, and allows simulation of these and other models – but does not give researchers access to the code or the ability to construct complex simulations.

\sphinxAtStartPar
In another effort to make models widely available a large number (more than 60 as of June 2023) customized stand\sphinxhyphen{}alone models (collectively known as called MFModSA \sphinxhyphen{} MacroFiscalModel StandAlones)  have been developed from the main model. Typically developed for a country\sphinxhyphen{}client (Ministry of Finance, Economy or Planning or Central Bank), these Stand Alones extend the standard model by incorporating additional details not in the standard model that are of specific import to different economies and the country\sphinxhyphen{}clients for whom they were built, including: a more detailed breakdown of the sectoral make up of an economy, more detailed fiscal and monetary accounts, and other economically important features of the economy that may exist only inside the aggregates of the standard model.

\sphinxAtStartPar
Training and dissemination around these customized versions of MFMod have been ongoing since 2013. In addition to making customized models available to client governments, Bank teams have run technical assistance program designed to train government officials in the use of these models and their maintenance, modification and revision.


\subsection{Climate change and the MFMod system}
\label{\detokenize{content/01_Introduction/Introduction:climate-change-and-the-mfmod-system}}
\sphinxAtStartPar
Most recently, the Bank has extended the standard MFMod framework to incorporate the main features of climate change ({[}\hyperlink{cite.content/litterature:id14}{BJS21a}{]})– both in terms of the impact of the economy on climate (principally through green\sphinxhyphen{}house gas emissions, like \(CO_2, N_{2}O, CH_4, ...\)) and the impact of the changing climate on the economy (higher temperatures, changes in rainfall quantity and variability, increased incidence of extreme weather) and their impacts on the economy (agricultural output, labor productivity, physical damages due to extreme weather events, sea\sphinxhyphen{}level rises etc.).

\sphinxAtStartPar
These climate enhanced versions of MFMod serve as one of the two main modelling systems (along with the Bank’s MANAGE CGE system) in the World Bank’s \sphinxhref{https://www.worldbank.org/en/publication/country-climate-development-reports}{Country Climate Development Reports}


\section{Moving the framework to an open\sphinxhyphen{}source footing}
\label{\detokenize{content/01_Introduction/Introduction:moving-the-framework-to-an-open-source-footing}}
\sphinxAtStartPar
Models in the MFMod family are normally built using the proprietary \sphinxhref{https://www.eviews.com}{EViews} econometric and modelling package. While offering many advantages for model development and maintenance, its cost may be a barrier to clients in developing countries.  As a result, the World Bank joined with Ib Hansen, a Danish economist formerly with the European Central Bank and the Danish Central Bank, who over the years has developed \sphinxcode{\sphinxupquote{modelflow}} a generalized solution engine written in Python for economic models. Together with World Bank, Hansen has worked to extend \sphinxcode{\sphinxupquote{modelflow}} so that MFMod models can be ported and run in the framework.

\sphinxAtStartPar
This paper reports on the results of these efforts. In particular, it provides step by step instructions on how to install the \sphinxcode{\sphinxupquote{modelflow}} framework, import a World Bank macrostructural model,  perform simulations with that model and report results using the many analytical and reporting tools that have been built into \sphinxcode{\sphinxupquote{modelflow}}.  It is not a manual for \sphinxcode{\sphinxupquote{modelflow}}, such a manual can be found \sphinxhref{https://ibhansen.github.io/doc/index}{here} nor is it documentation for the MFMod system, such documentation can be found here \hyperlink{cite.content/litterature:id15}{Burns \sphinxstyleemphasis{et al.}} and here {[}\hyperlink{cite.content/litterature:id18}{BJS21b}{]}, {[}\hyperlink{cite.content/litterature:id14}{BJS21a}{]}). Nor is it documentation for the specific models described and worked with below.


\bigskip\hrule\bigskip


\sphinxstepscope


\part{Macrostructural Models}

\sphinxstepscope


\chapter{Macrostructural models}
\label{\detokenize{content/02_MacrostructuralModels/MacroStructuralModels:macrostructural-models}}\label{\detokenize{content/02_MacrostructuralModels/MacroStructuralModels::doc}}
\sphinxAtStartPar
The economics profession uses a wide range of models for different purposes.  Macro\sphinxhyphen{}structural models (also known as semi\sphinxhyphen{}structural or Macro\sphinxhyphen{}econometric models) are a class of models that seek to summarize the most important interconnections and determinants of an economy. Computable General Equilibrium (CGE), and Dynamic Stochastic General Equilibrium (DSGE) models are other classes of models that also seek, using somewhat different methodologies, to capture the main economic channels by which the actions of agents (firms, households, governments) interact and help determine the structure, level and rate of growth of economic activity in an economy. Olivier Blanchard, former Chief Economist at the International Monetary Fund, in a series of articles published between 2016 and 2018 that were summarized in {[}\hyperlink{cite.content/litterature:id17}{Bla18}{]}. In these articles he lays out his views on the relative strengths and weaknesses of each of these systems, concluding that each has a role to play in helping economists analyze the macro\sphinxhyphen{}economy.

\sphinxAtStartPar
Typically organizations, including the World Bank, use all of these tools, privileging one or the other for specific purposes. Macrostructural models like the MFMod framework are widely used by Central Banks, Ministries of Finance; and professional forecasters both for the purposes of generating forecasts and policy analysis.


\section{A system of equations}
\label{\detokenize{content/02_MacrostructuralModels/MacroStructuralModels:a-system-of-equations}}
\sphinxAtStartPar
Macro\sphinxhyphen{}structural models are a system of equations comprised of two kinds of equations and three kinds of variables.
\begin{itemize}
\item {} 
\sphinxAtStartPar
\sphinxcode{\sphinxupquote{Identities}} are variables that are determined by a well defined accounting rule that always holds. The famous GDP Identity Y=C+I+G+(X\sphinxhyphen{}M) is one such identity, that indicates that GDP at market prices is definitionally equal to Consumption plus Investment plus Government spending plus Exports less Imports.

\item {} 
\sphinxAtStartPar
\sphinxcode{\sphinxupquote{Behavioural}} variables are determined by equations that typically attempt to summarize an economic (vs accounting) relationship. Thus, the equation that says real C = f(Disposable Income,the price level, and animal spirits) is a behavioural equation – where the relationship is drawn from economic theory. Because these equations do not fully explain the variation in the dependent variable and the sensitivities of variables to the changes in other variables are uncertain, these equations and their parameters are  typically estimated econometrically and are subject to error.

\item {} 
\sphinxAtStartPar
\sphinxcode{\sphinxupquote{Exogenous}} variables are not determined by the model. Typically there are set either by assumption or from data external to the model.  For an individual country model, would often be set as an exogenous variable because the level of activity of the economy itself is unlikely to affect the world price of oil.

\end{itemize}

\sphinxAtStartPar
In a fully general form it can be written as:
\begin{align*}
y_t^1  &=  f^1(y_{t+u}^1...,y_{t+u}^n...,y_t^2...,y_{t}^n...y_{t-r}^1...,y_{t-r}^n,x_t^1...x_{t}^k,...x_{t-s}^1...,x_{t-s}^k) \\
y_t^2  &=  f^2(y_{t+u}^1...,y_{t+u}^n...,y_t^1...,y_{t}^n...y_{t-r}^1...,y_{t-r}^n,x_t^1...x_{t}^k,...x_{t-s}^1...,x_{t-s}^k) \\
\vdots \\
y_t^n  &=  f^n(y_{t+u}^1...,y_{t+u}^n...,y_t^1...,y_{t}^{n-1}...y_{t-r}^1...,y_{t-r}^n,x_t^1...x_{t}^r,x..._{t-s}^1...,x_{t-s}^k)
\end{align*}
\sphinxAtStartPar
where \( y_t^1 \) is one of n endogenous variables and \(x_t^1\) is an exogenous variable and there are as many equations as there are unknown (endogenous variables).

\sphinxAtStartPar
Rewritten for our GDP identity and substituting the variable mnemonics Y,C,I,G,X,M we could write a simple model as a system of 6 equations in 6 unknowns:
\begin{align*}
Y_t  &=  C_t+I_t+G+t+ (X_t-M_t) \\
C_t &= c_t(C_{t-1},C_{t-2},I_t,G_t,X_t,M_t,P_t)\\
I_t &= c_t(I_{t-1},I_{t-2},C_t,G_t,X_t,M_t,P_t)\\
G_t &= c_t(G_{t-1},G_{t-2},C_t,I_t,X_t,M_t,P_t)\\
X_t &= c_t(X_{t-1},X_{t-2},C_t,I_t,G_t,M_t,P_t,P^f_t)\\
M_t &= c_t(M_{t-1},M_{t-2},C_t,I_t,G_t,X_t,P_t,P^f_t)
\end{align*}
\sphinxAtStartPar
and where \(P_t, P^f_t\) domestic and foreign prices respectively are exogenous in this simple model.


\section{Behavioural equations}
\label{\detokenize{content/02_MacrostructuralModels/MacroStructuralModels:behavioural-equations}}
\sphinxAtStartPar
Behavioural equations in a macrostructural equation are typically estimated. In MFMod they are often expressed in Error Correction form. In this approach the behaviour of the dependent variable (say Consumption) is assumed to be the joint product of a long\sphinxhyphen{}term economic relationship – usually drawn from economic theory, and various short\sphinxhyphen{}run influences which can be more ad hoc in nature. The ECM formulation has the advantage of tieing down the long run behavior of the economy to economic theory, while allowing its short\sphinxhyphen{}run dynamics (where short\sphinxhyphen{}run can in some cases be 5 or more years) to reflect the way the economy actually operates (not how textbooks say it should behave).

\sphinxAtStartPar
For the consumption equation, utility maximization subject to a budget constraint might lead us to define a long run relationship like this economic theory might lead us to something like this:
\begin{equation*}
\begin{split} C_t = \alpha + \beta_1{{rK_t + WB_{t} + GTR_{t}(1-\tau^{Direct})}\over {PC_t}}-\beta_3(r_t-\dot{p}_t) +\eta_t\end{split}
\end{equation*}
\sphinxAtStartPar
Where in the long run consumption (\(C_t\)) for a given interest rate is a stable share of real disposable income (\({{rK_t + WB_{t} + GTR_{t}}\over {PC_t}})\), implying a constant savings rate. And where real disposable income is given by interest earned on capital (\(rK_t\)) plus earnings from labour (\(WB_{t}\)) + Government transfers to households (\(GTR_{t}\)) multiplied by 1 less the direct  rate (\(\tau^{Direct}\)).  The final term reflects the influence of real interest rates on final consumption, such that as real interest rates rise consumption as a share of disposable income declines (the savings rate rises).

\sphinxAtStartPar
Replacing the expression following \(\beta\) with \(Y^{disp}_t\), the above simplifies and can be rewritten as:
\begin{equation*}
\begin{split} C_t= (\alpha + \beta_1{Y^{disp}_t}-\beta_3(r_t-\dot{p}_t))\end{split}
\end{equation*}
\sphinxAtStartPar
and dividing both sides by \(Y^{disp}_t\) gives:
\begin{equation*}
\begin{split}\frac{C_t}{Y^{disp}_t} = \beta_1 -\beta_3\frac{r_t-\dot{p}_t}{Y^{disp}_t}\end{split}
\end{equation*}
\sphinxAtStartPar
or in logarithms
\begin{equation*}
\begin{split}{c_{t-1}}-{y^{disp}_{t-1}} - ln(\beta_1) +\beta_3 ln(r_{t-1}-\dot{p}_{t-1} -{y^{disp}_{t-1}})=0\end{split}
\end{equation*}
\sphinxAtStartPar
we can then write our ECM equation as
\begin{equation*}
\begin{split} \Delta c_t = -\lambda(\eta_{t-1})+ SR_t \end{split}
\end{equation*}
\sphinxAtStartPar
Substituting the LR expression for the error term in t\sphinxhyphen{}1 we get
\begin{equation*}
\begin{split} \Delta c_t = -\lambda({c_{t-1}}-{y^{disp}_{t-1}} - ln(\beta_1) +\beta_3 ln(r_{t-1}-\dot{p}_{t-1} -{y^{disp}_{t-1}}))+ \beta_{SR1}{y^{disp}_{t}} - \beta_{SR2}ln(r_{t}-\dot{p}_{t} -{y^{disp}_{t}})  \end{split}
\end{equation*}
\sphinxAtStartPar
where \(\beta_{SR1}\) is the short run elasticity of consumption to disposable income; \(\beta_{SR2}\) is the short run real interest rate elasticity of consumption and \(\lambda\) is the speed of adjustment (the rate at which past errors are corrected in each period).

\sphinxAtStartPar
\hyperlink{cite.content/litterature:id15}{Burns \sphinxstyleemphasis{et al.}} provides more complete derivations of the functional forms for most of the behavioural equations in MFmod.

\sphinxstepscope


\chapter{Modelflow and the MFMod models of the World Bank}
\label{\detokenize{content/02_MacrostructuralModels/MFModAndModelFlow:modelflow-and-the-mfmod-models-of-the-world-bank}}\label{\detokenize{content/02_MacrostructuralModels/MFModAndModelFlow::doc}}
\sphinxAtStartPar
At the World Bank models built using the MFMod framework are developed in \sphinxhref{https://eviews.com}{EViews}. When disseminated to clients, the models are operated in a World Bank customized EViews environment. But as a systems of equations and associated data the models can be solved, and operated under any system capable of solving a system of simultaneous equations – as long as the equations and data can be transferred from EViews to the secondary system. \sphinxcode{\sphinxupquote{Modelflow}} is such a system and offers a wide range of features that permit not only solving the model, but also provide a rich and powerful suite of tools for analyzing the model and reporting results.


\section{A brief history of ModelFlow}
\label{\detokenize{content/02_MacrostructuralModels/MFModAndModelFlow:a-brief-history-of-modelflow}}
\sphinxAtStartPar
Modelflow is a python library that was developed by Ib Hansen over several years while working at the Danish Central Bank and the European Central Bank. The framework has been used both to port the U.S. Federal Reserve’s macro\sphinxhyphen{}structural  model to python, but also been used to bring several stress\sphinxhyphen{}testing models developed by European Central Banks and the European Central Bank into a python environment.

\sphinxAtStartPar
Beginning in 2019, Hansen has worked with the World Bank to develop additional features that facilitate working with models built using the Bank’s MFMod Framework, with the objective of creating an open source platform through which the Bank’s models can be made available to the public.

\sphinxAtStartPar
This paper, and the models that accompany it, are the product of this collaboration.

\sphinxstepscope


\part{Installation of modelflow}

\sphinxstepscope


\chapter{Installation of Modelflow}
\label{\detokenize{content/03_Installation/InstallingPython:installation-of-modelflow}}\label{\detokenize{content/03_Installation/InstallingPython::doc}}
\sphinxAtStartPar
Modelflow is a python package that defines the \sphinxcode{\sphinxupquote{model}} class, its methods and a number of other functions that extend and combine pre\sphinxhyphen{}existing python functions to allow the easy solution of complex systems of equations including macro\sphinxhyphen{}structural models like MFMod.  To work with \sphinxcode{\sphinxupquote{modelflow}}, a user needs to first install python (preferably the Anaconda variant), several supporting packages, and of course the \sphinxcode{\sphinxupquote{modelflow}} package itself.  While \sphinxcode{\sphinxupquote{modelflow}} can be run directly from the python command\sphinxhyphen{}line or IDEs (Interactive Development Environments) like \sphinxcode{\sphinxupquote{Spyder}} or Microsoft’s \sphinxcode{\sphinxupquote{Visual Code}}, it is suggested that users also install the Jupyter notebook system. Jupyter Notebook facilitates an interactive approach to building python programs, annotating them and ultimately doing simulations using MFMod under \sphinxcode{\sphinxupquote{modelflow}}. This entire manual and the examples in it were all written and executed in the Jupyter Notebook environment.


\section{Installation of Python}
\label{\detokenize{content/03_Installation/InstallingPython:installation-of-python}}
\sphinxAtStartPar
Python is an extremely powerful and versatile and extensible open\sphinxhyphen{}source language. It is widely used for artificial intelligence application, interactive web sites, and scientific processing. As of 14 November 2022, the Python Package Index (PyPI), the official repository for third\sphinxhyphen{}party Python software, contained over 415,000 packages that extend its functionality %
\begin{footnote}[1]\sphinxAtStartFootnote
\sphinxhref{https://en.wikipedia.org/wiki/Python\_(programming\_language)}{Wikipedia article on python}
%
\end{footnote}. Modelflow is one of these packages.

\sphinxAtStartPar
Python comes in many flavors and \sphinxcode{\sphinxupquote{modelflow}} will work with any of them.  However, it is strongly suggested that you use the Anaconda version of Python.  The remainder of this section points to instructions on how to install the Anaconda version of python (under Windows, MacOS and under Linux). Modelflow works equally well under all three.

\sphinxAtStartPar
This is followed by section that describes the steps necessary to create an anaconda environment with all the necessary packages to run \sphinxcode{\sphinxupquote{modelflow}}.


\subsection{Installation of Anaconda under Windows}
\label{\detokenize{content/03_Installation/InstallingPython:installation-of-anaconda-under-windows}}
\sphinxAtStartPar
The definitive source for installing Anaconda under windows can be found \sphinxhref{https://docs.anaconda.com/anaconda/install/windows/}{here}.

\sphinxAtStartPar
\sphinxstylestrong{It is strongly advised that Anaconda be installed for a single user (Just Me)}  This is much easier to maintain over time.  Installing “For all users on this computer” will substabitally increase the complexity of maintaining python on your computer.


\subsection{Installation of Python under macOS}
\label{\detokenize{content/03_Installation/InstallingPython:installation-of-python-under-macos}}
\sphinxAtStartPar
The definitive source for installing Anaconda under macOS can be found \sphinxhref{https://docs.anaconda.com/anaconda/install/mac-os/}{here}.


\subsection{Installation of Python under Linux}
\label{\detokenize{content/03_Installation/InstallingPython:installation-of-python-under-linux}}
\sphinxAtStartPar
The definitive source for installing Anaconda under Linux can be found \sphinxhref{https://docs.anaconda.com/anaconda/install/linux/}{here}.

\sphinxAtStartPar
Once Anaconda is fully installed, you can then go to the following instruction on how to install \sphinxcode{\sphinxupquote{modelflow}} and the packages on which it depends.


\bigskip\hrule\bigskip


\sphinxstepscope


\chapter{Installation of Modelflow}
\label{\detokenize{content/03_Installation/InstallingModelFlow:installation-of-modelflow}}\label{\detokenize{content/03_Installation/InstallingModelFlow::doc}}
\begin{sphinxadmonition}{warning}{Warning:}
\sphinxAtStartPar
The following instructions concern the installation of \sphinxcode{\sphinxupquote{modelflow}} within an Anaconda installation of python.  Different flavors of Python may require slight changes to this recipe, but are not covered here.

\sphinxAtStartPar
\sphinxcode{\sphinxupquote{Modelflow}} is built and tested using the anaconda python environment.  It is strongly recommended to use Anaconda with ````modelflow```.

\sphinxAtStartPar
If you have not already installed Anaconda following the instructions in the preceding chapter, please do so \sphinxstylestrong{Now}.
\end{sphinxadmonition}

\sphinxAtStartPar
\sphinxcode{\sphinxupquote{Modelflow}} is a python package that defines the modelflow class \sphinxcode{\sphinxupquote{model}} among others.  \sphinxcode{\sphinxupquote{Modelflow1}} has many dependencies. Installing the class the first time can take some time depending on your internet connection and computer speed.  It is essential that you follow all of the steps outlined below to ensure that your version of \sphinxcode{\sphinxupquote{modelflow}} operates as expected.


\section{Installation of \sphinxstyleliteralintitle{\sphinxupquote{modelflow}} under Anaconda}
\label{\detokenize{content/03_Installation/InstallingModelFlow:installation-of-modelflow-under-anaconda}}\begin{enumerate}
\sphinxsetlistlabels{\arabic}{enumi}{enumii}{}{.}%
\item {} 
\sphinxAtStartPar
Open the anaconda command prompt

\item {} 
\sphinxAtStartPar
Execute the following commands by copying and pasting them – either line by line or as a single mult\sphinxhyphen{}line step

\item {} 
\sphinxAtStartPar
Press enter

\end{enumerate}

\begin{sphinxVerbatim}[commandchars=\\\{\}]
\PYG{n}{conda} \PYG{n}{create} \PYG{o}{\PYGZhy{}}\PYG{n}{n} \PYG{n}{ModelFlow} \PYG{o}{\PYGZhy{}}\PYG{n}{c} \PYG{n}{ibh} \PYG{o}{\PYGZhy{}}\PYG{n}{c}  \PYG{n}{conda}\PYG{o}{\PYGZhy{}}\PYG{n}{forge} \PYG{n}{modelflow\PYGZus{}pinned\PYGZus{}developement\PYGZus{}test} \PYG{o}{\PYGZhy{}}\PYG{n}{y}
\PYG{n}{conda} \PYG{n}{activate} \PYG{n}{ModelFlow}
\PYG{n}{pip} \PYG{n}{install} \PYG{n}{dash\PYGZus{}interactive\PYGZus{}graphviz}
\PYG{n}{conda} \PYG{n}{install} \PYG{n}{pyeviews} \PYG{o}{\PYGZhy{}}\PYG{n}{c} \PYG{n}{conda}\PYG{o}{\PYGZhy{}}\PYG{n}{forge} \PYG{o}{\PYGZhy{}}\PYG{n}{y}
\PYG{n}{jupyter} \PYG{n}{contrib} \PYG{n}{nbextension} \PYG{n}{install} \PYG{o}{\PYGZhy{}}\PYG{o}{\PYGZhy{}}\PYG{n}{user}
\PYG{n}{jupyter} \PYG{n}{nbextension} \PYG{n}{enable} \PYG{n}{hide\PYGZus{}input\PYGZus{}all}\PYG{o}{/}\PYG{n}{main}
\PYG{n}{jupyter} \PYG{n}{nbextension} \PYG{n}{enable} \PYG{n}{splitcell}\PYG{o}{/}\PYG{n}{splitcellcd}
\PYG{n}{jupyter} \PYG{n}{nbextension} \PYG{n}{enable} \PYG{n}{toc2}\PYG{o}{/}\PYG{n}{main}

\end{sphinxVerbatim}

\sphinxAtStartPar
Depending on the speed of your computer and of your internet connection installation could take as little as 10 minutes or more than 1/2 an hour.

\sphinxAtStartPar
At the end of the process you will have a new conda environment ModelFlow, and this will have been activated.

\sphinxAtStartPar
Once modelflow is installed you are ready to work with it.  The following sections give a brief introduction to Jupyter notebook, which is a flexible tool that allows us to execute python code, interact with the modelflow class and World Bank Models and annotate what we have done for future replication.

\begin{sphinxadmonition}{note}{Note:}
\sphinxAtStartPar
NB: The next time you want to work with modelflow, you will need to activate the \sphinxcode{\sphinxupquote{modelflow}} environment by
\begin{enumerate}
\sphinxsetlistlabels{\arabic}{enumi}{enumii}{}{)}%
\item {} 
\sphinxAtStartPar
Opening the Anaconda command prompt window

\item {} 
\sphinxAtStartPar
Activate the ModelFlow environment we just created by executing the folling command

\end{enumerate}

\sphinxAtStartPar
\sphinxcode{\sphinxupquote{conda activate modelflow}}
\end{sphinxadmonition}

\sphinxstepscope


\chapter{Testing your installation of modelflow}
\label{\detokenize{content/03_Installation/TestingModelFlow:testing-your-installation-of-modelflow}}\label{\detokenize{content/03_Installation/TestingModelFlow::doc}}
\sphinxAtStartPar
To test that the installation of modelflow has worked properly, we will build a model using the modelflow framework and then simulate it.  A simple model that illustrates many of the functions of modelflow is the Solow growth model.

\sphinxAtStartPar
The code below first sets up the python environment by importing the modelflow  and pandas classes.  The initial two lines of code and the final two lines just set up the environment for optimal display and are not required.

\sphinxAtStartPar
To test the installation on your system you can copy this code into a Jupyter notebook and execute it.


\section{Specifying the model}
\label{\detokenize{content/03_Installation/TestingModelFlow:specifying-the-model}}
\sphinxAtStartPar
Having loaded the model class from the modelflow library, we can start constructing the model.

\sphinxAtStartPar
The first step is to define the equations of the model, using \sphinxcode{\sphinxupquote{modelflow}}’s Business Logic Language.

\begin{sphinxShadowBox}
\sphinxstylesidebartitle{\sphinxstylestrong{Business Logic Language}}

\sphinxAtStartPar
More on how to specify models \DUrole{xref,myst}{here}
\end{sphinxShadowBox}

\sphinxAtStartPar
The below code segment defines a string fsolow that contains the equations for the solow model, where:
\begin{itemize}
\item {} 
\sphinxAtStartPar
GDP is defined as a simple Cobb\sphinxhyphen{}Douglas production function as the product of TFP, Capital (raised to the share of capital in total income) and Labour (raised to the share of labor in total income)

\item {} 
\sphinxAtStartPar
Investment is equal to GDP less consumption

\item {} 
\sphinxAtStartPar
The change in capital is equal to investment this period less the depreciation of the capital stock from the previous period

\item {} 
\sphinxAtStartPar
Labor grows at the rate of growth of the variable \sphinxcode{\sphinxupquote{Labor\_growth}}

\item {} 
\sphinxAtStartPar
a pure reporting identity Capital\_intensity the ratio of the Capital Stock to the Labor input

\end{itemize}

\sphinxAtStartPar
We thus have a system of 6 equations with 6 unknowns (GDP, Consumption, Investment, Change in the Capital stock, and change in Labor supply, and the capital\_intensity) and exogenous variables (TFP, alfa,savings\_rate,Depreciation\_rate and Labor\_growth).

\sphinxAtStartPar
The equations for Labor and Capital have been entered as difference equations. The \sphinxcode{\sphinxupquote{modelflow}} object will automatically normalize them, generating an internal representation of \sphinxcode{\sphinxupquote{Labour=Labour(t\sphinxhyphen{}1)*(1+Labor\_growth)}} and \sphinxcode{\sphinxupquote{Capital=Capital(t\sphinxhyphen{}1)*(1\sphinxhyphen{}Depreciation\_rate)+Investment}}

\begin{sphinxuseclass}{cell}\begin{sphinxVerbatimInput}

\begin{sphinxuseclass}{cell_input}
\begin{sphinxVerbatim}[commandchars=\\\{\}]
\PYG{n}{fsolow} \PYG{o}{=} \PYG{l+s+s1}{\PYGZsq{}\PYGZsq{}\PYGZsq{}}\PYG{l+s+se}{\PYGZbs{}}
\PYG{l+s+s1}{GDP          = TFP  * Capital**alfa * Labor **(1\PYGZhy{}alfa) }
\PYG{l+s+s1}{Consumption     = (1\PYGZhy{}saving\PYGZus{}rate)  * GDP }
\PYG{l+s+s1}{Investment      = GDP \PYGZhy{} Consumption   }
\PYG{l+s+s1}{diff(Capital)   = Investment\PYGZhy{}Depreciation\PYGZus{}rate * Capital(\PYGZhy{}1)}
\PYG{l+s+s1}{diff(Labor)     = Labor\PYGZus{}growth * Labor(\PYGZhy{}1)  }
\PYG{l+s+s1}{Capital\PYGZus{}intensity = Capital/Labor }
\PYG{l+s+s1}{\PYGZsq{}\PYGZsq{}\PYGZsq{}}
\end{sphinxVerbatim}

\end{sphinxuseclass}\end{sphinxVerbatimInput}

\end{sphinxuseclass}
\sphinxAtStartPar
To create the model we instantiate (create) a variable \sphinxcode{\sphinxupquote{msolow}} (which will ultimately contain both the equations and data for the model) using the \sphinxcode{\sphinxupquote{.from\_eq()}} method of the \sphinxcode{\sphinxupquote{modelflow}} class – submitting to it the equations in string form, and giving it the name “Solow model”.

\begin{sphinxuseclass}{cell}\begin{sphinxVerbatimInput}

\begin{sphinxuseclass}{cell_input}
\begin{sphinxVerbatim}[commandchars=\\\{\}]
\PYG{n}{msolow} \PYG{o}{=} \PYG{n}{model}\PYG{o}{.}\PYG{n}{from\PYGZus{}eq}\PYG{p}{(}\PYG{n}{fsolow}\PYG{p}{,}\PYG{n}{modelname}\PYG{o}{=}\PYG{l+s+s1}{\PYGZsq{}}\PYG{l+s+s1}{Solow model}\PYG{l+s+s1}{\PYGZsq{}}\PYG{p}{)}
\end{sphinxVerbatim}

\end{sphinxuseclass}\end{sphinxVerbatimInput}

\end{sphinxuseclass}
\sphinxAtStartPar
The internal representation of the normalized equations can be displayed in normalized business language with the \sphinxcode{\sphinxupquote{modelflow}} method \sphinxcode{\sphinxupquote{.print\_model}}:

\begin{sphinxuseclass}{cell}\begin{sphinxVerbatimInput}

\begin{sphinxuseclass}{cell_input}
\begin{sphinxVerbatim}[commandchars=\\\{\}]
\PYG{n}{msolow}\PYG{o}{.}\PYG{n}{print\PYGZus{}model}
\end{sphinxVerbatim}

\end{sphinxuseclass}\end{sphinxVerbatimInput}
\begin{sphinxVerbatimOutput}

\begin{sphinxuseclass}{cell_output}
\begin{sphinxVerbatim}[commandchars=\\\{\}]
FRML \PYGZlt{}\PYGZgt{} GDP          = TFP  * CAPITAL**ALFA * LABOR **(1\PYGZhy{}ALFA)  \PYGZdl{}
FRML \PYGZlt{}\PYGZgt{} CONSUMPTION     = (1\PYGZhy{}SAVING\PYGZus{}RATE)  * GDP  \PYGZdl{}
FRML \PYGZlt{}\PYGZgt{} INVESTMENT      = GDP \PYGZhy{} CONSUMPTION    \PYGZdl{}
FRML \PYGZlt{}\PYGZgt{} CAPITAL=CAPITAL(\PYGZhy{}1)+(INVESTMENT\PYGZhy{}DEPRECIATION\PYGZus{}RATE * CAPITAL(\PYGZhy{}1))\PYGZdl{}
FRML \PYGZlt{}\PYGZgt{} LABOR=LABOR(\PYGZhy{}1)+(LABOR\PYGZus{}GROWTH * LABOR(\PYGZhy{}1))\PYGZdl{}
FRML \PYGZlt{}\PYGZgt{} CAPITAL\PYGZus{}INTENSITY = CAPITAL/LABOR  \PYGZdl{}
\end{sphinxVerbatim}

\end{sphinxuseclass}\end{sphinxVerbatimOutput}

\end{sphinxuseclass}

\section{Create some data}
\label{\detokenize{content/03_Installation/TestingModelFlow:create-some-data}}
\sphinxAtStartPar
For the moment \sphinxcode{\sphinxupquote{msolow}} has a mathematical representation of a system of equations but no data.

\sphinxAtStartPar
To add data we  create a pandas dataframe with initial values for our exogenous variables. Technically capital and labor are endogenous in the Solow model, but because they are specified as change equations their initial values are exogenous and need to be initialized.

\sphinxAtStartPar
The code below  instantiates (creates) a panda dataframe \sphinxcode{\sphinxupquote{df}} and fills it with the variables for our model, initializing these with a series of values over 300 datapoints.  The final command displays the first ten rows of the dataframe.

\begin{sphinxadmonition}{note}{Note:}
\begin{sphinxVerbatim}[commandchars=\\\{\}]
Pandas data frames is a foundational class of python.  There are thousands of web sites dedicated to understanding pandas.  Some notable ones include:
\end{sphinxVerbatim}
\end{sphinxadmonition}

\begin{sphinxuseclass}{cell}\begin{sphinxVerbatimInput}

\begin{sphinxuseclass}{cell_input}
\begin{sphinxVerbatim}[commandchars=\\\{\}]
\PYG{n}{N} \PYG{o}{=} \PYG{l+m+mi}{300}  
\PYG{n}{df} \PYG{o}{=} \PYG{n}{pd}\PYG{o}{.}\PYG{n}{DataFrame}\PYG{p}{(}\PYG{p}{\PYGZob{}}\PYG{l+s+s1}{\PYGZsq{}}\PYG{l+s+s1}{LABOR}\PYG{l+s+s1}{\PYGZsq{}}\PYG{p}{:}\PYG{p}{[}\PYG{l+m+mi}{100}\PYG{p}{]}\PYG{o}{*}\PYG{n}{N}\PYG{p}{,}
                   \PYG{l+s+s1}{\PYGZsq{}}\PYG{l+s+s1}{CAPITAL}\PYG{l+s+s1}{\PYGZsq{}}\PYG{p}{:}\PYG{p}{[}\PYG{l+m+mi}{100}\PYG{p}{]}\PYG{o}{*}\PYG{n}{N}\PYG{p}{,} 
                   \PYG{l+s+s1}{\PYGZsq{}}\PYG{l+s+s1}{ALFA}\PYG{l+s+s1}{\PYGZsq{}}\PYG{p}{:}\PYG{p}{[}\PYG{l+m+mf}{0.5}\PYG{p}{]}\PYG{o}{*}\PYG{n}{N}\PYG{p}{,} 
                   \PYG{l+s+s1}{\PYGZsq{}}\PYG{l+s+s1}{TFP}\PYG{l+s+s1}{\PYGZsq{}}\PYG{p}{:} \PYG{p}{[}\PYG{l+m+mi}{1}\PYG{p}{]}\PYG{o}{*}\PYG{n}{N}\PYG{p}{,} 
                   \PYG{l+s+s1}{\PYGZsq{}}\PYG{l+s+s1}{DEPRECIATION\PYGZus{}RATE}\PYG{l+s+s1}{\PYGZsq{}}\PYG{p}{:} \PYG{p}{[}\PYG{l+m+mf}{0.05}\PYG{p}{]}\PYG{o}{*}\PYG{n}{N}\PYG{p}{,} 
                   \PYG{l+s+s1}{\PYGZsq{}}\PYG{l+s+s1}{LABOR\PYGZus{}GROWTH}\PYG{l+s+s1}{\PYGZsq{}}\PYG{p}{:} \PYG{p}{[}\PYG{l+m+mf}{0.01}\PYG{p}{]}\PYG{o}{*}\PYG{n}{N}\PYG{p}{,} 
                   \PYG{l+s+s1}{\PYGZsq{}}\PYG{l+s+s1}{SAVING\PYGZus{}RATE}\PYG{l+s+s1}{\PYGZsq{}}\PYG{p}{:}\PYG{p}{[}\PYG{l+m+mf}{0.05}\PYG{p}{]}\PYG{o}{*}\PYG{n}{N}\PYG{p}{\PYGZcb{}}\PYG{p}{,}\PYG{n}{index}\PYG{o}{=}\PYG{p}{[}\PYG{n}{v} \PYG{k}{for} \PYG{n}{v} \PYG{o+ow}{in} \PYG{n+nb}{range}\PYG{p}{(}\PYG{l+m+mi}{2000}\PYG{p}{,}\PYG{l+m+mi}{2300}\PYG{p}{)}\PYG{p}{]}\PYG{p}{)}
\PYG{n}{df}\PYG{o}{.}\PYG{n}{head}\PYG{p}{(}\PYG{p}{)} \PYG{c+c1}{\PYGZsh{}this prints out the first 5 rows of the dataframe}
\end{sphinxVerbatim}

\end{sphinxuseclass}\end{sphinxVerbatimInput}
\begin{sphinxVerbatimOutput}

\begin{sphinxuseclass}{cell_output}
\begin{sphinxVerbatim}[commandchars=\\\{\}]
      LABOR  CAPITAL  ALFA  TFP  DEPRECIATION\PYGZus{}RATE  LABOR\PYGZus{}GROWTH  SAVING\PYGZus{}RATE
2000    100      100   0.5    1               0.05          0.01         0.05
2001    100      100   0.5    1               0.05          0.01         0.05
2002    100      100   0.5    1               0.05          0.01         0.05
2003    100      100   0.5    1               0.05          0.01         0.05
2004    100      100   0.5    1               0.05          0.01         0.05
\end{sphinxVerbatim}

\end{sphinxuseclass}\end{sphinxVerbatimOutput}

\end{sphinxuseclass}

\section{Putting it together}
\label{\detokenize{content/03_Installation/TestingModelFlow:putting-it-together}}
\sphinxAtStartPar
Having defined an initial data set for all the exogenous variables, we can combine these with the equations and solve the model.

\sphinxAtStartPar
The command below solves the model \sphinxcode{\sphinxupquote{msolow}} on the data contained in the dataframe \sphinxcode{\sphinxupquote{df}} and stores the output in a new dataframe called \sphinxcode{\sphinxupquote{result}}.

\sphinxAtStartPar
The last line displays the values of the simulated model, which now includes results for the endogenous variables, and different values for the Labor and Capital variables reflecting their endogeneity for periods 2 through 300.

\begin{sphinxuseclass}{cell}\begin{sphinxVerbatimInput}

\begin{sphinxuseclass}{cell_input}
\begin{sphinxVerbatim}[commandchars=\\\{\}]
\PYG{n}{result} \PYG{o}{=} \PYG{n}{msolow}\PYG{p}{(}\PYG{n}{df}\PYG{p}{,}\PYG{n}{keep}\PYG{o}{=}\PYG{l+s+s1}{\PYGZsq{}}\PYG{l+s+s1}{Baseline}\PYG{l+s+s1}{\PYGZsq{}}\PYG{p}{)} 
\PYG{c+c1}{\PYGZsh{} The model is simulated for all years possible }

\PYG{n}{result}\PYG{o}{.}\PYG{n}{head}\PYG{p}{(}\PYG{l+m+mi}{10}\PYG{p}{)}
\end{sphinxVerbatim}

\end{sphinxuseclass}\end{sphinxVerbatimInput}
\begin{sphinxVerbatimOutput}

\begin{sphinxuseclass}{cell_output}
\begin{sphinxVerbatim}[commandchars=\\\{\}]
           LABOR     CAPITAL  ALFA  TFP  DEPRECIATION\PYGZus{}RATE  LABOR\PYGZus{}GROWTH  \PYGZbs{}
2000  100.000000  100.000000   0.5  1.0               0.05          0.01   
2001  101.000000  100.025580   0.5  1.0               0.05          0.01   
2002  102.010000  100.076226   0.5  1.0               0.05          0.01   
2003  103.030100  100.151443   0.5  1.0               0.05          0.01   
2004  104.060401  100.250762   0.5  1.0               0.05          0.01   
2005  105.101005  100.373733   0.5  1.0               0.05          0.01   
2006  106.152015  100.519926   0.5  1.0               0.05          0.01   
2007  107.213535  100.688931   0.5  1.0               0.05          0.01   
2008  108.285671  100.880357   0.5  1.0               0.05          0.01   
2009  109.368527  101.093830   0.5  1.0               0.05          0.01   

      SAVING\PYGZus{}RATE         GDP  CAPITAL\PYGZus{}INTENSITY  INVESTMENT  CONSUMPTION  
2000         0.05    0.000000           0.000000    0.000000     0.000000  
2001         0.05  100.511609           0.990352    5.025580    95.486029  
2002         0.05  101.038487           0.981043    5.051924    95.986562  
2003         0.05  101.580575           0.972060    5.079029    96.501546  
2004         0.05  102.137821           0.963390    5.106891    97.030930  
2005         0.05  102.710176           0.955022    5.135509    97.574667  
2006         0.05  103.297593           0.946943    5.164880    98.132713  
2007         0.05  103.900030           0.939144    5.195002    98.705029  
2008         0.05  104.517449           0.931613    5.225872    99.291576  
2009         0.05  105.149813           0.924341    5.257491    99.892323  
\end{sphinxVerbatim}

\end{sphinxuseclass}\end{sphinxVerbatimOutput}

\end{sphinxuseclass}

\section{Create a scenario and run again}
\label{\detokenize{content/03_Installation/TestingModelFlow:create-a-scenario-and-run-again}}
\begin{sphinxShadowBox}
\sphinxstylesidebartitle{\sphinxstylestrong{dataframe.upd}}

\sphinxAtStartPar
When importing modelclass all pandas dataframes are enriched with a a handy way to create a new pandas dataframe as a copy of an existing one but with one or more series updated.

\sphinxAtStartPar
In this case df.upd will create a a new dataframe \sphinxcode{\sphinxupquote{dfscenaario}} with updated LABOR\_GROWTH

\sphinxAtStartPar
For more detail on the \sphinxcode{\sphinxupquote{.upd}} method look here \DUrole{xref,myst}{here}
\end{sphinxShadowBox}

\begin{sphinxuseclass}{cell}\begin{sphinxVerbatimInput}

\begin{sphinxuseclass}{cell_input}
\begin{sphinxVerbatim}[commandchars=\\\{\}]
\PYG{n}{dfscenario} \PYG{o}{=} \PYG{n}{df}\PYG{o}{.}\PYG{n}{mfcalc}\PYG{p}{(}\PYG{l+s+s1}{\PYGZsq{}}\PYG{l+s+s1}{\PYGZlt{}2023 2200\PYGZgt{} LABOR\PYGZus{}GROWTH = LABOR\PYGZus{}GROWTH + 0.002}\PYG{l+s+s1}{\PYGZsq{}}\PYG{p}{)}  \PYG{c+c1}{\PYGZsh{} create a new dataframe, increase LABOR\PYGZus{}GROWTH by 0.002}
\PYG{n}{scenario}   \PYG{o}{=} \PYG{n}{msolow}\PYG{p}{(}\PYG{n}{dfscenario}\PYG{p}{,}\PYG{n}{keep}\PYG{o}{=}\PYG{l+s+s1}{\PYGZsq{}}\PYG{l+s+s1}{Higher labor growth }\PYG{l+s+s1}{\PYGZsq{}}\PYG{p}{)} \PYG{c+c1}{\PYGZsh{} simulate the model }
\end{sphinxVerbatim}

\end{sphinxuseclass}\end{sphinxVerbatimInput}

\end{sphinxuseclass}

\section{Inspect results}
\label{\detokenize{content/03_Installation/TestingModelFlow:inspect-results}}
\sphinxAtStartPar
\sphinxcode{\sphinxupquote{Modelflow}} includes a range of methods to view data and results, either as graphs or as tables.  Some of these are part of standard python, others are additional features that \sphinxcode{\sphinxupquote{modelflow}} makes available.

\sphinxAtStartPar
Scenario results can be inspected either by referring to the scenario name given in the (optional) \sphinxcode{\sphinxupquote{keep}} statement when the model was solved, by referring to the \sphinxcode{\sphinxupquote{basedf}} and the \sphinxcode{\sphinxupquote{lastdf}}.
\begin{itemize}
\item {} 
\sphinxAtStartPar
\sphinxcode{\sphinxupquote{basedf}} is a dataframe that is automatically generated when the model is solved and contains a copy of the initial conditions of the model prior to the shock.

\item {} 
\sphinxAtStartPar
\sphinxcode{\sphinxupquote{lastdf}}is a dataframe that is automatically generated when the model is solved and contains a copy of the results from the simulation. Several built in display functions use these functions to display results.

\end{itemize}

\sphinxAtStartPar
Finally one could also look at the dataframe to which the results of the simulation were assigned \sphinxcode{\sphinxupquote{scenario}} in the example above.

\sphinxAtStartPar
Below is a small sub\sphinxhyphen{}set of the visualization options available.


\subsection{Graphical representations of results}
\label{\detokenize{content/03_Installation/TestingModelFlow:graphical-representations-of-results}}

\subsubsection{The .dif.plot() method}
\label{\detokenize{content/03_Installation/TestingModelFlow:the-dif-plot-method}}
\sphinxAtStartPar
The \sphinxcode{\sphinxupquote{.dif.plot}} method will plot the change in the level of requested variables.  Requested variables can be selected either directly by name or using wildcards.

\sphinxAtStartPar
In this example, a wild card specification is used, requesting the display of all variables that begin with the text ‘labor’.  Note that the selector is not case sensitive.

\sphinxAtStartPar
In this case we are displaying changes into the labor and labor growth variables due to the shock when we increased the growth rate of labor by .0002

\begin{sphinxuseclass}{cell}\begin{sphinxVerbatimInput}

\begin{sphinxuseclass}{cell_input}
\begin{sphinxVerbatim}[commandchars=\\\{\}]
\PYG{n}{msolow}\PYG{p}{[}\PYG{l+s+s1}{\PYGZsq{}}\PYG{l+s+s1}{labor*}\PYG{l+s+s1}{\PYGZsq{}}\PYG{p}{]}\PYG{o}{.}\PYG{n}{dif}\PYG{o}{.}\PYG{n}{plot}\PYG{p}{(}\PYG{p}{)} 
\end{sphinxVerbatim}

\end{sphinxuseclass}\end{sphinxVerbatimInput}
\begin{sphinxVerbatimOutput}

\begin{sphinxuseclass}{cell_output}
\noindent\sphinxincludegraphics{{28ae05d62fafe5fd35db4868226962fccd6269bec73310fece8437ae7f551422}.png}

\noindent\sphinxincludegraphics{{28ae05d62fafe5fd35db4868226962fccd6269bec73310fece8437ae7f551422}.png}

\end{sphinxuseclass}\end{sphinxVerbatimOutput}

\end{sphinxuseclass}
\sphinxAtStartPar
In this example, instead of using a wild card selector we requested a variable explicitly by name.

\begin{sphinxuseclass}{cell}\begin{sphinxVerbatimInput}

\begin{sphinxuseclass}{cell_input}
\begin{sphinxVerbatim}[commandchars=\\\{\}]
\PYG{n}{msolow}\PYG{p}{[}\PYG{l+s+s1}{\PYGZsq{}}\PYG{l+s+s1}{GDP LABOR\PYGZus{}GROWTH}\PYG{l+s+s1}{\PYGZsq{}}\PYG{p}{]}\PYG{o}{.}\PYG{n}{pct}\PYG{o}{.}\PYG{n}{plot}\PYG{p}{(}\PYG{p}{)} 
\end{sphinxVerbatim}

\end{sphinxuseclass}\end{sphinxVerbatimInput}
\begin{sphinxVerbatimOutput}

\begin{sphinxuseclass}{cell_output}
\noindent\sphinxincludegraphics{{49f43ba057063db290c43f05c27a589984c46ab42fe9078b7976044d7baec37f}.png}

\end{sphinxuseclass}\end{sphinxVerbatimOutput}

\end{sphinxuseclass}

\subsubsection{Using the kept solutions}
\label{\detokenize{content/03_Installation/TestingModelFlow:using-the-kept-solutions}}
\sphinxAtStartPar
Because the keyword \sphinxcode{\sphinxupquote{keep}} was used when running the simulations, we can refer to the scenarios by their names – or produce graphs from multiple scenarios – not just the first and last.

\begin{sphinxuseclass}{cell}\begin{sphinxVerbatimInput}

\begin{sphinxuseclass}{cell_input}
\begin{sphinxVerbatim}[commandchars=\\\{\}]
\PYG{n}{msolow}\PYG{o}{.}\PYG{n}{keep\PYGZus{}plot}\PYG{p}{(}\PYG{l+s+s1}{\PYGZsq{}}\PYG{l+s+s1}{GDP}\PYG{l+s+s1}{\PYGZsq{}}\PYG{p}{)}
    
\end{sphinxVerbatim}

\end{sphinxuseclass}\end{sphinxVerbatimInput}
\begin{sphinxVerbatimOutput}

\begin{sphinxuseclass}{cell_output}
\begin{sphinxVerbatim}[commandchars=\\\{\}]
\PYGZob{}\PYGZsq{}GDP\PYGZsq{}: \PYGZlt{}Figure size 1000x600 with 1 Axes\PYGZgt{}\PYGZcb{}
\end{sphinxVerbatim}

\end{sphinxuseclass}\end{sphinxVerbatimOutput}

\end{sphinxuseclass}

\subsection{Textual and tabular display of results}
\label{\detokenize{content/03_Installation/TestingModelFlow:textual-and-tabular-display-of-results}}
\sphinxAtStartPar
Standard pandas syntax can be used to display data in the results dataframes.

\sphinxAtStartPar
Here we use the standard pandas \sphinxcode{\sphinxupquote{.loc}} method to display every 10th data point for consumption from the results dataframe, beginning from observation 50 through 100.

\begin{sphinxuseclass}{cell}\begin{sphinxVerbatimInput}

\begin{sphinxuseclass}{cell_input}
\begin{sphinxVerbatim}[commandchars=\\\{\}]
\PYG{n}{msolow}\PYG{o}{.}\PYG{n}{lastdf}\PYG{o}{.}\PYG{n}{loc}\PYG{p}{[}\PYG{l+m+mi}{50}\PYG{p}{:}\PYG{l+m+mi}{100}\PYG{p}{:}\PYG{l+m+mi}{10}\PYG{p}{,}\PYG{l+s+s1}{\PYGZsq{}}\PYG{l+s+s1}{CONSUMPTION}\PYG{l+s+s1}{\PYGZsq{}}\PYG{p}{]}
\end{sphinxVerbatim}

\end{sphinxuseclass}\end{sphinxVerbatimInput}
\begin{sphinxVerbatimOutput}

\begin{sphinxuseclass}{cell_output}
\begin{sphinxVerbatim}[commandchars=\\\{\}]
Series([], Name: CONSUMPTION, dtype: float64)
\end{sphinxVerbatim}

\end{sphinxuseclass}\end{sphinxVerbatimOutput}

\end{sphinxuseclass}

\subsubsection{The \sphinxstyleliteralintitle{\sphinxupquote{.dif.df}} method}
\label{\detokenize{content/03_Installation/TestingModelFlow:the-dif-df-method}}
\sphinxAtStartPar
The \sphinxcode{\sphinxupquote{.dif.df}} method prints out the changes in variables, i.e. eh difference between the level of specified  variables in the \sphinxcode{\sphinxupquote{lastdf}} dataframe vs the \sphinxcode{\sphinxupquote{basedf}} dataframe.

\begin{sphinxuseclass}{cell}\begin{sphinxVerbatimInput}

\begin{sphinxuseclass}{cell_input}
\begin{sphinxVerbatim}[commandchars=\\\{\}]
\PYG{n}{msolow}\PYG{p}{[}\PYG{l+s+s1}{\PYGZsq{}}\PYG{l+s+s1}{GDP CONSUMPTION}\PYG{l+s+s1}{\PYGZsq{}}\PYG{p}{]}\PYG{o}{.}\PYG{n}{dif}\PYG{o}{.}\PYG{n}{df}
\end{sphinxVerbatim}

\end{sphinxuseclass}\end{sphinxVerbatimInput}
\begin{sphinxVerbatimOutput}

\begin{sphinxuseclass}{cell_output}
\begin{sphinxVerbatim}[commandchars=\\\{\}]
             GDP  CONSUMPTION
2001    0.000000     0.000000
2002    0.000000     0.000000
2003    0.000000     0.000000
2004    0.000000     0.000000
2005    0.000000     0.000000
...          ...          ...
2295  665.334581   632.067852
2296  672.097592   638.492713
2297  678.925939   644.979642
2298  685.820324   651.529308
2299  692.781453   658.142380

[299 rows x 2 columns]
\end{sphinxVerbatim}

\end{sphinxuseclass}\end{sphinxVerbatimOutput}

\end{sphinxuseclass}

\subsubsection{The \sphinxstyleliteralintitle{\sphinxupquote{.difpct.df}} method}
\label{\detokenize{content/03_Installation/TestingModelFlow:the-difpct-df-method}}
\sphinxAtStartPar
The \sphinxcode{\sphinxupquote{.dif.pct.df}} method express the changes between the last simulation and base simulation results as a percent differences in the level (\({\Delta X_t \over X^{basedf}_{t-1}} \) ).  In the example below the mul100 method multiplies the result by 100.

\begin{sphinxuseclass}{cell}\begin{sphinxVerbatimInput}

\begin{sphinxuseclass}{cell_input}
\begin{sphinxVerbatim}[commandchars=\\\{\}]
\PYG{n}{msolow}\PYG{p}{[}\PYG{l+s+s1}{\PYGZsq{}}\PYG{l+s+s1}{GDP CONSUMPTION}\PYG{l+s+s1}{\PYGZsq{}}\PYG{p}{]}\PYG{o}{.}\PYG{n}{difpct}\PYG{o}{.}\PYG{n}{mul100}\PYG{o}{.}\PYG{n}{df}
\end{sphinxVerbatim}

\end{sphinxuseclass}\end{sphinxVerbatimInput}
\begin{sphinxVerbatimOutput}

\begin{sphinxuseclass}{cell_output}
\begin{sphinxVerbatim}[commandchars=\\\{\}]
           GDP  CONSUMPTION
2001       NaN          NaN
2002  0.000000     0.000000
2003  0.000000     0.000000
2004  0.000000     0.000000
2005  0.000000     0.000000
...        ...          ...
2295  0.005047     0.005047
2296  0.004892     0.004892
2297  0.004742     0.004742
2298  0.004596     0.004596
2299  0.004456     0.004456

[299 rows x 2 columns]
\end{sphinxVerbatim}

\end{sphinxuseclass}\end{sphinxVerbatimOutput}

\end{sphinxuseclass}

\subsection{Interactive display of impacts}
\label{\detokenize{content/03_Installation/TestingModelFlow:interactive-display-of-impacts}}
\sphinxAtStartPar
When working within Jupyter notebook the {[} {]}  command will produce (without the .df termination) will generate a widget with the results expressed as level differences, percent differences, differences in the growth rate – both graphically and in table form.

\sphinxAtStartPar
Please consult \DUrole{xref,myst}{here} for a fuller presentation of the display routines built into \sphinxcode{\sphinxupquote{modelflfow}}.

\begin{sphinxuseclass}{cell}\begin{sphinxVerbatimInput}

\begin{sphinxuseclass}{cell_input}
\begin{sphinxVerbatim}[commandchars=\\\{\}]
\PYG{n}{msolow}\PYG{p}{[}\PYG{l+s+s1}{\PYGZsq{}}\PYG{l+s+s1}{GDP CONSUMPTION}\PYG{l+s+s1}{\PYGZsq{}}\PYG{p}{]}
\end{sphinxVerbatim}

\end{sphinxuseclass}\end{sphinxVerbatimInput}
\begin{sphinxVerbatimOutput}

\begin{sphinxuseclass}{cell_output}
\begin{sphinxVerbatim}[commandchars=\\\{\}]
Tab(children=(Tab(children=(HTML(value=\PYGZsq{}\PYGZlt{}?xml version=\PYGZdq{}1.0\PYGZdq{} encoding=\PYGZdq{}utf\PYGZhy{}8\PYGZdq{} standalone=\PYGZdq{}no\PYGZdq{}?\PYGZgt{}\PYGZbs{}n\PYGZlt{}!DOCTYPE svg …
\end{sphinxVerbatim}

\begin{sphinxVerbatim}[commandchars=\\\{\}]

\end{sphinxVerbatim}

\end{sphinxuseclass}\end{sphinxVerbatimOutput}

\end{sphinxuseclass}
\sphinxstepscope


\part{Features}

\sphinxstepscope


\chapter{Useful model instance properties and methods}
\label{\detokenize{content/notebooks/modelflow_features:useful-model-instance-properties-and-methods}}\label{\detokenize{content/notebooks/modelflow_features::doc}}
\sphinxAtStartPar
The focus of this chapter is to introduce some properties and methods of the model instance.

\sphinxAtStartPar
First a model and data is loaded, then a scenario is run. Then we have some content to use.

\sphinxAtStartPar
A model instance gives the user access to a number of properties and methods which helps in managing the model and its results.

\sphinxAtStartPar
If \sphinxcode{\sphinxupquote{mmodel}} is a model instance \sphinxcode{\sphinxupquote{mmodel.<property>}} will return a property. Some properties can also be assigned by the user just by:
\begin{quote}

\sphinxAtStartPar
mmodel.property = something
\end{quote}

\sphinxAtStartPar
The model class itself also have a few properties. These are simple accessed by  \sphinxcode{\sphinxupquote{model.<property>}}.

\sphinxAtStartPar
Enjoy


\section{Import the model class}
\label{\detokenize{content/notebooks/modelflow_features:import-the-model-class}}
\sphinxAtStartPar
This class incorporates most of the methods used to manage a model.

\sphinxAtStartPar
Assuming the ModelFlow library has been installed on your machine, the following imports set up your notebook so that you can run the cells in this notebook.

\sphinxAtStartPar
In order to manipulate plots later on matplotlib.pyplot is also imported.

\begin{sphinxuseclass}{cell}\begin{sphinxVerbatimInput}

\begin{sphinxuseclass}{cell_input}
\begin{sphinxVerbatim}[commandchars=\\\{\}]
\PYG{c+c1}{\PYGZsh{}\PYGZpc{}matplotlib notebook}
\PYG{o}{\PYGZpc{}}\PYG{k}{matplotlib} inline
\end{sphinxVerbatim}

\end{sphinxuseclass}\end{sphinxVerbatimInput}

\end{sphinxuseclass}
\begin{sphinxuseclass}{cell}\begin{sphinxVerbatimInput}

\begin{sphinxuseclass}{cell_input}
\begin{sphinxVerbatim}[commandchars=\\\{\}]
\PYG{k+kn}{from} \PYG{n+nn}{modelclass} \PYG{k+kn}{import} \PYG{n}{model} 
\end{sphinxVerbatim}

\end{sphinxuseclass}\end{sphinxVerbatimInput}

\end{sphinxuseclass}
\begin{sphinxuseclass}{cell}\begin{sphinxVerbatimInput}

\begin{sphinxuseclass}{cell_input}
\begin{sphinxVerbatim}[commandchars=\\\{\}]
\PYG{k+kn}{import} \PYG{n+nn}{matplotlib}\PYG{n+nn}{.}\PYG{n+nn}{pyplot} \PYG{k}{as} \PYG{n+nn}{plt} \PYG{c+c1}{\PYGZsh{} To manipulate plots }
\end{sphinxVerbatim}

\end{sphinxuseclass}\end{sphinxVerbatimInput}

\end{sphinxuseclass}

\section{Class methods to help in Jupyter Notebook}
\label{\detokenize{content/notebooks/modelflow_features:class-methods-to-help-in-jupyter-notebook}}

\subsection{.widescreen() use Jupyter Notebook in widescreen}
\label{\detokenize{content/notebooks/modelflow_features:widescreen-use-jupyter-notebook-in-widescreen}}
\sphinxAtStartPar
Enables the whole viewing area of the browser.

\begin{sphinxuseclass}{cell}\begin{sphinxVerbatimInput}

\begin{sphinxuseclass}{cell_input}
\begin{sphinxVerbatim}[commandchars=\\\{\}]
\PYG{n}{model}\PYG{o}{.}\PYG{n}{widescreen}\PYG{p}{(}\PYG{p}{)} 
\end{sphinxVerbatim}

\end{sphinxuseclass}\end{sphinxVerbatimInput}
\begin{sphinxVerbatimOutput}

\begin{sphinxuseclass}{cell_output}
\begin{sphinxVerbatim}[commandchars=\\\{\}]
\PYGZlt{}IPython.core.display.HTML object\PYGZgt{}
\end{sphinxVerbatim}

\end{sphinxuseclass}\end{sphinxVerbatimOutput}

\end{sphinxuseclass}

\subsection{.scroll\_off() Turn off scroll cells in Jupyter Notebook}
\label{\detokenize{content/notebooks/modelflow_features:scroll-off-turn-off-scroll-cells-in-jupyter-notebook}}
\sphinxAtStartPar
Can be useful

\begin{sphinxuseclass}{cell}\begin{sphinxVerbatimInput}

\begin{sphinxuseclass}{cell_input}
\begin{sphinxVerbatim}[commandchars=\\\{\}]
\PYG{n}{model}\PYG{o}{.}\PYG{n}{scroll\PYGZus{}off}\PYG{p}{(}\PYG{p}{)}
\end{sphinxVerbatim}

\end{sphinxuseclass}\end{sphinxVerbatimInput}

\end{sphinxuseclass}

\section{.modelload Load a pre\sphinxhyphen{}cooked model, data and descriptions}
\label{\detokenize{content/notebooks/modelflow_features:modelload-load-a-pre-cooked-model-data-and-descriptions}}
\sphinxAtStartPar
In this notebook, we will be using a pre\sphinxhyphen{}existing  model of Pakistan.

\sphinxAtStartPar
The file ‘pak.pcim’ has been created from a Eviews workspace. It contains all that is needed to run the model:
\begin{itemize}
\item {} 
\sphinxAtStartPar
Model equations

\item {} 
\sphinxAtStartPar
Data

\item {} 
\sphinxAtStartPar
Simulation options

\item {} 
\sphinxAtStartPar
Variable descriptions

\end{itemize}

\sphinxAtStartPar
Using the ‘modelload’ method of the  ‘model’ class, a model instance ‘mpak’ and a ‘result’ DataFrame is created.

\begin{sphinxuseclass}{cell}\begin{sphinxVerbatimInput}

\begin{sphinxuseclass}{cell_input}
\begin{sphinxVerbatim}[commandchars=\\\{\}]
\PYG{n}{mpak}\PYG{p}{,}\PYG{n}{baseline} \PYG{o}{=} \PYG{n}{model}\PYG{o}{.}\PYG{n}{modelload}\PYG{p}{(}\PYG{l+s+s1}{\PYGZsq{}}\PYG{l+s+s1}{../models/pak.pcim}\PYG{l+s+s1}{\PYGZsq{}}\PYG{p}{,}\PYG{n}{run}\PYG{o}{=}\PYG{l+m+mi}{1}\PYG{p}{,}\PYG{n}{silent}\PYG{o}{=}\PYG{l+m+mi}{1}\PYG{p}{,}\PYG{n}{keep}\PYG{o}{=}\PYG{l+s+s1}{\PYGZsq{}}\PYG{l+s+s1}{Baseline}\PYG{l+s+s1}{\PYGZsq{}}\PYG{p}{)}
\end{sphinxVerbatim}

\end{sphinxuseclass}\end{sphinxVerbatimInput}
\begin{sphinxVerbatimOutput}

\begin{sphinxuseclass}{cell_output}
\begin{sphinxVerbatim}[commandchars=\\\{\}]
file read:  C:\PYGZbs{}modelflow manual\PYGZbs{}papers\PYGZbs{}mfbook\PYGZbs{}content\PYGZbs{}models\PYGZbs{}pak.pcim
\end{sphinxVerbatim}

\end{sphinxuseclass}\end{sphinxVerbatimOutput}

\end{sphinxuseclass}
\sphinxAtStartPar
\sphinxstylestrong{mpak} 
The \sphinxstyleemphasis{modelload} method processes the file and initiates the model, that we call ‘mpak’ (m for model and pak for Pakistan) with both equations and the data.

\sphinxAtStartPar
‘mpak’ is an \sphinxstylestrong{instance}  of the  model object with which we will work.

\sphinxAtStartPar
\sphinxstylestrong{baseline}  
‘result’ is a Pandas dataframe containing the data that was loaded.

\sphinxAtStartPar
\sphinxstylestrong{run=1} the model is simulated. The simulation timeframe  and options from the time the file where dumped will be used. The two objects \sphinxstylestrong{mpak.basedf} and \sphinxstylestrong{mpak.lastdf} will contain the simulation result. If run=0 the model will not be simulated.

\sphinxAtStartPar
\sphinxstylestrong{silent=1} if silent is set to 0  information regarding the simulation will be displayed.

\sphinxAtStartPar
\sphinxstylestrong{keep=’Baseline’} This saves the result in a dictionary mpak.keep\_solutions.


\section{Create a scenario}
\label{\detokenize{content/notebooks/modelflow_features:create-a-scenario}}
\sphinxAtStartPar
Many objects relates to comparison of different scenarios. So first a scenario is created by updating some exogenous variables.
In this case the carbon tax rates for gas, oil and coal are all set to 29 from 2023 to 2100. Then the scenario is simulated.
Now the mpak object contains a number of useful properties and methods.

\sphinxAtStartPar
You can find more on this experiment \DUrole{xref,myst}{here}

\begin{sphinxuseclass}{cell}\begin{sphinxVerbatimInput}

\begin{sphinxuseclass}{cell_input}
\begin{sphinxVerbatim}[commandchars=\\\{\}]
\PYG{n}{scenario\PYGZus{}exo}  \PYG{o}{=}  \PYG{n}{baseline}\PYG{o}{.}\PYG{n}{upd}\PYG{p}{(}\PYG{l+s+s2}{\PYGZdq{}}\PYG{l+s+s2}{\PYGZlt{}2020 2100\PYGZgt{} PAKGGREVCO2CER PAKGGREVCO2GER PAKGGREVCO2OER = 29}\PYG{l+s+s2}{\PYGZdq{}}\PYG{p}{)}
\end{sphinxVerbatim}

\end{sphinxuseclass}\end{sphinxVerbatimInput}

\end{sphinxuseclass}

\section{() Simulate on a dataframe}
\label{\detokenize{content/notebooks/modelflow_features:simulate-on-a-dataframe}}
\sphinxAtStartPar
When calling the model instance like \sphinxcode{\sphinxupquote{mpak(dataframe,start, end)}} the model will be simulated for the time frame \sphinxcode{\sphinxupquote{start to end}} using the dataframe.  
Just above we created a dataframe \sphinxcode{\sphinxupquote{scenario\_exo}} where the tax variables are updated. Now the \sphinxcode{\sphinxupquote{mpak}} can be simulated. We simulate from 2020 to 2100.

\begin{sphinxuseclass}{cell}\begin{sphinxVerbatimInput}

\begin{sphinxuseclass}{cell_input}
\begin{sphinxVerbatim}[commandchars=\\\{\}]
\PYG{n}{scenario} \PYG{o}{=} \PYG{n}{mpak}\PYG{p}{(}\PYG{n}{scenario\PYGZus{}exo}\PYG{p}{,}\PYG{l+m+mi}{2020}\PYG{p}{,}\PYG{l+m+mi}{2100}\PYG{p}{,}\PYG{n}{keep}\PYG{o}{=}\PYG{l+s+sa}{f}\PYG{l+s+s1}{\PYGZsq{}}\PYG{l+s+s1}{Coal, Oil and Gastax : 29}\PYG{l+s+s1}{\PYGZsq{}}\PYG{p}{)} \PYG{c+c1}{\PYGZsh{} runs the simulation}
\end{sphinxVerbatim}

\end{sphinxuseclass}\end{sphinxVerbatimInput}

\end{sphinxuseclass}

\section{Access results}
\label{\detokenize{content/notebooks/modelflow_features:access-results}}
\sphinxAtStartPar
Now we have two dataframes with results \sphinxcode{\sphinxupquote{baseline}} and \sphinxcode{\sphinxupquote{scenario}}. These dataframes can be manipulated and visualized
with the tools provided by the \sphinxstylestrong{pandas} library and other like \sphinxstylestrong{Matplotlib} and \sphinxstylestrong{Plotly}. However to make things easy the first and
latest simulation result is also in the mpak object:
\begin{itemize}
\item {} 
\sphinxAtStartPar
\sphinxstylestrong{mpak.basedf}: Dataframe with the values for baseline

\item {} 
\sphinxAtStartPar
\sphinxstylestrong{mpak.lastdf}: Dataframe with the values for alternative

\end{itemize}

\sphinxAtStartPar
This means that .basedf and .lastdf will contain the same result after the first simulation. 
If new scenarios are simulated the data in .lastdf will then be replaced with the latest results.

\sphinxAtStartPar
These dataframes are used by a number of model instance methods as you will see later.

\sphinxAtStartPar
The user can assign dataframes to both .basedf and .lastdf. This is useful for comparing simulations which are not the first and last.

\begin{sphinxuseclass}{cell}\begin{sphinxVerbatimInput}

\begin{sphinxuseclass}{cell_input}
\begin{sphinxVerbatim}[commandchars=\\\{\}]
\PYG{n+nb}{print}\PYG{p}{(}\PYG{l+s+sa}{f}\PYG{l+s+s1}{\PYGZsq{}}\PYG{l+s+s1}{mpak.basedf: Dataframe: with }\PYG{l+s+si}{\PYGZob{}}\PYG{n}{mpak}\PYG{o}{.}\PYG{n}{basedf}\PYG{o}{.}\PYG{n}{shape}\PYG{p}{[}\PYG{l+m+mi}{0}\PYG{p}{]}\PYG{l+s+si}{\PYGZcb{}}\PYG{l+s+s1}{ years and }\PYG{l+s+si}{\PYGZob{}}\PYG{n}{mpak}\PYG{o}{.}\PYG{n}{basedf}\PYG{o}{.}\PYG{n}{shape}\PYG{p}{[}\PYG{l+m+mi}{1}\PYG{p}{]}\PYG{l+s+si}{\PYGZcb{}}\PYG{l+s+s1}{ variables}\PYG{l+s+s1}{\PYGZsq{}}\PYG{p}{)}
\PYG{n+nb}{print}\PYG{p}{(}\PYG{l+s+sa}{f}\PYG{l+s+s1}{\PYGZsq{}}\PYG{l+s+s1}{mpak.lastdf: Dataframe: with }\PYG{l+s+si}{\PYGZob{}}\PYG{n}{mpak}\PYG{o}{.}\PYG{n}{lastdf}\PYG{o}{.}\PYG{n}{shape}\PYG{p}{[}\PYG{l+m+mi}{0}\PYG{p}{]}\PYG{l+s+si}{\PYGZcb{}}\PYG{l+s+s1}{ years and }\PYG{l+s+si}{\PYGZob{}}\PYG{n}{mpak}\PYG{o}{.}\PYG{n}{lastdf}\PYG{o}{.}\PYG{n}{shape}\PYG{p}{[}\PYG{l+m+mi}{1}\PYG{p}{]}\PYG{l+s+si}{\PYGZcb{}}\PYG{l+s+s1}{ variables}\PYG{l+s+s1}{\PYGZsq{}}\PYG{p}{)}
\end{sphinxVerbatim}

\end{sphinxuseclass}\end{sphinxVerbatimInput}
\begin{sphinxVerbatimOutput}

\begin{sphinxuseclass}{cell_output}
\begin{sphinxVerbatim}[commandchars=\\\{\}]
mpak.basedf: Dataframe: with 121 years and 1290 variables
mpak.lastdf: Dataframe: with 121 years and 1290 variables
\end{sphinxVerbatim}

\end{sphinxuseclass}\end{sphinxVerbatimOutput}

\end{sphinxuseclass}

\subsection{.keep\_solutions, A dictionary of dataframes with results}
\label{\detokenize{content/notebooks/modelflow_features:keep-solutions-a-dictionary-of-dataframes-with-results}}
\sphinxAtStartPar
Create a dictionary of dataframes with .keep\_solutions. Sometimes we want to be able to compare more than two scenarios. Using \sphinxcode{\sphinxupquote{keep='some description'}} the dataframe with results can be saved into a dictionary with the description as key and the dataframe as value.

\sphinxAtStartPar
In our example we have created two scenarios. A baseline and a scenario with the tax set to 29. So mpak.keep\_solutions looks like this:

\begin{sphinxuseclass}{cell}\begin{sphinxVerbatimInput}

\begin{sphinxuseclass}{cell_input}
\begin{sphinxVerbatim}[commandchars=\\\{\}]
\PYG{n+nb}{print}\PYG{p}{(}\PYG{l+s+s1}{\PYGZsq{}}\PYG{l+s+s1}{mpak.keep\PYGZus{}solutions contains:}\PYG{l+s+s1}{\PYGZsq{}}\PYG{p}{)}
\PYG{k}{for} \PYG{n}{key}\PYG{p}{,}\PYG{n}{value} \PYG{o+ow}{in} \PYG{n}{mpak}\PYG{o}{.}\PYG{n}{keep\PYGZus{}solutions}\PYG{o}{.}\PYG{n}{items}\PYG{p}{(}\PYG{p}{)}\PYG{p}{:} 
    \PYG{n+nb}{print}\PYG{p}{(}\PYG{l+s+sa}{f}\PYG{l+s+s1}{\PYGZsq{}}\PYG{l+s+s1}{key = }\PYG{l+s+si}{\PYGZob{}}\PYG{n}{key}\PYG{l+s+si}{:}\PYG{l+s+s1}{25}\PYG{l+s+si}{\PYGZcb{}}\PYG{l+s+s1}{|Dataframe: }\PYG{l+s+si}{\PYGZob{}}\PYG{n}{value}\PYG{o}{.}\PYG{n}{shape}\PYG{p}{[}\PYG{l+m+mi}{0}\PYG{p}{]}\PYG{l+s+si}{\PYGZcb{}}\PYG{l+s+s1}{ years and }\PYG{l+s+si}{\PYGZob{}}\PYG{n}{value}\PYG{o}{.}\PYG{n}{shape}\PYG{p}{[}\PYG{l+m+mi}{1}\PYG{p}{]}\PYG{l+s+si}{\PYGZcb{}}\PYG{l+s+s1}{ variables}\PYG{l+s+s1}{\PYGZsq{}}\PYG{p}{)}
\end{sphinxVerbatim}

\end{sphinxuseclass}\end{sphinxVerbatimInput}
\begin{sphinxVerbatimOutput}

\begin{sphinxuseclass}{cell_output}
\begin{sphinxVerbatim}[commandchars=\\\{\}]
mpak.keep\PYGZus{}solutions contains:
key = Baseline                 |Dataframe: 121 years and 1290 variables
key = Coal, Oil and Gastax : 29|Dataframe: 121 years and 1290 variables
\end{sphinxVerbatim}

\end{sphinxuseclass}\end{sphinxVerbatimOutput}

\end{sphinxuseclass}
\sphinxAtStartPar
Sometime it can be useful to reset the \sphinxcode{\sphinxupquote{.keep\_solutions}}, so that a new set of solutions can be inspected. This is done by replacing it with an empty dictionary. Two methods can be used:
\begin{quote}

\sphinxAtStartPar
mpak.keep\_solutions = \{\}
\end{quote}

\sphinxAtStartPar
or in the simulation call:
\begin{quote}

\sphinxAtStartPar
mpak(,,keep=’’)
\end{quote}


\subsection{More on manipulating keep\_solution:}
\label{\detokenize{content/notebooks/modelflow_features:more-on-manipulating-keep-solution}}
\sphinxAtStartPar
\DUrole{xref,myst}{Here}


\subsection{.oldkwargs, Options in the simulation call is persistent between calls}
\label{\detokenize{content/notebooks/modelflow_features:oldkwargs-options-in-the-simulation-call-is-persistent-between-calls}}
\sphinxAtStartPar
When simulating a model the parameters are persistent. So the user just have to provide the
solution options once. These persistent parameters are located in the property .oldkwargs.

\sphinxAtStartPar
In this case the persistent parameters are:

\begin{sphinxuseclass}{cell}\begin{sphinxVerbatimInput}

\begin{sphinxuseclass}{cell_input}
\begin{sphinxVerbatim}[commandchars=\\\{\}]
\PYG{n}{mpak}\PYG{o}{.}\PYG{n}{oldkwargs}
\end{sphinxVerbatim}

\end{sphinxuseclass}\end{sphinxVerbatimInput}
\begin{sphinxVerbatimOutput}

\begin{sphinxuseclass}{cell_output}
\begin{sphinxVerbatim}[commandchars=\\\{\}]
\PYGZob{}\PYGZsq{}silent\PYGZsq{}: 1, \PYGZsq{}alfa\PYGZsq{}: 0.7, \PYGZsq{}ldumpvar\PYGZsq{}: 0, \PYGZsq{}keep\PYGZsq{}: \PYGZsq{}Coal, Oil and Gastax : 29\PYGZsq{}\PYGZcb{}
\end{sphinxVerbatim}

\end{sphinxuseclass}\end{sphinxVerbatimOutput}

\end{sphinxuseclass}
\sphinxAtStartPar
The user may have to reset the parameters, this is done like this:

\sphinxAtStartPar
To reset the options just do:
\begin{quote}

\sphinxAtStartPar
mpak.oldkwargs = \{\}
\end{quote}


\section{.current\_per, The time frame operations are performed on}
\label{\detokenize{content/notebooks/modelflow_features:current-per-the-time-frame-operations-are-performed-on}}
\sphinxAtStartPar
Most operations on a model class instance operates on the current time frame.
It is a subset of the row index of the dataframe which is simulated.

\sphinxAtStartPar
In this case it is:

\begin{sphinxuseclass}{cell}\begin{sphinxVerbatimInput}

\begin{sphinxuseclass}{cell_input}
\begin{sphinxVerbatim}[commandchars=\\\{\}]
\PYG{n}{mpak}\PYG{o}{.}\PYG{n}{current\PYGZus{}per}
\end{sphinxVerbatim}

\end{sphinxuseclass}\end{sphinxVerbatimInput}
\begin{sphinxVerbatimOutput}

\begin{sphinxuseclass}{cell_output}
\begin{sphinxVerbatim}[commandchars=\\\{\}]
Int64Index([2020, 2021, 2022, 2023, 2024, 2025, 2026, 2027, 2028, 2029, 2030,
            2031, 2032, 2033, 2034, 2035, 2036, 2037, 2038, 2039, 2040, 2041,
            2042, 2043, 2044, 2045, 2046, 2047, 2048, 2049, 2050, 2051, 2052,
            2053, 2054, 2055, 2056, 2057, 2058, 2059, 2060, 2061, 2062, 2063,
            2064, 2065, 2066, 2067, 2068, 2069, 2070, 2071, 2072, 2073, 2074,
            2075, 2076, 2077, 2078, 2079, 2080, 2081, 2082, 2083, 2084, 2085,
            2086, 2087, 2088, 2089, 2090, 2091, 2092, 2093, 2094, 2095, 2096,
            2097, 2098, 2099, 2100],
           dtype=\PYGZsq{}int64\PYGZsq{})
\end{sphinxVerbatim}

\end{sphinxuseclass}\end{sphinxVerbatimOutput}

\end{sphinxuseclass}
\sphinxAtStartPar
The possible times in the dataframe is contained in the \sphinxcode{\sphinxupquote{<dataframe>.index}} property.

\begin{sphinxuseclass}{cell}\begin{sphinxVerbatimInput}

\begin{sphinxuseclass}{cell_input}
\begin{sphinxVerbatim}[commandchars=\\\{\}]
\PYG{n}{scenario}\PYG{o}{.}\PYG{n}{index}  \PYG{c+c1}{\PYGZsh{} the index of the dataframe}
\end{sphinxVerbatim}

\end{sphinxuseclass}\end{sphinxVerbatimInput}
\begin{sphinxVerbatimOutput}

\begin{sphinxuseclass}{cell_output}
\begin{sphinxVerbatim}[commandchars=\\\{\}]
Int64Index([1980, 1981, 1982, 1983, 1984, 1985, 1986, 1987, 1988, 1989,
            ...
            2091, 2092, 2093, 2094, 2095, 2096, 2097, 2098, 2099, 2100],
           dtype=\PYGZsq{}int64\PYGZsq{}, length=121)
\end{sphinxVerbatim}

\end{sphinxuseclass}\end{sphinxVerbatimOutput}

\end{sphinxuseclass}

\subsection{.smpl, Set time frame}
\label{\detokenize{content/notebooks/modelflow_features:smpl-set-time-frame}}
\sphinxAtStartPar
The time frame can be set like this:

\begin{sphinxuseclass}{cell}\begin{sphinxVerbatimInput}

\begin{sphinxuseclass}{cell_input}
\begin{sphinxVerbatim}[commandchars=\\\{\}]
\PYG{n}{mpak}\PYG{o}{.}\PYG{n}{smpl}\PYG{p}{(}\PYG{l+m+mi}{2020}\PYG{p}{,}\PYG{l+m+mi}{2025}\PYG{p}{)}
\PYG{n}{mpak}\PYG{o}{.}\PYG{n}{current\PYGZus{}per}
\end{sphinxVerbatim}

\end{sphinxuseclass}\end{sphinxVerbatimInput}
\begin{sphinxVerbatimOutput}

\begin{sphinxuseclass}{cell_output}
\begin{sphinxVerbatim}[commandchars=\\\{\}]
Int64Index([2020, 2021, 2022, 2023, 2024, 2025], dtype=\PYGZsq{}int64\PYGZsq{})
\end{sphinxVerbatim}

\end{sphinxuseclass}\end{sphinxVerbatimOutput}

\end{sphinxuseclass}

\subsection{.set\_smpl, Set timeframe for a local scope}
\label{\detokenize{content/notebooks/modelflow_features:set-smpl-set-timeframe-for-a-local-scope}}
\sphinxAtStartPar
For many operations it can be useful to apply the operations for a shorter time frame, but retain the global time frame after the operation. 
This can be done  with a \sphinxcode{\sphinxupquote{with}} statement like this.

\begin{sphinxuseclass}{cell}\begin{sphinxVerbatimInput}

\begin{sphinxuseclass}{cell_input}
\begin{sphinxVerbatim}[commandchars=\\\{\}]
\PYG{n+nb}{print}\PYG{p}{(}\PYG{l+s+sa}{f}\PYG{l+s+s1}{\PYGZsq{}}\PYG{l+s+s1}{Global time  before   }\PYG{l+s+si}{\PYGZob{}}\PYG{n}{mpak}\PYG{o}{.}\PYG{n}{current\PYGZus{}per}\PYG{l+s+si}{\PYGZcb{}}\PYG{l+s+s1}{\PYGZsq{}}\PYG{p}{)}
\PYG{k}{with} \PYG{n}{mpak}\PYG{o}{.}\PYG{n}{set\PYGZus{}smpl}\PYG{p}{(}\PYG{l+m+mi}{2022}\PYG{p}{,}\PYG{l+m+mi}{2023}\PYG{p}{)}\PYG{p}{:}
    \PYG{n+nb}{print}\PYG{p}{(}\PYG{l+s+sa}{f}\PYG{l+s+s1}{\PYGZsq{}}\PYG{l+s+s1}{Local time frame      }\PYG{l+s+si}{\PYGZob{}}\PYG{n}{mpak}\PYG{o}{.}\PYG{n}{current\PYGZus{}per}\PYG{l+s+si}{\PYGZcb{}}\PYG{l+s+s1}{\PYGZsq{}}\PYG{p}{)}
\PYG{n+nb}{print}\PYG{p}{(}\PYG{l+s+sa}{f}\PYG{l+s+s1}{\PYGZsq{}}\PYG{l+s+s1}{Unchanged global time }\PYG{l+s+si}{\PYGZob{}}\PYG{n}{mpak}\PYG{o}{.}\PYG{n}{current\PYGZus{}per}\PYG{l+s+si}{\PYGZcb{}}\PYG{l+s+s1}{\PYGZsq{}}\PYG{p}{)}
\end{sphinxVerbatim}

\end{sphinxuseclass}\end{sphinxVerbatimInput}
\begin{sphinxVerbatimOutput}

\begin{sphinxuseclass}{cell_output}
\begin{sphinxVerbatim}[commandchars=\\\{\}]
Global time  before   Int64Index([2020, 2021, 2022, 2023, 2024, 2025], dtype=\PYGZsq{}int64\PYGZsq{})
Local time frame      Int64Index([2022, 2023], dtype=\PYGZsq{}int64\PYGZsq{})
Unchanged global time Int64Index([2020, 2021, 2022, 2023, 2024, 2025], dtype=\PYGZsq{}int64\PYGZsq{})
\end{sphinxVerbatim}

\end{sphinxuseclass}\end{sphinxVerbatimOutput}

\end{sphinxuseclass}

\subsection{.set\_smpl\_relative Set relative timeframe for a local scope}
\label{\detokenize{content/notebooks/modelflow_features:set-smpl-relative-set-relative-timeframe-for-a-local-scope}}
\sphinxAtStartPar
When creating a script it can be useful to set the time frame relative to the
current time.

\sphinxAtStartPar
Like this:

\begin{sphinxuseclass}{cell}\begin{sphinxVerbatimInput}

\begin{sphinxuseclass}{cell_input}
\begin{sphinxVerbatim}[commandchars=\\\{\}]
\PYG{n+nb}{print}\PYG{p}{(}\PYG{l+s+sa}{f}\PYG{l+s+s1}{\PYGZsq{}}\PYG{l+s+s1}{Global time  before   }\PYG{l+s+si}{\PYGZob{}}\PYG{n}{mpak}\PYG{o}{.}\PYG{n}{current\PYGZus{}per}\PYG{l+s+si}{\PYGZcb{}}\PYG{l+s+s1}{\PYGZsq{}}\PYG{p}{)}
\PYG{k}{with} \PYG{n}{mpak}\PYG{o}{.}\PYG{n}{set\PYGZus{}smpl\PYGZus{}relative} \PYG{p}{(}\PYG{o}{\PYGZhy{}}\PYG{l+m+mi}{1}\PYG{p}{,}\PYG{l+m+mi}{0}\PYG{p}{)}\PYG{p}{:}
    \PYG{n+nb}{print}\PYG{p}{(}\PYG{l+s+sa}{f}\PYG{l+s+s1}{\PYGZsq{}}\PYG{l+s+s1}{Local time frame      }\PYG{l+s+si}{\PYGZob{}}\PYG{n}{mpak}\PYG{o}{.}\PYG{n}{current\PYGZus{}per}\PYG{l+s+si}{\PYGZcb{}}\PYG{l+s+s1}{\PYGZsq{}}\PYG{p}{)}
\PYG{n+nb}{print}\PYG{p}{(}\PYG{l+s+sa}{f}\PYG{l+s+s1}{\PYGZsq{}}\PYG{l+s+s1}{Unchanged global time }\PYG{l+s+si}{\PYGZob{}}\PYG{n}{mpak}\PYG{o}{.}\PYG{n}{current\PYGZus{}per}\PYG{l+s+si}{\PYGZcb{}}\PYG{l+s+s1}{\PYGZsq{}}\PYG{p}{)}
\end{sphinxVerbatim}

\end{sphinxuseclass}\end{sphinxVerbatimInput}
\begin{sphinxVerbatimOutput}

\begin{sphinxuseclass}{cell_output}
\begin{sphinxVerbatim}[commandchars=\\\{\}]
Global time  before   Int64Index([2020, 2021, 2022, 2023, 2024, 2025], dtype=\PYGZsq{}int64\PYGZsq{})
Local time frame      Int64Index([2019, 2020, 2021, 2022, 2023, 2024, 2025], dtype=\PYGZsq{}int64\PYGZsq{})
Unchanged global time Int64Index([2020, 2021, 2022, 2023, 2024, 2025], dtype=\PYGZsq{}int64\PYGZsq{})
\end{sphinxVerbatim}

\end{sphinxuseclass}\end{sphinxVerbatimOutput}

\end{sphinxuseclass}

\section{Using the index operator {[} {]} to select and visualize variables.}
\label{\detokenize{content/notebooks/modelflow_features:using-the-index-operator-to-select-and-visualize-variables}}\label{\detokenize{content/notebooks/modelflow_features:index-operator}}
\sphinxAtStartPar
The index operator {[} {]} can be used to select variables and then process the values for quick analysis.

\sphinxAtStartPar
To select variables the method accept patterns which defines variable names. Wildcards:
\begin{itemize}
\item {} 
\sphinxAtStartPar
\sphinxcode{\sphinxupquote{\textbackslash{}*}} matches everything

\item {} 
\sphinxAtStartPar
\sphinxcode{\sphinxupquote{?}} matches any single character

\item {} 
\sphinxAtStartPar
\sphinxcode{\sphinxupquote{\textbackslash{}{[}seq{]}}} matches any character in seq

\item {} 
\sphinxAtStartPar
\sphinxcode{\sphinxupquote{\textbackslash{}{[}!seq{]}}} matches any character not in seq

\end{itemize}

\sphinxAtStartPar
For more how wildcards can be used, the specification can be found here (https://docs.python.org/3/library/fnmatch.html)

\sphinxAtStartPar
In the following example we are selecting the results of mpak{[}‘PAKNYGDPMKTPKN’{]}

\sphinxAtStartPar
This call will return a special class (called \sphinxcode{\sphinxupquote{vis}}). It implements a number
of methods and properties which comes in handy for quick analyses.

\sphinxAtStartPar
Several properties and methods can be chained. An example:

\begin{sphinxuseclass}{cell}\begin{sphinxVerbatimInput}

\begin{sphinxuseclass}{cell_input}
\begin{sphinxVerbatim}[commandchars=\\\{\}]
\PYG{k}{with} \PYG{n}{mpak}\PYG{o}{.}\PYG{n}{set\PYGZus{}smpl}\PYG{p}{(}\PYG{l+m+mi}{2020}\PYG{p}{,}\PYG{l+m+mi}{2100}\PYG{p}{)}\PYG{p}{:}
    \PYG{n}{mpak}\PYG{p}{[}\PYG{l+s+s1}{\PYGZsq{}}\PYG{l+s+s1}{PAKNYGDPMKTPKN}\PYG{l+s+s1}{\PYGZsq{}}\PYG{p}{]}\PYG{o}{.}\PYG{n}{difpctlevel}\PYG{o}{.}\PYG{n}{mul100}\PYG{o}{.}\PYG{n}{rename}\PYG{p}{(}\PYG{p}{)}\PYG{o}{.}\PYG{n}{plot}\PYG{p}{(}\PYG{n}{colrow}\PYG{o}{=}\PYG{l+m+mi}{1}\PYG{p}{,}
                \PYG{n}{title}\PYG{o}{=}\PYG{l+s+s1}{\PYGZsq{}}\PYG{l+s+s1}{Difference to baseline in percent}\PYG{l+s+s1}{\PYGZsq{}}\PYG{p}{,}\PYG{n}{top}\PYG{o}{=}\PYG{l+m+mf}{0.8}\PYG{p}{)}\PYG{p}{;}
\end{sphinxVerbatim}

\end{sphinxuseclass}\end{sphinxVerbatimInput}

\end{sphinxuseclass}
\sphinxAtStartPar
But first some basic information


\subsection{model{[}‘\#ENDO’{]}}
\label{\detokenize{content/notebooks/modelflow_features:model-endo}}
\sphinxAtStartPar
Use ‘\#ENDO’ to access all endogenous variables in your model instance.

\sphinxAtStartPar
For the sake of space, the result is saved in the variable ‘allendo’ and not printed.

\begin{sphinxuseclass}{cell}\begin{sphinxVerbatimInput}

\begin{sphinxuseclass}{cell_input}
\begin{sphinxVerbatim}[commandchars=\\\{\}]
\PYG{n}{allendo} \PYG{o}{=} \PYG{n}{mpak}\PYG{p}{[}\PYG{l+s+s1}{\PYGZsq{}}\PYG{l+s+s1}{\PYGZsh{}ENDO}\PYG{l+s+s1}{\PYGZsq{}}\PYG{p}{]}
\PYG{c+c1}{\PYGZsh{} allendo.show}
\end{sphinxVerbatim}

\end{sphinxuseclass}\end{sphinxVerbatimInput}

\end{sphinxuseclass}

\subsection{Access values in .lastdf and .basedf}
\label{\detokenize{content/notebooks/modelflow_features:access-values-in-lastdf-and-basedf}}
\sphinxAtStartPar
To limit the output printed, we set the time frame to 2020 to 2023.

\begin{sphinxuseclass}{cell}\begin{sphinxVerbatimInput}

\begin{sphinxuseclass}{cell_input}
\begin{sphinxVerbatim}[commandchars=\\\{\}]
\PYG{n}{mpak}\PYG{o}{.}\PYG{n}{smpl}\PYG{p}{(}\PYG{l+m+mi}{2020}\PYG{p}{,}\PYG{l+m+mi}{2023}\PYG{p}{)}\PYG{p}{;}
\end{sphinxVerbatim}

\end{sphinxuseclass}\end{sphinxVerbatimInput}

\end{sphinxuseclass}
\sphinxAtStartPar
To access the values of ‘PAKNYGDPMKTPKN’ and ‘PAKNECONPRVTKN’ from the latest simulation a small widget is displayed.

\begin{sphinxuseclass}{cell}\begin{sphinxVerbatimInput}

\begin{sphinxuseclass}{cell_input}
\begin{sphinxVerbatim}[commandchars=\\\{\}]
\PYG{n}{mpak}\PYG{p}{[}\PYG{l+s+s1}{\PYGZsq{}}\PYG{l+s+s1}{PAKNYGDPMKTPKN PAKNECONPRVTKN}\PYG{l+s+s1}{\PYGZsq{}}\PYG{p}{]} 
\end{sphinxVerbatim}

\end{sphinxuseclass}\end{sphinxVerbatimInput}
\begin{sphinxVerbatimOutput}

\begin{sphinxuseclass}{cell_output}
\begin{sphinxVerbatim}[commandchars=\\\{\}]
Tab(children=(Tab(children=(HTML(value=\PYGZsq{}\PYGZlt{}?xml version=\PYGZdq{}1.0\PYGZdq{} encoding=\PYGZdq{}utf\PYGZhy{}8\PYGZdq{} standalone=\PYGZdq{}no\PYGZdq{}?\PYGZgt{}\PYGZbs{}n\PYGZlt{}!DOCTYPE svg …
\end{sphinxVerbatim}

\begin{sphinxVerbatim}[commandchars=\\\{\}]

\end{sphinxVerbatim}

\end{sphinxuseclass}\end{sphinxVerbatimOutput}

\end{sphinxuseclass}
\sphinxAtStartPar
To access the values of ‘PAKNYGDPMKTPKN’ and ‘PAKNECONPRVTKN’ from the base dataframe, specify .base

\begin{sphinxuseclass}{cell}\begin{sphinxVerbatimInput}

\begin{sphinxuseclass}{cell_input}
\begin{sphinxVerbatim}[commandchars=\\\{\}]
\PYG{n}{mpak}\PYG{p}{[}\PYG{l+s+s1}{\PYGZsq{}}\PYG{l+s+s1}{PAKNYGDPMKTPKN PAKNECONPRVTKN}\PYG{l+s+s1}{\PYGZsq{}}\PYG{p}{]}\PYG{o}{.}\PYG{n}{base}\PYG{o}{.}\PYG{n}{df} 
\end{sphinxVerbatim}

\end{sphinxuseclass}\end{sphinxVerbatimInput}
\begin{sphinxVerbatimOutput}

\begin{sphinxuseclass}{cell_output}
\begin{sphinxVerbatim}[commandchars=\\\{\}]
      PAKNYGDPMKTPKN  PAKNECONPRVTKN
2020    2.627394e+07    2.367289e+07
2021    2.651137e+07    2.397282e+07
2022    2.668514e+07    2.416413e+07
2023    2.696308e+07    2.442786e+07
\end{sphinxVerbatim}

\end{sphinxuseclass}\end{sphinxVerbatimOutput}

\end{sphinxuseclass}

\subsection{.df  Pandas dataframe}
\label{\detokenize{content/notebooks/modelflow_features:df-pandas-dataframe}}
\sphinxAtStartPar
Sometime you need to perform additional operations on the values. Therefor the .df will return a dataframe with the selected variables.

\begin{sphinxuseclass}{cell}\begin{sphinxVerbatimInput}

\begin{sphinxuseclass}{cell_input}
\begin{sphinxVerbatim}[commandchars=\\\{\}]
\PYG{n}{mpak}\PYG{p}{[}\PYG{l+s+s1}{\PYGZsq{}}\PYG{l+s+s1}{PAKNYGDPMKTPKN PAKNECONPRVTKN}\PYG{l+s+s1}{\PYGZsq{}}\PYG{p}{]}\PYG{o}{.}\PYG{n}{df}
\end{sphinxVerbatim}

\end{sphinxuseclass}\end{sphinxVerbatimInput}
\begin{sphinxVerbatimOutput}

\begin{sphinxuseclass}{cell_output}
\begin{sphinxVerbatim}[commandchars=\\\{\}]
      PAKNYGDPMKTPKN  PAKNECONPRVTKN
2020    2.647002e+07    2.344055e+07
2021    2.676493e+07    2.366076e+07
2022    2.688965e+07    2.376966e+07
2023    2.708904e+07    2.395330e+07
\end{sphinxVerbatim}

\end{sphinxuseclass}\end{sphinxVerbatimOutput}

\end{sphinxuseclass}

\subsection{.show  as a html table with tooltips}
\label{\detokenize{content/notebooks/modelflow_features:show-as-a-html-table-with-tooltips}}
\sphinxAtStartPar
If you want the variable descriptions use this

\begin{sphinxuseclass}{cell}\begin{sphinxVerbatimInput}

\begin{sphinxuseclass}{cell_input}
\begin{sphinxVerbatim}[commandchars=\\\{\}]
\PYG{n}{mpak}\PYG{p}{[}\PYG{l+s+s1}{\PYGZsq{}}\PYG{l+s+s1}{PAKNYGDPMKTPKN PAKNECONPRVTKN}\PYG{l+s+s1}{\PYGZsq{}}\PYG{p}{]}\PYG{o}{.}\PYG{n}{show}
\end{sphinxVerbatim}

\end{sphinxuseclass}\end{sphinxVerbatimInput}
\begin{sphinxVerbatimOutput}

\begin{sphinxuseclass}{cell_output}
\begin{sphinxVerbatim}[commandchars=\\\{\}]
Tab(children=(Tab(children=(HTML(value=\PYGZsq{}\PYGZlt{}?xml version=\PYGZdq{}1.0\PYGZdq{} encoding=\PYGZdq{}utf\PYGZhy{}8\PYGZdq{} standalone=\PYGZdq{}no\PYGZdq{}?\PYGZgt{}\PYGZbs{}n\PYGZlt{}!DOCTYPE svg …
\end{sphinxVerbatim}

\end{sphinxuseclass}\end{sphinxVerbatimOutput}

\end{sphinxuseclass}

\subsection{.names Variable names}
\label{\detokenize{content/notebooks/modelflow_features:names-variable-names}}
\sphinxAtStartPar
If you select variables using wildcards, then you can access the names that correspond to your query.

\begin{sphinxuseclass}{cell}\begin{sphinxVerbatimInput}

\begin{sphinxuseclass}{cell_input}
\begin{sphinxVerbatim}[commandchars=\\\{\}]
\PYG{n}{mpak}\PYG{p}{[}\PYG{l+s+s1}{\PYGZsq{}}\PYG{l+s+s1}{PAKNYGDP??????}\PYG{l+s+s1}{\PYGZsq{}}\PYG{p}{]}\PYG{o}{.}\PYG{n}{names}
\end{sphinxVerbatim}

\end{sphinxuseclass}\end{sphinxVerbatimInput}
\begin{sphinxVerbatimOutput}

\begin{sphinxuseclass}{cell_output}
\begin{sphinxVerbatim}[commandchars=\\\{\}]
[\PYGZsq{}PAKNYGDPDISCCN\PYGZsq{},
 \PYGZsq{}PAKNYGDPDISCKN\PYGZsq{},
 \PYGZsq{}PAKNYGDPFCSTCN\PYGZsq{},
 \PYGZsq{}PAKNYGDPFCSTKN\PYGZsq{},
 \PYGZsq{}PAKNYGDPFCSTXN\PYGZsq{},
 \PYGZsq{}PAKNYGDPMKTPCD\PYGZsq{},
 \PYGZsq{}PAKNYGDPMKTPCN\PYGZsq{},
 \PYGZsq{}PAKNYGDPMKTPKD\PYGZsq{},
 \PYGZsq{}PAKNYGDPMKTPKN\PYGZsq{},
 \PYGZsq{}PAKNYGDPMKTPXN\PYGZsq{},
 \PYGZsq{}PAKNYGDPPOTLKN\PYGZsq{}]
\end{sphinxVerbatim}

\end{sphinxuseclass}\end{sphinxVerbatimOutput}

\end{sphinxuseclass}

\subsection{.frml The formulas}
\label{\detokenize{content/notebooks/modelflow_features:frml-the-formulas}}
\sphinxAtStartPar
Use .frml to access all the equations for the endogenous variables.

\begin{sphinxuseclass}{cell}\begin{sphinxVerbatimInput}

\begin{sphinxuseclass}{cell_input}
\begin{sphinxVerbatim}[commandchars=\\\{\}]
\PYG{n}{mpak}\PYG{p}{[}\PYG{l+s+s1}{\PYGZsq{}}\PYG{l+s+s1}{PAKNYGDPMKTPKN PAKNECONPRVTKN}\PYG{l+s+s1}{\PYGZsq{}}\PYG{p}{]}\PYG{o}{.}\PYG{n}{frml}
\end{sphinxVerbatim}

\end{sphinxuseclass}\end{sphinxVerbatimInput}
\begin{sphinxVerbatimOutput}

\begin{sphinxuseclass}{cell_output}
\begin{sphinxVerbatim}[commandchars=\\\{\}]
PAKNYGDPMKTPKN : FRML \PYGZlt{}\PYGZgt{} PAKNYGDPMKTPKN = PAKNECONPRVTKN+PAKNECONGOVTKN+PAKNEGDIFTOTKN+PAKNEGDISTKBKN+PAKNEEXPGNFSKN\PYGZhy{}PAKNEIMPGNFSKN+PAKNYGDPDISCKN+PAKADAP*PAKDISPREPKN \PYGZdl{}
PAKNECONPRVTKN : FRML \PYGZlt{}Z,EXO\PYGZgt{} PAKNECONPRVTKN = (PAKNECONPRVTKN(\PYGZhy{}1)*EXP(PAKNECONPRVTKN\PYGZus{}A+ (\PYGZhy{}0.2*(LOG(PAKNECONPRVTKN(\PYGZhy{}1))\PYGZhy{}LOG(1.21203101101442)\PYGZhy{}LOG((((PAKBXFSTREMTCD(\PYGZhy{}1)\PYGZhy{}PAKBMFSTREMTCD(\PYGZhy{}1))*PAKPANUSATLS(\PYGZhy{}1))+PAKGGEXPTRNSCN(\PYGZhy{}1)+PAKNYYWBTOTLCN(\PYGZhy{}1)*(1\PYGZhy{}PAKGGREVDRCTXN(\PYGZhy{}1)/100))/PAKNECONPRVTXN(\PYGZhy{}1)))+0.763938860758873*((LOG((((PAKBXFSTREMTCD\PYGZhy{}PAKBMFSTREMTCD)*PAKPANUSATLS)+PAKGGEXPTRNSCN+PAKNYYWBTOTLCN*(1\PYGZhy{}PAKGGREVDRCTXN/100))/PAKNECONPRVTXN))\PYGZhy{}(LOG((((PAKBXFSTREMTCD(\PYGZhy{}1)\PYGZhy{}PAKBMFSTREMTCD(\PYGZhy{}1))*PAKPANUSATLS(\PYGZhy{}1))+PAKGGEXPTRNSCN(\PYGZhy{}1)+PAKNYYWBTOTLCN(\PYGZhy{}1)*(1\PYGZhy{}PAKGGREVDRCTXN(\PYGZhy{}1)/100))/PAKNECONPRVTXN(\PYGZhy{}1))))\PYGZhy{}0.0634474791568939*DURING\PYGZus{}2009\PYGZhy{}0.3*(PAKFMLBLPOLYXN/100\PYGZhy{}((LOG(PAKNECONPRVTXN))\PYGZhy{}(LOG(PAKNECONPRVTXN(\PYGZhy{}1)))))) )) * (1\PYGZhy{}PAKNECONPRVTKN\PYGZus{}D)+ PAKNECONPRVTKN\PYGZus{}X*PAKNECONPRVTKN\PYGZus{}D \PYGZdl{}
\end{sphinxVerbatim}

\end{sphinxuseclass}\end{sphinxVerbatimOutput}

\end{sphinxuseclass}

\subsection{.rename() Rename variables to descriptions}
\label{\detokenize{content/notebooks/modelflow_features:rename-rename-variables-to-descriptions}}
\sphinxAtStartPar
Use .rename() to assign variable descriptions as variable names.

\sphinxAtStartPar
Handy when plotting!

\begin{sphinxuseclass}{cell}\begin{sphinxVerbatimInput}

\begin{sphinxuseclass}{cell_input}
\begin{sphinxVerbatim}[commandchars=\\\{\}]
\PYG{n}{mpak}\PYG{p}{[}\PYG{l+s+s1}{\PYGZsq{}}\PYG{l+s+s1}{PAKNYGDPMKTPKN PAKNECONPRVTKN}\PYG{l+s+s1}{\PYGZsq{}}\PYG{p}{]}\PYG{o}{.}\PYG{n}{rename}\PYG{p}{(}\PYG{p}{)}\PYG{o}{.}\PYG{n}{df}
\end{sphinxVerbatim}

\end{sphinxuseclass}\end{sphinxVerbatimInput}
\begin{sphinxVerbatimOutput}

\begin{sphinxuseclass}{cell_output}
\begin{sphinxVerbatim}[commandchars=\\\{\}]
          Real GDP  HH. Cons Real
2020  2.647002e+07   2.344055e+07
2021  2.676493e+07   2.366076e+07
2022  2.688965e+07   2.376966e+07
2023  2.708904e+07   2.395330e+07
\end{sphinxVerbatim}

\end{sphinxuseclass}\end{sphinxVerbatimOutput}

\end{sphinxuseclass}

\subsection{Transformations of solution results}
\label{\detokenize{content/notebooks/modelflow_features:transformations-of-solution-results}}
\sphinxAtStartPar
When the variables has been selected through the index operator a number of standard data transformations can be performed.


\begin{savenotes}\sphinxattablestart
\centering
\begin{tabulary}{\linewidth}[t]{|T|T|T|}
\hline
\sphinxstyletheadfamily 
\sphinxAtStartPar
Transfomation
&\sphinxstyletheadfamily 
\sphinxAtStartPar
Meaning
&\sphinxstyletheadfamily 
\sphinxAtStartPar
expression
\\
\hline
\sphinxAtStartPar
pct
&
\sphinxAtStartPar
Growth rates
&
\sphinxAtStartPar
\(\left(\cfrac{this_t}{this_{t-1}}-1\right )\)
\\
\hline
\sphinxAtStartPar
dif
&
\sphinxAtStartPar
Difference in level
&
\sphinxAtStartPar
\(l-b\)
\\
\hline
\sphinxAtStartPar
difpct
&
\sphinxAtStartPar
Differens in growth rate
&
\sphinxAtStartPar
\(\left( \cfrac{l_t}{l_{t-1}}-1 \right) - \left(\cfrac{b_t}{b_{t-1}}-1 \right)\)
\\
\hline
\sphinxAtStartPar
difpctlevel
&
\sphinxAtStartPar
differens in level in pct of baseline
&
\sphinxAtStartPar
\(\left( \cfrac{l_t-b_t}{b_{t}} \right) \)
\\
\hline
\sphinxAtStartPar
mul100
&
\sphinxAtStartPar
multiply by 100
&
\sphinxAtStartPar
\(this_t \times 100\)
\\
\hline
\end{tabulary}
\par
\sphinxattableend\end{savenotes}
\begin{itemize}
\item {} 
\sphinxAtStartPar
\(this\) is the chained value. Default lastdf but if preseeded by .base the values from .basedf will be used

\item {} 
\sphinxAtStartPar
\(b\) is the values from .basedf

\item {} 
\sphinxAtStartPar
\(l\) is the values from .lastdf

\end{itemize}


\subsection{.dif Difference in level}
\label{\detokenize{content/notebooks/modelflow_features:dif-difference-in-level}}
\sphinxAtStartPar
The ‘dif’ command displays the difference in levels of the latest and previous solutions.

\sphinxAtStartPar
\(l-b\)

\sphinxAtStartPar
where l is the variable from the .lastdf and b is the variable from .basedf.

\begin{sphinxuseclass}{cell}\begin{sphinxVerbatimInput}

\begin{sphinxuseclass}{cell_input}
\begin{sphinxVerbatim}[commandchars=\\\{\}]
\PYG{n}{mpak}\PYG{p}{[}\PYG{l+s+s1}{\PYGZsq{}}\PYG{l+s+s1}{PAKNYGDPMKTPKN PAKNECONPRVTKN}\PYG{l+s+s1}{\PYGZsq{}}\PYG{p}{]}\PYG{o}{.}\PYG{n}{dif}\PYG{o}{.}\PYG{n}{plot}\PYG{p}{(}\PYG{p}{)}
\end{sphinxVerbatim}

\end{sphinxuseclass}\end{sphinxVerbatimInput}
\begin{sphinxVerbatimOutput}

\begin{sphinxuseclass}{cell_output}
\noindent\sphinxincludegraphics{{2013e08f59cf2fda412b98a62255965110e3146d12ddcc13ecbf1cb343875952}.png}

\end{sphinxuseclass}\end{sphinxVerbatimOutput}

\end{sphinxuseclass}

\subsection{.pct  Growthrates}
\label{\detokenize{content/notebooks/modelflow_features:pct-growthrates}}
\sphinxAtStartPar
Display growth rates

\sphinxAtStartPar
\(\left(\cfrac{l_t}{l_{t-1}}-1\right )\)

\begin{sphinxuseclass}{cell}\begin{sphinxVerbatimInput}

\begin{sphinxuseclass}{cell_input}
\begin{sphinxVerbatim}[commandchars=\\\{\}]
\PYG{n}{mpak}\PYG{p}{[}\PYG{l+s+s1}{\PYGZsq{}}\PYG{l+s+s1}{PAKNYGDPMKTPKN PAKNECONPRVTKN}\PYG{l+s+s1}{\PYGZsq{}}\PYG{p}{]}\PYG{o}{.}\PYG{n}{pct}\PYG{o}{.}\PYG{n}{plot}\PYG{p}{(}\PYG{p}{)}\PYG{p}{;}
\end{sphinxVerbatim}

\end{sphinxuseclass}\end{sphinxVerbatimInput}

\end{sphinxuseclass}

\subsection{.difpct property difference in growthrate}
\label{\detokenize{content/notebooks/modelflow_features:difpct-property-difference-in-growthrate}}
\sphinxAtStartPar
The difference in the growth rates  between the last and base dataframe.

\sphinxAtStartPar
\(\left( \cfrac{l_t}{l_{t-1}}-1 \right) - \left(\cfrac{b_t}{b_{t-1}}-1 \right)\)

\begin{sphinxuseclass}{cell}\begin{sphinxVerbatimInput}

\begin{sphinxuseclass}{cell_input}
\begin{sphinxVerbatim}[commandchars=\\\{\}]
\PYG{n}{mpak}\PYG{p}{[}\PYG{l+s+s1}{\PYGZsq{}}\PYG{l+s+s1}{PAKNYGDPMKTPKN PAKNECONPRVTKN}\PYG{l+s+s1}{\PYGZsq{}}\PYG{p}{]}\PYG{o}{.}\PYG{n}{difpct}\PYG{o}{.}\PYG{n}{plot}\PYG{p}{(}\PYG{p}{)} \PYG{p}{;} 
\end{sphinxVerbatim}

\end{sphinxuseclass}\end{sphinxVerbatimInput}

\end{sphinxuseclass}

\subsection{.difpctlevel percent difference of  levels}
\label{\detokenize{content/notebooks/modelflow_features:difpctlevel-percent-difference-of-levels}}
\sphinxAtStartPar
\(\left( \cfrac{l_t-b_t}{b_{t}} \right) \)

\begin{sphinxuseclass}{cell}\begin{sphinxVerbatimInput}

\begin{sphinxuseclass}{cell_input}
\begin{sphinxVerbatim}[commandchars=\\\{\}]
\PYG{n}{mpak}\PYG{p}{[}\PYG{l+s+s1}{\PYGZsq{}}\PYG{l+s+s1}{PAKNYGDPMKTPKN PAKNECONPRVTKN}\PYG{l+s+s1}{\PYGZsq{}}\PYG{p}{]}\PYG{o}{.}\PYG{n}{difpctlevel}\PYG{o}{.}\PYG{n}{plot}\PYG{p}{(}\PYG{p}{)}\PYG{p}{;}  
\end{sphinxVerbatim}

\end{sphinxuseclass}\end{sphinxVerbatimInput}

\end{sphinxuseclass}

\subsection{mul100 multiply by 100}
\label{\detokenize{content/notebooks/modelflow_features:mul100-multiply-by-100}}
\sphinxAtStartPar
multiply growth rate by 100.

\begin{sphinxuseclass}{cell}\begin{sphinxVerbatimInput}

\begin{sphinxuseclass}{cell_input}
\begin{sphinxVerbatim}[commandchars=\\\{\}]
\PYG{n}{mpak}\PYG{p}{[}\PYG{l+s+s1}{\PYGZsq{}}\PYG{l+s+s1}{PAKNYGDPMKTPKN PAKNECONPRVTKN}\PYG{l+s+s1}{\PYGZsq{}}\PYG{p}{]}\PYG{o}{.}\PYG{n}{pct}\PYG{o}{.}\PYG{n}{mul100}\PYG{o}{.}\PYG{n}{plot}\PYG{p}{(}\PYG{p}{)} 
\end{sphinxVerbatim}

\end{sphinxuseclass}\end{sphinxVerbatimInput}
\begin{sphinxVerbatimOutput}

\begin{sphinxuseclass}{cell_output}
\noindent\sphinxincludegraphics{{c607053d8a5ad717aec35cfeb41a850f1b55560db799786d7ba7225f24c6dd21}.png}

\end{sphinxuseclass}\end{sphinxVerbatimOutput}

\end{sphinxuseclass}

\section{.plot chart the selected and transformed variables}
\label{\detokenize{content/notebooks/modelflow_features:plot-chart-the-selected-and-transformed-variables}}
\sphinxAtStartPar
After the varaibles has been selected and transformed, they can  be plotted. The .plot() method plots the selected variables separately

\begin{sphinxuseclass}{cell}\begin{sphinxVerbatimInput}

\begin{sphinxuseclass}{cell_input}
\begin{sphinxVerbatim}[commandchars=\\\{\}]
\PYG{n}{mpak}\PYG{o}{.}\PYG{n}{smpl}\PYG{p}{(}\PYG{l+m+mi}{2020}\PYG{p}{,}\PYG{l+m+mi}{2100}\PYG{p}{)}\PYG{p}{;}

\PYG{n}{mpak}\PYG{p}{[}\PYG{l+s+s1}{\PYGZsq{}}\PYG{l+s+s1}{PAKNYGDP??????}\PYG{l+s+s1}{\PYGZsq{}}\PYG{p}{]}\PYG{o}{.}\PYG{n}{rename}\PYG{p}{(}\PYG{p}{)}\PYG{o}{.}\PYG{n}{plot}\PYG{p}{(}\PYG{p}{)}\PYG{p}{;}
\end{sphinxVerbatim}

\end{sphinxuseclass}\end{sphinxVerbatimInput}
\begin{sphinxVerbatimOutput}

\begin{sphinxuseclass}{cell_output}
\begin{sphinxVerbatim}[commandchars=\\\{\}]
C:\PYGZbs{}Users\PYGZbs{}ibhan\PYGZbs{}miniconda3\PYGZbs{}envs\PYGZbs{}mfbooknew\PYGZbs{}lib\PYGZbs{}site\PYGZhy{}packages\PYGZbs{}pandas\PYGZbs{}plotting\PYGZbs{}\PYGZus{}matplotlib\PYGZbs{}tools.py:227: RuntimeWarning: More than 20 figures have been opened. Figures created through the pyplot interface (`matplotlib.pyplot.figure`) are retained until explicitly closed and may consume too much memory. (To control this warning, see the rcParam `figure.max\PYGZus{}open\PYGZus{}warning`). Consider using `matplotlib.pyplot.close()`.
  fig = plt.figure(**fig\PYGZus{}kw)
\end{sphinxVerbatim}

\end{sphinxuseclass}\end{sphinxVerbatimOutput}

\end{sphinxuseclass}

\subsection{Options to plot()}
\label{\detokenize{content/notebooks/modelflow_features:options-to-plot}}
\sphinxAtStartPar
Common:
\begin{itemize}
\item {} 
\sphinxAtStartPar
title (optional): title. Defaults to ‘’.

\item {} 
\sphinxAtStartPar
colrow (TYPE, optional): Columns per row . Defaults to 2.

\item {} 
\sphinxAtStartPar
sharey (TYPE, optional): Share y axis between plots. Defaults to False.

\item {} 
\sphinxAtStartPar
top (TYPE, optional): Relative position of the title. Defaults to 0.90.

\end{itemize}

\sphinxAtStartPar
More excotic:
\begin{itemize}
\item {} 
\sphinxAtStartPar
splitchar (TYPE, optional): If the name should be split . Defaults to ‘\_\_’.

\item {} 
\sphinxAtStartPar
savefig (TYPE, optional): Save figure. Defaults to ‘’.

\item {} 
\sphinxAtStartPar
xsize  (TYPE, optional): x size default to 10

\item {} 
\sphinxAtStartPar
ysize  (TYPE, optional): y size per row, defaults to 2

\item {} 
\sphinxAtStartPar
ppos (optional): \# of position to use if split. Defaults to \sphinxhyphen{}1.

\item {} 
\sphinxAtStartPar
kind (TYPE, optional): Matplotlib kind . Defaults to ‘line’.

\end{itemize}

\begin{sphinxuseclass}{cell}\begin{sphinxVerbatimInput}

\begin{sphinxuseclass}{cell_input}
\begin{sphinxVerbatim}[commandchars=\\\{\}]
\PYG{n}{mpak}\PYG{p}{[}\PYG{l+s+s1}{\PYGZsq{}}\PYG{l+s+s1}{PAKNYGDP??????}\PYG{l+s+s1}{\PYGZsq{}}\PYG{p}{]}\PYG{o}{.}\PYG{n}{difpct}\PYG{o}{.}\PYG{n}{mul100}\PYG{o}{.}\PYG{n}{rename}\PYG{p}{(}\PYG{p}{)}\PYG{o}{.}\PYG{n}{plot}\PYG{p}{(}\PYG{n}{title}\PYG{o}{=}\PYG{l+s+s1}{\PYGZsq{}}\PYG{l+s+s1}{GDP growth }\PYG{l+s+s1}{\PYGZsq{}}\PYG{p}{,}\PYG{n}{top} \PYG{o}{=} \PYG{l+m+mf}{0.92}\PYG{p}{)}\PYG{p}{;}
\end{sphinxVerbatim}

\end{sphinxuseclass}\end{sphinxVerbatimInput}

\end{sphinxuseclass}

\section{Plotting inspiration}
\label{\detokenize{content/notebooks/modelflow_features:plotting-inspiration}}
\sphinxAtStartPar
The following graph shows the components of GDP using the values of the baseline dataframe.

\begin{sphinxuseclass}{cell}\begin{sphinxVerbatimInput}

\begin{sphinxuseclass}{cell_input}
\begin{sphinxVerbatim}[commandchars=\\\{\}]
\PYG{n}{mpak}\PYG{p}{[}\PYG{l+s+s1}{\PYGZsq{}}\PYG{l+s+s1}{PAKNYGDPMKTPKN PAKNECONPRVTKN PAKNEGDIFTOTKN}\PYG{l+s+s1}{\PYGZsq{}}\PYG{p}{]}\PYG{o}{.}\PYGZbs{}
\PYG{n}{difpctlevel}\PYG{o}{.}\PYG{n}{mul100}\PYG{o}{.}\PYG{n}{rename}\PYG{p}{(}\PYG{p}{)}\PYG{o}{.}\PYGZbs{}
\PYG{n}{plot}\PYG{p}{(}\PYG{n}{title}\PYG{o}{=}\PYG{l+s+s1}{\PYGZsq{}}\PYG{l+s+s1}{Components of GDP in pct of baseline}\PYG{l+s+s1}{\PYGZsq{}}\PYG{p}{,}\PYG{n}{colrow}\PYG{o}{=}\PYG{l+m+mi}{1}\PYG{p}{,}\PYG{n}{top}\PYG{o}{=}\PYG{l+m+mf}{0.90}\PYG{p}{,}\PYG{n}{kind}\PYG{o}{=}\PYG{l+s+s1}{\PYGZsq{}}\PYG{l+s+s1}{bar}\PYG{l+s+s1}{\PYGZsq{}}\PYG{p}{)} \PYG{p}{;}
\end{sphinxVerbatim}

\end{sphinxuseclass}\end{sphinxVerbatimInput}

\end{sphinxuseclass}

\subsection{Heatmaps}
\label{\detokenize{content/notebooks/modelflow_features:heatmaps}}
\sphinxAtStartPar
For some model types heatmaps can be helpful, and they come out of the box. This feature was developed for use by bank stress test models.

\begin{sphinxuseclass}{cell}\begin{sphinxVerbatimInput}

\begin{sphinxuseclass}{cell_input}
\begin{sphinxVerbatim}[commandchars=\\\{\}]
\PYG{k}{with} \PYG{n}{mpak}\PYG{o}{.}\PYG{n}{set\PYGZus{}smpl}\PYG{p}{(}\PYG{l+m+mi}{2020}\PYG{p}{,}\PYG{l+m+mi}{2030}\PYG{p}{)}\PYG{p}{:}
    \PYG{n}{heatmap} \PYG{o}{=} \PYG{n}{mpak}\PYG{p}{[}\PYG{l+s+s1}{\PYGZsq{}}\PYG{l+s+s1}{PAKNYGDPMKTPKN PAKNECONPRVTKN}\PYG{l+s+s1}{\PYGZsq{}}\PYG{p}{]}\PYG{o}{.}\PYG{n}{pct}\PYG{o}{.}\PYG{n}{rename}\PYG{p}{(}\PYG{p}{)}\PYG{o}{.}\PYG{n}{mul100}\PYG{o}{.}\PYG{n}{heat}\PYG{p}{(}\PYG{n}{title}\PYG{o}{=}\PYG{l+s+s1}{\PYGZsq{}}\PYG{l+s+s1}{Growth rates}\PYG{l+s+s1}{\PYGZsq{}}\PYG{p}{,}\PYG{n}{annot}\PYG{o}{=}\PYG{k+kc}{True}\PYG{p}{,}\PYG{n}{dec}\PYG{o}{=}\PYG{l+m+mi}{1}\PYG{p}{,}\PYG{n}{size}\PYG{o}{=}\PYG{p}{(}\PYG{l+m+mi}{10}\PYG{p}{,}\PYG{l+m+mi}{3}\PYG{p}{)}\PYG{p}{)}  
\end{sphinxVerbatim}

\end{sphinxuseclass}\end{sphinxVerbatimInput}

\end{sphinxuseclass}
\sphinxAtStartPar



\subsection{Violin and boxplots,}
\label{\detokenize{content/notebooks/modelflow_features:violin-and-boxplots}}
\sphinxAtStartPar
Not obvious for macro models, but useful for stress test  models with many banks.

\begin{sphinxuseclass}{cell}\begin{sphinxVerbatimInput}

\begin{sphinxuseclass}{cell_input}
\begin{sphinxVerbatim}[commandchars=\\\{\}]
\PYG{k}{with} \PYG{n}{mpak}\PYG{o}{.}\PYG{n}{set\PYGZus{}smpl}\PYG{p}{(}\PYG{l+m+mi}{2020}\PYG{p}{,}\PYG{l+m+mi}{2030}\PYG{p}{)}\PYG{p}{:} 
    \PYG{n}{mpak}\PYG{p}{[}\PYG{l+s+s1}{\PYGZsq{}}\PYG{l+s+s1}{PAKNYGDPMKTPKN PAKNECONPRVTKN}\PYG{l+s+s1}{\PYGZsq{}}\PYG{p}{]}\PYG{o}{.}\PYG{n}{difpct}\PYG{o}{.}\PYG{n}{box}\PYG{p}{(}\PYG{p}{)}  
    \PYG{n}{mpak}\PYG{p}{[}\PYG{l+s+s1}{\PYGZsq{}}\PYG{l+s+s1}{PAKNYGDPMKTPKN PAKNECONPRVTKN}\PYG{l+s+s1}{\PYGZsq{}}\PYG{p}{]}\PYG{o}{.}\PYG{n}{difpct}\PYG{o}{.}\PYG{n}{violin}\PYG{p}{(}\PYG{p}{)}  
\end{sphinxVerbatim}

\end{sphinxuseclass}\end{sphinxVerbatimInput}

\end{sphinxuseclass}

\subsection{Plot baseline vs alternative}
\label{\detokenize{content/notebooks/modelflow_features:plot-baseline-vs-alternative}}
\sphinxAtStartPar
A raw routine, only showing levels.
To make it really useful it should be expanded.

\begin{sphinxuseclass}{cell}\begin{sphinxVerbatimInput}

\begin{sphinxuseclass}{cell_input}
\begin{sphinxVerbatim}[commandchars=\\\{\}]
\PYG{n}{mpak}\PYG{p}{[}\PYG{l+s+s1}{\PYGZsq{}}\PYG{l+s+s1}{PAKNYGDPMKTPKN PAKNECONPRVTKN}\PYG{l+s+s1}{\PYGZsq{}}\PYG{p}{]}\PYG{o}{.}\PYG{n}{plot\PYGZus{}alt}\PYG{p}{(}\PYG{p}{)} \PYG{p}{;}
\end{sphinxVerbatim}

\end{sphinxuseclass}\end{sphinxVerbatimInput}

\end{sphinxuseclass}

\section{.draw() Graphical presentation of relationships between variables}
\label{\detokenize{content/notebooks/modelflow_features:draw-graphical-presentation-of-relationships-between-variables}}
\sphinxAtStartPar
.draw() helps you understand the relationship between variables in your model better.

\sphinxAtStartPar
The thickness the arrow reflect the attribution of the the upstream variable to the impact on the downstream variable.


\subsection{.draw(up = level, down = level)}
\label{\detokenize{content/notebooks/modelflow_features:draw-up-level-down-level}}
\sphinxAtStartPar
You can specify how many levels up and down you want in your graphical presentation (Needs more explanation).

\sphinxAtStartPar
In this example all variables that depend directly upon GDP and consumption as well as those that are determined by them, are displayed. This means one step upstream in the model logic and one step downstream.

\sphinxAtStartPar
More on the how to visualize the logic structure \DUrole{xref,myst}{here}

\begin{sphinxuseclass}{cell}\begin{sphinxVerbatimInput}

\begin{sphinxuseclass}{cell_input}
\begin{sphinxVerbatim}[commandchars=\\\{\}]
\PYG{n}{mpak}\PYG{p}{[}\PYG{l+s+s1}{\PYGZsq{}}\PYG{l+s+s1}{PAKNYGDPMKTPKN PAKNECONPRVTKN}\PYG{l+s+s1}{\PYGZsq{}}\PYG{p}{]}\PYG{o}{.}\PYG{n}{draw}\PYG{p}{(}\PYG{n}{up}\PYG{o}{=}\PYG{l+m+mi}{1}\PYG{p}{,}\PYG{n}{down}\PYG{o}{=}\PYG{l+m+mi}{1}\PYG{p}{)}  \PYG{c+c1}{\PYGZsh{} diagram of all direct dependencies }
\end{sphinxVerbatim}

\end{sphinxuseclass}\end{sphinxVerbatimInput}
\begin{sphinxVerbatimOutput}

\begin{sphinxuseclass}{cell_output}
\begin{sphinxVerbatim}[commandchars=\\\{\}]
\PYGZlt{}IPython.core.display.SVG object\PYGZgt{}
\end{sphinxVerbatim}

\begin{sphinxVerbatim}[commandchars=\\\{\}]
\PYGZlt{}IPython.core.display.SVG object\PYGZgt{}
\end{sphinxVerbatim}

\end{sphinxuseclass}\end{sphinxVerbatimOutput}

\end{sphinxuseclass}

\subsection{.draw(filter =<minimal impact>)}
\label{\detokenize{content/notebooks/modelflow_features:draw-filter-minimal-impact}}
\sphinxAtStartPar
By specifying filter=  only links where the minimal impact is more than <minimal impact> are show. In this case 20\%

\begin{sphinxuseclass}{cell}\begin{sphinxVerbatimInput}

\begin{sphinxuseclass}{cell_input}
\begin{sphinxVerbatim}[commandchars=\\\{\}]
\PYG{n}{mpak}\PYG{p}{[}\PYG{l+s+s1}{\PYGZsq{}}\PYG{l+s+s1}{PAKNECONPRVTKN}\PYG{l+s+s1}{\PYGZsq{}}\PYG{p}{]}\PYG{o}{.}\PYG{n}{draw}\PYG{p}{(}\PYG{n}{up}\PYG{o}{=}\PYG{l+m+mi}{3}\PYG{p}{,}\PYG{n}{down}\PYG{o}{=}\PYG{l+m+mi}{1}\PYG{p}{,}\PYG{n+nb}{filter}\PYG{o}{=}\PYG{l+m+mi}{20}\PYG{p}{)}  
\end{sphinxVerbatim}

\end{sphinxuseclass}\end{sphinxVerbatimInput}
\begin{sphinxVerbatimOutput}

\begin{sphinxuseclass}{cell_output}
\begin{sphinxVerbatim}[commandchars=\\\{\}]
\PYGZlt{}IPython.core.display.SVG object\PYGZgt{}
\end{sphinxVerbatim}

\end{sphinxuseclass}\end{sphinxVerbatimOutput}

\end{sphinxuseclass}

\section{dekomp() Attrribution of right hand side variables to change in result.}
\label{\detokenize{content/notebooks/modelflow_features:dekomp-attrribution-of-right-hand-side-variables-to-change-in-result}}
\sphinxAtStartPar
For more information on attribution look \DUrole{xref,myst}{here}

\sphinxAtStartPar
The dekomp command decomposes the contributions of the right hand side variables to the observed change in the left hand side variables.

\begin{sphinxuseclass}{cell}\begin{sphinxVerbatimInput}

\begin{sphinxuseclass}{cell_input}
\begin{sphinxVerbatim}[commandchars=\\\{\}]
\PYG{k}{with} \PYG{n}{mpak}\PYG{o}{.}\PYG{n}{set\PYGZus{}smpl}\PYG{p}{(}\PYG{l+m+mi}{2021}\PYG{p}{,}\PYG{l+m+mi}{2025}\PYG{p}{)}\PYG{p}{:}
    \PYG{n}{mpak}\PYG{p}{[}\PYG{l+s+s1}{\PYGZsq{}}\PYG{l+s+s1}{PAKNYGDPMKTPKN PAKNECONPRVTKN}\PYG{l+s+s1}{\PYGZsq{}}\PYG{p}{]}\PYG{o}{.}\PYG{n}{dekomp}\PYG{p}{(}\PYG{p}{)}  \PYG{c+c1}{\PYGZsh{} frml attribution }
\end{sphinxVerbatim}

\end{sphinxuseclass}\end{sphinxVerbatimInput}
\begin{sphinxVerbatimOutput}

\begin{sphinxuseclass}{cell_output}
\begin{sphinxVerbatim}[commandchars=\\\{\}]
Formula        : FRML  \PYGZlt{}\PYGZgt{} PAKNYGDPMKTPKN = PAKNECONPRVTKN+PAKNECONGOVTKN+PAKNEGDIFTOTKN+PAKNEGDISTKBKN+PAKNEEXPGNFSKN\PYGZhy{}PAKNEIMPGNFSKN+PAKNYGDPDISCKN+PAKADAP*PAKDISPREPKN \PYGZdl{} 

                       2021        2022        2023        2024        2025
Variable    lag                                                            
Base        0   26511370.41 26685141.87 26963077.57 27393200.36 27963231.53
Alternative 0   26764926.87 26889649.52 27089036.50 27454422.35 27979057.19
Difference  0     253556.46   204507.65   125958.93    61221.99    15825.66
Percent     0          0.96        0.77        0.47        0.22        0.06

 Contributions to differende for  PAKNYGDPMKTPKN
                         2021       2022       2023       2024       2025
Variable       lag                                                       
PAKNECONPRVTKN 0   \PYGZhy{}312052.97 \PYGZhy{}394466.14 \PYGZhy{}474558.93 \PYGZhy{}531755.17 \PYGZhy{}563616.01
PAKNECONGOVTKN 0    303335.99  268694.45  232506.87  209988.19  197439.80
PAKNEGDIFTOTKN 0    188565.48  188222.74  177226.36  163571.78  148739.93
PAKNEGDISTKBKN 0        \PYGZhy{}0.02      \PYGZhy{}0.01      \PYGZhy{}0.02      \PYGZhy{}0.02      \PYGZhy{}0.05
PAKNEEXPGNFSKN 0     \PYGZhy{}2911.23   \PYGZhy{}5414.50   \PYGZhy{}7960.34  \PYGZhy{}10272.64  \PYGZhy{}12204.84
PAKNEIMPGNFSKN 0     76619.12  147471.06  198744.89  229689.74  245466.56
PAKNYGDPDISCKN 0        \PYGZhy{}0.02      \PYGZhy{}0.01      \PYGZhy{}0.02      \PYGZhy{}0.02      \PYGZhy{}0.05
PAKADAP        0        \PYGZhy{}0.02      \PYGZhy{}0.01      \PYGZhy{}0.02      \PYGZhy{}0.02      \PYGZhy{}0.05
PAKDISPREPKN   0        \PYGZhy{}0.02      \PYGZhy{}0.01      \PYGZhy{}0.02      \PYGZhy{}0.02      \PYGZhy{}0.05

 Share of contributions to differende for  PAKNYGDPMKTPKN
                          2021        2022        2023        2024        2025
Variable       lag                                                            
PAKNEIMPGNFSKN 0           30\PYGZpc{}         72\PYGZpc{}        158\PYGZpc{}        375\PYGZpc{}       1551\PYGZpc{}
PAKNECONGOVTKN 0          120\PYGZpc{}        131\PYGZpc{}        185\PYGZpc{}        343\PYGZpc{}       1248\PYGZpc{}
PAKNEGDIFTOTKN 0           74\PYGZpc{}         92\PYGZpc{}        141\PYGZpc{}        267\PYGZpc{}        940\PYGZpc{}
PAKNEGDISTKBKN 0           \PYGZhy{}0\PYGZpc{}         \PYGZhy{}0\PYGZpc{}         \PYGZhy{}0\PYGZpc{}         \PYGZhy{}0\PYGZpc{}         \PYGZhy{}0\PYGZpc{}
PAKNYGDPDISCKN 0           \PYGZhy{}0\PYGZpc{}         \PYGZhy{}0\PYGZpc{}         \PYGZhy{}0\PYGZpc{}         \PYGZhy{}0\PYGZpc{}         \PYGZhy{}0\PYGZpc{}
PAKADAP        0           \PYGZhy{}0\PYGZpc{}         \PYGZhy{}0\PYGZpc{}         \PYGZhy{}0\PYGZpc{}         \PYGZhy{}0\PYGZpc{}         \PYGZhy{}0\PYGZpc{}
PAKDISPREPKN   0           \PYGZhy{}0\PYGZpc{}         \PYGZhy{}0\PYGZpc{}         \PYGZhy{}0\PYGZpc{}         \PYGZhy{}0\PYGZpc{}         \PYGZhy{}0\PYGZpc{}
PAKNEEXPGNFSKN 0           \PYGZhy{}1\PYGZpc{}         \PYGZhy{}3\PYGZpc{}         \PYGZhy{}6\PYGZpc{}        \PYGZhy{}17\PYGZpc{}        \PYGZhy{}77\PYGZpc{}
PAKNECONPRVTKN 0         \PYGZhy{}123\PYGZpc{}       \PYGZhy{}193\PYGZpc{}       \PYGZhy{}377\PYGZpc{}       \PYGZhy{}869\PYGZpc{}      \PYGZhy{}3561\PYGZpc{}
Total          0          100\PYGZpc{}        100\PYGZpc{}        100\PYGZpc{}        100\PYGZpc{}        100\PYGZpc{}
Residual       0           \PYGZhy{}0\PYGZpc{}         \PYGZhy{}0\PYGZpc{}         \PYGZhy{}0\PYGZpc{}         \PYGZhy{}0\PYGZpc{}         \PYGZhy{}0\PYGZpc{}

 Contribution to growth rate PAKNYGDPMKTPKN
                          2021        2022        2023        2024        2025
Variable       lag                                                            
PAKNECONPRVTKN 0         \PYGZhy{}0.0\PYGZpc{}       \PYGZhy{}0.0\PYGZpc{}       \PYGZhy{}0.0\PYGZpc{}       \PYGZhy{}0.0\PYGZpc{}       \PYGZhy{}0.0\PYGZpc{}
PAKNECONGOVTKN 0          0.0\PYGZpc{}        0.0\PYGZpc{}        0.0\PYGZpc{}        0.0\PYGZpc{}        0.0\PYGZpc{}
PAKNEGDIFTOTKN 0          0.0\PYGZpc{}        0.0\PYGZpc{}        0.0\PYGZpc{}        0.0\PYGZpc{}        0.0\PYGZpc{}
PAKNEGDISTKBKN 0         \PYGZhy{}0.0\PYGZpc{}       \PYGZhy{}0.0\PYGZpc{}       \PYGZhy{}0.0\PYGZpc{}       \PYGZhy{}0.0\PYGZpc{}       \PYGZhy{}0.0\PYGZpc{}
PAKNEEXPGNFSKN 0         \PYGZhy{}0.0\PYGZpc{}       \PYGZhy{}0.0\PYGZpc{}       \PYGZhy{}0.0\PYGZpc{}       \PYGZhy{}0.0\PYGZpc{}       \PYGZhy{}0.0\PYGZpc{}
PAKNEIMPGNFSKN 0          0.0\PYGZpc{}        0.0\PYGZpc{}        0.0\PYGZpc{}        0.0\PYGZpc{}        0.0\PYGZpc{}
PAKNYGDPDISCKN 0         \PYGZhy{}0.0\PYGZpc{}       \PYGZhy{}0.0\PYGZpc{}       \PYGZhy{}0.0\PYGZpc{}       \PYGZhy{}0.0\PYGZpc{}       \PYGZhy{}0.0\PYGZpc{}
PAKADAP        0         \PYGZhy{}0.0\PYGZpc{}       \PYGZhy{}0.0\PYGZpc{}       \PYGZhy{}0.0\PYGZpc{}       \PYGZhy{}0.0\PYGZpc{}       \PYGZhy{}0.0\PYGZpc{}
PAKDISPREPKN   0         \PYGZhy{}0.0\PYGZpc{}       \PYGZhy{}0.0\PYGZpc{}       \PYGZhy{}0.0\PYGZpc{}       \PYGZhy{}0.0\PYGZpc{}       \PYGZhy{}0.0\PYGZpc{}

Formula        : FRML \PYGZlt{}Z,EXO\PYGZgt{} PAKNECONPRVTKN = (PAKNECONPRVTKN(\PYGZhy{}1)*EXP(PAKNECONPRVTKN\PYGZus{}A+ (\PYGZhy{}0.2*(LOG(PAKNECONPRVTKN(\PYGZhy{}1))\PYGZhy{}LOG(1.21203101101442)\PYGZhy{}LOG((((PAKBXFSTREMTCD(\PYGZhy{}1)\PYGZhy{}PAKBMFSTREMTCD(\PYGZhy{}1))*PAKPANUSATLS(\PYGZhy{}1))+PAKGGEXPTRNSCN(\PYGZhy{}1)+PAKNYYWBTOTLCN(\PYGZhy{}1)*(1\PYGZhy{}PAKGGREVDRCTXN(\PYGZhy{}1)/100))/PAKNECONPRVTXN(\PYGZhy{}1)))+0.763938860758873*((LOG((((PAKBXFSTREMTCD\PYGZhy{}PAKBMFSTREMTCD)*PAKPANUSATLS)+PAKGGEXPTRNSCN+PAKNYYWBTOTLCN*(1\PYGZhy{}PAKGGREVDRCTXN/100))/PAKNECONPRVTXN))\PYGZhy{}(LOG((((PAKBXFSTREMTCD(\PYGZhy{}1)\PYGZhy{}PAKBMFSTREMTCD(\PYGZhy{}1))*PAKPANUSATLS(\PYGZhy{}1))+PAKGGEXPTRNSCN(\PYGZhy{}1)+PAKNYYWBTOTLCN(\PYGZhy{}1)*(1\PYGZhy{}PAKGGREVDRCTXN(\PYGZhy{}1)/100))/PAKNECONPRVTXN(\PYGZhy{}1))))\PYGZhy{}0.0634474791568939*DURING\PYGZus{}2009\PYGZhy{}0.3*(PAKFMLBLPOLYXN/100\PYGZhy{}((LOG(PAKNECONPRVTXN))\PYGZhy{}(LOG(PAKNECONPRVTXN(\PYGZhy{}1)))))) )) * (1\PYGZhy{}PAKNECONPRVTKN\PYGZus{}D)+ PAKNECONPRVTKN\PYGZus{}X*PAKNECONPRVTKN\PYGZus{}D  \PYGZdl{} 

                       2021        2022        2023        2024        2025
Variable    lag                                                            
Base        0   23972815.36 24164128.02 24427863.05 24818524.47 25323255.17
Alternative 0   23660762.40 23769661.89 23953304.14 24286769.32 24759639.22
Difference  0    \PYGZhy{}312052.95  \PYGZhy{}394466.13  \PYGZhy{}474558.91  \PYGZhy{}531755.15  \PYGZhy{}563615.95
Percent     0         \PYGZhy{}1.30       \PYGZhy{}1.63       \PYGZhy{}1.94       \PYGZhy{}2.14       \PYGZhy{}2.23

 Contributions to differende for  PAKNECONPRVTKN
                           2021       2022       2023       2024       2025
Variable         lag                                                       
PAKNECONPRVTKN   \PYGZhy{}1  \PYGZhy{}187434.07 \PYGZhy{}250462.33 \PYGZhy{}317486.72 \PYGZhy{}384175.74 \PYGZhy{}432745.55
PAKNECONPRVTKN\PYGZus{}A  0       \PYGZhy{}0.01      \PYGZhy{}0.01      \PYGZhy{}0.01      \PYGZhy{}0.01      \PYGZhy{}0.04
PAKBXFSTREMTCD   \PYGZhy{}1   \PYGZhy{}38694.42  \PYGZhy{}49412.27  \PYGZhy{}52084.76  \PYGZhy{}50817.15  \PYGZhy{}48170.52
PAKBMFSTREMTCD   \PYGZhy{}1      120.58     140.33     135.37     121.42     106.31
PAKPANUSATLS     \PYGZhy{}1     3137.57    3566.27    3817.26    3916.63    3901.04
PAKGGEXPTRNSCN   \PYGZhy{}1    \PYGZhy{}2382.89   \PYGZhy{}4372.94   \PYGZhy{}5966.91   \PYGZhy{}7223.04   \PYGZhy{}8206.86
PAKNYYWBTOTLCN   \PYGZhy{}1   \PYGZhy{}78794.18 \PYGZhy{}120093.04 \PYGZhy{}145773.43 \PYGZhy{}156461.75 \PYGZhy{}167189.37
PAKGGREVDRCTXN   \PYGZhy{}1       \PYGZhy{}0.01      \PYGZhy{}0.01      \PYGZhy{}0.01      \PYGZhy{}0.01      \PYGZhy{}0.04
PAKNECONPRVTXN   \PYGZhy{}1   204247.65  231199.67  249025.22  258789.48  262005.59
PAKBXFSTREMTCD    0    66466.87   69836.78   67727.29   63861.51   59511.44
PAKBMFSTREMTCD    0     \PYGZhy{}189.25    \PYGZhy{}182.00    \PYGZhy{}162.26    \PYGZhy{}141.32    \PYGZhy{}122.29
PAKPANUSATLS      0    \PYGZhy{}4809.78   \PYGZhy{}5132.40   \PYGZhy{}5233.84   \PYGZhy{}5184.63   \PYGZhy{}5043.02
PAKGGEXPTRNSCN    0     5895.35    8018.71    9646.92   10900.77   11850.22
PAKNYYWBTOTLCN    0   160980.65  194563.25  207466.32  220404.39  237003.52
PAKGGREVDRCTXN    0       \PYGZhy{}0.01      \PYGZhy{}0.01      \PYGZhy{}0.01      \PYGZhy{}0.01      \PYGZhy{}0.04
PAKNECONPRVTXN    0  \PYGZhy{}410022.04 \PYGZhy{}440677.37 \PYGZhy{}455322.70 \PYGZhy{}458425.12 \PYGZhy{}453478.88
DURING\PYGZus{}2009       0       \PYGZhy{}0.01      \PYGZhy{}0.01      \PYGZhy{}0.01      \PYGZhy{}0.01      \PYGZhy{}0.04
PAKFMLBLPOLYXN    0   \PYGZhy{}34994.35  \PYGZhy{}36409.85  \PYGZhy{}35203.57  \PYGZhy{}32189.52  \PYGZhy{}28049.75
PAKNECONPRVTKN\PYGZus{}D  0       \PYGZhy{}0.01      \PYGZhy{}0.01      \PYGZhy{}0.01      \PYGZhy{}0.01      \PYGZhy{}0.04
PAKNECONPRVTKN\PYGZus{}X  0       \PYGZhy{}0.01      \PYGZhy{}0.01      \PYGZhy{}0.01      \PYGZhy{}0.01      \PYGZhy{}0.04

 Share of contributions to differende for  PAKNECONPRVTKN
                            2021        2022        2023        2024        2025
Variable         lag                                                            
PAKNECONPRVTXN    0         131\PYGZpc{}        112\PYGZpc{}         96\PYGZpc{}         86\PYGZpc{}         80\PYGZpc{}
PAKNECONPRVTKN   \PYGZhy{}1          60\PYGZpc{}         63\PYGZpc{}         67\PYGZpc{}         72\PYGZpc{}         77\PYGZpc{}
PAKNYYWBTOTLCN   \PYGZhy{}1          25\PYGZpc{}         30\PYGZpc{}         31\PYGZpc{}         29\PYGZpc{}         30\PYGZpc{}
PAKBXFSTREMTCD   \PYGZhy{}1          12\PYGZpc{}         13\PYGZpc{}         11\PYGZpc{}         10\PYGZpc{}          9\PYGZpc{}
PAKFMLBLPOLYXN    0          11\PYGZpc{}          9\PYGZpc{}          7\PYGZpc{}          6\PYGZpc{}          5\PYGZpc{}
PAKGGEXPTRNSCN   \PYGZhy{}1           1\PYGZpc{}          1\PYGZpc{}          1\PYGZpc{}          1\PYGZpc{}          1\PYGZpc{}
PAKPANUSATLS      0           2\PYGZpc{}          1\PYGZpc{}          1\PYGZpc{}          1\PYGZpc{}          1\PYGZpc{}
PAKBMFSTREMTCD    0           0\PYGZpc{}          0\PYGZpc{}          0\PYGZpc{}          0\PYGZpc{}          0\PYGZpc{}
PAKNECONPRVTKN\PYGZus{}A  0           0\PYGZpc{}          0\PYGZpc{}          0\PYGZpc{}          0\PYGZpc{}          0\PYGZpc{}
PAKGGREVDRCTXN   \PYGZhy{}1           0\PYGZpc{}          0\PYGZpc{}          0\PYGZpc{}          0\PYGZpc{}          0\PYGZpc{}
                  0           0\PYGZpc{}          0\PYGZpc{}          0\PYGZpc{}          0\PYGZpc{}          0\PYGZpc{}
DURING\PYGZus{}2009       0           0\PYGZpc{}          0\PYGZpc{}          0\PYGZpc{}          0\PYGZpc{}          0\PYGZpc{}
PAKNECONPRVTKN\PYGZus{}D  0           0\PYGZpc{}          0\PYGZpc{}          0\PYGZpc{}          0\PYGZpc{}          0\PYGZpc{}
PAKNECONPRVTKN\PYGZus{}X  0           0\PYGZpc{}          0\PYGZpc{}          0\PYGZpc{}          0\PYGZpc{}          0\PYGZpc{}
PAKBMFSTREMTCD   \PYGZhy{}1          \PYGZhy{}0\PYGZpc{}         \PYGZhy{}0\PYGZpc{}         \PYGZhy{}0\PYGZpc{}         \PYGZhy{}0\PYGZpc{}         \PYGZhy{}0\PYGZpc{}
PAKPANUSATLS     \PYGZhy{}1          \PYGZhy{}1\PYGZpc{}         \PYGZhy{}1\PYGZpc{}         \PYGZhy{}1\PYGZpc{}         \PYGZhy{}1\PYGZpc{}         \PYGZhy{}1\PYGZpc{}
PAKGGEXPTRNSCN    0          \PYGZhy{}2\PYGZpc{}         \PYGZhy{}2\PYGZpc{}         \PYGZhy{}2\PYGZpc{}         \PYGZhy{}2\PYGZpc{}         \PYGZhy{}2\PYGZpc{}
PAKBXFSTREMTCD    0         \PYGZhy{}21\PYGZpc{}        \PYGZhy{}18\PYGZpc{}        \PYGZhy{}14\PYGZpc{}        \PYGZhy{}12\PYGZpc{}        \PYGZhy{}11\PYGZpc{}
PAKNYYWBTOTLCN    0         \PYGZhy{}52\PYGZpc{}        \PYGZhy{}49\PYGZpc{}        \PYGZhy{}44\PYGZpc{}        \PYGZhy{}41\PYGZpc{}        \PYGZhy{}42\PYGZpc{}
PAKNECONPRVTXN   \PYGZhy{}1         \PYGZhy{}65\PYGZpc{}        \PYGZhy{}59\PYGZpc{}        \PYGZhy{}52\PYGZpc{}        \PYGZhy{}49\PYGZpc{}        \PYGZhy{}46\PYGZpc{}
Total             0         101\PYGZpc{}        101\PYGZpc{}        101\PYGZpc{}        101\PYGZpc{}        101\PYGZpc{}
Residual          0           1\PYGZpc{}          1\PYGZpc{}          1\PYGZpc{}          1\PYGZpc{}          1\PYGZpc{}

 Contribution to growth rate PAKNECONPRVTKN
                            2021        2022        2023        2024        2025
Variable         lag                                                            
PAKNECONPRVTKN   \PYGZhy{}1         0.0\PYGZpc{}        0.0\PYGZpc{}        0.0\PYGZpc{}        0.0\PYGZpc{}        0.0\PYGZpc{}
PAKNECONPRVTKN\PYGZus{}A  0        \PYGZhy{}0.0\PYGZpc{}       \PYGZhy{}0.0\PYGZpc{}       \PYGZhy{}0.0\PYGZpc{}       \PYGZhy{}0.0\PYGZpc{}       \PYGZhy{}0.0\PYGZpc{}
PAKBXFSTREMTCD   \PYGZhy{}1        \PYGZhy{}0.0\PYGZpc{}       \PYGZhy{}0.0\PYGZpc{}       \PYGZhy{}0.0\PYGZpc{}       \PYGZhy{}0.0\PYGZpc{}       \PYGZhy{}0.0\PYGZpc{}
PAKBMFSTREMTCD   \PYGZhy{}1         0.0\PYGZpc{}        0.0\PYGZpc{}        0.0\PYGZpc{}        0.0\PYGZpc{}        0.0\PYGZpc{}
PAKPANUSATLS     \PYGZhy{}1         0.0\PYGZpc{}        0.0\PYGZpc{}        0.0\PYGZpc{}        0.0\PYGZpc{}        0.0\PYGZpc{}
PAKGGEXPTRNSCN   \PYGZhy{}1        \PYGZhy{}0.0\PYGZpc{}       \PYGZhy{}0.0\PYGZpc{}       \PYGZhy{}0.0\PYGZpc{}       \PYGZhy{}0.0\PYGZpc{}       \PYGZhy{}0.0\PYGZpc{}
PAKNYYWBTOTLCN   \PYGZhy{}1        \PYGZhy{}0.0\PYGZpc{}       \PYGZhy{}0.0\PYGZpc{}       \PYGZhy{}0.0\PYGZpc{}       \PYGZhy{}0.0\PYGZpc{}       \PYGZhy{}0.0\PYGZpc{}
PAKGGREVDRCTXN   \PYGZhy{}1        \PYGZhy{}0.0\PYGZpc{}       \PYGZhy{}0.0\PYGZpc{}       \PYGZhy{}0.0\PYGZpc{}       \PYGZhy{}0.0\PYGZpc{}       \PYGZhy{}0.0\PYGZpc{}
PAKNECONPRVTXN   \PYGZhy{}1         0.0\PYGZpc{}        0.0\PYGZpc{}        0.0\PYGZpc{}        0.0\PYGZpc{}        0.0\PYGZpc{}
PAKBXFSTREMTCD    0         0.0\PYGZpc{}        0.0\PYGZpc{}        0.0\PYGZpc{}        0.0\PYGZpc{}        0.0\PYGZpc{}
PAKBMFSTREMTCD    0        \PYGZhy{}0.0\PYGZpc{}       \PYGZhy{}0.0\PYGZpc{}       \PYGZhy{}0.0\PYGZpc{}       \PYGZhy{}0.0\PYGZpc{}       \PYGZhy{}0.0\PYGZpc{}
PAKPANUSATLS      0        \PYGZhy{}0.0\PYGZpc{}       \PYGZhy{}0.0\PYGZpc{}       \PYGZhy{}0.0\PYGZpc{}       \PYGZhy{}0.0\PYGZpc{}       \PYGZhy{}0.0\PYGZpc{}
PAKGGEXPTRNSCN    0         0.0\PYGZpc{}        0.0\PYGZpc{}        0.0\PYGZpc{}        0.0\PYGZpc{}        0.0\PYGZpc{}
PAKNYYWBTOTLCN    0         0.0\PYGZpc{}        0.0\PYGZpc{}        0.0\PYGZpc{}        0.0\PYGZpc{}        0.0\PYGZpc{}
PAKGGREVDRCTXN    0        \PYGZhy{}0.0\PYGZpc{}       \PYGZhy{}0.0\PYGZpc{}       \PYGZhy{}0.0\PYGZpc{}       \PYGZhy{}0.0\PYGZpc{}       \PYGZhy{}0.0\PYGZpc{}
PAKNECONPRVTXN    0        \PYGZhy{}0.0\PYGZpc{}       \PYGZhy{}0.0\PYGZpc{}       \PYGZhy{}0.0\PYGZpc{}       \PYGZhy{}0.0\PYGZpc{}       \PYGZhy{}0.0\PYGZpc{}
DURING\PYGZus{}2009       0        \PYGZhy{}0.0\PYGZpc{}       \PYGZhy{}0.0\PYGZpc{}       \PYGZhy{}0.0\PYGZpc{}       \PYGZhy{}0.0\PYGZpc{}       \PYGZhy{}0.0\PYGZpc{}
PAKFMLBLPOLYXN    0        \PYGZhy{}0.0\PYGZpc{}       \PYGZhy{}0.0\PYGZpc{}       \PYGZhy{}0.0\PYGZpc{}       \PYGZhy{}0.0\PYGZpc{}       \PYGZhy{}0.0\PYGZpc{}
PAKNECONPRVTKN\PYGZus{}D  0        \PYGZhy{}0.0\PYGZpc{}       \PYGZhy{}0.0\PYGZpc{}       \PYGZhy{}0.0\PYGZpc{}       \PYGZhy{}0.0\PYGZpc{}       \PYGZhy{}0.0\PYGZpc{}
PAKNECONPRVTKN\PYGZus{}X  0        \PYGZhy{}0.0\PYGZpc{}       \PYGZhy{}0.0\PYGZpc{}       \PYGZhy{}0.0\PYGZpc{}       \PYGZhy{}0.0\PYGZpc{}       \PYGZhy{}0.0\PYGZpc{}
\end{sphinxVerbatim}

\end{sphinxuseclass}\end{sphinxVerbatimOutput}

\end{sphinxuseclass}

\section{Bespoken plots using matplotlib  (or plotly \sphinxhyphen{}later) (should go to a separate plot book}
\label{\detokenize{content/notebooks/modelflow_features:bespoken-plots-using-matplotlib-or-plotly-later-should-go-to-a-separate-plot-book}}
\sphinxAtStartPar
The predefined plots are not necessary created for presentation purpose. To create  bespoken plots the they can be
constructed directly in python scripts. The two main libraries are matplotlib, plotly but any ther python plotting library can be used. Here is an example using matplotlib.


\section{Plot four separate plots of multiple series in grid}
\label{\detokenize{content/notebooks/modelflow_features:plot-four-separate-plots-of-multiple-series-in-grid}}
\begin{sphinxuseclass}{cell}\begin{sphinxVerbatimInput}

\begin{sphinxuseclass}{cell_input}
\begin{sphinxVerbatim}[commandchars=\\\{\}]
\PYG{n}{figure}\PYG{p}{,}\PYG{n}{axs}\PYG{o}{=} \PYG{n}{plt}\PYG{o}{.}\PYG{n}{subplots}\PYG{p}{(}\PYG{l+m+mi}{2}\PYG{p}{,}\PYG{l+m+mi}{2}\PYG{p}{,}\PYG{n}{figsize}\PYG{o}{=}\PYG{p}{(}\PYG{l+m+mi}{11}\PYG{p}{,} \PYG{l+m+mi}{7}\PYG{p}{)}\PYG{p}{)}
\PYG{n}{axs}\PYG{p}{[}\PYG{l+m+mi}{0}\PYG{p}{,}\PYG{l+m+mi}{0}\PYG{p}{]}\PYG{o}{.}\PYG{n}{plot}\PYG{p}{(}\PYG{n}{mpak}\PYG{o}{.}\PYG{n}{basedf}\PYG{o}{.}\PYG{n}{loc}\PYG{p}{[}\PYG{l+m+mi}{2020}\PYG{p}{:}\PYG{l+m+mi}{2099}\PYG{p}{,}\PYG{l+s+s1}{\PYGZsq{}}\PYG{l+s+s1}{PAKGGBALOVRLCN\PYGZus{}}\PYG{l+s+s1}{\PYGZsq{}}\PYG{p}{]}\PYG{p}{,}\PYG{n}{label}\PYG{o}{=}\PYG{l+s+s1}{\PYGZsq{}}\PYG{l+s+s1}{Baseline}\PYG{l+s+s1}{\PYGZsq{}}\PYG{p}{)}
\PYG{n}{axs}\PYG{p}{[}\PYG{l+m+mi}{0}\PYG{p}{,}\PYG{l+m+mi}{0}\PYG{p}{]}\PYG{o}{.}\PYG{n}{plot}\PYG{p}{(}\PYG{n}{mpak}\PYG{o}{.}\PYG{n}{lastdf}\PYG{o}{.}\PYG{n}{loc}\PYG{p}{[}\PYG{l+m+mi}{2020}\PYG{p}{:}\PYG{l+m+mi}{2099}\PYG{p}{,}\PYG{l+s+s1}{\PYGZsq{}}\PYG{l+s+s1}{PAKGGBALOVRLCN\PYGZus{}}\PYG{l+s+s1}{\PYGZsq{}}\PYG{p}{]}\PYG{p}{,}\PYG{n}{label}\PYG{o}{=}\PYG{l+s+s1}{\PYGZsq{}}\PYG{l+s+s1}{Scenario}\PYG{l+s+s1}{\PYGZsq{}}\PYG{p}{)}
\PYG{c+c1}{\PYGZsh{}axs[0,0].legend()}

\PYG{n}{axs}\PYG{p}{[}\PYG{l+m+mi}{0}\PYG{p}{,}\PYG{l+m+mi}{1}\PYG{p}{]}\PYG{o}{.}\PYG{n}{plot}\PYG{p}{(}\PYG{n}{mpak}\PYG{o}{.}\PYG{n}{basedf}\PYG{o}{.}\PYG{n}{loc}\PYG{p}{[}\PYG{l+m+mi}{2020}\PYG{p}{:}\PYG{l+m+mi}{2099}\PYG{p}{,}\PYG{l+s+s1}{\PYGZsq{}}\PYG{l+s+s1}{PAKGGDBTTOTLCN\PYGZus{}}\PYG{l+s+s1}{\PYGZsq{}}\PYG{p}{]}\PYG{p}{,}\PYG{n}{label}\PYG{o}{=}\PYG{l+s+s1}{\PYGZsq{}}\PYG{l+s+s1}{Baseline}\PYG{l+s+s1}{\PYGZsq{}}\PYG{p}{)}
\PYG{n}{axs}\PYG{p}{[}\PYG{l+m+mi}{0}\PYG{p}{,}\PYG{l+m+mi}{1}\PYG{p}{]}\PYG{o}{.}\PYG{n}{plot}\PYG{p}{(}\PYG{n}{mpak}\PYG{o}{.}\PYG{n}{lastdf}\PYG{o}{.}\PYG{n}{loc}\PYG{p}{[}\PYG{l+m+mi}{2020}\PYG{p}{:}\PYG{l+m+mi}{2099}\PYG{p}{,}\PYG{l+s+s1}{\PYGZsq{}}\PYG{l+s+s1}{PAKGGDBTTOTLCN\PYGZus{}}\PYG{l+s+s1}{\PYGZsq{}}\PYG{p}{]}\PYG{p}{,}\PYG{n}{label}\PYG{o}{=}\PYG{l+s+s1}{\PYGZsq{}}\PYG{l+s+s1}{Scenario}\PYG{l+s+s1}{\PYGZsq{}}\PYG{p}{)}

\PYG{n}{axs}\PYG{p}{[}\PYG{l+m+mi}{1}\PYG{p}{,}\PYG{l+m+mi}{0}\PYG{p}{]}\PYG{o}{.}\PYG{n}{plot}\PYG{p}{(}\PYG{n}{mpak}\PYG{o}{.}\PYG{n}{basedf}\PYG{o}{.}\PYG{n}{loc}\PYG{p}{[}\PYG{l+m+mi}{2020}\PYG{p}{:}\PYG{l+m+mi}{2099}\PYG{p}{,}\PYG{l+s+s1}{\PYGZsq{}}\PYG{l+s+s1}{PAKGGREVTOTLCN}\PYG{l+s+s1}{\PYGZsq{}}\PYG{p}{]}\PYG{o}{/}\PYG{n}{mpak}\PYG{o}{.}\PYG{n}{basedf}\PYG{o}{.}\PYG{n}{loc}\PYG{p}{[}\PYG{l+m+mi}{2020}\PYG{p}{:}\PYG{l+m+mi}{2099}\PYG{p}{,}\PYG{l+s+s1}{\PYGZsq{}}\PYG{l+s+s1}{PAKNYGDPMKTPCN}\PYG{l+s+s1}{\PYGZsq{}}\PYG{p}{]}\PYG{o}{*}\PYG{l+m+mi}{100}\PYG{p}{,}\PYG{n}{label}\PYG{o}{=}\PYG{l+s+s1}{\PYGZsq{}}\PYG{l+s+s1}{Baseline}\PYG{l+s+s1}{\PYGZsq{}}\PYG{p}{)}
\PYG{n}{axs}\PYG{p}{[}\PYG{l+m+mi}{1}\PYG{p}{,}\PYG{l+m+mi}{0}\PYG{p}{]}\PYG{o}{.}\PYG{n}{plot}\PYG{p}{(}\PYG{n}{mpak}\PYG{o}{.}\PYG{n}{lastdf}\PYG{o}{.}\PYG{n}{loc}\PYG{p}{[}\PYG{l+m+mi}{2020}\PYG{p}{:}\PYG{l+m+mi}{2099}\PYG{p}{,}\PYG{l+s+s1}{\PYGZsq{}}\PYG{l+s+s1}{PAKGGREVTOTLCN}\PYG{l+s+s1}{\PYGZsq{}}\PYG{p}{]}\PYG{o}{/}\PYG{n}{mpak}\PYG{o}{.}\PYG{n}{lastdf}\PYG{o}{.}\PYG{n}{loc}\PYG{p}{[}\PYG{l+m+mi}{2020}\PYG{p}{:}\PYG{l+m+mi}{2099}\PYG{p}{,}\PYG{l+s+s1}{\PYGZsq{}}\PYG{l+s+s1}{PAKNYGDPMKTPCN}\PYG{l+s+s1}{\PYGZsq{}}\PYG{p}{]}\PYG{o}{*}\PYG{l+m+mi}{100}\PYG{p}{,}\PYG{n}{label}\PYG{o}{=}\PYG{l+s+s1}{\PYGZsq{}}\PYG{l+s+s1}{Scenario}\PYG{l+s+s1}{\PYGZsq{}}\PYG{p}{)}

\PYG{n}{axs}\PYG{p}{[}\PYG{l+m+mi}{1}\PYG{p}{,}\PYG{l+m+mi}{1}\PYG{p}{]}\PYG{o}{.}\PYG{n}{plot}\PYG{p}{(}\PYG{n}{mpak}\PYG{o}{.}\PYG{n}{basedf}\PYG{o}{.}\PYG{n}{loc}\PYG{p}{[}\PYG{l+m+mi}{2020}\PYG{p}{:}\PYG{l+m+mi}{2099}\PYG{p}{,}\PYG{l+s+s1}{\PYGZsq{}}\PYG{l+s+s1}{PAKGGREVGRNTCN}\PYG{l+s+s1}{\PYGZsq{}}\PYG{p}{]}\PYG{o}{/}\PYG{n}{mpak}\PYG{o}{.}\PYG{n}{basedf}\PYG{o}{.}\PYG{n}{loc}\PYG{p}{[}\PYG{l+m+mi}{2020}\PYG{p}{:}\PYG{l+m+mi}{2099}\PYG{p}{,}\PYG{l+s+s1}{\PYGZsq{}}\PYG{l+s+s1}{PAKNYGDPMKTPCN}\PYG{l+s+s1}{\PYGZsq{}}\PYG{p}{]}\PYG{o}{*}\PYG{l+m+mi}{100}\PYG{p}{,}\PYG{n}{label}\PYG{o}{=}\PYG{l+s+s1}{\PYGZsq{}}\PYG{l+s+s1}{Baseline}\PYG{l+s+s1}{\PYGZsq{}}\PYG{p}{)}
\PYG{n}{axs}\PYG{p}{[}\PYG{l+m+mi}{1}\PYG{p}{,}\PYG{l+m+mi}{1}\PYG{p}{]}\PYG{o}{.}\PYG{n}{plot}\PYG{p}{(}\PYG{n}{mpak}\PYG{o}{.}\PYG{n}{lastdf}\PYG{o}{.}\PYG{n}{loc}\PYG{p}{[}\PYG{l+m+mi}{2020}\PYG{p}{:}\PYG{l+m+mi}{2099}\PYG{p}{,}\PYG{l+s+s1}{\PYGZsq{}}\PYG{l+s+s1}{PAKGGREVGRNTCN}\PYG{l+s+s1}{\PYGZsq{}}\PYG{p}{]}\PYG{o}{/}\PYG{n}{mpak}\PYG{o}{.}\PYG{n}{lastdf}\PYG{o}{.}\PYG{n}{loc}\PYG{p}{[}\PYG{l+m+mi}{2020}\PYG{p}{:}\PYG{l+m+mi}{2099}\PYG{p}{,}\PYG{l+s+s1}{\PYGZsq{}}\PYG{l+s+s1}{PAKNYGDPMKTPCN}\PYG{l+s+s1}{\PYGZsq{}}\PYG{p}{]}\PYG{o}{*}\PYG{l+m+mi}{100}\PYG{p}{,}\PYG{n}{label}\PYG{o}{=}\PYG{l+s+s1}{\PYGZsq{}}\PYG{l+s+s1}{Scenario}\PYG{l+s+s1}{\PYGZsq{}}\PYG{p}{)}
\PYG{c+c1}{\PYGZsh{}axs2[4].plot(mpak.lastdf.loc[2000:2099,\PYGZsq{}PAKGGREVGRNTCN\PYGZsq{}]/mpak.basedf.loc[2000:2099,\PYGZsq{}PAKNYGDPMKTPCN\PYGZsq{}]*100,label=\PYGZsq{}Scenario\PYGZsq{})}

\PYG{n}{axs}\PYG{p}{[}\PYG{l+m+mi}{0}\PYG{p}{,}\PYG{l+m+mi}{0}\PYG{p}{]}\PYG{o}{.}\PYG{n}{title}\PYG{o}{.}\PYG{n}{set\PYGZus{}text}\PYG{p}{(}\PYG{l+s+s2}{\PYGZdq{}}\PYG{l+s+s2}{Fiscal balance (}\PYG{l+s+si}{\PYGZpc{} o}\PYG{l+s+s2}{f GDP)}\PYG{l+s+s2}{\PYGZdq{}}\PYG{p}{)}
\PYG{n}{axs}\PYG{p}{[}\PYG{l+m+mi}{0}\PYG{p}{,}\PYG{l+m+mi}{1}\PYG{p}{]}\PYG{o}{.}\PYG{n}{title}\PYG{o}{.}\PYG{n}{set\PYGZus{}text}\PYG{p}{(}\PYG{l+s+s2}{\PYGZdq{}}\PYG{l+s+s2}{Gov}\PYG{l+s+s2}{\PYGZsq{}}\PYG{l+s+s2}{t Debt (}\PYG{l+s+si}{\PYGZpc{} o}\PYG{l+s+s2}{f GDP)}\PYG{l+s+s2}{\PYGZdq{}}\PYG{p}{)}
\PYG{n}{axs}\PYG{p}{[}\PYG{l+m+mi}{1}\PYG{p}{,}\PYG{l+m+mi}{0}\PYG{p}{]}\PYG{o}{.}\PYG{n}{title}\PYG{o}{.}\PYG{n}{set\PYGZus{}text}\PYG{p}{(}\PYG{l+s+s2}{\PYGZdq{}}\PYG{l+s+s2}{Total revenues (}\PYG{l+s+si}{\PYGZpc{} o}\PYG{l+s+s2}{f GDP)}\PYG{l+s+s2}{\PYGZdq{}}\PYG{p}{)}
\PYG{n}{axs}\PYG{p}{[}\PYG{l+m+mi}{1}\PYG{p}{,}\PYG{l+m+mi}{1}\PYG{p}{]}\PYG{o}{.}\PYG{n}{title}\PYG{o}{.}\PYG{n}{set\PYGZus{}text}\PYG{p}{(}\PYG{l+s+s2}{\PYGZdq{}}\PYG{l+s+s2}{Grant Revenues (}\PYG{l+s+si}{\PYGZpc{} o}\PYG{l+s+s2}{f GDP)}\PYG{l+s+s2}{\PYGZdq{}}\PYG{p}{)}
\PYG{n}{figure}\PYG{o}{.}\PYG{n}{suptitle}\PYG{p}{(}\PYG{l+s+s2}{\PYGZdq{}}\PYG{l+s+s2}{Fiscal outcomes}\PYG{l+s+s2}{\PYGZdq{}}\PYG{p}{)}

\PYG{n}{plt}\PYG{o}{.}\PYG{n}{figlegend}\PYG{p}{(}\PYG{p}{[}\PYG{l+s+s1}{\PYGZsq{}}\PYG{l+s+s1}{Baseline}\PYG{l+s+s1}{\PYGZsq{}}\PYG{p}{,}\PYG{l+s+s1}{\PYGZsq{}}\PYG{l+s+s1}{Scenario}\PYG{l+s+s1}{\PYGZsq{}}\PYG{p}{]}\PYG{p}{,}\PYG{n}{loc}\PYG{o}{=}\PYG{l+s+s1}{\PYGZsq{}}\PYG{l+s+s1}{lower left}\PYG{l+s+s1}{\PYGZsq{}}\PYG{p}{,}\PYG{n}{ncol}\PYG{o}{=}\PYG{l+m+mi}{5}\PYG{p}{)}  
\PYG{n}{figure}\PYG{o}{.}\PYG{n}{tight\PYGZus{}layout}\PYG{p}{(}\PYG{n}{pad}\PYG{o}{=}\PYG{l+m+mf}{2.3}\PYG{p}{)} \PYG{c+c1}{\PYGZsh{}Ensures legend does not overlap dates}
\PYG{n}{figure}
\end{sphinxVerbatim}

\end{sphinxuseclass}\end{sphinxVerbatimInput}
\begin{sphinxVerbatimOutput}

\begin{sphinxuseclass}{cell_output}
\noindent\sphinxincludegraphics{{af263b8b7718467e4adcce4f07a1db6ce46b66ee3c78d4e8505248794bb2d008}.png}

\end{sphinxuseclass}\end{sphinxVerbatimOutput}

\end{sphinxuseclass}
\sphinxstepscope


\part{More}

\sphinxstepscope

\begin{sphinxthebibliography}{BCJ+19}
\bibitem[Bla18]{content/litterature:id17}
\sphinxAtStartPar
Olivier Blanchard. On the future of Macroeconomic models. \sphinxstyleemphasis{Oxford Review of Economic Policy}, 34(1\sphinxhyphen{}2):43–54, 2018. URL: \sphinxurl{https://academic.oup.com/oxrep/article/34/1-2/43/4781808}, \sphinxhref{https://doi.org/https://doi.org/10.1093/oxrep/grx045}{doi:https://doi.org/10.1093/oxrep/grx045}.
\bibitem[BCJ+19]{content/litterature:id15}
\sphinxAtStartPar
Andrew Burns, Benoit Campagne, Charl Jooste, David Stephan, and Thi Thanh Bui. \sphinxstyleemphasis{The World Bank Macro\sphinxhyphen{}Fiscal Model Technical Description}. Number 8965 in Policy Research Working Papers. World Bank, Washington DC., 2019. URL: \sphinxurl{https://openknowledge.worldbank.org/handle/10986/32217}.
\bibitem[BJS21a]{content/litterature:id14}
\sphinxAtStartPar
Andrew Burns, Charl Jooste, and Gregor Schwerhoff. \sphinxstyleemphasis{Climate Modeling for Macroeconomic Policy : A Case Study for Pakistan}. Number 9780 in Policy Research Working Papers. World Bank, Washington, DC, 2021. URL: \sphinxurl{https://openknowledge.worldbank.org/bitstream/handle/10986/36307/Climate-Modeling-for-Macroeconomic-Policy-A-Case-Study-for-Pakistan.pdf?sequence=1\&isAllowed=y}.
\bibitem[BJS21b]{content/litterature:id18}
\sphinxAtStartPar
Andrew Burns, Charl Jooste, and Gregor Schwerhoff. \sphinxstyleemphasis{Macroeconomic Modeling of Managing Hurricane Damage in the Caribbean: The Case of Jamaica}. Volume 9505 of Policy Research Working Paper. World Bank, Washington DC., 2021. URL: \sphinxurl{https://documents1.worldbank.org/curated/en/593351609776234361/pdf/Macroeconomic-Modeling-of-Managing-Hurricane-Damage-in-the-Caribbean-The-Case-of-Jamaica.pdf}.
\bibitem[DJ13]{content/litterature:id19}
\sphinxAtStartPar
Peter B Dixon and DFale W. Jorgenson. \sphinxstyleemphasis{Handbook of Computable General Equilibrium Modelling}. Volume 1A. Elsevier B.V., 2013. ISBN ISSN: 2211\sphinxhyphen{}6885. URL: \sphinxurl{https://www.sciencedirect.com/handbook/handbook-of-computable-general-equilibrium-modeling/}.
\end{sphinxthebibliography}







\renewcommand{\indexname}{Index}
\printindex
\end{document}