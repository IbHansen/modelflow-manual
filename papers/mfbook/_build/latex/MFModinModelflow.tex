%% Generated by Sphinx.
\def\sphinxdocclass{jupyterBook}
\documentclass[letterpaper,10pt,english]{jupyterBook}
\ifdefined\pdfpxdimen
   \let\sphinxpxdimen\pdfpxdimen\else\newdimen\sphinxpxdimen
\fi \sphinxpxdimen=.75bp\relax
\ifdefined\pdfimageresolution
    \pdfimageresolution= \numexpr \dimexpr1in\relax/\sphinxpxdimen\relax
\fi
%% let collapsible pdf bookmarks panel have high depth per default
\PassOptionsToPackage{bookmarksdepth=5}{hyperref}
%% turn off hyperref patch of \index as sphinx.xdy xindy module takes care of
%% suitable \hyperpage mark-up, working around hyperref-xindy incompatibility
\PassOptionsToPackage{hyperindex=false}{hyperref}
%% memoir class requires extra handling
\makeatletter\@ifclassloaded{memoir}
{\ifdefined\memhyperindexfalse\memhyperindexfalse\fi}{}\makeatother

\PassOptionsToPackage{warn}{textcomp}

\catcode`^^^^00a0\active\protected\def^^^^00a0{\leavevmode\nobreak\ }
\usepackage{cmap}
\usepackage{fontspec}
\defaultfontfeatures[\rmfamily,\sffamily,\ttfamily]{}
\usepackage{amsmath,amssymb,amstext}
\usepackage{polyglossia}
\setmainlanguage{english}



\setmainfont{FreeSerif}[
  Extension      = .otf,
  UprightFont    = *,
  ItalicFont     = *Italic,
  BoldFont       = *Bold,
  BoldItalicFont = *BoldItalic
]
\setsansfont{FreeSans}[
  Extension      = .otf,
  UprightFont    = *,
  ItalicFont     = *Oblique,
  BoldFont       = *Bold,
  BoldItalicFont = *BoldOblique,
]
\setmonofont{FreeMono}[
  Extension      = .otf,
  UprightFont    = *,
  ItalicFont     = *Oblique,
  BoldFont       = *Bold,
  BoldItalicFont = *BoldOblique,
]



\usepackage[Bjarne]{fncychap}
\usepackage[,numfigreset=1,mathnumfig]{sphinx}

\fvset{fontsize=\small}
\usepackage{geometry}


% Include hyperref last.
\usepackage{hyperref}
% Fix anchor placement for figures with captions.
\usepackage{hypcap}% it must be loaded after hyperref.
% Set up styles of URL: it should be placed after hyperref.
\urlstyle{same}

\addto\captionsenglish{\renewcommand{\contentsname}{The World Bank's MFMod Framework and Modelflow}}

\usepackage{sphinxmessages}



        % Start of preamble defined in sphinx-jupyterbook-latex %
         \usepackage[Latin,Greek]{ucharclasses}
        \usepackage{unicode-math}
        % fixing title of the toc
        \addto\captionsenglish{\renewcommand{\contentsname}{Contents}}
        \hypersetup{
            pdfencoding=auto,
            psdextra
        }
        % End of preamble defined in sphinx-jupyterbook-latex %
        

\title{The World Bank's MFMod Framework in Python with Modelflow}
\date{May 31, 2023}
\release{}
\author{Andrew Burns and Ib Hansen}
\newcommand{\sphinxlogo}{\vbox{}}
\renewcommand{\releasename}{}
\makeindex
  \renewcommand{\cleardoublepage}{\clearpage}
\begin{document}

\pagestyle{empty}
\sphinxmaketitle
\pagestyle{plain}
\sphinxtableofcontents
\pagestyle{normal}
\phantomsection\label{\detokenize{content/Introduction::doc}}


\begin{sphinxadmonition}{warning}{Warning:}
\sphinxAtStartPar
This Jupyter Book is work in progress.
\end{sphinxadmonition}

\sphinxAtStartPar
This paper describes the implementation of the World Bank’s MacroFiscalModel (MFMod, see Burns \sphinxstyleemphasis{et al.} {[}\hyperlink{cite.content/99_BackMatter/References:id15}{2019}{]}) using the open source solution program ModelFlow (\sphinxhref{https://ibhansen.github.io/doc/index}{Hansen, 2023}).

\sphinxAtStartPar
The impetus for this paper and the work that it summarizes was to make available to a wider constituency the work that the Bank has done over the past several decades to disseminate Macro\sphinxhyphen{}structural models%
\begin{footnote}[1]\sphinxAtStartFootnote
Economic modelling has a long tradition at the World Bank.  The Bank has had a long\sphinxhyphen{}standing involvement in macroeconomic modelling, initally with linear programming polanning models {[}\hyperlink{cite.content/99_BackMatter/References:id21}{Chenery, 1971}{]}, and then CGE models {[}{]}. Indeed, the popular modelling package GAMS, which is widely used to solve CGE and Linear Programming models, \sphinxhref{https://www.gams.com/about/company/}{started out} as a project begun at the World Bank in the 1976 {[}\hyperlink{cite.content/99_BackMatter/References:id20}{Addison, 1989}{]}.
%
\end{footnote} – notably those that form part of its MFMod (MacroFiscalModel) framework.

\begin{DUlineblock}{0em}
\item[] \sphinxstylestrong{\Large The MFMod Framework at the World Bank}
\end{DUlineblock}

\sphinxAtStartPar
MFMod is the World Bank’s work\sphinxhyphen{}horse macro\sphinxhyphen{}structural economic modelling framework. It exists both a linked system of 184 country specific models that can be solved either independently or as a larger system (\sphinxcode{\sphinxupquote{MFMod}}), and as a series of  standalone customized models, known collectively as MFMod Standalones (MFMod SAs) that have been developed from the central model to the fit the specific needs of individual countries. Both modelling systems can be solved using the EViews modelling languiage, or through the intermediation of an easy\sphinxhyphen{}to\sphinxhyphen{}use excel front end developed by the Bank.

\sphinxAtStartPar
The main \sphinxcode{\sphinxupquote{MFMod}} global model evolved from earlier macro\sphinxhyphen{}structural models developed during the 2000s to strengthen the basis for the forecasts produced by the World Bank. Some examples of these models were released on the World Bank’s isimulate platform early in 2010 along with several CGE models dating from this period. These earlier models were substantially extended into what has become the main MFMod (MacroFiscalModel) model during 2014. Since 2015, MFMod replaced the Bank’s RMSIM\sphinxhyphen{}X model ({[}\hyperlink{cite.content/99_BackMatter/References:id20}{Addison, 1989}{]}), as the Bank’s main tool for forecasting and economic analysis, and is used for the World Bank’s twice annual forecasting exercise \sphinxhref{https://www.worldbank.org/en/publication/macro-poverty-outlook}{The Macro Poverty Outlook}.

\sphinxAtStartPar
The main documentation for \sphinxcode{\sphinxupquote{MFMod}} are Burns \sphinxstyleemphasis{et al.} {[}\hyperlink{cite.content/99_BackMatter/References:id15}{2019}{]}.

\begin{DUlineblock}{0em}
\item[] \sphinxstylestrong{\large Climate aware version of MFMod}
\end{DUlineblock}

\sphinxAtStartPar
Most recently, the Bank has extended the standard MFMod framework to incorporate the main features of climate change {[}\hyperlink{cite.content/99_BackMatter/References:id14}{Burns \sphinxstyleemphasis{et al.}, 2021}{]}– both in terms of the impact of the economy on climate (principally through green\sphinxhyphen{}house gas emissions, like \(CO_2, N_{2}O, CH_4, ...\)) and the impact of the changing climate on the economy (higher temperatures, changes in rainfall quantity and variability, increased incidence of extreme weather) and their impacts on the economy (agricultural output, labor productivity, physical damages due to extreme weather events, sea\sphinxhyphen{}level rises etc.).

\sphinxAtStartPar
Variants on the model initially described in  Burns \sphinxstyleemphasis{et al.} {[}\hyperlink{cite.content/99_BackMatter/References:id14}{2021}{]}, have been developed for {[}xx{]} countries and underpin the economic analysis contained in many of the World Bank’s  \sphinxhref{https://www.worldbank.org/en/publication/country-climate-development-reports}{Country Climate Development Reports}.

\begin{DUlineblock}{0em}
\item[] \sphinxstylestrong{\large Early steps to bring the MFMod system to the broader economics community}
\end{DUlineblock}

\sphinxAtStartPar
Bank staff were quick to recognize that the models built for its own needs could be of use to the broader economics community. An initial project \sphinxcode{\sphinxupquote{isimulate}} made several versions of this earlier model available for simulation on the \sphinxhref{https://isimulate.worldbank.org}{isimulate platform} in 2007, and these models continue to be available there.  The \sphinxcode{\sphinxupquote{isimulate}} platform housed (and continues to house) public access to earlier versions of the MFMod system, and allows simulation of these and other models – but does not give researchers access to the code or the ability to construct complex simulations.

\sphinxAtStartPar
In another effort to make models widely available a large number (more than 60 as of June 2023) customized stand\sphinxhyphen{}alone models (collectively known as called MFModSA \sphinxhyphen{} MacroFiscalModel StandAlones)  have been developed from the main model. Typically developed for a country\sphinxhyphen{}client (Ministry of Finance, Economy or Planning or Central Bank), these Stand Alones extend the standard model by incorporating additional details not in the standard model that are of specific import to different economies and the country\sphinxhyphen{}clients for whom they were built, including: a more detailed breakdown of the sectoral make up of an economy, more detailed fiscal and monetary accounts, and other economically important features of the economy that may exist only inside the aggregates of the standard model.

\sphinxAtStartPar
Training and dissemination around these customized versions of MFMod have been ongoing since 2013. In addition to making customized models available to client governments, Bank teams have run technical assistance program designed to train government officials in the use of these models, their maintenance, modification and revision.

\begin{DUlineblock}{0em}
\item[] \sphinxstylestrong{\large Moving the framework to an open\sphinxhyphen{}source footing}
\end{DUlineblock}

\sphinxAtStartPar
Models in the MFMod family are normally built using the proprietary \DUrole{xref,myst}{EViews} econometric and modelling package. While offering many advantages for model development and maintenance, its cost may be a barrier to clients in developing countries.  As a result, the World Bank joined with Ib Hansen, a Danish economist formerly with the European Central Bank and the Danish Central Bank, who over the years has developed \sphinxcode{\sphinxupquote{modelflow}} a generalized solution engine written in Python for economic models. Together with World Bank, Hansen has worked to extend \sphinxcode{\sphinxupquote{modelflow}} so that MFMod models can be ported and run in the framework.

\sphinxAtStartPar
This paper reports on the results of these efforts. In particular, it provides step by step instructions on how to install the \sphinxcode{\sphinxupquote{modelflow}} framework, import a World Bank macrostructural model,  perform simulations with that model and report results using the many analytical and reporting tools that have been built into \sphinxcode{\sphinxupquote{modelflow}}.  It is not a manual for \sphinxcode{\sphinxupquote{modelflow}}, such a manual can be found \sphinxhref{https://ibhansen.github.io/doc/index}{here} nor is it documentation for the MFMod system, such documentation can be found here {[}\hyperlink{cite.content/99_BackMatter/References:id15}{Burns \sphinxstyleemphasis{et al.}, 2019}{]} and here {[}\hyperlink{cite.content/99_BackMatter/References:id18}{Burns \sphinxstyleemphasis{et al.}, 2021}{]}, {[}\hyperlink{cite.content/99_BackMatter/References:id14}{Burns \sphinxstyleemphasis{et al.}, 2021}{]}). Nor is it documentation for the specific models described and worked with below.


\bigskip\hrule\bigskip


\sphinxstepscope


\part{The World Bank's MFMod Framework and Modelflow}

\sphinxstepscope


\chapter{Macrostructural models}
\label{\detokenize{content/02_MacrostructuralModels/MacroStructuralModels:macrostructural-models}}\label{\detokenize{content/02_MacrostructuralModels/MacroStructuralModels::doc}}
\sphinxAtStartPar
The economics profession uses a wide range of models for different purposes.  Macro\sphinxhyphen{}structural models (also known as semi\sphinxhyphen{}structural or Macro\sphinxhyphen{}econometric models) are a class of models that seek to summarize the most important interconnections and determinants of economic activity in an economy. Computable General Equilibrium (CGE), and Dynamic Stochastic General Equilibrium (DSGE) models are other classes of models that also seek, using somewhat different methodologies, to capture the main economic channels by which the actions of agents (firms, households, governments) interact and help determine the structure, level and rate of growth of economic activity in an economy.

\sphinxAtStartPar
Olivier Blanchard, former Chief Economist at the International Monetary Fund, in a series of articles published between 2016 and 2018 that were summarized in Blanchard {[}\hyperlink{cite.content/99_BackMatter/References:id17}{2018}{]}, lays out his views on the relative strengths and weaknesses of each of these systems, concluding that each has a role to play in helping economists analyze the macro\sphinxhyphen{}economy. Typically, organizations, including the World Bank, use all of these tools, privileging one or the other for specific purposes. Macrostructural models like the MFMod framework are widely used by Central Banks, Ministries of Finance; and professional forecasters both for the purposes of generating forecasts and policy analysis.


\section{A system of equations}
\label{\detokenize{content/02_MacrostructuralModels/MacroStructuralModels:a-system-of-equations}}
\sphinxAtStartPar
Mathematically, macro\sphinxhyphen{}structural models are a system of equations comprised of two kinds of equations and three kinds of variables.

\sphinxAtStartPar
\sphinxstylestrong{Types of variables in macro\sphinxhyphen{}structural models}
\begin{itemize}
\item {} 
\sphinxAtStartPar
\sphinxcode{\sphinxupquote{Identities}} are variables that are determined by a well defined accounting rule that always holds. The famous GDP Identity Y=C+I+G+(X\sphinxhyphen{}M) is one such identity, that indicates that GDP at market prices is definitionally equal to Consumption plus Investment plus Government spending plus Exports less Imports.

\item {} 
\sphinxAtStartPar
\sphinxcode{\sphinxupquote{Behavioural}} variables are determined by equations that typically attempt to summarize an economic (vs accounting) relationship. Thus, the equation that says Real Consumption = f(Disposable Income,the price level, and animal spirits) is a behavioural equation – where the relationship is drawn from economic theory. Because these equations do not fully explain the variation in the dependent variable and the sensitivities of variables to the changes in other variables are uncertain, these equations and their parameters are  typically estimated econometrically and are subject to error.

\item {} 
\sphinxAtStartPar
\sphinxcode{\sphinxupquote{Exogenous}} variables are not determined by the model. Typically there are set either by assumption or from data external to the model.  For an individual country model, the exogenous variables would often include the global price of crude oil  because the level of activity of the economy itself is unlikely to affect the world price of oil.

\end{itemize}

\sphinxAtStartPar
In a fully general form it can be written as:

\begin{sphinxadmonition}{warning}{Warning:}
\sphinxAtStartPar
IB: function names changed some typoes
\end{sphinxadmonition}
\begin{align*}
y_t^1  &=  f^1(y_{t+u}^1...,y_{t+u}^n...,y_t^2...,y_{t}^n...y_{t-r}^1...,y_{t-r}^n,x_t^1...x_{t}^k,...x_{t-s}^1...,x_{t-s}^k) \\
y_t^2  &=  f^2(y_{t+u}^1...,y_{t+u}^n...,y_t^1...,y_{t}^n...y_{t-r}^1...,y_{t-r}^n,x_t^1...x_{t}^k,...x_{t-s}^1...,x_{t-s}^k) \\
\vdots \\
y_t^n  &=  f^n(y_{t+u}^1...,y_{t+u}^n...,y_t^1...,y_{t}^{n-1}...y_{t-r}^1...,y_{t-r}^n,x_t^1...x_{t}^r,x..._{t-s}^1...,x_{t-s}^k)
\end{align*}
\sphinxAtStartPar
where \( y_t^1 \) is one of n endogenous variables and \(x_t^1\) is an exogenous variable and there are as many equations as there are unknown (endogenous variables).

\sphinxAtStartPar
Substituting the variable mnemonics Y, C, I, G, X, M for the simple model the above can be rewritten as as a system of 6 equations in 6 unknowns:
\begin{align*}
Y_t  &=  C_t+I_t+G_t+ (X_t-M_t) \\
C_t &= c_t(C_{t-1},C_{t-2},I_t,G_t,X_t,M_t,P_t)\\
I_t &= i_t(I_{t-1},I_{t-2},C_t,G_t,X_t,M_t,P_t)\\
G_t &= g_t(G_{t-1},G_{t-2},C_t,I_t,X_t,M_t,P_t)\\
X_t &= x_t(X_{t-1},X_{t-2},C_t,I_t,G_t,M_t,P_t,P^f_t)\\
M_t &= m_t(M_{t-1},M_{t-2},C_t,I_t,G_t,X_t,P_t,P^f_t)
\end{align*}
\sphinxAtStartPar
and where \(P_t, P^f_t\) (domestic and foreign prices, respectively) are exogenous in this simple model.

\sphinxstepscope


\chapter{Modelflow and the MFMod models of the World Bank}
\label{\detokenize{content/02_MacrostructuralModels/MFModAndModelFlow:modelflow-and-the-mfmod-models-of-the-world-bank}}\label{\detokenize{content/02_MacrostructuralModels/MFModAndModelFlow::doc}}
\sphinxAtStartPar
At the World Bank models built using the MFMod framework are developed in \DUrole{xref,myst}{EViews}. When disseminated to clients, the models are operated in a World Bank customized EViews environment. But as a systems of equations and associated data the models can be solved, and operated under any system capable of solving a system of simultaneous equations – as long as the equations and data can be transferred from EViews to the secondary system. \sphinxcode{\sphinxupquote{Modelflow}} is such a system and offers a wide range of features that permit not only solving the model, but also provide a rich and powerful suite of tools for analyzing the model and reporting results.

\begin{sphinxadmonition}{warning}{Warning:}
\sphinxAtStartPar
Ib: headings not allowed in margins so it is just there
\end{sphinxadmonition}


\section{A brief history of ModelFlow}
\label{\detokenize{content/02_MacrostructuralModels/MFModAndModelFlow:a-brief-history-of-modelflow}}
\sphinxAtStartPar
Modelflow is a python library that was developed by Ib Hansen over several years while working at the Danish Central Bank and the European Central Bank. The framework has been used both to port the U.S. Federal Reserve’s macro\sphinxhyphen{}structural  model to python, but also been used to bring several stress\sphinxhyphen{}testing models developed by the Danish Central Bank and the European Central Bank into a python environment.

\sphinxAtStartPar
Beginning in 2019, Hansen has worked with the World Bank to develop additional features that facilitate working with models built using the Bank’s MFMod Framework, with the objective of creating an open source platform through which the Bank’s models can be made available to the public.

\sphinxAtStartPar
This paper, and the models that accompany it, are the product of this collaboration.

\sphinxstepscope


\chapter{Installation}
\label{\detokenize{content/03_Installation/Installing:installation}}\label{\detokenize{content/03_Installation/Installing::doc}}
\begin{sphinxadmonition}{warning}{Warning:}
\sphinxAtStartPar
IB:Changed to Miniconda.
consolidated  install python and modelflow for ease of use
\end{sphinxadmonition}

\sphinxAtStartPar
Modelflow is a python package that defines the \sphinxcode{\sphinxupquote{model}} class, its methods and a number of other functions that extend and combine pre\sphinxhyphen{}existing python functions to allow the easy solution of complex systems of equations including macro\sphinxhyphen{}structural models like MFMod.  To work with \sphinxcode{\sphinxupquote{modelflow}}, a user needs to first install python (preferably the Anaconda variant), several supporting packages, and of course the \sphinxcode{\sphinxupquote{modelflow}} package itself.

\sphinxAtStartPar
\sphinxcode{\sphinxupquote{modelflow}} can be run through the Jupyter notebook system, directly from the python command\sphinxhyphen{}line or from IDEs (Interactive Development Environments) like \sphinxcode{\sphinxupquote{Spyder}} or Microsoft’s \sphinxcode{\sphinxupquote{Visual Code}}.

\sphinxAtStartPar
The Jupyter Notebook facilitates an interactive approach to building python programs, annotating them and ultimately doing simulations using MFMod under \sphinxcode{\sphinxupquote{modelflow}}. This entire manual and the examples in it were all written and executed in the Jupyter Notebook environment.


\section{Installation of Python}
\label{\detokenize{content/03_Installation/Installing:installation-of-python}}
\sphinxAtStartPar
\sphinxcode{\sphinxupquote{Python}} is an extremely powerful, versatile and extensible open\sphinxhyphen{}source language. It is widely used for artificial intelligence application, interactive web sites, and scientific processing. As of 14 November 2022, the \sphinxcode{\sphinxupquote{Python Package Index}} (PyPI), the official repository for third\sphinxhyphen{}party Python software, contained over 415,000 packages that extend its functionality %
\begin{footnote}[1]\sphinxAtStartFootnote
\sphinxhref{https://en.wikipedia.org/wiki/Python\_(programming\_language)}{Wikipedia article on python}
%
\end{footnote}. Modelflow is one of these packages.

\sphinxAtStartPar
Python comes in many flavors and \sphinxcode{\sphinxupquote{modelflow}} will work with any of them.  However, \sphinxstylestrong{users are strongly advised to use the Anaconda version of Python}. It is advised to use the \sphinxstylestrong{Miniconda} version of the package manager, this is will only install the minimum needed packages. However the full \sphinxstylestrong{Anaconda} version can also be used.

\sphinxAtStartPar
The remainder of this section points to instructions on how to install the Miniconda version of python under Windows. \sphinxcode{\sphinxupquote{Modelflow}} works equally well under linux and should work under MacOS.

\sphinxAtStartPar
This is followed by section that describes the steps necessary to create an anaconda environment with all the necessary packages to run \sphinxcode{\sphinxupquote{modelflow}}.


\section{Installation of Miniconda under Windows:}
\label{\detokenize{content/03_Installation/Installing:installation-of-miniconda-under-windows}}
\sphinxAtStartPar
The easy way to install is to use the \sphinxstylestrong{Miniconda} package manager from Anaconda. You can find a link \sphinxhref{https://docs.conda.io/en/latest/miniconda.html}{here}. The latest version can be downloaded and installed by execution these command in a command window.

\sphinxAtStartPar
Open the a command prompt

\sphinxAtStartPar
\sphinxstylestrong{First} copy/paste these lines to the command prompt

\begin{sphinxVerbatim}[commandchars=\\\{\}]
\PYG{n}{curl} \PYG{o}{\PYGZhy{}}\PYG{n}{L} \PYG{l+s+s2}{\PYGZdq{}}\PYG{l+s+s2}{https://repo.anaconda.com/miniconda/Miniconda3\PYGZhy{}latest\PYGZhy{}Windows\PYGZhy{}x86\PYGZus{}64.exe}\PYG{l+s+s2}{\PYGZdq{}} \PYG{o}{\PYGZhy{}}\PYG{o}{\PYGZhy{}}\PYG{n}{output} \PYG{o}{\PYGZpc{}}\PYG{n}{TEMP}\PYG{o}{\PYGZpc{}}\PYGZbs{}\PYG{n}{miniconda}\PYG{o}{.}\PYG{n}{exe}
\PYG{o}{\PYGZpc{}}\PYG{n}{temp}\PYG{o}{\PYGZpc{}}\PYGZbs{}\PYG{n}{miniconda}\PYG{o}{.}\PYG{n}{exe} \PYG{o}{/}\PYG{n}{S} \PYG{o}{/}\PYG{n}{D}\PYG{o}{=}\PYG{o}{\PYGZpc{}}\PYG{n}{USERPROFILE}\PYG{o}{\PYGZpc{}}\PYGZbs{}\PYG{n}{miniconda3}

\end{sphinxVerbatim}

\sphinxAtStartPar
These lines will download miniconda and install it in the users folder

\sphinxAtStartPar
\sphinxstylestrong{Then} copy/paste these lines to the command prompt

\begin{sphinxVerbatim}[commandchars=\\\{\}]
\PYG{n}{call} \PYG{o}{\PYGZpc{}}\PYG{n}{USERPROFILE}\PYG{o}{\PYGZpc{}}\PYGZbs{}\PYG{n}{miniconda3}\PYGZbs{}\PYG{n}{Scripts}\PYGZbs{}\PYG{n}{activate}\PYG{o}{.}\PYG{n}{bat} \PYG{o}{\PYGZpc{}}\PYG{n}{USERPROFILE}\PYG{o}{\PYGZpc{}}\PYGZbs{}\PYG{n}{miniconda3}
\PYG{n}{call} \PYG{n}{conda} \PYG{n}{install} \PYG{n}{mamba} \PYG{o}{\PYGZhy{}}\PYG{n}{c} \PYG{n}{conda}\PYG{o}{\PYGZhy{}}\PYG{n}{forge} \PYG{o}{\PYGZhy{}}\PYG{n}{y}

\end{sphinxVerbatim}

\sphinxAtStartPar
These lines will install the mamba package manager. Mamba is a faster alternative to the conda default package manager in miniconda.


\subsection{Installation of Python under macOS}
\label{\detokenize{content/03_Installation/Installing:installation-of-python-under-macos}}
\sphinxAtStartPar
The definitive source for installing Anaconda under macOS can be found \sphinxhref{https://docs.anaconda.com/anaconda/install/mac-os/}{here}.


\subsection{Installation of Python under Linux}
\label{\detokenize{content/03_Installation/Installing:installation-of-python-under-linux}}
\sphinxAtStartPar
The definitive source for installing Anaconda under Linux can be found \sphinxhref{https://docs.anaconda.com/anaconda/install/linux/}{here}.


\section{Installation of Modelflow}
\label{\detokenize{content/03_Installation/Installing:installation-of-modelflow}}
\sphinxAtStartPar
\sphinxcode{\sphinxupquote{Modelflow}} is a python package that defines the modelflow class \sphinxcode{\sphinxupquote{model}} among others.  \sphinxcode{\sphinxupquote{Modelflow}} has many dependencies. Installing the class the first time can take some time depending on your internet connection and computer speed.  It is essential that you follow all of the steps outlined below to ensure that your version of \sphinxcode{\sphinxupquote{modelflow}} operates as expected.

\begin{sphinxadmonition}{warning}{Warning:}
\sphinxAtStartPar
The following instructions concern the installation of \sphinxcode{\sphinxupquote{modelflow}} within an Anaconda installation of python.  Different flavors of Python may require slight changes to this recipe, but are not covered here.

\sphinxAtStartPar
\sphinxcode{\sphinxupquote{Modelflow}} is built and tested using the anaconda python environment.  It is strongly recommended to use Anaconda with \sphinxcode{\sphinxupquote{modelflow}}.

\sphinxAtStartPar
If you have not already installed Anaconda following the instructions in the preceding section, please do so \sphinxstylestrong{Now}.
\end{sphinxadmonition}


\section{Installation of \sphinxstyleliteralintitle{\sphinxupquote{modelflow}} using Miniconda}
\label{\detokenize{content/03_Installation/Installing:installation-of-modelflow-using-miniconda}}
\sphinxAtStartPar
Open the a command prompt or continue in the command prompt used to install python.

\sphinxAtStartPar
copy/paste these lines to the command prompt

\begin{sphinxVerbatim}[commandchars=\\\{\}]
\PYG{o}{\PYGZpc{}}\PYG{n}{USERPROFILE}\PYG{o}{\PYGZpc{}}\PYGZbs{}\PYG{n}{miniconda3}\PYGZbs{}\PYG{n}{Scripts}\PYGZbs{}\PYG{n}{activate}\PYG{o}{.}\PYG{n}{bat} \PYG{o}{\PYGZpc{}}\PYG{n}{USERPROFILE}\PYG{o}{\PYGZpc{}}\PYGZbs{}\PYG{n}{miniconda3} 

\PYG{n}{mamba} \PYG{n}{create} \PYG{o}{\PYGZhy{}}\PYG{n}{n} \PYG{n}{modelflow} \PYG{o}{\PYGZhy{}}\PYG{n}{c} \PYG{n}{ibh} \PYG{o}{\PYGZhy{}}\PYG{n}{c}  \PYG{n}{conda}\PYG{o}{\PYGZhy{}}\PYG{n}{forge} \PYG{n}{modelflow\PYGZus{}stable} \PYG{o}{\PYGZhy{}}\PYG{n}{y}

\PYG{n}{conda} \PYG{n}{activate} \PYG{n}{modelflow}
\PYG{n}{pip} \PYG{n}{install} \PYG{n}{dash\PYGZus{}interactive\PYGZus{}graphviz}
\PYG{n}{jupyter} \PYG{n}{contrib} \PYG{n}{nbextension} \PYG{n}{install} \PYG{o}{\PYGZhy{}}\PYG{o}{\PYGZhy{}}\PYG{n}{user}
\PYG{n}{jupyter} \PYG{n}{nbextension} \PYG{n}{enable} \PYG{n}{hide\PYGZus{}input\PYGZus{}all}\PYG{o}{/}\PYG{n}{main}
\PYG{n}{jupyter} \PYG{n}{nbextension} \PYG{n}{enable} \PYG{n}{splitcell}\PYG{o}{/}\PYG{n}{splitcell}
\PYG{n}{jupyter} \PYG{n}{nbextension} \PYG{n}{enable} \PYG{n}{toc2}\PYG{o}{/}\PYG{n}{main}
\PYG{n}{jupyter} \PYG{n}{nbextension} \PYG{n}{enable} \PYG{n}{varInspector}\PYG{o}{/}\PYG{n}{main}


\end{sphinxVerbatim}

\sphinxAtStartPar
Depending on the speed of your computer and of your internet connection installation could take as little as 10 minutes or more than 1/2 an hour.

\sphinxAtStartPar
At the end of the process you will have a new conda environment called \sphinxcode{\sphinxupquote{modelflow}}, and this will have been activated. The computer set up is complete and the user is ready to work with \sphinxcode{\sphinxupquote{modelflow}}.

\sphinxAtStartPar
The following sections give a brief introduction to Jupyter notebook, which is a flexible tool that allows us to execute python code, interact with the modelflow class and World Bank Models and annotate what we have done for future replication.


\section{Updating Modelflow}
\label{\detokenize{content/03_Installation/Installing:updating-modelflow}}
\sphinxAtStartPar
Once installed, a \sphinxcode{\sphinxupquote{modelflow}} environment can be updated. This is done first by activating the \sphinxcode{\sphinxupquote{modelflow}} environment created above.

\begin{sphinxVerbatim}[commandchars=\\\{\}]
\PYG{o}{\PYGZpc{}}\PYG{n}{USERPROFILE}\PYG{o}{\PYGZpc{}}\PYGZbs{}\PYG{n}{miniconda3}\PYGZbs{}\PYG{n}{Scripts}\PYGZbs{}\PYG{n}{activate}\PYG{o}{.}\PYG{n}{bat} \PYG{o}{\PYGZpc{}}\PYG{n}{USERPROFILE}\PYG{o}{\PYGZpc{}}\PYGZbs{}\PYG{n}{miniconda3} 
\PYG{n}{conda} \PYG{n}{activate} \PYG{n}{modelflow}

\end{sphinxVerbatim}

\sphinxAtStartPar
And then executing the command:

\begin{sphinxVerbatim}[commandchars=\\\{\}]
\PYG{n}{mamba} \PYG{n}{install} \PYG{n}{modelflow} \PYG{o}{\PYGZhy{}}\PYG{n}{c} \PYG{n}{ibh} \PYG{o}{\PYGZhy{}}\PYG{o}{\PYGZhy{}}\PYG{n}{no}\PYG{o}{\PYGZhy{}}\PYG{n}{deps}



\end{sphinxVerbatim}


\section{Starting a Python session with modelflow.}
\label{\detokenize{content/03_Installation/Installing:starting-a-python-session-with-modelflow}}
\sphinxAtStartPar
This can be done by starting the \sphinxcode{\sphinxupquote{Anaconda Prompt (miniconda3)}} app. This will create a command window with a base python environment. Now you want to change the environment to the modelflow environment. This is done like this:

\begin{sphinxVerbatim}[commandchars=\\\{\}]
\PYG{n}{conda} \PYG{n}{activate} \PYG{n}{modelflow}

\end{sphinxVerbatim}

\sphinxAtStartPar
Now we are ready to start jupyter by:

\begin{sphinxVerbatim}[commandchars=\\\{\}]
\PYG{n}{cd} \PYG{o}{\PYGZlt{}}\PYG{n}{path} \PYG{n}{where} \PYG{n}{you} \PYG{n}{want} \PYG{n}{to} \PYG{n}{start}\PYG{o}{\PYGZgt{}} 
\PYG{n}{jupyter} \PYG{n}{notebook}
\end{sphinxVerbatim}


\section{Creating a .lnk file to start Jupyter with modelflow.}
\label{\detokenize{content/03_Installation/Installing:creating-a-lnk-file-to-start-jupyter-with-modelflow}}
\sphinxAtStartPar
It can be convenient to create a icon on the desktop which start up jupyter in a modelflow environment. This is done by creating a new shortcut where the \sphinxstylestrong{destination/target} is:

\begin{sphinxVerbatim}[commandchars=\\\{\}]
\PYG{o}{\PYGZpc{}}\PYG{n}{windir}\PYG{o}{\PYGZpc{}}\PYGZbs{}\PYG{n}{System32}\PYGZbs{}\PYG{n}{cmd}\PYG{o}{.}\PYG{n}{exe} \PYG{l+s+s2}{\PYGZdq{}}\PYG{l+s+s2}{/K}\PYG{l+s+s2}{\PYGZdq{}} \PYG{o}{\PYGZpc{}}\PYG{n}{USERPROFILE}\PYG{o}{\PYGZpc{}}\PYGZbs{}\PYG{n}{miniconda3}\PYGZbs{}\PYG{n}{Scripts}\PYGZbs{}\PYG{n}{activate}\PYG{o}{.}\PYG{n}{bat} \PYG{o}{\PYGZpc{}}\PYG{n}{USERPROFILE}\PYG{o}{\PYGZpc{}}\PYGZbs{}\PYG{n}{miniconda3}\PYGZbs{}\PYG{n}{envs}\PYGZbs{}\PYG{n}{modelflow}\PYG{o}{\PYGZam{}}\PYG{o}{\PYGZam{}}\PYG{n}{jupyter} \PYG{n}{notebook}
\end{sphinxVerbatim}

\sphinxAtStartPar
This will create a command window, activate modelflow and start jupyter notebook

\sphinxAtStartPar
and the \sphinxstylestrong{Start in}  is set to the folder where you want jupyter to start from. 
Be aware that jupyter can’t access notebooks “before” the \sphinxcode{\sphinxupquote{Start in}} location. \sphinxcode{\sphinxupquote{C:\textbackslash{}}} could be a choice.


\bigskip\hrule\bigskip


\sphinxstepscope


\chapter{Testing your installation of modelflow}
\label{\detokenize{content/03_Installation/TestingModelFlow:testing-your-installation-of-modelflow}}\label{\detokenize{content/03_Installation/TestingModelFlow::doc}}
\sphinxAtStartPar
To test that the installation of modelflow has worked properly, we will build a model using the modelflow framework and then simulate it.  A simple model that illustrates many of the functions of modelflow is the Solow growth model.

\sphinxAtStartPar
The code below first sets up the python environment by importing the modelflow  and pandas classes.  The initial two lines of code and the final two lines just set up the environment for optimal display and are not required.

\sphinxAtStartPar
To test the installation on your system you can copy this code into a Jupyter notebook and execute it.

\begin{sphinxuseclass}{cell}\begin{sphinxVerbatimInput}

\begin{sphinxuseclass}{cell_input}
\begin{sphinxVerbatim}[commandchars=\\\{\}]
\PYG{c+c1}{\PYGZsh{}Required packages}
\PYG{k+kn}{import} \PYG{n+nn}{pandas} \PYG{k}{as} \PYG{n+nn}{pd}

\PYG{c+c1}{\PYGZsh{} Modelflow modules}
\PYG{k+kn}{from} \PYG{n+nn}{modelclass} \PYG{k+kn}{import} \PYG{n}{model}
\PYG{k+kn}{from} \PYG{n+nn}{modelmf} \PYG{k+kn}{import} \PYG{n}{model}

   
\PYG{c+c1}{\PYGZsh{}for publication }
\PYG{n}{latex}\PYG{o}{=}\PYG{l+m+mi}{0}
\PYG{n}{model}\PYG{o}{.}\PYG{n}{widescreen}\PYG{p}{(}\PYG{p}{)}
\end{sphinxVerbatim}

\end{sphinxuseclass}\end{sphinxVerbatimInput}
\begin{sphinxVerbatimOutput}

\begin{sphinxuseclass}{cell_output}
\begin{sphinxVerbatim}[commandchars=\\\{\}]
\PYGZlt{}IPython.core.display.HTML object\PYGZgt{}
\end{sphinxVerbatim}

\end{sphinxuseclass}\end{sphinxVerbatimOutput}

\end{sphinxuseclass}
\index{citations@\spxentry{citations}|see{bibliographies}}\ignorespaces 
\sphinxAtStartPar
\DUrole{xref,myst}{using markdown link syntax}: using markdown link syntax

\index{Entry name@\spxentry{Entry name}}\ignorespaces 

\section{Specifying the model}
\label{\detokenize{content/03_Installation/TestingModelFlow:specifying-the-model}}\label{\detokenize{content/03_Installation/TestingModelFlow:index-1}}
\sphinxAtStartPar
Having loaded the model class from the modelflow library, we can start constructing the model.

\sphinxAtStartPar
The first step is to define the equations of the model, using \sphinxcode{\sphinxupquote{modelflow}}’s Business Logic Language.

\begin{sphinxShadowBox}
\sphinxstylesidebartitle{\sphinxstylestrong{Business Logic Language}}

\sphinxAtStartPar
More on how to specify models \DUrole{xref,myst}{here}
\end{sphinxShadowBox}

\sphinxAtStartPar
The below code segment defines a string fsolow that contains the equations for the solow model, where:
\begin{itemize}
\item {} 
\sphinxAtStartPar
GDP is defined as a simple Cobb\sphinxhyphen{}Douglas production function as the product of TFP, Capital (raised to the share of capital in total income) and Labour (raised to the share of labor in total income)

\item {} 
\sphinxAtStartPar
Investment is equal to GDP less consumption

\item {} 
\sphinxAtStartPar
The change in capital is equal to investment this period less the depreciation of the capital stock from the previous period

\item {} 
\sphinxAtStartPar
Labor grows at the rate of growth of the variable \sphinxcode{\sphinxupquote{Labor\_growth}}

\item {} 
\sphinxAtStartPar
a pure reporting identity Capital\_intensity the ratio of the Capital Stock to the Labor input

\end{itemize}

\sphinxAtStartPar
We thus have a system of 6 equations with 6 unknowns (GDP, Consumption, Investment, Change in the Capital stock, and change in Labor supply, and the capital\_intensity) and exogenous variables (TFP, alfa,savings\_rate,Depreciation\_rate and Labor\_growth).

\begin{sphinxadmonition}{note}{Note}

\sphinxAtStartPar
The equations for Labor and Capital have been entered as difference equations.
The \sphinxcode{\sphinxupquote{modelflow}} object will automatically normalize them, generating an internal representation of \sphinxcode{\sphinxupquote{Labour=Labour(t\sphinxhyphen{}1)*(1+Labor\_growth)}}
and \sphinxcode{\sphinxupquote{Capital=Capital(t\sphinxhyphen{}1)*(1\sphinxhyphen{}Depreciation\_rate)+Investment}}
\end{sphinxadmonition}

\begin{sphinxuseclass}{cell}\begin{sphinxVerbatimInput}

\begin{sphinxuseclass}{cell_input}
\begin{sphinxVerbatim}[commandchars=\\\{\}]
\PYG{n}{fsolow} \PYG{o}{=} \PYG{l+s+s1}{\PYGZsq{}\PYGZsq{}\PYGZsq{}}\PYG{l+s+se}{\PYGZbs{}}
\PYG{l+s+s1}{GDP          = TFP  * Capital**alfa * Labor **(1\PYGZhy{}alfa) }
\PYG{l+s+s1}{Consumption     = (1\PYGZhy{}saving\PYGZus{}rate)  * GDP }
\PYG{l+s+s1}{Investment      = GDP \PYGZhy{} Consumption   }
\PYG{l+s+s1}{diff(Capital)   = Investment\PYGZhy{}Depreciation\PYGZus{}rate * Capital(\PYGZhy{}1)}
\PYG{l+s+s1}{diff(Labor)     = Labor\PYGZus{}growth * Labor(\PYGZhy{}1)  }
\PYG{l+s+s1}{Capital\PYGZus{}intensity = Capital/Labor }
\PYG{l+s+s1}{\PYGZsq{}\PYGZsq{}\PYGZsq{}}
\end{sphinxVerbatim}

\end{sphinxuseclass}\end{sphinxVerbatimInput}

\end{sphinxuseclass}
\sphinxAtStartPar
To create the model we instantiate (create) a variable \sphinxcode{\sphinxupquote{msolow}} (which will ultimately contain both the equations and data for the model) using the \sphinxcode{\sphinxupquote{.from\_eq()}} method of the \sphinxcode{\sphinxupquote{modelflow}} class – submitting to it the equations in string form, and giving it the name “Solow model”.

\begin{sphinxuseclass}{cell}\begin{sphinxVerbatimInput}

\begin{sphinxuseclass}{cell_input}
\begin{sphinxVerbatim}[commandchars=\\\{\}]
\PYG{n}{msolow} \PYG{o}{=} \PYG{n}{model}\PYG{o}{.}\PYG{n}{from\PYGZus{}eq}\PYG{p}{(}\PYG{n}{fsolow}\PYG{p}{,}\PYG{n}{modelname}\PYG{o}{=}\PYG{l+s+s1}{\PYGZsq{}}\PYG{l+s+s1}{Solow model}\PYG{l+s+s1}{\PYGZsq{}}\PYG{p}{)}
\end{sphinxVerbatim}

\end{sphinxuseclass}\end{sphinxVerbatimInput}

\end{sphinxuseclass}
\sphinxAtStartPar
The internal representation of the normalized equations can be displayed in normalized business language with the \sphinxcode{\sphinxupquote{modelflow}} method \sphinxcode{\sphinxupquote{.print\_model}}:

\begin{sphinxuseclass}{cell}\begin{sphinxVerbatimInput}

\begin{sphinxuseclass}{cell_input}
\begin{sphinxVerbatim}[commandchars=\\\{\}]
\PYG{n}{msolow}\PYG{o}{.}\PYG{n}{print\PYGZus{}model}
\end{sphinxVerbatim}

\end{sphinxuseclass}\end{sphinxVerbatimInput}
\begin{sphinxVerbatimOutput}

\begin{sphinxuseclass}{cell_output}
\begin{sphinxVerbatim}[commandchars=\\\{\}]
FRML \PYGZlt{}\PYGZgt{} GDP          = TFP  * CAPITAL**ALFA * LABOR **(1\PYGZhy{}ALFA)  \PYGZdl{}
FRML \PYGZlt{}\PYGZgt{} CONSUMPTION     = (1\PYGZhy{}SAVING\PYGZus{}RATE)  * GDP  \PYGZdl{}
FRML \PYGZlt{}\PYGZgt{} INVESTMENT      = GDP \PYGZhy{} CONSUMPTION    \PYGZdl{}
FRML \PYGZlt{}\PYGZgt{} CAPITAL=CAPITAL(\PYGZhy{}1)+(INVESTMENT\PYGZhy{}DEPRECIATION\PYGZus{}RATE * CAPITAL(\PYGZhy{}1))\PYGZdl{}
FRML \PYGZlt{}\PYGZgt{} LABOR=LABOR(\PYGZhy{}1)+(LABOR\PYGZus{}GROWTH * LABOR(\PYGZhy{}1))\PYGZdl{}
FRML \PYGZlt{}\PYGZgt{} CAPITAL\PYGZus{}INTENSITY = CAPITAL/LABOR  \PYGZdl{}
\end{sphinxVerbatim}

\end{sphinxuseclass}\end{sphinxVerbatimOutput}

\end{sphinxuseclass}

\section{Create some data}
\label{\detokenize{content/03_Installation/TestingModelFlow:create-some-data}}
\sphinxAtStartPar
For the moment \sphinxcode{\sphinxupquote{msolow}} has a mathematical representation of a system of equations but no data.

\sphinxAtStartPar
To add data we  create a pandas dataframe with initial values for our exogenous variables. Technically capital and labor are endogenous in the Solow model, but because they are specified as change equations their initial values are exogenous and need to be initialized.

\sphinxAtStartPar
The code below  instantiates (creates) a panda dataframe \sphinxcode{\sphinxupquote{df}} and fills it with the variables for our model, initializing these with a series of values over 300 datapoints.  The final command displays the first ten rows of the dataframe.

\begin{sphinxadmonition}{note}{Note:}
\begin{sphinxVerbatim}[commandchars=\\\{\}]
Pandas data frames is a foundational class of python.  There are thousands of web sites dedicated to understanding pandas.  Some notable ones include:
\end{sphinxVerbatim}
\end{sphinxadmonition}

\begin{sphinxuseclass}{cell}\begin{sphinxVerbatimInput}

\begin{sphinxuseclass}{cell_input}
\begin{sphinxVerbatim}[commandchars=\\\{\}]
\PYG{n}{N} \PYG{o}{=} \PYG{l+m+mi}{300}  
\PYG{n}{df} \PYG{o}{=} \PYG{n}{pd}\PYG{o}{.}\PYG{n}{DataFrame}\PYG{p}{(}\PYG{p}{\PYGZob{}}\PYG{l+s+s1}{\PYGZsq{}}\PYG{l+s+s1}{LABOR}\PYG{l+s+s1}{\PYGZsq{}}\PYG{p}{:}\PYG{p}{[}\PYG{l+m+mi}{100}\PYG{p}{]}\PYG{o}{*}\PYG{n}{N}\PYG{p}{,}
                   \PYG{l+s+s1}{\PYGZsq{}}\PYG{l+s+s1}{CAPITAL}\PYG{l+s+s1}{\PYGZsq{}}\PYG{p}{:}\PYG{p}{[}\PYG{l+m+mi}{100}\PYG{p}{]}\PYG{o}{*}\PYG{n}{N}\PYG{p}{,} 
                   \PYG{l+s+s1}{\PYGZsq{}}\PYG{l+s+s1}{ALFA}\PYG{l+s+s1}{\PYGZsq{}}\PYG{p}{:}\PYG{p}{[}\PYG{l+m+mf}{0.5}\PYG{p}{]}\PYG{o}{*}\PYG{n}{N}\PYG{p}{,} 
                   \PYG{l+s+s1}{\PYGZsq{}}\PYG{l+s+s1}{TFP}\PYG{l+s+s1}{\PYGZsq{}}\PYG{p}{:} \PYG{p}{[}\PYG{l+m+mi}{1}\PYG{p}{]}\PYG{o}{*}\PYG{n}{N}\PYG{p}{,} 
                   \PYG{l+s+s1}{\PYGZsq{}}\PYG{l+s+s1}{DEPRECIATION\PYGZus{}RATE}\PYG{l+s+s1}{\PYGZsq{}}\PYG{p}{:} \PYG{p}{[}\PYG{l+m+mf}{0.05}\PYG{p}{]}\PYG{o}{*}\PYG{n}{N}\PYG{p}{,} 
                   \PYG{l+s+s1}{\PYGZsq{}}\PYG{l+s+s1}{LABOR\PYGZus{}GROWTH}\PYG{l+s+s1}{\PYGZsq{}}\PYG{p}{:} \PYG{p}{[}\PYG{l+m+mf}{0.01}\PYG{p}{]}\PYG{o}{*}\PYG{n}{N}\PYG{p}{,} 
                   \PYG{l+s+s1}{\PYGZsq{}}\PYG{l+s+s1}{SAVING\PYGZus{}RATE}\PYG{l+s+s1}{\PYGZsq{}}\PYG{p}{:}\PYG{p}{[}\PYG{l+m+mf}{0.05}\PYG{p}{]}\PYG{o}{*}\PYG{n}{N}\PYG{p}{\PYGZcb{}}\PYG{p}{,}\PYG{n}{index}\PYG{o}{=}\PYG{p}{[}\PYG{n}{v} \PYG{k}{for} \PYG{n}{v} \PYG{o+ow}{in} \PYG{n+nb}{range}\PYG{p}{(}\PYG{l+m+mi}{2000}\PYG{p}{,}\PYG{l+m+mi}{2300}\PYG{p}{)}\PYG{p}{]}\PYG{p}{)}
\PYG{n}{df}\PYG{o}{.}\PYG{n}{head}\PYG{p}{(}\PYG{p}{)} \PYG{c+c1}{\PYGZsh{}this prints out the first 5 rows of the dataframe}
\end{sphinxVerbatim}

\end{sphinxuseclass}\end{sphinxVerbatimInput}
\begin{sphinxVerbatimOutput}

\begin{sphinxuseclass}{cell_output}
\begin{sphinxVerbatim}[commandchars=\\\{\}]
      LABOR  CAPITAL  ALFA  TFP  DEPRECIATION\PYGZus{}RATE  LABOR\PYGZus{}GROWTH  SAVING\PYGZus{}RATE
2000    100      100   0.5    1               0.05          0.01         0.05
2001    100      100   0.5    1               0.05          0.01         0.05
2002    100      100   0.5    1               0.05          0.01         0.05
2003    100      100   0.5    1               0.05          0.01         0.05
2004    100      100   0.5    1               0.05          0.01         0.05
\end{sphinxVerbatim}

\end{sphinxuseclass}\end{sphinxVerbatimOutput}

\end{sphinxuseclass}

\section{Putting it together}
\label{\detokenize{content/03_Installation/TestingModelFlow:putting-it-together}}
\sphinxAtStartPar
Having defined an initial data set for all the exogenous variables, we can combine these with the equations and solve the model.

\sphinxAtStartPar
The command below solves the model \sphinxcode{\sphinxupquote{msolow}} on the data contained in the dataframe \sphinxcode{\sphinxupquote{df}} and stores the output in a new dataframe called \sphinxcode{\sphinxupquote{result}}.

\sphinxAtStartPar
The last line displays the values of the simulated model, which now includes results for the endogenous variables, and different values for the Labor and Capital variables reflecting their endogeneity for periods 2 through 300.

\begin{sphinxuseclass}{cell}\begin{sphinxVerbatimInput}

\begin{sphinxuseclass}{cell_input}
\begin{sphinxVerbatim}[commandchars=\\\{\}]
\PYG{n}{result} \PYG{o}{=} \PYG{n}{msolow}\PYG{p}{(}\PYG{n}{df}\PYG{p}{,}\PYG{n}{keep}\PYG{o}{=}\PYG{l+s+s1}{\PYGZsq{}}\PYG{l+s+s1}{Baseline}\PYG{l+s+s1}{\PYGZsq{}}\PYG{p}{)} 
\PYG{c+c1}{\PYGZsh{} The model is simulated for all years possible }

\PYG{n}{result}\PYG{o}{.}\PYG{n}{head}\PYG{p}{(}\PYG{l+m+mi}{10}\PYG{p}{)}
\end{sphinxVerbatim}

\end{sphinxuseclass}\end{sphinxVerbatimInput}
\begin{sphinxVerbatimOutput}

\begin{sphinxuseclass}{cell_output}
\begin{sphinxVerbatim}[commandchars=\\\{\}]
           LABOR     CAPITAL  ALFA  TFP  DEPRECIATION\PYGZus{}RATE  LABOR\PYGZus{}GROWTH   
2000  100.000000  100.000000   0.5  1.0               0.05          0.01  \PYGZbs{}
2001  101.000000  100.025580   0.5  1.0               0.05          0.01   
2002  102.010000  100.076226   0.5  1.0               0.05          0.01   
2003  103.030100  100.151443   0.5  1.0               0.05          0.01   
2004  104.060401  100.250762   0.5  1.0               0.05          0.01   
2005  105.101005  100.373733   0.5  1.0               0.05          0.01   
2006  106.152015  100.519926   0.5  1.0               0.05          0.01   
2007  107.213535  100.688931   0.5  1.0               0.05          0.01   
2008  108.285671  100.880357   0.5  1.0               0.05          0.01   
2009  109.368527  101.093830   0.5  1.0               0.05          0.01   

      SAVING\PYGZus{}RATE  INVESTMENT  CAPITAL\PYGZus{}INTENSITY  CONSUMPTION         GDP  
2000         0.05    0.000000           0.000000     0.000000    0.000000  
2001         0.05    5.025580           0.990352    95.486029  100.511609  
2002         0.05    5.051924           0.981043    95.986562  101.038487  
2003         0.05    5.079029           0.972060    96.501546  101.580575  
2004         0.05    5.106891           0.963390    97.030930  102.137821  
2005         0.05    5.135509           0.955022    97.574667  102.710176  
2006         0.05    5.164880           0.946943    98.132713  103.297593  
2007         0.05    5.195002           0.939144    98.705029  103.900030  
2008         0.05    5.225872           0.931613    99.291576  104.517449  
2009         0.05    5.257491           0.924341    99.892323  105.149813  
\end{sphinxVerbatim}

\end{sphinxuseclass}\end{sphinxVerbatimOutput}

\end{sphinxuseclass}

\section{Create a scenario and run again}
\label{\detokenize{content/03_Installation/TestingModelFlow:create-a-scenario-and-run-again}}
\begin{sphinxShadowBox}
\sphinxstylesidebartitle{\sphinxstylestrong{dataframe.upd}}

\sphinxAtStartPar
When importing modelclass all pandas dataframes are enriched with a a handy way to create a new pandas dataframe as a copy of an existing one but with one or more series updated.

\sphinxAtStartPar
In this case df.upd will create a a new dataframe \sphinxcode{\sphinxupquote{dfscenaario}} with updated LABOR\_GROWTH

\sphinxAtStartPar
For more detail on the \sphinxcode{\sphinxupquote{.upd}} method look here \DUrole{xref,myst}{here}
\end{sphinxShadowBox}

\begin{sphinxuseclass}{cell}\begin{sphinxVerbatimInput}

\begin{sphinxuseclass}{cell_input}
\begin{sphinxVerbatim}[commandchars=\\\{\}]
\PYG{n}{dfscenario} \PYG{o}{=} \PYG{n}{df}\PYG{o}{.}\PYG{n}{mfcalc}\PYG{p}{(}\PYG{l+s+s1}{\PYGZsq{}}\PYG{l+s+s1}{\PYGZlt{}2023 2200\PYGZgt{} LABOR\PYGZus{}GROWTH = LABOR\PYGZus{}GROWTH + 0.002}\PYG{l+s+s1}{\PYGZsq{}}\PYG{p}{)}  \PYG{c+c1}{\PYGZsh{} create a new dataframe, increase LABOR\PYGZus{}GROWTH by 0.002}
\PYG{n}{scenario}   \PYG{o}{=} \PYG{n}{msolow}\PYG{p}{(}\PYG{n}{dfscenario}\PYG{p}{,}\PYG{n}{keep}\PYG{o}{=}\PYG{l+s+s1}{\PYGZsq{}}\PYG{l+s+s1}{Higher labor growth }\PYG{l+s+s1}{\PYGZsq{}}\PYG{p}{)} \PYG{c+c1}{\PYGZsh{} simulate the model }
\end{sphinxVerbatim}

\end{sphinxuseclass}\end{sphinxVerbatimInput}

\end{sphinxuseclass}

\section{Inspect results}
\label{\detokenize{content/03_Installation/TestingModelFlow:inspect-results}}
\sphinxAtStartPar
\sphinxcode{\sphinxupquote{Modelflow}} includes a range of methods to view data and results, either as graphs or as tables.  Some of these are part of standard python, others are additional features that \sphinxcode{\sphinxupquote{modelflow}} makes available.

\sphinxAtStartPar
Scenario results can be inspected either by referring to the scenario name given in the (optional) \sphinxcode{\sphinxupquote{keep}} statement when the model was solved, by referring to the \sphinxcode{\sphinxupquote{basedf}} and the \sphinxcode{\sphinxupquote{lastdf}}.
\begin{itemize}
\item {} 
\sphinxAtStartPar
\sphinxcode{\sphinxupquote{basedf}} is a dataframe that is automatically generated when the model is solved and contains a copy of the initial conditions of the model prior to the shock.

\item {} 
\sphinxAtStartPar
\sphinxcode{\sphinxupquote{lastdf}}is a dataframe that is automatically generated when the model is solved and contains a copy of the results from the simulation. Several built in display functions use these functions to display results.

\end{itemize}

\sphinxAtStartPar
Finally one could also look at the dataframe to which the results of the simulation were assigned \sphinxcode{\sphinxupquote{scenario}} in the example above.

\sphinxAtStartPar
Below is a small sub\sphinxhyphen{}set of the visualization options available.


\subsection{Graphical representations of results}
\label{\detokenize{content/03_Installation/TestingModelFlow:graphical-representations-of-results}}

\subsubsection{The .dif.plot() method}
\label{\detokenize{content/03_Installation/TestingModelFlow:the-dif-plot-method}}
\sphinxAtStartPar
The \sphinxcode{\sphinxupquote{.dif.plot}} method will plot the change in the level of requested variables.  Requested variables can be selected either directly by name or using wildcards.

\sphinxAtStartPar
In this example, a wild card specification is used, requesting the display of all variables that begin with the text ‘labor’.  Note that the selector is not case sensitive.

\sphinxAtStartPar
In this case we are displaying changes into the labor and labor growth variables due to the shock when we increased the growth rate of labor by .0002

\begin{sphinxuseclass}{cell}\begin{sphinxVerbatimInput}

\begin{sphinxuseclass}{cell_input}
\begin{sphinxVerbatim}[commandchars=\\\{\}]
\PYG{n}{msolow}\PYG{p}{[}\PYG{l+s+s1}{\PYGZsq{}}\PYG{l+s+s1}{labor*}\PYG{l+s+s1}{\PYGZsq{}}\PYG{p}{]}\PYG{o}{.}\PYG{n}{dif}\PYG{o}{.}\PYG{n}{plot}\PYG{p}{(}\PYG{p}{)} 
\end{sphinxVerbatim}

\end{sphinxuseclass}\end{sphinxVerbatimInput}
\begin{sphinxVerbatimOutput}

\begin{sphinxuseclass}{cell_output}
\noindent\sphinxincludegraphics{{28ae05d62fafe5fd35db4868226962fccd6269bec73310fece8437ae7f551422}.png}

\end{sphinxuseclass}\end{sphinxVerbatimOutput}

\end{sphinxuseclass}
\sphinxAtStartPar
In this example, instead of using a wild card selector we requested a variable explicitly by name.

\begin{sphinxuseclass}{cell}\begin{sphinxVerbatimInput}

\begin{sphinxuseclass}{cell_input}
\begin{sphinxVerbatim}[commandchars=\\\{\}]
\PYG{n}{msolow}\PYG{p}{[}\PYG{l+s+s1}{\PYGZsq{}}\PYG{l+s+s1}{GDP LABOR\PYGZus{}GROWTH}\PYG{l+s+s1}{\PYGZsq{}}\PYG{p}{]}\PYG{o}{.}\PYG{n}{pct}\PYG{o}{.}\PYG{n}{plot}\PYG{p}{(}\PYG{p}{)} 
\end{sphinxVerbatim}

\end{sphinxuseclass}\end{sphinxVerbatimInput}
\begin{sphinxVerbatimOutput}

\begin{sphinxuseclass}{cell_output}
\noindent\sphinxincludegraphics{{49f43ba057063db290c43f05c27a589984c46ab42fe9078b7976044d7baec37f}.png}

\end{sphinxuseclass}\end{sphinxVerbatimOutput}

\end{sphinxuseclass}

\subsubsection{Using the kept solutions}
\label{\detokenize{content/03_Installation/TestingModelFlow:using-the-kept-solutions}}
\sphinxAtStartPar
Because the keyword \sphinxcode{\sphinxupquote{keep}} was used when running the simulations, we can refer to the scenarios by their names – or produce graphs from multiple scenarios – not just the first and last.

\begin{sphinxuseclass}{cell}\begin{sphinxVerbatimInput}

\begin{sphinxuseclass}{cell_input}
\begin{sphinxVerbatim}[commandchars=\\\{\}]
\PYG{n}{msolow}\PYG{o}{.}\PYG{n}{keep\PYGZus{}plot}\PYG{p}{(}\PYG{l+s+s1}{\PYGZsq{}}\PYG{l+s+s1}{GDP}\PYG{l+s+s1}{\PYGZsq{}}\PYG{p}{)}\PYG{p}{;}
    
\end{sphinxVerbatim}

\end{sphinxuseclass}\end{sphinxVerbatimInput}
\begin{sphinxVerbatimOutput}

\begin{sphinxuseclass}{cell_output}
\noindent\sphinxincludegraphics{{d226200c19295f5ac6c2bcdb7b0d7f79658e204e0b397db9e49da9baa34f83ba}.png}

\end{sphinxuseclass}\end{sphinxVerbatimOutput}

\end{sphinxuseclass}

\subsection{Textual and tabular display of results}
\label{\detokenize{content/03_Installation/TestingModelFlow:textual-and-tabular-display-of-results}}
\sphinxAtStartPar
Standard pandas syntax can be used to display data in the results dataframes.

\sphinxAtStartPar
Here we use the standard pandas \sphinxcode{\sphinxupquote{.loc}} method to display every 10th data point for consumption from the results dataframe, beginning from observation 2000 through 2100.

\begin{sphinxuseclass}{cell}\begin{sphinxVerbatimInput}

\begin{sphinxuseclass}{cell_input}
\begin{sphinxVerbatim}[commandchars=\\\{\}]
\PYG{n}{msolow}\PYG{o}{.}\PYG{n}{lastdf}\PYG{o}{.}\PYG{n}{loc}\PYG{p}{[}\PYG{l+m+mi}{2000}\PYG{p}{:}\PYG{l+m+mi}{2100}\PYG{p}{:}\PYG{l+m+mi}{10}\PYG{p}{,}\PYG{l+s+s1}{\PYGZsq{}}\PYG{l+s+s1}{CONSUMPTION}\PYG{l+s+s1}{\PYGZsq{}}\PYG{p}{]}
\end{sphinxVerbatim}

\end{sphinxuseclass}\end{sphinxVerbatimInput}
\begin{sphinxVerbatimOutput}

\begin{sphinxuseclass}{cell_output}
\begin{sphinxVerbatim}[commandchars=\\\{\}]
2000      0.000000
2010    100.507238
2020    107.431233
2030    116.791578
2040    128.304359
2050    141.874040
2060    157.652568
2070    175.830726
2080    196.639578
2090    220.352686
2100    247.289125
Name: CONSUMPTION, dtype: float64
\end{sphinxVerbatim}

\end{sphinxuseclass}\end{sphinxVerbatimOutput}

\end{sphinxuseclass}

\subsubsection{The \sphinxstyleliteralintitle{\sphinxupquote{.dif.df}} method}
\label{\detokenize{content/03_Installation/TestingModelFlow:the-dif-df-method}}
\sphinxAtStartPar
The \sphinxcode{\sphinxupquote{.dif.df}} method prints out the changes in variables, i.e. eh difference between the level of specified  variables in the \sphinxcode{\sphinxupquote{lastdf}} dataframe vs the \sphinxcode{\sphinxupquote{basedf}} dataframe.

\begin{sphinxuseclass}{cell}\begin{sphinxVerbatimInput}

\begin{sphinxuseclass}{cell_input}
\begin{sphinxVerbatim}[commandchars=\\\{\}]
\PYG{n}{msolow}\PYG{p}{[}\PYG{l+s+s1}{\PYGZsq{}}\PYG{l+s+s1}{GDP CONSUMPTION}\PYG{l+s+s1}{\PYGZsq{}}\PYG{p}{]}\PYG{o}{.}\PYG{n}{dif}\PYG{o}{.}\PYG{n}{df}
\end{sphinxVerbatim}

\end{sphinxuseclass}\end{sphinxVerbatimInput}
\begin{sphinxVerbatimOutput}

\begin{sphinxuseclass}{cell_output}
\begin{sphinxVerbatim}[commandchars=\\\{\}]
             GDP  CONSUMPTION
2001    0.000000     0.000000
2002    0.000000     0.000000
2003    0.000000     0.000000
2004    0.000000     0.000000
2005    0.000000     0.000000
...          ...          ...
2295  665.334581   632.067852
2296  672.097592   638.492713
2297  678.925939   644.979642
2298  685.820324   651.529308
2299  692.781453   658.142380

[299 rows x 2 columns]
\end{sphinxVerbatim}

\end{sphinxuseclass}\end{sphinxVerbatimOutput}

\end{sphinxuseclass}

\subsubsection{The \sphinxstyleliteralintitle{\sphinxupquote{.difpct.df}} method}
\label{\detokenize{content/03_Installation/TestingModelFlow:the-difpct-df-method}}
\sphinxAtStartPar
The \sphinxcode{\sphinxupquote{.dif.pct.df}} method express the changes between the last simulation and base simulation results as a percent differences in the level (\({\Delta X_t \over X^{basedf}_{t-1}} \) ).  In the example below the mul100 method multiplies the result by 100.

\begin{sphinxuseclass}{cell}\begin{sphinxVerbatimInput}

\begin{sphinxuseclass}{cell_input}
\begin{sphinxVerbatim}[commandchars=\\\{\}]
\PYG{n}{msolow}\PYG{p}{[}\PYG{l+s+s1}{\PYGZsq{}}\PYG{l+s+s1}{GDP CONSUMPTION}\PYG{l+s+s1}{\PYGZsq{}}\PYG{p}{]}\PYG{o}{.}\PYG{n}{difpct}\PYG{o}{.}\PYG{n}{mul100}\PYG{o}{.}\PYG{n}{df}
\end{sphinxVerbatim}

\end{sphinxuseclass}\end{sphinxVerbatimInput}
\begin{sphinxVerbatimOutput}

\begin{sphinxuseclass}{cell_output}
\begin{sphinxVerbatim}[commandchars=\\\{\}]
           GDP  CONSUMPTION
2001       NaN          NaN
2002  0.000000     0.000000
2003  0.000000     0.000000
2004  0.000000     0.000000
2005  0.000000     0.000000
...        ...          ...
2295  0.005047     0.005047
2296  0.004892     0.004892
2297  0.004742     0.004742
2298  0.004596     0.004596
2299  0.004456     0.004456

[299 rows x 2 columns]
\end{sphinxVerbatim}

\end{sphinxuseclass}\end{sphinxVerbatimOutput}

\end{sphinxuseclass}

\subsection{Interactive display of impacts}
\label{\detokenize{content/03_Installation/TestingModelFlow:interactive-display-of-impacts}}
\sphinxAtStartPar
When working within Jupyter notebook the dif command will produce (without the .df termination) will generate a widget with the results expressed as level differences, percent differences, differences in the growth rate – both graphically and in table form.

\sphinxAtStartPar
Please consult \DUrole{xref,myst}{here} for a fuller presentation of the display routines built into \sphinxcode{\sphinxupquote{modelflfow}}.

\begin{sphinxuseclass}{cell}\begin{sphinxVerbatimInput}

\begin{sphinxuseclass}{cell_input}
\begin{sphinxVerbatim}[commandchars=\\\{\}]
\PYG{n}{msolow}\PYG{p}{[}\PYG{l+s+s1}{\PYGZsq{}}\PYG{l+s+s1}{GDP CONSUMPTION}\PYG{l+s+s1}{\PYGZsq{}}\PYG{p}{]}
\end{sphinxVerbatim}

\end{sphinxuseclass}\end{sphinxVerbatimInput}
\begin{sphinxVerbatimOutput}

\begin{sphinxuseclass}{cell_output}
\begin{sphinxVerbatim}[commandchars=\\\{\}]
Tab(children=(Tab(children=(HTML(value=\PYGZsq{}\PYGZlt{}?xml version=\PYGZdq{}1.0\PYGZdq{} encoding=\PYGZdq{}utf\PYGZhy{}8\PYGZdq{} standalone=\PYGZdq{}no\PYGZdq{}?\PYGZgt{}\PYGZbs{}n\PYGZlt{}!DOCTYPE svg …
\end{sphinxVerbatim}

\begin{sphinxVerbatim}[commandchars=\\\{\}]

\end{sphinxVerbatim}

\end{sphinxuseclass}\end{sphinxVerbatimOutput}

\end{sphinxuseclass}
\begin{sphinxuseclass}{cell}\begin{sphinxVerbatimInput}

\begin{sphinxuseclass}{cell_input}
\begin{sphinxVerbatim}[commandchars=\\\{\}]
\PYG{n}{msolow}\PYG{o}{.}\PYG{n}{capital}\PYG{o}{.}\PYG{n}{dash}
\end{sphinxVerbatim}

\end{sphinxuseclass}\end{sphinxVerbatimInput}
\begin{sphinxVerbatimOutput}

\begin{sphinxuseclass}{cell_output}
\begin{sphinxVerbatim}[commandchars=\\\{\}]
apprun
Dash is running on http://127.0.0.1:5001/
\end{sphinxVerbatim}

\begin{sphinxVerbatim}[commandchars=\\\{\}]
No Dash, Address \PYGZsq{}http://127.0.0.1:5001\PYGZsq{} already in use.
    Try passing a different port to run\PYGZus{}server.
\end{sphinxVerbatim}

\end{sphinxuseclass}\end{sphinxVerbatimOutput}

\end{sphinxuseclass}
\begin{sphinxuseclass}{cell}\begin{sphinxVerbatimInput}

\begin{sphinxuseclass}{cell_input}
\begin{sphinxVerbatim}[commandchars=\\\{\}]
\PYG{n}{msolow}\PYG{o}{.}\PYG{n}{gdp}\PYG{o}{.}\PYG{n}{dashport}\PYG{p}{(}\PYG{p}{)}
\end{sphinxVerbatim}

\end{sphinxuseclass}\end{sphinxVerbatimInput}
\begin{sphinxVerbatimOutput}

\begin{sphinxuseclass}{cell_output}
\begin{sphinxVerbatim}[commandchars=\\\{\}]
apprun
Dash is running on http://127.0.0.1:5002/
\end{sphinxVerbatim}

\begin{sphinxVerbatim}[commandchars=\\\{\}]
Dash app running on http://127.0.0.1:5002/
\end{sphinxVerbatim}

\end{sphinxuseclass}\end{sphinxVerbatimOutput}

\end{sphinxuseclass}
\sphinxstepscope


\part{Some python essentials for using WorldBank models with modelflow}

\sphinxstepscope


\chapter{Introduction to  Jupyter Notebook}
\label{\detokenize{content/04_PythonEssentials/Intro_Jupyter_notebook:introduction-to-jupyter-notebook}}\label{\detokenize{content/04_PythonEssentials/Intro_Jupyter_notebook::doc}}
\sphinxAtStartPar
Jupyter Notebook is a web application for creating, annotating, simulating and working with computational documents.  Originally developed for python, the latest versions of EViews also support Jupyter Notebooks. Jupyter Notebook offers a simple, streamlined, document\sphinxhyphen{}centric experience and can be a great environment for documenting the work you are doing, and trying alternative methods of achieving desirable results.  Many of the methods in \sphinxcode{\sphinxupquote{modelflow}} have been developed to work well with Jupyter Notebook. Indeed this documentation was written as a series of Jupyter Notebooks bound together with Jupyter Book.

\sphinxAtStartPar
Jupyter Notebook is not the only way to work with modelflow or Python.  As users become more advanced they are likely to migrate to a more program\sphinxhyphen{}centric IDE (Interactive Development Environment) like Spyder or Microsoft Visual Code.

\sphinxAtStartPar
However, to start Jupyter Notebooks are a great way to learn, follow work done by others and tweak them to fit your own needs.

\sphinxAtStartPar
There are many fine tutorials on Jupyter Notebook on the web, and \sphinxhref{https://docs.jupyter.org/en/latest/}{The official Jupyter site} is a good starting point. The following aims to provide enough information to get a user started.  Another good reference is \sphinxhref{https://jupyter.brynmawr.edu/services/public/dblank/Jupyter\%20Notebook\%20Users\%20Manual.ipynb}{here.}

\begin{sphinxadmonition}{warning}{Warning:}
\sphinxAtStartPar
*** Ib new book replaced by new cell second time belowq
\end{sphinxadmonition}


\section{Starting Jupyter Notebook}
\label{\detokenize{content/04_PythonEssentials/Intro_Jupyter_notebook:starting-jupyter-notebook}}
\sphinxAtStartPar
Each time, a user wants to work with \sphinxcode{\sphinxupquote{modelflow}}, they will need to activate the \sphinxcode{\sphinxupquote{modelflow}} environment by
\begin{enumerate}
\sphinxsetlistlabels{\arabic}{enumi}{enumii}{}{)}%
\item {} 
\sphinxAtStartPar
Opening the Anaconda command prompt window

\item {} 
\sphinxAtStartPar
Activate the ModelFlow environment we just created by executing the following command

\end{enumerate}

\sphinxAtStartPar
\sphinxcode{\sphinxupquote{conda activate modelflow}}

\sphinxAtStartPar
From here, any number of mechanisms can be used to interact with \sphinxcode{\sphinxupquote{modelflow}} and World Bank models.

\sphinxAtStartPar
\sphinxstylestrong{To use Jupyter Notebook} the Jupyter notebook, must be first started.  Following steps 1\sphinxhyphen{}2 above, a user would need to execute from the conda command line:

\sphinxAtStartPar
\sphinxcode{\sphinxupquote{jupyter notebook}}

\sphinxAtStartPar
This will launch the Jupyter environment in your default web browser, which should look something like this:

\sphinxAtStartPar
\sphinxincludegraphics{{NewJNSession}.png}

\sphinxAtStartPar
where the directory structure presented is that of the directory from the \sphinxcode{\sphinxupquote{jupyter notebook}} command was executed.

\begin{sphinxadmonition}{warning}{Warning:}
\sphinxAtStartPar
Note the directory from which you execute the \sphinxcode{\sphinxupquote{jupyter notebook}} \sphinxstylestrong{mfbook} in the example above will be the \sphinxstylestrong{root} directory for the jupyter session, and only directories and files below this root directory will be accessible by jupyter.
\end{sphinxadmonition}


\section{Creating a notebook}
\label{\detokenize{content/04_PythonEssentials/Intro_Jupyter_notebook:creating-a-notebook}}
\sphinxAtStartPar
The idea behind jupyter notebook was to create an interactive version of the notebooks that scientists use(d) to:
\begin{itemize}
\item {} 
\sphinxAtStartPar
record what they have done

\item {} 
\sphinxAtStartPar
perhaps explain why

\item {} 
\sphinxAtStartPar
document how data was generated, and

\item {} 
\sphinxAtStartPar
record the results of their experiments

\end{itemize}

\sphinxAtStartPar
The motivation for these notebooks and Jupyter notebook is to record the precise steps taken to produce a set of results, which if followed by others would allow the to generate the same results.

\sphinxAtStartPar
To create a notebook you must select from the Jupyter Notebook menu

\sphinxAtStartPar
File\sphinxhyphen{}> New Notebook

\begin{figure}[htbp]
\centering
\capstart

\noindent\sphinxincludegraphics[height=150\sphinxpxdimen]{{NewNotebook}.png}
\caption{A newly created Jupyter Notebook session}\label{\detokenize{content/04_PythonEssentials/Intro_Jupyter_notebook:new-notebook}}\end{figure}

\sphinxAtStartPar
This will generate a blank unnamed notebook with one empty cell, that looks something like this:

\sphinxAtStartPar
\sphinxcode{\sphinxupquote{!{[}NewCell{]}(./Newcell.png)}}

\begin{figure}[htbp]
\centering
\capstart

\noindent\sphinxincludegraphics[height=150\sphinxpxdimen]{{Newcell}.png}
\caption{A newly created Jupyter Notebook}\label{\detokenize{content/04_PythonEssentials/Intro_Jupyter_notebook:new-cell}}\end{figure}

\begin{sphinxadmonition}{warning}{Warning:}
\sphinxAtStartPar
Each notebook has associated with it a “Kernel”, which is an instance of the computing environment in which code will be executed. For Jupyter Notebooks that work with \sphinxcode{\sphinxupquote{modelflow}} this will be a Python Kernel. If your computer has more than one “kernel’s” installed on it, you may be prompted when creating a new notebook for the kernel with which to associate it.  Typically this should be the Python Kernel under which your modelflow was built – currently python 3.9 in April 2023.
\end{sphinxadmonition}


\section{Jupter Notebook cells}
\label{\detokenize{content/04_PythonEssentials/Intro_Jupyter_notebook:jupter-notebook-cells}}
\sphinxAtStartPar
A Jupyter Notebook is comprised of a series of cells.

\sphinxAtStartPar
\sphinxstyleemphasis{\sphinxstylestrong{Jupyter Notebook cells can contain:}}
\begin{itemize}
\item {} 
\sphinxAtStartPar
\sphinxstylestrong{computer code} (typically python code, but as noted other kernels – like Eviews – can be used with jupyter).

\item {} 
\sphinxAtStartPar
\sphinxstylestrong{markdown text}: plain text that can include special characters that make some text appear as bold, or indicate the text is a header, or instruct Jupyter Notebook to render the text as a mathematical formula.  All of the text in this document was entered using Jupyter Notebook’s markdown language

\item {} 
\sphinxAtStartPar
Results (in the form of tables or graphs) from the execution of computer code specified in a code cell

\end{itemize}

\sphinxAtStartPar
\sphinxstylestrong{Every cell has two modes:}
\begin{enumerate}
\sphinxsetlistlabels{\arabic}{enumi}{enumii}{}{.}%
\item {} 
\sphinxAtStartPar
Edit mode – indicated by a green vertical bar. In edit mode the user can change the code, or the markdown.

\item {} 
\sphinxAtStartPar
Select/Copy mode – indicated by a blue vertical bar.  This will be teh state of the cell when its content has been executed.  For markdown cells this means that the text and special characters have been rendered into formatted text.  For code cells, this means the code has been executed and its output (if any) displayed in an output cell.

\end{enumerate}

\sphinxAtStartPar
\sphinxstylestrong{Users can switch between Edit and Select/Copy Mode by hitting Enter}

\sphinxAtStartPar
This entire book was generated using markdown cells, code cells and output cells from Jupyter Notebooks.

\begin{sphinxadmonition}{note}{Note:}
\sphinxAtStartPar
Jupyter Notebooks were designed to facilitate \sphinxstyleemphasis{replicability}: the idea that a scientific analysis should contain \sphinxhyphen{} in addition to the final output (text, graphs, tables) \sphinxhyphen{} all the computational steps needed to get from raw input data to the results.
\end{sphinxadmonition}


\subsection{How to add, delete and move cells}
\label{\detokenize{content/04_PythonEssentials/Intro_Jupyter_notebook:how-to-add-delete-and-move-cells}}
\sphinxAtStartPar
The newly created Jupyter Notebook will have a code cell by default.  Cells can be added, deleted and moved either via mouse using the toolbar or by keyboard shortcut.

\sphinxAtStartPar
\sphinxstylestrong{Using the Toolbar}
\begin{itemize}
\item {} 
\sphinxAtStartPar
\sphinxstylestrong{+ button}: add a cell below the current cell

\item {} 
\sphinxAtStartPar
\sphinxstylestrong{scissors}: cut  current cell (can be undone from “Edit” tab)

\item {} 
\sphinxAtStartPar
\sphinxstylestrong{clipboard}: paste a previously cut cell to the current location

\item {} 
\sphinxAtStartPar
\sphinxstylestrong{up\sphinxhyphen{} and down arrows}: move cells (cell must be in Select/Copy mode – vertical side bar must be blue)

\item {} 
\sphinxAtStartPar
\sphinxstylestrong{hold shift + click cells in left margin}: select multiple cells (vertical bar must be blue)

\end{itemize}

\sphinxAtStartPar
\sphinxstylestrong{Using keyboard short cuts}
\begin{itemize}
\item {} 
\sphinxAtStartPar
\sphinxstylestrong{esc + a}: add a cell above the current cell

\item {} 
\sphinxAtStartPar
\sphinxstylestrong{esc + b}: add a cell below the current cell

\item {} 
\sphinxAtStartPar
\sphinxstylestrong{esc + d+d}: delete the current cell

\end{itemize}


\subsection{Change the type of a cell}
\label{\detokenize{content/04_PythonEssentials/Intro_Jupyter_notebook:change-the-type-of-a-cell}}
\sphinxAtStartPar
You can also change the type of a cell. New cells are by default “code” cells.

\sphinxAtStartPar
\sphinxstylestrong{Using the Toolbar}
\begin{itemize}
\item {} 
\sphinxAtStartPar
Select the desired type from the drop down.  options include
\begin{itemize}
\item {} 
\sphinxAtStartPar
Markdown

\item {} 
\sphinxAtStartPar
Code

\item {} 
\sphinxAtStartPar
Raw NBConvert

\item {} 
\sphinxAtStartPar
Heading

\end{itemize}

\end{itemize}

\sphinxAtStartPar
\sphinxstylestrong{Using keyboard short cuts}
\begin{itemize}
\item {} 
\sphinxAtStartPar
\sphinxstylestrong{esc + m}: make the current cell a markdown cell

\item {} 
\sphinxAtStartPar
\sphinxstylestrong{esc + y}: make the current cell a code  cell

\end{itemize}

\sphinxAtStartPar
\sphinxstylestrong{Auto\sphinxhyphen{}complete and context\sphinxhyphen{}sensitive help}

\sphinxAtStartPar
When editing a code cell, you can use these short\sphinxhyphen{}cuts to autocomplete and or call up documentation for a command.
\begin{itemize}
\item {} 
\sphinxAtStartPar
\sphinxstylestrong{tab}: autocomplete and  method selection

\item {} 
\sphinxAtStartPar
\sphinxstylestrong{double tab}: documention (double tab for full doc)

\end{itemize}


\section{Execution of cells}
\label{\detokenize{content/04_PythonEssentials/Intro_Jupyter_notebook:execution-of-cells}}
\sphinxAtStartPar
Every cell in a Jupyter Notebook can be executed, either by using the Run button on the Jupyter Notebook menu, or by using one of \sphinxstylestrong{two keyboard shortcuts}:
\begin{itemize}
\item {} 
\sphinxAtStartPar
\sphinxstylestrong{ctrl + Enter}: Executes the code in the cell or formats the markdown of a cell.  The current cell retains the focus – cursor stays on cell executed.

\item {} 
\sphinxAtStartPar
\sphinxstylestrong{shift + enter}: Executes the code in the cell or formats the markdown of a cell. Focus (cursor) jumps to the next cell

\end{itemize}

\sphinxAtStartPar
For other useful shortcuts see “Help” => “Keyboard Shortcuts” or simply press keyboard icon in the toolbar.


\subsection{Executing python code}
\label{\detokenize{content/04_PythonEssentials/Intro_Jupyter_notebook:executing-python-code}}
\sphinxAtStartPar
Below is a code with some standard python that declares a variable “x”, assigns it the value 10, declares a second variable “y” and assigns it the value 45.  The final line of y alone, instructs python to display the value of the variable y.  The results of the operation appear in Jupyter Notebook as an output cell Out{[}\#{]}.  By pressing \sphinxstylestrong{Ctrl\sphinxhyphen{}Enter} the code will be executed and the output displayed below.

\begin{sphinxuseclass}{cell}\begin{sphinxVerbatimInput}

\begin{sphinxuseclass}{cell_input}
\begin{sphinxVerbatim}[commandchars=\\\{\}]
\PYG{n}{x} \PYG{o}{=} \PYG{l+m+mi}{10}
\PYG{n}{y} \PYG{o}{=} \PYG{l+m+mi}{45}
\PYG{n}{y}
\end{sphinxVerbatim}

\end{sphinxuseclass}\end{sphinxVerbatimInput}
\begin{sphinxVerbatimOutput}

\begin{sphinxuseclass}{cell_output}
\begin{sphinxVerbatim}[commandchars=\\\{\}]
45
\end{sphinxVerbatim}

\end{sphinxuseclass}\end{sphinxVerbatimOutput}

\end{sphinxuseclass}
\sphinxAtStartPar
\sphinxstylestrong{The semi\sphinxhyphen{}colon “;” suppresses output in Jupyter Notebook}

\sphinxAtStartPar
In the example below, a semi\sphinxhyphen{}colon “;” has been appended to the final line.  This suppresses the display of the value contained by y;  As a result there is no output cell.

\begin{sphinxuseclass}{cell}\begin{sphinxVerbatimInput}

\begin{sphinxuseclass}{cell_input}
\begin{sphinxVerbatim}[commandchars=\\\{\}]
\PYG{n}{x} \PYG{o}{=} \PYG{l+m+mi}{10}
\PYG{n}{y} \PYG{o}{=} \PYG{l+m+mi}{45}
\PYG{n}{y}\PYG{p}{;}
\end{sphinxVerbatim}

\end{sphinxuseclass}\end{sphinxVerbatimInput}

\end{sphinxuseclass}
\sphinxAtStartPar
Another way to display results is to use the print function.

\begin{sphinxuseclass}{cell}\begin{sphinxVerbatimInput}

\begin{sphinxuseclass}{cell_input}
\begin{sphinxVerbatim}[commandchars=\\\{\}]
\PYG{n}{x} \PYG{o}{=} \PYG{l+m+mi}{10}
\PYG{n+nb}{print}\PYG{p}{(}\PYG{n}{x}\PYG{p}{)}
\end{sphinxVerbatim}

\end{sphinxuseclass}\end{sphinxVerbatimInput}
\begin{sphinxVerbatimOutput}

\begin{sphinxuseclass}{cell_output}
\begin{sphinxVerbatim}[commandchars=\\\{\}]
10
\end{sphinxVerbatim}

\end{sphinxuseclass}\end{sphinxVerbatimOutput}

\end{sphinxuseclass}
\sphinxAtStartPar
Variables in a Jupyter Notebook session are persistent, as a result in the subsequent cell, we can declare a variable ‘z’ equal to 2*y and it will have the value 90.

\begin{sphinxuseclass}{cell}\begin{sphinxVerbatimInput}

\begin{sphinxuseclass}{cell_input}
\begin{sphinxVerbatim}[commandchars=\\\{\}]
\PYG{n}{z}\PYG{o}{=}\PYG{n}{y}\PYG{o}{*}\PYG{l+m+mi}{2}
\PYG{n}{z}
\end{sphinxVerbatim}

\end{sphinxuseclass}\end{sphinxVerbatimInput}
\begin{sphinxVerbatimOutput}

\begin{sphinxuseclass}{cell_output}
\begin{sphinxVerbatim}[commandchars=\\\{\}]
90
\end{sphinxVerbatim}

\end{sphinxuseclass}\end{sphinxVerbatimOutput}

\end{sphinxuseclass}

\section{Markdown cells and the markdown scripting language in Jupyter Notebook}
\label{\detokenize{content/04_PythonEssentials/Intro_Jupyter_notebook:markdown-cells-and-the-markdown-scripting-language-in-jupyter-notebook}}
\sphinxAtStartPar
Text cells in a notebook can be made more interesting by using markdown.

\sphinxAtStartPar
Cells designated as markdown cells when executed are rendered in a rich text format (html).

\sphinxAtStartPar
Markdown is a lightweight markup language for creating formatted text using a plain\sphinxhyphen{}text editor.  Used in a markdown cell of Jupyter Notebook it can be used to produce nicely formatted text that mixes text, mathematical formulae, code and outputs from executed python code.

\sphinxAtStartPar
Rather than the relatively complex commands of html <h1></h1>, markdown uses a simplified set of commands to control how text elements should be rendered.


\subsection{Common markdown commands}
\label{\detokenize{content/04_PythonEssentials/Intro_Jupyter_notebook:common-markdown-commands}}
\sphinxAtStartPar
Some of the most common of these include:


\begin{savenotes}\sphinxattablestart
\centering
\begin{tabulary}{\linewidth}[t]{|T|T|}
\hline
\sphinxstyletheadfamily 
\sphinxAtStartPar
symbol
&\sphinxstyletheadfamily 
\sphinxAtStartPar
Effect
\\
\hline
\sphinxAtStartPar
\#
&
\sphinxAtStartPar
Header
\\
\hline
\sphinxAtStartPar
\#\#
&
\sphinxAtStartPar
second level
\\
\hline
\sphinxAtStartPar
\#\#\#
&
\sphinxAtStartPar
third level etc.
\\
\hline
\sphinxAtStartPar
**Bold text**
&
\sphinxAtStartPar
\sphinxstylestrong{Bold text}
\\
\hline
\sphinxAtStartPar
*Italics text*
&
\sphinxAtStartPar
\sphinxstyleemphasis{Italics text}
\\
\hline
\sphinxAtStartPar
* text
&
\sphinxAtStartPar
Bulleted text or dot notes
\\
\hline
\sphinxAtStartPar
1. text
&
\sphinxAtStartPar
1. Numbered bullets
\\
\hline
\end{tabulary}
\par
\sphinxattableend\end{savenotes}


\subsection{Tables in markdown}
\label{\detokenize{content/04_PythonEssentials/Intro_Jupyter_notebook:tables-in-markdown}}
\sphinxAtStartPar
Tables like the one above can be constructed using | as separators.

\sphinxAtStartPar
The |:–|:—————–| on the second line tells the Table generator how to justifythe cintents of columns.  :– means left justify :–: means center justify and –: means right justify.

\sphinxAtStartPar
Below is the markdown code that generated the above table:

\begin{sphinxVerbatim}[commandchars=\\\{\}]
\PYG{o}{|} \PYG{n}{symbol}           \PYG{o}{|} \PYG{n}{Effect}          \PYG{o}{|}
\PYG{o}{|}\PYG{p}{:}\PYG{o}{\PYGZhy{}}\PYG{o}{\PYGZhy{}}\PYG{o}{|}\PYG{p}{:}\PYG{o}{\PYGZhy{}}\PYG{o}{\PYGZhy{}}\PYG{o}{\PYGZhy{}}\PYG{o}{\PYGZhy{}}\PYG{o}{\PYGZhy{}}\PYG{o}{\PYGZhy{}}\PYG{o}{|}                                \PYG{c+c1}{\PYGZsh{} Specifies the justification for the columns of the table.}
\PYG{o}{|} \PYGZbs{}\PYG{c+c1}{\PYGZsh{}               | Header        |}
\PYG{o}{|} \PYGZbs{}\PYG{c+c1}{\PYGZsh{}\PYGZbs{}\PYGZsh{}             | second level |}
\PYG{o}{|} \PYGZbs{}\PYG{o}{*}\PYGZbs{}\PYG{o}{*}\PYG{n}{Bold} \PYG{n}{text}\PYGZbs{}\PYG{o}{*}\PYGZbs{}\PYG{o}{*} \PYG{o}{|} \PYG{o}{*}\PYG{o}{*}\PYG{n}{Bold} \PYG{n}{text}\PYG{o}{*}\PYG{o}{*}   \PYG{o}{|}
\PYG{o}{|} \PYGZbs{}\PYG{o}{*}\PYG{n}{Italics} \PYG{n}{text}\PYGZbs{}\PYG{o}{*} \PYG{o}{|} \PYG{o}{*}\PYG{n}{Italics} \PYG{n}{text}\PYG{o}{*}   \PYG{o}{|}
\PYG{o}{|} 
\PYG{o}{|} \PYG{l+m+mi}{1}\PYGZbs{}\PYG{o}{.} \PYG{n}{text}  \PYG{o}{|} \PYG{l+m+mf}{1.} \PYG{n}{Numbered} \PYG{n}{bullets}   \PYG{o}{|}

\end{sphinxVerbatim}


\subsection{Displaying code}
\label{\detokenize{content/04_PythonEssentials/Intro_Jupyter_notebook:displaying-code}}
\sphinxAtStartPar
To display a (unexecutable)  block of code within a markdown cell, encapsulate it (surround it) with backticks `.

\sphinxAtStartPar
For a multiline section of code use three backticks at the beginning and end.

\sphinxAtStartPar
```
Multi line
text to be rendered as code
```.

\sphinxAtStartPar
will be rendered as: \sphinxcode{\sphinxupquote{text to be rendered as code}}.

\begin{sphinxVerbatim}[commandchars=\\\{\}]

\PYG{n}{Multi} \PYG{n}{line} 
\PYG{n}{text} \PYG{n}{to} \PYG{n}{be} \PYG{n}{rendered} \PYG{k}{as} \PYG{n}{code} 
\end{sphinxVerbatim}

\sphinxAtStartPar
For inline code references ‘ a single back tick at the beginning and end suffices.

\sphinxAtStartPar
\sphinxstylestrong{This sentence:}

\sphinxAtStartPar
An example sentence with some back\sphinxhyphen{}ticked `text as code` in the middle.

\sphinxAtStartPar
\sphinxstylestrong{will render as:}

\sphinxAtStartPar
An example sentence with some back\sphinxhyphen{}ticked \sphinxcode{\sphinxupquote{text as code}} in the middle.

\begin{sphinxadmonition}{warning}{Warning:}
\sphinxAtStartPar
*** Ib functiuon names corrected below
\end{sphinxadmonition}


\subsection{Rendering mathematics in markdown}
\label{\detokenize{content/04_PythonEssentials/Intro_Jupyter_notebook:rendering-mathematics-in-markdown}}
\sphinxAtStartPar
Jupyter Notebook’s implementation of Markdown supports \sphinxcode{\sphinxupquote{latex}} mathematical notation.

\sphinxAtStartPar
Inline enclose the \sphinxcode{\sphinxupquote{latex}} code in \sphinxcode{\sphinxupquote{\$}}:

\sphinxAtStartPar
An Equation: \sphinxcode{\sphinxupquote{\$y\_t = \textbackslash{}beta\_0 + \textbackslash{}beta\_1 x\_t + u\_t\textbackslash{}\$}} will renders as: \(y_t = \beta_0 + \beta_1 x_t + u_t\)

\sphinxAtStartPar
if enclosed in \sphinxcode{\sphinxupquote{\$\$}} \sphinxcode{\sphinxupquote{\$\$}} it will be centered on its own line.
\begin{equation*}
\begin{split}y_t = \beta_0 + \beta_1 x_t + u_t\end{split}
\end{equation*}

\subsubsection{Complex and multi\sphinxhyphen{}line math}
\label{\detokenize{content/04_PythonEssentials/Intro_Jupyter_notebook:complex-and-multi-line-math}}
\begin{sphinxVerbatim}[commandchars=\\\{\}]
\PYGZbs{}\PYG{n}{begin}\PYG{p}{\PYGZob{}}\PYG{n}{align}\PYG{p}{\PYGZcb{}}
\PYG{n}{Y\PYGZus{}t}  \PYG{o}{\PYGZam{}}\PYG{o}{=}  \PYG{n}{C\PYGZus{}t}\PYG{o}{+}\PYG{n}{I\PYGZus{}t}\PYG{o}{+}\PYG{n}{G}\PYG{o}{+}\PYG{n}{t}\PYG{o}{+} \PYG{p}{(}\PYG{n}{X\PYGZus{}t}\PYG{o}{\PYGZhy{}}\PYG{n}{M\PYGZus{}t}\PYG{p}{)} \PYGZbs{}\PYGZbs{}
\PYG{n}{C\PYGZus{}t} \PYG{o}{\PYGZam{}}\PYG{o}{=} \PYG{n}{y\PYGZus{}t}\PYG{p}{(}\PYG{n}{C\PYGZus{}}\PYG{p}{\PYGZob{}}\PYG{n}{t}\PYG{o}{\PYGZhy{}}\PYG{l+m+mi}{1}\PYG{p}{\PYGZcb{}}\PYG{p}{,}\PYG{n}{C\PYGZus{}}\PYG{p}{\PYGZob{}}\PYG{n}{t}\PYG{o}{\PYGZhy{}}\PYG{l+m+mi}{2}\PYG{p}{\PYGZcb{}}\PYG{p}{,}\PYG{n}{I\PYGZus{}t}\PYG{p}{,}\PYG{n}{G\PYGZus{}t}\PYG{p}{,}\PYG{n}{X\PYGZus{}t}\PYG{p}{,}\PYG{n}{M\PYGZus{}t}\PYG{p}{,}\PYG{n}{P\PYGZus{}t}\PYG{p}{)}\PYGZbs{}\PYGZbs{}
\PYG{n}{I\PYGZus{}t} \PYG{o}{\PYGZam{}}\PYG{o}{=} \PYG{n}{i\PYGZus{}t}\PYG{p}{(}\PYG{n}{I\PYGZus{}}\PYG{p}{\PYGZob{}}\PYG{n}{t}\PYG{o}{\PYGZhy{}}\PYG{l+m+mi}{1}\PYG{p}{\PYGZcb{}}\PYG{p}{,}\PYG{n}{I\PYGZus{}}\PYG{p}{\PYGZob{}}\PYG{n}{t}\PYG{o}{\PYGZhy{}}\PYG{l+m+mi}{2}\PYG{p}{\PYGZcb{}}\PYG{p}{,}\PYG{n}{C\PYGZus{}t}\PYG{p}{,}\PYG{n}{G\PYGZus{}t}\PYG{p}{,}\PYG{n}{X\PYGZus{}t}\PYG{p}{,}\PYG{n}{M\PYGZus{}t}\PYG{p}{,}\PYG{n}{P\PYGZus{}t}\PYG{p}{)}\PYGZbs{}\PYGZbs{}
\PYG{n}{G\PYGZus{}t} \PYG{o}{\PYGZam{}}\PYG{o}{=} \PYG{n}{g\PYGZus{}t}\PYG{p}{(}\PYG{n}{G\PYGZus{}}\PYG{p}{\PYGZob{}}\PYG{n}{t}\PYG{o}{\PYGZhy{}}\PYG{l+m+mi}{1}\PYG{p}{\PYGZcb{}}\PYG{p}{,}\PYG{n}{G\PYGZus{}}\PYG{p}{\PYGZob{}}\PYG{n}{t}\PYG{o}{\PYGZhy{}}\PYG{l+m+mi}{2}\PYG{p}{\PYGZcb{}}\PYG{p}{,}\PYG{n}{C\PYGZus{}t}\PYG{p}{,}\PYG{n}{I\PYGZus{}t}\PYG{p}{,}\PYG{n}{X\PYGZus{}t}\PYG{p}{,}\PYG{n}{M\PYGZus{}t}\PYG{p}{,}\PYG{n}{P\PYGZus{}t}\PYG{p}{)}\PYGZbs{}\PYGZbs{}
\PYG{n}{X\PYGZus{}t} \PYG{o}{\PYGZam{}}\PYG{o}{=} \PYG{n}{x\PYGZus{}t}\PYG{p}{(}\PYG{n}{X\PYGZus{}}\PYG{p}{\PYGZob{}}\PYG{n}{t}\PYG{o}{\PYGZhy{}}\PYG{l+m+mi}{1}\PYG{p}{\PYGZcb{}}\PYG{p}{,}\PYG{n}{X\PYGZus{}}\PYG{p}{\PYGZob{}}\PYG{n}{t}\PYG{o}{\PYGZhy{}}\PYG{l+m+mi}{2}\PYG{p}{\PYGZcb{}}\PYG{p}{,}\PYG{n}{C\PYGZus{}t}\PYG{p}{,}\PYG{n}{I\PYGZus{}t}\PYG{p}{,}\PYG{n}{G\PYGZus{}t}\PYG{p}{,}\PYG{n}{M\PYGZus{}t}\PYG{p}{,}\PYG{n}{P\PYGZus{}t}\PYG{p}{,}\PYG{n}{P}\PYG{o}{\PYGZca{}}\PYG{n}{f\PYGZus{}t}\PYG{p}{)}\PYGZbs{}\PYGZbs{}
\PYG{n}{M\PYGZus{}t} \PYG{o}{\PYGZam{}}\PYG{o}{=} \PYG{n}{m\PYGZus{}t}\PYG{p}{(}\PYG{n}{M\PYGZus{}}\PYG{p}{\PYGZob{}}\PYG{n}{t}\PYG{o}{\PYGZhy{}}\PYG{l+m+mi}{1}\PYG{p}{\PYGZcb{}}\PYG{p}{,}\PYG{n}{M\PYGZus{}}\PYG{p}{\PYGZob{}}\PYG{n}{t}\PYG{o}{\PYGZhy{}}\PYG{l+m+mi}{2}\PYG{p}{\PYGZcb{}}\PYG{p}{,}\PYG{n}{C\PYGZus{}t}\PYG{p}{,}\PYG{n}{I\PYGZus{}t}\PYG{p}{,}\PYG{n}{G\PYGZus{}t}\PYG{p}{,}\PYG{n}{X\PYGZus{}t}\PYG{p}{,}\PYG{n}{P\PYGZus{}t}\PYG{p}{,}\PYG{n}{P}\PYG{o}{\PYGZca{}}\PYG{n}{f\PYGZus{}t}\PYG{p}{)}
\PYGZbs{}\PYG{n}{end}\PYG{p}{\PYGZob{}}\PYG{n}{align}\PYG{p}{\PYGZcb{}}
\end{sphinxVerbatim}

\sphinxAtStartPar
The above \sphinxcode{\sphinxupquote{latex}} mathematics code uses the  \sphinxcode{\sphinxupquote{\&}} symbol to tell \sphinxcode{\sphinxupquote{latex}} to align the different lines (separated by \sphinxcode{\sphinxupquote{\textbackslash{}\textbackslash{}}}) on the character immediately after the \sphinxcode{\sphinxupquote{\&}}. In this instance the equals “=” sign.
\label{equation:content/04_PythonEssentials/Intro_Jupyter_notebook:aec49902-cb53-4c63-b689-989a68602faf}\begin{align}
Y_t  &=  C_t+I_t+G+t+ (X_t-M_t) \\
C_t &= c_t(C_{t-1},C_{t-2},I_t,G_t,X_t,M_t,P_t)\\
I_t &= i_t(I_{t-1},I_{t-2},C_t,G_t,X_t,M_t,P_t)\\
G_t &= g_t(G_{t-1},G_{t-2},C_t,I_t,X_t,M_t,P_t)\\
X_t &= x_t(X_{t-1},X_{t-2},C_t,I_t,G_t,M_t,P_t,P^f_t)\\
M_t &= m_t(M_{t-1},M_{t-2},C_t,I_t,G_t,X_t,P_t,P^f_t)
\end{align}

\subsection{links to more info on markdown}
\label{\detokenize{content/04_PythonEssentials/Intro_Jupyter_notebook:links-to-more-info-on-markdown}}
\sphinxAtStartPar
There are several very good markdown cheatsheets on the internet, one of these is \sphinxhref{https://www.markdownguide.org/cheat-sheet/}{here}

\sphinxstepscope


\chapter{Some Python basics}
\label{\detokenize{content/04_PythonEssentials/PythonPandasDataframes:some-python-basics}}\label{\detokenize{content/04_PythonEssentials/PythonPandasDataframes::doc}}
\sphinxAtStartPar
Before using \sphinxcode{\sphinxupquote{modelflow}} with the World Bank’s MFMod models, users  will have to understand at least some basic elements of \sphinxcode{\sphinxupquote{python}} syntax and usage.  Notably they will need to understand about packages, libraries and classes, how to access them.


\section{Starting python in windows}
\label{\detokenize{content/04_PythonEssentials/PythonPandasDataframes:starting-python-in-windows}}
\sphinxAtStartPar
To begin using \sphinxcode{\sphinxupquote{modelflow}}, python itself needs to be started.  This can be done either using the \sphinxcode{\sphinxupquote{Anaconda}} navigator or from the command line shell. In either case, the user will need to start python and select the \sphinxcode{\sphinxupquote{modelflow}} environment.

\begin{sphinxadmonition}{warning}{Warning:}
\sphinxAtStartPar
*** Ib recipie, no navigator the \sphinxhyphen{}old version is saved
\end{sphinxadmonition}


\section{Starting a Python session with modelflow.}
\label{\detokenize{content/04_PythonEssentials/PythonPandasDataframes:starting-a-python-session-with-modelflow}}
\sphinxAtStartPar
This can be done by starting the \sphinxcode{\sphinxupquote{Anaconda Prompt (miniconda3)}} app. This will create a command window with a base python environment. Now you want to change the environment to the modelflow environment. This is done like this:

\begin{sphinxVerbatim}[commandchars=\\\{\}]
\PYG{n}{conda} \PYG{n}{activate} \PYG{n}{modelflow}

\end{sphinxVerbatim}

\sphinxAtStartPar
Now we are ready to start jupyter by:

\begin{sphinxVerbatim}[commandchars=\\\{\}]
\PYG{n}{cd} \PYG{o}{\PYGZlt{}}\PYG{n}{path} \PYG{n}{where} \PYG{n}{you} \PYG{n}{want} \PYG{n}{to} \PYG{n}{start}\PYG{o}{\PYGZgt{}} 
\PYG{n}{jupyter} \PYG{n}{notebook}
\end{sphinxVerbatim}


\section{Start python from command line}
\label{\detokenize{content/04_PythonEssentials/PythonPandasDataframes:start-python-from-command-line}}
\sphinxAtStartPar
First you have to make anaconda/conda active this is done by opening a command prompt and issue:
\begin{quote}
\end{quote}

\sphinxAtStartPar
Then the modelflow enviroment has to be activated in order to get access to the library this is done like this:


\section{Python  packages, libraries and classes}
\label{\detokenize{content/04_PythonEssentials/PythonPandasDataframes:python-packages-libraries-and-classes}}
\sphinxAtStartPar
Some features of \sphinxcode{\sphinxupquote{python}} are built\sphinxhyphen{}in out\sphinxhyphen{}of\sphinxhyphen{}the\sphinxhyphen{}box.  Others build up on these basic features.

\sphinxAtStartPar
A \sphinxstylestrong{python class} is a code template that defines a python object. Classes can have properties {[}variables or data{]} associated with them and methods (behaviours or functions) associated with them. In python a class is created by the keyword class. An object of type class is created (instantiated) using the class’s “constructor” – a special method that creates an object that is an instance of a class.

\sphinxAtStartPar
A \sphinxstylestrong{module} is a Python object consisting of Python code. A module can define functions, classes and variables. A module can also include runnable code.

\sphinxAtStartPar
A \sphinxstylestrong{python package} is a collection of modules that are related to each other. When a module from an external package is required by a program, that package (or module in the package) must  be \sphinxstylestrong{imported} into the current session for its modules can be put to use.

\sphinxAtStartPar
A \sphinxstylestrong{python library} is a collection of related modules or packages.

\sphinxAtStartPar
\sphinxcode{\sphinxupquote{Modelflow}} is a python package that \sphinxstyleemphasis{inherits} (build on or adds to) the methods and properties of other \sphinxcode{\sphinxupquote{python}} classes like \sphinxcode{\sphinxupquote{pandas}}, \sphinxcode{\sphinxupquote{numpy}} and \sphinxcode{\sphinxupquote{mathplotlib}}.

\begin{sphinxadmonition}{note}{Note:}
\sphinxAtStartPar
In modelflow the model is a class and we can create an instance of a model (an object filled with the characteristics of the class) by executing the code \sphinxcode{\sphinxupquote{mymodel = model(myformulas)}} see below for a working example.
\end{sphinxadmonition}


\section{Importing packages, libraries, modules and classes}
\label{\detokenize{content/04_PythonEssentials/PythonPandasDataframes:importing-packages-libraries-modules-and-classes}}
\sphinxAtStartPar
Some libraries, packages, and modules are part of the core python package and will be available (importable) from the get\sphinxhyphen{}go.  Others are not, and need to be installed before importing them into a session.

\sphinxAtStartPar
If you followed the modelflow installation instructions you have already downloaded and installed on your computer all the packages necessary for running World Bank models under modelflow.  But to work with them in a given Jupyter Notebook session or in a program context, you will also need to \sphinxcode{\sphinxupquote{import}} them into your session before you call them.

\begin{sphinxadmonition}{note}{Note:}
\sphinxAtStartPar
\sphinxstylestrong{Installation} of a package is not the same as \sphinxstylestrong{import}ing a package. To be imported a package must be installed once on the computer that wishes to use it.  Once it has been installed, the package must be imported into each python session where it is to be used.
\end{sphinxadmonition}

\sphinxAtStartPar
Typically a python program will start with the importation of the libraries, classes and modules that will be used.  Because a Jupyter Notebook is essentially a heavily annotated program, it also requires that packages used be imported.

\sphinxAtStartPar
As described above packages, libraries and modules are containers that can include other elements.  Take for example the package Math.

\sphinxAtStartPar
To import the Math Package we execute the command \sphinxcode{\sphinxupquote{ import math}}.  Having done that we can can call the functions and data that are defined in it.

\begin{sphinxuseclass}{cell}\begin{sphinxVerbatimInput}

\begin{sphinxuseclass}{cell_input}
\begin{sphinxVerbatim}[commandchars=\\\{\}]
\PYG{c+c1}{\PYGZsh{} the \PYGZdq{}\PYGZsh{}\PYGZdq{}\PYGZdq{} in a code cell indicates a comment, test after the \PYGZsh{} will not be executed}
\PYG{k+kn}{import} \PYG{n+nn}{math}

\PYG{c+c1}{\PYGZsh{} Now that we have imported math we can access some of the elements identified in the package, }
\PYG{c+c1}{\PYGZsh{} For example math contains a definition for pi, we can access that by executing the pi method }
\PYG{c+c1}{\PYGZsh{} of the library math}
\PYG{n}{math}\PYG{o}{.}\PYG{n}{pi}
\end{sphinxVerbatim}

\end{sphinxuseclass}\end{sphinxVerbatimInput}
\begin{sphinxVerbatimOutput}

\begin{sphinxuseclass}{cell_output}
\begin{sphinxVerbatim}[commandchars=\\\{\}]
3.141592653589793
\end{sphinxVerbatim}

\end{sphinxuseclass}\end{sphinxVerbatimOutput}

\end{sphinxuseclass}

\subsection{Import specific elements or classes from a module or library}
\label{\detokenize{content/04_PythonEssentials/PythonPandasDataframes:import-specific-elements-or-classes-from-a-module-or-library}}
\sphinxAtStartPar
The python package \sphinxcode{\sphinxupquote{math}} contains several functions and classes.

\sphinxAtStartPar
If I want I can import them directly. Then when I call them I will not have to precede them with the name of their libary. to do this I use the \sphinxstylestrong{from} syntax.  \sphinxcode{\sphinxupquote{from math import pi,cos,sin}} will import the pi constant and the two functions cos and sin and allow me to call them directly.

\sphinxAtStartPar
Compared these calls with the one in the preceding section – there the call to the method pi has to be preceded by its namespace designator math.  i.e. \sphinxcode{\sphinxupquote{math.pi}}. Below we import pi directly and can just call it with pi.

\begin{sphinxuseclass}{cell}\begin{sphinxVerbatimInput}

\begin{sphinxuseclass}{cell_input}
\begin{sphinxVerbatim}[commandchars=\\\{\}]
\PYG{k+kn}{from} \PYG{n+nn}{math} \PYG{k+kn}{import} \PYG{n}{pi}\PYG{p}{,}\PYG{n}{cos}\PYG{p}{,}\PYG{n}{sin}

\PYG{n+nb}{print}\PYG{p}{(}\PYG{n}{pi}\PYG{p}{)}
\PYG{n+nb}{print}\PYG{p}{(}\PYG{n}{cos}\PYG{p}{(}\PYG{l+m+mi}{3}\PYG{p}{)}\PYG{p}{)}
\end{sphinxVerbatim}

\end{sphinxuseclass}\end{sphinxVerbatimInput}
\begin{sphinxVerbatimOutput}

\begin{sphinxuseclass}{cell_output}
\begin{sphinxVerbatim}[commandchars=\\\{\}]
3.141592653589793
\PYGZhy{}0.9899924966004454
\end{sphinxVerbatim}

\end{sphinxuseclass}\end{sphinxVerbatimOutput}

\end{sphinxuseclass}

\subsection{import a class but give it an alias}
\label{\detokenize{content/04_PythonEssentials/PythonPandasDataframes:import-a-class-but-give-it-an-alias}}
\sphinxAtStartPar
A class and instead of using its full name as above or it can be given an alias, that is hopefully shorter but still obvious enough that the user knows what class is being referred to.

\sphinxAtStartPar
For example  \sphinxcode{\sphinxupquote{import math as m}} allows a call to pi using the more succint syntax \sphinxcode{\sphinxupquote{m.py}}.

\begin{sphinxuseclass}{cell}\begin{sphinxVerbatimInput}

\begin{sphinxuseclass}{cell_input}
\begin{sphinxVerbatim}[commandchars=\\\{\}]
\PYG{k+kn}{import} \PYG{n+nn}{math} \PYG{k}{as} \PYG{n+nn}{m}
\PYG{n+nb}{print}\PYG{p}{(}\PYG{n}{m}\PYG{o}{.}\PYG{n}{pi}\PYG{p}{)}
\PYG{n+nb}{print}\PYG{p}{(}\PYG{n}{m}\PYG{o}{.}\PYG{n}{cos}\PYG{p}{(}\PYG{l+m+mi}{3}\PYG{p}{)}\PYG{p}{)}
\end{sphinxVerbatim}

\end{sphinxuseclass}\end{sphinxVerbatimInput}
\begin{sphinxVerbatimOutput}

\begin{sphinxuseclass}{cell_output}
\begin{sphinxVerbatim}[commandchars=\\\{\}]
3.141592653589793
\PYGZhy{}0.9899924966004454
\end{sphinxVerbatim}

\end{sphinxuseclass}\end{sphinxVerbatimOutput}

\end{sphinxuseclass}

\subsection{Standard aliases}
\label{\detokenize{content/04_PythonEssentials/PythonPandasDataframes:standard-aliases}}
\sphinxAtStartPar
Some packages are so frequently used that by convention they have been “assigned” specific aliases.

\sphinxAtStartPar
For example:

\sphinxAtStartPar
\sphinxstylestrong{Common aliases}


\begin{savenotes}\sphinxattablestart
\centering
\begin{tabulary}{\linewidth}[t]{|T|T|T|T|}
\hline
\sphinxstyletheadfamily 
\sphinxAtStartPar
Alias
&\sphinxstyletheadfamily 
\sphinxAtStartPar
aliased package
&\sphinxstyletheadfamily 
\sphinxAtStartPar
example
&\sphinxstyletheadfamily 
\sphinxAtStartPar
functionalty
\\
\hline
\sphinxAtStartPar
pd
&
\sphinxAtStartPar
pandas
&
\sphinxAtStartPar
import pandas as pd
&
\sphinxAtStartPar
Pandas are used for storing and retriveing data
\\
\hline
\sphinxAtStartPar
np
&
\sphinxAtStartPar
numpy
&
\sphinxAtStartPar
import numpy as np
&
\sphinxAtStartPar
Numpy gives access to some advanced mathematical features
\\
\hline
\end{tabulary}
\par
\sphinxattableend\end{savenotes}

\sphinxAtStartPar
You don’t have to use those conventions but it will make your code easier to read by others who are familiar with it.


\chapter{Introduction to Pandas dataframes}
\label{\detokenize{content/04_PythonEssentials/PythonPandasDataframes:introduction-to-pandas-dataframes}}
\sphinxAtStartPar
Modelflow is built on top of the Pandas library. Pandas is the Swiss knife of data science and can perform an impressing array of date oriented tasks.

\sphinxAtStartPar
This tutorial is a very short introduction to how pandas dataframes are used with Modelflow. For a more complete discussion see any of the many tutorials on the internet, notably:
\begin{itemize}
\item {} 
\sphinxAtStartPar
\sphinxhref{https://pandas.pydata.org/}{Pandas homepage}

\item {} 
\sphinxAtStartPar
\sphinxhref{https://pandas.pydata.org/pandas-docs/stable/getting\_started/tutorials.html}{Pandas community tutorials}

\end{itemize}


\section{Import the pandas library}
\label{\detokenize{content/04_PythonEssentials/PythonPandasDataframes:import-the-pandas-library}}
\sphinxAtStartPar
As with any python program, in order to use a package or library it must first be imported into the session. As noted above, by  convention pandas is imported as pd

\begin{sphinxuseclass}{cell}\begin{sphinxVerbatimInput}

\begin{sphinxuseclass}{cell_input}
\begin{sphinxVerbatim}[commandchars=\\\{\}]
\PYG{k+kn}{import} \PYG{n+nn}{pandas} \PYG{k}{as} \PYG{n+nn}{pd} 
\end{sphinxVerbatim}

\end{sphinxuseclass}\end{sphinxVerbatimInput}

\end{sphinxuseclass}
\sphinxAtStartPar
Pandas, like any library, contains many classes and methods.  The discussion below focuses on \sphinxstylestrong{Series} and \sphinxstylestrong{DataFrames} two classes that are part of the pandas library.  Both \sphinxcode{\sphinxupquote{series}} and \sphinxcode{\sphinxupquote{dataframes}} are containers that can be used to store time\sphinxhyphen{}series data and that have associated with them a number of very useful methods for displaying and manipulating time\sphinxhyphen{}series data.\\
Unlike other statistical packages neither \sphinxcode{\sphinxupquote{series}} nor \sphinxcode{\sphinxupquote{dataframes}} are inherently or exclusively time\sphinxhyphen{}series in nature.  \sphinxcode{\sphinxupquote{Modelflow}} and macro\sphinxhyphen{}economists use them in this way, but the classes themselves are not dated in anyway out\sphinxhyphen{}of\sphinxhyphen{}the\sphinxhyphen{}box.


\section{The \sphinxstyleliteralintitle{\sphinxupquote{Pandas}} class \sphinxstyleliteralintitle{\sphinxupquote{series}}}
\label{\detokenize{content/04_PythonEssentials/PythonPandasDataframes:the-pandas-class-series}}
\sphinxAtStartPar
A pandas series is class that can be used to instantiate an object that holds a two dimensional array comprised of values and an index.

\sphinxAtStartPar
The constructor for a \sphinxcode{\sphinxupquote{Series}} object is \sphinxcode{\sphinxupquote{pandas.Series()}}.  The content inside the parentheses will determine the nature of the series\sphinxhyphen{}object generated.  As an object\sphinxhyphen{}oriented language Python supports \sphinxcode{\sphinxupquote{overrides}} (which is to say a method can have more than one way in which it can be called). Specifically there can be different constructors defined for a class, depending on how the data that is to be used to initialize it is organized.


\subsection{Series declared from a list}
\label{\detokenize{content/04_PythonEssentials/PythonPandasDataframes:series-declared-from-a-list}}
\sphinxAtStartPar
The simplest way to create a Series is to pass an array of values as a Python list to the Series constructor.

\begin{sphinxadmonition}{note}{Note:}
\sphinxAtStartPar
A list in python is a comma delimited collection of items.  It could be text, numbers or even more complex objects.  When declared (and returned) list are enclosed in square brackets.

\sphinxAtStartPar
For example both of the following two lines are perfectly good examples of lists.

\sphinxAtStartPar
mylist={[}2,7,8,9{]}
mylist2={[}“Some text”,”Some more Text”,2,3{]}

\sphinxAtStartPar
The list is entirely agnostic about the type of data it contains.
\end{sphinxadmonition}

\sphinxAtStartPar
In the examples below Simplest, Simple and simple3 are all series – although series3 which is derived from a list mixing text and numeric values would be hard to interpret as an economic series.

\begin{sphinxuseclass}{cell}\begin{sphinxVerbatimInput}

\begin{sphinxuseclass}{cell_input}
\begin{sphinxVerbatim}[commandchars=\\\{\}]
\PYG{n}{values}\PYG{o}{=}\PYG{p}{[}\PYG{l+m+mi}{7}\PYG{p}{,}\PYG{l+m+mi}{8}\PYG{p}{,}\PYG{l+m+mi}{9}\PYG{p}{,}\PYG{l+m+mi}{10}\PYG{p}{,}\PYG{l+m+mi}{11}\PYG{p}{]}
\PYG{n}{weird}\PYG{o}{=}\PYG{p}{[}\PYG{l+s+s2}{\PYGZdq{}}\PYG{l+s+s2}{Some text}\PYG{l+s+s2}{\PYGZdq{}}\PYG{p}{,}\PYG{l+s+s2}{\PYGZdq{}}\PYG{l+s+s2}{Some more Text}\PYG{l+s+s2}{\PYGZdq{}}\PYG{p}{,}\PYG{l+m+mi}{2}\PYG{p}{,}\PYG{l+m+mi}{3}\PYG{p}{]}

\PYG{c+c1}{\PYGZsh{} Here the constructor is passed a numeric list}
\PYG{n}{Simplest}\PYG{o}{=}\PYG{n}{pd}\PYG{o}{.}\PYG{n}{Series}\PYG{p}{(}\PYG{p}{[}\PYG{l+m+mi}{2}\PYG{p}{,}\PYG{l+m+mi}{3}\PYG{p}{,}\PYG{l+m+mi}{4}\PYG{p}{,}\PYG{l+m+mi}{5}\PYG{p}{,}\PYG{l+m+mi}{6}\PYG{p}{]}\PYG{p}{)}
\PYG{n}{Simplest}
\end{sphinxVerbatim}

\end{sphinxuseclass}\end{sphinxVerbatimInput}
\begin{sphinxVerbatimOutput}

\begin{sphinxuseclass}{cell_output}
\begin{sphinxVerbatim}[commandchars=\\\{\}]
0    2
1    3
2    4
3    5
4    6
dtype: int64
\end{sphinxVerbatim}

\end{sphinxuseclass}\end{sphinxVerbatimOutput}

\end{sphinxuseclass}
\begin{sphinxuseclass}{cell}\begin{sphinxVerbatimInput}

\begin{sphinxuseclass}{cell_input}
\begin{sphinxVerbatim}[commandchars=\\\{\}]
\PYG{c+c1}{\PYGZsh{} In this case the constructor is passed a variable that contains a list}
\PYG{n}{simple2}\PYG{o}{=}\PYG{n}{pd}\PYG{o}{.}\PYG{n}{Series}\PYG{p}{(}\PYG{n}{values}\PYG{p}{)}
\PYG{n}{simple2}
\end{sphinxVerbatim}

\end{sphinxuseclass}\end{sphinxVerbatimInput}
\begin{sphinxVerbatimOutput}

\begin{sphinxuseclass}{cell_output}
\begin{sphinxVerbatim}[commandchars=\\\{\}]
0     7
1     8
2     9
3    10
4    11
dtype: int64
\end{sphinxVerbatim}

\end{sphinxuseclass}\end{sphinxVerbatimOutput}

\end{sphinxuseclass}
\begin{sphinxuseclass}{cell}\begin{sphinxVerbatimInput}

\begin{sphinxuseclass}{cell_input}
\begin{sphinxVerbatim}[commandchars=\\\{\}]
\PYG{c+c1}{\PYGZsh{} Here the constructor is passed a variable containing a list that is a mix of }
\PYG{c+c1}{\PYGZsh{} alphanumerics and numerical values}
\PYG{n}{simple3}\PYG{o}{=}\PYG{n}{pd}\PYG{o}{.}\PYG{n}{Series}\PYG{p}{(}\PYG{n}{weird}\PYG{p}{)}
\PYG{n}{simple3}
\end{sphinxVerbatim}

\end{sphinxuseclass}\end{sphinxVerbatimInput}
\begin{sphinxVerbatimOutput}

\begin{sphinxuseclass}{cell_output}
\begin{sphinxVerbatim}[commandchars=\\\{\}]
0         Some text
1    Some more Text
2                 2
3                 3
dtype: object
\end{sphinxVerbatim}

\end{sphinxuseclass}\end{sphinxVerbatimOutput}

\end{sphinxuseclass}
\sphinxAtStartPar
Note that all three series have different length.

\sphinxAtStartPar
Moreover, constructed in this way (by passing a list to the constructor) each of these \sphinxcode{\sphinxupquote{Series}} are automatically assigned a zero\sphinxhyphen{}based index (a numerial index that starts with 0).


\subsection{Series declared using a specific index}
\label{\detokenize{content/04_PythonEssentials/PythonPandasDataframes:series-declared-using-a-specific-index}}
\sphinxAtStartPar
In this example the series Simple and Simple2 are recreated (overwritten), but this time an index is specified. Here the index is declared as a(nother) list.

\begin{sphinxuseclass}{cell}\begin{sphinxVerbatimInput}

\begin{sphinxuseclass}{cell_input}
\begin{sphinxVerbatim}[commandchars=\\\{\}]
\PYG{c+c1}{\PYGZsh{} In this example the constructor is given both the values }
\PYG{c+c1}{\PYGZsh{} and specific values for the index}
\PYG{n}{Simplest}\PYG{o}{=}\PYG{n}{pd}\PYG{o}{.}\PYG{n}{Series}\PYG{p}{(}\PYG{p}{[}\PYG{l+m+mi}{2}\PYG{p}{,}\PYG{l+m+mi}{3}\PYG{p}{,}\PYG{l+m+mi}{4}\PYG{p}{,}\PYG{l+m+mi}{5}\PYG{p}{,}\PYG{l+m+mi}{6}\PYG{p}{]}\PYG{p}{,}\PYG{n}{index}\PYG{o}{=}\PYG{p}{[}\PYG{l+m+mi}{1966}\PYG{p}{,}\PYG{l+m+mi}{1967}\PYG{p}{,}\PYG{l+m+mi}{1996}\PYG{p}{,}\PYG{l+m+mi}{1999}\PYG{p}{,}\PYG{l+m+mi}{2000}\PYG{p}{]}\PYG{p}{)}
\PYG{n}{Simplest}
\end{sphinxVerbatim}

\end{sphinxuseclass}\end{sphinxVerbatimInput}
\begin{sphinxVerbatimOutput}

\begin{sphinxuseclass}{cell_output}
\begin{sphinxVerbatim}[commandchars=\\\{\}]
1966    2
1967    3
1996    4
1999    5
2000    6
dtype: int64
\end{sphinxVerbatim}

\end{sphinxuseclass}\end{sphinxVerbatimOutput}

\end{sphinxuseclass}
\begin{sphinxuseclass}{cell}\begin{sphinxVerbatimInput}

\begin{sphinxuseclass}{cell_input}
\begin{sphinxVerbatim}[commandchars=\\\{\}]
\PYG{n}{simple2}\PYG{o}{=}\PYG{n}{pd}\PYG{o}{.}\PYG{n}{Series}\PYG{p}{(}\PYG{n}{values}\PYG{p}{,}\PYG{n}{index}\PYG{o}{=}\PYG{p}{[}\PYG{l+m+mi}{1966}\PYG{p}{,}\PYG{l+m+mi}{1967}\PYG{p}{,}\PYG{l+m+mi}{1996}\PYG{p}{,}\PYG{l+m+mi}{1999}\PYG{p}{,}\PYG{l+m+mi}{2000}\PYG{p}{]}\PYG{p}{)}
\PYG{n}{simple2}
\end{sphinxVerbatim}

\end{sphinxuseclass}\end{sphinxVerbatimInput}
\begin{sphinxVerbatimOutput}

\begin{sphinxuseclass}{cell_output}
\begin{sphinxVerbatim}[commandchars=\\\{\}]
1966     7
1967     8
1996     9
1999    10
2000    11
dtype: int64
\end{sphinxVerbatim}

\end{sphinxuseclass}\end{sphinxVerbatimOutput}

\end{sphinxuseclass}
\sphinxAtStartPar
Now the Series look more like time series data!


\subsection{Create Series from a dictionary}
\label{\detokenize{content/04_PythonEssentials/PythonPandasDataframes:create-series-from-a-dictionary}}
\sphinxAtStartPar
In python a dictionary is a data structure that is more generally known in computer science as an associative array. A dictionary consists of a collection of key\sphinxhyphen{}value pairs, where each key\sphinxhyphen{}value pair \sphinxstyleemphasis{maps} or \sphinxstyleemphasis{links} the key to its associated value.

\begin{sphinxadmonition}{note}{Note:}
\sphinxAtStartPar
A dictionary is enclosed in curly brackets \{\}, versus a list which is enclosed in square brackets{[}{]}.
\end{sphinxadmonition}

\sphinxAtStartPar
Thus mydict=\{“1966”:2,”1967”:3,”1968”:4,”1969”:5,”2000”:\sphinxhyphen{}15\} creates an object called mydict.   \sphinxcode{\sphinxupquote{mydict}}maps (or links) the key “1966” links to the value 2.

\begin{sphinxadmonition}{note}{Note:}
\sphinxAtStartPar
In this example the Key was a string but we could just as easily made it a numerical value:
\end{sphinxadmonition}

\sphinxAtStartPar
mydict2=\{1966:2,1967:3,1968:4,1969:5,2000:\sphinxhyphen{}15\} creates an object called mydict2 that links (maps) the key “1966” to the value 2.

\sphinxAtStartPar
The series constructor also accepts a dictionary, and maps the key to the index of the Series.

\begin{sphinxuseclass}{cell}\begin{sphinxVerbatimInput}

\begin{sphinxuseclass}{cell_input}
\begin{sphinxVerbatim}[commandchars=\\\{\}]
\PYG{n}{mydict2}\PYG{o}{=}\PYG{p}{\PYGZob{}}\PYG{l+m+mi}{1966}\PYG{p}{:}\PYG{l+m+mi}{2}\PYG{p}{,}\PYG{l+m+mi}{1967}\PYG{p}{:}\PYG{l+m+mi}{3}\PYG{p}{,}\PYG{l+m+mi}{1968}\PYG{p}{:}\PYG{l+m+mi}{4}\PYG{p}{,}\PYG{l+m+mi}{1969}\PYG{p}{:}\PYG{l+m+mi}{5}\PYG{p}{,}\PYG{l+m+mi}{2000}\PYG{p}{:}\PYG{l+m+mi}{6}\PYG{p}{\PYGZcb{}}
\PYG{n}{simple2}\PYG{o}{=}\PYG{n}{pd}\PYG{o}{.}\PYG{n}{Series}\PYG{p}{(}\PYG{n}{mydict2}\PYG{p}{)}
\PYG{n}{simple2}
\end{sphinxVerbatim}

\end{sphinxuseclass}\end{sphinxVerbatimInput}
\begin{sphinxVerbatimOutput}

\begin{sphinxuseclass}{cell_output}
\begin{sphinxVerbatim}[commandchars=\\\{\}]
1966    2
1967    3
1968    4
1969    5
2000    6
dtype: int64
\end{sphinxVerbatim}

\end{sphinxuseclass}\end{sphinxVerbatimOutput}

\end{sphinxuseclass}

\section{Properties and methods of \sphinxstyleliteralintitle{\sphinxupquote{DataFrames}} in \sphinxstyleliteralintitle{\sphinxupquote{modelflow}}}
\label{\detokenize{content/04_PythonEssentials/PythonPandasDataframes:properties-and-methods-of-dataframes-in-modelflow}}
\sphinxAtStartPar
Any class can have both properties (data) and methods (functions that operate on the data of the particular instance of the class). With object\sphinxhyphen{}oriented programming languages like python, classes can be built as supersets of existing classes. The \sphinxcode{\sphinxupquote{modelflow}} class \sphinxcode{\sphinxupquote{model}} inherits or encapsulates all of the features of the pandas dataframe and extends it in many important ways.  Some of the methods below are standard pandas methods, others have been added to it by \sphinxcode{\sphinxupquote{modelflow}} features

\sphinxAtStartPar
Much more detail on standard pandas dataframes can be found on the \sphinxhref{https://pandas.pydata.org/docs/reference/frame.html}{official pandas website}.


\subsection{\sphinxstyleliteralintitle{\sphinxupquote{DataFrame}}s}
\label{\detokenize{content/04_PythonEssentials/PythonPandasDataframes:dataframes}}
\sphinxAtStartPar
The \sphinxcode{\sphinxupquote{DataFrame}} is the primary structure of pandas and is a two\sphinxhyphen{}dimensional data structure with named rows and columns.  Each columns can have different data types (numeric, string, etc).

\sphinxAtStartPar
By convention, a dataframe if often called df or some other modifier followed by df, to assist in reading the code.


\subsection{Creating or instantiating a dataframe}
\label{\detokenize{content/04_PythonEssentials/PythonPandasDataframes:creating-or-instantiating-a-dataframe}}
\sphinxAtStartPar
Like any object, a \sphinxcode{\sphinxupquote{DataFrame}} can be created by calling the constructor of the pandas class \sphinxcode{\sphinxupquote{DataFrame}}.

\sphinxAtStartPar
Each class has many constructors, so there are very many ways to create a \sphinxcode{\sphinxupquote{dataframe}}. The \sphinxcode{\sphinxupquote{pandas.DataFrame()}} method is constructor for the \sphinxcode{\sphinxupquote{DataFrame}} class. It takes several forms (as with \sphinxcode{\sphinxupquote{Series}}), but always returns an instance (instantiates) of a \sphinxcode{\sphinxupquote{DataFrame}} object – i.e. a variable whose contents are a \sphinxcode{\sphinxupquote{DataFrame}}.

\sphinxAtStartPar
The code example below creates a \sphinxcode{\sphinxupquote{DataFrame}} of three columns A,B,C; indexed between 2019 and 2021.  Macroeconomists may interpret the index as dates, but for pandas they are just numbers.

\sphinxAtStartPar
Below a \sphinxcode{\sphinxupquote{DataFrame}} named \sphinxcode{\sphinxupquote{df}} is instantiated from a dictionary and assigned a specific index by passing a list of years as the index.

\begin{sphinxuseclass}{cell}\begin{sphinxVerbatimInput}

\begin{sphinxuseclass}{cell_input}
\begin{sphinxVerbatim}[commandchars=\\\{\}]
\PYG{n}{df} \PYG{o}{=} \PYG{n}{pd}\PYG{o}{.}\PYG{n}{DataFrame}\PYG{p}{(}\PYG{p}{\PYGZob{}}\PYG{l+s+s1}{\PYGZsq{}}\PYG{l+s+s1}{B}\PYG{l+s+s1}{\PYGZsq{}}\PYG{p}{:} \PYG{p}{[}\PYG{l+m+mi}{1}\PYG{p}{,}\PYG{l+m+mi}{1}\PYG{p}{,}\PYG{l+m+mi}{1}\PYG{p}{,}\PYG{l+m+mi}{1}\PYG{p}{]}\PYG{p}{,}\PYG{l+s+s1}{\PYGZsq{}}\PYG{l+s+s1}{C}\PYG{l+s+s1}{\PYGZsq{}}\PYG{p}{:}\PYG{p}{[}\PYG{l+m+mi}{1}\PYG{p}{,}\PYG{l+m+mi}{2}\PYG{p}{,}\PYG{l+m+mi}{3}\PYG{p}{,}\PYG{l+m+mi}{6}\PYG{p}{]}\PYG{p}{,}\PYG{l+s+s1}{\PYGZsq{}}\PYG{l+s+s1}{E}\PYG{l+s+s1}{\PYGZsq{}}\PYG{p}{:}\PYG{p}{[}\PYG{l+m+mi}{4}\PYG{p}{,}\PYG{l+m+mi}{4}\PYG{p}{,}\PYG{l+m+mi}{4}\PYG{p}{,}\PYG{l+m+mi}{4}\PYG{p}{]}\PYG{p}{\PYGZcb{}}\PYG{p}{,}\PYG{n}{index}\PYG{o}{=}\PYG{p}{[}\PYG{l+m+mi}{2018}\PYG{p}{,}\PYG{l+m+mi}{2019}\PYG{p}{,}\PYG{l+m+mi}{2020}\PYG{p}{,}\PYG{l+m+mi}{2021}\PYG{p}{]}\PYG{p}{)}
\PYG{n}{df} 
\end{sphinxVerbatim}

\end{sphinxuseclass}\end{sphinxVerbatimInput}
\begin{sphinxVerbatimOutput}

\begin{sphinxuseclass}{cell_output}
\begin{sphinxVerbatim}[commandchars=\\\{\}]
      B  C  E
2018  1  1  4
2019  1  2  4
2020  1  3  4
2021  1  6  4
\end{sphinxVerbatim}

\end{sphinxuseclass}\end{sphinxVerbatimOutput}

\end{sphinxuseclass}
\begin{sphinxadmonition}{note}{Note:}
\sphinxAtStartPar
In the \sphinxcode{\sphinxupquote{DataFrame}}s that are used in macrostructural models like MFMod, each  column is often interpreted as a time\sphinxhyphen{}series of an economic variable. So in this dataframe,  normally A, B and C each be interpreted as economic time series.

\sphinxAtStartPar
That said, there is nothing in the \sphinxcode{\sphinxupquote{DataFrame}} class that suggests that the data it stores must be time\sphinxhyphen{}series or even numeric in nature.
\end{sphinxadmonition}


\subsection{Adding a column to a dataframe}
\label{\detokenize{content/04_PythonEssentials/PythonPandasDataframes:adding-a-column-to-a-dataframe}}
\sphinxAtStartPar
If a value is assigned to a column that does not exist, pandas will add a column with that name and fill it with values resulting from  the calculation.

\begin{sphinxadmonition}{note}{Note:}
\sphinxAtStartPar
The size of the object assigned to the new column must match the size (number of rows) of the pre\sphinxhyphen{}existing \sphinxcode{\sphinxupquote{DataFarame}}.
\end{sphinxadmonition}

\begin{sphinxuseclass}{cell}\begin{sphinxVerbatimInput}

\begin{sphinxuseclass}{cell_input}
\begin{sphinxVerbatim}[commandchars=\\\{\}]
\PYG{n}{df}\PYG{p}{[}\PYG{l+s+s2}{\PYGZdq{}}\PYG{l+s+s2}{NEW}\PYG{l+s+s2}{\PYGZdq{}}\PYG{p}{]}\PYG{o}{=}\PYG{p}{[}\PYG{l+m+mi}{10}\PYG{p}{,}\PYG{l+m+mi}{12}\PYG{p}{,}\PYG{l+m+mi}{10}\PYG{p}{,}\PYG{l+m+mi}{13}\PYG{p}{]}
\PYG{n}{df}
\end{sphinxVerbatim}

\end{sphinxuseclass}\end{sphinxVerbatimInput}
\begin{sphinxVerbatimOutput}

\begin{sphinxuseclass}{cell_output}
\begin{sphinxVerbatim}[commandchars=\\\{\}]
      B  C  E  NEW
2018  1  1  4   10
2019  1  2  4   12
2020  1  3  4   10
2021  1  6  4   13
\end{sphinxVerbatim}

\end{sphinxuseclass}\end{sphinxVerbatimOutput}

\end{sphinxuseclass}

\subsection{Revising values}
\label{\detokenize{content/04_PythonEssentials/PythonPandasDataframes:revising-values}}
\sphinxAtStartPar
If the column exists than the = method will revise the values of the rows with the values assigned in the statement.

\begin{sphinxadmonition}{warning}{Warning:}
\sphinxAtStartPar
The dimensions of the list assigned via the \sphinxcode{\sphinxupquote{=}} method must be the same as the \sphinxcode{\sphinxupquote{DataFrame}} (i.e. there must be exactly as many values as there are rows).  Alternatively if only one value is provided, then that value will replace all of the values in the specified column (be broadcast to the other rows in the column).
\end{sphinxadmonition}

\begin{sphinxuseclass}{cell}\begin{sphinxVerbatimInput}

\begin{sphinxuseclass}{cell_input}
\begin{sphinxVerbatim}[commandchars=\\\{\}]
\PYG{n}{df}\PYG{p}{[}\PYG{l+s+s2}{\PYGZdq{}}\PYG{l+s+s2}{NEW}\PYG{l+s+s2}{\PYGZdq{}}\PYG{p}{]}\PYG{o}{=}\PYG{p}{[}\PYG{l+m+mi}{11}\PYG{p}{,}\PYG{l+m+mi}{12}\PYG{p}{,}\PYG{l+m+mi}{10}\PYG{p}{,}\PYG{l+m+mi}{14}\PYG{p}{]}

\PYG{n}{df}
\end{sphinxVerbatim}

\end{sphinxuseclass}\end{sphinxVerbatimInput}
\begin{sphinxVerbatimOutput}

\begin{sphinxuseclass}{cell_output}
\begin{sphinxVerbatim}[commandchars=\\\{\}]
      B  C  E  NEW
2018  1  1  4   11
2019  1  2  4   12
2020  1  3  4   10
2021  1  6  4   14
\end{sphinxVerbatim}

\end{sphinxuseclass}\end{sphinxVerbatimOutput}

\end{sphinxuseclass}
\begin{sphinxuseclass}{cell}\begin{sphinxVerbatimInput}

\begin{sphinxuseclass}{cell_input}
\begin{sphinxVerbatim}[commandchars=\\\{\}]
\PYG{c+c1}{\PYGZsh{} replace all of the rows of column B with the same value}
\PYG{n}{df}\PYG{p}{[}\PYG{l+s+s1}{\PYGZsq{}}\PYG{l+s+s1}{B}\PYG{l+s+s1}{\PYGZsq{}}\PYG{p}{]}\PYG{o}{=}\PYG{l+m+mi}{17}
\PYG{n}{df}
\end{sphinxVerbatim}

\end{sphinxuseclass}\end{sphinxVerbatimInput}
\begin{sphinxVerbatimOutput}

\begin{sphinxuseclass}{cell_output}
\begin{sphinxVerbatim}[commandchars=\\\{\}]
       B  C  E  NEW
2018  17  1  4   11
2019  17  2  4   12
2020  17  3  4   10
2021  17  6  4   14
\end{sphinxVerbatim}

\end{sphinxuseclass}\end{sphinxVerbatimOutput}

\end{sphinxuseclass}

\section{Column names in  Modelflow}
\label{\detokenize{content/04_PythonEssentials/PythonPandasDataframes:column-names-in-modelflow}}
\begin{sphinxShadowBox}
\sphinxstylesidebartitle{Modelflow variable names}

\sphinxAtStartPar
Modelflow places more restrictions on column names than do pandas \sphinxstyleemphasis{per se}.
\end{sphinxShadowBox}

\sphinxAtStartPar
While pandas dataframes are very liberal in what names can be given to columns, \sphinxcode{\sphinxupquote{modelflow}} is more restrictive.

\sphinxAtStartPar
Specifically, in modelflow a variable name must:
\begin{itemize}
\item {} 
\sphinxAtStartPar
start with a letter

\item {} 
\sphinxAtStartPar
be upper case

\end{itemize}

\sphinxAtStartPar
Thus while all these are legal column names in pandas, some are illegal in modelflow.


\begin{savenotes}\sphinxattablestart
\centering
\begin{tabulary}{\linewidth}[t]{|T|T|T|}
\hline
\sphinxstyletheadfamily 
\sphinxAtStartPar
Variable Name
&\sphinxstyletheadfamily 
\sphinxAtStartPar
Legal in modelfow?
&\sphinxstyletheadfamily 
\sphinxAtStartPar
Reason
\\
\hline
\sphinxAtStartPar
IB
&
\sphinxAtStartPar
yes
&
\sphinxAtStartPar
Starts with a letter and is uppercase
\\
\hline
\sphinxAtStartPar
ib
&
\sphinxAtStartPar
no
&
\sphinxAtStartPar
 lowercase letters are not allowed
\\
\hline
\sphinxAtStartPar
42ANSWER
&
\sphinxAtStartPar
No
&
\sphinxAtStartPar
 does not start with a letter 
\\
\hline
\sphinxAtStartPar
\_HORSE1
&
\sphinxAtStartPar
No
&
\sphinxAtStartPar
does not start with a letter 
\\
\hline
\sphinxAtStartPar
A\_VERY\_LONG\_NAME\_THAT\_IS\_LEGAL\_3
&
\sphinxAtStartPar
Yes
&
\sphinxAtStartPar
 Starts with a letter and is uppercase 
\\
\hline
\end{tabulary}
\par
\sphinxattableend\end{savenotes}


\section{.index and time dimensions in Modelflow}
\label{\detokenize{content/04_PythonEssentials/PythonPandasDataframes:index-and-time-dimensions-in-modelflow}}
\sphinxAtStartPar
As we saw above, series have indices.  Dataframes also have indices, which are the row names of the dataframe.

\sphinxAtStartPar
In \sphinxcode{\sphinxupquote{modelflow}} the index series is typically understood to represent a date.

\sphinxAtStartPar
For yearly models a list of integers like in the above example works fine.

\sphinxAtStartPar
For higher frequency models the index can be one of pandas datatypes.

\begin{sphinxadmonition}{warning}{Warning:}
\sphinxAtStartPar
Not all datetypes work well with the graphics routines of modelflow.  Users are advised to use the \sphinxcode{\sphinxupquote{pd.period\_range()}} method to generate date indexes.

\sphinxAtStartPar
For example:

\begin{sphinxVerbatim}[commandchars=\\\{\}]
    \PYG{n}{dates} \PYG{o}{=} \PYG{n}{pd}\PYG{o}{.}\PYG{n}{period\PYGZus{}range}\PYG{p}{(}\PYG{n}{start}\PYG{o}{=}\PYG{l+s+s1}{\PYGZsq{}}\PYG{l+s+s1}{1975q1}\PYG{l+s+s1}{\PYGZsq{}}\PYG{p}{,}\PYG{n}{end}\PYG{o}{=}\PYG{l+s+s1}{\PYGZsq{}}\PYG{l+s+s1}{2125q4}\PYG{l+s+s1}{\PYGZsq{}}\PYG{p}{,}\PYG{n}{freq}\PYG{o}{=}\PYG{l+s+s1}{\PYGZsq{}}\PYG{l+s+s1}{Q}\PYG{l+s+s1}{\PYGZsq{}}\PYG{p}{)}
    \PYG{n}{df}\PYG{o}{.}\PYG{n}{index}\PYG{o}{=}\PYG{n}{dates}
\end{sphinxVerbatim}
\end{sphinxadmonition}


\subsection{Leads and lags}
\label{\detokenize{content/04_PythonEssentials/PythonPandasDataframes:leads-and-lags}}
\sphinxAtStartPar
In modelflow leads and lags can be indicated by following the variable with a parenthesis and either \sphinxhyphen{}1 or \sphinxhyphen{}2 two for one or two period lags (where the number following the negative sign indicates the number of time periods that are lagged). Positive numbers are used for forward leads (no +sign required).

\sphinxAtStartPar
When a method defined by the \sphinxcode{\sphinxupquote{modelflow}} class encounters something like \sphinxcode{\sphinxupquote{A(\sphinxhyphen{}1)}}, it will take the value from the row above the current row. No matter if the index is an integer, a year, quarter or a millisecond. The same goes for leads, \sphinxcode{\sphinxupquote{A(+1)}} will return the value of \sphinxcode{\sphinxupquote{A}} in the next row.

\sphinxAtStartPar
As a result in a quarterly model \sphinxcode{\sphinxupquote{B=A(\sphinxhyphen{}4)}} would assign B the value of A from the same quarter in the previous year.


\subsection{.columns lists the column names of a dataframe}
\label{\detokenize{content/04_PythonEssentials/PythonPandasDataframes:columns-lists-the-column-names-of-a-dataframe}}
\sphinxAtStartPar
The method \sphinxcode{\sphinxupquote{.columns}} returns the names of the columns in the dataframe.

\begin{sphinxuseclass}{cell}\begin{sphinxVerbatimInput}

\begin{sphinxuseclass}{cell_input}
\begin{sphinxVerbatim}[commandchars=\\\{\}]
\PYG{n}{df}\PYG{o}{.}\PYG{n}{columns}
\end{sphinxVerbatim}

\end{sphinxuseclass}\end{sphinxVerbatimInput}
\begin{sphinxVerbatimOutput}

\begin{sphinxuseclass}{cell_output}
\begin{sphinxVerbatim}[commandchars=\\\{\}]
Index([\PYGZsq{}B\PYGZsq{}, \PYGZsq{}C\PYGZsq{}, \PYGZsq{}E\PYGZsq{}, \PYGZsq{}NEW\PYGZsq{}], dtype=\PYGZsq{}object\PYGZsq{})
\end{sphinxVerbatim}

\end{sphinxuseclass}\end{sphinxVerbatimOutput}

\end{sphinxuseclass}

\subsection{.size indicates the dimension of a list}
\label{\detokenize{content/04_PythonEssentials/PythonPandasDataframes:size-indicates-the-dimension-of-a-list}}
\sphinxAtStartPar
so \sphinxcode{\sphinxupquote{df.columns.size}} returns the number of columns in a dataframe.

\begin{sphinxuseclass}{cell}\begin{sphinxVerbatimInput}

\begin{sphinxuseclass}{cell_input}
\begin{sphinxVerbatim}[commandchars=\\\{\}]
\PYG{n}{df}\PYG{o}{.}\PYG{n}{columns}\PYG{o}{.}\PYG{n}{size}
\end{sphinxVerbatim}

\end{sphinxuseclass}\end{sphinxVerbatimInput}
\begin{sphinxVerbatimOutput}

\begin{sphinxuseclass}{cell_output}
\begin{sphinxVerbatim}[commandchars=\\\{\}]
4
\end{sphinxVerbatim}

\end{sphinxuseclass}\end{sphinxVerbatimOutput}

\end{sphinxuseclass}
\sphinxAtStartPar
The dataframe df has 4 columns.


\subsection{.eval() evaluates calculates an expression on the data of a dataframe}
\label{\detokenize{content/04_PythonEssentials/PythonPandasDataframes:eval-evaluates-calculates-an-expression-on-the-data-of-a-dataframe}}
\sphinxAtStartPar
\sphinxcode{\sphinxupquote{.eval}} is a native dataframe method, which does calculations on a \sphinxcode{\sphinxupquote{dataframe}} and returns a revised \sphinxcode{\sphinxupquote{datafame}}. With this method expressions can be evaluated and new columns created.

\begin{sphinxuseclass}{cell}\begin{sphinxVerbatimInput}

\begin{sphinxuseclass}{cell_input}
\begin{sphinxVerbatim}[commandchars=\\\{\}]
\PYG{n}{df}\PYG{o}{.}\PYG{n}{eval}\PYG{p}{(}\PYG{l+s+s1}{\PYGZsq{}\PYGZsq{}\PYGZsq{}}\PYG{l+s+s1}{X = B*C}
\PYG{l+s+s1}{           THE\PYGZus{}ANSWER = 42}\PYG{l+s+s1}{\PYGZsq{}\PYGZsq{}\PYGZsq{}}\PYG{p}{)}
\end{sphinxVerbatim}

\end{sphinxuseclass}\end{sphinxVerbatimInput}
\begin{sphinxVerbatimOutput}

\begin{sphinxuseclass}{cell_output}
\begin{sphinxVerbatim}[commandchars=\\\{\}]
       B  C  E  NEW    X  THE\PYGZus{}ANSWER
2018  17  1  4   11   17          42
2019  17  2  4   12   34          42
2020  17  3  4   10   51          42
2021  17  6  4   14  102          42
\end{sphinxVerbatim}

\end{sphinxuseclass}\end{sphinxVerbatimOutput}

\end{sphinxuseclass}
\begin{sphinxuseclass}{cell}\begin{sphinxVerbatimInput}

\begin{sphinxuseclass}{cell_input}
\begin{sphinxVerbatim}[commandchars=\\\{\}]
\PYG{n}{df}
\end{sphinxVerbatim}

\end{sphinxuseclass}\end{sphinxVerbatimInput}
\begin{sphinxVerbatimOutput}

\begin{sphinxuseclass}{cell_output}
\begin{sphinxVerbatim}[commandchars=\\\{\}]
       B  C  E  NEW
2018  17  1  4   11
2019  17  2  4   12
2020  17  3  4   10
2021  17  6  4   14
\end{sphinxVerbatim}

\end{sphinxuseclass}\end{sphinxVerbatimOutput}

\end{sphinxuseclass}
\sphinxAtStartPar
In the above example the resulting dataframe is displayed but is not stored.

\sphinxAtStartPar
To store it, the results of the calculation must be assigned to a variable.  The pre\sphinxhyphen{}existing dataframe can be overwritten by assigning it the result of the eval statement.

\begin{sphinxuseclass}{cell}\begin{sphinxVerbatimInput}

\begin{sphinxuseclass}{cell_input}
\begin{sphinxVerbatim}[commandchars=\\\{\}]
\PYG{n}{df}\PYG{o}{=}\PYG{n}{df}\PYG{o}{.}\PYG{n}{eval}\PYG{p}{(}\PYG{l+s+s1}{\PYGZsq{}\PYGZsq{}\PYGZsq{}}\PYG{l+s+s1}{X = B*C}
\PYG{l+s+s1}{           THE\PYGZus{}ANSWER = 42}\PYG{l+s+s1}{\PYGZsq{}\PYGZsq{}\PYGZsq{}}\PYG{p}{)}
\PYG{n}{df}
\end{sphinxVerbatim}

\end{sphinxuseclass}\end{sphinxVerbatimInput}
\begin{sphinxVerbatimOutput}

\begin{sphinxuseclass}{cell_output}
\begin{sphinxVerbatim}[commandchars=\\\{\}]
       B  C  E  NEW    X  THE\PYGZus{}ANSWER
2018  17  1  4   11   17          42
2019  17  2  4   12   34          42
2020  17  3  4   10   51          42
2021  17  6  4   14  102          42
\end{sphinxVerbatim}

\end{sphinxuseclass}\end{sphinxVerbatimOutput}

\end{sphinxuseclass}
\sphinxAtStartPar
With this operation the new columns, x and THE\_ANSWER have been appended to the dataframe df.

\begin{sphinxadmonition}{note}{Note:}
\sphinxAtStartPar
The \sphinxcode{\sphinxupquote{.eval()}} method is a native pandas method.  As such it cannot handle lagged variables (because pandas do not support the idea of a lagged variable.

\sphinxAtStartPar
The \sphinxcode{\sphinxupquote{.mfcalc()}} and the \sphinxcode{\sphinxupquote{.upd()}} methods discussed below are \sphinxcode{\sphinxupquote{modelflow}} features that extend the functionalities native to \sphinxcode{\sphinxupquote{dataframe}} that allows such calculations to be performed.
\end{sphinxadmonition}


\subsection{.loc{[}{]} selects a portion (slice) of a dataframe}
\label{\detokenize{content/04_PythonEssentials/PythonPandasDataframes:loc-selects-a-portion-slice-of-a-dataframe}}
\sphinxAtStartPar
The \sphinxcode{\sphinxupquote{.loc{[}{]}}} method allows you to display and/or revise specific sub\sphinxhyphen{}sections of a column or row in a dataframe.


\subsubsection{.loc{[}row,column{]} A single element}
\label{\detokenize{content/04_PythonEssentials/PythonPandasDataframes:loc-row-column-a-single-element}}
\sphinxAtStartPar
\sphinxcode{\sphinxupquote{.loc{[}row,column{]}}} operates on a single cell in the dataframe.  Thus the below displays the value of the cell with index=2019 observation from the  column C.

\begin{sphinxuseclass}{cell}\begin{sphinxVerbatimInput}

\begin{sphinxuseclass}{cell_input}
\begin{sphinxVerbatim}[commandchars=\\\{\}]
\PYG{n}{df}\PYG{o}{.}\PYG{n}{loc}\PYG{p}{[}\PYG{l+m+mi}{2019}\PYG{p}{,}\PYG{l+s+s1}{\PYGZsq{}}\PYG{l+s+s1}{C}\PYG{l+s+s1}{\PYGZsq{}}\PYG{p}{]}
\end{sphinxVerbatim}

\end{sphinxuseclass}\end{sphinxVerbatimInput}
\begin{sphinxVerbatimOutput}

\begin{sphinxuseclass}{cell_output}
\begin{sphinxVerbatim}[commandchars=\\\{\}]
2
\end{sphinxVerbatim}

\end{sphinxuseclass}\end{sphinxVerbatimOutput}

\end{sphinxuseclass}

\subsubsection{.loc{[}:,column{]} A single column}
\label{\detokenize{content/04_PythonEssentials/PythonPandasDataframes:loc-column-a-single-column}}
\sphinxAtStartPar
The lone colon in a loc statement indicates all the rows or columns.  Here all of the rows.

\begin{sphinxuseclass}{cell}\begin{sphinxVerbatimInput}

\begin{sphinxuseclass}{cell_input}
\begin{sphinxVerbatim}[commandchars=\\\{\}]
\PYG{n}{df}\PYG{o}{.}\PYG{n}{loc}\PYG{p}{[}\PYG{p}{:}\PYG{p}{,}\PYG{l+s+s1}{\PYGZsq{}}\PYG{l+s+s1}{C}\PYG{l+s+s1}{\PYGZsq{}}\PYG{p}{]}
\end{sphinxVerbatim}

\end{sphinxuseclass}\end{sphinxVerbatimInput}
\begin{sphinxVerbatimOutput}

\begin{sphinxuseclass}{cell_output}
\begin{sphinxVerbatim}[commandchars=\\\{\}]
2018    1
2019    2
2020    3
2021    6
Name: C, dtype: int64
\end{sphinxVerbatim}

\end{sphinxuseclass}\end{sphinxVerbatimOutput}

\end{sphinxuseclass}

\subsubsection{.loc{[}row,:{]} A single row}
\label{\detokenize{content/04_PythonEssentials/PythonPandasDataframes:loc-row-a-single-row}}
\sphinxAtStartPar
Here all of the columns, for the selected row.

\begin{sphinxuseclass}{cell}\begin{sphinxVerbatimInput}

\begin{sphinxuseclass}{cell_input}
\begin{sphinxVerbatim}[commandchars=\\\{\}]
\PYG{n}{df}\PYG{o}{.}\PYG{n}{loc}\PYG{p}{[}\PYG{l+m+mi}{2019}\PYG{p}{,}\PYG{p}{:}\PYG{p}{]}
\end{sphinxVerbatim}

\end{sphinxuseclass}\end{sphinxVerbatimInput}
\begin{sphinxVerbatimOutput}

\begin{sphinxuseclass}{cell_output}
\begin{sphinxVerbatim}[commandchars=\\\{\}]
B             17
C              2
E              4
NEW           12
X             34
THE\PYGZus{}ANSWER    42
Name: 2019, dtype: int64
\end{sphinxVerbatim}

\end{sphinxuseclass}\end{sphinxVerbatimOutput}

\end{sphinxuseclass}

\subsubsection{.loc{[}:,{[}names…{]}{]} Several columns}
\label{\detokenize{content/04_PythonEssentials/PythonPandasDataframes:loc-names-several-columns}}
\sphinxAtStartPar
Passing a list in either the rows or columns portion of the loc statement will allow multiple rows or columns to be displayed.

\begin{sphinxuseclass}{cell}\begin{sphinxVerbatimInput}

\begin{sphinxuseclass}{cell_input}
\begin{sphinxVerbatim}[commandchars=\\\{\}]
\PYG{n}{df}\PYG{o}{.}\PYG{n}{loc}\PYG{p}{[}\PYG{p}{[}\PYG{l+m+mi}{2018}\PYG{p}{,}\PYG{l+m+mi}{2021}\PYG{p}{]}\PYG{p}{,}\PYG{p}{[}\PYG{l+s+s1}{\PYGZsq{}}\PYG{l+s+s1}{B}\PYG{l+s+s1}{\PYGZsq{}}\PYG{p}{,}\PYG{l+s+s1}{\PYGZsq{}}\PYG{l+s+s1}{C}\PYG{l+s+s1}{\PYGZsq{}}\PYG{p}{]}\PYG{p}{]}
\end{sphinxVerbatim}

\end{sphinxuseclass}\end{sphinxVerbatimInput}
\begin{sphinxVerbatimOutput}

\begin{sphinxuseclass}{cell_output}
\begin{sphinxVerbatim}[commandchars=\\\{\}]
       B  C
2018  17  1
2021  17  6
\end{sphinxVerbatim}

\end{sphinxuseclass}\end{sphinxVerbatimOutput}

\end{sphinxuseclass}

\subsubsection{.loc using the colon to select a range}
\label{\detokenize{content/04_PythonEssentials/PythonPandasDataframes:loc-using-the-colon-to-select-a-range}}
\sphinxAtStartPar
with the colon operator we can also select a range of results.

\sphinxAtStartPar
Here from 2018 to 2019.

\begin{sphinxuseclass}{cell}\begin{sphinxVerbatimInput}

\begin{sphinxuseclass}{cell_input}
\begin{sphinxVerbatim}[commandchars=\\\{\}]
\PYG{n}{df}\PYG{o}{.}\PYG{n}{loc}\PYG{p}{[}\PYG{l+m+mi}{2018}\PYG{p}{:}\PYG{l+m+mi}{2020}\PYG{p}{,}\PYG{p}{[}\PYG{l+s+s1}{\PYGZsq{}}\PYG{l+s+s1}{B}\PYG{l+s+s1}{\PYGZsq{}}\PYG{p}{,}\PYG{l+s+s1}{\PYGZsq{}}\PYG{l+s+s1}{C}\PYG{l+s+s1}{\PYGZsq{}}\PYG{p}{]}\PYG{p}{]}
\end{sphinxVerbatim}

\end{sphinxuseclass}\end{sphinxVerbatimInput}
\begin{sphinxVerbatimOutput}

\begin{sphinxuseclass}{cell_output}
\begin{sphinxVerbatim}[commandchars=\\\{\}]
       B  C
2018  17  1
2019  17  2
2020  17  3
\end{sphinxVerbatim}

\end{sphinxuseclass}\end{sphinxVerbatimOutput}

\end{sphinxuseclass}

\subsubsection{.loc{[}{]} can also be used on the left hand side to assign values to specific cells}
\label{\detokenize{content/04_PythonEssentials/PythonPandasDataframes:loc-can-also-be-used-on-the-left-hand-side-to-assign-values-to-specific-cells}}
\sphinxAtStartPar
This can be very handy when updating scenarios.

\begin{sphinxuseclass}{cell}\begin{sphinxVerbatimInput}

\begin{sphinxuseclass}{cell_input}
\begin{sphinxVerbatim}[commandchars=\\\{\}]
\PYG{n}{df}\PYG{o}{.}\PYG{n}{loc}\PYG{p}{[}\PYG{l+m+mi}{2019}\PYG{p}{:}\PYG{l+m+mi}{2020}\PYG{p}{,}\PYG{l+s+s1}{\PYGZsq{}}\PYG{l+s+s1}{C}\PYG{l+s+s1}{\PYGZsq{}}\PYG{p}{]} \PYG{o}{=} \PYG{l+m+mi}{17}
\PYG{n}{df}
\end{sphinxVerbatim}

\end{sphinxuseclass}\end{sphinxVerbatimInput}
\begin{sphinxVerbatimOutput}

\begin{sphinxuseclass}{cell_output}
\begin{sphinxVerbatim}[commandchars=\\\{\}]
       B   C  E  NEW    X  THE\PYGZus{}ANSWER
2018  17   1  4   11   17          42
2019  17  17  4   12   34          42
2020  17  17  4   10   51          42
2021  17   6  4   14  102          42
\end{sphinxVerbatim}

\end{sphinxuseclass}\end{sphinxVerbatimOutput}

\end{sphinxuseclass}
\begin{sphinxadmonition}{warning}{Warning:}
\sphinxAtStartPar
The dimensions on the right hand side of = and the left hand side should match. That is: either the dimensions should be the same, or the right hand side should be \sphinxcode{\sphinxupquote{broadcasted}} into the left hand slice.

\sphinxAtStartPar
For more on broadcasting \sphinxhref{https://jakevdp.github.io/PythonDataScienceHandbook/02.05-computation-on-arrays-broadcasting.html}{see here}
\end{sphinxadmonition}

\sphinxAtStartPar
\sphinxstylestrong{For more info on the .loc{[}{]} method}
\begin{itemize}
\item {} 
\sphinxAtStartPar
\sphinxhref{https://pandas.pydata.org/docs/reference/api/pandas.DataFrame.loc.html}{Description}

\item {} 
\sphinxAtStartPar
\sphinxhref{https://www.google.com/search?q=pandas+dataframe+loc\&newwindow=1}{Search}

\end{itemize}

\sphinxAtStartPar
\sphinxstylestrong{For more info on pandas:}
\begin{itemize}
\item {} 
\sphinxAtStartPar
\sphinxhref{https://pandas.pydata.org/}{Pandas homepage}

\item {} 
\sphinxAtStartPar
\sphinxhref{https://pandas.pydata.org/pandas-docs/stable/getting\_started/tutorials.html}{Pandas community tutorials}

\end{itemize}

\sphinxstepscope


\chapter{Modelflow extensions to pandas}
\label{\detokenize{content/04_PythonEssentials/UpdateCommand:modelflow-extensions-to-pandas}}\label{\detokenize{content/04_PythonEssentials/UpdateCommand::doc}}
\sphinxAtStartPar
Modeflow inherits all the capabilities of pandas and extends some as well.

\sphinxAtStartPar
Data in a dataframe can be modified directly with built\sphinxhyphen{}in pandas functionalities like \sphinxcode{\sphinxupquote{.loc{[}{]}}} and \sphinxcode{\sphinxupquote{eval()}}, but \sphinxcode{\sphinxupquote{modelflow}} extends these capabilities with in important ways with the \sphinxcode{\sphinxupquote{.upd()}} and \sphinxcode{\sphinxupquote{.mfcalc()}} methods.


\section{.upd() method of modelflow}
\label{\detokenize{content/04_PythonEssentials/UpdateCommand:upd-method-of-modelflow}}
\sphinxAtStartPar
The \sphinxcode{\sphinxupquote{.upd()}} method extends pandas by giving the user a concise and expressive way to modify data in a dataframe using a syntax that a database\sphinxhyphen{}manager or macroeconomic modeler might find more natural.

\sphinxAtStartPar
Notably it allows the user to employ formula’s to do updates, and supports both lags and leads on variables.

\sphinxAtStartPar
\sphinxcode{\sphinxupquote{.upd()}} can be used to:
\begin{itemize}
\item {} 
\sphinxAtStartPar
Perform different types of  updates

\item {} 
\sphinxAtStartPar
Perform multiple updates each on a new line

\item {} 
\sphinxAtStartPar
Perform changes over specific periods

\item {} 
\sphinxAtStartPar
Use one input which is used for all time frames, or a separate input for each time

\item {} 
\sphinxAtStartPar
Preserve pre\sphinxhyphen{}shock growth rates for out of sample time\sphinxhyphen{}periods

\item {} 
\sphinxAtStartPar
Display results

\end{itemize}


\subsection{\sphinxstyleliteralintitle{\sphinxupquote{.upd()}} method operators}
\label{\detokenize{content/04_PythonEssentials/UpdateCommand:upd-method-operators}}
\sphinxAtStartPar
Below are some of the operators that can be used in the \sphinxcode{\sphinxupquote{.upd()}} method

\begin{sphinxadmonition}{warning}{Warning:}
\sphinxAtStartPar
IB: typos reworked
\end{sphinxadmonition}

\sphinxAtStartPar
\sphinxstylestrong{Types of update:}


\begin{savenotes}\sphinxattablestart
\centering
\begin{tabulary}{\linewidth}[t]{|T|T|}
\hline
\sphinxstyletheadfamily 
\sphinxAtStartPar
Update to perform
&\sphinxstyletheadfamily 
\sphinxAtStartPar
Use this operator
\\
\hline
\sphinxAtStartPar
Set a variable equal to the input
&
\sphinxAtStartPar
=
\\
\hline
\sphinxAtStartPar
Add the input to the input
&
\sphinxAtStartPar
+
\\
\hline
\sphinxAtStartPar
Set the variable to itself multiplied by the input
&
\sphinxAtStartPar
*
\\
\hline
\sphinxAtStartPar
Increase/Decrease the variable by a percent of itself (1+input/100)
&
\sphinxAtStartPar
\%
\\
\hline
\sphinxAtStartPar
Set the growth rate of the variable to the input
&
\sphinxAtStartPar
=growth
\\
\hline
\sphinxAtStartPar
Change the growth rate of the variable to its current growth rate plus the input value in percentage points
&
\sphinxAtStartPar
+growth
\\
\hline
\sphinxAtStartPar
Specify the amount by which the variable should increase from its previous period level (\(\Delta = var_t - var_{t-1}\))
&
\sphinxAtStartPar
=diff
\\
\hline
\end{tabulary}
\par
\sphinxattableend\end{savenotes}

\begin{sphinxadmonition}{danger}{Danger:}
\sphinxAtStartPar
Note: the syntax of an update command requires that there be a space between variable names and the operators.

\sphinxAtStartPar
Thus \sphinxcode{\sphinxupquote{df.upd("A = 7")}} is fine, but \sphinxcode{\sphinxupquote{df.upd("A =7")}} will generate an error.

\sphinxAtStartPar
Similarly  \sphinxcode{\sphinxupquote{df.upd("A * 1.1")}} is fine, but \sphinxcode{\sphinxupquote{df.upd("A* 1.1")}} will generate an error.
\end{sphinxadmonition}


\subsection{\sphinxstyleliteralintitle{\sphinxupquote{.upd()}} some examples}
\label{\detokenize{content/04_PythonEssentials/UpdateCommand:upd-some-examples}}

\subsection{Setting up the python environment}
\label{\detokenize{content/04_PythonEssentials/UpdateCommand:setting-up-the-python-environment}}
\sphinxAtStartPar
In order to use \sphinxcode{\sphinxupquote{.upd()}} all of the necessary libraries must be \sphinxstylestrong{imported} into the python session.

\begin{sphinxuseclass}{cell}\begin{sphinxVerbatimInput}

\begin{sphinxuseclass}{cell_input}
\begin{sphinxVerbatim}[commandchars=\\\{\}]
\PYG{c+c1}{\PYGZsh{} First import pandas and the model into the  workspace}
\PYG{c+c1}{\PYGZsh{} There is no problem importing the same library multiple times. }
\PYG{c+c1}{\PYGZsh{} After the first import there is not any significant penalty for additional mports }
\PYG{k+kn}{import} \PYG{n+nn}{pandas} \PYG{k}{as} \PYG{n+nn}{pd}

\PYG{k+kn}{from} \PYG{n+nn}{modelclass} \PYG{k+kn}{import} \PYG{n}{model} 
\PYG{c+c1}{\PYGZsh{} functions that improve rendering of modelflow outputs under Jupyter Notebook}
\PYG{n}{model}\PYG{o}{.}\PYG{n}{widescreen}\PYG{p}{(}\PYG{p}{)}
\PYG{n}{model}\PYG{o}{.}\PYG{n}{scroll\PYGZus{}off}\PYG{p}{(}\PYG{p}{)}
\end{sphinxVerbatim}

\end{sphinxuseclass}\end{sphinxVerbatimInput}
\begin{sphinxVerbatimOutput}

\begin{sphinxuseclass}{cell_output}
\begin{sphinxVerbatim}[commandchars=\\\{\}]
\PYGZlt{}IPython.core.display.HTML object\PYGZgt{}
\end{sphinxVerbatim}

\end{sphinxuseclass}\end{sphinxVerbatimOutput}

\end{sphinxuseclass}
\sphinxAtStartPar
Now create a dataframe using standard pandas syntax.  In this instance with years as the index and a dictionary defining the variables and their data.

\begin{sphinxuseclass}{cell}\begin{sphinxVerbatimInput}

\begin{sphinxuseclass}{cell_input}
\begin{sphinxVerbatim}[commandchars=\\\{\}]
\PYG{c+c1}{\PYGZsh{} Create a dataframe using standard pandas}

\PYG{n}{df} \PYG{o}{=} \PYG{n}{pd}\PYG{o}{.}\PYG{n}{DataFrame}\PYG{p}{(}\PYG{p}{\PYGZob{}}\PYG{l+s+s1}{\PYGZsq{}}\PYG{l+s+s1}{B}\PYG{l+s+s1}{\PYGZsq{}}\PYG{p}{:} \PYG{p}{[}\PYG{l+m+mi}{1}\PYG{p}{,}\PYG{l+m+mi}{1}\PYG{p}{,}\PYG{l+m+mi}{1}\PYG{p}{,}\PYG{l+m+mi}{1}\PYG{p}{]}\PYG{p}{,}\PYG{l+s+s1}{\PYGZsq{}}\PYG{l+s+s1}{C}\PYG{l+s+s1}{\PYGZsq{}}\PYG{p}{:}\PYG{p}{[}\PYG{l+m+mi}{1}\PYG{p}{,}\PYG{l+m+mi}{2}\PYG{p}{,}\PYG{l+m+mi}{3}\PYG{p}{,}\PYG{l+m+mi}{6}\PYG{p}{]}\PYG{p}{,}\PYG{l+s+s1}{\PYGZsq{}}\PYG{l+s+s1}{E}\PYG{l+s+s1}{\PYGZsq{}}\PYG{p}{:}\PYG{p}{[}\PYG{l+m+mi}{4}\PYG{p}{,}\PYG{l+m+mi}{4}\PYG{p}{,}\PYG{l+m+mi}{4}\PYG{p}{,}\PYG{l+m+mi}{4}\PYG{p}{]}\PYG{p}{\PYGZcb{}}\PYG{p}{,}\PYG{n}{index}\PYG{o}{=}\PYG{p}{[}\PYG{l+m+mi}{2018}\PYG{p}{,}\PYG{l+m+mi}{2019}\PYG{p}{,}\PYG{l+m+mi}{2020}\PYG{p}{,}\PYG{l+m+mi}{2021}\PYG{p}{]}\PYG{p}{)}
\PYG{n}{df} 
\end{sphinxVerbatim}

\end{sphinxuseclass}\end{sphinxVerbatimInput}
\begin{sphinxVerbatimOutput}

\begin{sphinxuseclass}{cell_output}
\begin{sphinxVerbatim}[commandchars=\\\{\}]
      B  C  E
2018  1  1  4
2019  1  2  4
2020  1  3  4
2021  1  6  4
\end{sphinxVerbatim}

\end{sphinxuseclass}\end{sphinxVerbatimOutput}

\end{sphinxuseclass}
\sphinxAtStartPar
A somewhat more creative way to initialize the dataframe for dates would use a loop to specify the dates that get passed to the constructor as an argument.

\sphinxAtStartPar
Below a dataframe df with two Series (A and B), is initialized with the values 100 for all data points.

\sphinxAtStartPar
The index is defined dynamically by a loop \sphinxcode{\sphinxupquote{index={[}2020+v for v in range(number\_of\_rows){]}}} that runs for number\_of\_rows times (6 times in this example) setting v equal to 2020+0, 2020+1,…,202+5. The resulting list whose values are assigned to index is {[}2020,2021,2022,2023,2024,2025{]}.

\sphinxAtStartPar
The big advantage of this method is that if the user wanted to have data created for the period 1990 to 2030, they would only have to change number\_of\_rows from 6 to 41 and 2020 in the loop to 1990.

\sphinxAtStartPar
The second example simplifies further by just specifying the begin and end point of the range.

\begin{sphinxuseclass}{cell}\begin{sphinxVerbatimInput}

\begin{sphinxuseclass}{cell_input}
\begin{sphinxVerbatim}[commandchars=\\\{\}]
\PYG{c+c1}{\PYGZsh{}define the number of years for which the data is to be created.}
\PYG{n}{number\PYGZus{}of\PYGZus{}rows} \PYG{o}{=} \PYG{l+m+mi}{6} 

\PYG{c+c1}{\PYGZsh{} call the dataframe constructor}
\PYG{n}{df} \PYG{o}{=} \PYG{n}{pd}\PYG{o}{.}\PYG{n}{DataFrame}\PYG{p}{(}\PYG{l+m+mi}{100}\PYG{p}{,}
       \PYG{n}{index}\PYG{o}{=}\PYG{p}{[}\PYG{l+m+mi}{2020}\PYG{o}{+}\PYG{n}{v} \PYG{k}{for} \PYG{n}{v} \PYG{o+ow}{in} \PYG{n+nb}{range}\PYG{p}{(}\PYG{n}{number\PYGZus{}of\PYGZus{}rows}\PYG{p}{)}\PYG{p}{]}\PYG{p}{,} \PYG{c+c1}{\PYGZsh{} create row index}
       \PYG{c+c1}{\PYGZsh{} equivalent to index=[2020,2021,2022,2023,2024,2025] }
       \PYG{n}{columns}\PYG{o}{=}\PYG{p}{[}\PYG{l+s+s1}{\PYGZsq{}}\PYG{l+s+s1}{A}\PYG{l+s+s1}{\PYGZsq{}}\PYG{p}{,}\PYG{l+s+s1}{\PYGZsq{}}\PYG{l+s+s1}{B}\PYG{l+s+s1}{\PYGZsq{}}\PYG{p}{]}\PYG{p}{)}                                 \PYG{c+c1}{\PYGZsh{} create column name }
\PYG{n}{df}

\PYG{n}{df1} \PYG{o}{=} \PYG{n}{pd}\PYG{o}{.}\PYG{n}{DataFrame}\PYG{p}{(}\PYG{l+m+mi}{200}\PYG{p}{,}
       \PYG{n}{index}\PYG{o}{=}\PYG{p}{[}\PYG{n}{v} \PYG{k}{for} \PYG{n}{v} \PYG{o+ow}{in} \PYG{n+nb}{range}\PYG{p}{(}\PYG{l+m+mi}{2020}\PYG{p}{,}\PYG{l+m+mi}{2030}\PYG{p}{)}\PYG{p}{]}\PYG{p}{,} \PYG{c+c1}{\PYGZsh{} create row index}
       \PYG{c+c1}{\PYGZsh{} equivalent to index=[2020,2021,...,2030] }
       \PYG{n}{columns}\PYG{o}{=}\PYG{p}{[}\PYG{l+s+s1}{\PYGZsq{}}\PYG{l+s+s1}{A1}\PYG{l+s+s1}{\PYGZsq{}}\PYG{p}{,}\PYG{l+s+s1}{\PYGZsq{}}\PYG{l+s+s1}{B1}\PYG{l+s+s1}{\PYGZsq{}}\PYG{p}{]}\PYG{p}{)}                                 \PYG{c+c1}{\PYGZsh{} create column name }
\PYG{n}{df1}
\end{sphinxVerbatim}

\end{sphinxuseclass}\end{sphinxVerbatimInput}
\begin{sphinxVerbatimOutput}

\begin{sphinxuseclass}{cell_output}
\begin{sphinxVerbatim}[commandchars=\\\{\}]
       A1   B1
2020  200  200
2021  200  200
2022  200  200
2023  200  200
2024  200  200
2025  200  200
2026  200  200
2027  200  200
2028  200  200
2029  200  200
\end{sphinxVerbatim}

\end{sphinxuseclass}\end{sphinxVerbatimOutput}

\end{sphinxuseclass}

\subsection{Use .upd to create a new variable (= operator)}
\label{\detokenize{content/04_PythonEssentials/UpdateCommand:use-upd-to-create-a-new-variable-operator}}
\sphinxAtStartPar
With standard pandas a user can add a column (series) to a dataframe simply by assigning a adding to a dataframe.  For example:

\sphinxAtStartPar
\sphinxcode{\sphinxupquote{df{[}'NEW2'{]}={[}17,12,14,15{]}}}

\sphinxAtStartPar
\sphinxcode{\sphinxupquote{.upd()}} provides this functionality as well.

\begin{sphinxuseclass}{cell}\begin{sphinxVerbatimInput}

\begin{sphinxuseclass}{cell_input}
\begin{sphinxVerbatim}[commandchars=\\\{\}]
\PYG{n}{df2}\PYG{o}{=}\PYG{n}{df}\PYG{o}{.}\PYG{n}{upd}\PYG{p}{(}\PYG{l+s+s1}{\PYGZsq{}}\PYG{l+s+s1}{c = 142}\PYG{l+s+s1}{\PYGZsq{}}\PYG{p}{)} 
\PYG{n}{df2}
\end{sphinxVerbatim}

\end{sphinxuseclass}\end{sphinxVerbatimInput}
\begin{sphinxVerbatimOutput}

\begin{sphinxuseclass}{cell_output}
\begin{sphinxVerbatim}[commandchars=\\\{\}]
        A    B      C
2020  100  100  142.0
2021  100  100  142.0
2022  100  100  142.0
2023  100  100  142.0
2024  100  100  142.0
2025  100  100  142.0
\end{sphinxVerbatim}

\end{sphinxuseclass}\end{sphinxVerbatimOutput}

\end{sphinxuseclass}
\begin{sphinxadmonition}{note}{Note:}
\sphinxAtStartPar
Note that the new variable name was entered as a lower case ‘c’ here.  Lowercase letters are not legal \sphinxcode{\sphinxupquote{modelflow}} variable names.  The \sphinxcode{\sphinxupquote{.upd()}} method knows is part of modelflow and knows this rule, so it automatically translates lowercase entries into upper case so that the statement works.
\end{sphinxadmonition}


\subsection{Multiple updates and specific time periods}
\label{\detokenize{content/04_PythonEssentials/UpdateCommand:multiple-updates-and-specific-time-periods}}
\sphinxAtStartPar
The modelflow method \sphinxcode{\sphinxupquote{.upd()}} takes a string as an argument.  That string can contain a single update command or can contain multiple commands.

\sphinxAtStartPar
Moreover by including a <Begin End> date clause in a given update command, the update will be restricted to the associated time period.

\sphinxAtStartPar
The below illustrates this, modifying two existing variables A, B over different time periods and creating a new variable.

\begin{sphinxadmonition}{danger}{Danger:}
\sphinxAtStartPar
Note that the third line inherits the time period of the previous line.

\sphinxAtStartPar
Note also the submitted string can include comments as well (denoted with the standard python \# indicator).
\end{sphinxadmonition}

\begin{sphinxuseclass}{cell}\begin{sphinxVerbatimInput}

\begin{sphinxuseclass}{cell_input}
\begin{sphinxVerbatim}[commandchars=\\\{\}]
\PYG{n}{df}\PYG{o}{.}\PYG{n}{upd}\PYG{p}{(}\PYG{l+s+s2}{\PYGZdq{}\PYGZdq{}\PYGZdq{}}
\PYG{l+s+s2}{\PYGZsh{} Same number of values as years}
\PYG{l+s+s2}{\PYGZlt{}2021 2024\PYGZgt{} A = 42 44 45 46    \PYGZsh{} 4 years}
\PYG{l+s+s2}{\PYGZlt{}2020     \PYGZgt{} B = 200            \PYGZsh{} 1 year }
\PYG{l+s+s2}{c = 500                        \PYGZsh{} Same period as previous line}
\PYG{l+s+s2}{\PYGZlt{}\PYGZhy{}0 \PYGZhy{}1\PYGZgt{} D = 33                   \PYGZsh{} All years }
\PYG{l+s+s2}{\PYGZdq{}\PYGZdq{}\PYGZdq{}}\PYG{p}{)}
\end{sphinxVerbatim}

\end{sphinxuseclass}\end{sphinxVerbatimInput}
\begin{sphinxVerbatimOutput}

\begin{sphinxuseclass}{cell_output}
\begin{sphinxVerbatim}[commandchars=\\\{\}]
        A    B      C     D
2020  100  200  500.0  33.0
2021   42  100    0.0  33.0
2022   44  100    0.0  33.0
2023   45  100    0.0  33.0
2024   46  100    0.0  33.0
2025  100  100    0.0  33.0
\end{sphinxVerbatim}

\end{sphinxuseclass}\end{sphinxVerbatimOutput}

\end{sphinxuseclass}
\begin{sphinxShadowBox}
\sphinxstylesidebartitle{}

\sphinxAtStartPar
\sphinxstylestrong{Time scope of .upd()}

\sphinxAtStartPar
Made this a margin just to see

\sphinxAtStartPar
The update command takes a variety of mathematical operators \sphinxcode{\sphinxupquote{=, +, *, \% =GROWTH, +GROWTH, =DIFF}} and applies them to data for the period set in the leading <>.

\sphinxAtStartPar
If the user wants to modify a series or group of series for only a specific point in time or a period of time, she can indicate the period in the command line.
\begin{itemize}
\item {} 
\sphinxAtStartPar
If \sphinxstylestrong{one date} is specified the operation is applied to a single point in time

\item {} 
\sphinxAtStartPar
If \sphinxstylestrong{two dates}  are specifies the operation is applied over a period of time.

\end{itemize}

\sphinxAtStartPar
The selected time period will persist until re\sphinxhyphen{}set with a new time specification. Useful to avoid visual noise if several variables are going to be updated for the same time period.

\sphinxAtStartPar
The time period can be rest to the full time\sphinxhyphen{}period by using the special <\sphinxhyphen{}0 \sphinxhyphen{}1> time period.  More generally:
\begin{itemize}
\item {} 
\sphinxAtStartPar
Indicates the start of the dataframe use \sphinxhyphen{}0

\item {} 
\sphinxAtStartPar
Indicates the end of the dataframe use \sphinxhyphen{}1

\end{itemize}

\sphinxAtStartPar
If no time is provided the dataframe start and end period will be used.
\end{sphinxShadowBox}


\subsection{Setting specific datapoints to specific values}
\label{\detokenize{content/04_PythonEssentials/UpdateCommand:setting-specific-datapoints-to-specific-values}}
\sphinxAtStartPar
This example, demonstrates the equals operator.  The \sphinxcode{\sphinxupquote{=}} operator indicates that the variable a should be set equal to the indicated values following the \sphinxcode{\sphinxupquote{=}} operator (42 44 45 46 in the first line, 200 in the second and 500 inthe third). The dates enclosed in <> indicate the period over which the change should be applied.

\sphinxAtStartPar
Either:
\begin{itemize}
\item {} 
\sphinxAtStartPar
The number of data points provided must match the number of dates in the period, Or

\item {} 
\sphinxAtStartPar
Only one data point is provided, it is applied to all dates in the period.

\end{itemize}

\sphinxAtStartPar
If only one period is to be modified then it can be followed by just one date.

\sphinxAtStartPar
Note that the final line inherited the time period set in the second line.

\begin{sphinxuseclass}{cell}\begin{sphinxVerbatimInput}

\begin{sphinxuseclass}{cell_input}
\begin{sphinxVerbatim}[commandchars=\\\{\}]
\PYG{n}{df}\PYG{o}{.}\PYG{n}{upd}\PYG{p}{(}\PYG{l+s+s2}{\PYGZdq{}\PYGZdq{}\PYGZdq{}}
\PYG{l+s+s2}{\PYGZsh{} Same number of values as years}
\PYG{l+s+s2}{\PYGZlt{}2021 2024\PYGZgt{} A = 42 44 45 46    \PYGZsh{} 4 years}
\PYG{l+s+s2}{\PYGZlt{}2023     \PYGZgt{} B = 200            \PYGZsh{} 1 year }
\PYG{l+s+s2}{c = 500}
\PYG{l+s+s2}{\PYGZdq{}\PYGZdq{}\PYGZdq{}}\PYG{p}{)}
\end{sphinxVerbatim}

\end{sphinxuseclass}\end{sphinxVerbatimInput}
\begin{sphinxVerbatimOutput}

\begin{sphinxuseclass}{cell_output}
\begin{sphinxVerbatim}[commandchars=\\\{\}]
        A    B      C
2020  100  100    0.0
2021   42  100    0.0
2022   44  100    0.0
2023   45  200  500.0
2024   46  100    0.0
2025  100  100    0.0
\end{sphinxVerbatim}

\end{sphinxuseclass}\end{sphinxVerbatimOutput}

\end{sphinxuseclass}

\subsection{Adding  the specified  values to all values in a range (the + operator)}
\label{\detokenize{content/04_PythonEssentials/UpdateCommand:adding-the-specified-values-to-all-values-in-a-range-the-operator}}
\sphinxAtStartPar
NB: Here upd with the  + operator indicates that we are adding 42.

\begin{sphinxuseclass}{cell}\begin{sphinxVerbatimInput}

\begin{sphinxuseclass}{cell_input}
\begin{sphinxVerbatim}[commandchars=\\\{\}]
\PYG{n}{df}\PYG{o}{.}\PYG{n}{upd}\PYG{p}{(}\PYG{l+s+s1}{\PYGZsq{}\PYGZsq{}\PYGZsq{}}
\PYG{l+s+s1}{\PYGZsh{} Or one number to all years in between start and end }
\PYG{l+s+s1}{\PYGZlt{}2022 2024\PYGZgt{} B  +  42    \PYGZsh{} one value broadcast to 3 years }
\PYG{l+s+s1}{\PYGZsq{}\PYGZsq{}\PYGZsq{}}\PYG{p}{)}
\end{sphinxVerbatim}

\end{sphinxuseclass}\end{sphinxVerbatimInput}
\begin{sphinxVerbatimOutput}

\begin{sphinxuseclass}{cell_output}
\begin{sphinxVerbatim}[commandchars=\\\{\}]
        A    B
2020  100  100
2021  100  100
2022  100  142
2023  100  142
2024  100  142
2025  100  100
\end{sphinxVerbatim}

\end{sphinxuseclass}\end{sphinxVerbatimOutput}

\end{sphinxuseclass}

\subsection{Multiplying all values in a range by the specified values (the * operator)}
\label{\detokenize{content/04_PythonEssentials/UpdateCommand:multiplying-all-values-in-a-range-by-the-specified-values-the-operator}}
\begin{sphinxuseclass}{cell}\begin{sphinxVerbatimInput}

\begin{sphinxuseclass}{cell_input}
\begin{sphinxVerbatim}[commandchars=\\\{\}]
\PYG{n}{df}\PYG{o}{.}\PYG{n}{upd}\PYG{p}{(}\PYG{l+s+s1}{\PYGZsq{}\PYGZsq{}\PYGZsq{}}
\PYG{l+s+s1}{\PYGZsh{} Same number of values as years}
\PYG{l+s+s1}{\PYGZlt{}2021 2023\PYGZgt{} A *  42 44 55}
\PYG{l+s+s1}{\PYGZsq{}\PYGZsq{}\PYGZsq{}}\PYG{p}{)}
\end{sphinxVerbatim}

\end{sphinxuseclass}\end{sphinxVerbatimInput}
\begin{sphinxVerbatimOutput}

\begin{sphinxuseclass}{cell_output}
\begin{sphinxVerbatim}[commandchars=\\\{\}]
         A    B
2020   100  100
2021  4200  100
2022  4400  100
2023  5500  100
2024   100  100
2025   100  100
\end{sphinxVerbatim}

\end{sphinxuseclass}\end{sphinxVerbatimOutput}

\end{sphinxuseclass}

\subsection{Increasing all  values in a range by a  specified percent amount (the \% operator)}
\label{\detokenize{content/04_PythonEssentials/UpdateCommand:increasing-all-values-in-a-range-by-a-specified-percent-amount-the-operator}}
\sphinxAtStartPar
In this example:
\begin{itemize}
\item {} 
\sphinxAtStartPar
A is increased by 42 and 44\% over the range 2021 through 2022.

\item {} 
\sphinxAtStartPar
B is increased by 10 percent in all years

\item {} 
\sphinxAtStartPar
C is a new variable, is created and set to 100 for the whole range

\item {} 
\sphinxAtStartPar
C is decreased by 12 percent over the range 2023 through 2025.

\end{itemize}

\begin{sphinxuseclass}{cell}\begin{sphinxVerbatimInput}

\begin{sphinxuseclass}{cell_input}
\begin{sphinxVerbatim}[commandchars=\\\{\}]
\PYG{n}{df}\PYG{o}{.}\PYG{n}{upd}\PYG{p}{(}\PYG{l+s+s1}{\PYGZsq{}\PYGZsq{}\PYGZsq{}}
\PYG{l+s+s1}{\PYGZlt{}2021 2022 \PYGZgt{} A }\PYG{l+s+s1}{\PYGZpc{}}\PYG{l+s+s1}{  42 44   }
\PYG{l+s+s1}{\PYGZlt{}\PYGZhy{}0 \PYGZhy{}1\PYGZgt{} B }\PYG{l+s+s1}{\PYGZpc{}}\PYG{l+s+s1}{ 10            \PYGZsh{} all rows }
\PYG{l+s+s1}{C = 100                   \PYGZsh{} all rows persist }
\PYG{l+s+s1}{\PYGZlt{}2023 2025\PYGZgt{} C }\PYG{l+s+s1}{\PYGZpc{}}\PYG{l+s+s1}{ \PYGZhy{}12       \PYGZsh{} now only for 3 years }
\PYG{l+s+s1}{\PYGZsq{}\PYGZsq{}\PYGZsq{}}\PYG{p}{)}
\end{sphinxVerbatim}

\end{sphinxuseclass}\end{sphinxVerbatimInput}
\begin{sphinxVerbatimOutput}

\begin{sphinxuseclass}{cell_output}
\begin{sphinxVerbatim}[commandchars=\\\{\}]
        A      B      C
2020  100  110.0  100.0
2021  142  110.0  100.0
2022  144  110.0  100.0
2023  100  110.0   88.0
2024  100  110.0   88.0
2025  100  110.0   88.0
\end{sphinxVerbatim}

\end{sphinxuseclass}\end{sphinxVerbatimOutput}

\end{sphinxuseclass}

\subsection{Set the percent growth rate to specified values (=GROWTH)}
\label{\detokenize{content/04_PythonEssentials/UpdateCommand:set-the-percent-growth-rate-to-specified-values-growth}}
\begin{sphinxuseclass}{cell}\begin{sphinxVerbatimInput}

\begin{sphinxuseclass}{cell_input}
\begin{sphinxVerbatim}[commandchars=\\\{\}]
\PYG{n}{res} \PYG{o}{=} \PYG{n}{df}\PYG{o}{.}\PYG{n}{upd}\PYG{p}{(}\PYG{l+s+s1}{\PYGZsq{}\PYGZsq{}\PYGZsq{}}
\PYG{l+s+s1}{\PYGZsh{} Same number of values as years}
\PYG{l+s+s1}{\PYGZlt{}2021 2022\PYGZgt{} A =GROWTH  1 5  }
\PYG{l+s+s1}{\PYGZlt{}2020\PYGZgt{} c = 100 }
\PYG{l+s+s1}{\PYGZlt{}2021 2025\PYGZgt{} c =GROWTH 2 }
\PYG{l+s+s1}{\PYGZsq{}\PYGZsq{}\PYGZsq{}}\PYG{p}{)}
\PYG{n+nb}{print}\PYG{p}{(}\PYG{l+s+sa}{f}\PYG{l+s+s1}{\PYGZsq{}}\PYG{l+s+s1}{Dataframe:}\PYG{l+s+se}{\PYGZbs{}n}\PYG{l+s+si}{\PYGZob{}}\PYG{n}{res}\PYG{l+s+si}{\PYGZcb{}}\PYG{l+s+se}{\PYGZbs{}n}\PYG{l+s+se}{\PYGZbs{}n}\PYG{l+s+s1}{Growth:}\PYG{l+s+se}{\PYGZbs{}n}\PYG{l+s+si}{\PYGZob{}}\PYG{n}{res}\PYG{o}{.}\PYG{n}{pct\PYGZus{}change}\PYG{p}{(}\PYG{p}{)}\PYG{o}{*}\PYG{l+m+mi}{100}\PYG{l+s+si}{\PYGZcb{}}\PYG{l+s+se}{\PYGZbs{}n}\PYG{l+s+s1}{\PYGZsq{}}\PYG{p}{)} \PYG{c+c1}{\PYGZsh{} Explained b}
\end{sphinxVerbatim}

\end{sphinxuseclass}\end{sphinxVerbatimInput}
\begin{sphinxVerbatimOutput}

\begin{sphinxuseclass}{cell_output}
\begin{sphinxVerbatim}[commandchars=\\\{\}]
Dataframe:
           A    B           C
2020  100.00  100  100.000000
2021  101.00  100  102.000000
2022  106.05  100  104.040000
2023  100.00  100  106.120800
2024  100.00  100  108.243216
2025  100.00  100  110.408080

Growth:
             A    B    C
2020       NaN  NaN  NaN
2021  1.000000  0.0  2.0
2022  5.000000  0.0  2.0
2023 \PYGZhy{}5.704856  0.0  2.0
2024  0.000000  0.0  2.0
2025  0.000000  0.0  2.0
\end{sphinxVerbatim}

\end{sphinxuseclass}\end{sphinxVerbatimOutput}

\end{sphinxuseclass}

\subsection{Add or subtract from the existing percent growth rate (+GROWTH operator)}
\label{\detokenize{content/04_PythonEssentials/UpdateCommand:add-or-subtract-from-the-existing-percent-growth-rate-growth-operator}}
\sphinxAtStartPar
The below example is a bit more complicated.

\sphinxAtStartPar
The first line sets the growth rate of A to 1\% in all periods beginning in  2021

\sphinxAtStartPar
The second command adds 2 3 4 5 6 to the growth rates in each period after 2021, resulting in growth rates of 3,4,5,6,7.

\begin{sphinxuseclass}{cell}\begin{sphinxVerbatimInput}

\begin{sphinxuseclass}{cell_input}
\begin{sphinxVerbatim}[commandchars=\\\{\}]
\PYG{n}{res} \PYG{o}{=}\PYG{n}{df}\PYG{o}{.}\PYG{n}{upd}\PYG{p}{(}\PYG{l+s+s1}{\PYGZsq{}\PYGZsq{}\PYGZsq{}}
\PYG{l+s+s1}{\PYGZlt{}2021 \PYGZgt{} A =GROWTH  1  \PYGZsh{} All selected years set to the same growth rate}
\PYG{l+s+s1}{a +growth   2  \PYGZsh{} Add to the existing growth rate these numbers  }
\PYG{l+s+s1}{\PYGZsq{}\PYGZsq{}\PYGZsq{}}\PYG{p}{)}
\PYG{n+nb}{print}\PYG{p}{(}\PYG{l+s+sa}{f}\PYG{l+s+s1}{\PYGZsq{}}\PYG{l+s+s1}{Dataframe:}\PYG{l+s+se}{\PYGZbs{}n}\PYG{l+s+si}{\PYGZob{}}\PYG{n}{res}\PYG{l+s+si}{\PYGZcb{}}\PYG{l+s+se}{\PYGZbs{}n}\PYG{l+s+se}{\PYGZbs{}n}\PYG{l+s+s1}{Growth:}\PYG{l+s+se}{\PYGZbs{}n}\PYG{l+s+si}{\PYGZob{}}\PYG{n}{res}\PYG{o}{.}\PYG{n}{pct\PYGZus{}change}\PYG{p}{(}\PYG{p}{)}\PYG{o}{*}\PYG{l+m+mi}{100}\PYG{l+s+si}{\PYGZcb{}}\PYG{l+s+se}{\PYGZbs{}n}\PYG{l+s+s1}{\PYGZsq{}}\PYG{p}{)}
\end{sphinxVerbatim}

\end{sphinxuseclass}\end{sphinxVerbatimInput}
\begin{sphinxVerbatimOutput}

\begin{sphinxuseclass}{cell_output}
\begin{sphinxVerbatim}[commandchars=\\\{\}]
Dataframe:
        A    B
2020  100  100
2021  103  100
2022  100  100
2023  100  100
2024  100  100
2025  100  100

Growth:
             A    B
2020       NaN  NaN
2021  3.000000  0.0
2022 \PYGZhy{}2.912621  0.0
2023  0.000000  0.0
2024  0.000000  0.0
2025  0.000000  0.0
\end{sphinxVerbatim}

\end{sphinxuseclass}\end{sphinxVerbatimOutput}

\end{sphinxuseclass}

\subsection{Set the change in a variable to specific values (=diff operator)}
\label{\detokenize{content/04_PythonEssentials/UpdateCommand:set-the-change-in-a-variable-to-specific-values-diff-operator}}
\sphinxAtStartPar
\(\Delta = var_t - var_{t-1} = some number\)

\sphinxAtStartPar
Here sets the value of A in 2021 to 2 more than the value of 2020, and the 2022 value as 4 more than the \sphinxstylestrong{revised} value of 2021.

\sphinxAtStartPar
The second line creates a new variable “UPBY2” to the data frame and sets it equal to 100 for all periods,

\sphinxAtStartPar
The third line adds 2 to the previous periods value UPBY2.

\begin{sphinxuseclass}{cell}\begin{sphinxVerbatimInput}

\begin{sphinxuseclass}{cell_input}
\begin{sphinxVerbatim}[commandchars=\\\{\}]
\PYG{n}{df}\PYG{o}{.}\PYG{n}{upd}\PYG{p}{(}\PYG{l+s+s1}{\PYGZsq{}\PYGZsq{}\PYGZsq{}}
\PYG{l+s+s1}{\PYGZlt{} 2021 2022\PYGZgt{} A =diff  2 4   \PYGZsh{} Same number of values as years}
\PYG{l+s+s1}{\PYGZlt{}2020 \PYGZgt{} UpBy2 = 100 \PYGZsh{} sets rows equal to the same  number for all years in between start and end }
\PYG{l+s+s1}{\PYGZlt{}2021 2025\PYGZgt{} UpBy2 =diff  2  }

\PYG{l+s+s1}{\PYGZsq{}\PYGZsq{}\PYGZsq{}}\PYG{p}{)}
\end{sphinxVerbatim}

\end{sphinxuseclass}\end{sphinxVerbatimInput}
\begin{sphinxVerbatimOutput}

\begin{sphinxuseclass}{cell_output}
\begin{sphinxVerbatim}[commandchars=\\\{\}]
        A    B  UPBY2
2020  100  100  100.0
2021  102  100  102.0
2022  106  100  104.0
2023  100  100  106.0
2024  100  100  108.0
2025  100  100  110.0
\end{sphinxVerbatim}

\end{sphinxuseclass}\end{sphinxVerbatimOutput}

\end{sphinxuseclass}

\subsection{Recall  that we have not overwritten df, so the df dataframe is unchanged.}
\label{\detokenize{content/04_PythonEssentials/UpdateCommand:recall-that-we-have-not-overwritten-df-so-the-df-dataframe-is-unchanged}}
\begin{sphinxuseclass}{cell}\begin{sphinxVerbatimInput}

\begin{sphinxuseclass}{cell_input}
\begin{sphinxVerbatim}[commandchars=\\\{\}]
\PYG{n}{df}
\end{sphinxVerbatim}

\end{sphinxuseclass}\end{sphinxVerbatimInput}
\begin{sphinxVerbatimOutput}

\begin{sphinxuseclass}{cell_output}
\begin{sphinxVerbatim}[commandchars=\\\{\}]
        A    B
2020  100  100
2021  100  100
2022  100  100
2023  100  100
2024  100  100
2025  100  100
\end{sphinxVerbatim}

\end{sphinxuseclass}\end{sphinxVerbatimOutput}

\end{sphinxuseclass}
\begin{sphinxadmonition}{note}{Note:}
\sphinxAtStartPar
The method \sphinxcode{\sphinxupquote{.upd()}} only operates on on variable.  A command like \sphinxcode{\sphinxupquote{.upd('A = B')}} would not work. For these kind of functions, use \sphinxcode{\sphinxupquote{.mfcalc()}} (see next section).
\end{sphinxadmonition}


\subsection{Keep growth rates after the update time – the –kg option}
\label{\detokenize{content/04_PythonEssentials/UpdateCommand:keep-growth-rates-after-the-update-time-the-kg-option}}
\sphinxAtStartPar
In a long projection it can sometime be useful to be able to update variables for which new information is available, but for the subsequent periods keep the growth rate the same as before the update. In database management this is frequently done when two time\sphinxhyphen{}series with different levels are spliced together.

\sphinxAtStartPar
The \sphinxhyphen{}kg or –keep\_growth option instructs modelview to calculate the growth rate of the existing pre\sphinxhyphen{}change series, and then use it to preserve the pre\sphinxhyphen{}change growth rates of the series for the periods that were \sphinxstylestrong{not} changed.

\sphinxAtStartPar
This allows to update variables for which new information is available, but keep the growth rate the same as before the update in the period after the update time.


\subsubsection{The default keep\_growth behaviour}
\label{\detokenize{content/04_PythonEssentials/UpdateCommand:the-default-keep-growth-behaviour}}
\sphinxAtStartPar
The \sphinxcode{\sphinxupquote{upd()}} method has a parameter \sphinxcode{\sphinxupquote{keep\_growth}}, which by default is equal to \sphinxcode{\sphinxupquote{False}}.

\sphinxAtStartPar
\sphinxcode{\sphinxupquote{keep\_growth}} determines how data in  the time periods after those where an update is executed are treated.

\sphinxAtStartPar
If \sphinxcode{\sphinxupquote{keep\_growth}} is \sphinxcode{\sphinxupquote{False}} then data in the sub\sphinxhyphen{}period after a change is left unchanged.

\sphinxAtStartPar
if \sphinxcode{\sphinxupquote{keep\_growth}} is set to “\sphinxcode{\sphinxupquote{True}}” then the system will preserve the pre\sphinxhyphen{}change growth rate of the affected variable in the time period \sphinxstyleemphasis{after the change}.

\begin{sphinxadmonition}{note}{Note:}
\sphinxAtStartPar
At the line level:
\begin{itemize}
\item {} 
\sphinxAtStartPar
\sphinxcode{\sphinxupquote{keep\_growth=True}} can be expressed as –kg

\item {} 
\sphinxAtStartPar
\sphinxcode{\sphinxupquote{keep\_growth=False}} can be expressed as –nkg

\end{itemize}
\end{sphinxadmonition}

\sphinxAtStartPar
Let’s see this in a concrete example.  Consider the following \sphinxcode{\sphinxupquote{dataframe}} df with two variables A and B, that each grow by 2\% per period, with A initialized at a level of 100 and B at a level of 110 so that we can see each separately on a graph.

\begin{sphinxuseclass}{cell}\begin{sphinxVerbatimInput}

\begin{sphinxuseclass}{cell_input}
\begin{sphinxVerbatim}[commandchars=\\\{\}]
\PYG{n}{df} \PYG{o}{=} \PYG{n}{pd}\PYG{o}{.}\PYG{n}{DataFrame}\PYG{p}{(}\PYG{l+m+mi}{100}\PYG{p}{,}
       \PYG{n}{index}\PYG{o}{=}\PYG{p}{[}\PYG{l+m+mi}{2020}\PYG{o}{+}\PYG{n}{v} \PYG{k}{for} \PYG{n}{v} \PYG{o+ow}{in} \PYG{n+nb}{range}\PYG{p}{(}\PYG{n}{number\PYGZus{}of\PYGZus{}rows}\PYG{p}{)}\PYG{p}{]}\PYG{p}{,} \PYG{c+c1}{\PYGZsh{} create row index}
       \PYG{c+c1}{\PYGZsh{} equivalent to index=[2020,2021,2022,2023,2024,2025] }
       \PYG{n}{columns}\PYG{o}{=}\PYG{p}{[}\PYG{l+s+s1}{\PYGZsq{}}\PYG{l+s+s1}{A}\PYG{l+s+s1}{\PYGZsq{}}\PYG{p}{,}\PYG{l+s+s1}{\PYGZsq{}}\PYG{l+s+s1}{B}\PYG{l+s+s1}{\PYGZsq{}}\PYG{p}{]}\PYG{p}{)} 

\PYG{n}{df}\PYG{o}{=}\PYG{n}{df}\PYG{o}{.}\PYG{n}{upd}\PYG{p}{(}\PYG{l+s+s2}{\PYGZdq{}\PYGZdq{}\PYGZdq{}}\PYG{l+s+s2}{\PYGZlt{}2021 \PYGZhy{}1\PYGZgt{} A =growth 2}
\PYG{l+s+s2}{           \PYGZlt{}2020 \PYGZhy{}1\PYGZgt{}   B = 110}
\PYG{l+s+s2}{          \PYGZlt{}2021 \PYGZhy{}1\PYGZgt{}    B =growth 2}
\PYG{l+s+s2}{          }\PYG{l+s+s2}{\PYGZdq{}\PYGZdq{}\PYGZdq{}}\PYG{p}{)}
\PYG{c+c1}{\PYGZsh{} Store these variables for later use in comparisons}
\PYG{n}{df}\PYG{p}{[}\PYG{l+s+s1}{\PYGZsq{}}\PYG{l+s+s1}{A\PYGZus{}ORIG}\PYG{l+s+s1}{\PYGZsq{}}\PYG{p}{]}\PYG{o}{=}\PYG{n}{df}\PYG{p}{[}\PYG{l+s+s1}{\PYGZsq{}}\PYG{l+s+s1}{A}\PYG{l+s+s1}{\PYGZsq{}}\PYG{p}{]}
\PYG{n}{df}\PYG{p}{[}\PYG{l+s+s1}{\PYGZsq{}}\PYG{l+s+s1}{B\PYGZus{}ORIG}\PYG{l+s+s1}{\PYGZsq{}}\PYG{p}{]}\PYG{o}{=}\PYG{n}{df}\PYG{p}{[}\PYG{l+s+s1}{\PYGZsq{}}\PYG{l+s+s1}{B}\PYG{l+s+s1}{\PYGZsq{}}\PYG{p}{]}
\PYG{n}{df}
\end{sphinxVerbatim}

\end{sphinxuseclass}\end{sphinxVerbatimInput}
\begin{sphinxVerbatimOutput}

\begin{sphinxuseclass}{cell_output}
\begin{sphinxVerbatim}[commandchars=\\\{\}]
               A           B      A\PYGZus{}ORIG      B\PYGZus{}ORIG
2020  100.000000  110.000000  100.000000  110.000000
2021  102.000000  112.200000  102.000000  112.200000
2022  104.040000  114.444000  104.040000  114.444000
2023  106.120800  116.732880  106.120800  116.732880
2024  108.243216  119.067538  108.243216  119.067538
2025  110.408080  121.448888  110.408080  121.448888
\end{sphinxVerbatim}

\end{sphinxuseclass}\end{sphinxVerbatimOutput}

\end{sphinxuseclass}
\begin{sphinxuseclass}{cell}\begin{sphinxVerbatimInput}

\begin{sphinxuseclass}{cell_input}
\begin{sphinxVerbatim}[commandchars=\\\{\}]
\PYG{n}{df}\PYG{p}{[}\PYG{p}{[}\PYG{l+s+s1}{\PYGZsq{}}\PYG{l+s+s1}{A}\PYG{l+s+s1}{\PYGZsq{}}\PYG{p}{,}\PYG{l+s+s1}{\PYGZsq{}}\PYG{l+s+s1}{B}\PYG{l+s+s1}{\PYGZsq{}}\PYG{p}{]}\PYG{p}{]}\PYG{o}{.}\PYG{n}{plot}\PYG{p}{(}\PYG{p}{)}\PYG{p}{;}
\end{sphinxVerbatim}

\end{sphinxuseclass}\end{sphinxVerbatimInput}
\begin{sphinxVerbatimOutput}

\begin{sphinxuseclass}{cell_output}
\noindent\sphinxincludegraphics{{c02b4af773579f433eace4f785e543742d5ef841906e6b994877f90732f66fd1}.png}

\end{sphinxuseclass}\end{sphinxVerbatimOutput}

\end{sphinxuseclass}
\sphinxAtStartPar
Now lets modify each by adding 5 to the level in 2022 and 2023.
\begin{itemize}
\item {} 
\sphinxAtStartPar
For ‘B’ we set the keep\_growth option as False

\item {} 
\sphinxAtStartPar
for ‘A’ we set the keep\_growth True.

\end{itemize}

\begin{sphinxuseclass}{cell}\begin{sphinxVerbatimInput}

\begin{sphinxuseclass}{cell_input}
\begin{sphinxVerbatim}[commandchars=\\\{\}]
\PYG{n}{df}\PYG{o}{=}\PYG{n}{df}\PYG{o}{.}\PYG{n}{upd}\PYG{p}{(}\PYG{l+s+s2}{\PYGZdq{}\PYGZdq{}\PYGZdq{}}
\PYG{l+s+s2}{            \PYGZlt{}2022 2023\PYGZgt{} A + 5 \PYGZhy{}\PYGZhy{}kg}
\PYG{l+s+s2}{            \PYGZlt{}2022 2023\PYGZgt{} B + 5 \PYGZhy{}\PYGZhy{}nkg}
\PYG{l+s+s2}{            }\PYG{l+s+s2}{\PYGZdq{}\PYGZdq{}\PYGZdq{}}\PYG{p}{)}

\PYG{n}{df}\PYG{p}{[}\PYG{p}{[}\PYG{l+s+s1}{\PYGZsq{}}\PYG{l+s+s1}{A}\PYG{l+s+s1}{\PYGZsq{}}\PYG{p}{,}\PYG{l+s+s1}{\PYGZsq{}}\PYG{l+s+s1}{B}\PYG{l+s+s1}{\PYGZsq{}}\PYG{p}{,}\PYG{l+s+s1}{\PYGZsq{}}\PYG{l+s+s1}{A\PYGZus{}ORIG}\PYG{l+s+s1}{\PYGZsq{}}\PYG{p}{,}\PYG{l+s+s1}{\PYGZsq{}}\PYG{l+s+s1}{B\PYGZus{}ORIG}\PYG{l+s+s1}{\PYGZsq{}}\PYG{p}{]}\PYG{p}{]}\PYG{o}{.}\PYG{n}{plot}\PYG{p}{(}\PYG{p}{)}\PYG{p}{;}
    
\end{sphinxVerbatim}

\end{sphinxuseclass}\end{sphinxVerbatimInput}
\begin{sphinxVerbatimOutput}

\begin{sphinxuseclass}{cell_output}
\noindent\sphinxincludegraphics{{b190fe24d2cb993f17006d33ba0fb9966a5860917c21cee0cbc7a3ba4322c9df}.png}

\end{sphinxuseclass}\end{sphinxVerbatimOutput}

\end{sphinxuseclass}
\sphinxAtStartPar
In the first example ‘A’ (the green and blue lines) the level of A is increased by 5 for two periods (2021\sphinxhyphen{}2022). The subsequent values are also increased and they were calculated to maintain the growth rate of the original series.

\sphinxAtStartPar
For the ‘B’ variable the same level change was input but because of the \sphinxcode{\sphinxupquote{\sphinxhyphen{}\sphinxhyphen{}nkg}} (equivalent to \sphinxcode{\sphinxupquote{keep\_growth=False}}) the periods after the change were unaffected and retained their old values.

\sphinxAtStartPar
Below are plots the growth rates of the two transformed series.

\sphinxAtStartPar
Here the growth in both series accelerates in 2022, by slightly less than 5 percentage points because a) the base of each is more than 100, with the base of B being higher (it was initialized at 110). In 2023 the growth rate of A returns to 2 percent, while the growth rate of B is actually negative because the level (see earlier graph) has fallen back to its original level.

\begin{sphinxuseclass}{cell}\begin{sphinxVerbatimInput}

\begin{sphinxuseclass}{cell_input}
\begin{sphinxVerbatim}[commandchars=\\\{\}]
\PYG{n}{dfg}\PYG{o}{=}\PYG{n}{df}\PYG{p}{[}\PYG{p}{[}\PYG{l+s+s1}{\PYGZsq{}}\PYG{l+s+s1}{A}\PYG{l+s+s1}{\PYGZsq{}}\PYG{p}{,}\PYG{l+s+s1}{\PYGZsq{}}\PYG{l+s+s1}{B}\PYG{l+s+s1}{\PYGZsq{}}\PYG{p}{]}\PYG{p}{]}\PYG{o}{.}\PYG{n}{pct\PYGZus{}change}\PYG{p}{(}\PYG{p}{)}\PYG{o}{*}\PYG{l+m+mi}{100}
\PYG{n}{dfg}\PYG{o}{.}\PYG{n}{plot}\PYG{p}{(}\PYG{p}{)}\PYG{p}{;}
\end{sphinxVerbatim}

\end{sphinxuseclass}\end{sphinxVerbatimInput}
\begin{sphinxVerbatimOutput}

\begin{sphinxuseclass}{cell_output}
\noindent\sphinxincludegraphics{{93fc87a8e8ede996fa596d14d140f87c73411dfd2bd0aa8433f137f2597a0b85}.png}

\end{sphinxuseclass}\end{sphinxVerbatimOutput}

\end{sphinxuseclass}

\subsection{.upd(,,,keep\_growth) some more examples}
\label{\detokenize{content/04_PythonEssentials/UpdateCommand:upd-keep-growth-some-more-examples}}

\subsection{Initialize a new dataframe First make a dataframe with some growth rate}
\label{\detokenize{content/04_PythonEssentials/UpdateCommand:initialize-a-new-dataframe-first-make-a-dataframe-with-some-growth-rate}}
\begin{sphinxuseclass}{cell}\begin{sphinxVerbatimInput}

\begin{sphinxuseclass}{cell_input}
\begin{sphinxVerbatim}[commandchars=\\\{\}]
\PYG{c+c1}{\PYGZsh{} instantiate a new dataframe with one column \PYGZsq{}A\PYGZsq{} with value 100 everywhere and index 2020\PYGZhy{}2025}
\PYG{n}{dftest} \PYG{o}{=} \PYG{n}{pd}\PYG{o}{.}\PYG{n}{DataFrame}\PYG{p}{(}\PYG{l+m+mi}{100}\PYG{p}{,}
       \PYG{n}{index}\PYG{o}{=}\PYG{p}{[}\PYG{l+m+mi}{2020}\PYG{o}{+}\PYG{n}{v} \PYG{k}{for} \PYG{n}{v} \PYG{o+ow}{in} \PYG{n+nb}{range}\PYG{p}{(}\PYG{n}{number\PYGZus{}of\PYGZus{}rows}\PYG{p}{)}\PYG{p}{]}\PYG{p}{,} \PYG{c+c1}{\PYGZsh{} create row index}
       \PYG{c+c1}{\PYGZsh{} equivalent to index=[2020,2021,2022,2023,2024,2025] }
       \PYG{n}{columns}\PYG{o}{=}\PYG{p}{[}\PYG{l+s+s1}{\PYGZsq{}}\PYG{l+s+s1}{A}\PYG{l+s+s1}{\PYGZsq{}}\PYG{p}{]}\PYG{p}{)}                                 \PYG{c+c1}{\PYGZsh{} create column name }

\PYG{c+c1}{\PYGZsh{} Update a to have growth rate accelerationg linearly by 1 from 1 percent to 5 percent}
\PYG{n}{original} \PYG{o}{=} \PYG{n}{dftest}\PYG{o}{.}\PYG{n}{upd}\PYG{p}{(}\PYG{l+s+s1}{\PYGZsq{}}\PYG{l+s+s1}{\PYGZlt{}2021 2025\PYGZgt{} a =growth 1 2 3 4 5}\PYG{l+s+s1}{\PYGZsq{}}\PYG{p}{)}  
\PYG{n+nb}{print}\PYG{p}{(}\PYG{l+s+sa}{f}\PYG{l+s+s1}{\PYGZsq{}}\PYG{l+s+s1}{Levels:}\PYG{l+s+se}{\PYGZbs{}n}\PYG{l+s+si}{\PYGZob{}}\PYG{n}{original}\PYG{l+s+si}{\PYGZcb{}}\PYG{l+s+se}{\PYGZbs{}n}\PYG{l+s+se}{\PYGZbs{}n}\PYG{l+s+s1}{Growth:}\PYG{l+s+se}{\PYGZbs{}n}\PYG{l+s+si}{\PYGZob{}}\PYG{n}{original}\PYG{o}{.}\PYG{n}{pct\PYGZus{}change}\PYG{p}{(}\PYG{p}{)}\PYG{o}{*}\PYG{l+m+mi}{100}\PYG{l+s+si}{\PYGZcb{}}\PYG{l+s+se}{\PYGZbs{}n}\PYG{l+s+s1}{\PYGZsq{}}\PYG{p}{)}
\end{sphinxVerbatim}

\end{sphinxuseclass}\end{sphinxVerbatimInput}
\begin{sphinxVerbatimOutput}

\begin{sphinxuseclass}{cell_output}
\begin{sphinxVerbatim}[commandchars=\\\{\}]
Levels:
               A
2020  100.000000
2021  101.000000
2022  103.020000
2023  106.110600
2024  110.355024
2025  115.872775

Growth:
        A
2020  NaN
2021  1.0
2022  2.0
2023  3.0
2024  4.0
2025  5.0
\end{sphinxVerbatim}

\end{sphinxuseclass}\end{sphinxVerbatimOutput}

\end{sphinxuseclass}

\subsection{now update A in 2021 to 2023 to a new value}
\label{\detokenize{content/04_PythonEssentials/UpdateCommand:now-update-a-in-2021-to-2023-to-a-new-value}}
\sphinxAtStartPar
Below performs the same operation, the first time the updated value is assigned to the \sphinxcode{\sphinxupquote{dataframe}} \sphinxcode{\sphinxupquote{nkg}} and the default behaviour of \sphinxcode{\sphinxupquote{keep\_growth}} is \sphinxcode{\sphinxupquote{False}}

\sphinxAtStartPar
In the second example the \sphinxcode{\sphinxupquote{\sphinxhyphen{}kg}} line option is specified, telling modelflow to maintain the growth rates of the dependent variable in the periods after the update is executed.

\begin{sphinxuseclass}{cell}\begin{sphinxVerbatimInput}

\begin{sphinxuseclass}{cell_input}
\begin{sphinxVerbatim}[commandchars=\\\{\}]
\PYG{n}{nokg} \PYG{o}{=} \PYG{n}{original}\PYG{o}{.}\PYG{n}{upd}\PYG{p}{(}\PYG{l+s+s1}{\PYGZsq{}\PYGZsq{}\PYGZsq{}}
\PYG{l+s+s1}{\PYGZlt{}2021 2025\PYGZgt{}  a =growth 1 2 3 4 5 }
\PYG{l+s+s1}{\PYGZlt{}2021 2023\PYGZgt{}  a = 120  }
\PYG{l+s+s1}{\PYGZsq{}\PYGZsq{}\PYGZsq{}}\PYG{p}{,}\PYG{n}{lprint}\PYG{o}{=}\PYG{l+m+mi}{0}\PYG{p}{)}

\PYG{n}{kg} \PYG{o}{=} \PYG{n}{original}\PYG{o}{.}\PYG{n}{upd}\PYG{p}{(}\PYG{l+s+s1}{\PYGZsq{}\PYGZsq{}\PYGZsq{}}
\PYG{l+s+s1}{\PYGZlt{}2021 2025\PYGZgt{}  a =growth 1 2 3 4 5 }
\PYG{l+s+s1}{\PYGZlt{}2021 2023\PYGZgt{}  a = 120  \PYGZhy{}\PYGZhy{}kg}
\PYG{l+s+s1}{\PYGZsq{}\PYGZsq{}\PYGZsq{}}\PYG{p}{,}\PYG{n}{lprint}\PYG{o}{=}\PYG{l+m+mi}{0}\PYG{p}{)}


\PYG{n}{kg}\PYG{o}{=}\PYG{n}{kg}\PYG{o}{.}\PYG{n}{rename}\PYG{p}{(}\PYG{n}{columns}\PYG{o}{=}\PYG{p}{\PYGZob{}}\PYG{l+s+s2}{\PYGZdq{}}\PYG{l+s+s2}{A}\PYG{l+s+s2}{\PYGZdq{}}\PYG{p}{:}\PYG{l+s+s2}{\PYGZdq{}}\PYG{l+s+s2}{KG}\PYG{l+s+s2}{\PYGZdq{}}\PYG{p}{\PYGZcb{}}\PYG{p}{)}       \PYG{c+c1}{\PYGZsh{}rename cols to facilitate display}
\PYG{n}{nokg}\PYG{o}{=}\PYG{n}{nokg}\PYG{o}{.}\PYG{n}{rename}\PYG{p}{(}\PYG{n}{columns}\PYG{o}{=}\PYG{p}{\PYGZob{}}\PYG{l+s+s2}{\PYGZdq{}}\PYG{l+s+s2}{A}\PYG{l+s+s2}{\PYGZdq{}}\PYG{p}{:}\PYG{l+s+s2}{\PYGZdq{}}\PYG{l+s+s2}{NOKG}\PYG{l+s+s2}{\PYGZdq{}}\PYG{p}{\PYGZcb{}}\PYG{p}{)} \PYG{c+c1}{\PYGZsh{}rename cols to facilitate display}

\PYG{n}{combo}\PYG{o}{=}\PYG{n}{pd}\PYG{o}{.}\PYG{n}{concat}\PYG{p}{(}\PYG{p}{[}\PYG{n}{kg}\PYG{p}{,}\PYG{n}{nokg}\PYG{p}{]}\PYG{p}{,} \PYG{n}{axis}\PYG{o}{=}\PYG{l+m+mi}{1}\PYG{p}{)}
\PYG{n}{combo}


\PYG{n+nb}{print}\PYG{p}{(}\PYG{l+s+sa}{f}\PYG{l+s+s1}{\PYGZsq{}}\PYG{l+s+s1}{Levels}\PYG{l+s+se}{\PYGZbs{}n}\PYG{l+s+si}{\PYGZob{}}\PYG{n}{combo}\PYG{l+s+si}{\PYGZcb{}}\PYG{l+s+se}{\PYGZbs{}n}\PYG{l+s+se}{\PYGZbs{}n}\PYG{l+s+s1}{Growth}\PYG{l+s+se}{\PYGZbs{}n}\PYG{l+s+si}{\PYGZob{}}\PYG{n}{combo}\PYG{o}{.}\PYG{n}{pct\PYGZus{}change}\PYG{p}{(}\PYG{p}{)}\PYG{o}{*}\PYG{l+m+mi}{100}\PYG{l+s+si}{\PYGZcb{}}\PYG{l+s+s1}{\PYGZsq{}}\PYG{p}{)}
\end{sphinxVerbatim}

\end{sphinxuseclass}\end{sphinxVerbatimInput}
\begin{sphinxVerbatimOutput}

\begin{sphinxuseclass}{cell_output}
\begin{sphinxVerbatim}[commandchars=\\\{\}]
Levels
          KG        NOKG
2020  100.00  100.000000
2021  120.00  120.000000
2022  120.00  120.000000
2023  120.00  120.000000
2024  124.80  110.355024
2025  131.04  115.872775

Growth
        KG      NOKG
2020   NaN       NaN
2021  20.0  20.00000
2022   0.0   0.00000
2023   0.0   0.00000
2024   4.0  \PYGZhy{}8.03748
2025   5.0   5.00000
\end{sphinxVerbatim}

\end{sphinxuseclass}\end{sphinxVerbatimOutput}

\end{sphinxuseclass}
\begin{sphinxadmonition}{note}{Note:}
\sphinxAtStartPar
In the first example where KG (keep\_growth) \sphinxstylestrong{was not set}, because the level was set constant for three periods at 120 the rate of growth was 0 for the final two years of the set period.  But following this update, the level of A in 2023 is 120. With \sphinxcode{\sphinxupquote{keep\_Growth=False}} (its default value)m the level of A in 2024 remains at its unchanged  unchanged (lower) level of 100.35. As a result, the growth rate in 2024 is negative.

\sphinxAtStartPar
In the \sphinxstylestrong{–kg} example, the pre\sphinxhyphen{}exsting growth rate (of 4\%) is applied to the new value of 120 and so the level in 2024 is (120*1.04)=124.8 and 2025 is 131.04.
\end{sphinxadmonition}


\subsubsection{.upd() with the option keep\_growth set globally}
\label{\detokenize{content/04_PythonEssentials/UpdateCommand:upd-with-the-option-keep-growth-set-globally}}
\sphinxAtStartPar
Above the line level option \sphinxcode{\sphinxupquote{\sphinxhyphen{}\sphinxhyphen{}keep\_growth}} or \sphinxcode{\sphinxupquote{\sphinxhyphen{}\sphinxhyphen{}kg}} was used to keep the growth rate(or not) for a given operation.

\sphinxAtStartPar
This works because by default the option \sphinxcode{\sphinxupquote{Keep\_growth}} is set to false, implementing \sphinxcode{\sphinxupquote{\sphinxhyphen{}\sphinxhyphen{}kg}} at the line level temporarily set the keep\_growth flag to  true for the specific line (and those following).

\sphinxAtStartPar
The \sphinxcode{\sphinxupquote{keep\_growth}} flag can also be set globally for all the lines by setting the option in the command line.

\sphinxAtStartPar
\sphinxcode{\sphinxupquote{keep\_growth=True}}.

\sphinxAtStartPar
Now as default, all lines will keep the growth rate (unless overridden at the line level with \sphinxcode{\sphinxupquote{\sphinxhyphen{}\sphinxhyphen{}nkg}} or \sphinxcode{\sphinxupquote{\sphinxhyphen{}\sphinxhyphen{}no\_keep\_growth}}.
\begin{itemize}
\item {} 
\sphinxAtStartPar
c,d are updated in 2022 and 2023 and keep the growth rates afterwards

\item {} 
\sphinxAtStartPar
e the \sphinxcode{\sphinxupquote{\sphinxhyphen{}\sphinxhyphen{}no\_keep\_growth}} in this line prevents the updating 2024\sphinxhyphen{}2025

\end{itemize}

\begin{sphinxuseclass}{cell}\begin{sphinxVerbatimInput}

\begin{sphinxuseclass}{cell_input}
\begin{sphinxVerbatim}[commandchars=\\\{\}]
\PYG{c+c1}{\PYGZsh{} Create a data frame}
\PYG{n}{dftest} \PYG{o}{=} \PYG{n}{pd}\PYG{o}{.}\PYG{n}{DataFrame}\PYG{p}{(}\PYG{l+m+mi}{100}\PYG{p}{,}
       \PYG{n}{index}\PYG{o}{=}\PYG{p}{[}\PYG{l+m+mi}{2020}\PYG{o}{+}\PYG{n}{v} \PYG{k}{for} \PYG{n}{v} \PYG{o+ow}{in} \PYG{n+nb}{range}\PYG{p}{(}\PYG{n}{number\PYGZus{}of\PYGZus{}rows}\PYG{p}{)}\PYG{p}{]}\PYG{p}{,} \PYG{c+c1}{\PYGZsh{} create row index}
       \PYG{c+c1}{\PYGZsh{} equivalent to index=[2020,2021,2022,2023,2024,2025] }
       \PYG{n}{columns}\PYG{o}{=}\PYG{p}{[}\PYG{l+s+s1}{\PYGZsq{}}\PYG{l+s+s1}{A}\PYG{l+s+s1}{\PYGZsq{}}\PYG{p}{,}\PYG{l+s+s1}{\PYGZsq{}}\PYG{l+s+s1}{B}\PYG{l+s+s1}{\PYGZsq{}}\PYG{p}{,}\PYG{l+s+s1}{\PYGZsq{}}\PYG{l+s+s1}{C}\PYG{l+s+s1}{\PYGZsq{}}\PYG{p}{,}\PYG{l+s+s1}{\PYGZsq{}}\PYG{l+s+s1}{D}\PYG{l+s+s1}{\PYGZsq{}}\PYG{p}{,}\PYG{l+s+s1}{\PYGZsq{}}\PYG{l+s+s1}{E}\PYG{l+s+s1}{\PYGZsq{}}\PYG{p}{]}\PYG{p}{)}                                 \PYG{c+c1}{\PYGZsh{} create column name }
\PYG{n}{df}
\end{sphinxVerbatim}

\end{sphinxuseclass}\end{sphinxVerbatimInput}
\begin{sphinxVerbatimOutput}

\begin{sphinxuseclass}{cell_output}
\begin{sphinxVerbatim}[commandchars=\\\{\}]
               A           B      A\PYGZus{}ORIG      B\PYGZus{}ORIG
2020  100.000000  110.000000  100.000000  110.000000
2021  102.000000  112.200000  102.000000  112.200000
2022  109.040000  119.444000  104.040000  114.444000
2023  111.120800  121.732880  106.120800  116.732880
2024  113.343216  119.067538  108.243216  119.067538
2025  115.610080  121.448888  110.408080  121.448888
\end{sphinxVerbatim}

\end{sphinxuseclass}\end{sphinxVerbatimOutput}

\end{sphinxuseclass}
\begin{sphinxuseclass}{cell}\begin{sphinxVerbatimInput}

\begin{sphinxuseclass}{cell_input}
\begin{sphinxVerbatim}[commandchars=\\\{\}]
\PYG{n}{dfres} \PYG{o}{=} \PYG{n}{dftest}\PYG{o}{.}\PYG{n}{upd}\PYG{p}{(}\PYG{l+s+s1}{\PYGZsq{}\PYGZsq{}\PYGZsq{}}
\PYG{l+s+s1}{\PYGZlt{}2022 2023\PYGZgt{} c = 200 }
\PYG{l+s+s1}{\PYGZlt{}2022 2023\PYGZgt{} d = 300  }
\PYG{l+s+s1}{\PYGZlt{}2022 2023\PYGZgt{} e = 400  \PYGZhy{}\PYGZhy{}no\PYGZus{}keep\PYGZus{}growth }
\PYG{l+s+s1}{\PYGZsq{}\PYGZsq{}\PYGZsq{}}\PYG{p}{,}\PYG{n}{keep\PYGZus{}growth}\PYG{o}{=}\PYG{k+kc}{True}\PYG{p}{)}  \PYG{c+c1}{\PYGZsh{} \PYGZlt{}=  Set keep\PYGZus{}growth to True for the entirety of the command, }
                       \PYG{c+c1}{\PYGZsh{} except for e where it is overridden by the \PYGZhy{}\PYGZhy{}no\PYGZus{}keep\PYGZus{}growth flag}
\PYG{n+nb}{print}\PYG{p}{(}\PYG{l+s+sa}{f}\PYG{l+s+s1}{\PYGZsq{}}\PYG{l+s+s1}{Dataframe:}\PYG{l+s+se}{\PYGZbs{}n}\PYG{l+s+si}{\PYGZob{}}\PYG{n}{dfres}\PYG{l+s+si}{\PYGZcb{}}\PYG{l+s+se}{\PYGZbs{}n}\PYG{l+s+se}{\PYGZbs{}n}\PYG{l+s+s1}{Growth:}\PYG{l+s+se}{\PYGZbs{}n}\PYG{l+s+si}{\PYGZob{}}\PYG{n}{dfres}\PYG{o}{.}\PYG{n}{pct\PYGZus{}change}\PYG{p}{(}\PYG{p}{)}\PYG{o}{*}\PYG{l+m+mi}{100}\PYG{l+s+si}{\PYGZcb{}}\PYG{l+s+se}{\PYGZbs{}n}\PYG{l+s+s1}{\PYGZsq{}}\PYG{p}{)}
\end{sphinxVerbatim}

\end{sphinxuseclass}\end{sphinxVerbatimInput}
\begin{sphinxVerbatimOutput}

\begin{sphinxuseclass}{cell_output}
\begin{sphinxVerbatim}[commandchars=\\\{\}]
Dataframe:
        A    B      C      D    E
2020  100  100  100.0  100.0  100
2021  100  100  100.0  100.0  100
2022  100  100  200.0  300.0  400
2023  100  100  200.0  300.0  400
2024  100  100  200.0  300.0  100
2025  100  100  200.0  300.0  100

Growth:
        A    B      C      D      E
2020  NaN  NaN    NaN    NaN    NaN
2021  0.0  0.0    0.0    0.0    0.0
2022  0.0  0.0  100.0  200.0  300.0
2023  0.0  0.0    0.0    0.0    0.0
2024  0.0  0.0    0.0    0.0  \PYGZhy{}75.0
2025  0.0  0.0    0.0    0.0    0.0
\end{sphinxVerbatim}

\end{sphinxuseclass}\end{sphinxVerbatimOutput}

\end{sphinxuseclass}

\subsection{Update several variable in one line}
\label{\detokenize{content/04_PythonEssentials/UpdateCommand:update-several-variable-in-one-line}}
\sphinxAtStartPar
Sometime there is a need to update several variable with the same value over the same time frame. To ease this case .update can accept several variables in one line

\begin{sphinxuseclass}{cell}\begin{sphinxVerbatimInput}

\begin{sphinxuseclass}{cell_input}
\begin{sphinxVerbatim}[commandchars=\\\{\}]
\PYG{n}{df}\PYG{o}{.}\PYG{n}{upd}\PYG{p}{(}\PYG{l+s+s1}{\PYGZsq{}\PYGZsq{}\PYGZsq{}}
\PYG{l+s+s1}{\PYGZlt{}2022 2024\PYGZgt{} h i j k =      40      \PYGZsh{} earlier values are set to zero by default}
\PYG{l+s+s1}{\PYGZlt{}2020\PYGZgt{}      p q r s =       1000   \PYGZsh{} All values beginning in 2020 set to 1000}
\PYG{l+s+s1}{\PYGZlt{}2021 \PYGZhy{}1\PYGZgt{}   p q r s =growth 2      \PYGZsh{} \PYGZhy{}1 indicates the last year of dataframe}
\PYG{l+s+s1}{\PYGZsq{}\PYGZsq{}\PYGZsq{}}\PYG{p}{)}
\end{sphinxVerbatim}

\end{sphinxuseclass}\end{sphinxVerbatimInput}
\begin{sphinxVerbatimOutput}

\begin{sphinxuseclass}{cell_output}
\begin{sphinxVerbatim}[commandchars=\\\{\}]
               A           B      A\PYGZus{}ORIG      B\PYGZus{}ORIG     H     I     J     K   
2020  100.000000  110.000000  100.000000  110.000000   0.0   0.0   0.0   0.0  \PYGZbs{}
2021  102.000000  112.200000  102.000000  112.200000   0.0   0.0   0.0   0.0   
2022  109.040000  119.444000  104.040000  114.444000  40.0  40.0  40.0  40.0   
2023  111.120800  121.732880  106.120800  116.732880  40.0  40.0  40.0  40.0   
2024  113.343216  119.067538  108.243216  119.067538  40.0  40.0  40.0  40.0   
2025  115.610080  121.448888  110.408080  121.448888   0.0   0.0   0.0   0.0   

                P            Q            R            S  
2020  1000.000000  1000.000000  1000.000000  1000.000000  
2021  1020.000000  1020.000000  1020.000000  1020.000000  
2022  1040.400000  1040.400000  1040.400000  1040.400000  
2023  1061.208000  1061.208000  1061.208000  1061.208000  
2024  1082.432160  1082.432160  1082.432160  1082.432160  
2025  1104.080803  1104.080803  1104.080803  1104.080803  
\end{sphinxVerbatim}

\end{sphinxuseclass}\end{sphinxVerbatimOutput}

\end{sphinxuseclass}

\subsection{.upd(,,scale=<number, default=1>) Scale the updates}
\label{\detokenize{content/04_PythonEssentials/UpdateCommand:upd-scale-number-default-1-scale-the-updates}}
\sphinxAtStartPar
When running a scenario it can be useful to be able to create a number of scenarios based on one update but with different scale.

\sphinxAtStartPar
This can be particularly useful when we want to do sensitivity analyses of model results, depending on how heavily a shocked variable is hit

\sphinxAtStartPar
When using the scale option, scale=0  the baseline while scale=0.5 is a scenario half
the severity.

\sphinxAtStartPar
In the example below the values of the dataframes are printed. We use the scale option (setting to to 0, 0.5 and 1) to run three scenarios using the same code but where the update in each case is multiplied by either 0, 0.5 or 1.

\begin{sphinxadmonition}{note}{Note:}
\sphinxAtStartPar
Here we are just printing the outputs, a more interesting example would involve the solving a  model using different levels of a given shock.
\end{sphinxadmonition}

\begin{sphinxuseclass}{cell}\begin{sphinxVerbatimInput}

\begin{sphinxuseclass}{cell_input}
\begin{sphinxVerbatim}[commandchars=\\\{\}]
\PYG{n+nb}{print}\PYG{p}{(}\PYG{l+s+sa}{f}\PYG{l+s+s1}{\PYGZsq{}}\PYG{l+s+s1}{input dataframe: }\PYG{l+s+se}{\PYGZbs{}n}\PYG{l+s+si}{\PYGZob{}}\PYG{n}{df}\PYG{l+s+si}{\PYGZcb{}}\PYG{l+s+se}{\PYGZbs{}n}\PYG{l+s+se}{\PYGZbs{}n}\PYG{l+s+s1}{\PYGZsq{}}\PYG{p}{)}
\PYG{k}{for} \PYG{n}{severity} \PYG{o+ow}{in} \PYG{p}{[}\PYG{l+m+mi}{0}\PYG{p}{,}\PYG{l+m+mf}{0.5}\PYG{p}{,}\PYG{l+m+mi}{1}\PYG{p}{]}\PYG{p}{:} 
    \PYG{c+c1}{\PYGZsh{} First make a dataframe with some growth rate }
    \PYG{n}{res} \PYG{o}{=} \PYG{n}{df}\PYG{o}{.}\PYG{n}{upd}\PYG{p}{(}\PYG{l+s+s1}{\PYGZsq{}\PYGZsq{}\PYGZsq{}}
\PYG{l+s+s1}{    \PYGZlt{}2021 2025\PYGZgt{}}
\PYG{l+s+s1}{    a =growth 1 2 3 4 5 }
\PYG{l+s+s1}{    b + 10}
\PYG{l+s+s1}{    }\PYG{l+s+s1}{\PYGZsq{}\PYGZsq{}\PYGZsq{}}\PYG{p}{,}\PYG{n}{scale}\PYG{o}{=}\PYG{n}{severity}\PYG{p}{)}
    \PYG{n+nb}{print}\PYG{p}{(}\PYG{l+s+sa}{f}\PYG{l+s+s1}{\PYGZsq{}}\PYG{l+s+si}{\PYGZob{}}\PYG{n}{severity}\PYG{l+s+si}{=\PYGZcb{}}\PYG{l+s+se}{\PYGZbs{}n}\PYG{l+s+s1}{Dataframe:}\PYG{l+s+se}{\PYGZbs{}n}\PYG{l+s+si}{\PYGZob{}}\PYG{n}{res}\PYG{l+s+si}{\PYGZcb{}}\PYG{l+s+se}{\PYGZbs{}n}\PYG{l+s+se}{\PYGZbs{}n}\PYG{l+s+s1}{Growth:}\PYG{l+s+se}{\PYGZbs{}n}\PYG{l+s+si}{\PYGZob{}}\PYG{n}{res}\PYG{o}{.}\PYG{n}{pct\PYGZus{}change}\PYG{p}{(}\PYG{p}{)}\PYG{o}{*}\PYG{l+m+mi}{100}\PYG{l+s+si}{\PYGZcb{}}\PYG{l+s+se}{\PYGZbs{}n}\PYG{l+s+se}{\PYGZbs{}n}\PYG{l+s+s1}{\PYGZsq{}}\PYG{p}{)}
    \PYG{c+c1}{\PYGZsh{}  }
    \PYG{c+c1}{\PYGZsh{} Here the updated dataframe is only printed. }
    \PYG{c+c1}{\PYGZsh{} A more realistic use case is to simulate a model like this: }
    \PYG{c+c1}{\PYGZsh{} dummy\PYGZus{} = mpak(res,keep=\PYGZsq{}Severity \PYGZob{}serverity\PYGZcb{}\PYGZsq{})    \PYGZsh{} more realistic }
\end{sphinxVerbatim}

\end{sphinxuseclass}\end{sphinxVerbatimInput}
\begin{sphinxVerbatimOutput}

\begin{sphinxuseclass}{cell_output}
\begin{sphinxVerbatim}[commandchars=\\\{\}]
input dataframe: 
               A           B      A\PYGZus{}ORIG      B\PYGZus{}ORIG
2020  100.000000  110.000000  100.000000  110.000000
2021  102.000000  112.200000  102.000000  112.200000
2022  109.040000  119.444000  104.040000  114.444000
2023  111.120800  121.732880  106.120800  116.732880
2024  113.343216  119.067538  108.243216  119.067538
2025  115.610080  121.448888  110.408080  121.448888


severity=0
Dataframe:
          A           B      A\PYGZus{}ORIG      B\PYGZus{}ORIG
2020  100.0  110.000000  100.000000  110.000000
2021  100.0  112.200000  102.000000  112.200000
2022  100.0  119.444000  104.040000  114.444000
2023  100.0  121.732880  106.120800  116.732880
2024  100.0  119.067538  108.243216  119.067538
2025  100.0  121.448888  110.408080  121.448888

Growth:
        A         B  A\PYGZus{}ORIG  B\PYGZus{}ORIG
2020  NaN       NaN     NaN     NaN
2021  0.0  2.000000     2.0     2.0
2022  0.0  6.456328     2.0     2.0
2023  0.0  1.916279     2.0     2.0
2024  0.0 \PYGZhy{}2.189501     2.0     2.0
2025  0.0  2.000000     2.0     2.0


severity=0.5
Dataframe:
               A           B      A\PYGZus{}ORIG      B\PYGZus{}ORIG
2020  100.000000  110.000000  100.000000  110.000000
2021  100.500000  117.200000  102.000000  112.200000
2022  101.505000  124.444000  104.040000  114.444000
2023  103.027575  126.732880  106.120800  116.732880
2024  105.088126  124.067538  108.243216  119.067538
2025  107.715330  126.448888  110.408080  121.448888

Growth:
        A         B  A\PYGZus{}ORIG  B\PYGZus{}ORIG
2020  NaN       NaN     NaN     NaN
2021  0.5  6.545455     2.0     2.0
2022  1.0  6.180887     2.0     2.0
2023  1.5  1.839285     2.0     2.0
2024  2.0 \PYGZhy{}2.103118     2.0     2.0
2025  2.5  1.919399     2.0     2.0


severity=1
Dataframe:
               A           B      A\PYGZus{}ORIG      B\PYGZus{}ORIG
2020  100.000000  110.000000  100.000000  110.000000
2021  101.000000  122.200000  102.000000  112.200000
2022  103.020000  129.444000  104.040000  114.444000
2023  106.110600  131.732880  106.120800  116.732880
2024  110.355024  129.067538  108.243216  119.067538
2025  115.872775  131.448888  110.408080  121.448888

Growth:
        A          B  A\PYGZus{}ORIG  B\PYGZus{}ORIG
2020  NaN        NaN     NaN     NaN
2021  1.0  11.090909     2.0     2.0
2022  2.0   5.927987     2.0     2.0
2023  3.0   1.768240     2.0     2.0
2024  4.0  \PYGZhy{}2.023293     2.0     2.0
2025  5.0   1.845042     2.0     2.0
\end{sphinxVerbatim}

\end{sphinxuseclass}\end{sphinxVerbatimOutput}

\end{sphinxuseclass}

\subsection{.upd(,,lprint=True ) prints values the before and after update}
\label{\detokenize{content/04_PythonEssentials/UpdateCommand:upd-lprint-true-prints-values-the-before-and-after-update}}
\sphinxAtStartPar
The \sphinxcode{\sphinxupquote{lPrint}} option of the method \sphinxcode{\sphinxupquote{upd()}} is by defualt \sphinxcode{\sphinxupquote{= False}}.  By setting it true an update command will output the results of the calculation comapriong the values of the dataframe (over the impacted period) before, after and the difference between the two.

\begin{sphinxuseclass}{cell}\begin{sphinxVerbatimInput}

\begin{sphinxuseclass}{cell_input}
\begin{sphinxVerbatim}[commandchars=\\\{\}]
\PYG{n}{df}\PYG{o}{.}\PYG{n}{upd}\PYG{p}{(}\PYG{l+s+s1}{\PYGZsq{}\PYGZsq{}\PYGZsq{}}
\PYG{l+s+s1}{\PYGZsh{} Same number of values as years}
\PYG{l+s+s1}{\PYGZlt{}2021 2022\PYGZgt{} A *  42 44}
\PYG{l+s+s1}{\PYGZsq{}\PYGZsq{}\PYGZsq{}}\PYG{p}{,}\PYG{n}{lprint}\PYG{o}{=}\PYG{l+m+mi}{1}\PYG{p}{)}
\end{sphinxVerbatim}

\end{sphinxuseclass}\end{sphinxVerbatimInput}
\begin{sphinxVerbatimOutput}

\begin{sphinxuseclass}{cell_output}
\begin{sphinxVerbatim}[commandchars=\\\{\}]
Update * [42.0, 44.0] 2021 2022
A                    Before                After                 Diff
2021               102.0000            4284.0000            4182.0000
2022               109.0400            4797.7600            4688.7200
\end{sphinxVerbatim}

\begin{sphinxVerbatim}[commandchars=\\\{\}]
                A           B      A\PYGZus{}ORIG      B\PYGZus{}ORIG
2020   100.000000  110.000000  100.000000  110.000000
2021  4284.000000  112.200000  102.000000  112.200000
2022  4797.760000  119.444000  104.040000  114.444000
2023   111.120800  121.732880  106.120800  116.732880
2024   113.343216  119.067538  108.243216  119.067538
2025   115.610080  121.448888  110.408080  121.448888
\end{sphinxVerbatim}

\end{sphinxuseclass}\end{sphinxVerbatimOutput}

\end{sphinxuseclass}

\subsection{.upd(,,create=True ) Requires the variable to exist}
\label{\detokenize{content/04_PythonEssentials/UpdateCommand:upd-create-true-requires-the-variable-to-exist}}
\sphinxAtStartPar
Until now .upd has created variables if they did not exist in the input dataframe.

\sphinxAtStartPar
To catch misspellings the parameter \sphinxcode{\sphinxupquote{create}} can be set to False.
New variables will not be created, and an exception will be raised.

\sphinxAtStartPar
Here Pythons exception handling is used, so the notebook will continue to run the cells below.

\begin{sphinxuseclass}{cell}\begin{sphinxVerbatimInput}

\begin{sphinxuseclass}{cell_input}
\begin{sphinxVerbatim}[commandchars=\\\{\}]
\PYG{k}{try}\PYG{p}{:}
    \PYG{n}{xx} \PYG{o}{=} \PYG{n}{df}\PYG{o}{.}\PYG{n}{upd}\PYG{p}{(}\PYG{l+s+s1}{\PYGZsq{}\PYGZsq{}\PYGZsq{}}
\PYG{l+s+s1}{    \PYGZsh{} Same number of values as years}
\PYG{l+s+s1}{    \PYGZlt{}2021 2022\PYGZgt{} Aa *  42 44}
\PYG{l+s+s1}{    }\PYG{l+s+s1}{\PYGZsq{}\PYGZsq{}\PYGZsq{}}\PYG{p}{,}\PYG{n}{create}\PYG{o}{=}\PYG{k+kc}{False}\PYG{p}{)}
    \PYG{n+nb}{print}\PYG{p}{(}\PYG{n}{xx}\PYG{p}{)}
\PYG{k}{except} \PYG{n+ne}{Exception} \PYG{k}{as} \PYG{n}{inst}\PYG{p}{:}
    \PYG{n}{xx} \PYG{o}{=} \PYG{k+kc}{None}
    \PYG{n+nb}{print}\PYG{p}{(}\PYG{n}{inst}\PYG{p}{)} 
\end{sphinxVerbatim}

\end{sphinxuseclass}\end{sphinxVerbatimInput}
\begin{sphinxVerbatimOutput}

\begin{sphinxuseclass}{cell_output}
\begin{sphinxVerbatim}[commandchars=\\\{\}]
Variable to update not found:AA, timespan = [2021 2022] 
Set create=True if you want the variable created: 
\end{sphinxVerbatim}

\end{sphinxuseclass}\end{sphinxVerbatimOutput}

\end{sphinxuseclass}
\sphinxstepscope


\section{\sphinxstyleliteralintitle{\sphinxupquote{.mfcalc()}} an extension of standard Pandas}
\label{\detokenize{content/04_PythonEssentials/mfcalc:mfcalc-an-extension-of-standard-pandas}}\label{\detokenize{content/04_PythonEssentials/mfcalc::doc}}
\sphinxAtStartPar
Like\sphinxcode{\sphinxupquote{.upd()}}, the \sphinxcode{\sphinxupquote{.mfcalc()}} method can be used to extend the functionality of standard pandas.  It is actually a much more powerful method that can be used to solve models or mini\sphinxhyphen{}models or see how modelflow normalizes equations.  It can be particularly useful when creating scenarios – uses that are presented elsewhere.

\sphinxAtStartPar
Here , the focus is but is on using \sphinxcode{\sphinxupquote{mfcalc()}}to perform quick and dirty calculations and modify datafames.


\subsection{workspace initialization}
\label{\detokenize{content/04_PythonEssentials/mfcalc:workspace-initialization}}
\sphinxAtStartPar
Setting up our python session to use pandas and modelflow by importing their packages.  \sphinxcode{\sphinxupquote{modelmf}} is an extension of dataframes that is part of the modelflow installation package (and also used by modelflow itself).

\begin{sphinxuseclass}{cell}\begin{sphinxVerbatimInput}

\begin{sphinxuseclass}{cell_input}
\begin{sphinxVerbatim}[commandchars=\\\{\}]
\PYG{k+kn}{import} \PYG{n+nn}{pandas} \PYG{k}{as} \PYG{n+nn}{pd}  \PYG{c+c1}{\PYGZsh{} Python data science library}
\PYG{k+kn}{import} \PYG{n+nn}{modelmf}       \PYG{c+c1}{\PYGZsh{} Add useful features to pandas dataframes }
                     \PYG{c+c1}{\PYGZsh{} using utlities initially developed for modelflow}
\end{sphinxVerbatim}

\end{sphinxuseclass}\end{sphinxVerbatimInput}

\end{sphinxuseclass}

\subsection{Create a  simple dataframe}
\label{\detokenize{content/04_PythonEssentials/mfcalc:create-a-simple-dataframe}}
\sphinxAtStartPar
Create a Pandas dataframe with one column with the name A and 6 rows.

\sphinxAtStartPar
Set set the index to 2020 through 2026 and set the values of all the cells to 100.
\begin{itemize}
\item {} 
\sphinxAtStartPar
\sphinxcode{\sphinxupquote{pd.DataFrame}} creates a dataframe  \sphinxhref{https://pandas.pydata.org/docs/reference/api/pandas.DataFrame.html\#pandas.DataFrame}{Description}

\item {} 
\sphinxAtStartPar
The expression \sphinxcode{\sphinxupquote{{[}v for v in range(2020,2026){]}}} dynamically creates a  python list, and fills it with  integers beginning with 2020 and ending 2025

\end{itemize}

\begin{sphinxuseclass}{cell}\begin{sphinxVerbatimInput}

\begin{sphinxuseclass}{cell_input}
\begin{sphinxVerbatim}[commandchars=\\\{\}]
\PYG{n}{df} \PYG{o}{=} \PYG{n}{pd}\PYG{o}{.}\PYG{n}{DataFrame}\PYG{p}{(}                                 \PYG{c+c1}{\PYGZsh{} call the dataframe constructure }
    \PYG{l+m+mf}{100.000}\PYG{p}{,}                                           \PYG{c+c1}{\PYGZsh{} the values }
    \PYG{n}{index}\PYG{o}{=}\PYG{p}{[}\PYG{n}{v} \PYG{k}{for} \PYG{n}{v} \PYG{o+ow}{in} \PYG{n+nb}{range}\PYG{p}{(}\PYG{l+m+mi}{2020}\PYG{p}{,}\PYG{l+m+mi}{2026}\PYG{p}{)}\PYG{p}{]}\PYG{p}{,}           \PYG{c+c1}{\PYGZsh{}index}
    \PYG{n}{columns}\PYG{o}{=}\PYG{p}{[}\PYG{l+s+s1}{\PYGZsq{}}\PYG{l+s+s1}{A}\PYG{l+s+s1}{\PYGZsq{}}\PYG{p}{]}                                  \PYG{c+c1}{\PYGZsh{} the column name }
                 \PYG{p}{)}
\PYG{n}{df}   \PYG{c+c1}{\PYGZsh{} the result of the last statement is displayed in the output cell }
\end{sphinxVerbatim}

\end{sphinxuseclass}\end{sphinxVerbatimInput}
\begin{sphinxVerbatimOutput}

\begin{sphinxuseclass}{cell_output}
\begin{sphinxVerbatim}[commandchars=\\\{\}]
          A
2020  100.0
2021  100.0
2022  100.0
2023  100.0
2024  100.0
2025  100.0
\end{sphinxVerbatim}

\end{sphinxuseclass}\end{sphinxVerbatimOutput}

\end{sphinxuseclass}

\subsection{\sphinxstyleliteralintitle{\sphinxupquote{.mfcalc()}} example to calculate a new series}
\label{\detokenize{content/04_PythonEssentials/mfcalc:mfcalc-example-to-calculate-a-new-series}}
\sphinxAtStartPar
Use  mfcalc to calculate a new column (series) as a function of the existing A column series

\sphinxAtStartPar
The below call creates a new column x.

\begin{sphinxuseclass}{cell}\begin{sphinxVerbatimInput}

\begin{sphinxuseclass}{cell_input}
\begin{sphinxVerbatim}[commandchars=\\\{\}]
\PYG{n}{df}\PYG{o}{.}\PYG{n}{mfcalc}\PYG{p}{(}\PYG{l+s+s1}{\PYGZsq{}}\PYG{l+s+s1}{x = x(\PYGZhy{}1) + a}\PYG{l+s+s1}{\PYGZsq{}}\PYG{p}{)}
\end{sphinxVerbatim}

\end{sphinxuseclass}\end{sphinxVerbatimInput}
\begin{sphinxVerbatimOutput}

\begin{sphinxuseclass}{cell_output}
\begin{sphinxVerbatim}[commandchars=\\\{\}]
* Take care. Lags or leads in the equations, mfcalc run for 2021 to 2022
\end{sphinxVerbatim}

\begin{sphinxVerbatim}[commandchars=\\\{\}]
          A      X
2020  100.0    0.0
2021  100.0  100.0
2022  100.0  200.0
2023  100.0  300.0
2024  100.0  400.0
2025  100.0  500.0
\end{sphinxVerbatim}

\end{sphinxuseclass}\end{sphinxVerbatimOutput}

\end{sphinxuseclass}
\begin{sphinxadmonition}{warning}{Warning:}
\sphinxAtStartPar
By default \sphinxcode{\sphinxupquote{.mfcalc}} will initialize a new variable with zeroes.

\sphinxAtStartPar
Moreover, if a formula passed to \sphinxcode{\sphinxupquote{.mfcalc}} contains a lag a value will be calculated for the a row only if there is data in the series for the preceding row.

\sphinxAtStartPar
These two behaviors affects how calculations generated with \sphinxcode{\sphinxupquote{.mfcalc}} are executed and can generate results that may sometimes by unexpected.
\end{sphinxadmonition}

\sphinxAtStartPar
The initialization of new variables with zero and the treatment of lags combined means that when the command \sphinxcode{\sphinxupquote{df.mfcalc('x = x(\sphinxhyphen{}1) + a')}} is executed, the value for X in 2020 will be zero (not n/a). This results because there was no X variable defined for 2019 (no such row exists). \sphinxcode{\sphinxupquote{modelflow}} first initializes all values of X with zero.  It then goes to calculate X in 2020.  There is no X value for 2019 so it skips ahead to 2021 and calculates X as equal to 0 (the value of x in 2020) + the value for a in 2021 – etc.

\begin{sphinxuseclass}{cell}\begin{sphinxVerbatimInput}

\begin{sphinxuseclass}{cell_input}
\begin{sphinxVerbatim}[commandchars=\\\{\}]
\PYG{n}{df}
\end{sphinxVerbatim}

\end{sphinxuseclass}\end{sphinxVerbatimInput}
\begin{sphinxVerbatimOutput}

\begin{sphinxuseclass}{cell_output}
\begin{sphinxVerbatim}[commandchars=\\\{\}]
          A
2020  100.0
2021  100.0
2022  100.0
2023  100.0
2024  100.0
2025  100.0
\end{sphinxVerbatim}

\end{sphinxuseclass}\end{sphinxVerbatimOutput}

\end{sphinxuseclass}

\subsection{Storing the result of an \sphinxstyleliteralintitle{\sphinxupquote{.mfcalc()}} call}
\label{\detokenize{content/04_PythonEssentials/mfcalc:storing-the-result-of-an-mfcalc-call}}
\sphinxAtStartPar
Above the results of the \sphinxcode{\sphinxupquote{.mfcalc()}} operation was not assigned to an object – the \sphinxcode{\sphinxupquote{DataFrame}} object df itself was not changed.

\sphinxAtStartPar
Below the results of the same operation are assigned to the variable df2 and therefore stored.

\begin{sphinxuseclass}{cell}\begin{sphinxVerbatimInput}

\begin{sphinxuseclass}{cell_input}
\begin{sphinxVerbatim}[commandchars=\\\{\}]
\PYG{n}{df2}\PYG{o}{=}\PYG{n}{df}\PYG{o}{.}\PYG{n}{mfcalc}\PYG{p}{(}\PYG{l+s+s1}{\PYGZsq{}}\PYG{l+s+s1}{x = x(\PYGZhy{}1) + a}\PYG{l+s+s1}{\PYGZsq{}}\PYG{p}{)} \PYG{c+c1}{\PYGZsh{} Assign the result to df2}
\PYG{n}{df2}
\end{sphinxVerbatim}

\end{sphinxuseclass}\end{sphinxVerbatimInput}
\begin{sphinxVerbatimOutput}

\begin{sphinxuseclass}{cell_output}
\begin{sphinxVerbatim}[commandchars=\\\{\}]
* Take care. Lags or leads in the equations, mfcalc run for 2021 to 2022
\end{sphinxVerbatim}

\begin{sphinxVerbatim}[commandchars=\\\{\}]
          A      X
2020  100.0    0.0
2021  100.0  100.0
2022  100.0  200.0
2023  100.0  300.0
2024  100.0  400.0
2025  100.0  500.0
\end{sphinxVerbatim}

\end{sphinxuseclass}\end{sphinxVerbatimOutput}

\end{sphinxuseclass}

\subsection{Recalculate A so  it grows by 2 percent}
\label{\detokenize{content/04_PythonEssentials/mfcalc:recalculate-a-so-it-grows-by-2-percent}}
\sphinxAtStartPar
\sphinxcode{\sphinxupquote{mfcalc()}}knows that it can not start to calculate in 2020 A (the lagged variable) has no value in 2019.

\sphinxAtStartPar
\sphinxcode{\sphinxupquote{.mfcalc()}} therefore begins its calculation in 2021. Note, the existing value for 2020 is preserved.  This behaviour differs from other programs that might return a n/a value for the 2020.

\begin{sphinxuseclass}{cell}\begin{sphinxVerbatimInput}

\begin{sphinxuseclass}{cell_input}
\begin{sphinxVerbatim}[commandchars=\\\{\}]
\PYG{n}{res} \PYG{o}{=} \PYG{n}{df}\PYG{o}{.}\PYG{n}{mfcalc}\PYG{p}{(}\PYG{l+s+s1}{\PYGZsq{}}\PYG{l+s+s1}{a =  1.02 *  a(\PYGZhy{}1)}\PYG{l+s+s1}{\PYGZsq{}}\PYG{p}{)}
\PYG{n}{res}
\end{sphinxVerbatim}

\end{sphinxuseclass}\end{sphinxVerbatimInput}
\begin{sphinxVerbatimOutput}

\begin{sphinxuseclass}{cell_output}
\begin{sphinxVerbatim}[commandchars=\\\{\}]
* Take care. Lags or leads in the equations, mfcalc run for 2021 to 2022
\end{sphinxVerbatim}

\begin{sphinxVerbatim}[commandchars=\\\{\}]
               A
2020  100.000000
2021  102.000000
2022  104.040000
2023  106.120800
2024  108.243216
2025  110.408080
\end{sphinxVerbatim}

\end{sphinxuseclass}\end{sphinxVerbatimOutput}

\end{sphinxuseclass}
\begin{sphinxuseclass}{cell}\begin{sphinxVerbatimInput}

\begin{sphinxuseclass}{cell_input}
\begin{sphinxVerbatim}[commandchars=\\\{\}]
\PYG{n}{res}\PYG{o}{.}\PYG{n}{pct\PYGZus{}change}\PYG{p}{(}\PYG{p}{)}\PYG{o}{*}\PYG{l+m+mi}{100} \PYG{c+c1}{\PYGZsh{} to display the percent changes}
\end{sphinxVerbatim}

\end{sphinxuseclass}\end{sphinxVerbatimInput}
\begin{sphinxVerbatimOutput}

\begin{sphinxuseclass}{cell_output}
\begin{sphinxVerbatim}[commandchars=\\\{\}]
        A
2020  NaN
2021  2.0
2022  2.0
2023  2.0
2024  2.0
2025  2.0
\end{sphinxVerbatim}

\end{sphinxuseclass}\end{sphinxVerbatimOutput}

\end{sphinxuseclass}

\subsection{\sphinxstyleliteralintitle{\sphinxupquote{.mfcalc()}} \sphinxhyphen{} the showeq option}
\label{\detokenize{content/04_PythonEssentials/mfcalc:mfcalc-the-showeq-option}}
\sphinxAtStartPar
The \sphinxcode{\sphinxupquote{showeq}} option is by default \sphinxcode{\sphinxupquote{= False}}.

\sphinxAtStartPar
By setting equal to \sphinxcode{\sphinxupquote{True}}, mfcalc can be used to express the normalization of an entered equation.

\begin{sphinxuseclass}{cell}\begin{sphinxVerbatimInput}

\begin{sphinxuseclass}{cell_input}
\begin{sphinxVerbatim}[commandchars=\\\{\}]
\PYG{n}{df}\PYG{o}{.}\PYG{n}{mfcalc}\PYG{p}{(}\PYG{l+s+s1}{\PYGZsq{}}\PYG{l+s+s1}{dlog( a) =  0.02}\PYG{l+s+s1}{\PYGZsq{}}\PYG{p}{,}\PYG{n}{showeq}\PYG{o}{=}\PYG{l+m+mi}{1}\PYG{p}{)}\PYG{p}{;}
\end{sphinxVerbatim}

\end{sphinxuseclass}\end{sphinxVerbatimInput}
\begin{sphinxVerbatimOutput}

\begin{sphinxuseclass}{cell_output}
\begin{sphinxVerbatim}[commandchars=\\\{\}]
* Take care. Lags or leads in the equations, mfcalc run for 2021 to 2022
FRML \PYGZlt{}\PYGZgt{} A=EXP(LOG(A(\PYGZhy{}1))+0.02)\PYGZdl{}
\end{sphinxVerbatim}

\end{sphinxuseclass}\end{sphinxVerbatimOutput}

\end{sphinxuseclass}
\sphinxAtStartPar
In \sphinxcode{\sphinxupquote{modelflow}} the expression \sphinxcode{\sphinxupquote{dlog(a)}} refers to the difference in the natural logarithm \(dlog(x_t) \equiv ln(x_t)-ln(x_{t-1})\) and is equal to the growth rate for the variable.

\sphinxAtStartPar
\sphinxcode{\sphinxupquote{.mfcalc()}} normalizes the equation such that the systems solves for a as follows:
\begin{align*}
dlog(a) &= 0.02\\
log(a)-log(a_{t-1}) &= .02\\
log(a) &=log(a_{t-1})+.02\\
a &= e^{log(a_{t-1})+0.02}\\
a &=a_{t-1}*e^{0.02}\\
\end{align*}
\sphinxAtStartPar
which expressed in the business logic language of \sphinxcode{\sphinxupquote{modelflow}} is:

\sphinxAtStartPar
A=EXP(LOG(A(\sphinxhyphen{}1))+0.02)


\subsection{Using the diff() operator with mfcalc}
\label{\detokenize{content/04_PythonEssentials/mfcalc:using-the-diff-operator-with-mfcalc}}
\sphinxAtStartPar
The diff() operator, effectively normalizes to an equation that will add the value to the right of the equals sign to the lagged variable inserted in the diff operator.  Thus,  diff(a)=x normalizes to a=a(\sphinxhyphen{}1)+x

\begin{sphinxuseclass}{cell}\begin{sphinxVerbatimInput}

\begin{sphinxuseclass}{cell_input}
\begin{sphinxVerbatim}[commandchars=\\\{\}]
\PYG{n}{df}\PYG{o}{.}\PYG{n}{mfcalc}\PYG{p}{(}\PYG{l+s+s1}{\PYGZsq{}}\PYG{l+s+s1}{diff(a) =  2}\PYG{l+s+s1}{\PYGZsq{}}\PYG{p}{,}\PYG{n}{showeq}\PYG{o}{=}\PYG{l+m+mi}{1}\PYG{p}{)}
\end{sphinxVerbatim}

\end{sphinxuseclass}\end{sphinxVerbatimInput}
\begin{sphinxVerbatimOutput}

\begin{sphinxuseclass}{cell_output}
\begin{sphinxVerbatim}[commandchars=\\\{\}]
* Take care. Lags or leads in the equations, mfcalc run for 2021 to 2022
FRML \PYGZlt{}\PYGZgt{} A=A(\PYGZhy{}1)+(2)\PYGZdl{}
\end{sphinxVerbatim}

\begin{sphinxVerbatim}[commandchars=\\\{\}]
          A
2020  100.0
2021  102.0
2022  104.0
2023  106.0
2024  108.0
2025  110.0
\end{sphinxVerbatim}

\end{sphinxuseclass}\end{sphinxVerbatimOutput}

\end{sphinxuseclass}

\subsection{mfcalc with several equations and arguments}
\label{\detokenize{content/04_PythonEssentials/mfcalc:mfcalc-with-several-equations-and-arguments}}
\sphinxAtStartPar
In addition to a single equation multiple commands can be executed with one command.

\sphinxAtStartPar
However, \sphinxstylestrong{be careful} because the equation commands are executed simultaneously, which, combined with the treatments of lags, means that results may differ from what they would be if the commands were run sequentially.

\sphinxAtStartPar
For example:

\begin{sphinxuseclass}{cell}\begin{sphinxVerbatimInput}

\begin{sphinxuseclass}{cell_input}
\begin{sphinxVerbatim}[commandchars=\\\{\}]
\PYG{n}{res} \PYG{o}{=} \PYG{n}{df}\PYG{o}{.}\PYG{n}{mfcalc}\PYG{p}{(}\PYG{l+s+s1}{\PYGZsq{}\PYGZsq{}\PYGZsq{}}
\PYG{l+s+s1}{diff(a) =  2}
\PYG{l+s+s1}{x = a + 42 }
\PYG{l+s+s1}{\PYGZsq{}\PYGZsq{}\PYGZsq{}}\PYG{p}{)}

\PYG{n}{res}

\PYG{c+c1}{\PYGZsh{} use res.diff() to see the difference}
\end{sphinxVerbatim}

\end{sphinxuseclass}\end{sphinxVerbatimInput}
\begin{sphinxVerbatimOutput}

\begin{sphinxuseclass}{cell_output}
\begin{sphinxVerbatim}[commandchars=\\\{\}]
* Take care. Lags or leads in the equations, mfcalc run for 2021 to 2022
\end{sphinxVerbatim}

\begin{sphinxVerbatim}[commandchars=\\\{\}]
          A      X
2020  100.0    0.0
2021  102.0  144.0
2022  104.0  146.0
2023  106.0  148.0
2024  108.0  150.0
2025  110.0  152.0
\end{sphinxVerbatim}

\end{sphinxuseclass}\end{sphinxVerbatimOutput}

\end{sphinxuseclass}
\sphinxAtStartPar
In this example the \sphinxcode{\sphinxupquote{DataFarame}} df was initialized to 100 for the period 2020 through 2025.

\sphinxAtStartPar
The first line of the \sphinxcode{\sphinxupquote{.mfcalc()}} routine produces results only for the period 2021 \sphinxhyphen{} 2025 because there is no value for \sphinxcode{\sphinxupquote{a}} in 2019.  The value of a in 2020 is unchanged, and the following values rise by 2 in each period.

\sphinxAtStartPar
When calculating X however, \sphinxcode{\sphinxupquote{.mfcalc}} does not use the final result of the calculation of \sphinxcode{\sphinxupquote{A}}, but the intermediate result (the values for 2021 through 2025.

\sphinxAtStartPar
As a result, it is this series that is passed to the second question which adds 42 to that result.

\sphinxAtStartPar
\sphinxstylestrong{X in 2020 is not 142 as one might have expected but zero, the value to which the newly created variable defaults.}

\sphinxAtStartPar
Compare the results above with the results (below) when the two steps are now undertaken in two separate calls to \sphinxcode{\sphinxupquote{.mfcalc()}}.

\begin{sphinxuseclass}{cell}\begin{sphinxVerbatimInput}

\begin{sphinxuseclass}{cell_input}
\begin{sphinxVerbatim}[commandchars=\\\{\}]
\PYG{n}{res1} \PYG{o}{=} \PYG{n}{df}\PYG{o}{.}\PYG{n}{mfcalc}\PYG{p}{(}\PYG{l+s+s1}{\PYGZsq{}\PYGZsq{}\PYGZsq{}}
\PYG{l+s+s1}{diff(a) =  2}
\PYG{l+s+s1}{\PYGZsq{}\PYGZsq{}\PYGZsq{}}\PYG{p}{)}

\PYG{n}{res2} \PYG{o}{=} \PYG{n}{res1}\PYG{o}{.}\PYG{n}{mfcalc}\PYG{p}{(}\PYG{l+s+s1}{\PYGZsq{}\PYGZsq{}\PYGZsq{}}
\PYG{l+s+s1}{x = a + 42 }
\PYG{l+s+s1}{\PYGZsq{}\PYGZsq{}\PYGZsq{}}\PYG{p}{)}
\PYG{n}{res2}
\end{sphinxVerbatim}

\end{sphinxuseclass}\end{sphinxVerbatimInput}
\begin{sphinxVerbatimOutput}

\begin{sphinxuseclass}{cell_output}
\begin{sphinxVerbatim}[commandchars=\\\{\}]
* Take care. Lags or leads in the equations, mfcalc run for 2021 to 2022
\end{sphinxVerbatim}

\begin{sphinxVerbatim}[commandchars=\\\{\}]
          A      X
2020  100.0  142.0
2021  102.0  144.0
2022  104.0  146.0
2023  106.0  148.0
2024  108.0  150.0
2025  110.0  152.0
\end{sphinxVerbatim}

\end{sphinxuseclass}\end{sphinxVerbatimOutput}

\end{sphinxuseclass}
\begin{sphinxadmonition}{danger}{Danger:}
\sphinxAtStartPar
In \sphinxcode{\sphinxupquote{.mfcalc()}}, when there are multiple equation commands in a single call, they are executed simultaneously. This, combined with \sphinxcode{\sphinxupquote{mfcalc}}’s  treatments of lags, means only the results of the lagged calculation will be passed to other commands equations defined in the \sphinxcode{\sphinxupquote{.mfcalc}} command. As a consequence, results may differ from what would be expected and what would be seen if the two commands were run sequentially.
\end{sphinxadmonition}


\subsection{Setting a time frame with mfcalc.}
\label{\detokenize{content/04_PythonEssentials/mfcalc:setting-a-time-frame-with-mfcalc}}
\sphinxAtStartPar
It can useful in some circumstances to limit the time frame for which the calculations are performed. Specifying a start date and end date enclosed in <> in a line restricts the time period over which subsequent calculations are performed.

\sphinxAtStartPar
In the example below zeroes are generated for x prior to 2023 when the expressions are executed.

\begin{sphinxuseclass}{cell}\begin{sphinxVerbatimInput}

\begin{sphinxuseclass}{cell_input}
\begin{sphinxVerbatim}[commandchars=\\\{\}]
\PYG{n}{res} \PYG{o}{=} \PYG{n}{df}\PYG{o}{.}\PYG{n}{mfcalc}\PYG{p}{(}\PYG{l+s+s1}{\PYGZsq{}\PYGZsq{}\PYGZsq{}}
\PYG{l+s+s1}{\PYGZlt{}2023 2025\PYGZgt{}}
\PYG{l+s+s1}{diff(a) =  2}
\PYG{l+s+s1}{x = a + 42 }
\PYG{l+s+s1}{\PYGZsq{}\PYGZsq{}\PYGZsq{}}\PYG{p}{)}

\PYG{n}{res}\PYG{o}{.}\PYG{n}{diff}\PYG{p}{(}\PYG{p}{)}

\PYG{n}{res}
\end{sphinxVerbatim}

\end{sphinxuseclass}\end{sphinxVerbatimInput}
\begin{sphinxVerbatimOutput}

\begin{sphinxuseclass}{cell_output}
\begin{sphinxVerbatim}[commandchars=\\\{\}]
          A      X
2020  100.0    0.0
2021  100.0    0.0
2022  100.0    0.0
2023  102.0  144.0
2024  104.0  146.0
2025  106.0  148.0
\end{sphinxVerbatim}

\end{sphinxuseclass}\end{sphinxVerbatimOutput}

\end{sphinxuseclass}
\sphinxstepscope


\part{Using modelflow with World Bank models}

\sphinxstepscope


\chapter{Using \sphinxstyleliteralintitle{\sphinxupquote{modelflow}} with World Bank models}
\label{\detokenize{content/05_WBModels/LoadingWBModel:using-modelflow-with-world-bank-models}}\label{\detokenize{content/05_WBModels/LoadingWBModel::doc}}
\sphinxAtStartPar
The \sphinxcode{\sphinxupquote{Modelflow}} python package has been developed to solve a wide range of models, see the modelflow \DUrole{xref,myst}{github} web site for working examples of the Solow Model, the FR/USB model and others.

\sphinxAtStartPar
The package has been substantially expanded to include special features that enable it to work with World Bank models originally developed in EViews and designed to use EViews Model Object for simuation.

\sphinxAtStartPar
This chapter illustrates how to access these models, how to load them into a \sphinxcode{\sphinxupquote{modelflow}} anaconda environment on your computer and how to perform a variety of simulations


\section{Accessing a world bank model}
\label{\detokenize{content/05_WBModels/LoadingWBModel:accessing-a-world-bank-model}}
\sphinxAtStartPar
At this time several World bank macrostructural models are available to download and use with \sphinxcode{\sphinxupquote{modelflow}}.  These include a macrostructural model for:
\begin{itemize}
\item {} 
\sphinxAtStartPar
Indonesia

\item {} 
\sphinxAtStartPar
Nepal

\item {} 
\sphinxAtStartPar
Croatia

\item {} 
\sphinxAtStartPar
Iraq

\item {} 
\sphinxAtStartPar
Kenya

\item {} 
\sphinxAtStartPar
Bolivia

\end{itemize}

\sphinxAtStartPar
Each of these models has been developed as part of the outreach work of the World Bank.  The basic modelling framework of each of these models is outlined in Burns \sphinxstyleemphasis{et al.} {[}\hyperlink{cite.content/99_BackMatter/References:id15}{2019}{]} with specific extensions reflecting features of the individual country modelled.

\sphinxAtStartPar
This book uses as an example a climate aware model for Pakistan developed in 2020 and described in Burns \sphinxstyleemphasis{et al.} {[}\hyperlink{cite.content/99_BackMatter/References:id14}{2021}{]} .

\sphinxAtStartPar
The World Bank models are distributed in the \sphinxcode{\sphinxupquote{pcim}} file format of the \sphinxcode{\sphinxupquote{modelflow}} and can be downloaded by right clicking on the links above.  The Pakistan model can be downloaded \DUrole{xref,download,myst}{here} by right clicking on the above link and selecting Save Link as and placing the file on a directory accessible by your \sphinxcode{\sphinxupquote{modelflow}} installation.

\begin{sphinxadmonition}{note}{Question}

\begin{sphinxVerbatim}[commandchars=\\\{\}]
Ib can\PYGZsq{}t we have a package WorldBankMFModModels that one could import?  I seem to see this for other packages that have geographic data on countries or their population.   
\end{sphinxVerbatim}

\sphinxAtStartPar
I imagining something like:
\end{sphinxadmonition}

\sphinxAtStartPar
\sphinxcode{\sphinxupquote{from worldbankMFModModels import pak}}

\begin{sphinxadmonition}{note}{Ansver}

\sphinxAtStartPar
As of now, a model which is not located in the specified path will be sought in a global model repo on github. The user can specify any URL where the model is located instead, or a URL for a repo which is seached for the model name. So, if I am not wrong it is better to place each model at some location instead of creating a python packagde where all models are located,
\end{sphinxadmonition}


\section{Preparing your python environment}
\label{\detokenize{content/05_WBModels/LoadingWBModel:preparing-your-python-environment}}
\sphinxAtStartPar
As always, the \sphinxcode{\sphinxupquote{modelflow}} and other python packages that will be used need to be imported into your python session.  The examples here and this book were written and solved in a \sphinxstyleemphasis{Jupyter Notebook}. There are some Jupyter specific commands included in these examples and these are annotated. However, the bulk of the content of the programs can be run in other environments, including Interactive Development Environments (IDE) like \sphinxcode{\sphinxupquote{Spyder}}or \sphinxcode{\sphinxupquote{MS Visual Code}}.  All the programs have been tested under \sphinxcode{\sphinxupquote{spyder}} as well as Jupyter Notebook.

\sphinxAtStartPar
It is assumed that:
\begin{enumerate}
\sphinxsetlistlabels{\arabic}{enumi}{enumii}{}{.}%
\item {} 
\sphinxAtStartPar
you have already installed \sphinxcode{\sphinxupquote{modelflow}} and its various support packages following the instructions in Chapter xx

\item {} 
\sphinxAtStartPar
you are using Anaconda, and that

\item {} 
\sphinxAtStartPar
you have activated your \sphinxcode{\sphinxupquote{modelflow}} environment by executing the following command from your python command line:

\end{enumerate}

\begin{sphinxVerbatim}[commandchars=\\\{\}]
\PYG{n}{conda} \PYG{n}{activate} \PYG{n}{modelflow}
\end{sphinxVerbatim}

\sphinxAtStartPar
where \sphinxcode{\sphinxupquote{modelflow}} is the name you have given to the \sphinxcode{\sphinxupquote{conda}} environment into which you installed \sphinxcode{\sphinxupquote{modelflow}}.


\chapter{Working with PakMod under modelflow}
\label{\detokenize{content/05_WBModels/LoadingWBModel:working-with-pakmod-under-modelflow}}
\sphinxAtStartPar
The basic method for working with any model is the same. Indeed the initial steps followed here are the same as were followed during the simple model discussion.

\sphinxAtStartPar
Process:
\begin{enumerate}
\sphinxsetlistlabels{\arabic}{enumi}{enumii}{}{.}%
\item {} 
\sphinxAtStartPar
Prepare the workspace

\item {} 
\sphinxAtStartPar
Load the model Modelflow

\item {} 
\sphinxAtStartPar
Design some scenarios

\item {} 
\sphinxAtStartPar
Simulate the model

\item {} 
\sphinxAtStartPar
Visualize the results

\end{enumerate}


\section{Load a pre\sphinxhyphen{}existing model, data and descriptions}
\label{\detokenize{content/05_WBModels/LoadingWBModel:load-a-pre-existing-model-data-and-descriptions}}
\sphinxAtStartPar
To load a model use the \sphinxcode{\sphinxupquote{model.modelload()}} method of \sphinxcode{\sphinxupquote{modelflow}}.

\sphinxAtStartPar
The command below

\begin{sphinxVerbatim}[commandchars=\\\{\}]
\PYG{n}{mpak}\PYG{p}{,}\PYG{n}{bline} \PYG{o}{=} \PYG{n}{model}\PYG{o}{.}\PYG{n}{modelload}\PYG{p}{(}\PYG{l+s+s1}{\PYGZsq{}}\PYG{l+s+s1}{..}\PYG{l+s+s1}{\PYGZbs{}}\PYG{l+s+s1}{models}\PYG{l+s+s1}{\PYGZbs{}}\PYG{l+s+s1}{pak.pcim}\PYG{l+s+s1}{\PYGZsq{}}\PYG{p}{,} \PYG{n}{alfa}\PYG{o}{=}\PYG{l+m+mf}{0.7}\PYG{p}{,}\PYG{n}{run}\PYG{o}{=}\PYG{l+m+mi}{1}\PYG{p}{,}\PYG{n}{keep}\PYG{o}{=} \PYG{l+s+s1}{\PYGZsq{}}\PYG{l+s+s1}{Baseline}\PYG{l+s+s1}{\PYGZsq{}}\PYG{p}{)}
\end{sphinxVerbatim}

\sphinxAtStartPar
instantiates (creates an instance of) a \sphinxcode{\sphinxupquote{modelflow model}} object and assigns it to the variable name mpak.  The \sphinxcode{\sphinxupquote{run=1}} option executes the model and assigns the result of the model execution to the dataframe \sphinxcode{\sphinxupquote{bline}}.  The model is solved with the parameter alfa set to 0.7.  The \(alfa \in (0,1)\) parameter determines the step size of the solution engine. The larger alfa the larger the step size. Larger step sizes may solve faster, but may have trouble finding a unique solution.  Smaller step sizes take longer to solve but are more likely to find a unique solution.  Values of alfa=.7 work well for World Bank models.

\sphinxAtStartPar
The \sphinxcode{\sphinxupquote{keep}} option instructs \sphinxcode{\sphinxupquote{modelflow}} to maintain in the model object (\sphinxcode{\sphinxupquote{mpak}}) the results of the initial scenario, assigning it the text name \sphinxcode{\sphinxupquote{Baseline}}.

\begin{sphinxadmonition}{note}{Note:}
\sphinxAtStartPar
If modelflow cannot find the file at the position indicated it will look it \sphinxstylestrong{the global Model repository} where some models are stored.
\end{sphinxadmonition}

\begin{sphinxuseclass}{cell}\begin{sphinxVerbatimInput}

\begin{sphinxuseclass}{cell_input}
\begin{sphinxVerbatim}[commandchars=\\\{\}]
\PYG{c+c1}{\PYGZsh{}Replace the path below with the location of the pak.pcim file on your computer}
\PYG{n}{mpak}\PYG{p}{,}\PYG{n}{bline} \PYG{o}{=} \PYG{n}{model}\PYG{o}{.}\PYG{n}{modelload}\PYG{p}{(}\PYG{l+s+s1}{\PYGZsq{}}\PYG{l+s+s1}{..}\PYG{l+s+s1}{\PYGZbs{}}\PYG{l+s+s1}{models}\PYG{l+s+s1}{\PYGZbs{}}\PYG{l+s+s1}{pak.pcim}\PYG{l+s+s1}{\PYGZsq{}}\PYG{p}{,} \PYGZbs{}
                                \PYG{n}{alfa}\PYG{o}{=}\PYG{l+m+mf}{0.7}\PYG{p}{,}\PYG{n}{run}\PYG{o}{=}\PYG{l+m+mi}{1}\PYG{p}{,}\PYG{n}{keep}\PYG{o}{=} \PYG{l+s+s1}{\PYGZsq{}}\PYG{l+s+s1}{Baseline}\PYG{l+s+s1}{\PYGZsq{}}\PYG{p}{)}
\end{sphinxVerbatim}

\end{sphinxuseclass}\end{sphinxVerbatimInput}
\begin{sphinxVerbatimOutput}

\begin{sphinxuseclass}{cell_output}
\begin{sphinxVerbatim}[commandchars=\\\{\}]
file read:  C:\PYGZbs{}modelflow manual\PYGZbs{}papers\PYGZbs{}mfbook\PYGZbs{}content\PYGZbs{}models\PYGZbs{}pak.pcim
\end{sphinxVerbatim}

\end{sphinxuseclass}\end{sphinxVerbatimOutput}

\end{sphinxuseclass}
\begin{sphinxadmonition}{note}{Note:}
\sphinxAtStartPar
the variable \sphinxcode{\sphinxupquote{bline}} contains the dataframe with the results of the simulation.  This is distinct from the data that is stored by the \sphinxcode{\sphinxupquote{keep=}} command. That said, the data associated with each, while stored separately, have the same numerical values. The \sphinxcode{\sphinxupquote{keep}} option is described in more detail toward the end of this section.
\end{sphinxadmonition}


\chapter{Extracting information about the model}
\label{\detokenize{content/05_WBModels/LoadingWBModel:extracting-information-about-the-model}}
\sphinxAtStartPar
The newly loaded python object  \sphinxcode{\sphinxupquote{mpak}} is an instance of the model class and as such inherits the \sphinxcode{\sphinxupquote{methods}} (functions) and \sphinxcode{\sphinxupquote{properties}} (data) of that class. To learn about the model there are a variety of methods that can be used to extract information about the model and its data.

\sphinxAtStartPar
A World Bank model in \sphinxcode{\sphinxupquote{modelflow}} will contain a wide range of objects.
\begin{itemize}
\item {} 
\sphinxAtStartPar
variables  – time series variables comprised of mnemonics and data

\item {} 
\sphinxAtStartPar
variables descriptions – when avaiable

\item {} 
\sphinxAtStartPar
dataframes – data for each variable generated in  different simulations

\item {} 
\sphinxAtStartPar
groups     – lists of variables

\item {} 
\sphinxAtStartPar
equations  – identities and behaviourals

\item {} 
\sphinxAtStartPar
model      – the model object itself

\end{itemize}

\sphinxAtStartPar
Extracting information about each of these objects is central to working with WBG models in \sphinxcode{\sphinxupquote{modelflow}}.


\section{Model information}
\label{\detokenize{content/05_WBModels/LoadingWBModel:model-information}}
\sphinxAtStartPar
The model object contains information about the model itself, its name, its structure (does it contain simultaneous equations or is it recursive), the number of variables it contains and the number that are exogenous and endogenous (have associated equations).

\begin{sphinxuseclass}{cell}\begin{sphinxVerbatimInput}

\begin{sphinxuseclass}{cell_input}
\begin{sphinxVerbatim}[commandchars=\\\{\}]
\PYG{n}{mpak}
\end{sphinxVerbatim}

\end{sphinxuseclass}\end{sphinxVerbatimInput}
\begin{sphinxVerbatimOutput}

\begin{sphinxuseclass}{cell_output}
\begin{sphinxVerbatim}[commandchars=\\\{\}]
\PYGZlt{}
Model name                              :                  PAK 
Model structure                         :         Simultaneous 
Number of variables                     :                  839 
Number of exogeneous  variables         :                  461 
Number of endogeneous variables         :                  378 
\PYGZgt{}
\end{sphinxVerbatim}

\end{sphinxuseclass}\end{sphinxVerbatimOutput}

\end{sphinxuseclass}
\sphinxAtStartPar
The model work space also has a time dimension, its sample period.  This can be retrieved and changed. The current time dimension is contained in the variable `mpak.per\_current’

\begin{sphinxuseclass}{cell}\begin{sphinxVerbatimInput}

\begin{sphinxuseclass}{cell_input}
\begin{sphinxVerbatim}[commandchars=\\\{\}]
\PYG{n}{mpak}\PYG{o}{.}\PYG{n}{current\PYGZus{}per}
\end{sphinxVerbatim}

\end{sphinxuseclass}\end{sphinxVerbatimInput}
\begin{sphinxVerbatimOutput}

\begin{sphinxuseclass}{cell_output}
\begin{sphinxVerbatim}[commandchars=\\\{\}]
Index([2016, 2017, 2018, 2019, 2020, 2021, 2022, 2023, 2024, 2025, 2026, 2027,
       2028, 2029, 2030],
      dtype=\PYGZsq{}int64\PYGZsq{})
\end{sphinxVerbatim}

\end{sphinxuseclass}\end{sphinxVerbatimOutput}

\end{sphinxuseclass}
\begin{sphinxuseclass}{cell}\begin{sphinxVerbatimInput}

\begin{sphinxuseclass}{cell_input}
\begin{sphinxVerbatim}[commandchars=\\\{\}]
\PYG{n}{mpak}\PYG{o}{.}\PYG{n}{model\PYGZus{}description}\PYG{o}{=}\PYG{l+s+s2}{\PYGZdq{}}\PYG{l+s+s2}{World Bank climate aware model of Pakistan as described in Burns et al. (2019)}\PYG{l+s+s2}{\PYGZdq{}}
\PYG{n}{mpak}\PYG{o}{.}\PYG{n}{model\PYGZus{}description}
\end{sphinxVerbatim}

\end{sphinxuseclass}\end{sphinxVerbatimInput}
\begin{sphinxVerbatimOutput}

\begin{sphinxuseclass}{cell_output}
\begin{sphinxVerbatim}[commandchars=\\\{\}]
\PYGZsq{}World Bank climate aware model of Pakistan as described in Burns et al. (2019)\PYGZsq{}
\end{sphinxVerbatim}

\end{sphinxuseclass}\end{sphinxVerbatimOutput}

\end{sphinxuseclass}

\section{Information about variables}
\label{\detokenize{content/05_WBModels/LoadingWBModel:information-about-variables}}
\sphinxAtStartPar
The model object \sphinxcode{\sphinxupquote{mpak}} contains lists of all the variables that form part of the model, and these lists can be interrogated to garner information about the model.  The Table below indicates some of the most important of these.  The variables for which information is sought can be specified directly or through a wildcard specification (see note).


\begin{savenotes}\sphinxattablestart
\centering
\begin{tabulary}{\linewidth}[t]{|T|T|T|}
\hline
\sphinxstyletheadfamily 
\sphinxAtStartPar
Method
&\sphinxstyletheadfamily 
\sphinxAtStartPar
Example
&\sphinxstyletheadfamily 
\sphinxAtStartPar
Information returned
\\
\hline
\sphinxAtStartPar
\sphinxcode{\sphinxupquote{.names}}
&
\sphinxAtStartPar
\sphinxcode{\sphinxupquote{modelname{[}'PAKNECON*XN{]}.name}}
&
\sphinxAtStartPar
returns a python list of the mnemnics of all the variables defined and contained in the model object that match the search pattern in the \sphinxcode{\sphinxupquote{{[}{]}}}
\\
\hline
\sphinxAtStartPar
\sphinxcode{\sphinxupquote{.des}}
&
\sphinxAtStartPar
\sphinxcode{\sphinxupquote{modelname{[}'PAKNECONPRVT?N'{]}.des}}
&
\sphinxAtStartPar
returns a list  of mnemonics and their variable descriptions
\\
\hline
\sphinxAtStartPar
\sphinxcode{\sphinxupquote{.<var name>.show}}
&
\sphinxAtStartPar
\sphinxcode{\sphinxupquote{modelname.PAKNECONPRVTXN.show}}
&
\sphinxAtStartPar
Lists the equation (formula), variable descriptions and variable values
\\
\hline
\end{tabulary}
\par
\sphinxattableend\end{savenotes}

\begin{sphinxadmonition}{note}{Note:}
\sphinxAtStartPar
\sphinxstylestrong{Wildcards}

\sphinxAtStartPar
Most of the information commands accept wildcard specifications in the search parameter.

\sphinxAtStartPar
The \sphinxcode{\sphinxupquote{*}} character in the command \sphinxcode{\sphinxupquote{mpak{[}'PAKNECON*XN'{]}.names}} example is a \sphinxcode{\sphinxupquote{wildcard}} character and the expression will return all variables that begin PAKNECON and end XN.

\sphinxAtStartPar
The \sphinxcode{\sphinxupquote{?}} in the \sphinxcode{\sphinxupquote{.des}} example is another wildcard expression. It will match only single characters.  Thus \sphinxcode{\sphinxupquote{mpak{[}'PAKNECONPRVT?N'{]}.names}}  would return three variables: \sphinxcode{\sphinxupquote{PAKNECONPRVTKN}}, \sphinxcode{\sphinxupquote{PAKNECONPRVTXN}}, and \sphinxcode{\sphinxupquote{PAKNECONPRVTXN}}.  The real, current value, and deflators for household consumption expenditure.

\sphinxAtStartPar
Note the final show example uses a slightly different syntax where the variable to be operated upon is specified directly: \sphinxcode{\sphinxupquote{modelname.PAKNECONPRVTXN.show}}.
\end{sphinxadmonition}

\sphinxAtStartPar
The example below returns the mnemonics and descriptions of all variables matching the pattern \sphinxcode{\sphinxupquote{PAKNYGDP*KN}}, i.e. Pakistani variables from the National Income Accounts from the main sub\sphinxhyphen{}category GDP that are also real variables.

\begin{sphinxuseclass}{cell}\begin{sphinxVerbatimInput}

\begin{sphinxuseclass}{cell_input}
\begin{sphinxVerbatim}[commandchars=\\\{\}]
\PYG{n}{mpak}\PYG{p}{[}\PYG{l+s+s1}{\PYGZsq{}}\PYG{l+s+s1}{PAKNYGDP*KN}\PYG{l+s+s1}{\PYGZsq{}}\PYG{p}{]}\PYG{o}{.}\PYG{n}{des}
\end{sphinxVerbatim}

\end{sphinxuseclass}\end{sphinxVerbatimInput}
\begin{sphinxVerbatimOutput}

\begin{sphinxuseclass}{cell_output}
\begin{sphinxVerbatim}[commandchars=\\\{\}]
PAKNYGDPDISCKN : GDP Disc., 2000 LCU mn
PAKNYGDPFCSTKN : GDP Factor Cost Local Currency units Volumes National base year
PAKNYGDPMKTPKN : Real GDP
PAKNYGDPPOTLKN : Potential Output, constant LCU
\end{sphinxVerbatim}

\end{sphinxuseclass}\end{sphinxVerbatimOutput}

\end{sphinxuseclass}
\begin{sphinxadmonition}{note}{Box {[}\textasciicircum{}BoxWBMnemonics{]}: World Bank Mnemonics}

\sphinxAtStartPar
A typical World Bank model will have in excess of 300 variables.  Each has a mnemonic that is structured in a specific way, The root for almost all are 14 characters long (some special variables have additional characters appended to this root) (see discussion in section).
\begin{equation*}
\begin{split}\texttt{12345678901234}\end{split}
\end{equation*}\begin{equation*}
\begin{split}\color{green}{\texttt{CCC}}\color{red}{\texttt{AA}}\color{lime}{\texttt{MMM}}\color{blue}{\texttt{NNNN}}\color{magenta}{\texttt{U}}\color{black}{\texttt{C}}\end{split}
\end{equation*}
\sphinxAtStartPar
where:


\begin{savenotes}\sphinxattablestart
\centering
\begin{tabulary}{\linewidth}[t]{|T|T|}
\hline
\sphinxstyletheadfamily 
\sphinxAtStartPar
Letters
&\sphinxstyletheadfamily 
\sphinxAtStartPar
Meaning
\\
\hline
\sphinxAtStartPar
\(\color{green}{\texttt{CCC}}\)
&
\sphinxAtStartPar
The three\sphinxhyphen{}leter ISO code for a country – i.e. IDN for Indonesia, RUS for Russia
\\
\hline
\sphinxAtStartPar
\(\color{red}{\texttt{AA}}\)
&
\sphinxAtStartPar
The two\sphinxhyphen{}letter major accounting system to which the variable attaches,
\\
\hline
\sphinxAtStartPar
\(\color{lime}{\texttt{MMM}}\)
&
\sphinxAtStartPar
The three\sphinxhyphen{}letter major sub\sphinxhyphen{}category of the data \sphinxhyphen{}  i.e. GDP, EXP \sphinxhyphen{} expenditure
\\
\hline
\sphinxAtStartPar
\(\color{blue}{\texttt{NNNN}}\)
&
\sphinxAtStartPar
The four\sphinxhyphen{}letter minor sub\sphinxhyphen{}category  MKTP for market prices
\\
\hline
\sphinxAtStartPar
\(\color{magenta}{\texttt{U}}\)
&
\sphinxAtStartPar
The measure  (K: real variable;C: Current Values; X: Prices)
\\
\hline
\sphinxAtStartPar
\(\color{black}{\texttt{C}}\)
&
\sphinxAtStartPar
denotes the Currency (N: National currency; D: USD; P: PPP)
\\
\hline
\end{tabulary}
\par
\sphinxattableend\end{savenotes}

\sphinxAtStartPar
Common major accounting systems mnemonics: the, \(\color{red}{\texttt{AA}}\)s from above:


\begin{savenotes}\sphinxattablestart
\centering
\begin{tabulary}{\linewidth}[t]{|T|T|}
\hline
\sphinxstyletheadfamily 
\sphinxAtStartPar
Code
&\sphinxstyletheadfamily 
\sphinxAtStartPar
Meaning
\\
\hline
\sphinxAtStartPar
NY
&
\sphinxAtStartPar
National income
\\
\hline
\sphinxAtStartPar
NE
&
\sphinxAtStartPar
National expenditure Accounts
\\
\hline
\sphinxAtStartPar
NV
&
\sphinxAtStartPar
Value added accounts
\\
\hline
\sphinxAtStartPar
GG
&
\sphinxAtStartPar
General Government Accounts
\\
\hline
\sphinxAtStartPar
BX
&
\sphinxAtStartPar
Balance of Payments: Exports
\\
\hline
\sphinxAtStartPar
BM
&
\sphinxAtStartPar
Balance of Payments: Imports
\\
\hline
\sphinxAtStartPar
BN
&
\sphinxAtStartPar
Balance of Payments: Net
\\
\hline
\sphinxAtStartPar
BF
&
\sphinxAtStartPar
Balance of Payments: Financial Account
\\
\hline
\end{tabulary}
\par
\sphinxattableend\end{savenotes}

\sphinxAtStartPar
Thus


\begin{savenotes}\sphinxattablestart
\centering
\begin{tabulary}{\linewidth}[t]{|T|T|}
\hline
\sphinxstyletheadfamily 
\sphinxAtStartPar
Mnemonic
&\sphinxstyletheadfamily 
\sphinxAtStartPar
Meaning
\\
\hline
\sphinxAtStartPar
IDNNYGDPMKTPKN
&
\sphinxAtStartPar
Indonesia GDP at market prices, real in Indonesian Rupiah
\\
\hline
\sphinxAtStartPar
KENNECPNPRVTXN
&
\sphinxAtStartPar
Kenya Private (household) consumption expenditure schillings deflator
\\
\hline
\sphinxAtStartPar
BOLGGEXPGNFSCN
&
\sphinxAtStartPar
Bolivia Government Expenditure on Goods and services (GNFS) in current Bolivars
\\
\hline
\sphinxAtStartPar
HRVGGREVDCITCN
&
\sphinxAtStartPar
Croatia Government Revenues Direct Corporate Income Taxes in current Euros
\\
\hline
\sphinxAtStartPar
NPLBXGSRNFSVCD
&
\sphinxAtStartPar
Nepal BOP Exports of non\sphinxhyphen{}factor services (goods and services) in current USD
\\
\hline
\end{tabulary}
\par
\sphinxattableend\end{savenotes}
\end{sphinxadmonition}

\sphinxAtStartPar
If executed, the command \sphinxcode{\sphinxupquote{mpak{[}'*'{]}.des}} would return a dictionary of all the mnemonics and descriptions of all the variables in the \sphinxcode{\sphinxupquote{mpak}} model object.


\subsection{The \sphinxstyleliteralintitle{\sphinxupquote{!}} operator – searching on the variable description}
\label{\detokenize{content/05_WBModels/LoadingWBModel:the-operator-searching-on-the-variable-description}}
\sphinxAtStartPar
The same methods can be used to retrieve information about variables, based on their descriptions (vs mnemonic), by pre\sphinxhyphen{}pending the search string with the  \sphinxcode{\sphinxupquote{!}} operator.

\begin{sphinxadmonition}{note}{Note:}
\sphinxAtStartPar
\sphinxstylestrong{The ! operator}
If a wildcard is preceded by an exclamation mark \sphinxstylestrong{!} the search will be done over the description of variables instead of the mnemonic
\end{sphinxadmonition}

\sphinxAtStartPar
The below expression returns all variables whose description includes the word Carbon.

\begin{sphinxuseclass}{cell}\begin{sphinxVerbatimInput}

\begin{sphinxuseclass}{cell_input}
\begin{sphinxVerbatim}[commandchars=\\\{\}]
\PYG{n}{mpak}\PYG{p}{[}\PYG{l+s+s1}{\PYGZsq{}}\PYG{l+s+s1}{!*Carbon*}\PYG{l+s+s1}{\PYGZsq{}}\PYG{p}{]}\PYG{o}{.}\PYG{n}{des}
\end{sphinxVerbatim}

\end{sphinxuseclass}\end{sphinxVerbatimInput}
\begin{sphinxVerbatimOutput}

\begin{sphinxuseclass}{cell_output}
\begin{sphinxVerbatim}[commandchars=\\\{\}]
PAKGGREVCO2CER : Carbon tax on coal (USD/t)
PAKGGREVCO2GER : Carbon tax on gas (USD/t)
PAKGGREVCO2OER : Carbon tax on oil (USD/t)
\end{sphinxVerbatim}

\end{sphinxuseclass}\end{sphinxVerbatimOutput}

\end{sphinxuseclass}

\section{Groups}
\label{\detokenize{content/05_WBModels/LoadingWBModel:groups}}
\sphinxAtStartPar
Modelflow inherits a variant of the idea of groups from \sphinxcode{\sphinxupquote{EViews}}.  In \sphinxcode{\sphinxupquote{modelflow}} the groups defined in an imported EViews workfile are converted into entries in a dictionary called \sphinxcode{\sphinxupquote{var\_groups}} which can be interrogated, added to and amended like any dictionary in python.

\sphinxAtStartPar
The command
\sphinxcode{\sphinxupquote{mpak.var\_groups}} will return all of the groups already defined in mpak.

\begin{sphinxuseclass}{cell}\begin{sphinxVerbatimInput}

\begin{sphinxuseclass}{cell_input}
\begin{sphinxVerbatim}[commandchars=\\\{\}]
\PYG{n}{mpak}\PYG{o}{.}\PYG{n}{var\PYGZus{}groups}
\end{sphinxVerbatim}

\end{sphinxuseclass}\end{sphinxVerbatimInput}
\begin{sphinxVerbatimOutput}

\begin{sphinxuseclass}{cell_output}
\begin{sphinxVerbatim}[commandchars=\\\{\}]
\PYGZob{}\PYGZsq{}Headline\PYGZsq{}: \PYGZsq{}???GDPpckn ???NRTOTLCN ???LMEMPTOTL ???BFFINCABDCD  ???BFBOPTOTLCD ???GGBALEXGRCN ???BNCABLOCLCD\PYGZus{} ???FPCPITOTLXN\PYGZsq{},
 \PYGZsq{}National income accounts\PYGZsq{}: \PYGZsq{}???NY*\PYGZsq{},
 \PYGZsq{}National expenditure accounts\PYGZsq{}: \PYGZsq{}???NE*\PYGZsq{},
 \PYGZsq{}Value added accounts\PYGZsq{}: \PYGZsq{}???NV*\PYGZsq{},
 \PYGZsq{}Balance of payments exports\PYGZsq{}: \PYGZsq{}???BX*\PYGZsq{},
 \PYGZsq{}Balance of payments exports and value added \PYGZsq{}: \PYGZsq{}???BX* ???NV*\PYGZsq{},
 \PYGZsq{}Balance of Payments Financial Account\PYGZsq{}: \PYGZsq{}???BF*\PYGZsq{},
 \PYGZsq{}General government fiscal accounts\PYGZsq{}: \PYGZsq{}???GG*\PYGZsq{},
 \PYGZsq{}World all\PYGZsq{}: \PYGZsq{}WLD*\PYGZsq{},
 \PYGZsq{}PAK all\PYGZsq{}: \PYGZsq{}PAK*\PYGZsq{}\PYGZcb{}
\end{sphinxVerbatim}

\end{sphinxuseclass}\end{sphinxVerbatimOutput}

\end{sphinxuseclass}
\sphinxAtStartPar
A group can be added to the dictionary by giving it a unique identifier (key) and associating with it a string defining the group, using a wildcard specification or just a space de\sphinxhyphen{}limited list of mnemonics.

\sphinxAtStartPar
Thus the command

\sphinxAtStartPar
\sphinxcode{\sphinxupquote{mpak.var\_groups{[}'Mygroup{]}='PAKGGREV*CN PAKGGEBALOVRLCN'}}

\sphinxAtStartPar
Will create a new group containing the variables matching the pattern

\begin{sphinxuseclass}{cell}\begin{sphinxVerbatimInput}

\begin{sphinxuseclass}{cell_input}
\begin{sphinxVerbatim}[commandchars=\\\{\}]
\PYG{n}{mpak}\PYG{o}{.}\PYG{n}{var\PYGZus{}groups}\PYG{p}{[}\PYG{l+s+s1}{\PYGZsq{}}\PYG{l+s+s1}{Mygroup}\PYG{l+s+s1}{\PYGZsq{}}\PYG{p}{]}\PYG{o}{=}\PYG{l+s+s1}{\PYGZsq{}}\PYG{l+s+s1}{PAKGGREV*CN PAKGGEBALOVRLCN}\PYG{l+s+s1}{\PYGZsq{}}
                
\end{sphinxVerbatim}

\end{sphinxuseclass}\end{sphinxVerbatimInput}

\end{sphinxuseclass}
\begin{sphinxuseclass}{cell}\begin{sphinxVerbatimInput}

\begin{sphinxuseclass}{cell_input}
\begin{sphinxVerbatim}[commandchars=\\\{\}]
\PYG{n}{mpak}\PYG{p}{[}\PYG{l+s+s1}{\PYGZsq{}}\PYG{l+s+s1}{\PYGZsh{}Mygroup}\PYG{l+s+s1}{\PYGZsq{}}\PYG{p}{]}\PYG{o}{.}\PYG{n}{names}
\end{sphinxVerbatim}

\end{sphinxuseclass}\end{sphinxVerbatimInput}
\begin{sphinxVerbatimOutput}

\begin{sphinxuseclass}{cell_output}
\begin{sphinxVerbatim}[commandchars=\\\{\}]
[\PYGZsq{}PAKGGREVDRCTCN\PYGZsq{},
 \PYGZsq{}PAKGGREVEMISCN\PYGZsq{},
 \PYGZsq{}PAKGGREVGNFSCN\PYGZsq{},
 \PYGZsq{}PAKGGREVGRNTCN\PYGZsq{},
 \PYGZsq{}PAKGGREVOTHRCN\PYGZsq{},
 \PYGZsq{}PAKGGREVTOTLCN\PYGZsq{},
 \PYGZsq{}PAKGGREVTRDECN\PYGZsq{}]
\end{sphinxVerbatim}

\end{sphinxuseclass}\end{sphinxVerbatimOutput}

\end{sphinxuseclass}

\section{Information about data}
\label{\detokenize{content/05_WBModels/LoadingWBModel:information-about-data}}
\sphinxAtStartPar
Note the same search functions can be used to display the data associated with the returned variables.

\sphinxAtStartPar
Thus to see the data for the \sphinxcode{\sphinxupquote{Mygroup}} group of variables, one could use the \sphinxcode{\sphinxupquote{.df}} method or \sphinxcode{\sphinxupquote{.plot()}} methods – here modified by pct (to show growth rates) and mul100 to multiply them by 100 to display them as percent change.

\begin{sphinxuseclass}{cell}\begin{sphinxVerbatimInput}

\begin{sphinxuseclass}{cell_input}
\begin{sphinxVerbatim}[commandchars=\\\{\}]
\PYG{n}{mpak}\PYG{p}{[}\PYG{l+s+s1}{\PYGZsq{}}\PYG{l+s+s1}{\PYGZsh{}Mygroup}\PYG{l+s+s1}{\PYGZsq{}}\PYG{p}{]}\PYG{o}{.}\PYG{n}{pct}\PYG{o}{.}\PYG{n}{mul100}\PYG{o}{.}\PYG{n}{plot}\PYG{p}{(}\PYG{p}{)}
\end{sphinxVerbatim}

\end{sphinxuseclass}\end{sphinxVerbatimInput}
\begin{sphinxVerbatimOutput}

\begin{sphinxuseclass}{cell_output}
\noindent\sphinxincludegraphics{{1d2f04bb581f8d29beb1f0e3b2e49b09c51c3e2c0d137a42ee517a0517b6affd}.png}

\end{sphinxuseclass}\end{sphinxVerbatimOutput}

\end{sphinxuseclass}
\sphinxAtStartPar
Below the same logic is used to display the data from variables matching a mnemonic search.  The results have been placed inside a \sphinxcode{\sphinxupquote{with m{[}pak.set\_smpl()}} clause to restrict the output to a shorter period.  If it was not used the output would cover the whole time period of the \sphinxcode{\sphinxupquote{.lastdf}} DataFrame from which all of this data is drawm.

\begin{sphinxuseclass}{cell}\begin{sphinxVerbatimInput}

\begin{sphinxuseclass}{cell_input}
\begin{sphinxVerbatim}[commandchars=\\\{\}]
\PYG{k}{with} \PYG{n}{mpak}\PYG{o}{.}\PYG{n}{set\PYGZus{}smpl}\PYG{p}{(}\PYG{l+m+mi}{2020}\PYG{p}{,}\PYG{l+m+mi}{2024}\PYG{p}{)}\PYG{p}{:}
    \PYG{n+nb}{print}\PYG{p}{(}\PYG{n+nb}{round}\PYG{p}{(}\PYG{n}{mpak}\PYG{p}{[}\PYG{l+s+s1}{\PYGZsq{}}\PYG{l+s+s1}{\PYGZsh{}Mygroup}\PYG{l+s+s1}{\PYGZsq{}}\PYG{p}{]}\PYG{o}{.}\PYG{n}{pct}\PYG{o}{.}\PYG{n}{mul100}\PYG{o}{.}\PYG{n}{df}\PYG{p}{,}\PYG{l+m+mi}{2}\PYG{p}{)}\PYG{p}{)}
\end{sphinxVerbatim}

\end{sphinxuseclass}\end{sphinxVerbatimInput}
\begin{sphinxVerbatimOutput}

\begin{sphinxuseclass}{cell_output}
\begin{sphinxVerbatim}[commandchars=\\\{\}]
      PAKGGREVDRCTCN  PAKGGREVEMISCN  PAKGGREVGNFSCN  PAKGGREVGRNTCN   
2020           13.30            1.10           13.25           39.48  \PYGZbs{}
2021           11.69            0.21           11.33           29.52   
2022           10.48            0.28           10.11           23.40   
2023            9.84            0.82            9.60           19.62   
2024            9.48            1.42            9.36           17.09   

      PAKGGREVOTHRCN  PAKGGREVTOTLCN  PAKGGREVTRDECN  
2020           17.83           16.77           18.25  
2021           15.34           14.39           15.28  
2022           13.45           12.69           13.71  
2023           12.29           11.72           12.89  
2024           11.51           11.10           12.35  
\end{sphinxVerbatim}

\end{sphinxuseclass}\end{sphinxVerbatimOutput}

\end{sphinxuseclass}
\sphinxAtStartPar
Jupyter truncates the output by showing the first and last five observations of the active sample period when the same call is  made without the with clause.

\begin{sphinxuseclass}{cell}\begin{sphinxVerbatimInput}

\begin{sphinxuseclass}{cell_input}
\begin{sphinxVerbatim}[commandchars=\\\{\}]
\PYG{n}{mpak}\PYG{o}{.}\PYG{n}{smpl}\PYG{p}{(}\PYG{l+m+mi}{2000}\PYG{p}{,}\PYG{l+m+mi}{2100}\PYG{p}{)}
\PYG{n+nb}{print}\PYG{p}{(}\PYG{n+nb}{round}\PYG{p}{(}\PYG{n}{mpak}\PYG{p}{[}\PYG{l+s+s1}{\PYGZsq{}}\PYG{l+s+s1}{\PYGZsh{}Mygroup}\PYG{l+s+s1}{\PYGZsq{}}\PYG{p}{]}\PYG{o}{.}\PYG{n}{pct}\PYG{o}{.}\PYG{n}{mul100}\PYG{o}{.}\PYG{n}{df}\PYG{p}{,}\PYG{l+m+mi}{2}\PYG{p}{)}\PYG{p}{)}
\end{sphinxVerbatim}

\end{sphinxuseclass}\end{sphinxVerbatimInput}
\begin{sphinxVerbatimOutput}

\begin{sphinxuseclass}{cell_output}
\begin{sphinxVerbatim}[commandchars=\\\{\}]
      PAKGGREVDRCTCN  PAKGGREVEMISCN  PAKGGREVGNFSCN  PAKGGREVGRNTCN   
2000            9.55          101.83           70.02             NaN  \PYGZbs{}
2001           11.14           15.37           31.46             inf   
2002           14.66          \PYGZhy{}13.23            8.55           94.82   
2003            7.11           35.47           17.12          \PYGZhy{}36.96   
2004            8.43           21.64           13.05          \PYGZhy{}39.98   
...              ...             ...             ...             ...   
2096            9.03            2.84            9.07            9.03   
2097            9.02            2.84            9.06            9.02   
2098            9.02            2.84            9.06            9.02   
2099            9.01            2.84            9.06            9.01   
2100            9.01            2.84            9.05            9.01   

      PAKGGREVOTHRCN  PAKGGREVTOTLCN  PAKGGREVTRDECN  
2000             NaN            7.30          \PYGZhy{}21.68  
2001             inf           16.34            5.52  
2002           17.59           22.84          \PYGZhy{}26.44  
2003           15.20            6.04           43.96  
2004           26.30           15.68           32.11  
...              ...             ...             ...  
2096            9.03            9.03            8.96  
2097            9.02            9.02            8.96  
2098            9.02            9.02            8.95  
2099            9.01            9.02            8.95  
2100            9.01            9.01            8.95  

[101 rows x 7 columns]
\end{sphinxVerbatim}

\end{sphinxuseclass}\end{sphinxVerbatimOutput}

\end{sphinxuseclass}

\subsection{Some examples}
\label{\detokenize{content/05_WBModels/LoadingWBModel:some-examples}}

\subsubsection{\sphinxstyleliteralintitle{\sphinxupquote{.names}} property}
\label{\detokenize{content/05_WBModels/LoadingWBModel:names-property}}
\sphinxAtStartPar
\sphinxcode{\sphinxupquote{mpak{[}'PAKNECON*XN'{]}.names}}

\sphinxAtStartPar
Return the names (mnemonmics) of all variables that begin \sphinxcode{\sphinxupquote{PAKNECON}} and end \sphinxcode{\sphinxupquote{XN}} – i.e. Price deflators for various types of consumption demand.

\begin{sphinxuseclass}{cell}\begin{sphinxVerbatimInput}

\begin{sphinxuseclass}{cell_input}
\begin{sphinxVerbatim}[commandchars=\\\{\}]
\PYG{n}{mpak}\PYG{p}{[}\PYG{l+s+s1}{\PYGZsq{}}\PYG{l+s+s1}{PAKNECON*XN}\PYG{l+s+s1}{\PYGZsq{}}\PYG{p}{]}\PYG{o}{.}\PYG{n}{names}
\end{sphinxVerbatim}

\end{sphinxuseclass}\end{sphinxVerbatimInput}
\begin{sphinxVerbatimOutput}

\begin{sphinxuseclass}{cell_output}
\begin{sphinxVerbatim}[commandchars=\\\{\}]
[\PYGZsq{}PAKNECONENGYXN\PYGZsq{}, \PYGZsq{}PAKNECONGOVTXN\PYGZsq{}, \PYGZsq{}PAKNECONOTHRXN\PYGZsq{}, \PYGZsq{}PAKNECONPRVTXN\PYGZsq{}]
\end{sphinxVerbatim}

\end{sphinxuseclass}\end{sphinxVerbatimOutput}

\end{sphinxuseclass}

\subsubsection{The \sphinxstyleliteralintitle{\sphinxupquote{.des}} property}
\label{\detokenize{content/05_WBModels/LoadingWBModel:the-des-property}}
\sphinxAtStartPar
\sphinxcode{\sphinxupquote{mpak{[}'PAKNECONPRVT?N'{]}.des}}

\sphinxAtStartPar
Returns a dictionary comprised of the mnemonics and the descriptions of all the variables that begin \sphinxcode{\sphinxupquote{PAKNECONPRVT}} and end \sphinxcode{\sphinxupquote{N}}, but have only one character between the T and the N.

\begin{sphinxuseclass}{cell}\begin{sphinxVerbatimInput}

\begin{sphinxuseclass}{cell_input}
\begin{sphinxVerbatim}[commandchars=\\\{\}]
\PYG{n}{mpak}\PYG{p}{[}\PYG{l+s+s1}{\PYGZsq{}}\PYG{l+s+s1}{PAKNECONPRVT?N}\PYG{l+s+s1}{\PYGZsq{}}\PYG{p}{]}\PYG{o}{.}\PYG{n}{des}
\end{sphinxVerbatim}

\end{sphinxuseclass}\end{sphinxVerbatimInput}
\begin{sphinxVerbatimOutput}

\begin{sphinxuseclass}{cell_output}
\begin{sphinxVerbatim}[commandchars=\\\{\}]
PAKNECONPRVTCN : Pvt. Cons., LCU mn
PAKNECONPRVTKN : HH. Cons Real
PAKNECONPRVTXN : Implicit LCU defl., Pvt. Cons., 2000 = 1
\end{sphinxVerbatim}

\end{sphinxuseclass}\end{sphinxVerbatimOutput}

\end{sphinxuseclass}

\subsubsection{\sphinxstyleliteralintitle{\sphinxupquote{.var\_description}} method}
\label{\detokenize{content/05_WBModels/LoadingWBModel:var-description-method}}
\sphinxAtStartPar
The property \sphinxcode{\sphinxupquote{.var\_description}}returns a dictionaryy with descriptions of all variables.  Modified to a specific variable it returns the description of that one variable.  As a dictionary it do not accept wildcards as arguments.

\begin{sphinxuseclass}{cell}\begin{sphinxVerbatimInput}

\begin{sphinxuseclass}{cell_input}
\begin{sphinxVerbatim}[commandchars=\\\{\}]
\PYG{c+c1}{\PYGZsh{}mpak.var\PYGZus{}description \PYGZsh{} returns the descirptions for all variables}
\PYG{n}{mpak}\PYG{o}{.}\PYG{n}{var\PYGZus{}description}\PYG{p}{[}\PYG{l+s+s1}{\PYGZsq{}}\PYG{l+s+s1}{PAKNYGDPMKTPCN}\PYG{l+s+s1}{\PYGZsq{}}\PYG{p}{]} \PYG{c+c1}{\PYGZsh{} returns the description of a specific variable}
\end{sphinxVerbatim}

\end{sphinxuseclass}\end{sphinxVerbatimInput}
\begin{sphinxVerbatimOutput}

\begin{sphinxuseclass}{cell_output}
\begin{sphinxVerbatim}[commandchars=\\\{\}]
\PYGZsq{}GDP, Market Prices, LCU mn\PYGZsq{}
\end{sphinxVerbatim}

\end{sphinxuseclass}\end{sphinxVerbatimOutput}

\end{sphinxuseclass}

\section{Information about equations}
\label{\detokenize{content/05_WBModels/LoadingWBModel:information-about-equations}}
\sphinxAtStartPar
For endogenous variables the property: \sphinxcode{\sphinxupquote{.<variable name>.frml}} returns informations on the equation for the variable.

\begin{sphinxuseclass}{cell}\begin{sphinxVerbatimInput}

\begin{sphinxuseclass}{cell_input}
\begin{sphinxVerbatim}[commandchars=\\\{\}]
\PYG{n}{mpak}\PYG{o}{.}\PYG{n}{PAKNECONPRVTKN}\PYG{o}{.}\PYG{n}{frml}
\end{sphinxVerbatim}

\end{sphinxuseclass}\end{sphinxVerbatimInput}
\begin{sphinxVerbatimOutput}

\begin{sphinxuseclass}{cell_output}
\begin{sphinxVerbatim}[commandchars=\\\{\}]
Endogeneous: PAKNECONPRVTKN: HH. Cons Real
Formular: FRML \PYGZlt{}DAMP,STOC\PYGZgt{} PAKNECONPRVTKN = (PAKNECONPRVTKN(\PYGZhy{}1)*EXP(PAKNECONPRVTKN\PYGZus{}A+ (\PYGZhy{}0.2*(LOG(PAKNECONPRVTKN(\PYGZhy{}1))\PYGZhy{}LOG(1.21203101101442)\PYGZhy{}LOG((((PAKBXFSTREMTCD(\PYGZhy{}1)\PYGZhy{}PAKBMFSTREMTCD(\PYGZhy{}1))*PAKPANUSATLS(\PYGZhy{}1))+PAKGGEXPTRNSCN(\PYGZhy{}1)+PAKNYYWBTOTLCN(\PYGZhy{}1)*(1\PYGZhy{}PAKGGREVDRCTXN(\PYGZhy{}1)/100))/PAKNECONPRVTXN(\PYGZhy{}1)))+0.763938860758873*((LOG((((PAKBXFSTREMTCD\PYGZhy{}PAKBMFSTREMTCD)*PAKPANUSATLS)+PAKGGEXPTRNSCN+PAKNYYWBTOTLCN*(1\PYGZhy{}PAKGGREVDRCTXN/100))/PAKNECONPRVTXN))\PYGZhy{}(LOG((((PAKBXFSTREMTCD(\PYGZhy{}1)\PYGZhy{}PAKBMFSTREMTCD(\PYGZhy{}1))*PAKPANUSATLS(\PYGZhy{}1))+PAKGGEXPTRNSCN(\PYGZhy{}1)+PAKNYYWBTOTLCN(\PYGZhy{}1)*(1\PYGZhy{}PAKGGREVDRCTXN(\PYGZhy{}1)/100))/PAKNECONPRVTXN(\PYGZhy{}1))))\PYGZhy{}0.0634474791568939*DURING\PYGZus{}2009\PYGZhy{}0.3*(PAKFMLBLPOLYXN/100\PYGZhy{}((LOG(PAKNECONPRVTXN))\PYGZhy{}(LOG(PAKNECONPRVTXN(\PYGZhy{}1)))))) )) * (1\PYGZhy{}PAKNECONPRVTKN\PYGZus{}D)+ PAKNECONPRVTKN\PYGZus{}X*PAKNECONPRVTKN\PYGZus{}D  \PYGZdl{}

PAKNECONPRVTKN  : HH. Cons Real
DURING\PYGZus{}2009     : 
PAKBMFSTREMTCD  : Imp., Remittances (BOP), US\PYGZdl{} mn
PAKBXFSTREMTCD  : Exp., Remittances (BOP), US\PYGZdl{} mn
PAKFMLBLPOLYXN  : Key Policy Interest Rate
PAKGGEXPTRNSCN  : Current Transfers
PAKGGREVDRCTXN  : Direct Revenue Tax Rate
PAKNECONPRVTKN\PYGZus{}A: Add factor:HH. Cons Real
PAKNECONPRVTKN\PYGZus{}D: Fix dummy:HH. Cons Real
PAKNECONPRVTKN\PYGZus{}X: Fix value:HH. Cons Real
PAKNECONPRVTXN  : Implicit LCU defl., Pvt. Cons., 2000 = 1
PAKNYYWBTOTLCN  : Total Wage Bill
PAKPANUSATLS    : Exchange rate LCU / US\PYGZdl{} \PYGZhy{} Pakistan
\end{sphinxVerbatim}

\end{sphinxuseclass}\end{sphinxVerbatimOutput}

\end{sphinxuseclass}

\subsection{The \sphinxstyleliteralintitle{\sphinxupquote{endogene}} property}
\label{\detokenize{content/05_WBModels/LoadingWBModel:the-endogene-property}}
\sphinxAtStartPar
The  \sphinxcode{\sphinxupquote{endogene}} property either returns a list of all variables in the model that are endogenous (have an equation). It can also be used to test whether a a specific mnemonic has an equation associated with it.

\sphinxAtStartPar
The expression \sphinxcode{\sphinxupquote{'PAKNECONPRVTKN' in mpak.endogene}} returns True if the passed mnemonic is in the list returned by \sphinxcode{\sphinxupquote{mpak.endogene}}.

\begin{sphinxuseclass}{cell}\begin{sphinxVerbatimInput}

\begin{sphinxuseclass}{cell_input}
\begin{sphinxVerbatim}[commandchars=\\\{\}]
\PYG{l+s+s1}{\PYGZsq{}}\PYG{l+s+s1}{PAKNECONPRVTKN}\PYG{l+s+s1}{\PYGZsq{}} \PYG{o+ow}{in} \PYG{n}{mpak}\PYG{o}{.}\PYG{n}{endogene}
\end{sphinxVerbatim}

\end{sphinxuseclass}\end{sphinxVerbatimInput}
\begin{sphinxVerbatimOutput}

\begin{sphinxuseclass}{cell_output}
\begin{sphinxVerbatim}[commandchars=\\\{\}]
True
\end{sphinxVerbatim}

\end{sphinxuseclass}\end{sphinxVerbatimOutput}

\end{sphinxuseclass}

\subsection{Retrieving info on many equations}
\label{\detokenize{content/05_WBModels/LoadingWBModel:retrieving-info-on-many-equations}}
\sphinxAtStartPar
There are three functions to extract the equations from a model. They all operates on the selection pattern menionend above.


\begin{savenotes}\sphinxattablestart
\centering
\begin{tabulary}{\linewidth}[t]{|T|T|}
\hline
\sphinxstyletheadfamily 
\sphinxAtStartPar
Command
&\sphinxstyletheadfamily 
\sphinxAtStartPar
Effect
\\
\hline
\sphinxAtStartPar
\sphinxcode{\sphinxupquote{mpak{[}'PAKNECONPRVTKN'{]}.frml}}
&
\sphinxAtStartPar
Returns a \sphinxstylestrong{normalized} version of the equation (the one actually used in modelflow)
\\
\hline
\sphinxAtStartPar
\sphinxcode{\sphinxupquote{mpak{[}'PAKNECONPRVTKN'{]}.eviews}}
&
\sphinxAtStartPar
In models imported from Eviews, reports the original eviews specification
\\
\hline
\sphinxAtStartPar
\sphinxcode{\sphinxupquote{mpak.PAKNECONPRVTXN.show}}
&
\sphinxAtStartPar
The equation (formula), variable descriptions variable values
\\
\hline
\end{tabulary}
\par
\sphinxattableend\end{savenotes}

\sphinxAtStartPar
The equation for consumption in \sphinxcode{\sphinxupquote{mpak}} we see that it follows something very close to this formulation.


\subsubsection{The \sphinxstyleliteralintitle{\sphinxupquote{.frml}} property}
\label{\detokenize{content/05_WBModels/LoadingWBModel:the-frml-property}}
\sphinxAtStartPar
The \sphinxcode{\sphinxupquote{.frml}} method returns the normalized equation that is actually used in modelflow.  In this instance it is not called for the results of a search operation but by referencing directly the equation (which is itself a property of the mpak model).

\sphinxAtStartPar
Note that following the normalized equation is a listing of all the dependent variables of the equation.

\begin{sphinxuseclass}{cell}\begin{sphinxVerbatimInput}

\begin{sphinxuseclass}{cell_input}
\begin{sphinxVerbatim}[commandchars=\\\{\}]
\PYG{n}{mpak}\PYG{o}{.}\PYG{n}{PAKNECONPRVTKN}\PYG{o}{.}\PYG{n}{frml}
\end{sphinxVerbatim}

\end{sphinxuseclass}\end{sphinxVerbatimInput}
\begin{sphinxVerbatimOutput}

\begin{sphinxuseclass}{cell_output}
\begin{sphinxVerbatim}[commandchars=\\\{\}]
Endogeneous: PAKNECONPRVTKN: HH. Cons Real
Formular: FRML \PYGZlt{}DAMP,STOC\PYGZgt{} PAKNECONPRVTKN = (PAKNECONPRVTKN(\PYGZhy{}1)*EXP(PAKNECONPRVTKN\PYGZus{}A+ (\PYGZhy{}0.2*(LOG(PAKNECONPRVTKN(\PYGZhy{}1))\PYGZhy{}LOG(1.21203101101442)\PYGZhy{}LOG((((PAKBXFSTREMTCD(\PYGZhy{}1)\PYGZhy{}PAKBMFSTREMTCD(\PYGZhy{}1))*PAKPANUSATLS(\PYGZhy{}1))+PAKGGEXPTRNSCN(\PYGZhy{}1)+PAKNYYWBTOTLCN(\PYGZhy{}1)*(1\PYGZhy{}PAKGGREVDRCTXN(\PYGZhy{}1)/100))/PAKNECONPRVTXN(\PYGZhy{}1)))+0.763938860758873*((LOG((((PAKBXFSTREMTCD\PYGZhy{}PAKBMFSTREMTCD)*PAKPANUSATLS)+PAKGGEXPTRNSCN+PAKNYYWBTOTLCN*(1\PYGZhy{}PAKGGREVDRCTXN/100))/PAKNECONPRVTXN))\PYGZhy{}(LOG((((PAKBXFSTREMTCD(\PYGZhy{}1)\PYGZhy{}PAKBMFSTREMTCD(\PYGZhy{}1))*PAKPANUSATLS(\PYGZhy{}1))+PAKGGEXPTRNSCN(\PYGZhy{}1)+PAKNYYWBTOTLCN(\PYGZhy{}1)*(1\PYGZhy{}PAKGGREVDRCTXN(\PYGZhy{}1)/100))/PAKNECONPRVTXN(\PYGZhy{}1))))\PYGZhy{}0.0634474791568939*DURING\PYGZus{}2009\PYGZhy{}0.3*(PAKFMLBLPOLYXN/100\PYGZhy{}((LOG(PAKNECONPRVTXN))\PYGZhy{}(LOG(PAKNECONPRVTXN(\PYGZhy{}1)))))) )) * (1\PYGZhy{}PAKNECONPRVTKN\PYGZus{}D)+ PAKNECONPRVTKN\PYGZus{}X*PAKNECONPRVTKN\PYGZus{}D  \PYGZdl{}

PAKNECONPRVTKN  : HH. Cons Real
DURING\PYGZus{}2009     : 
PAKBMFSTREMTCD  : Imp., Remittances (BOP), US\PYGZdl{} mn
PAKBXFSTREMTCD  : Exp., Remittances (BOP), US\PYGZdl{} mn
PAKFMLBLPOLYXN  : Key Policy Interest Rate
PAKGGEXPTRNSCN  : Current Transfers
PAKGGREVDRCTXN  : Direct Revenue Tax Rate
PAKNECONPRVTKN\PYGZus{}A: Add factor:HH. Cons Real
PAKNECONPRVTKN\PYGZus{}D: Fix dummy:HH. Cons Real
PAKNECONPRVTKN\PYGZus{}X: Fix value:HH. Cons Real
PAKNECONPRVTXN  : Implicit LCU defl., Pvt. Cons., 2000 = 1
PAKNYYWBTOTLCN  : Total Wage Bill
PAKPANUSATLS    : Exchange rate LCU / US\PYGZdl{} \PYGZhy{} Pakistan
\end{sphinxVerbatim}

\end{sphinxuseclass}\end{sphinxVerbatimOutput}

\end{sphinxuseclass}

\subsubsection{The \sphinxstyleliteralintitle{\sphinxupquote{.eviews}} method}
\label{\detokenize{content/05_WBModels/LoadingWBModel:the-eviews-method}}
\sphinxAtStartPar
The \sphinxcode{\sphinxupquote{mpak{[}'PAKNECONPRVTKN'{]}.eviews}} command returns the equations before they were normalized. In most cases this is a slightly more legible form. Here following the EViews syntax, \(\Delta ln()\) is written as dlog().

\begin{sphinxuseclass}{cell}\begin{sphinxVerbatimInput}

\begin{sphinxuseclass}{cell_input}
\begin{sphinxVerbatim}[commandchars=\\\{\}]
\PYG{n}{mpak}\PYG{p}{[}\PYG{l+s+s1}{\PYGZsq{}}\PYG{l+s+s1}{PAKNECONPRVTKN}\PYG{l+s+s1}{\PYGZsq{}}\PYG{p}{]}\PYG{o}{.}\PYG{n}{eviews}
\end{sphinxVerbatim}

\end{sphinxuseclass}\end{sphinxVerbatimInput}
\begin{sphinxVerbatimOutput}

\begin{sphinxuseclass}{cell_output}
\begin{sphinxVerbatim}[commandchars=\\\{\}]
PAKNECONPRVTKN : DLOG(PAKNECONPRVTKN) =\PYGZhy{} 0.2*(LOG(PAKNECONPRVTKN( \PYGZhy{} 1)) \PYGZhy{} LOG(1.21203101101442) \PYGZhy{} LOG((((PAKBXFSTREMTCD( \PYGZhy{} 1) \PYGZhy{} PAKBMFSTREMTCD( \PYGZhy{} 1))*PAKPANUSATLS( \PYGZhy{} 1)) + PAKGGEXPTRNSCN( \PYGZhy{} 1) + PAKNYYWBTOTLCN( \PYGZhy{} 1)*(1 \PYGZhy{} PAKGGREVDRCTXN( \PYGZhy{} 1)/100))/PAKNECONPRVTXN( \PYGZhy{} 1))) + 0.763938860758873*DLOG((((PAKBXFSTREMTCD \PYGZhy{} PAKBMFSTREMTCD)*PAKPANUSATLS) + PAKGGEXPTRNSCN + PAKNYYWBTOTLCN*(1 \PYGZhy{} PAKGGREVDRCTXN/100))/PAKNECONPRVTXN) \PYGZhy{} 0.0634474791568939*@DURING(\PYGZdq{}2009\PYGZdq{}) \PYGZhy{} 0.3*(PAKFMLBLPOLYXN/100 \PYGZhy{} DLOG(PAKNECONPRVTXN))
\end{sphinxVerbatim}

\end{sphinxuseclass}\end{sphinxVerbatimOutput}

\end{sphinxuseclass}

\subsubsection{The \sphinxstyleliteralintitle{\sphinxupquote{.show}} method}
\label{\detokenize{content/05_WBModels/LoadingWBModel:the-show-method}}
\sphinxAtStartPar
The \sphinxcode{\sphinxupquote{.show}} method returns:
\begin{enumerate}
\sphinxsetlistlabels{\arabic}{enumi}{enumii}{}{.}%
\item {} 
\sphinxAtStartPar
The description of the variable

\item {} 
\sphinxAtStartPar
The normalized equation that is actually used in modelflow.

\item {} 
\sphinxAtStartPar
A listing of the mnemonics and descriptions of the RHS variables

\item {} 
\sphinxAtStartPar
The data of that variable (drawn from the \sphinxcode{\sphinxupquote{basedf}} and \sphinxcode{\sphinxupquote{.lastdf}} DataFrames in the model object as well as the data of the RHS variables of the equation from both the \sphinxcode{\sphinxupquote{basedf}} and \sphinxcode{\sphinxupquote{.lastdf}} DataFrames.

\end{enumerate}

\begin{sphinxuseclass}{cell}\begin{sphinxVerbatimInput}

\begin{sphinxuseclass}{cell_input}
\begin{sphinxVerbatim}[commandchars=\\\{\}]
\PYG{n}{mpak}\PYG{o}{.}\PYG{n}{smpl}\PYG{p}{(}\PYG{l+m+mi}{2020}\PYG{p}{,}\PYG{l+m+mi}{2025}\PYG{p}{)} \PYG{c+c1}{\PYGZsh{}change the actual sample range to limit the number of columns displayed}
\PYG{n}{mpak}\PYG{o}{.}\PYG{n}{PAKNECONPRVTKN}\PYG{o}{.}\PYG{n}{show}
\end{sphinxVerbatim}

\end{sphinxuseclass}\end{sphinxVerbatimInput}
\begin{sphinxVerbatimOutput}

\begin{sphinxuseclass}{cell_output}
\begin{sphinxVerbatim}[commandchars=\\\{\}]
Endogeneous: PAKNECONPRVTKN: HH. Cons Real
Formular: FRML \PYGZlt{}DAMP,STOC\PYGZgt{} PAKNECONPRVTKN = (PAKNECONPRVTKN(\PYGZhy{}1)*EXP(PAKNECONPRVTKN\PYGZus{}A+ (\PYGZhy{}0.2*(LOG(PAKNECONPRVTKN(\PYGZhy{}1))\PYGZhy{}LOG(1.21203101101442)\PYGZhy{}LOG((((PAKBXFSTREMTCD(\PYGZhy{}1)\PYGZhy{}PAKBMFSTREMTCD(\PYGZhy{}1))*PAKPANUSATLS(\PYGZhy{}1))+PAKGGEXPTRNSCN(\PYGZhy{}1)+PAKNYYWBTOTLCN(\PYGZhy{}1)*(1\PYGZhy{}PAKGGREVDRCTXN(\PYGZhy{}1)/100))/PAKNECONPRVTXN(\PYGZhy{}1)))+0.763938860758873*((LOG((((PAKBXFSTREMTCD\PYGZhy{}PAKBMFSTREMTCD)*PAKPANUSATLS)+PAKGGEXPTRNSCN+PAKNYYWBTOTLCN*(1\PYGZhy{}PAKGGREVDRCTXN/100))/PAKNECONPRVTXN))\PYGZhy{}(LOG((((PAKBXFSTREMTCD(\PYGZhy{}1)\PYGZhy{}PAKBMFSTREMTCD(\PYGZhy{}1))*PAKPANUSATLS(\PYGZhy{}1))+PAKGGEXPTRNSCN(\PYGZhy{}1)+PAKNYYWBTOTLCN(\PYGZhy{}1)*(1\PYGZhy{}PAKGGREVDRCTXN(\PYGZhy{}1)/100))/PAKNECONPRVTXN(\PYGZhy{}1))))\PYGZhy{}0.0634474791568939*DURING\PYGZus{}2009\PYGZhy{}0.3*(PAKFMLBLPOLYXN/100\PYGZhy{}((LOG(PAKNECONPRVTXN))\PYGZhy{}(LOG(PAKNECONPRVTXN(\PYGZhy{}1)))))) )) * (1\PYGZhy{}PAKNECONPRVTKN\PYGZus{}D)+ PAKNECONPRVTKN\PYGZus{}X*PAKNECONPRVTKN\PYGZus{}D  \PYGZdl{}

PAKNECONPRVTKN  : HH. Cons Real
DURING\PYGZus{}2009     : 
PAKBMFSTREMTCD  : Imp., Remittances (BOP), US\PYGZdl{} mn
PAKBXFSTREMTCD  : Exp., Remittances (BOP), US\PYGZdl{} mn
PAKFMLBLPOLYXN  : Key Policy Interest Rate
PAKGGEXPTRNSCN  : Current Transfers
PAKGGREVDRCTXN  : Direct Revenue Tax Rate
PAKNECONPRVTKN\PYGZus{}A: Add factor:HH. Cons Real
PAKNECONPRVTKN\PYGZus{}D: Fix dummy:HH. Cons Real
PAKNECONPRVTKN\PYGZus{}X: Fix value:HH. Cons Real
PAKNECONPRVTXN  : Implicit LCU defl., Pvt. Cons., 2000 = 1
PAKNYYWBTOTLCN  : Total Wage Bill
PAKPANUSATLS    : Exchange rate LCU / US\PYGZdl{} \PYGZhy{} Pakistan

Values :
\end{sphinxVerbatim}

\begin{sphinxVerbatim}[commandchars=\\\{\}]
\PYGZlt{}IPython.core.display.HTML object\PYGZgt{}
\end{sphinxVerbatim}

\begin{sphinxVerbatim}[commandchars=\\\{\}]
Input last run:
\end{sphinxVerbatim}

\begin{sphinxVerbatim}[commandchars=\\\{\}]
\PYGZlt{}pandas.io.formats.style.Styler at 0x23f38e59090\PYGZgt{}
\end{sphinxVerbatim}

\begin{sphinxVerbatim}[commandchars=\\\{\}]
Input base run:
\end{sphinxVerbatim}

\begin{sphinxVerbatim}[commandchars=\\\{\}]
\PYGZlt{}pandas.io.formats.style.Styler at 0x23f38e1f4f0\PYGZgt{}
\end{sphinxVerbatim}

\begin{sphinxVerbatim}[commandchars=\\\{\}]
Difference for input variables
\end{sphinxVerbatim}

\begin{sphinxVerbatim}[commandchars=\\\{\}]
\PYGZlt{}pandas.io.formats.style.Styler at 0x23f38e41690\PYGZgt{}
\end{sphinxVerbatim}

\begin{sphinxVerbatim}[commandchars=\\\{\}]

\end{sphinxVerbatim}

\end{sphinxuseclass}\end{sphinxVerbatimOutput}

\end{sphinxuseclass}

\section{Behavioural equations in the MFMod framework}
\label{\detokenize{content/05_WBModels/LoadingWBModel:behavioural-equations-in-the-mfmod-framework}}
\sphinxAtStartPar
Recall a behavioural equation determines the value of an endogenous variable. For many of the variables in Wold Bank models, behavioural functions are estimated using an Error Correction Framework that splits the equation into a theoretically determined long run component and a more idiosyncratic short\sphinxhyphen{}run component.

\sphinxAtStartPar
Looking at the eviews representation of the consumption function:

\sphinxAtStartPar
\sphinxcode{\sphinxupquote{DLOG(PAKNECONPRVTKN) =\sphinxhyphen{} 0.2*(LOG(PAKNECONPRVTKN( \sphinxhyphen{} 1)) \sphinxhyphen{} LOG(1.21203101101442) \sphinxhyphen{} LOG((((PAKBXFSTREMTCD( \sphinxhyphen{} 1) \sphinxhyphen{} PAKBMFSTREMTCD( \sphinxhyphen{} 1))*PAKPANUSATLS( \sphinxhyphen{} 1)) + PAKGGEXPTRNSCN( \sphinxhyphen{} 1) + PAKNYYWBTOTLCN( \sphinxhyphen{} 1)*(1 \sphinxhyphen{} PAKGGREVDRCTXN( \sphinxhyphen{} 1)/100))/PAKNECONPRVTXN( \sphinxhyphen{} 1))) + 0.763938860758873*DLOG((((PAKBXFSTREMTCD \sphinxhyphen{} PAKBMFSTREMTCD)*PAKPANUSATLS) + PAKGGEXPTRNSCN + PAKNYYWBTOTLCN*(1 \sphinxhyphen{} PAKGGREVDRCTXN/100))/PAKNECONPRVTXN) \sphinxhyphen{} 0.0634474791568939*@DURING("2009") \sphinxhyphen{} 0.3*(PAKFMLBLPOLYXN/100 \sphinxhyphen{} DLOG(PAKNECONPRVTXN))}}

\sphinxAtStartPar
Below the mnemonics are simplified to ease reading of the equation using:


\begin{savenotes}\sphinxattablestart
\centering
\begin{tabulary}{\linewidth}[t]{|T|T|T|}
\hline
\sphinxstyletheadfamily 
\sphinxAtStartPar
Model Mnemonic
&\sphinxstyletheadfamily 
\sphinxAtStartPar
Simplified
&\sphinxstyletheadfamily 
\sphinxAtStartPar
Meaning
\\
\hline
\sphinxAtStartPar
PAKNECONPRVTKN
&
\sphinxAtStartPar
\(CON^{KN}_t\)
&
\sphinxAtStartPar
Household Consumption
\\
\hline
\sphinxAtStartPar
(PAKBXFSTREMTCD \sphinxhyphen{} PAKBMFSTREMTCD)*PAKPANUSATLS
&
\sphinxAtStartPar
\(Remit^{net}_t\)
&
\sphinxAtStartPar
Net remittances inflows in LCU
\\
\hline
\sphinxAtStartPar
PAKGGEXPTRNSCN
&
\sphinxAtStartPar
\(TRANSF^{hhld}_t\)
&
\sphinxAtStartPar
Government transfers to households
\\
\hline
\sphinxAtStartPar
DURING\_2010
&
\sphinxAtStartPar
\(D^{2010}_t\)
&
\sphinxAtStartPar
A dummy
\\
\hline
\sphinxAtStartPar
PAKFMLBLPOLYXN
&
\sphinxAtStartPar
\(r^{policy}_t\)
&
\sphinxAtStartPar
Policy Rate
\\
\hline
\sphinxAtStartPar
PAKGGREVDRCTXN
&
\sphinxAtStartPar
\(DirectTxR_t\)
&
\sphinxAtStartPar
Direct Taxes: Effective rate
\\
\hline
\sphinxAtStartPar
PAKNECONPRVTKN\_A
&
\sphinxAtStartPar
\(CON^{KN_AF}_t\)
&
\sphinxAtStartPar
Add factor:Household Consumption
\\
\hline
\sphinxAtStartPar
PAKNECONPRVTXN
&
\sphinxAtStartPar
\(CON^{XN}_t\)
&
\sphinxAtStartPar
Household Consumption Deflator
\\
\hline
\sphinxAtStartPar
PAKNYYWBTOTLCN
&
\sphinxAtStartPar
\(WAGEBILL^{CN}_t\)
&
\sphinxAtStartPar
Economy\sphinxhyphen{}wide wage bill
\\
\hline
\end{tabulary}
\par
\sphinxattableend\end{savenotes}

\sphinxAtStartPar
With those substitutions the equation can be rewritten as:
\begin{align*}
\Delta log(CON^{KN}_t) = &-0.2*\bigg[LOG(CON^{KN}_{t-1})-LOG\bigg({\frac{(Remit^{net}_{t-1}+WAGEBILL^{CN}_{t-1}+TRANSF^{hhld}_{t-1})*(1-DirectTxR_{t-1}/100)}{CON^{XN}_{t-1}}}\bigg)\bigg]  \\
&+0.76*\Delta log \bigg({\frac{(Remit^{net}_{t}+WAGEBILL^{CN}_{t}+TRANSF^{hhld}_{t})*(1-DirectTxR_{t}/100)}{CON^{XN}_{t}}}\bigg)  \\
&+0.030 + 0.016*D^{2010}_t-0.3*\bigg(r^{policy}_t/100-\Delta log(CON^{XN}_{t})\bigg) -CON^{KN_AF}_t
\end{align*}
\sphinxAtStartPar
Where in this instance the short\sphinxhyphen{}run elasticity of consumption to disposable income is .76 , and the short run elasticity of consumption to the real interest rate is 0.3.


\subsection{The ECM specification}
\label{\detokenize{content/05_WBModels/LoadingWBModel:the-ecm-specification}}
\sphinxAtStartPar
 Pretty sure this repeats and earlier section.  Delete one 

\sphinxAtStartPar
The ECM approach used in World Bank models is described in {[}\hyperlink{cite.content/99_BackMatter/References:id16}{Wickens and Breusch, 1988}{]}, and addresses the above challenge by modelling both the long run relationship and the short run short run behaviour and brings them together into one equation.

\sphinxAtStartPar
The ECM specification is therefore a single equation comprised of two parts (the long run relationship, and the short\sphinxhyphen{}run relationship).

\sphinxAtStartPar
Consider as an example two variables say consumption and disposable income.  Both have an underlying trend or in the parlance are co\sphinxhyphen{}integrated to degree 1.  For simplicity we call them y an x.


\subsubsection{The short run relationship}
\label{\detokenize{content/05_WBModels/LoadingWBModel:the-short-run-relationship}}
\sphinxAtStartPar
In its simplest form we might have a short run relationship between the growth rates of our two variables such that:
\begin{equation*}
\begin{split}\Delta ln(Y_t) = \alpha + \beta \Delta ln(X_t) +\epsilon_t\end{split}
\end{equation*}
\sphinxAtStartPar
or substituting lower case letters for the logged values.
\begin{equation*}
\begin{split}\Delta y_t = \alpha + \beta \Delta x_t +\epsilon_t\end{split}
\end{equation*}

\subsubsection{The long run equation}
\label{\detokenize{content/05_WBModels/LoadingWBModel:the-long-run-equation}}
\sphinxAtStartPar
The long run relates the level of the two (or more) variables.  A simplified version of that equation can be written as:
\begin{equation*}
\begin{split}Y_t=αX_t^β+ \eta_t\end{split}
\end{equation*}
\sphinxAtStartPar
Rewriting this (in logarithms) it can be expressed as:
\begin{equation*}
\begin{split}y_t = ln⁡(α) + βx_t + \eta_t\end{split}
\end{equation*}

\subsection{The long run equation in the steady state}
\label{\detokenize{content/05_WBModels/LoadingWBModel:the-long-run-equation-in-the-steady-state}}
\sphinxAtStartPar
Note that in the steady state the expected value of the error term in the long run equation is zero (\(\eta_t=0 \)) so in those conditions the long run relationship can be simplified to:
\begin{equation*}
\begin{split}y_t=ln⁡(α)+\beta x_t\end{split}
\end{equation*}
\sphinxAtStartPar
or equivalently (substituting A for the log of \(\alpha\)).
\begin{equation*}
\begin{split}y_t-A-βx_t=0\end{split}
\end{equation*}
\sphinxAtStartPar
Moreover if this expression is multiplied by some arbitrary constant, say \(-\lambda\), it would still equal zero.
\begin{equation*}
\begin{split}-\lambda(y_t -A-βx_t)\end{split}
\end{equation*}
\sphinxAtStartPar
and in the steady state this will also be true for the lagged variables
\begin{equation*}
\begin{split}-\lambda(y_{t-1}- A - βx_{t-1})\end{split}
\end{equation*}

\section{Putting it together}
\label{\detokenize{content/05_WBModels/LoadingWBModel:putting-it-together}}
\sphinxAtStartPar
From before we have the short run equation:
\begin{equation*}
\begin{split}\Delta y_t = \alpha + \beta \Delta x_t +\epsilon_t\end{split}
\end{equation*}
\sphinxAtStartPar
Inserting the steady state expression for the long\sphinxhyphen{}run into the short run equation makes no difference (in the long run) because in the long run it is equal to zero.
\begin{equation*}
\begin{split}\Delta y_t = -\lambda(y_{t-1}-A-βx_{t-1})  + \alpha + \beta \Delta x_t +\epsilon_t\end{split}
\end{equation*}
\sphinxAtStartPar
When the model is not in the steady state the expression \(y_{t-1}-A-βx_{t-1}\) is of course the error term from the long run equation (a measure of how far the dependent variable is from equilibrium).


\subsection{Lambda, the speed of adjustment}
\label{\detokenize{content/05_WBModels/LoadingWBModel:lambda-the-speed-of-adjustment}}
\sphinxAtStartPar
The parameter \(\lambda\) can be interpreted as the speed of adjustment.  As long as \(\lambda\) is greater than zero and less or equal to one if there are no further disturbances ( \(\epsilon_t=0\)) the expression multiplied by lambda will slowly decline toward zero. How fast depends on how large or small is \(\lambda\).

\sphinxAtStartPar
To be convergent \(\lambda\) must be between 0 and 2, if its is negative or greater than one, then the long run portion of the equation will cause the disequilibrium to grow each period (\(\lambda\) >1) not diminish or if  (\(\lambda\) >1<2) output will oscillate from positive to negative (\(\lambda <0\)) but will slowly converge.

\sphinxAtStartPar
Intuitively, the long\sphinxhyphen{}run error\sphinxhyphen{}term measures how far the model was from equilibrium one period earlier (at t\sphinxhyphen{}1). The ECM term (multiplied by \(\lambda\) ensures the model will slowly converge to equilibrium – the point at which the long run equation holds exactly. If \(\lambda\) is greater than zero but less than one (or equal to one) some portion of the previous period year’s disequilibrium will be absorbed each year. How much is absorbed depends on the size of estimated speed of the adjustment coefficient \(\lambda\). 

\sphinxAtStartPar
An ECM equation can, therefore be broken into its component parts.  For the consumption function it will look something like this:
\begin{equation*}
\begin{split}\Delta c_t = -\lambda (\underbrace{
        log(C_{t-1})-log(Wages_{t-1}-Taxes_{t-1}+Transfers_{t-1}) -log(\alpha))}  _\text{Long run}
+\beta \underbrace{\Delta x_t}_\text{short run}\end{split}
\end{equation*}
\sphinxstepscope


\chapter{Scenario analysis}
\label{\detokenize{content/05_WBModels/ScenarioAnalysis:scenario-analysis}}\label{\detokenize{content/05_WBModels/ScenarioAnalysis::doc}}
\sphinxAtStartPar
An essential feature of a model is that when given a specific set of inputs (the exogenous variables to the model) it will always return the same results.

\sphinxAtStartPar
Below a new \sphinxcode{\sphinxupquote{ModelFlow}} sessionis prepared, initializing a pandas session and importing and solving a saved WBG model (NB: these are precisely the same commands they used to start the previous chapter) and would form the essential initialization commands of any python session using \sphinxcode{\sphinxupquote{ModelFlow}}.


\section{Ib:}
\label{\detokenize{content/05_WBModels/ScenarioAnalysis:ib}}
\sphinxAtStartPar
I have introduced a rename() in the {[}{]} expressions, to replace the variable name with the descriptions.

\begin{sphinxuseclass}{cell}\begin{sphinxVerbatimInput}

\begin{sphinxuseclass}{cell_input}
\begin{sphinxVerbatim}[commandchars=\\\{\}]
\PYG{o}{\PYGZpc{}}\PYG{k}{matplotlib} inline
\end{sphinxVerbatim}

\end{sphinxuseclass}\end{sphinxVerbatimInput}

\end{sphinxuseclass}
\begin{sphinxadmonition}{note}{Recall}

\sphinxAtStartPar
The \sphinxcode{\sphinxupquote{model}} object (\sphinxcode{\sphinxupquote{mpak}} in this instance) always contains two \sphinxcode{\sphinxupquote{DataFrames}}, the \sphinxcode{\sphinxupquote{.basedf}} dataframe that contains the initial values for the variables of the model, and the \sphinxcode{\sphinxupquote{.lastdf}} which contain the results of the most recently executed scenario.
\end{sphinxadmonition}

\sphinxAtStartPar
As noted, when the model is solved without changing any inputs (as was the case of the load) the model should return (reproduce) exactly the same data as before{[}\textasciicircum{}fn2{]}.  To test this for \sphinxcode{\sphinxupquote{mpak}} the results from the simulation can be compared by using the \sphinxcode{\sphinxupquote{basedf}} and \sphinxcode{\sphinxupquote{lastdf}} DataFrames.

\sphinxAtStartPar
{[}\textasciicircum{}fn2:{]} If it does not, the model has violated he principle of reproducibility and there is something wrong (usually one of the identities does not hold).

\sphinxAtStartPar
Below, the percent difference between the values of the variables for real GDP and Consumer demand in the two \sphinxcode{\sphinxupquote{dataframes}} \sphinxcode{\sphinxupquote{.basedf}} and \sphinxcode{\sphinxupquote{lastdf}} is zero following a simulation where the inputs were not changed – confirming the reproduction of results.

\begin{sphinxuseclass}{cell}\begin{sphinxVerbatimInput}

\begin{sphinxuseclass}{cell_input}
\begin{sphinxVerbatim}[commandchars=\\\{\}]
\PYG{c+c1}{\PYGZsh{} Need statement to change the default format}
\PYG{n}{mpak}\PYG{o}{.}\PYG{n}{smpl}\PYG{p}{(}\PYG{l+m+mi}{2020}\PYG{p}{,}\PYG{l+m+mi}{2030}\PYG{p}{)}
\PYG{n}{mpak}\PYG{p}{[}\PYG{l+s+s1}{\PYGZsq{}}\PYG{l+s+s1}{PAKNYGDPMKTPKN PAKNECONPRVTKN}\PYG{l+s+s1}{\PYGZsq{}}\PYG{p}{]}\PYG{o}{.}\PYG{n}{difpctlevel}\PYG{o}{.}\PYG{n}{mul100}\PYG{o}{.}\PYG{n}{df}
\end{sphinxVerbatim}

\end{sphinxuseclass}\end{sphinxVerbatimInput}
\begin{sphinxVerbatimOutput}

\begin{sphinxuseclass}{cell_output}
\begin{sphinxVerbatim}[commandchars=\\\{\}]
      PAKNYGDPMKTPKN  PAKNECONPRVTKN
2020             0.0             0.0
2021             0.0             0.0
2022             0.0             0.0
2023             0.0             0.0
2024             0.0             0.0
2025             0.0             0.0
2026             0.0             0.0
2027             0.0             0.0
2028             0.0             0.0
2029             0.0             0.0
2030             0.0             0.0
\end{sphinxVerbatim}

\end{sphinxuseclass}\end{sphinxVerbatimOutput}

\end{sphinxuseclass}

\section{Different kinds of simulations}
\label{\detokenize{content/05_WBModels/ScenarioAnalysis:different-kinds-of-simulations}}
\sphinxAtStartPar
The \sphinxcode{\sphinxupquote{modelflow}} package performs 4 different kinds of simulation:
\begin{enumerate}
\sphinxsetlistlabels{\arabic}{enumi}{enumii}{}{.}%
\item {} 
\sphinxAtStartPar
A shock to an exogenous variable in the model

\item {} 
\sphinxAtStartPar
An exogenous shock of a behavioural variable, executed by exogenizing the variable

\item {} 
\sphinxAtStartPar
An endogenous shock of a behavioural variable, executed by shocking the add factor of the variable.

\item {} 
\sphinxAtStartPar
A mixed shock of a behavioural variable, achieved by temporarily exogenixing the variable.

\end{enumerate}

\sphinxAtStartPar
Although technically modelflow would allow us to shock identities, that would violate their nature as accounting rules. \sphinxstylestrong{Effectively such a shock would break the economic sense of the model.}

\sphinxAtStartPar
As a result, this we possibility is not discussed.


\subsection{A shock to an exogenous variable}
\label{\detokenize{content/05_WBModels/ScenarioAnalysis:a-shock-to-an-exogenous-variable}}
\sphinxAtStartPar
A World Bank model will reproduce the same values if inputs (exogenous variables) are not changed.  In the simulation below, the oil price is changed – increasing by \$25 for the three years between 2025 and 2027 inclusive.

\sphinxAtStartPar
As a first step a new input dataframe is created as a copy of the original and then the oil price in that data frame is modified using the \sphinxcode{\sphinxupquote{mfcalc}} method to change the value for the three years in question.

\sphinxAtStartPar
Finally pandas math is used to display the initial value, the changed value and the difference between the two, confirming that the mfcalc statement revised the oil price data.

\begin{sphinxuseclass}{cell}\begin{sphinxVerbatimInput}

\begin{sphinxuseclass}{cell_input}
\begin{sphinxVerbatim}[commandchars=\\\{\}]
\PYG{c+c1}{\PYGZsh{}Make a copy of the baseline dataframe}
\PYG{n}{oilshockdf}\PYG{o}{=}\PYG{n}{mpak}\PYG{o}{.}\PYG{n}{basedf}
\PYG{n}{oilshockdf}\PYG{o}{=}\PYG{n}{oilshockdf}\PYG{o}{.}\PYG{n}{mfcalc}\PYG{p}{(}\PYG{l+s+s2}{\PYGZdq{}}\PYG{l+s+s2}{\PYGZlt{}2025 2027\PYGZgt{} WLDFCRUDE\PYGZus{}PETRO = WLDFCRUDE\PYGZus{}PETRO +25}\PYG{l+s+s2}{\PYGZdq{}}\PYG{p}{)}

\PYG{n}{compdf}\PYG{o}{=}\PYG{n}{mpak}\PYG{o}{.}\PYG{n}{basedf}\PYG{o}{.}\PYG{n}{loc}\PYG{p}{[}\PYG{l+m+mi}{2000}\PYG{p}{:}\PYG{l+m+mi}{2030}\PYG{p}{,}\PYG{p}{[}\PYG{l+s+s1}{\PYGZsq{}}\PYG{l+s+s1}{WLDFCRUDE\PYGZus{}PETRO}\PYG{l+s+s1}{\PYGZsq{}}\PYG{p}{]}\PYG{p}{]}
\PYG{n}{compdf}\PYG{p}{[}\PYG{l+s+s1}{\PYGZsq{}}\PYG{l+s+s1}{LASTDF}\PYG{l+s+s1}{\PYGZsq{}}\PYG{p}{]}\PYG{o}{=}\PYG{n}{oilshockdf}\PYG{o}{.}\PYG{n}{loc}\PYG{p}{[}\PYG{l+m+mi}{2000}\PYG{p}{:}\PYG{l+m+mi}{2030}\PYG{p}{,}\PYG{p}{[}\PYG{l+s+s1}{\PYGZsq{}}\PYG{l+s+s1}{WLDFCRUDE\PYGZus{}PETRO}\PYG{l+s+s1}{\PYGZsq{}}\PYG{p}{]}\PYG{p}{]}
\PYG{n}{compdf}\PYG{p}{[}\PYG{l+s+s1}{\PYGZsq{}}\PYG{l+s+s1}{Dif}\PYG{l+s+s1}{\PYGZsq{}}\PYG{p}{]}\PYG{o}{=}\PYG{n}{compdf}\PYG{p}{[}\PYG{l+s+s1}{\PYGZsq{}}\PYG{l+s+s1}{LASTDF}\PYG{l+s+s1}{\PYGZsq{}}\PYG{p}{]}\PYG{o}{\PYGZhy{}}\PYG{n}{compdf}\PYG{p}{[}\PYG{l+s+s1}{\PYGZsq{}}\PYG{l+s+s1}{WLDFCRUDE\PYGZus{}PETRO}\PYG{l+s+s1}{\PYGZsq{}}\PYG{p}{]}

\PYG{n}{compdf}\PYG{o}{.}\PYG{n}{loc}\PYG{p}{[}\PYG{l+m+mi}{2024}\PYG{p}{:}\PYG{l+m+mi}{2030}\PYG{p}{]}
\end{sphinxVerbatim}

\end{sphinxuseclass}\end{sphinxVerbatimInput}
\begin{sphinxVerbatimOutput}

\begin{sphinxuseclass}{cell_output}
\begin{sphinxVerbatim}[commandchars=\\\{\}]
      WLDFCRUDE\PYGZus{}PETRO      LASTDF   Dif
2024        80.367180   80.367180   0.0
2025        85.336809  110.336809  25.0
2026        90.613742  115.613742  25.0
2027        96.216983  121.216983  25.0
2028       102.166709  102.166709   0.0
2029       108.484346  108.484346   0.0
2030       115.192643  115.192643   0.0
\end{sphinxVerbatim}

\end{sphinxuseclass}\end{sphinxVerbatimOutput}

\end{sphinxuseclass}
\begin{sphinxadmonition}{warning}{Warning:}
\sphinxAtStartPar
IB: alternative use upd and no nedd to copy
\end{sphinxadmonition}

\begin{sphinxuseclass}{cell}\begin{sphinxVerbatimInput}

\begin{sphinxuseclass}{cell_input}
\begin{sphinxVerbatim}[commandchars=\\\{\}]
\PYG{n}{oilshockdf}\PYG{o}{=}\PYG{n}{bline}\PYG{o}{.}\PYG{n}{upd}\PYG{p}{(}\PYG{l+s+s2}{\PYGZdq{}}\PYG{l+s+s2}{\PYGZlt{}2025 2027\PYGZgt{} WLDFCRUDE\PYGZus{}PETRO + 25 }\PYG{l+s+s2}{\PYGZdq{}}\PYG{p}{)}

\PYG{n}{compdf}\PYG{o}{=}\PYG{n}{mpak}\PYG{o}{.}\PYG{n}{basedf}\PYG{o}{.}\PYG{n}{loc}\PYG{p}{[}\PYG{l+m+mi}{2000}\PYG{p}{:}\PYG{l+m+mi}{2030}\PYG{p}{,}\PYG{p}{[}\PYG{l+s+s1}{\PYGZsq{}}\PYG{l+s+s1}{WLDFCRUDE\PYGZus{}PETRO}\PYG{l+s+s1}{\PYGZsq{}}\PYG{p}{]}\PYG{p}{]}
\PYG{n}{compdf}\PYG{p}{[}\PYG{l+s+s1}{\PYGZsq{}}\PYG{l+s+s1}{LASTDF}\PYG{l+s+s1}{\PYGZsq{}}\PYG{p}{]}\PYG{o}{=}\PYG{n}{oilshockdf}\PYG{o}{.}\PYG{n}{loc}\PYG{p}{[}\PYG{l+m+mi}{2000}\PYG{p}{:}\PYG{l+m+mi}{2030}\PYG{p}{,}\PYG{p}{[}\PYG{l+s+s1}{\PYGZsq{}}\PYG{l+s+s1}{WLDFCRUDE\PYGZus{}PETRO}\PYG{l+s+s1}{\PYGZsq{}}\PYG{p}{]}\PYG{p}{]}
\PYG{n}{compdf}\PYG{p}{[}\PYG{l+s+s1}{\PYGZsq{}}\PYG{l+s+s1}{Dif}\PYG{l+s+s1}{\PYGZsq{}}\PYG{p}{]}\PYG{o}{=}\PYG{n}{compdf}\PYG{p}{[}\PYG{l+s+s1}{\PYGZsq{}}\PYG{l+s+s1}{LASTDF}\PYG{l+s+s1}{\PYGZsq{}}\PYG{p}{]}\PYG{o}{\PYGZhy{}}\PYG{n}{compdf}\PYG{p}{[}\PYG{l+s+s1}{\PYGZsq{}}\PYG{l+s+s1}{WLDFCRUDE\PYGZus{}PETRO}\PYG{l+s+s1}{\PYGZsq{}}\PYG{p}{]}

\PYG{n}{compdf}\PYG{o}{.}\PYG{n}{loc}\PYG{p}{[}\PYG{l+m+mi}{2024}\PYG{p}{:}\PYG{l+m+mi}{2030}\PYG{p}{]}
\end{sphinxVerbatim}

\end{sphinxuseclass}\end{sphinxVerbatimInput}
\begin{sphinxVerbatimOutput}

\begin{sphinxuseclass}{cell_output}
\begin{sphinxVerbatim}[commandchars=\\\{\}]
      WLDFCRUDE\PYGZus{}PETRO      LASTDF   Dif
2024        80.367180   80.367180   0.0
2025        85.336809  110.336809  25.0
2026        90.613742  115.613742  25.0
2027        96.216983  121.216983  25.0
2028       102.166709  102.166709   0.0
2029       108.484346  108.484346   0.0
2030       115.192643  115.192643   0.0
\end{sphinxVerbatim}

\end{sphinxuseclass}\end{sphinxVerbatimOutput}

\end{sphinxuseclass}

\subsubsection{Running the simulation}
\label{\detokenize{content/05_WBModels/ScenarioAnalysis:running-the-simulation}}
\sphinxAtStartPar
Having created a new dataframe comprised of all the old data plus the changed data for the oil price, a simulation can now be run.

\sphinxAtStartPar
In the command below, the simulation is run from 2020 to 2040, using the \sphinxcode{\sphinxupquote{oilshockdf}} as the input \sphinxcode{\sphinxupquote{DataFrame}}.  The results of the simulation are assigned to a new \sphinxcode{\sphinxupquote{DataFrame}}  named \sphinxcode{\sphinxupquote{ExogOilSimul}}.  The \sphinxcode{\sphinxupquote{Keep}} command ensures that the mpak model object stores (keeps) a copy of the results identified by the text name \sphinxstyleemphasis{‘\$25 increase in oil prices 2025\sphinxhyphen{}27’}.

\begin{sphinxuseclass}{cell}\begin{sphinxVerbatimInput}

\begin{sphinxuseclass}{cell_input}
\begin{sphinxVerbatim}[commandchars=\\\{\}]
\PYG{c+c1}{\PYGZsh{}Simulate the model }
\PYG{n}{ExogOilSimul} \PYG{o}{=} \PYG{n}{mpak}\PYG{p}{(}\PYG{n}{oilshockdf}\PYG{p}{,}\PYG{l+m+mi}{2020}\PYG{p}{,}\PYG{l+m+mi}{2040}\PYG{p}{,}\PYG{n}{keep}\PYG{o}{=}\PYG{l+s+s1}{\PYGZsq{}}\PYG{l+s+s1}{\PYGZdl{}25 increase in oil prices 2025\PYGZhy{}27}\PYG{l+s+s1}{\PYGZsq{}}\PYG{p}{)} 
\end{sphinxVerbatim}

\end{sphinxuseclass}\end{sphinxVerbatimInput}

\end{sphinxuseclass}

\paragraph{Results}
\label{\detokenize{content/05_WBModels/ScenarioAnalysis:results}}
\sphinxAtStartPar
\sphinxcode{\sphinxupquote{ModelFlow}} tools can be used to visualize the impacts of the shock; as a print out; as charts and within Jupyter notebook as an interactive widget.

\sphinxAtStartPar
The display below confirms that the shock was executed as desired. The \sphinxcode{\sphinxupquote{dif.df}} method returns the difference between the \sphinxcode{\sphinxupquote{.lastdf}} and \sphinxcode{\sphinxupquote{.basedf}} values of the selected variable(s) as a \sphinxcode{\sphinxupquote{DataFrame}}. The \sphinxcode{\sphinxupquote{with mpak.set\_smpl(2020,2030):}} clause temporarily restricts the sample period over which the following \sphinxstylestrong{indented} commands are executed.

\sphinxAtStartPar
Alternatively the \sphinxcode{\sphinxupquote{mpak.smpl(2020,2030)}}could be used. This would restricts the time period of over which \sphinxstylestrong{all} subsequent commands are executed.

\begin{sphinxuseclass}{cell}\begin{sphinxVerbatimInput}

\begin{sphinxuseclass}{cell_input}
\begin{sphinxVerbatim}[commandchars=\\\{\}]
\PYG{k}{with} \PYG{n}{mpak}\PYG{o}{.}\PYG{n}{set\PYGZus{}smpl}\PYG{p}{(}\PYG{l+m+mi}{2020}\PYG{p}{,}\PYG{l+m+mi}{2030}\PYG{p}{)}\PYG{p}{:}
    \PYG{n+nb}{print}\PYG{p}{(}\PYG{n}{mpak}\PYG{p}{[}\PYG{l+s+s1}{\PYGZsq{}}\PYG{l+s+s1}{WLDFCRUDE\PYGZus{}PETRO}\PYG{l+s+s1}{\PYGZsq{}}\PYG{p}{]}\PYG{o}{.}\PYG{n}{dif}\PYG{o}{.}\PYG{n}{df}\PYG{p}{)}\PYG{p}{;}
\end{sphinxVerbatim}

\end{sphinxuseclass}\end{sphinxVerbatimInput}
\begin{sphinxVerbatimOutput}

\begin{sphinxuseclass}{cell_output}
\begin{sphinxVerbatim}[commandchars=\\\{\}]
      WLDFCRUDE\PYGZus{}PETRO
2020              0.0
2021              0.0
2022              0.0
2023              0.0
2024              0.0
2025             25.0
2026             25.0
2027             25.0
2028              0.0
2029              0.0
2030              0.0
\end{sphinxVerbatim}

\end{sphinxuseclass}\end{sphinxVerbatimOutput}

\end{sphinxuseclass}
\sphinxAtStartPar
Below the impact of this change on a few variables are expressed graphically and in a table.

\sphinxAtStartPar
The first variable \sphinxcode{\sphinxupquote{PAKNYGDPMKTPKN}} is Pakistan’s real GDP, the second \sphinxcode{\sphinxupquote{PAKNECONPRVTKN}} is real consumption and the third is the Consumer price deflator \sphinxcode{\sphinxupquote{PAKNECONPRVTXN}}.

\begin{sphinxadmonition}{warning}{Warning:}
\sphinxAtStartPar
IB: the rename()  will use the descriptions.
added in percent.
\end{sphinxadmonition}

\begin{sphinxuseclass}{cell}\begin{sphinxVerbatimInput}

\begin{sphinxuseclass}{cell_input}
\begin{sphinxVerbatim}[commandchars=\\\{\}]
\PYG{n}{mpak}\PYG{p}{[}\PYG{l+s+s1}{\PYGZsq{}}\PYG{l+s+s1}{PAKNYGDPMKTPKN PAKNECONPRVTKN PAKNEIMPGNFSKN PAKNECONPRVTXN}\PYG{l+s+s1}{\PYGZsq{}}\PYG{p}{]}\PYG{o}{.}\PYG{n}{difpctlevel}\PYG{o}{.}\PYG{n}{rename}\PYG{p}{(}\PYG{p}{)}\PYG{o}{.}\PYG{n}{mul100}\PYG{o}{.}\PYG{n}{plot}\PYG{p}{(}\PYG{n}{title}\PYG{o}{=}\PYG{l+s+s2}{\PYGZdq{}}\PYG{l+s+s2}{Impact in percent of temporary \PYGZdl{}25 hike in oil prices}\PYG{l+s+s2}{\PYGZdq{}}\PYG{p}{)}
\end{sphinxVerbatim}

\end{sphinxuseclass}\end{sphinxVerbatimInput}
\begin{sphinxVerbatimOutput}

\begin{sphinxuseclass}{cell_output}
\noindent\sphinxincludegraphics{{35fae98be2df7277e2088e8086f5cf175d221b94cd55c790fe46622d617adfd4}.png}

\end{sphinxuseclass}\end{sphinxVerbatimOutput}

\end{sphinxuseclass}
\begin{sphinxuseclass}{cell}\begin{sphinxVerbatimInput}

\begin{sphinxuseclass}{cell_input}
\begin{sphinxVerbatim}[commandchars=\\\{\}]
\PYG{n+nb}{print}\PYG{p}{(}\PYG{n+nb}{round}\PYG{p}{(}\PYG{n}{mpak}\PYG{p}{[}\PYG{l+s+s1}{\PYGZsq{}}\PYG{l+s+s1}{PAKNYGDPMKTPKN PAKNECONPRVTKN PAKNEIMPGNFSKN PAKNECONPRVTXN}\PYG{l+s+s1}{\PYGZsq{}}\PYG{p}{]}\PYG{o}{.}\PYG{n}{difpctlevel}\PYG{o}{.}\PYG{n}{mul100}\PYG{o}{.}\PYG{n}{rename}\PYG{p}{(}\PYG{p}{)}\PYG{o}{.}\PYG{n}{df}\PYG{p}{,}\PYG{l+m+mi}{2}\PYG{p}{)}\PYG{p}{)}
\end{sphinxVerbatim}

\end{sphinxuseclass}\end{sphinxVerbatimInput}
\begin{sphinxVerbatimOutput}

\begin{sphinxuseclass}{cell_output}
\begin{sphinxVerbatim}[commandchars=\\\{\}]
      Real GDP  HH. Cons Real  Imports real   
2020      0.00           0.00          0.00  \PYGZbs{}
2021      0.00           0.00          0.00   
2022      0.00           0.00          0.00   
2023      0.00           0.00          0.00   
2024      0.00           0.00          0.00   
2025     \PYGZhy{}0.89          \PYGZhy{}1.32         \PYGZhy{}1.49   
2026     \PYGZhy{}0.85          \PYGZhy{}1.48         \PYGZhy{}2.65   
2027     \PYGZhy{}0.64          \PYGZhy{}1.37         \PYGZhy{}3.19   
2028      0.34          \PYGZhy{}0.08         \PYGZhy{}2.17   
2029      0.50           0.20         \PYGZhy{}1.25   
2030      0.45           0.19         \PYGZhy{}0.80   
2031     \PYGZhy{}0.04           0.02         \PYGZhy{}0.10   
2032     \PYGZhy{}0.06           0.01         \PYGZhy{}0.01   
2033     \PYGZhy{}0.08           0.00          0.03   
2034     \PYGZhy{}0.08           0.01          0.04   
2035     \PYGZhy{}0.07           0.03          0.05   
2036     \PYGZhy{}0.06           0.04          0.05   
2037     \PYGZhy{}0.04           0.05          0.05   
2038     \PYGZhy{}0.03           0.05          0.05   
2039     \PYGZhy{}0.03           0.04          0.04   
2040     \PYGZhy{}0.04           0.03          0.02   

      Implicit LCU defl., Pvt. Cons., 2000 = 1  
2020                                      0.00  
2021                                      0.00  
2022                                      0.00  
2023                                      0.00  
2024                                      0.00  
2025                                      1.64  
2026                                      1.35  
2027                                      1.08  
2028                                     \PYGZhy{}0.51  
2029                                     \PYGZhy{}0.43  
2030                                     \PYGZhy{}0.31  
2031                                      0.26  
2032                                      0.20  
2033                                      0.15  
2034                                      0.11  
2035                                      0.08  
2036                                      0.06  
2037                                      0.06  
2038                                      0.08  
2039                                      0.09  
2040                                      0.11  
\end{sphinxVerbatim}

\end{sphinxuseclass}\end{sphinxVerbatimOutput}

\end{sphinxuseclass}
\sphinxAtStartPar
The graphs show the change in the level as a percent of the previous level. They suggest that a temporary \$25 oil price hike would reduce GDP in the first year by about 0.9 percent, that the impact would diminish by the third year to \sphinxhyphen{}.64 percent, and then turn positive in the fourth year when the price effect was eliminated.

\sphinxAtStartPar
The impacts on household consumption are stronger but follow a similar pattern.

\sphinxAtStartPar
The GDP impact is smaller partly because the decline in domestic demand reduces imports.  Because imports enter into the GDP identity with a negative sign. Therefore a reduction in imports actually increase aggregate GDP – or in this case partially offsets the declines coming from reduced consumption (and investment).

\sphinxAtStartPar
Finally as could be expected, initially prices rise sharply with higher oil prices. However, as the slow down in growth is felt, inflationary pressures turn negative and the overall impact on the price level turns negative.  The graph and table above shows what is happening to the \sphinxstylestrong{price level}. To see the impact on inflation (the rate of growth of prices), a separate graph can be generated using \sphinxcode{\sphinxupquote{difpct.mul100}}, which shows the change in the rate of growth of variables where the growth rate is expressed as a per cent \(\bigg[\bigg(\frac{x^{shock}_t}{x^{shock}_{t-1}}-1\bigg)\) \( - \bigg(\frac{x^{baseline}_t}{x^{baseline}_{t-1}}-1\bigg)\Bigg]*100\).

\begin{sphinxuseclass}{cell}\begin{sphinxVerbatimInput}

\begin{sphinxuseclass}{cell_input}
\begin{sphinxVerbatim}[commandchars=\\\{\}]
\PYG{n}{mpak}\PYG{p}{[}\PYG{l+s+s1}{\PYGZsq{}}\PYG{l+s+s1}{PAKNECONPRVTXN}\PYG{l+s+s1}{\PYGZsq{}}\PYG{p}{]}\PYG{o}{.}\PYG{n}{difpct}\PYG{o}{.}\PYG{n}{mul100}\PYG{o}{.}\PYG{n}{rename}\PYG{p}{(}\PYG{p}{)}\PYG{o}{.}\PYG{n}{plot}\PYG{p}{(}\PYG{n}{title}\PYG{o}{=}\PYG{l+s+s2}{\PYGZdq{}}\PYG{l+s+s2}{Change in inflation from a temporary \PYGZdl{}25 hike in oil prices}\PYG{l+s+s2}{\PYGZdq{}}\PYG{p}{)}
\end{sphinxVerbatim}

\end{sphinxuseclass}\end{sphinxVerbatimInput}
\begin{sphinxVerbatimOutput}

\begin{sphinxuseclass}{cell_output}
\noindent\sphinxincludegraphics{{821655a2fceb04636af1626e595f35576c55d5eaad45722566b7bb116cf9d791}.png}

\end{sphinxuseclass}\end{sphinxVerbatimOutput}

\end{sphinxuseclass}
\sphinxAtStartPar
 Ib how come this graph shows up so small.  How can we affect its size?

\sphinxAtStartPar
This view, gives a more nuanced result.  The inflation rate increases initially by about 1.2 percentage points, but falls compared with the baseline below in the 2026\sphinxhyphen{}2027 period as the as the influence of the slowdown in GDP more than offsets the continued inflationary impetus from the lagged increase in oil prices. In 2028, when oil prices drop back to their previous level, there is an additional dis\sphinxhyphen{}inflationary force and sharp drop in inflation as compared with the baseline. Overtime, the boost to demand from lower prices translates into an acceleration in growth and a return of inflation back to its trend rate.

\sphinxAtStartPar
Change size of chart: By xsize=  and ysize =

\begin{sphinxuseclass}{cell}\begin{sphinxVerbatimInput}

\begin{sphinxuseclass}{cell_input}
\begin{sphinxVerbatim}[commandchars=\\\{\}]
\PYG{k+kn}{import} \PYG{n+nn}{modelvis} \PYG{k}{as} \PYG{n+nn}{mv}
\PYG{n}{help}\PYG{p}{(}\PYG{n}{mv}\PYG{o}{.}\PYG{n}{plotshow}\PYG{p}{)}
\end{sphinxVerbatim}

\end{sphinxuseclass}\end{sphinxVerbatimInput}
\begin{sphinxVerbatimOutput}

\begin{sphinxuseclass}{cell_output}
\begin{sphinxVerbatim}[commandchars=\\\{\}]
Help on function plotshow in module modelvis:

plotshow(df, name=\PYGZsq{}\PYGZsq{}, ppos=\PYGZhy{}1, kind=\PYGZsq{}line\PYGZsq{}, colrow=2, sharey=False, top=None, splitchar=\PYGZsq{}\PYGZus{}\PYGZus{}\PYGZsq{}, savefig=\PYGZsq{}\PYGZsq{}, *args, **kwargs)
    Args:
        df (TYPE): Dataframe .
        name (TYPE, optional): title. Defaults to \PYGZsq{}\PYGZsq{}.
        ppos (TYPE, optional): \PYGZsh{} of position to use if split. Defaults to \PYGZhy{}1.
        kind (TYPE, optional): matplotlib kind . Defaults to \PYGZsq{}line\PYGZsq{}.
        colrow (TYPE, optional): columns per row . Defaults to 6.
        sharey (TYPE, optional): Share y axis between plots. Defaults to True.
        top (TYPE, optional): relative position of the title. Defaults to 0.90.
        splitchar (TYPE, optional): if the name should be split . Defaults to \PYGZsq{}\PYGZus{}\PYGZus{}\PYGZsq{}.
        savefig (TYPE, optional): save figure. Defaults to \PYGZsq{}\PYGZsq{}.
        xsize  (TYPE, optional): x size default to 10 
        ysize  (TYPE, optional): y size per row, defaults to 2
    
    Returns:
        a matplotlib fig.
\end{sphinxVerbatim}

\end{sphinxuseclass}\end{sphinxVerbatimOutput}

\end{sphinxuseclass}
\begin{sphinxuseclass}{cell}\begin{sphinxVerbatimInput}

\begin{sphinxuseclass}{cell_input}
\begin{sphinxVerbatim}[commandchars=\\\{\}]
\PYG{n}{mpak}\PYG{p}{[}\PYG{l+s+s1}{\PYGZsq{}}\PYG{l+s+s1}{PAKNECONPRVTXN}\PYG{l+s+s1}{\PYGZsq{}}\PYG{p}{]}\PYG{o}{.}\PYG{n}{difpct}\PYG{o}{.}\PYG{n}{mul100}\PYG{o}{.}\PYG{n}{rename}\PYG{p}{(}\PYG{p}{)}\PYG{o}{.}\PYG{n}{plot}\PYG{p}{(}\PYG{n}{title}\PYG{o}{=}\PYG{l+s+s2}{\PYGZdq{}}\PYG{l+s+s2}{Change in inflation from a temporary \PYGZdl{}25 hike in oil prices}\PYG{l+s+s2}{\PYGZdq{}}
                                                   \PYG{p}{,}\PYG{n}{xsize}\PYG{o}{=}\PYG{l+m+mi}{10}\PYG{p}{,}\PYG{n}{ysize}\PYG{o}{=}\PYG{l+m+mi}{4}\PYG{p}{)}
\end{sphinxVerbatim}

\end{sphinxuseclass}\end{sphinxVerbatimInput}
\begin{sphinxVerbatimOutput}

\begin{sphinxuseclass}{cell_output}
\noindent\sphinxincludegraphics{{1e673e3378d890569a61cfea5f901f0b6376a20f25a49f02be32fc3262c15664}.png}

\end{sphinxuseclass}\end{sphinxVerbatimOutput}

\end{sphinxuseclass}
\sphinxAtStartPar
IB inserted below


\subsection{An exogenous shock to a Behavioural variable}
\label{\detokenize{content/05_WBModels/ScenarioAnalysis:an-exogenous-shock-to-a-behavioural-variable}}
\sphinxAtStartPar
Behavioural equations can be de\sphinxhyphen{}activated by exogenizing them, either for the entire simulation period, or for a selected sub period.  In this example, consumption is exogenized for the entire simulation period.

\sphinxAtStartPar
To motivate the simulation, it is assumed that a change in weather patterns has increased the number of sunny days by 10 percent. This increases households happiness and causes them to permanently increase their spending by 2.5\% beginning in 2025.

\sphinxAtStartPar
Such a shock can be specified either manually or by using the\sphinxcode{\sphinxupquote{.fix()}} method. Below the simpler \sphinxcode{\sphinxupquote{.fix()}} method is used, but the equivalent manual steps performed by \sphinxcode{\sphinxupquote{.fix()}} are also explained.

\sphinxAtStartPar
To exogenize \sphinxcode{\sphinxupquote{PAKNECONPRVTKN}} for the entire simulation period, initially a new \sphinxcode{\sphinxupquote{DataFrame}} \sphinxcode{\sphinxupquote{Cfixed}} is created as a slightly modified version of  \sphinxcode{\sphinxupquote{mpak.basedf}} using the \sphinxcode{\sphinxupquote{.fix()}} command.

\sphinxAtStartPar
\sphinxcode{\sphinxupquote{Cfixed=mpak.fix(mpak.basedf,PAKNECONPRVTKN)}}

\sphinxAtStartPar
This does two things, that could have been done manually.  First it sets the dummy variable \sphinxcode{\sphinxupquote{PAKNECONPRVTKN\_D=1}} for the entire simulation period. Recall the consumption equation like all behavioural equations of World Banjk models implemented in \sphinxcode{\sphinxupquote{ModelFlow}}is expressed in tow parts.
\begin{equation*}
\begin{split} cons= (1-cons_D)*\bigg[C'(X)\bigg] + cons_d*cons_x\end{split}
\end{equation*}
\sphinxAtStartPar
When \(cons_D=1\) the first part (as it does in this scenario) the equation evaluate to zero and consumption is equal to (1)* \(cons_x\).  If instead (which would be the normal case \(cons_d\) were set to zero. the the equation would simplify to \( cons= C'(X) \)

\sphinxAtStartPar
Then \sphinxcode{\sphinxupquote{.fix()}} method then sets the variable \sphinxcode{\sphinxupquote{PAKNECONPRVTKN\_X}} in the \sphinxcode{\sphinxupquote{Cfixed}} dataframe equal to the value of \sphinxcode{\sphinxupquote{PAKNECONPRVTKN}} in the \sphinxcode{\sphinxupquote{basedf}} \sphinxcode{\sphinxupquote{.DataFrame}}. All the other variables are  just copies of their values in \sphinxcode{\sphinxupquote{.basedf}}.

\sphinxAtStartPar
With \sphinxcode{\sphinxupquote{PAKNECONPRVTKN\_D=1}} throughout the normal behavioral equation is effectively de\sphinxhyphen{}activated or exogenized … \(PAKNECONPRVTKN=PAKNECONPRVTKN\_X\).

\begin{sphinxuseclass}{cell}\begin{sphinxVerbatimInput}

\begin{sphinxuseclass}{cell_input}
\begin{sphinxVerbatim}[commandchars=\\\{\}]
\PYG{n}{mpak}\PYG{o}{.}\PYG{n}{smpl}\PYG{p}{(}\PYG{p}{)} \PYG{c+c1}{\PYGZsh{} reset the active sample period to the full model.}
\PYG{n}{Cfixed}\PYG{o}{=}\PYG{n}{mpak}\PYG{o}{.}\PYG{n}{fix}\PYG{p}{(}\PYG{n}{bline}\PYG{p}{,}\PYG{l+s+s1}{\PYGZsq{}}\PYG{l+s+s1}{PAKNECONPRVTKN}\PYG{l+s+s1}{\PYGZsq{}}\PYG{p}{)}
\end{sphinxVerbatim}

\end{sphinxuseclass}\end{sphinxVerbatimInput}
\begin{sphinxVerbatimOutput}

\begin{sphinxuseclass}{cell_output}
\begin{sphinxVerbatim}[commandchars=\\\{\}]
The folowing variables are fixed
PAKNECONPRVTKN
\end{sphinxVerbatim}

\end{sphinxuseclass}\end{sphinxVerbatimOutput}

\end{sphinxuseclass}
\sphinxAtStartPar
For the moment, the equation is exogenized but the values have been set to the same values as the \sphinxcode{\sphinxupquote{.basedf}} dataframe, so solving the model will not change anything.

\sphinxAtStartPar
The \sphinxcode{\sphinxupquote{.upd()}} method can be used to implement the assumption that Real consumption (\sphinxcode{\sphinxupquote{ PAKNECONPRVTYKN}}) would be 2.5\% stronger.

\begin{sphinxuseclass}{cell}\begin{sphinxVerbatimInput}

\begin{sphinxuseclass}{cell_input}
\begin{sphinxVerbatim}[commandchars=\\\{\}]
\PYG{n}{Cfixed}\PYG{o}{=}\PYG{n}{Cfixed}\PYG{o}{.}\PYG{n}{upd}\PYG{p}{(}\PYG{l+s+s2}{\PYGZdq{}}\PYG{l+s+s2}{\PYGZlt{}2025 2040\PYGZgt{} PAKNECONPRVTKN\PYGZus{}X  * 1.025}\PYG{l+s+s2}{\PYGZdq{}}\PYG{p}{)}
\end{sphinxVerbatim}

\end{sphinxuseclass}\end{sphinxVerbatimInput}

\end{sphinxuseclass}
\sphinxAtStartPar
To perform the simulation, the revised \sphinxcode{\sphinxupquote{CFixed}} DataFrame is input to the \sphinxcode{\sphinxupquote{mpak}} model solve routine.

\sphinxAtStartPar
\sphinxcode{\sphinxupquote{CFixedRes = mpak(Cfixed,2020,2040,keep='2.5\% increase in C 2025\sphinxhyphen{}40 (fix)')}}

\sphinxAtStartPar
And then the results can be examined graphically as before.

\begin{sphinxuseclass}{cell}\begin{sphinxVerbatimInput}

\begin{sphinxuseclass}{cell_input}
\begin{sphinxVerbatim}[commandchars=\\\{\}]
\PYG{n}{CFixedRes} \PYG{o}{=} \PYG{n}{mpak}\PYG{p}{(}\PYG{n}{Cfixed}\PYG{p}{,}\PYG{l+m+mi}{2020}\PYG{p}{,}\PYG{l+m+mi}{2040}\PYG{p}{,}\PYG{n}{keep}\PYG{o}{=}\PYG{l+s+s1}{\PYGZsq{}}\PYG{l+s+s1}{2.5}\PYG{l+s+si}{\PYGZpc{} i}\PYG{l+s+s1}{ncrease in C 2025\PYGZhy{}40}\PYG{l+s+s1}{\PYGZsq{}}\PYG{p}{)} \PYG{c+c1}{\PYGZsh{} simulates the model }


\PYG{n}{mpak}\PYG{p}{[}\PYG{l+s+s1}{\PYGZsq{}}\PYG{l+s+s1}{PAKNYGDPMKTPKN PAKNECONPRVTKN PAKNEIMPGNFSKN PAKNECONPRVTXN}\PYG{l+s+s1}{\PYGZsq{}}\PYG{p}{]}\PYG{o}{.}\PYG{n}{difpctlevel}\PYG{o}{.}\PYG{n}{mul100}\PYG{o}{.}\PYG{n}{rename}\PYG{p}{(}\PYG{p}{)}\PYG{o}{.}\PYG{n}{plot}\PYG{p}{(}
    \PYG{n}{title}\PYG{o}{=}\PYG{l+s+s2}{\PYGZdq{}}\PYG{l+s+s2}{Impact in percent of a permanent 2.5}\PYG{l+s+si}{\PYGZpc{} i}\PYG{l+s+s2}{ncrease in Consumption}\PYG{l+s+s2}{\PYGZdq{}}\PYG{p}{)}
\end{sphinxVerbatim}

\end{sphinxuseclass}\end{sphinxVerbatimInput}
\begin{sphinxVerbatimOutput}

\begin{sphinxuseclass}{cell_output}
\noindent\sphinxincludegraphics{{dcca51aadc4a3afd31fb024063d6d7340645dd648d5f4e5c482a4b00e8dd3827}.png}

\end{sphinxuseclass}\end{sphinxVerbatimOutput}

\end{sphinxuseclass}
\sphinxAtStartPar
Or the results can be displayed as numbers with a convinient format

\begin{sphinxuseclass}{cell}\begin{sphinxVerbatimInput}

\begin{sphinxuseclass}{cell_input}
\begin{sphinxVerbatim}[commandchars=\\\{\}]
\PYG{k+kn}{import} \PYG{n+nn}{pandas} \PYG{k}{as} \PYG{n+nn}{pd}
\PYG{k}{with} \PYG{n}{pd}\PYG{o}{.}\PYG{n}{option\PYGZus{}context}\PYG{p}{(}\PYG{l+s+s1}{\PYGZsq{}}\PYG{l+s+s1}{display.float\PYGZus{}format}\PYG{l+s+s1}{\PYGZsq{}}\PYG{p}{,} \PYG{l+s+s1}{\PYGZsq{}}\PYG{l+s+si}{\PYGZob{}:,.2f\PYGZcb{}}\PYG{l+s+s1}{\PYGZpc{}}\PYG{l+s+s1}{\PYGZsq{}}\PYG{o}{.}\PYG{n}{format}\PYG{p}{)}\PYG{p}{:}
    \PYG{k}{with} \PYG{n}{mpak}\PYG{o}{.}\PYG{n}{set\PYGZus{}smpl}\PYG{p}{(}\PYG{l+m+mi}{2020}\PYG{p}{,}\PYG{l+m+mi}{2040}\PYG{p}{)}\PYG{p}{:}
        \PYG{n+nb}{print}\PYG{p}{(}\PYG{n}{mpak}\PYG{p}{[}\PYG{l+s+s1}{\PYGZsq{}}\PYG{l+s+s1}{PAKNYGDPMKTPKN PAKNECONPRVTKN PAKNEIMPGNFSKN PAKNECONPRVTXN}\PYG{l+s+s1}{\PYGZsq{}}\PYG{p}{]}\PYG{o}{.}\PYG{n}{difpctlevel}\PYG{o}{.}\PYG{n}{mul100}\PYG{o}{.}\PYG{n}{df}\PYG{p}{)}
\end{sphinxVerbatim}

\end{sphinxuseclass}\end{sphinxVerbatimInput}
\begin{sphinxVerbatimOutput}

\begin{sphinxuseclass}{cell_output}
\begin{sphinxVerbatim}[commandchars=\\\{\}]
      PAKNYGDPMKTPKN  PAKNECONPRVTKN  PAKNEIMPGNFSKN  PAKNECONPRVTXN
2020           0.00\PYGZpc{}           0.00\PYGZpc{}           0.00\PYGZpc{}           0.00\PYGZpc{}
2021           0.00\PYGZpc{}           0.00\PYGZpc{}           0.00\PYGZpc{}           0.00\PYGZpc{}
2022           0.00\PYGZpc{}           0.00\PYGZpc{}           0.00\PYGZpc{}           0.00\PYGZpc{}
2023           0.00\PYGZpc{}           0.00\PYGZpc{}           0.00\PYGZpc{}           0.00\PYGZpc{}
2024           0.00\PYGZpc{}           0.00\PYGZpc{}           0.00\PYGZpc{}           0.00\PYGZpc{}
2025           2.01\PYGZpc{}           2.50\PYGZpc{}           2.27\PYGZpc{}           0.44\PYGZpc{}
2026           2.07\PYGZpc{}           2.50\PYGZpc{}           2.43\PYGZpc{}           1.06\PYGZpc{}
2027           2.05\PYGZpc{}           2.50\PYGZpc{}           2.59\PYGZpc{}           1.69\PYGZpc{}
2028           1.99\PYGZpc{}           2.50\PYGZpc{}           2.78\PYGZpc{}           2.31\PYGZpc{}
2029           1.92\PYGZpc{}           2.50\PYGZpc{}           2.99\PYGZpc{}           2.90\PYGZpc{}
2030           1.83\PYGZpc{}           2.50\PYGZpc{}           3.22\PYGZpc{}           3.47\PYGZpc{}
2031           1.43\PYGZpc{}           2.50\PYGZpc{}           4.03\PYGZpc{}           4.53\PYGZpc{}
2032           1.37\PYGZpc{}           2.50\PYGZpc{}           4.18\PYGZpc{}           4.92\PYGZpc{}
2033           1.30\PYGZpc{}           2.50\PYGZpc{}           4.34\PYGZpc{}           5.29\PYGZpc{}
2034           1.23\PYGZpc{}           2.50\PYGZpc{}           4.50\PYGZpc{}           5.64\PYGZpc{}
2035           1.16\PYGZpc{}           2.50\PYGZpc{}           4.66\PYGZpc{}           5.97\PYGZpc{}
2036           1.09\PYGZpc{}           2.50\PYGZpc{}           4.81\PYGZpc{}           6.28\PYGZpc{}
2037           1.03\PYGZpc{}           2.50\PYGZpc{}           4.96\PYGZpc{}           6.56\PYGZpc{}
2038           0.96\PYGZpc{}           2.50\PYGZpc{}           5.10\PYGZpc{}           6.82\PYGZpc{}
2039           0.90\PYGZpc{}           2.50\PYGZpc{}           5.24\PYGZpc{}           7.06\PYGZpc{}
2040           0.84\PYGZpc{}           2.50\PYGZpc{}           5.36\PYGZpc{}           7.28\PYGZpc{}
\end{sphinxVerbatim}

\end{sphinxuseclass}\end{sphinxVerbatimOutput}

\end{sphinxuseclass}
\sphinxAtStartPar
The permanent rise in consumption by 2.5 percent causes a temporary increase in GDP of close to 2\% (1.86). Higher imports tend to diminish the effect on GDP. Over time higher prices due to the inflationary pressures caused by the additional demand cause the GDP impact to diminish to close to less than 1 percent by 2040.


\subsection{Temporarily exogenize a behavioural variable}
\label{\detokenize{content/05_WBModels/ScenarioAnalysis:temporarily-exogenize-a-behavioural-variable}}
\sphinxAtStartPar
The third method of formulating a scenario involves temporarily exogenizing a variable. The methodology is the same except the period for which the variable is exogenized is different.

\sphinxAtStartPar
Here the set up is basically the same as before.

\begin{sphinxuseclass}{cell}\begin{sphinxVerbatimInput}

\begin{sphinxuseclass}{cell_input}
\begin{sphinxVerbatim}[commandchars=\\\{\}]
\PYG{c+c1}{\PYGZsh{}reset the active sample period to the full period}
\PYG{n}{mpak}\PYG{o}{.}\PYG{n}{smpl}\PYG{p}{(}\PYG{p}{)}                                  
\PYG{c+c1}{\PYGZsh{} create a copy of the bline DataFrame, but setting the PAKNECONPRVTKN\PYGZus{}D variable to 1 for the period 2025 through 2027}
\PYG{n}{CTempExogAll}\PYG{o}{=}\PYG{n}{mpak}\PYG{o}{.}\PYG{n}{fix}\PYG{p}{(}\PYG{n}{bline}\PYG{p}{,}\PYG{l+s+s1}{\PYGZsq{}}\PYG{l+s+s1}{PAKNECONPRVTKN}\PYG{l+s+s1}{\PYGZsq{}}\PYG{p}{)} 
\PYG{c+c1}{\PYGZsh{} multiply the exogenized value of consumption by 2.5\PYGZpc{} for 2025 through 2027}
\PYG{n}{CTempExogAll}\PYG{o}{=}\PYG{n}{CTempExogAll}\PYG{o}{.}\PYG{n}{upd}\PYG{p}{(}\PYG{l+s+s2}{\PYGZdq{}}\PYG{l+s+s2}{\PYGZlt{}2025 2027\PYGZgt{} PAKNECONPRVTKN\PYGZus{}X * 1.025}\PYG{l+s+s2}{\PYGZdq{}}\PYG{p}{)}  

\PYG{c+c1}{\PYGZsh{}Solve the model}
\PYG{n}{CTempXAllRes} \PYG{o}{=} \PYG{n}{mpak}\PYG{p}{(}\PYG{n}{CTempExogAll}\PYG{p}{,}\PYG{l+m+mi}{2020}\PYG{p}{,}\PYG{l+m+mi}{2040}\PYG{p}{,}\PYG{n}{keep}\PYG{o}{=}\PYG{l+s+s1}{\PYGZsq{}}\PYG{l+s+s1}{2.5}\PYG{l+s+si}{\PYGZpc{} i}\PYG{l+s+s1}{ncrease in C 2025\PYGZhy{}27 \PYGZhy{}\PYGZhy{} exog whole period}\PYG{l+s+s1}{\PYGZsq{}}\PYG{p}{)} \PYG{c+c1}{\PYGZsh{} simulates the model }
\PYG{n}{mpak}\PYG{p}{[}\PYG{l+s+s1}{\PYGZsq{}}\PYG{l+s+s1}{PAKNYGDPMKTPKN PAKNECONPRVTKN PAKNEIMPGNFSKN PAKNECONPRVTXN}\PYG{l+s+s1}{\PYGZsq{}}\PYG{p}{]}\PYG{o}{.}\PYG{n}{difpctlevel}\PYG{o}{.}\PYG{n}{mul100}\PYG{o}{.}\PYG{n}{rename}\PYG{p}{(}\PYG{p}{)}\PYG{o}{.}\PYG{n}{plot}\PYG{p}{(}
    \PYG{n}{title}\PYG{o}{=}\PYG{l+s+s2}{\PYGZdq{}}\PYG{l+s+s2}{Temporary hike in Consumption 2025\PYGZhy{}2027}\PYG{l+s+s2}{\PYGZdq{}}\PYG{p}{)}
\end{sphinxVerbatim}

\end{sphinxuseclass}\end{sphinxVerbatimInput}
\begin{sphinxVerbatimOutput}

\begin{sphinxuseclass}{cell_output}
\begin{sphinxVerbatim}[commandchars=\\\{\}]
The folowing variables are fixed
PAKNECONPRVTKN
\end{sphinxVerbatim}

\noindent\sphinxincludegraphics{{a324dc15dd086017fbe72019a96ec019b031daeb8102e15356a1f156faac530d}.png}

\end{sphinxuseclass}\end{sphinxVerbatimOutput}

\end{sphinxuseclass}
\sphinxAtStartPar
The results are quite different.  GDP is boosted initially as before but when consumption drops back to its pre\sphinxhyphen{}shock level, GDP and imports decline sharply.

\sphinxAtStartPar
Prices (and inflation) are higher initially but when the economy starts to slow after 2025 prices actually fall (deflation). While prices are falling, the level of prices remains higher at the end of the simulation.


\subsubsection{Temporary shock exogenized for the whole period}
\label{\detokenize{content/05_WBModels/ScenarioAnalysis:temporary-shock-exogenized-for-the-whole-period}}
\sphinxAtStartPar
This scenario is the same as the previous, but this time the \sphinxcode{\sphinxupquote{\sphinxhyphen{}\sphinxhyphen{}KG}} (keep\_growth) option is used to maintain the pre\sphinxhyphen{}shock growth rates of consumption in the post\sphinxhyphen{}shock period.  Effectively this is the same as a permanent increase in the level of consumption by 2.5\% because the final shocked value of consumption (which was 2.5\% higher then its pre\sphinxhyphen{}shock level) is grown at the same pre\sphinxhyphen{}shock rate – ensuring that all post\sphinxhyphen{}shock variables are also up by 2.5\%.

\begin{sphinxuseclass}{cell}\begin{sphinxVerbatimInput}

\begin{sphinxuseclass}{cell_input}
\begin{sphinxVerbatim}[commandchars=\\\{\}]
\PYG{n}{mpak}\PYG{o}{.}\PYG{n}{smpl}\PYG{p}{(}\PYG{p}{)} \PYG{c+c1}{\PYGZsh{} reset the active sample period to the full model.}
\PYG{n}{CTempExogAllKG}\PYG{o}{=}\PYG{n}{mpak}\PYG{o}{.}\PYG{n}{fix}\PYG{p}{(}\PYG{n}{bline}\PYG{p}{,}\PYG{l+s+s1}{\PYGZsq{}}\PYG{l+s+s1}{PAKNECONPRVTKN}\PYG{l+s+s1}{\PYGZsq{}}\PYG{p}{)}
\PYG{n}{CTempExogAllKG} \PYG{o}{=} \PYG{n}{CTempExogAllKG}\PYG{o}{.}\PYG{n}{upd}\PYG{p}{(}\PYG{l+s+s1}{\PYGZsq{}\PYGZsq{}\PYGZsq{}}
\PYG{l+s+s1}{\PYGZlt{}2025 2027\PYGZgt{} PAKNECONPRVTKN\PYGZus{}X * 1.025 \PYGZhy{}\PYGZhy{}kg}
\PYG{l+s+s1}{\PYGZsq{}\PYGZsq{}\PYGZsq{}}\PYG{p}{,}\PYG{n}{lprint}\PYG{o}{=}\PYG{l+m+mi}{0}\PYG{p}{)}

\PYG{c+c1}{\PYGZsh{}Now we solve the model}
\PYG{n}{CTempXAllResKG} \PYG{o}{=} \PYG{n}{mpak}\PYG{p}{(}\PYG{n}{CTempExogAllKG}\PYG{p}{,}\PYG{l+m+mi}{2020}\PYG{p}{,}\PYG{l+m+mi}{2040}\PYG{p}{,}\PYG{n}{keep}\PYG{o}{=}\PYG{l+s+s1}{\PYGZsq{}}\PYG{l+s+s1}{2.5}\PYG{l+s+si}{\PYGZpc{} i}\PYG{l+s+s1}{ncrease in C 2025\PYGZhy{}27 \PYGZhy{}\PYGZhy{} exog whole period \PYGZhy{}\PYGZhy{}KG=True}\PYG{l+s+s1}{\PYGZsq{}}\PYG{p}{)} \PYG{c+c1}{\PYGZsh{} simulates the model }
\PYG{n}{mpak}\PYG{p}{[}\PYG{l+s+s1}{\PYGZsq{}}\PYG{l+s+s1}{PAKNYGDPMKTPKN PAKNECONPRVTKN PAKNEIMPGNFSKN PAKNECONPRVTXN}\PYG{l+s+s1}{\PYGZsq{}}\PYG{p}{]}\PYG{o}{.}\PYG{n}{difpctlevel}\PYG{o}{.}\PYG{n}{mul100}\PYG{o}{.}\PYG{n}{rename}\PYG{p}{(}\PYG{p}{)}\PYG{o}{.}\PYG{n}{plot}\PYG{p}{(}
    \PYG{n}{title}\PYG{o}{=}\PYG{l+s+s2}{\PYGZdq{}}\PYG{l+s+s2}{2.5}\PYG{l+s+s2}{\PYGZpc{}}\PYG{l+s+s2}{ boost to cons 2025\PYGZhy{}27 \PYGZhy{}\PYGZhy{}kg=True}\PYG{l+s+s2}{\PYGZdq{}}\PYG{p}{)}
\end{sphinxVerbatim}

\end{sphinxuseclass}\end{sphinxVerbatimInput}
\begin{sphinxVerbatimOutput}

\begin{sphinxuseclass}{cell_output}
\begin{sphinxVerbatim}[commandchars=\\\{\}]
The folowing variables are fixed
PAKNECONPRVTKN
\end{sphinxVerbatim}

\noindent\sphinxincludegraphics{{4a303cc10ac9f0a0fdf3f929fb1cbcbe9df7f0fa999edf823c8a9e1e69f38142}.png}

\end{sphinxuseclass}\end{sphinxVerbatimOutput}

\end{sphinxuseclass}

\subsection{Exogenize the variable only for the period during which it is shocked}
\label{\detokenize{content/05_WBModels/ScenarioAnalysis:exogenize-the-variable-only-for-the-period-during-which-it-is-shocked}}
\sphinxAtStartPar
This scenario introduces a subtle but import difference.  Here we the variable is again exogenized using the fix syntax. But this time it is exogonized only for the period where the variable is shocked.

\sphinxAtStartPar
This means that the consumption function will be de\sphinxhyphen{}activated for only three years (instead of the whole period as in the previous examples).  As a result, the values that consumption takes in 2028, 2029, … 2040 depend on the model, not the level it was set to when exogenized (which was the case in the 3 previous versions).

\sphinxAtStartPar
Looking at the maths of the model the consumption equation is effectively split into two.

\sphinxAtStartPar
for the period before 2025 \(cons_d=0\) and the consumption equation simplifies to:

\sphinxAtStartPar
\(cons=C(X)\)

\sphinxAtStartPar
for the period 2025\sphinxhyphen{}2028 it is exogenized (\(cons_d=1\)) so it simplifies to:

\sphinxAtStartPar
\(cons=cons_x\)

\sphinxAtStartPar
but in the final period 2028\sphinxhyphen{}2040 (\(cons_d=0\)) and the equation reverts to:

\sphinxAtStartPar
\(cons=C(X)\)

\begin{sphinxuseclass}{cell}\begin{sphinxVerbatimInput}

\begin{sphinxuseclass}{cell_input}
\begin{sphinxVerbatim}[commandchars=\\\{\}]
\PYG{n}{mpak}\PYG{o}{.}\PYG{n}{smpl}\PYG{p}{(}\PYG{p}{)} \PYG{c+c1}{\PYGZsh{} reset the active sample period to the full model.}
\PYG{n}{CExogTemp}\PYG{o}{=}\PYG{n}{mpak}\PYG{o}{.}\PYG{n}{fix}\PYG{p}{(}\PYG{n}{bline}\PYG{p}{,}\PYG{l+s+s1}{\PYGZsq{}}\PYG{l+s+s1}{PAKNECONPRVTKN}\PYG{l+s+s1}{\PYGZsq{}}\PYG{p}{,}\PYG{l+m+mi}{2025}\PYG{p}{,}\PYG{l+m+mi}{2027}\PYG{p}{)}                             \PYG{c+c1}{\PYGZsh{}Consumption is exogenized only for three years 2025 2026 and 2027 PAKNECONPRVTKN\PYGZus{}D=1 for 2025,2026, 2027 0 elsewhere.}
\PYG{n}{CExogTemp} \PYG{o}{=} \PYG{n}{CExogTemp}\PYG{o}{.}\PYG{n}{upd}\PYG{p}{(}\PYG{l+s+s1}{\PYGZsq{}}\PYG{l+s+s1}{\PYGZlt{}2025 2027\PYGZgt{} PAKNECONPRVTKN\PYGZus{}X * 1.025}\PYG{l+s+s1}{\PYGZsq{}}\PYG{p}{,}\PYG{n}{lprint}\PYG{o}{=}\PYG{l+m+mi}{0}\PYG{p}{)}       \PYG{c+c1}{\PYGZsh{}In subsequent years it\PYGZsq{}s level will be determined by the equation }

\PYG{c+c1}{\PYGZsh{}Solve the model}
\PYG{n}{CExogTempRes} \PYG{o}{=} \PYG{n}{mpak}\PYG{p}{(}\PYG{n}{CExogTemp}\PYG{p}{,}\PYG{l+m+mi}{2020}\PYG{p}{,}\PYG{l+m+mi}{2040}\PYG{p}{,}\PYG{n}{keep}\PYG{o}{=}\PYG{l+s+s1}{\PYGZsq{}}\PYG{l+s+s1}{2.5}\PYG{l+s+si}{\PYGZpc{} i}\PYG{l+s+s1}{ncrease in C 2025\PYGZhy{}27 \PYGZhy{}\PYGZhy{} temporarily exogenized}\PYG{l+s+s1}{\PYGZsq{}}\PYG{p}{)} \PYG{c+c1}{\PYGZsh{} simulates the model }
\PYG{n}{mpak}\PYG{p}{[}\PYG{l+s+s1}{\PYGZsq{}}\PYG{l+s+s1}{PAKNYGDPMKTPKN PAKNECONPRVTKN PAKNEIMPGNFSKN PAKNECONPRVTXN}\PYG{l+s+s1}{\PYGZsq{}}\PYG{p}{]}\PYG{o}{.}\PYG{n}{difpctlevel}\PYG{o}{.}\PYG{n}{mul100}\PYG{o}{.}\PYG{n}{rename}\PYG{p}{(}\PYG{p}{)}\PYG{o}{.}\PYG{n}{plot}\PYG{p}{(}
    \PYG{n}{title}\PYG{o}{=}\PYG{l+s+s2}{\PYGZdq{}}\PYG{l+s+s2}{Temporary 2.5}\PYG{l+s+s2}{\PYGZpc{}}\PYG{l+s+s2}{ boost to cons 2025\PYGZhy{}27 \PYGZhy{} equation active}\PYG{l+s+s2}{\PYGZdq{}}\PYG{p}{)}
\end{sphinxVerbatim}

\end{sphinxuseclass}\end{sphinxVerbatimInput}
\begin{sphinxVerbatimOutput}

\begin{sphinxuseclass}{cell_output}
\begin{sphinxVerbatim}[commandchars=\\\{\}]
The folowing variables are fixed
PAKNECONPRVTKN
\end{sphinxVerbatim}

\noindent\sphinxincludegraphics{{6eb0dbddd5519959c3fe876950d058bfa5bfee0250d5d8ca3c4519433d40dc39}.png}

\end{sphinxuseclass}\end{sphinxVerbatimOutput}

\end{sphinxuseclass}
\sphinxAtStartPar
These results have subtle differences compared with the previous.  The most obvious is visible in looking at the graph for Consumption.  Rather than reverting immediately to its earlier pre\sphinxhyphen{}shock level, it falls more gradually and actually overshoots (falls below its earlier level), before returning slowly to its pre\sphinxhyphen{}shock level.  That is because unlike in the previous shocks, its path is being determined endogenously and reacting to changes elsewhere in the model, notably changes to prices, wages and government spending as well as the lagged level of consumption.

\begin{sphinxuseclass}{cell}\begin{sphinxVerbatimInput}

\begin{sphinxuseclass}{cell_input}
\begin{sphinxVerbatim}[commandchars=\\\{\}]
\PYG{n+nb}{print}\PYG{p}{(}\PYG{l+s+s1}{\PYGZsq{}}\PYG{l+s+s1}{Consumption base and shock levels}\PYG{l+s+se}{\PYGZbs{}r}\PYG{l+s+se}{\PYGZbs{}n}\PYG{l+s+s1}{\PYGZsq{}}\PYG{p}{)}\PYG{p}{;}

\PYG{n+nb}{print}\PYG{p}{(}\PYG{l+s+s1}{\PYGZsq{}}\PYG{l+s+s1}{Real values in 2030}\PYG{l+s+s1}{\PYGZsq{}}\PYG{p}{)}\PYG{p}{;}
\PYG{n+nb}{print}\PYG{p}{(}\PYG{l+s+sa}{f}\PYG{l+s+s1}{\PYGZsq{}}\PYG{l+s+s1}{Base value:  }\PYG{l+s+si}{\PYGZob{}}\PYG{n}{bline}\PYG{o}{.}\PYG{n}{loc}\PYG{p}{[}\PYG{l+m+mi}{2028}\PYG{p}{,}\PYG{l+s+s2}{\PYGZdq{}}\PYG{l+s+s2}{PAKNECONPRVTKN}\PYG{l+s+s2}{\PYGZdq{}}\PYG{p}{]}\PYG{l+s+si}{:}\PYG{l+s+s1}{,.0f}\PYG{l+s+si}{\PYGZcb{}}\PYG{l+s+s1}{.}\PYG{l+s+se}{\PYGZbs{}t}\PYG{l+s+s1}{Shocked value: }\PYG{l+s+si}{\PYGZob{}}\PYG{n}{CExogTempRes}\PYG{o}{.}\PYG{n}{loc}\PYG{p}{[}\PYG{l+m+mi}{2028}\PYG{p}{,}\PYG{l+s+s2}{\PYGZdq{}}\PYG{l+s+s2}{PAKNECONPRVTKN}\PYG{l+s+s2}{\PYGZdq{}}\PYG{p}{]}\PYG{l+s+si}{:}\PYG{l+s+s1}{,.0f}\PYG{l+s+si}{\PYGZcb{}}\PYG{l+s+s1}{.}\PYG{l+s+se}{\PYGZbs{}r}\PYG{l+s+se}{\PYGZbs{}n}\PYG{l+s+s1}{\PYGZsq{}}
    \PYG{l+s+sa}{f}\PYG{l+s+s1}{\PYGZsq{}}\PYG{l+s+s1}{Percent difference: }\PYG{l+s+si}{\PYGZob{}}\PYG{n+nb}{round}\PYG{p}{(}\PYG{l+m+mi}{100}\PYG{o}{*}\PYG{p}{(}\PYG{p}{(}\PYG{n}{CExogTempRes}\PYG{o}{.}\PYG{n}{loc}\PYG{p}{[}\PYG{l+m+mi}{2030}\PYG{p}{,}\PYG{l+s+s2}{\PYGZdq{}}\PYG{l+s+s2}{PAKNECONPRVTKN}\PYG{l+s+s2}{\PYGZdq{}}\PYG{p}{]}\PYG{o}{\PYGZhy{}}\PYG{n}{bline}\PYG{o}{.}\PYG{n}{loc}\PYG{p}{[}\PYG{l+m+mi}{2028}\PYG{p}{,}\PYG{l+s+s2}{\PYGZdq{}}\PYG{l+s+s2}{PAKNECONPRVTKN}\PYG{l+s+s2}{\PYGZdq{}}\PYG{p}{]}\PYG{p}{)}\PYG{o}{/}\PYG{n}{bline}\PYG{o}{.}\PYG{n}{loc}\PYG{p}{[}\PYG{l+m+mi}{2028}\PYG{p}{,}\PYG{l+s+s2}{\PYGZdq{}}\PYG{l+s+s2}{PAKNECONPRVTKN}\PYG{l+s+s2}{\PYGZdq{}}\PYG{p}{]}\PYG{p}{)}\PYG{p}{,}\PYG{l+m+mi}{2}\PYG{p}{)}\PYG{l+s+si}{\PYGZcb{}}\PYG{l+s+s1}{\PYGZsq{}}\PYG{p}{)}
\PYG{n+nb}{print}\PYG{p}{(}\PYG{l+s+s1}{\PYGZsq{}}\PYG{l+s+se}{\PYGZbs{}r}\PYG{l+s+se}{\PYGZbs{}n}\PYG{l+s+s1}{Real values in 2040}\PYG{l+s+s1}{\PYGZsq{}}\PYG{p}{)}\PYG{p}{;}
\PYG{n+nb}{print}\PYG{p}{(}\PYG{l+s+sa}{f}\PYG{l+s+s1}{\PYGZsq{}}\PYG{l+s+s1}{Base value:  }\PYG{l+s+si}{\PYGZob{}}\PYG{n}{bline}\PYG{o}{.}\PYG{n}{loc}\PYG{p}{[}\PYG{l+m+mi}{2040}\PYG{p}{,}\PYG{l+s+s2}{\PYGZdq{}}\PYG{l+s+s2}{PAKNECONPRVTKN}\PYG{l+s+s2}{\PYGZdq{}}\PYG{p}{]}\PYG{l+s+si}{:}\PYG{l+s+s1}{,.0f}\PYG{l+s+si}{\PYGZcb{}}\PYG{l+s+s1}{.}\PYG{l+s+se}{\PYGZbs{}t}\PYG{l+s+s1}{Shocked value: }\PYG{l+s+si}{\PYGZob{}}\PYG{n}{CExogTempRes}\PYG{o}{.}\PYG{n}{loc}\PYG{p}{[}\PYG{l+m+mi}{2040}\PYG{p}{,}\PYG{l+s+s2}{\PYGZdq{}}\PYG{l+s+s2}{PAKNECONPRVTKN}\PYG{l+s+s2}{\PYGZdq{}}\PYG{p}{]}\PYG{l+s+si}{:}\PYG{l+s+s1}{,.0f}\PYG{l+s+si}{\PYGZcb{}}\PYG{l+s+s1}{.}\PYG{l+s+se}{\PYGZbs{}r}\PYG{l+s+se}{\PYGZbs{}n}\PYG{l+s+s1}{\PYGZsq{}}
    \PYG{l+s+sa}{f}\PYG{l+s+s1}{\PYGZsq{}}\PYG{l+s+s1}{Percent difference: }\PYG{l+s+si}{\PYGZob{}}\PYG{n+nb}{round}\PYG{p}{(}\PYG{l+m+mi}{100}\PYG{o}{*}\PYG{p}{(}\PYG{p}{(}\PYG{n}{CExogTempRes}\PYG{o}{.}\PYG{n}{loc}\PYG{p}{[}\PYG{l+m+mi}{2040}\PYG{p}{,}\PYG{l+s+s2}{\PYGZdq{}}\PYG{l+s+s2}{PAKNECONPRVTKN}\PYG{l+s+s2}{\PYGZdq{}}\PYG{p}{]}\PYG{o}{\PYGZhy{}}\PYG{n}{bline}\PYG{o}{.}\PYG{n}{loc}\PYG{p}{[}\PYG{l+m+mi}{2040}\PYG{p}{,}\PYG{l+s+s2}{\PYGZdq{}}\PYG{l+s+s2}{PAKNECONPRVTKN}\PYG{l+s+s2}{\PYGZdq{}}\PYG{p}{]}\PYG{p}{)}\PYG{o}{/}\PYG{n}{bline}\PYG{o}{.}\PYG{n}{loc}\PYG{p}{[}\PYG{l+m+mi}{2040}\PYG{p}{,}\PYG{l+s+s2}{\PYGZdq{}}\PYG{l+s+s2}{PAKNECONPRVTKN}\PYG{l+s+s2}{\PYGZdq{}}\PYG{p}{]}\PYG{p}{)}\PYG{p}{,}\PYG{l+m+mi}{2}\PYG{p}{)}\PYG{l+s+si}{\PYGZcb{}}\PYG{l+s+s1}{\PYGZsq{}}\PYG{p}{)}
\end{sphinxVerbatim}

\end{sphinxuseclass}\end{sphinxVerbatimInput}
\begin{sphinxVerbatimOutput}

\begin{sphinxuseclass}{cell_output}
\begin{sphinxVerbatim}[commandchars=\\\{\}]
Consumption base and shock levels

Real values in 2030
Base value:  27,241,278.	Shocked value: 27,616,949.
Percent difference: 5.36

Real values in 2040
Base value:  38,676,995.	Shocked value: 38,693,167.
Percent difference: 0.04
\end{sphinxVerbatim}

\end{sphinxuseclass}\end{sphinxVerbatimOutput}

\end{sphinxuseclass}

\subsection{Simulation with Add factors}
\label{\detokenize{content/05_WBModels/ScenarioAnalysis:simulation-with-add-factors}}
\sphinxAtStartPar
Add factors are a crucial element of the macromodels of the World Bank and serve multiple purposes.

\sphinxAtStartPar
In simulation, add\sphinxhyphen{}factors allow simulations to be conducted \sphinxstylestrong{without} de\sphinxhyphen{}activating behavioural equations.  Such shocks are often referred to as \sphinxstylestrong{endogenous} shocks because the equation of the behavioural variable that is shocked remains  active throughout.

\sphinxAtStartPar
In some ways they are very similar to a temporary exogenous shock. Both ways of producing the shock allow the shocked variable to respond endogenously in the period after the shock.  The main difference between the two approaches is that:
\begin{itemize}
\item {} 
\sphinxAtStartPar
\sphinxstylestrong{Endogenous} shocks (Add\sphinxhyphen{}Factor shocks) allow the shocked variable to respond to changed circumstances that occur during the period of the shock.
\begin{itemize}
\item {} 
\sphinxAtStartPar
This approach makes most sense for “animal spirits”, shocks where the underlying behaviour is expected to change.

\item {} 
\sphinxAtStartPar
It also makes sense when actions of one part of an aggregate is likely to impact behaviour of other sectors within an aggregate

\item {} 
\sphinxAtStartPar
increased investment by a particular sector would be an example here as the associated increase in activity is likely to increase investment incentives in other sectors, while increased demand for savings will increase interest rates and the cost of capital operating in the opposite direction.

\item {} 
\sphinxAtStartPar
Sustained changes in behaviour, for example increased propensity to invest because of improved recognition

\end{itemize}

\item {} 
\sphinxAtStartPar
\sphinxstylestrong{Exogenous} shocks to endogenous variables fix the level of the shocked variable during the shock period.
\begin{itemize}
\item {} 
\sphinxAtStartPar
Changes in government spending policy, something that is often largely an economically exogenous decision.

\end{itemize}

\end{itemize}


\subsubsection{Simulating the impact of a planned investment}
\label{\detokenize{content/05_WBModels/ScenarioAnalysis:simulating-the-impact-of-a-planned-investment}}
\sphinxAtStartPar
The below simulation uses the add\sphinxhyphen{}factor to simulate the impact of a 3 year investment  program beginning in 2025 of 1 percent of GDP per year, that is financed through an increase in foreign direct investment. This might reflect a specific large scale plant that is being constructed due to a deal reached by the government with a foreign manufacturer.  The add\sphinxhyphen{}factor approach is chosen because the additional investment is likely to increase demand for the products of other firms, which is likely to incite them to add to their investments as well.


\paragraph{How to translate the economic shock into a model shock}
\label{\detokenize{content/05_WBModels/ScenarioAnalysis:how-to-translate-the-economic-shock-into-a-model-shock}}
\sphinxAtStartPar
Add\sphinxhyphen{}factors in the \sphinxcode{\sphinxupquote{MFMod}} framework are applied to the intercept of an equation (not the level of the dependent variable).  This preserves the estimated elasticities of the equation, but makes introduction of an add\sphinxhyphen{}factor shock somewhat more complicated than the exogenous approach.  Below a step\sphinxhyphen{}by\sphinxhyphen{}step how\sphinxhyphen{}to guide:
\begin{enumerate}
\sphinxsetlistlabels{\arabic}{enumi}{enumii}{}{.}%
\item {} 
\sphinxAtStartPar
Identify numerical size of the shock

\item {} 
\sphinxAtStartPar
Examine the functional form of the equation, to determine the nature of the add factor.  If the equation is expressed as a:
\begin{itemize}
\item {} 
\sphinxAtStartPar
\sphinxstylestrong{growth rate} then the add\sphinxhyphen{}factor will be an addition or subtraction to the growth rate

\item {} 
\sphinxAtStartPar
\sphinxstylestrong{percent of GDP (or some other level)} then the add\sphinxhyphen{}factor will be an addition or subtraction to the share of growth.

\item {} 
\sphinxAtStartPar
\sphinxstylestrong{Level} then the add\sphinxhyphen{}factor will be a direct addition to the level of the dependent variable

\end{itemize}

\item {} 
\sphinxAtStartPar
Convert the economic shock into the units of the add\sphinxhyphen{}factor

\item {} 
\sphinxAtStartPar
Shock the add\sphinxhyphen{}factor by the above amount and run the model
\begin{itemize}
\item {} 
\sphinxAtStartPar
Note the add\sphinxhyphen{}factor is an exogenous variable in the model, so shocking it follows the well established process for shocking an exogenous variable.

\end{itemize}

\end{enumerate}


\paragraph{Determine the size of shock}
\label{\detokenize{content/05_WBModels/ScenarioAnalysis:determine-the-size-of-shock}}
\sphinxAtStartPar
Above we identified the shock as to be a 1 percent of GDP increase in FDI that flows directly into private\sphinxhyphen{}sector investment.  A first step would be to determine the variables that need to be shocked (FDI) and private investment. To do this we can query the variable dictionary.

\begin{sphinxuseclass}{cell}\begin{sphinxVerbatimInput}

\begin{sphinxuseclass}{cell_input}
\begin{sphinxVerbatim}[commandchars=\\\{\}]
\PYG{n}{mpak}\PYG{p}{[}\PYG{l+s+s1}{\PYGZsq{}}\PYG{l+s+s1}{*NY*}\PYG{l+s+s1}{\PYGZsq{}}\PYG{p}{]}\PYG{o}{.}\PYG{n}{des}
\end{sphinxVerbatim}

\end{sphinxuseclass}\end{sphinxVerbatimInput}
\begin{sphinxVerbatimOutput}

\begin{sphinxuseclass}{cell_output}
\begin{sphinxVerbatim}[commandchars=\\\{\}]
PAKNYGDPDISCCN            : GDP Disc., LCU mn
PAKNYGDPDISCKN            : GDP Disc., 2000 LCU mn
PAKNYGDPFCSTCN            : GDP Factor Cost Local Currency units Volumes National base year
PAKNYGDPFCSTKN            : GDP Factor Cost Local Currency units Volumes National base year
PAKNYGDPFCSTXN            : GDP Factor Cost Local Currency units Implicit Price deflator
PAKNYGDPFCSTXN\PYGZus{}A          : Add factor:GDP Factor Cost Local Currency units Implicit Price deflator
PAKNYGDPFCSTXN\PYGZus{}D          : Fix dummy:GDP Factor Cost Local Currency units Implicit Price deflator
PAKNYGDPFCSTXN\PYGZus{}FITTED     : Fitted  value:GDP Factor Cost Local Currency units Implicit Price deflator
PAKNYGDPFCSTXN\PYGZus{}X          : Fix value:GDP Factor Cost Local Currency units Implicit Price deflator
PAKNYGDPGAP\PYGZus{}              : Output Gap (\PYGZpc{} of Potential GDP)
PAKNYGDPMKTPCD            : GDP, Market Prices, US\PYGZdl{} mn
PAKNYGDPMKTPCN            : GDP, Market Prices, LCU mn
PAKNYGDPMKTPCN\PYGZus{}VALUE\PYGZus{}2010 : PAKNYGDPMKTPCN\PYGZus{}VALUE\PYGZus{}2010
PAKNYGDPMKTPKD            : GDP, Market Prices, 2000 US\PYGZdl{} mn
PAKNYGDPMKTPKN            : Real GDP
PAKNYGDPMKTPKN\PYGZus{}VALUE\PYGZus{}2010 : PAKNYGDPMKTPKN\PYGZus{}VALUE\PYGZus{}2010
PAKNYGDPMKTPXN            : GDP, Marker Prices, LCU Price defl., 2000 = 1
PAKNYGDPPOTLKN            : Potential Output, constant LCU
PAKNYGDPTFP               : Total factor productivity
PAKNYTAXNINDCN            : Net Indirect Taxes Local Currency units Values
PAKNYTAXNINDKN            : Net Indirect Taxes Local Currency units Volumes National base year
PAKNYWBFORMSH             : PAKNYWBFORMSH
PAKNYWBINFMSH             : PAKNYWBINFMSH
PAKNYWRTFORMCN            : PAKNYWRTFORMCN
PAKNYWRTFORMCN\PYGZus{}A          : Add factor:PAKNYWRTFORMCN
PAKNYWRTFORMCN\PYGZus{}D          : Fix dummy:PAKNYWRTFORMCN
PAKNYWRTFORMCN\PYGZus{}FITTED     : Fitted  value:PAKNYWRTFORMCN
PAKNYWRTFORMCN\PYGZus{}X          : Fix value:PAKNYWRTFORMCN
PAKNYWRTINFMCN            : PAKNYWRTINFMCN
PAKNYWRTINFMCN\PYGZus{}A          : Add factor:PAKNYWRTINFMCN
PAKNYWRTINFMCN\PYGZus{}D          : Fix dummy:PAKNYWRTINFMCN
PAKNYWRTINFMCN\PYGZus{}FITTED     : Fitted  value:PAKNYWRTINFMCN
PAKNYWRTINFMCN\PYGZus{}X          : Fix value:PAKNYWRTINFMCN
PAKNYWRTTOTLCN            : PAKNYWRTTOTLCN
PAKNYYGOSOTLCN            : PAKNYYGOSOTLCN
PAKNYYWBFORMCN            : PAKNYYWBFORMCN
PAKNYYWBINFMCN            : PAKNYYWBINFMCN
PAKNYYWBINFMCN\PYGZus{}           : PAKNYYWBINFMCN\PYGZus{}
PAKNYYWBTOTLCN            : Total Wage Bill
PAKNYYWBTOTLCN\PYGZus{}           : Labor Share of Income
\end{sphinxVerbatim}

\end{sphinxuseclass}\end{sphinxVerbatimOutput}

\end{sphinxuseclass}

\paragraph{Identify the functional form(s)}
\label{\detokenize{content/05_WBModels/ScenarioAnalysis:identify-the-functional-form-s}}
\sphinxAtStartPar
To understand how to shock using the add factor, it is essential to understand how the add\sphinxhyphen{}factor enters into the equation.


\begin{savenotes}\sphinxattablestart
\centering
\begin{tabulary}{\linewidth}[t]{|T|T|}
\hline
\sphinxstyletheadfamily 
\sphinxAtStartPar
Addfactor is on the intercept of
&\sphinxstyletheadfamily 
\sphinxAtStartPar
Shock needs to be calculated as
\\
\hline
\sphinxAtStartPar
a growth equation
&
\sphinxAtStartPar
a change in th growth rate
\\
\hline
\sphinxAtStartPar
Share of GDP
&
\sphinxAtStartPar
a percent of GDP
\\
\hline
\sphinxAtStartPar
Level
&
\sphinxAtStartPar
as change in the level
\\
\hline
\end{tabulary}
\par
\sphinxattableend\end{savenotes}

\sphinxAtStartPar
Use the .eviews command or .original command to identify the functional forms if the equation to be shocked.

\begin{sphinxuseclass}{cell}\begin{sphinxVerbatimInput}

\begin{sphinxuseclass}{cell_input}
\begin{sphinxVerbatim}[commandchars=\\\{\}]
\PYG{n}{mpak}\PYG{p}{[}\PYG{l+s+s1}{\PYGZsq{}}\PYG{l+s+s1}{PAKNEGDIFPRVKN}\PYG{l+s+s1}{\PYGZsq{}}\PYG{p}{]}\PYG{o}{.}\PYG{n}{des}
\end{sphinxVerbatim}

\end{sphinxuseclass}\end{sphinxVerbatimInput}
\begin{sphinxVerbatimOutput}

\begin{sphinxuseclass}{cell_output}
\begin{sphinxVerbatim}[commandchars=\\\{\}]
PAKNEGDIFPRVKN : Prvt. Investment real
\end{sphinxVerbatim}

\end{sphinxuseclass}\end{sphinxVerbatimOutput}

\end{sphinxuseclass}
\sphinxAtStartPar
This needs to be rewritten to use the eviews expression when published / Done Ib

\begin{sphinxuseclass}{cell}\begin{sphinxVerbatimInput}

\begin{sphinxuseclass}{cell_input}
\begin{sphinxVerbatim}[commandchars=\\\{\}]
\PYG{c+c1}{\PYGZsh{} This needs to be rewritten to use the eviews expression when published / Done Ib }
\PYG{n}{mpak}\PYG{p}{[}\PYG{l+s+s1}{\PYGZsq{}}\PYG{l+s+s1}{PAKNEGDIFPRVKN}\PYG{l+s+s1}{\PYGZsq{}}\PYG{p}{]}\PYG{o}{.}\PYG{n}{eviews}
\end{sphinxVerbatim}

\end{sphinxuseclass}\end{sphinxVerbatimInput}
\begin{sphinxVerbatimOutput}

\begin{sphinxuseclass}{cell_output}
\begin{sphinxVerbatim}[commandchars=\\\{\}]
PAKNEGDIFPRVKN : (PAKNEGDIFPRVKN/PAKNEGDIKSTKKN( \PYGZhy{} 1)) = 0.00212272413966296 + 0.970234989019907*(PAKNEGDIFPRVKN( \PYGZhy{} 1)/PAKNEGDIKSTKKN( \PYGZhy{} 2)) + (1 \PYGZhy{} 0.970234989019907)*(DLOG(PAKNYGDPPOTLKN) + PAKDEPR) + 0.0525240494260597*DLOG(PAKNEKRTTOTLCN/PAKNYGDPFCSTXN)
\end{sphinxVerbatim}

\end{sphinxuseclass}\end{sphinxVerbatimOutput}

\end{sphinxuseclass}

\paragraph{Calculate the size of the required add factor shock}
\label{\detokenize{content/05_WBModels/ScenarioAnalysis:calculate-the-size-of-the-required-add-factor-shock}}
\sphinxAtStartPar
The shock to be executed is 0.5 percent of GDP.

\sphinxAtStartPar
It is assumed that the financing will come from FDI and that all the money will be spent in one year on private investment.

\sphinxAtStartPar
The private investment equation is written as a share of the capital stock.  Therefore, the add\sphinxhyphen{}factor needs to be shocked by adding 1 percent of GDP to private investment in 2028 divided by the capital stock in 2028.

\begin{sphinxuseclass}{cell}\begin{sphinxVerbatimInput}

\begin{sphinxuseclass}{cell_input}
\begin{sphinxVerbatim}[commandchars=\\\{\}]
\PYG{c+c1}{\PYGZsh{}Create a DataFrame AFShock that is equal tothe baseline}
\PYG{n}{AFShock}\PYG{o}{=}\PYG{n}{bline}

\PYG{c+c1}{\PYGZsh{}Display the level of the AF}
\PYG{n}{mpak}\PYG{p}{[}\PYG{p}{[}\PYG{l+s+s1}{\PYGZsq{}}\PYG{l+s+s1}{PAKNEGDIFPRVKN\PYGZus{}A}\PYG{l+s+s1}{\PYGZsq{}}\PYG{p}{,}\PYG{l+s+s1}{\PYGZsq{}}\PYG{l+s+s1}{PAKNEGDIFPRVKN}\PYG{l+s+s1}{\PYGZsq{}}\PYG{p}{,}\PYG{l+s+s1}{\PYGZsq{}}\PYG{l+s+s1}{PAKNEGDIKSTKKN}\PYG{l+s+s1}{\PYGZsq{}}\PYG{p}{]}\PYG{p}{]}\PYG{o}{.}\PYG{n}{des}
\PYG{n+nb}{print}\PYG{p}{(}\PYG{l+s+s2}{\PYGZdq{}}\PYG{l+s+s2}{Pre shock levels}\PYG{l+s+s2}{\PYGZdq{}}\PYG{p}{)}
\PYG{n}{AFShock}\PYG{o}{.}\PYG{n}{loc}\PYG{p}{[}\PYG{l+m+mi}{2025}\PYG{p}{:}\PYG{l+m+mi}{2030}\PYG{p}{,}\PYG{p}{[}\PYG{l+s+s1}{\PYGZsq{}}\PYG{l+s+s1}{PAKNEGDIFPRVKN\PYGZus{}A}\PYG{l+s+s1}{\PYGZsq{}}\PYG{p}{,}\PYG{l+s+s1}{\PYGZsq{}}\PYG{l+s+s1}{PAKNEGDIFPRVKN}\PYG{l+s+s1}{\PYGZsq{}}\PYG{p}{,}\PYG{l+s+s1}{\PYGZsq{}}\PYG{l+s+s1}{PAKNEGDIKSTKKN}\PYG{l+s+s1}{\PYGZsq{}}\PYG{p}{]}\PYG{p}{]}

\PYG{c+c1}{\PYGZsh{}print(AFShock.loc[2025:2030,\PYGZsq{}PAKNEGDIFPRVKN\PYGZsq{}]/AFShock.loc[2025:2030,\PYGZsq{}PAKNYGDPMKTPKN\PYGZsq{}]*100)}
\end{sphinxVerbatim}

\end{sphinxuseclass}\end{sphinxVerbatimInput}
\begin{sphinxVerbatimOutput}

\begin{sphinxuseclass}{cell_output}
\begin{sphinxVerbatim}[commandchars=\\\{\}]
PAKNEGDIFPRVKN\PYGZus{}A : Add factor:Prvt. Investment real
PAKNEGDIFPRVKN   : Prvt. Investment real
PAKNEGDIKSTKKN   : Capital stock, LCU
Pre shock levels
\end{sphinxVerbatim}

\begin{sphinxVerbatim}[commandchars=\\\{\}]
      PAKNEGDIFPRVKN\PYGZus{}A  PAKNEGDIFPRVKN  PAKNEGDIKSTKKN
2025         \PYGZhy{}0.000458    1.602854e+06    4.730392e+07
2026         \PYGZhy{}0.000389    1.581104e+06    4.814879e+07
2027         \PYGZhy{}0.000331    1.569541e+06    4.900980e+07
2028         \PYGZhy{}0.000281    1.569141e+06    4.989869e+07
2029         \PYGZhy{}0.000239    1.580577e+06    5.082694e+07
2030         \PYGZhy{}0.000203    1.604394e+06    5.180590e+07
\end{sphinxVerbatim}

\end{sphinxuseclass}\end{sphinxVerbatimOutput}

\end{sphinxuseclass}
\sphinxAtStartPar
Below the mfcalc routine is used to set the addfactor variable equal to its previous value plus the equivalent of 1 percent of GDP when expressed as a percent of the previous period’s level of private investment.

\begin{sphinxuseclass}{cell}\begin{sphinxVerbatimInput}

\begin{sphinxuseclass}{cell_input}
\begin{sphinxVerbatim}[commandchars=\\\{\}]
\PYG{n}{AFShock}\PYG{o}{=}\PYG{n}{AFShock}\PYG{o}{.}\PYG{n}{mfcalc}\PYG{p}{(}\PYG{l+s+s2}{\PYGZdq{}}\PYG{l+s+s2}{\PYGZlt{}2028 2028\PYGZgt{} PAKNEGDIFPRVKN\PYGZus{}A = PAKNEGDIFPRVKN\PYGZus{}A + (.01*PAKNYGDPMKTPKN/PAKNEGDIKSTKKN)}\PYG{l+s+s2}{\PYGZdq{}}\PYG{p}{)}\PYG{p}{;}

\PYG{n+nb}{print}\PYG{p}{(}\PYG{l+s+s2}{\PYGZdq{}}\PYG{l+s+s2}{Post shock levels}\PYG{l+s+s2}{\PYGZdq{}}\PYG{p}{)}
\PYG{n}{AFShock}\PYG{o}{.}\PYG{n}{loc}\PYG{p}{[}\PYG{l+m+mi}{2025}\PYG{p}{:}\PYG{l+m+mi}{2030}\PYG{p}{,}\PYG{l+s+s1}{\PYGZsq{}}\PYG{l+s+s1}{PAKNEGDIFPRVKN\PYGZus{}A}\PYG{l+s+s1}{\PYGZsq{}}\PYG{p}{]}
\end{sphinxVerbatim}

\end{sphinxuseclass}\end{sphinxVerbatimInput}
\begin{sphinxVerbatimOutput}

\begin{sphinxuseclass}{cell_output}
\begin{sphinxVerbatim}[commandchars=\\\{\}]
Post shock levels
\end{sphinxVerbatim}

\begin{sphinxVerbatim}[commandchars=\\\{\}]
2025   \PYGZhy{}0.000458
2026   \PYGZhy{}0.000389
2027   \PYGZhy{}0.000331
2028    0.005774
2029   \PYGZhy{}0.000239
2030   \PYGZhy{}0.000203
Name: PAKNEGDIFPRVKN\PYGZus{}A, dtype: float64
\end{sphinxVerbatim}

\end{sphinxuseclass}\end{sphinxVerbatimOutput}

\end{sphinxuseclass}

\paragraph{Run the shock}
\label{\detokenize{content/05_WBModels/ScenarioAnalysis:run-the-shock}}
\begin{sphinxuseclass}{cell}\begin{sphinxVerbatimInput}

\begin{sphinxuseclass}{cell_input}
\begin{sphinxVerbatim}[commandchars=\\\{\}]
\PYG{n}{AFShockRes} \PYG{o}{=} \PYG{n}{mpak}\PYG{p}{(}\PYG{n}{AFShock}\PYG{p}{,}\PYG{l+m+mi}{2020}\PYG{p}{,}\PYG{l+m+mi}{2040}\PYG{p}{,}\PYG{n}{keep}\PYG{o}{=}\PYG{l+s+s1}{\PYGZsq{}}\PYG{l+s+s1}{1}\PYG{l+s+si}{\PYGZpc{} o}\PYG{l+s+s1}{f GDP increase in FDI and private investment (AF shock)}\PYG{l+s+s1}{\PYGZsq{}}\PYG{p}{)}
\PYG{n}{mpak}\PYG{p}{[}\PYG{l+s+s1}{\PYGZsq{}}\PYG{l+s+s1}{PAKNYGDPMKTPKN PAKNEGDIFPRVKN PAKNECONPRVTKN PAKNEIMPGNFSKN PAKNEGDIFTOTKN PAKNECONPRVTXN}\PYG{l+s+s1}{\PYGZsq{}}\PYG{p}{]}\PYG{o}{.}\PYG{n}{difpctlevel}\PYG{o}{.}\PYG{n}{mul100}\PYG{o}{.}\PYG{n}{rename}\PYG{p}{(}\PYG{p}{)}\PYG{o}{.}\PYG{n}{plot}\PYG{p}{(}
    \PYG{n}{title}\PYG{o}{=}\PYG{l+s+s2}{\PYGZdq{}}\PYG{l+s+s2}{Add factor shock on private investment 1}\PYG{l+s+si}{\PYGZpc{} o}\PYG{l+s+s2}{f GDP}\PYG{l+s+s2}{\PYGZdq{}}\PYG{p}{)}
\end{sphinxVerbatim}

\end{sphinxuseclass}\end{sphinxVerbatimInput}
\begin{sphinxVerbatimOutput}

\begin{sphinxuseclass}{cell_output}
\noindent\sphinxincludegraphics{{5c2b9ccd33bf587a8e61239e4645a20db2d55f97794d5316270c8c95dd990db2}.png}

\end{sphinxuseclass}\end{sphinxVerbatimOutput}

\end{sphinxuseclass}

\chapter{Report writing and scenario results}
\label{\detokenize{content/05_WBModels/ScenarioAnalysis:report-writing-and-scenario-results}}
\sphinxAtStartPar
\sphinxcode{\sphinxupquote{ModelFlow}}, standard pandas routines and other python libraries like \sphinxcode{\sphinxupquote{Matplotlib}} and \sphinxcode{\sphinxupquote{Plotly}} can be used to visualize and compare dataframes and therefore the results, from scenarios – as indeed has been done in the preceding paragraphs.

\sphinxAtStartPar
In addition, \sphinxcode{\sphinxupquote{ModelFlow}} also provides several specific routines that make such comparisons easier.


\section{The Keep option}
\label{\detokenize{content/05_WBModels/ScenarioAnalysis:the-keep-option}}
\sphinxAtStartPar
The \sphinxstylestrong{Keep} option facilitates the comparison of results from different scenarios run on a give model object.  In each of the simulations executed above, the \sphinxcode{\sphinxupquote{keep}} option was activated. This causes the results from each simulation in a unique \sphinxcode{\sphinxupquote{DataFrame}} that can be identified by the descriptor given to it.


\section{The \sphinxstyleliteralintitle{\sphinxupquote{.keep\_plot()}} method}
\label{\detokenize{content/05_WBModels/ScenarioAnalysis:the-keep-plot-method}}
\sphinxAtStartPar
The keep\_plot method can be used to plot and compare results from the various scenarios that had been run earlier using the \sphinxcode{\sphinxupquote{keep=}} option.

\sphinxAtStartPar
By default the results across all scenarios for each selected variables will be shown on one chart at a time.

\begin{sphinxuseclass}{cell}\begin{sphinxVerbatimInput}

\begin{sphinxuseclass}{cell_input}
\begin{sphinxVerbatim}[commandchars=\\\{\}]
\PYG{n}{mpak}\PYG{o}{.}\PYG{n}{keep\PYGZus{}plot}\PYG{p}{(}\PYG{l+s+s1}{\PYGZsq{}}\PYG{l+s+s1}{PAKNYGDPMKTPCN PAKNECONPRVTKN PAKNEIMPGNFSKN}\PYG{l+s+s1}{\PYGZsq{}}\PYG{p}{,}  \PYG{n}{legend}\PYG{o}{=}\PYG{k+kc}{True}\PYG{p}{)}\PYG{p}{;}
\PYG{c+c1}{\PYGZsh{}show for each variable on a separate chart the results from each kept scenario}
\end{sphinxVerbatim}

\end{sphinxuseclass}\end{sphinxVerbatimInput}
\begin{sphinxVerbatimOutput}

\begin{sphinxuseclass}{cell_output}
\noindent\sphinxincludegraphics{{a80779fa393ddbd79cc8cdcfbb692a5a2f8f66ce26c19f461502efe8437620d8}.png}

\noindent\sphinxincludegraphics{{b3467fb82209da66debb03d797dea4955ccf452d3188b4ad5f9ae52c54ce4147}.png}

\noindent\sphinxincludegraphics{{eff7ebb23c31bb7fc082355adbdeb28ff091cd52c9e3a7596051d60822da9a9f}.png}

\end{sphinxuseclass}\end{sphinxVerbatimOutput}

\end{sphinxuseclass}

\subsection{\sphinxstyleliteralintitle{\sphinxupquote{keep\_plot()}} options}
\label{\detokenize{content/05_WBModels/ScenarioAnalysis:keep-plot-options}}
\sphinxAtStartPar
The \sphinxstylestrong{variables} to be displayed are listed as first argument. Variable names can include
wildcards (using * for any string and ? for any character).

\sphinxAtStartPar
\sphinxstylestrong{Transformation of data displayed:}


\begin{savenotes}\sphinxattablestart
\centering
\begin{tabulary}{\linewidth}[t]{|T|T|}
\hline
\sphinxstyletheadfamily 
\sphinxAtStartPar
showtype=
&\sphinxstyletheadfamily 
\sphinxAtStartPar
Use this operator
\\
\hline
\sphinxAtStartPar
‘level’ (default)
&
\sphinxAtStartPar
No transformation
\\
\hline
\sphinxAtStartPar
‘growth’
&
\sphinxAtStartPar
The growth rate  in percent
\\
\hline
\sphinxAtStartPar
‘change’
&
\sphinxAtStartPar
The yearly change (\(\Delta\))
\\
\hline
\end{tabulary}
\par
\sphinxattableend\end{savenotes}

\sphinxAtStartPar
\sphinxstylestrong{legend placement}


\begin{savenotes}\sphinxattablestart
\centering
\begin{tabulary}{\linewidth}[t]{|T|T|}
\hline
\sphinxstyletheadfamily 
\sphinxAtStartPar
legend=
&\sphinxstyletheadfamily 
\sphinxAtStartPar
Use this operator
\\
\hline
\sphinxAtStartPar
False (default)
&
\sphinxAtStartPar
The legends are  placed at the end of the corresponding line
\\
\hline
\sphinxAtStartPar
True
&
\sphinxAtStartPar
The legends are places in a legend box
\\
\hline
\end{tabulary}
\par
\sphinxattableend\end{savenotes}

\sphinxAtStartPar
Often it is useful to compare the scenario results with the baseline result. This is done with the diff argument.


\begin{savenotes}\sphinxattablestart
\centering
\begin{tabulary}{\linewidth}[t]{|T|T|}
\hline
\sphinxstyletheadfamily 
\sphinxAtStartPar
diff=
&\sphinxstyletheadfamily 
\sphinxAtStartPar
Use this operator
\\
\hline
\sphinxAtStartPar
False (default)
&
\sphinxAtStartPar
All entries in the keep\_solution dictionary are displayed
\\
\hline
\sphinxAtStartPar
True
&
\sphinxAtStartPar
The difference to the first entry is shown.
\\
\hline
\end{tabulary}
\par
\sphinxattableend\end{savenotes}

\sphinxAtStartPar
It can also be useful to compare the scenario results with the baseline result \sphinxstylestrong{measured in percent}. This is done with the diffpct argument.


\begin{savenotes}\sphinxattablestart
\centering
\begin{tabulary}{\linewidth}[t]{|T|T|}
\hline
\sphinxstyletheadfamily 
\sphinxAtStartPar
diffpct=
&\sphinxstyletheadfamily 
\sphinxAtStartPar
Use this operator
\\
\hline
\sphinxAtStartPar
False (default)
&
\sphinxAtStartPar
All entries in the keep\_solution dictionary is displayed
\\
\hline
\sphinxAtStartPar
True
&
\sphinxAtStartPar
The difference in percent to the first entry is shown
\\
\hline
\end{tabulary}
\par
\sphinxattableend\end{savenotes}

\begin{sphinxadmonition}{note}{Note:}
\sphinxAtStartPar
\sphinxcode{\sphinxupquote{'keep\_plot()}} and \sphinxcode{\sphinxupquote{.keep\_plot\_multi()}} return a python object that points to the in memory version of the rendered figure(s).  This object can be used to modify the graph (see examples towards the end of this chapter.
\end{sphinxadmonition}

\sphinxAtStartPar
\sphinxcode{\sphinxupquote{savefig='{[}path/{]}<prefix>.<extension>'}}
Will create a number of files with the charts.
The files will be saved location with name \sphinxcode{\sphinxupquote{<path>/<prefix><variable name>.<extension>}} The extension determines the
format of the saved file: \sphinxcode{\sphinxupquote{pdf}}, \sphinxcode{\sphinxupquote{svg}} and \sphinxcode{\sphinxupquote{png}} are the most common extensions.


\subsection{An example using the diff=TRUE option}
\label{\detokenize{content/05_WBModels/ScenarioAnalysis:an-example-using-the-diff-true-option}}
\sphinxAtStartPar
When \sphinxcode{\sphinxupquote{diff=True}} (or 1) results will be shown all of the selected scenarios presented as the change in selected variables with respect to the first scenario – in this instance the scenario saves with the name \sphinxcode{\sphinxupquote{baseline}}.

\begin{sphinxadmonition}{warning}{Warning:}
\sphinxAtStartPar
Note in this instance \sphinxcode{\sphinxupquote{baseline}} and \sphinxcode{\sphinxupquote{basedf}} are the same because they were defined that way.  However, there is nothing in the system that guarantees that the first \sphinxcode{\sphinxupquote{keep}} scenario will be the baseline or the \sphinxcode{\sphinxupquote{basedf}} scenario.
\end{sphinxadmonition}

\begin{sphinxuseclass}{cell}\begin{sphinxVerbatimInput}

\begin{sphinxuseclass}{cell_input}
\begin{sphinxVerbatim}[commandchars=\\\{\}]
\PYG{n}{mpak}\PYG{o}{.}\PYG{n}{keep\PYGZus{}plot}\PYG{p}{(}\PYG{l+s+s1}{\PYGZsq{}}\PYG{l+s+s1}{PAKNYGDPMKTPCN PAKNECONPRVTKN PAKNEIMPGNFSKN}\PYG{l+s+s1}{\PYGZsq{}}\PYG{p}{,} \PYG{n}{diff}\PYG{o}{=}\PYG{l+m+mi}{1}\PYG{p}{,} \PYG{n}{legend}\PYG{o}{=}\PYG{k+kc}{True}\PYG{p}{)}\PYG{p}{;}
\end{sphinxVerbatim}

\end{sphinxuseclass}\end{sphinxVerbatimInput}
\begin{sphinxVerbatimOutput}

\begin{sphinxuseclass}{cell_output}
\noindent\sphinxincludegraphics{{4ed3a6faba74255578850aa87591a3ef66edfb86d94f60abf444c9784c98ddcb}.png}

\noindent\sphinxincludegraphics{{6068f7cfc39d5183906a46c9b317b7b30387deeef6e1f3b9f64f632ab44b955d}.png}

\noindent\sphinxincludegraphics{{1df642e2b515fb53cb88dfe69da3ca7c6ba66dea3e36f58cd65910bd96e4b7d1}.png}

\end{sphinxuseclass}\end{sphinxVerbatimOutput}

\end{sphinxuseclass}

\subsection{The \sphinxstyleliteralintitle{\sphinxupquote{showtype}} option}
\label{\detokenize{content/05_WBModels/ScenarioAnalysis:the-showtype-option}}
\sphinxAtStartPar
In this example the difference with respect first \sphinxcode{\sphinxupquote{'keep}} scenario \sphinxcode{\sphinxupquote{baseline}} values are once again shown. This time the \sphinxcode{\sphinxupquote{showtype}} option has been set to \sphinxcode{\sphinxupquote{growth}}.  As a result the data is displayed as the  difference in the growth rate.

\begin{sphinxuseclass}{cell}\begin{sphinxVerbatimInput}

\begin{sphinxuseclass}{cell_input}
\begin{sphinxVerbatim}[commandchars=\\\{\}]
\PYG{n}{mpak}\PYG{o}{.}\PYG{n}{keep\PYGZus{}plot}\PYG{p}{(}\PYG{l+s+s1}{\PYGZsq{}}\PYG{l+s+s1}{PAKNYGDPMKTPCN PAKNEIMPGNFSKN}\PYG{l+s+s1}{\PYGZsq{}}\PYG{p}{,} \PYG{n}{diff}\PYG{o}{=}\PYG{l+m+mi}{1}\PYG{p}{,}\PYG{n}{showtype}\PYG{o}{=}\PYG{l+s+s1}{\PYGZsq{}}\PYG{l+s+s1}{growth}\PYG{l+s+s1}{\PYGZsq{}}\PYG{p}{,} \PYG{n}{legend}\PYG{o}{=}\PYG{k+kc}{True}\PYG{p}{)}\PYG{p}{;}
\end{sphinxVerbatim}

\end{sphinxuseclass}\end{sphinxVerbatimInput}
\begin{sphinxVerbatimOutput}

\begin{sphinxuseclass}{cell_output}
\noindent\sphinxincludegraphics{{90bbf1d36ebb87ef3938c818ea87f9e7d4a087c25792b6c4ce335d9dd7ddf4e3}.png}

\noindent\sphinxincludegraphics{{193460901e5b43bf8e7c979cb8b4ebd41896b733bd18742b47e277ba8c4aec72}.png}

\end{sphinxuseclass}\end{sphinxVerbatimOutput}

\end{sphinxuseclass}

\subsection{The diffpct option}
\label{\detokenize{content/05_WBModels/ScenarioAnalysis:the-diffpct-option}}
\sphinxAtStartPar
Setting \sphinxcode{\sphinxupquote{diffpct=True}} instructs \sphinxcode{\sphinxupquote{.keep\_plot()}} to display the data as a percent deviation from the first \sphinxcode{\sphinxupquote{kep}} scenario.

\begin{sphinxuseclass}{cell}\begin{sphinxVerbatimInput}

\begin{sphinxuseclass}{cell_input}
\begin{sphinxVerbatim}[commandchars=\\\{\}]
\PYG{n}{mpak}\PYG{o}{.}\PYG{n}{keep\PYGZus{}plot}\PYG{p}{(}\PYG{l+s+s1}{\PYGZsq{}}\PYG{l+s+s1}{PAKNYGDPMKTPCN  PAKNEIMPGNFSKN}\PYG{l+s+s1}{\PYGZsq{}}\PYG{p}{,} \PYG{n}{diffpct}\PYG{o}{=}\PYG{l+m+mi}{1}\PYG{p}{,}\PYG{n}{legend}\PYG{o}{=}\PYG{l+s+s2}{\PYGZdq{}}\PYG{l+s+s2}{Change in level as a }\PYG{l+s+si}{\PYGZpc{} o}\PYG{l+s+s2}{f first keep scenario}\PYG{l+s+s2}{\PYGZdq{}}\PYG{p}{)}\PYG{p}{;}
\end{sphinxVerbatim}

\end{sphinxuseclass}\end{sphinxVerbatimInput}
\begin{sphinxVerbatimOutput}

\begin{sphinxuseclass}{cell_output}
\noindent\sphinxincludegraphics{{4b7a0f485c3bda9b03cd6ef8415efe8594fce756039adbbc3eba4b1a904fccb9}.png}

\noindent\sphinxincludegraphics{{6278f82621bf975487c10a155f909c73cf0dfeb85d9725aa26c5989599943638}.png}

\end{sphinxuseclass}\end{sphinxVerbatimOutput}

\end{sphinxuseclass}

\subsubsection{Differences in percent of baseline values}
\label{\detokenize{content/05_WBModels/ScenarioAnalysis:differences-in-percent-of-baseline-values}}
\sphinxAtStartPar
In this plot, the same results are presented, but as percent deviations from the baseline values of the displayed data.

\begin{sphinxuseclass}{cell}\begin{sphinxVerbatimInput}

\begin{sphinxuseclass}{cell_input}
\begin{sphinxVerbatim}[commandchars=\\\{\}]
\PYG{n}{mpak}\PYG{o}{.}\PYG{n}{keep\PYGZus{}plot}\PYG{p}{(}\PYG{l+s+s1}{\PYGZsq{}}\PYG{l+s+s1}{PAKNYGDPMKTPCN PAKNECONPRVTKN }\PYG{l+s+s1}{\PYGZsq{}}\PYG{p}{,} \PYG{n}{diffpct}\PYG{o}{=}\PYG{l+m+mi}{1}\PYG{p}{,}\PYG{n}{showtype}\PYG{o}{=}\PYG{l+s+s1}{\PYGZsq{}}\PYG{l+s+s1}{level}\PYG{l+s+s1}{\PYGZsq{}}\PYG{p}{,} \PYG{n}{legend}\PYG{o}{=}\PYG{k+kc}{True}\PYG{p}{)}\PYG{p}{;}
\end{sphinxVerbatim}

\end{sphinxuseclass}\end{sphinxVerbatimInput}
\begin{sphinxVerbatimOutput}

\begin{sphinxuseclass}{cell_output}
\noindent\sphinxincludegraphics{{4b7a0f485c3bda9b03cd6ef8415efe8594fce756039adbbc3eba4b1a904fccb9}.png}

\noindent\sphinxincludegraphics{{1be61a7c5fe15786c3eab55cc284214194a1eed7ec84ff124b3ef644f10f4215}.png}

\end{sphinxuseclass}\end{sphinxVerbatimOutput}

\end{sphinxuseclass}

\subsection{The \sphinxstyleliteralintitle{\sphinxupquote{.keep\_switch()}} method. Selection of scenarios}
\label{\detokenize{content/05_WBModels/ScenarioAnalysis:the-keep-switch-method-selection-of-scenarios}}
\sphinxAtStartPar
The \sphinxcode{\sphinxupquote{.keep\_switch()}} method restricts the number of scenarios on which subsequent calls to \sphinxcode{\sphinxupquote{.keep\_plot()}} (and \sphinxcode{\sphinxupquote{.keep\_plot\_multi()}}) are executed on.  \sphinxcode{\sphinxupquote{.keep\_switch()}} can be passed a list of scenarios or using a wildcard selector.


\subsubsection{The \sphinxstyleliteralintitle{\sphinxupquote{.keep\_solutions.keys()}} method}
\label{\detokenize{content/05_WBModels/ScenarioAnalysis:the-keep-solutions-keys-method}}
\sphinxAtStartPar
The \sphinxcode{\sphinxupquote{.keep\_solutions.keys()}} method generates a list of the solutions that have been kept previously.

\begin{sphinxuseclass}{cell}\begin{sphinxVerbatimInput}

\begin{sphinxuseclass}{cell_input}
\begin{sphinxVerbatim}[commandchars=\\\{\}]
\PYG{n+nb}{print}\PYG{p}{(}\PYG{o}{*}\PYG{n}{mpak}\PYG{o}{.}\PYG{n}{keep\PYGZus{}solutions}\PYG{o}{.}\PYG{n}{keys}\PYG{p}{(}\PYG{p}{)}\PYG{p}{,}\PYG{n}{sep}\PYG{o}{=}\PYG{l+s+s1}{\PYGZsq{}}\PYG{l+s+se}{\PYGZbs{}n}\PYG{l+s+s1}{\PYGZsq{}}\PYG{p}{)}
\end{sphinxVerbatim}

\end{sphinxuseclass}\end{sphinxVerbatimInput}
\begin{sphinxVerbatimOutput}

\begin{sphinxuseclass}{cell_output}
\begin{sphinxVerbatim}[commandchars=\\\{\}]
Baseline
\PYGZdl{}25 increase in oil prices 2025\PYGZhy{}27
2.5\PYGZpc{} increase in C 2025\PYGZhy{}40
2.5\PYGZpc{} increase in C 2025\PYGZhy{}27 \PYGZhy{}\PYGZhy{} exog whole period
2.5\PYGZpc{} increase in C 2025\PYGZhy{}27 \PYGZhy{}\PYGZhy{} exog whole period \PYGZhy{}\PYGZhy{}KG=True
2.5\PYGZpc{} increase in C 2025\PYGZhy{}27 \PYGZhy{}\PYGZhy{} temporarily exogenized
1\PYGZpc{} of GDP increase in FDI and private investment (AF shock)
\end{sphinxVerbatim}

\end{sphinxuseclass}\end{sphinxVerbatimOutput}

\end{sphinxuseclass}
\sphinxAtStartPar
To specify exactly which scenarios to show in a keep\_plot, the \sphinxcode{\sphinxupquote{scenarios=}} option of \sphinxcode{\sphinxupquote{.keepswitch()}} must be initialized with a “|” delimited string of the names of the scenarios (retrieved above) that are to be displayed.

\sphinxAtStartPar
By placing the \sphinxcode{\sphinxupquote{.keepswitch()}} in a \sphinxcode{\sphinxupquote{with}} clause the scenario restriction will only apply to indented lines that follow the with construct.

\begin{sphinxuseclass}{cell}\begin{sphinxVerbatimInput}

\begin{sphinxuseclass}{cell_input}
\begin{sphinxVerbatim}[commandchars=\\\{\}]
\PYG{k}{with} \PYG{n}{mpak}\PYG{o}{.}\PYG{n}{keepswitch}\PYG{p}{(}\PYG{n}{scenarios}\PYG{o}{=}\PYG{l+s+s1}{\PYGZsq{}}\PYG{l+s+s1}{2.5}\PYG{l+s+si}{\PYGZpc{} i}\PYG{l+s+s1}{ncrease in C 2025\PYGZhy{}40|2.5}\PYG{l+s+si}{\PYGZpc{} i}\PYG{l+s+s1}{ncrease in C 2025\PYGZhy{}27 \PYGZhy{}\PYGZhy{} exog whole period|2.5}\PYG{l+s+si}{\PYGZpc{} i}\PYG{l+s+s1}{ncrease in C 2025\PYGZhy{}27 \PYGZhy{}\PYGZhy{} exog whole period \PYGZhy{}\PYGZhy{}KG=True|2.5}\PYG{l+s+si}{\PYGZpc{} i}\PYG{l+s+s1}{ncrease in C 2025\PYGZhy{}27 \PYGZhy{}\PYGZhy{} temporarily exogenized}\PYG{l+s+s1}{\PYGZsq{}}\PYG{p}{)}\PYG{p}{:}
    \PYG{n}{mpak}\PYG{o}{.}\PYG{n}{keep\PYGZus{}plot}\PYG{p}{(}\PYG{l+s+s1}{\PYGZsq{}}\PYG{l+s+s1}{PAKNYGDPMKTPKN PAKGGBALOVRLCN PAKGGDEBTTOTLCN}\PYG{l+s+s1}{\PYGZsq{}}\PYG{p}{,}\PYG{n}{diff}\PYG{o}{=}\PYG{k+kc}{False}\PYG{p}{,}\PYG{n}{showtype}\PYG{o}{=}\PYG{l+s+s1}{\PYGZsq{}}\PYG{l+s+s1}{growth}\PYG{l+s+s1}{\PYGZsq{}}\PYG{p}{,}\PYG{n}{legend}\PYG{o}{=}\PYG{k+kc}{True}\PYG{p}{)}\PYG{p}{;}
\end{sphinxVerbatim}

\end{sphinxuseclass}\end{sphinxVerbatimInput}
\begin{sphinxVerbatimOutput}

\begin{sphinxuseclass}{cell_output}
\noindent\sphinxincludegraphics{{4d4082f1eb35f31728b16dfc3ce40cde3588dab0fdb1f28afb94397881f0437f}.png}

\noindent\sphinxincludegraphics{{79c4faa9eebd093cebdf32aa5f4137a7d51775132845aedc327b97c04275222a}.png}

\end{sphinxuseclass}\end{sphinxVerbatimOutput}

\end{sphinxuseclass}

\subsubsection{Keepswitch with wildcard selection}
\label{\detokenize{content/05_WBModels/ScenarioAnalysis:keepswitch-with-wildcard-selection}}
\sphinxAtStartPar
Below we generate a series of plots using

\begin{sphinxuseclass}{cell}\begin{sphinxVerbatimInput}

\begin{sphinxuseclass}{cell_input}
\begin{sphinxVerbatim}[commandchars=\\\{\}]
\PYG{k}{with} \PYG{n}{mpak}\PYG{o}{.}\PYG{n}{keepswitch}\PYG{p}{(}\PYG{n}{scenarios}\PYG{o}{=}\PYG{l+s+s1}{\PYGZsq{}}\PYG{l+s+s1}{*2025*}\PYG{l+s+s1}{\PYGZsq{}}\PYG{p}{)}\PYG{p}{:}
    \PYG{n}{mpak}\PYG{o}{.}\PYG{n}{keep\PYGZus{}plot}\PYG{p}{(}\PYG{l+s+s1}{\PYGZsq{}}\PYG{l+s+s1}{PAKNYGDPMKTPKN PAKNECONPRVTKN}\PYG{l+s+s1}{\PYGZsq{}}\PYG{p}{,}\PYG{n}{showtype}\PYG{o}{=}\PYG{l+s+s1}{\PYGZsq{}}\PYG{l+s+s1}{growth}\PYG{l+s+s1}{\PYGZsq{}}\PYG{p}{,}\PYG{n}{legend}\PYG{o}{=}\PYG{k+kc}{True}\PYG{p}{)}\PYG{p}{;}
\end{sphinxVerbatim}

\end{sphinxuseclass}\end{sphinxVerbatimInput}
\begin{sphinxVerbatimOutput}

\begin{sphinxuseclass}{cell_output}
\noindent\sphinxincludegraphics{{7d1aa1ca7b80bc840d25f12a2679f4c190d82eecf72cf6f12a4d247df9830ea2}.png}

\noindent\sphinxincludegraphics{{53ecd99caf80c691a952bb6131692e98a3a3fe68aac682da277baf00fe32b45e}.png}

\end{sphinxuseclass}\end{sphinxVerbatimOutput}

\end{sphinxuseclass}

\section{The \sphinxstyleliteralintitle{\sphinxupquote{.keep\_plot\_multi()}} method}
\label{\detokenize{content/05_WBModels/ScenarioAnalysis:the-keep-plot-multi-method}}
\sphinxAtStartPar
The \sphinxcode{\sphinxupquote{.keep\_plot\_multi()}} method allows several charts to be displayed in a grid.  The size of each chart can be set with the \sphinxcode{\sphinxupquote{size=(w,h)}} option, where the units of with and height are in centimetres.

\begin{sphinxuseclass}{cell}\begin{sphinxVerbatimInput}

\begin{sphinxuseclass}{cell_input}
\begin{sphinxVerbatim}[commandchars=\\\{\}]
\PYG{k}{with} \PYG{n}{mpak}\PYG{o}{.}\PYG{n}{set\PYGZus{}smpl}\PYG{p}{(}\PYG{l+m+mi}{2000}\PYG{p}{,}\PYG{l+m+mi}{2040}\PYG{p}{)}\PYG{p}{:}
    \PYG{k}{with} \PYG{n}{mpak}\PYG{o}{.}\PYG{n}{keepswitch}\PYG{p}{(}\PYG{n}{scenarios}\PYG{o}{=}\PYG{l+s+s2}{\PYGZdq{}}\PYG{l+s+s2}{baseline *exog*}\PYG{l+s+s2}{\PYGZdq{}}\PYG{p}{)}\PYG{p}{:}
        \PYG{n}{var\PYGZus{}figs} \PYG{o}{=} \PYG{n}{mpak}\PYG{o}{.}\PYG{n}{keep\PYGZus{}plot\PYGZus{}multi}\PYG{p}{(}\PYG{l+s+s1}{\PYGZsq{}}\PYG{l+s+s1}{PAKNYGDPMKTPKN PAKNECONPRVTKN PAKNEGDIFTOTKN PAKNEIMPGNFSKN PAKNEEXPGNFSKN}\PYG{l+s+s1}{\PYGZsq{}}\PYG{p}{,}\PYG{l+m+mi}{2010}\PYG{p}{,}\PYG{l+m+mi}{2100}\PYG{p}{,}\PYG{n}{keep\PYGZus{}dim}\PYG{o}{=}\PYG{l+m+mi}{0}\PYG{p}{,}\PYG{n}{legend}\PYG{o}{=}\PYG{l+m+mi}{1}
                                \PYG{p}{,}\PYG{n}{size}\PYG{o}{=}\PYG{p}{(}\PYG{l+m+mi}{20}\PYG{p}{,}\PYG{l+m+mi}{20}\PYG{p}{)} \PYG{p}{,}\PYG{n}{diffpct}\PYG{o}{=}\PYG{k+kc}{True}\PYG{p}{,}\PYG{n}{title}\PYG{o}{=}\PYG{l+s+s1}{\PYGZsq{}}\PYG{l+s+s1}{\PYGZsq{}}  \PYG{p}{)}\PYG{p}{;}
\end{sphinxVerbatim}

\end{sphinxuseclass}\end{sphinxVerbatimInput}
\begin{sphinxVerbatimOutput}

\begin{sphinxuseclass}{cell_output}
\noindent\sphinxincludegraphics{{110b073013bb48ae907f8e787aab3de274c42258c568c7e059e8bf4220299939}.png}

\noindent\sphinxincludegraphics{{d0207b592c68e6eba3c886c8bf3e4c0e913066366b28e2733a47ee824b534250}.png}

\noindent\sphinxincludegraphics{{2aa184207903a01c4f7797817edf3b2310faa67a91efd0eba85f80edc6e9dad5}.png}

\end{sphinxuseclass}\end{sphinxVerbatimOutput}

\end{sphinxuseclass}
\sphinxAtStartPar
As indicated earlier both \sphinxcode{\sphinxupquote{keep\_plot()}} and \sphinxcode{\sphinxupquote{keep\_plot\_multi()}} return a variable that can be used to embellish or modify the figures produced by the automatic routines.

\sphinxAtStartPar
For example the charts can be resized.

\begin{sphinxuseclass}{cell}\begin{sphinxVerbatimInput}

\begin{sphinxuseclass}{cell_input}
\begin{sphinxVerbatim}[commandchars=\\\{\}]
\PYG{n}{var\PYGZus{}figs}\PYG{o}{.}\PYG{n}{set\PYGZus{}size\PYGZus{}inches}\PYG{p}{(}\PYG{l+m+mi}{15}\PYG{p}{,}\PYG{l+m+mi}{10}\PYG{p}{)}
\PYG{n}{var\PYGZus{}figs}
\end{sphinxVerbatim}

\end{sphinxuseclass}\end{sphinxVerbatimInput}
\begin{sphinxVerbatimOutput}

\begin{sphinxuseclass}{cell_output}
\noindent\sphinxincludegraphics{{0fde1c76130e7f66f8852126d948739ff4016752f6fcb6d6abe4b255a3d4e1b8}.png}

\end{sphinxuseclass}\end{sphinxVerbatimOutput}

\end{sphinxuseclass}
\sphinxAtStartPar
Individual charts can be deleted from the grid.

\begin{sphinxadmonition}{note}{Note:}
\sphinxAtStartPar
The grid representation of the individual charts is returned as a 0\sphinxhyphen{}based vector of charts.  Thus the first figure is the zeroeth and the second is the first.
\end{sphinxadmonition}


\subsection{Delete a chart from th grid}
\label{\detokenize{content/05_WBModels/ScenarioAnalysis:delete-a-chart-from-th-grid}}
\sphinxAtStartPar
A chart can be deleted from the grid by referencing it and calling the \sphinxcode{\sphinxupquote{.remove()}} method.

\begin{sphinxuseclass}{cell}\begin{sphinxVerbatimInput}

\begin{sphinxuseclass}{cell_input}
\begin{sphinxVerbatim}[commandchars=\\\{\}]
\PYG{n}{var\PYGZus{}figs}\PYG{o}{.}\PYG{n}{axes}\PYG{p}{[}\PYG{l+m+mi}{1}\PYG{p}{]}\PYG{o}{.}\PYG{n}{remove}\PYG{p}{(}\PYG{p}{)}
\PYG{n}{var\PYGZus{}figs}
\end{sphinxVerbatim}

\end{sphinxuseclass}\end{sphinxVerbatimInput}
\begin{sphinxVerbatimOutput}

\begin{sphinxuseclass}{cell_output}
\noindent\sphinxincludegraphics{{cecac040212a456116c95e7cd162e606f5d337a4f87674b97c1c1c58922d1234}.png}

\end{sphinxuseclass}\end{sphinxVerbatimOutput}

\end{sphinxuseclass}
\sphinxAtStartPar
The same mechanism can be used to revise the titles of the indidividual charts and annotate them.

\begin{sphinxuseclass}{cell}\begin{sphinxVerbatimInput}

\begin{sphinxuseclass}{cell_input}
\begin{sphinxVerbatim}[commandchars=\\\{\}]
\PYG{n}{var\PYGZus{}figs}\PYG{o}{.}\PYG{n}{axes}\PYG{p}{[}\PYG{l+m+mi}{0}\PYG{p}{]}\PYG{o}{.}\PYG{n}{set\PYGZus{}title}\PYG{p}{(}\PYG{l+s+s1}{\PYGZsq{}}\PYG{l+s+s1}{Impact of a 2.5}\PYG{l+s+si}{\PYGZpc{} i}\PYG{l+s+s1}{ncrease in C using exog method}\PYG{l+s+s1}{\PYGZsq{}}\PYG{p}{)}\PYG{p}{;}    \PYG{c+c1}{\PYGZsh{} many properties can be set afterward }
\PYG{n}{var\PYGZus{}figs}\PYG{o}{.}\PYG{n}{axes}\PYG{p}{[}\PYG{l+m+mi}{1}\PYG{p}{]}\PYG{o}{.}\PYG{n}{set\PYGZus{}title}\PYG{p}{(}\PYG{l+s+s1}{\PYGZsq{}}\PYG{l+s+s1}{Impact of a 2.5}\PYG{l+s+si}{\PYGZpc{} i}\PYG{l+s+s1}{ncrease in C using temporary exog method}\PYG{l+s+s1}{\PYGZsq{}}\PYG{p}{)}\PYG{p}{;}
\PYG{n}{var\PYGZus{}figs}
\end{sphinxVerbatim}

\end{sphinxuseclass}\end{sphinxVerbatimInput}
\begin{sphinxVerbatimOutput}

\begin{sphinxuseclass}{cell_output}
\noindent\sphinxincludegraphics{{da77be21155cd4f2f0166647e86314721d8a52769a63a2c2239a87fe8fcee618}.png}

\end{sphinxuseclass}\end{sphinxVerbatimOutput}

\end{sphinxuseclass}
\begin{sphinxuseclass}{cell}\begin{sphinxVerbatimInput}

\begin{sphinxuseclass}{cell_input}
\begin{sphinxVerbatim}[commandchars=\\\{\}]
\PYG{n}{var\PYGZus{}figs}\PYG{o}{.}\PYG{n}{axes}\PYG{p}{[}\PYG{l+m+mi}{0}\PYG{p}{]}\PYG{o}{.}\PYG{n}{set\PYGZus{}xlabel}\PYG{p}{(}\PYG{l+s+s1}{\PYGZsq{}}\PYG{l+s+s1}{Year}\PYG{l+s+s1}{\PYGZsq{}}\PYG{p}{)}
\PYG{n}{var\PYGZus{}figs}\PYG{o}{.}\PYG{n}{axes}\PYG{p}{[}\PYG{l+m+mi}{0}\PYG{p}{]}\PYG{o}{.}\PYG{n}{set\PYGZus{}ylabel}\PYG{p}{(}\PYG{l+s+s1}{\PYGZsq{}}\PYG{l+s+s1}{Change percent}\PYG{l+s+se}{\PYGZbs{}n}\PYG{l+s+s1}{change in level}\PYG{l+s+s1}{\PYGZsq{}}\PYG{p}{,}\PYG{n}{fontsize}\PYG{o}{=}\PYG{l+m+mi}{10}\PYG{p}{)}
\PYG{n}{var\PYGZus{}figs}\PYG{o}{.}\PYG{n}{axes}\PYG{p}{[}\PYG{l+m+mi}{0}\PYG{p}{]}\PYG{o}{.}\PYG{n}{yaxis}\PYG{o}{.}\PYG{n}{set\PYGZus{}label\PYGZus{}coords}\PYG{p}{(}\PYG{o}{\PYGZhy{}}\PYG{l+m+mf}{0.1}\PYG{p}{,}\PYG{l+m+mf}{1.02}\PYG{p}{)}
\PYG{n}{var\PYGZus{}figs}\PYG{o}{.}\PYG{n}{axes}\PYG{p}{[}\PYG{l+m+mi}{0}\PYG{p}{]}\PYG{o}{.}\PYG{n}{xaxis}\PYG{o}{.}\PYG{n}{set\PYGZus{}label\PYGZus{}coords}\PYG{p}{(}\PYG{l+m+mf}{.95}\PYG{p}{,}\PYG{o}{\PYGZhy{}}\PYG{l+m+mf}{.06}\PYG{p}{)}
\PYG{n}{var\PYGZus{}figs}
\end{sphinxVerbatim}

\end{sphinxuseclass}\end{sphinxVerbatimInput}
\begin{sphinxVerbatimOutput}

\begin{sphinxuseclass}{cell_output}
\noindent\sphinxincludegraphics{{8854e8c856d491acf263a49b3187e73e76e83c41f1c7a15802bec25c688ffb45}.png}

\end{sphinxuseclass}\end{sphinxVerbatimOutput}

\end{sphinxuseclass}
\sphinxAtStartPar
Variables pointing to the individual charts can be defined and used to make modifications to individual charts within the overall figure.

\begin{sphinxuseclass}{cell}\begin{sphinxVerbatimInput}

\begin{sphinxuseclass}{cell_input}
\begin{sphinxVerbatim}[commandchars=\\\{\}]
\PYG{n}{fig1}\PYG{o}{=}\PYG{n}{var\PYGZus{}figs}\PYG{o}{.}\PYG{n}{axes}\PYG{p}{[}\PYG{l+m+mi}{0}\PYG{p}{]}
\PYG{n}{fig2}\PYG{o}{=}\PYG{n}{var\PYGZus{}figs}\PYG{o}{.}\PYG{n}{axes}\PYG{p}{[}\PYG{l+m+mi}{1}\PYG{p}{]}
\end{sphinxVerbatim}

\end{sphinxuseclass}\end{sphinxVerbatimInput}

\end{sphinxuseclass}
\begin{sphinxuseclass}{cell}\begin{sphinxVerbatimInput}

\begin{sphinxuseclass}{cell_input}
\begin{sphinxVerbatim}[commandchars=\\\{\}]
\PYG{n}{fig2}\PYG{o}{.}\PYG{n}{set\PYGZus{}ylabel}\PYG{p}{(}\PYG{l+s+s1}{\PYGZsq{}}\PYG{l+s+s1}{Percent change}\PYG{l+s+se}{\PYGZbs{}n}\PYG{l+s+s1}{in level}\PYG{l+s+s1}{\PYGZsq{}}\PYG{p}{,}\PYG{n}{fontsize}\PYG{o}{=}\PYG{l+m+mi}{10}\PYG{p}{)}
\PYG{n}{fig2}\PYG{o}{.}\PYG{n}{yaxis}\PYG{o}{.}\PYG{n}{set\PYGZus{}label\PYGZus{}coords}\PYG{p}{(}\PYG{o}{\PYGZhy{}}\PYG{l+m+mf}{0.1}\PYG{p}{,}\PYG{l+m+mf}{1.02}\PYG{p}{)} \PYG{c+c1}{\PYGZsh{}place axes labels}
\PYG{n}{fig2}\PYG{o}{.}\PYG{n}{xaxis}\PYG{o}{.}\PYG{n}{set\PYGZus{}label\PYGZus{}coords}\PYG{p}{(}\PYG{l+m+mf}{.95}\PYG{p}{,}\PYG{o}{\PYGZhy{}}\PYG{l+m+mf}{.06}\PYG{p}{)}



\PYG{n}{fig2}\PYG{o}{.}\PYG{n}{text}\PYG{p}{(}\PYG{l+m+mf}{2040.}\PYG{p}{,}\PYG{l+m+mf}{0.4}\PYG{p}{,} \PYG{l+s+s1}{\PYGZsq{}}\PYG{l+s+s1}{Some text in a box}\PYG{l+s+s1}{\PYGZsq{}}\PYG{p}{,} 
          \PYG{n}{color}\PYG{o}{=}\PYG{l+s+s1}{\PYGZsq{}}\PYG{l+s+s1}{yellow}\PYG{l+s+s1}{\PYGZsq{}}\PYG{p}{,}\PYG{n}{bbox}\PYG{o}{=}\PYG{n+nb}{dict}\PYG{p}{(}\PYG{n}{facecolor}\PYG{o}{=}\PYG{l+s+s1}{\PYGZsq{}}\PYG{l+s+s1}{red}\PYG{l+s+s1}{\PYGZsq{}}\PYG{p}{,} \PYG{n}{alpha}\PYG{o}{=}\PYG{l+m+mf}{0.5}\PYG{p}{)}\PYG{p}{)}\PYG{p}{;}
\PYG{n}{fig2}\PYG{o}{.}\PYG{n}{text}\PYG{p}{(}\PYG{l+m+mf}{2040.}\PYG{p}{,}\PYG{l+m+mf}{0.1}\PYG{p}{,} \PYG{l+s+s1}{\PYGZsq{}}\PYG{l+s+s1}{Some nice text}\PYG{l+s+s1}{\PYGZsq{}}\PYG{p}{,} 
          \PYG{n}{style}\PYG{o}{=}\PYG{l+s+s1}{\PYGZsq{}}\PYG{l+s+s1}{italic}\PYG{l+s+s1}{\PYGZsq{}}\PYG{p}{,}\PYG{n}{color}\PYG{o}{=}\PYG{l+s+s1}{\PYGZsq{}}\PYG{l+s+s1}{green}\PYG{l+s+s1}{\PYGZsq{}}\PYG{p}{)}\PYG{p}{;}
          
\PYG{n}{var\PYGZus{}figs}
          
\end{sphinxVerbatim}

\end{sphinxuseclass}\end{sphinxVerbatimInput}
\begin{sphinxVerbatimOutput}

\begin{sphinxuseclass}{cell_output}
\noindent\sphinxincludegraphics{{6731336bfbd25841772227371b0c5d2df598a1407e399fd694bdf65d4e024f82}.png}

\end{sphinxuseclass}\end{sphinxVerbatimOutput}

\end{sphinxuseclass}

\subsection{Alternative one\sphinxhyphen{}stop shop. The results visualization widget view}
\label{\detokenize{content/05_WBModels/ScenarioAnalysis:alternative-one-stop-shop-the-results-visualization-widget-view}}
\sphinxAtStartPar
When working in Jupyter Notebook, referencing a selection of series will cause a data visualization widget to be generated that allows you to look at results (\sphinxcode{\sphinxupquote{basesdf}} vs \sphinxcode{\sphinxupquote{latestdf}}) for the selected variables as tables or charts, as levels, as growth rates and as percent differences from baseline.

\begin{sphinxuseclass}{cell}\begin{sphinxVerbatimInput}

\begin{sphinxuseclass}{cell_input}
\begin{sphinxVerbatim}[commandchars=\\\{\}]
\PYG{n}{mpak}\PYG{p}{[}\PYG{l+s+s1}{\PYGZsq{}}\PYG{l+s+s1}{PAKNYGDPMKTPCN PAKNYGDPMKTPKN PAKGGEXPTOTLCN PAKGGREVTOTLCN PAKNECONGOVTKN}\PYG{l+s+s1}{\PYGZsq{}}\PYG{p}{]}
\end{sphinxVerbatim}

\end{sphinxuseclass}\end{sphinxVerbatimInput}
\begin{sphinxVerbatimOutput}

\begin{sphinxuseclass}{cell_output}
\begin{sphinxVerbatim}[commandchars=\\\{\}]
Tab(children=(Tab(children=(HTML(value=\PYGZsq{}\PYGZlt{}?xml version=\PYGZdq{}1.0\PYGZdq{} encoding=\PYGZdq{}utf\PYGZhy{}8\PYGZdq{} standalone=\PYGZdq{}no\PYGZdq{}?\PYGZgt{}\PYGZbs{}n\PYGZlt{}!DOCTYPE svg …
\end{sphinxVerbatim}

\begin{sphinxVerbatim}[commandchars=\\\{\}]

\end{sphinxVerbatim}

\end{sphinxuseclass}\end{sphinxVerbatimOutput}

\end{sphinxuseclass}
\sphinxstepscope

\begin{sphinxuseclass}{cell}\begin{sphinxVerbatimInput}

\begin{sphinxuseclass}{cell_input}
\begin{sphinxVerbatim}[commandchars=\\\{\}]
\PYG{o}{\PYGZpc{}}\PYG{k}{matplotlib} inline
\end{sphinxVerbatim}

\end{sphinxuseclass}\end{sphinxVerbatimInput}

\end{sphinxuseclass}

\chapter{More complex scenarios}
\label{\detokenize{content/05_WBModels/MoreComplexScenarios:more-complex-scenarios}}\label{\detokenize{content/05_WBModels/MoreComplexScenarios::doc}}
\sphinxAtStartPar
The preceding chapter introduced four different ways of preparing a solution and the forms the backbone of running simulations on World Bank models in \sphinxcode{\sphinxupquote{modelflow}}. This chapter builds on those examples and delves into some of the challenges involved in translating a real\sphinxhyphen{}world policy challenge into the model\sphinxhyphen{}world and then back again.

\sphinxAtStartPar
The particular problem to be examined is in the introduction of a Carbon Tax. The model used and the example presented are both taken from the model of Pakistan presented here: .


\section{Setting up the environment}
\label{\detokenize{content/05_WBModels/MoreComplexScenarios:setting-up-the-environment}}
\sphinxAtStartPar
As always the modelflow and other python libraries that are to be used must be imported into the current session.

\begin{sphinxuseclass}{cell}\begin{sphinxVerbatimInput}

\begin{sphinxuseclass}{cell_input}
\begin{sphinxVerbatim}[commandchars=\\\{\}]
\PYG{k+kn}{from} \PYG{n+nn}{modelclass} \PYG{k+kn}{import} \PYG{n}{model} 
\PYG{n}{model}\PYG{o}{.}\PYG{n}{widescreen}\PYG{p}{(}\PYG{p}{)}
\PYG{n}{model}\PYG{o}{.}\PYG{n}{scroll\PYGZus{}off}\PYG{p}{(}\PYG{p}{)}
\end{sphinxVerbatim}

\end{sphinxuseclass}\end{sphinxVerbatimInput}
\begin{sphinxVerbatimOutput}

\begin{sphinxuseclass}{cell_output}
\begin{sphinxVerbatim}[commandchars=\\\{\}]
\PYGZlt{}IPython.core.display.HTML object\PYGZgt{}
\end{sphinxVerbatim}

\end{sphinxuseclass}\end{sphinxVerbatimOutput}

\end{sphinxuseclass}
\begin{sphinxuseclass}{cell}\begin{sphinxVerbatimInput}

\begin{sphinxuseclass}{cell_input}
\begin{sphinxVerbatim}[commandchars=\\\{\}]
\PYG{n}{mpak}\PYG{p}{,}\PYG{n}{baseline} \PYG{o}{=} \PYG{n}{model}\PYG{o}{.}\PYG{n}{modelload}\PYG{p}{(}\PYG{l+s+s1}{\PYGZsq{}}\PYG{l+s+s1}{../models/pak.pcim}\PYG{l+s+s1}{\PYGZsq{}}\PYG{p}{,}\PYG{n}{alfa}\PYG{o}{=}\PYG{l+m+mf}{0.7}\PYG{p}{,}\PYG{n}{run}\PYG{o}{=}\PYG{l+m+mi}{1}\PYG{p}{,}\PYG{n}{keep}\PYG{o}{=}\PYG{l+s+s2}{\PYGZdq{}}\PYG{l+s+s2}{Baseline}\PYG{l+s+s2}{\PYGZdq{}}\PYG{p}{)}
\end{sphinxVerbatim}

\end{sphinxuseclass}\end{sphinxVerbatimInput}
\begin{sphinxVerbatimOutput}

\begin{sphinxuseclass}{cell_output}
\begin{sphinxVerbatim}[commandchars=\\\{\}]
file read:  C:\PYGZbs{}modelflow manual\PYGZbs{}papers\PYGZbs{}mfbook\PYGZbs{}content\PYGZbs{}models\PYGZbs{}pak.pcim
\end{sphinxVerbatim}

\end{sphinxuseclass}\end{sphinxVerbatimOutput}

\end{sphinxuseclass}

\section{The policy problem}
\label{\detokenize{content/05_WBModels/MoreComplexScenarios:the-policy-problem}}
\sphinxAtStartPar
The objective of this chapter is to produce a simulation of the economic and climate effects of the introduction of a carbon tax in Pakistan.

\sphinxAtStartPar
The variable \sphinxcode{\sphinxupquote{mpak}} loaded above contains the model instance, the variables, equations and the data for the model.  On load the model was solved, and the results of that initiaql solve was assigned to the \sphinxcode{\sphinxupquote{DataFrame}} \sphinxcode{\sphinxupquote{baseline}}.

\sphinxAtStartPar
The Pakistan model contains three carbon tax variables:


\begin{savenotes}\sphinxattablestart
\centering
\begin{tabulary}{\linewidth}[t]{|T|T|}
\hline
\sphinxstyletheadfamily 
\sphinxAtStartPar
Mnemonic
&\sphinxstyletheadfamily 
\sphinxAtStartPar
Meaning
\\
\hline
\sphinxAtStartPar
PAKGGREVCO2CER
&
\sphinxAtStartPar
The effective carbon tax rate on Coal
\\
\hline
\sphinxAtStartPar
PAKGGREVCO2GER
&
\sphinxAtStartPar
The effective carbon tax rate on Gas
\\
\hline
\sphinxAtStartPar
PAKGGREVCO2OER
&
\sphinxAtStartPar
The effective carbon tax rate on Crude Oil
\\
\hline
\end{tabulary}
\par
\sphinxattableend\end{savenotes}

\sphinxAtStartPar
As discussed in earlier chapters the meaning of the mnemonics can be retrieved from the model:

\begin{sphinxuseclass}{cell}\begin{sphinxVerbatimInput}

\begin{sphinxuseclass}{cell_input}
\begin{sphinxVerbatim}[commandchars=\\\{\}]
\PYG{n}{mpak}\PYG{p}{[}\PYG{l+s+s1}{\PYGZsq{}}\PYG{l+s+s1}{PAKGGREVCO2*ER}\PYG{l+s+s1}{\PYGZsq{}}\PYG{p}{]}\PYG{o}{.}\PYG{n}{des}
    
\end{sphinxVerbatim}

\end{sphinxuseclass}\end{sphinxVerbatimInput}
\begin{sphinxVerbatimOutput}

\begin{sphinxuseclass}{cell_output}
\begin{sphinxVerbatim}[commandchars=\\\{\}]
PAKGGREVCO2CER : Carbon tax on coal (USD/t)
PAKGGREVCO2GER : Carbon tax on gas (USD/t)
PAKGGREVCO2OER : Carbon tax on oil (USD/t)
\end{sphinxVerbatim}

\end{sphinxuseclass}\end{sphinxVerbatimOutput}

\end{sphinxuseclass}
\sphinxAtStartPar
Alternatively one can search on the variable descriptions to retrieve the mnemonics of variables. Below the exclamation sign at the beginning of the string notifies the matching algorithm to search the variables descriptions (not he mnemonics) and return all variables that match.

\begin{sphinxuseclass}{cell}\begin{sphinxVerbatimInput}

\begin{sphinxuseclass}{cell_input}
\begin{sphinxVerbatim}[commandchars=\\\{\}]
\PYG{n}{mpak}\PYG{p}{[}\PYG{l+s+s1}{\PYGZsq{}}\PYG{l+s+s1}{!*Carbon*}\PYG{l+s+s1}{\PYGZsq{}}\PYG{p}{]}\PYG{o}{.}\PYG{n}{des}
\end{sphinxVerbatim}

\end{sphinxuseclass}\end{sphinxVerbatimInput}
\begin{sphinxVerbatimOutput}

\begin{sphinxuseclass}{cell_output}
\begin{sphinxVerbatim}[commandchars=\\\{\}]
PAKGGREVCO2CER : Carbon tax on coal (USD/t)
PAKGGREVCO2GER : Carbon tax on gas (USD/t)
PAKGGREVCO2OER : Carbon tax on oil (USD/t)
\end{sphinxVerbatim}

\end{sphinxuseclass}\end{sphinxVerbatimOutput}

\end{sphinxuseclass}

\section{Add variable descriptions}
\label{\detokenize{content/05_WBModels/MoreComplexScenarios:add-variable-descriptions}}
\sphinxAtStartPar
A modelflow model imported from EViews will inherit the variable descriptors coming from Eviews.  The variable descriptions are stored in a dictionary \sphinxcode{\sphinxupquote{mpak.var\_description}}. Not all EViews variables will necessarily have a description so such descriptions can be added to the existing by joining the \sphinxcode{\sphinxupquote{mpak.var\_description}} with a dictionary of additional variable description. This is done like this:

\sphinxAtStartPar
\sphinxcode{\sphinxupquote{mpak.var\_description = \{**mpak.var\_description,**extra\_description\}}}

\sphinxAtStartPar
adds the extra\_description dictionary to the pre\sphinxhyphen{}existing mpak,var\_description dictionary.

\sphinxAtStartPar
Several \sphinxcode{\sphinxupquote{modelflow}} methods include a \sphinxcode{\sphinxupquote{rename}} option, which if set to True will substitute the description for the variable name in any outputs.  Variables can also be selected for by using the \sphinxcode{\sphinxupquote{mpak{[}'!*subtext*'{]}}} syntax, where subtext is some text that appears in the variable descriptor.

\begin{sphinxuseclass}{cell}\begin{sphinxVerbatimInput}

\begin{sphinxuseclass}{cell_input}
\begin{sphinxVerbatim}[commandchars=\\\{\}]
\PYG{n}{extra\PYGZus{}description} \PYG{o}{=} \PYG{p}{\PYGZob{}}\PYG{l+s+s1}{\PYGZsq{}}\PYG{l+s+s1}{PAKNYGDPMKTPKN}\PYG{l+s+s1}{\PYGZsq{}}\PYG{p}{:} \PYG{l+s+s1}{\PYGZsq{}}\PYG{l+s+s1}{GDP}\PYG{l+s+s1}{\PYGZsq{}}\PYG{p}{,}
\PYG{l+s+s1}{\PYGZsq{}}\PYG{l+s+s1}{EMISCOAL}\PYG{l+s+s1}{\PYGZsq{}}        \PYG{p}{:} \PYG{l+s+s1}{\PYGZsq{}}\PYG{l+s+s1}{Coal emissions}\PYG{l+s+s1}{\PYGZsq{}}\PYG{p}{,}
\PYG{l+s+s1}{\PYGZsq{}}\PYG{l+s+s1}{EMISGAS}\PYG{l+s+s1}{\PYGZsq{}}         \PYG{p}{:} \PYG{l+s+s1}{\PYGZsq{}}\PYG{l+s+s1}{Gas Emissions}\PYG{l+s+s1}{\PYGZsq{}}\PYG{p}{,}
\PYG{l+s+s1}{\PYGZsq{}}\PYG{l+s+s1}{EMISOIL}\PYG{l+s+s1}{\PYGZsq{}}         \PYG{p}{:} \PYG{l+s+s1}{\PYGZsq{}}\PYG{l+s+s1}{Gas Emissions}\PYG{l+s+s1}{\PYGZsq{}}\PYG{p}{,}
\PYG{l+s+s1}{\PYGZsq{}}\PYG{l+s+s1}{PAKCCEMISCO2CKN}\PYG{l+s+s1}{\PYGZsq{}} \PYG{p}{:} \PYG{l+s+s1}{\PYGZsq{}}\PYG{l+s+s1}{Coal emissions, tCO2e}\PYG{l+s+s1}{\PYGZsq{}}\PYG{p}{,}
\PYG{l+s+s1}{\PYGZsq{}}\PYG{l+s+s1}{PAKCCEMISCO2GKN}\PYG{l+s+s1}{\PYGZsq{}} \PYG{p}{:} \PYG{l+s+s1}{\PYGZsq{}}\PYG{l+s+s1}{Natural Gas emissions, tCO2e}\PYG{l+s+s1}{\PYGZsq{}}\PYG{p}{,}
\PYG{l+s+s1}{\PYGZsq{}}\PYG{l+s+s1}{PAKCCEMISCO2OKN}\PYG{l+s+s1}{\PYGZsq{}} \PYG{p}{:} \PYG{l+s+s1}{\PYGZsq{}}\PYG{l+s+s1}{Crude Oil emissions, tCO2e}\PYG{l+s+s1}{\PYGZsq{}}\PYG{p}{,}
\PYG{l+s+s1}{\PYGZsq{}}\PYG{l+s+s1}{PAKCCEMISCO2TKN}\PYG{l+s+s1}{\PYGZsq{}} \PYG{p}{:} \PYG{l+s+s1}{\PYGZsq{}}\PYG{l+s+s1}{Total emissions, tCO2e}\PYG{l+s+s1}{\PYGZsq{}}\PYG{p}{,}
\PYG{l+s+s1}{\PYGZsq{}}\PYG{l+s+s1}{PAKGGREVEMISCN}\PYG{l+s+s1}{\PYGZsq{}}  \PYG{p}{:} \PYG{l+s+s1}{\PYGZsq{}}\PYG{l+s+s1}{Revenue from emissions taxes}\PYG{l+s+s1}{\PYGZsq{}}\PYG{p}{,}
 \PYG{l+s+s1}{\PYGZsq{}}\PYG{l+s+s1}{PAKLMUNRTOTLCN}\PYG{l+s+s1}{\PYGZsq{}}\PYG{p}{:} \PYG{l+s+s1}{\PYGZsq{}}\PYG{l+s+s1}{Unemployment rate}\PYG{l+s+s1}{\PYGZsq{}}\PYG{p}{,}
 \PYG{l+s+s1}{\PYGZsq{}}\PYG{l+s+s1}{PAKGGDBTTOTLCN\PYGZus{}}\PYG{l+s+s1}{\PYGZsq{}}\PYG{p}{:} \PYG{l+s+s1}{\PYGZsq{}}\PYG{l+s+s1}{Debt (}\PYG{l+s+si}{\PYGZpc{}G}\PYG{l+s+s1}{DP)}\PYG{l+s+s1}{\PYGZsq{}}\PYG{p}{,}
 \PYG{l+s+s1}{\PYGZsq{}}\PYG{l+s+s1}{PAKGGREVTOTLCN}\PYG{l+s+s1}{\PYGZsq{}}\PYG{p}{:} \PYG{l+s+s1}{\PYGZsq{}}\PYG{l+s+s1}{Fiscal revenues}\PYG{l+s+s1}{\PYGZsq{}}\PYG{p}{,}
 \PYG{l+s+s1}{\PYGZsq{}}\PYG{l+s+s1}{PAKWDL}\PYG{l+s+s1}{\PYGZsq{}}\PYG{p}{:} \PYG{l+s+s1}{\PYGZsq{}}\PYG{l+s+s1}{Working days lost due to pollution}\PYG{l+s+s1}{\PYGZsq{}}\PYG{p}{\PYGZcb{}}


\PYG{n}{mpak}\PYG{o}{.}\PYG{n}{var\PYGZus{}description} \PYG{o}{=} \PYG{p}{\PYGZob{}}\PYG{o}{*}\PYG{o}{*}\PYG{n}{mpak}\PYG{o}{.}\PYG{n}{var\PYGZus{}description}\PYG{p}{,}\PYG{o}{*}\PYG{o}{*}\PYG{n}{extra\PYGZus{}description}\PYG{p}{\PYGZcb{}}
\end{sphinxVerbatim}

\end{sphinxuseclass}\end{sphinxVerbatimInput}

\end{sphinxuseclass}

\section{Simulating the impact of a imposing a carbon price}
\label{\detokenize{content/05_WBModels/MoreComplexScenarios:simulating-the-impact-of-a-imposing-a-carbon-price}}
\sphinxAtStartPar
To run a simulation, the following steps must invariably be followed.
\begin{enumerate}
\sphinxsetlistlabels{\arabic}{enumi}{enumii}{}{.}%
\item {} 
\sphinxAtStartPar
Create a new DataFrame, typically a copy of an existing one.

\item {} 
\sphinxAtStartPar
Change the value  in the new df of the variable(s) to be shocked.

\item {} 
\sphinxAtStartPar
Solve the model using the newly altered df as the input df.

\end{enumerate}

\begin{sphinxuseclass}{cell}\begin{sphinxVerbatimInput}

\begin{sphinxuseclass}{cell_input}
\begin{sphinxVerbatim}[commandchars=\\\{\}]
\PYG{c+c1}{\PYGZsh{} Create copy of the baseline df}
\PYG{n}{alternative\PYGZus{}df} \PYG{o}{=} \PYG{n}{baseline}\PYG{o}{.}\PYG{n}{copy}\PYG{p}{(}\PYG{p}{)}
\PYG{c+c1}{\PYGZsh{}set the effective carbon tax of all three carbon tax variables equal to 30 USD}
\PYG{n}{alternative\PYGZus{}df}\PYG{o}{.}\PYG{n}{loc}\PYG{p}{[}\PYG{l+m+mi}{2025}\PYG{p}{:}\PYG{l+m+mi}{2100}\PYG{p}{,}\PYG{p}{[}\PYG{l+s+s1}{\PYGZsq{}}\PYG{l+s+s1}{PAKGGREVCO2CER}\PYG{l+s+s1}{\PYGZsq{}}\PYG{p}{,}\PYG{l+s+s1}{\PYGZsq{}}\PYG{l+s+s1}{PAKGGREVCO2GER}\PYG{l+s+s1}{\PYGZsq{}}\PYG{p}{,} \PYG{l+s+s1}{\PYGZsq{}}\PYG{l+s+s1}{PAKGGREVCO2OER}\PYG{l+s+s1}{\PYGZsq{}}\PYG{p}{]}\PYG{p}{]} \PYG{o}{=} \PYG{l+m+mi}{30} 
\end{sphinxVerbatim}

\end{sphinxuseclass}\end{sphinxVerbatimInput}

\end{sphinxuseclass}
\sphinxAtStartPar
The above used the \sphinxcode{\sphinxupquote{pandas}} function \sphinxcode{\sphinxupquote{.loc{[}{]}}} to change the Carbon Tax rate variables.

\sphinxAtStartPar
The \sphinxcode{\sphinxupquote{modelflow}} method \sphinxcode{\sphinxupquote{.upd()}} could be used to perform the same change.

\begin{sphinxuseclass}{cell}\begin{sphinxVerbatimInput}

\begin{sphinxuseclass}{cell_input}
\begin{sphinxVerbatim}[commandchars=\\\{\}]
\PYG{c+c1}{\PYGZsh{} This modelflow command is equivalent to the previous standard pandas command abive that used the .loc[] syntax}
\PYG{n}{CT30df}  \PYG{o}{=}  \PYG{n}{baseline}\PYG{o}{.}\PYG{n}{upd}\PYG{p}{(}\PYG{l+s+s2}{\PYGZdq{}}\PYG{l+s+s2}{\PYGZlt{}2025 2100\PYGZgt{} PAKGGREVCO2CER PAKGGREVCO2GER PAKGGREVCO2OER = 30}\PYG{l+s+s2}{\PYGZdq{}}\PYG{p}{)}
\end{sphinxVerbatim}

\end{sphinxuseclass}\end{sphinxVerbatimInput}

\end{sphinxuseclass}

\subsection{Solve the model}
\label{\detokenize{content/05_WBModels/MoreComplexScenarios:solve-the-model}}
\sphinxAtStartPar
Solving the model is as simple as calling the mpak function with the altered \sphinxcode{\sphinxupquote{DataFrame}} and assigning the results to dataframe (\sphinxcode{\sphinxupquote{resultsdf}} in this instance).  The \sphinxcode{\sphinxupquote{keep}} option causes a copy of the dataframe to be stores within the mpak model obkect.

\begin{sphinxuseclass}{cell}\begin{sphinxVerbatimInput}

\begin{sphinxuseclass}{cell_input}
\begin{sphinxVerbatim}[commandchars=\\\{\}]
\PYG{n}{resultsdf} \PYG{o}{=} \PYG{n}{mpak}\PYG{p}{(}\PYG{n}{CT30df}\PYG{p}{,}\PYG{l+m+mi}{2020}\PYG{p}{,}\PYG{l+m+mi}{2100}\PYG{p}{,}\PYG{n}{keep}\PYG{o}{=}\PYG{l+s+s2}{\PYGZdq{}}\PYG{l+s+s2}{Nominal \PYGZdl{}30USD Carbon tax}\PYG{l+s+s2}{\PYGZdq{}}\PYG{p}{)} \PYG{c+c1}{\PYGZsh{} simulates the model }
\end{sphinxVerbatim}

\end{sphinxuseclass}\end{sphinxVerbatimInput}

\end{sphinxuseclass}

\subsubsection{Examining the results}
\label{\detokenize{content/05_WBModels/MoreComplexScenarios:examining-the-results}}
\sphinxAtStartPar
Every time the model is solved the results of the simulation are assigned to a variable on the left hands side of the solve call (\sphinxcode{\sphinxupquote{resultdf}} in the example above.  The results of the most recent scenario are also always stored in the \sphinxcode{\sphinxupquote{.lastdf}} \sphinxcode{\sphinxupquote{DataFrame}} that is part of the properties of any \sphinxcode{\sphinxupquote{modelflow}} model. the \sphinxcode{\sphinxupquote{basedf}} property of \sphinxcode{\sphinxupquote{mpak}} (an instantiation of a modelflow model object) contains a copy of the initial DataFrame from which the model was built.

\sphinxAtStartPar
The \sphinxcode{\sphinxupquote{DataFrames}} \sphinxcode{\sphinxupquote{Baseline}} and \sphinxcode{\sphinxupquote{Resultsdf}} were created by us when we solved the model (initially on load) and now with the simulation.  Currently their contents are the same as, but separate from the contents of \sphinxcode{\sphinxupquote{basedf}} and \sphinxcode{\sphinxupquote{lastdf}}.

\begin{sphinxadmonition}{note}{Note:}
\sphinxAtStartPar
The standard dataframes are part of the \sphinxcode{\sphinxupquote{modelflow}} object and managed by it.
\begin{itemize}
\item {} 
\sphinxAtStartPar
\sphinxstylestrong{mpak.basedf}: Dataframe with the values for baseline

\item {} 
\sphinxAtStartPar
\sphinxstylestrong{mpak.lastdf}: Dataframe with the values from the most recent simulation

\end{itemize}
\end{sphinxadmonition}

\sphinxAtStartPar
The impact of the imposition of the carbon tax in the model is relatively quick, resulting in an overall decline in emissions of 21.8\% in the first year, with coal emissions recording the biggest hit at \sphinxhyphen{}40.5 percent.

\sphinxAtStartPar
Abstracting from the fact that the impact is occurring too quickly (it would take time for the substitution towards alternative sources of power to occur), the fact that impacts are fading with time suggests an error in the specification of the shock.  High domestic inflation means that the relative price change of a given Carbon price is declining over time.

\begin{sphinxuseclass}{cell}\begin{sphinxVerbatimInput}

\begin{sphinxuseclass}{cell_input}
\begin{sphinxVerbatim}[commandchars=\\\{\}]
\PYG{k}{with} \PYG{n}{mpak}\PYG{o}{.}\PYG{n}{set\PYGZus{}smpl}\PYG{p}{(}\PYG{l+m+mi}{2023}\PYG{p}{,}\PYG{l+m+mi}{2030}\PYG{p}{)}\PYG{p}{:}
    \PYG{n+nb}{print}\PYG{p}{(}\PYG{n+nb}{round}\PYG{p}{(}\PYG{n}{mpak}\PYG{p}{[}\PYG{l+s+s1}{\PYGZsq{}}\PYG{l+s+s1}{PAKCCEMISCO2*}\PYG{l+s+s1}{\PYGZsq{}}\PYG{p}{]}\PYG{o}{.}\PYG{n}{difpctlevel}\PYG{o}{.}\PYG{n}{mul100}\PYG{o}{.}\PYG{n}{df}\PYG{p}{,}\PYG{l+m+mi}{2}\PYG{p}{)}\PYG{p}{)}\PYG{p}{;}
\end{sphinxVerbatim}

\end{sphinxuseclass}\end{sphinxVerbatimInput}
\begin{sphinxVerbatimOutput}

\begin{sphinxuseclass}{cell_output}
\begin{sphinxVerbatim}[commandchars=\\\{\}]
      PAKCCEMISCO2CKN  PAKCCEMISCO2GKN  PAKCCEMISCO2OKN  PAKCCEMISCO2TKN
2023             0.00             0.00             0.00             0.00
2024             0.00             0.00             0.00             0.00
2025           \PYGZhy{}41.19           \PYGZhy{}26.99           \PYGZhy{}10.93           \PYGZhy{}22.17
2026           \PYGZhy{}40.06           \PYGZhy{}25.72           \PYGZhy{}10.98           \PYGZhy{}21.56
2027           \PYGZhy{}38.85           \PYGZhy{}24.48           \PYGZhy{}10.89           \PYGZhy{}20.89
2028           \PYGZhy{}37.59           \PYGZhy{}23.30           \PYGZhy{}10.68           \PYGZhy{}20.17
2029           \PYGZhy{}36.26           \PYGZhy{}22.14           \PYGZhy{}10.35           \PYGZhy{}19.38
2030           \PYGZhy{}34.89           \PYGZhy{}21.01            \PYGZhy{}9.95           \PYGZhy{}18.55
\end{sphinxVerbatim}

\end{sphinxuseclass}\end{sphinxVerbatimOutput}

\end{sphinxuseclass}
\begin{sphinxuseclass}{cell}\begin{sphinxVerbatimInput}

\begin{sphinxuseclass}{cell_input}
\begin{sphinxVerbatim}[commandchars=\\\{\}]
\PYG{n}{mpak}\PYG{p}{[}\PYG{l+s+s1}{\PYGZsq{}}\PYG{l+s+s1}{PAKCCEMISCO2?KN}\PYG{l+s+s1}{\PYGZsq{}}\PYG{p}{]}\PYG{o}{.}\PYG{n}{difpctlevel}\PYG{o}{.}\PYG{n}{mul100}\PYG{o}{.}\PYG{n}{plot}\PYG{p}{(}\PYG{n}{title}\PYG{o}{=}\PYG{l+s+s2}{\PYGZdq{}}\PYG{l+s+s2}{Emissions impact of a \PYGZdl{}30 USD Carbon tax}\PYG{l+s+s2}{\PYGZdq{}}\PYG{p}{)}
\end{sphinxVerbatim}

\end{sphinxuseclass}\end{sphinxVerbatimInput}
\begin{sphinxVerbatimOutput}

\begin{sphinxuseclass}{cell_output}
\noindent\sphinxincludegraphics{{34e14977fb5369f8e881ba18f47e2b8853f07c037e8f0b61abb6e8ecf41ce3b6}.png}

\end{sphinxuseclass}\end{sphinxVerbatimOutput}

\end{sphinxuseclass}
\sphinxAtStartPar
Or using the rename feature. So the variable \sphinxstylestrong{descriptions} are used instead of the variable \sphinxstylestrong{names}.

\begin{sphinxuseclass}{cell}\begin{sphinxVerbatimInput}

\begin{sphinxuseclass}{cell_input}
\begin{sphinxVerbatim}[commandchars=\\\{\}]
\PYG{n}{mpak}\PYG{p}{[}\PYG{l+s+s1}{\PYGZsq{}}\PYG{l+s+s1}{PAKCCEMISCO2?KN}\PYG{l+s+s1}{\PYGZsq{}}\PYG{p}{]}\PYG{o}{.}\PYG{n}{difpctlevel}\PYG{o}{.}\PYG{n}{mul100}\PYG{o}{.}\PYG{n}{rename}\PYG{p}{(}\PYG{p}{)}\PYG{o}{.}\PYG{n}{plot}\PYG{p}{(}\PYG{n}{title}\PYG{o}{=}\PYG{l+s+s2}{\PYGZdq{}}\PYG{l+s+s2}{Emissions impact of a \PYGZdl{}30 USD Carbon tax}\PYG{l+s+s2}{\PYGZdq{}}\PYG{p}{)}
\end{sphinxVerbatim}

\end{sphinxuseclass}\end{sphinxVerbatimInput}
\begin{sphinxVerbatimOutput}

\begin{sphinxuseclass}{cell_output}
\noindent\sphinxincludegraphics{{be10a8b0a0bc3437657b244c7a1a62ee98bb485d030b53b7ee6a5f03993555c1}.png}

\end{sphinxuseclass}\end{sphinxVerbatimOutput}

\end{sphinxuseclass}
\sphinxAtStartPar
Abstracting from the fact that the impact is occurring too quickly (it would take time for the substitution towards alternative sources of power to occur), the fact that impacts are fading with time suggests an error in the specification of the shock.  High domestic inflation means that the relative price change of a given Carbon price is declining over time.


\section{Re\sphinxhyphen{}thinking the shock as an ex\sphinxhyphen{}ante real shock}
\label{\detokenize{content/05_WBModels/MoreComplexScenarios:re-thinking-the-shock-as-an-ex-ante-real-shock}}
\sphinxAtStartPar
Inflation in Pakistan is relatively high so a \$30 shock quickly loses its relative price effect. Increasing the nominal value of the Carbon Tax by the amount of domestic inflation (converted into USD each year) would resolve the problem.

\sphinxAtStartPar
Below a new dataframe is created as acopy of the baseline and the three Carbon taxes are first set to \$30 in 2025 and then grown at the rate of domestic inflation to keep the relative price of the Carbon Tax constant.

\sphinxAtStartPar
Finally the model is re\sphinxhyphen{}solved.

\begin{sphinxuseclass}{cell}\begin{sphinxVerbatimInput}

\begin{sphinxuseclass}{cell_input}
\begin{sphinxVerbatim}[commandchars=\\\{\}]
\PYG{k+kn}{import} \PYG{n+nn}{modelmf}  \PYG{c+c1}{\PYGZsh{} import the mfcalc functionality and append it to standard pandas}
\PYG{n}{CT30realdf}\PYG{o}{=}\PYG{n}{baseline}\PYG{o}{.}\PYG{n}{upd}\PYG{p}{(}\PYG{l+s+s2}{\PYGZdq{}}\PYG{l+s+s2}{\PYGZlt{}2025 2025\PYGZgt{} PAKGGREVCO2CER PAKGGREVCO2OER PAKGGREVCO2GER = 30}\PYG{l+s+s2}{\PYGZdq{}}\PYG{p}{)}

\PYG{n}{CT30realdf}\PYG{o}{=}\PYG{n}{CT30realdf}\PYG{o}{.}\PYG{n}{mfcalc}\PYG{p}{(}\PYG{l+s+s1}{\PYGZsq{}\PYGZsq{}\PYGZsq{}}
\PYG{l+s+s1}{                              \PYGZlt{}2026 2100\PYGZgt{} PAKGGREVCO2CER = PAKGGREVCO2CER(\PYGZhy{}1)*(PAKNECONPRVTXN*PAKPANUSATLS)/(PAKNECONPRVTXN(\PYGZhy{}1)*PAKPANUSATLS(\PYGZhy{}1))}
\PYG{l+s+s1}{                                      PAKGGREVCO2OER = PAKGGREVCO2OER(\PYGZhy{}1)*(PAKNECONPRVTXN*PAKPANUSATLS)/(PAKNECONPRVTXN(\PYGZhy{}1)*PAKPANUSATLS(\PYGZhy{}1))}
\PYG{l+s+s1}{                                      PAKGGREVCO2GER = PAKGGREVCO2CER(\PYGZhy{}1)*(PAKNECONPRVTXN*PAKPANUSATLS)/(PAKNECONPRVTXN(\PYGZhy{}1)*PAKPANUSATLS(\PYGZhy{}1))}
\PYG{l+s+s1}{                          }\PYG{l+s+s1}{\PYGZsq{}\PYGZsq{}\PYGZsq{}}\PYG{p}{)}                         

\PYG{n}{CT30realdf}\PYG{o}{.}\PYG{n}{loc}\PYG{p}{[}\PYG{l+m+mi}{2023}\PYG{p}{:}\PYG{l+m+mi}{2030}\PYG{p}{,}\PYG{l+s+s1}{\PYGZsq{}}\PYG{l+s+s1}{PAKGGREVCO2CER}\PYG{l+s+s1}{\PYGZsq{}}\PYG{p}{]}


\PYG{n}{resultsdf} \PYG{o}{=} \PYG{n}{mpak}\PYG{p}{(}\PYG{n}{CT30realdf}\PYG{p}{,}\PYG{l+m+mi}{2020}\PYG{p}{,}\PYG{l+m+mi}{2100}\PYG{p}{,}\PYG{n}{keep}\PYG{o}{=}\PYG{l+s+s2}{\PYGZdq{}}\PYG{l+s+s2}{Real \PYGZdl{}30USD Carbon tax}\PYG{l+s+s2}{\PYGZdq{}}\PYG{p}{)} \PYG{c+c1}{\PYGZsh{} simulates the model }
\end{sphinxVerbatim}

\end{sphinxuseclass}\end{sphinxVerbatimInput}

\end{sphinxuseclass}
\begin{sphinxuseclass}{cell}\begin{sphinxVerbatimInput}

\begin{sphinxuseclass}{cell_input}
\begin{sphinxVerbatim}[commandchars=\\\{\}]
\PYG{k}{with} \PYG{n}{mpak}\PYG{o}{.}\PYG{n}{set\PYGZus{}smpl}\PYG{p}{(}\PYG{l+m+mi}{2023}\PYG{p}{,}\PYG{l+m+mi}{2030}\PYG{p}{)}\PYG{p}{:}
    \PYG{n+nb}{print}\PYG{p}{(}\PYG{n}{mpak}\PYG{p}{[}\PYG{l+s+s1}{\PYGZsq{}}\PYG{l+s+s1}{PAKGG*ER}\PYG{l+s+s1}{\PYGZsq{}}\PYG{p}{]}\PYG{o}{.}\PYG{n}{df}\PYG{p}{)}
\end{sphinxVerbatim}

\end{sphinxuseclass}\end{sphinxVerbatimInput}
\begin{sphinxVerbatimOutput}

\begin{sphinxuseclass}{cell_output}
\begin{sphinxVerbatim}[commandchars=\\\{\}]
      PAKGGREVCO2CER  PAKGGREVCO2GER  PAKGGREVCO2OER
2023       \PYGZhy{}5.549839      \PYGZhy{}41.000884       \PYGZhy{}8.710650
2024       \PYGZhy{}5.549839      \PYGZhy{}41.000884       \PYGZhy{}8.710650
2025       30.000000       30.000000       30.000000
2026       31.827691       31.827691       31.827691
2027       33.642459       33.642459       33.642459
2028       35.451255       35.451255       35.451255
2029       37.263353       37.263353       37.263353
2030       39.091481       39.091481       39.091481
\end{sphinxVerbatim}

\end{sphinxuseclass}\end{sphinxVerbatimOutput}

\end{sphinxuseclass}
\begin{sphinxuseclass}{cell}\begin{sphinxVerbatimInput}

\begin{sphinxuseclass}{cell_input}
\begin{sphinxVerbatim}[commandchars=\\\{\}]
\PYG{n}{mpak}\PYG{p}{[}\PYG{l+s+s1}{\PYGZsq{}}\PYG{l+s+s1}{PAKCCEMISCO2?KN}\PYG{l+s+s1}{\PYGZsq{}}\PYG{p}{]}\PYG{o}{.}\PYG{n}{difpctlevel}\PYG{o}{.}\PYG{n}{mul100}\PYG{o}{.}\PYG{n}{rename}\PYG{p}{(}\PYG{p}{)}\PYG{o}{.}\PYG{n}{plot}\PYG{p}{(}\PYG{n}{title}\PYG{o}{=}\PYG{l+s+s2}{\PYGZdq{}}\PYG{l+s+s2}{Emissions impact of a \PYGZdl{}30 USD Carbon tax}\PYG{l+s+s2}{\PYGZdq{}}\PYG{p}{)}
\end{sphinxVerbatim}

\end{sphinxuseclass}\end{sphinxVerbatimInput}
\begin{sphinxVerbatimOutput}

\begin{sphinxuseclass}{cell_output}
\noindent\sphinxincludegraphics{{7f30030b361a0be9abdccc1b5337719919249d5db27c2976415a5f54ac9a61d1}.png}

\end{sphinxuseclass}\end{sphinxVerbatimOutput}

\end{sphinxuseclass}
\sphinxAtStartPar
These results are better, but still there is an erosion of the effect of the tax.

\sphinxAtStartPar
On introspection, this is likely due to the fact that the carbon tax itself is inflationary.  As a result, prices probably rose to a higher level than supposed by the ex ante calculation.

\sphinxAtStartPar
To deal with this, a different approach is needed.  Rather than maintaining the carbon price as an exogenous variable, instead it should be made an endogenous variable by changing the model and adding equations for all of the carbon tax variables.

\sphinxAtStartPar
Before doing so lets save the current version of the model for further work later.

\begin{sphinxuseclass}{cell}\begin{sphinxVerbatimInput}

\begin{sphinxuseclass}{cell_input}
\begin{sphinxVerbatim}[commandchars=\\\{\}]
\PYG{n}{mpak}\PYG{o}{.}\PYG{n}{modeldump}\PYG{p}{(}\PYG{l+s+s1}{\PYGZsq{}}\PYG{l+s+s1}{..}\PYG{l+s+s1}{\PYGZbs{}}\PYG{l+s+s1}{models}\PYG{l+s+s1}{\PYGZbs{}}\PYG{l+s+s1}{pakCarbonTaxScenarios.pcim}\PYG{l+s+s1}{\PYGZsq{}}\PYG{p}{)}
\end{sphinxVerbatim}

\end{sphinxuseclass}\end{sphinxVerbatimInput}

\end{sphinxuseclass}

\section{Changing the model – modifying and or adding equations}
\label{\detokenize{content/05_WBModels/MoreComplexScenarios:changing-the-model-modifying-and-or-adding-equations}}
\sphinxAtStartPar
To endogenize the carbon price, an equation for each carbon price has to be added to the model.   This cane be done with the \sphinxcode{\sphinxupquote{.equpdate()}} method.

\begin{sphinxuseclass}{cell}\begin{sphinxVerbatimInput}

\begin{sphinxuseclass}{cell_input}
\begin{sphinxVerbatim}[commandchars=\\\{\}]
\PYG{n}{mpak1}\PYG{p}{,}\PYG{n}{baseline} \PYG{o}{=} \PYG{n}{model}\PYG{o}{.}\PYG{n}{modelload}\PYG{p}{(}\PYG{l+s+s1}{\PYGZsq{}}\PYG{l+s+s1}{../models/pak.pcim}\PYG{l+s+s1}{\PYGZsq{}}\PYG{p}{,}\PYG{n}{alfa}\PYG{o}{=}\PYG{l+m+mf}{0.7}\PYG{p}{,}\PYG{n}{run}\PYG{o}{=}\PYG{l+m+mi}{1}\PYG{p}{,}\PYG{n}{keep}\PYG{o}{=}\PYG{l+s+s2}{\PYGZdq{}}\PYG{l+s+s2}{Baseline}\PYG{l+s+s2}{\PYGZdq{}}\PYG{p}{)}
\PYG{n}{mpak1}\PYG{o}{.}\PYG{n}{var\PYGZus{}description} \PYG{o}{=} \PYG{p}{\PYGZob{}}\PYG{o}{*}\PYG{o}{*}\PYG{n}{mpak1}\PYG{o}{.}\PYG{n}{var\PYGZus{}description}\PYG{p}{,}\PYG{o}{*}\PYG{o}{*}\PYG{n}{extra\PYGZus{}description}\PYG{p}{\PYGZcb{}} \PYG{c+c1}{\PYGZsh{} as before we want the extra variable descriptions}


\PYG{n}{mpakreal}\PYG{p}{,}\PYG{n}{baselinereal} \PYG{o}{=} \PYG{n}{mpak1}\PYG{o}{.}\PYG{n}{equpdate}\PYG{p}{(}\PYG{l+s+s1}{\PYGZsq{}\PYGZsq{}\PYGZsq{}}
\PYG{l+s+s1}{\PYGZlt{}fixable\PYGZgt{} PAKGGREVCO2CER = PAKGGREVCO2CER(\PYGZhy{}1) * (PAKNYGDPMKTPXN*PAKPANUSATLS) / (PAKNYGDPMKTPXN(\PYGZhy{}1)*PAKPANUSATLS(\PYGZhy{}1))}
\PYG{l+s+s1}{\PYGZlt{}fixable\PYGZgt{} PAKGGREVCO2OER = PAKGGREVCO2OER(\PYGZhy{}1) * (PAKNYGDPMKTPXN*PAKPANUSATLS) / (PAKNYGDPMKTPXN(\PYGZhy{}1)*PAKPANUSATLS(\PYGZhy{}1))}
\PYG{l+s+s1}{\PYGZlt{}fixable\PYGZgt{} PAKGGREVCO2GER = PAKGGREVCO2GER(\PYGZhy{}1) * (PAKNYGDPMKTPXN*PAKPANUSATLS) / (PAKNYGDPMKTPXN(\PYGZhy{}1)*PAKPANUSATLS(\PYGZhy{}1))}
\PYG{l+s+s1}{\PYGZsq{}\PYGZsq{}\PYGZsq{}}\PYG{p}{,}\PYG{n}{add\PYGZus{}add\PYGZus{}factor}\PYG{o}{=}\PYG{k+kc}{False}\PYG{p}{,} \PYG{n}{calc\PYGZus{}add}\PYG{o}{=}\PYG{k+kc}{False}\PYG{p}{,}\PYG{n}{newname}\PYG{o}{=}\PYG{l+s+s1}{\PYGZsq{}}\PYG{l+s+s1}{Pak model, with real Carbon price equations}\PYG{l+s+s1}{\PYGZsq{}}\PYG{p}{)}
\end{sphinxVerbatim}

\end{sphinxuseclass}\end{sphinxVerbatimInput}
\begin{sphinxVerbatimOutput}

\begin{sphinxuseclass}{cell_output}
\begin{sphinxVerbatim}[commandchars=\\\{\}]
file read:  C:\PYGZbs{}modelflow manual\PYGZbs{}papers\PYGZbs{}mfbook\PYGZbs{}content\PYGZbs{}models\PYGZbs{}pak.pcim
\end{sphinxVerbatim}

\begin{sphinxVerbatim}[commandchars=\\\{\}]
The model:\PYGZdq{}PAK\PYGZdq{} got new equations, new model name is:\PYGZdq{}Pak model, with real Carbon price equations\PYGZdq{}
New equation for For PAKGGREVCO2CER
Old frml   :new endogeneous variable 
New frml   :FRML \PYGZlt{}fixable\PYGZgt{} PAKGGREVCO2CER = (PAKGGREVCO2CER(\PYGZhy{}1)*(PAKNYGDPMKTPXN*PAKPANUSATLS)/(PAKNYGDPMKTPXN(\PYGZhy{}1)*PAKPANUSATLS(\PYGZhy{}1)))* (1\PYGZhy{}PAKGGREVCO2CER\PYGZus{}D)+ PAKGGREVCO2CER\PYGZus{}X*PAKGGREVCO2CER\PYGZus{}D\PYGZdl{}
Adjust calc:No frml for adjustment calc  

New equation for For PAKGGREVCO2OER
Old frml   :new endogeneous variable 
New frml   :FRML \PYGZlt{}fixable\PYGZgt{} PAKGGREVCO2OER = (PAKGGREVCO2OER(\PYGZhy{}1)*(PAKNYGDPMKTPXN*PAKPANUSATLS)/(PAKNYGDPMKTPXN(\PYGZhy{}1)*PAKPANUSATLS(\PYGZhy{}1)))* (1\PYGZhy{}PAKGGREVCO2OER\PYGZus{}D)+ PAKGGREVCO2OER\PYGZus{}X*PAKGGREVCO2OER\PYGZus{}D\PYGZdl{}
Adjust calc:No frml for adjustment calc  

New equation for For PAKGGREVCO2GER
Old frml   :new endogeneous variable 
New frml   :FRML \PYGZlt{}fixable\PYGZgt{} PAKGGREVCO2GER = (PAKGGREVCO2GER(\PYGZhy{}1)*(PAKNYGDPMKTPXN*PAKPANUSATLS)/(PAKNYGDPMKTPXN(\PYGZhy{}1)*PAKPANUSATLS(\PYGZhy{}1)))* (1\PYGZhy{}PAKGGREVCO2GER\PYGZus{}D)+ PAKGGREVCO2GER\PYGZus{}X*PAKGGREVCO2GER\PYGZus{}D\PYGZdl{}
Adjust calc:No frml for adjustment calc  
\end{sphinxVerbatim}

\end{sphinxuseclass}\end{sphinxVerbatimOutput}

\end{sphinxuseclass}
\sphinxAtStartPar
As written, the \sphinxcode{\sphinxupquote{.equpdate()}} command creates a new model, which is a copy of the existing model with three new equations.

\sphinxAtStartPar
Each equation grows the nominal rate of the carbon tax at the same rate as inflation (\sphinxcode{\sphinxupquote{PAKNECONPRVTXN}}) converted into USD via the exchange rate \sphinxcode{\sphinxupquote{PAKPANUSATLS}}. The equations are introduced as exogenizable equations (as distinct from an identity which must always hold), by adding the <fixable> prefix to each equation. The equations are not estimated, so no add\sphinxhyphen{}factors are included in the equations.

\sphinxAtStartPar
The output for the \sphinxcode{\sphinxupquote{.equpdate()}} reports the actual formulae included in the model.

\begin{sphinxVerbatim}[commandchars=\\\{\}]
New equation for For PAKGGREVCO2CER
Old frml   :new endogeneous variable 
New frml   :FRML \PYGZlt{}fixable\PYGZgt{} PAKGGREVCO2CER = (PAKGGREVCO2CER(\PYGZhy{}1)*(PAKNECONPRVTXN*PAKPANUSATLS)/(PAKNECONPRVTXN(\PYGZhy{}1)*PAKPANUSATLS(\PYGZhy{}1)))* (1\PYGZhy{}PAKGGREVCO2CER\PYGZus{}D)+ PAKGGREVCO2CER\PYGZus{}X*PAKGGREVCO2CER\PYGZus{}D\PYGZdl{}
Adjust calc:No frml for adjustment calc 
\end{sphinxVerbatim}

\sphinxAtStartPar
Note that because the equations are to be fixable, an \_X and \_D variable are added to the specified equations. Combined they effectively split each equation into  two:
\begin{enumerate}
\sphinxsetlistlabels{\arabic}{enumi}{enumii}{}{.}%
\item {} 
\sphinxAtStartPar
the specified equations when \_D equals zero

\item {} 
\sphinxAtStartPar
equal to \_X when the \_D equals one.

\end{enumerate}

\sphinxAtStartPar
The newly created model is given the name mpakreal and is given a text description.

\sphinxAtStartPar
Following the addition of the equations, the new variables (\_D and \_X) must be initialized. The \_X variables are made equal to the current values of the various tax rates, while the \_D is set to 1 everywhere – effectively turning the equation off and re\sphinxhyphen{}creating the same situation as the initial model where the tax rates are fully exogenous.

\begin{sphinxuseclass}{cell}\begin{sphinxVerbatimInput}

\begin{sphinxuseclass}{cell_input}
\begin{sphinxVerbatim}[commandchars=\\\{\}]
\PYG{c+c1}{\PYGZsh{}Exogenizes the newly added equations and sets the dummy =1 amd the \PYGZus{}x to the current value of the dependent variable  }
\PYG{n}{baseline\PYGZus{}real}\PYG{o}{=}\PYG{n}{mpakreal}\PYG{o}{.}\PYG{n}{fix}\PYG{p}{(}\PYG{n}{baselinereal}\PYG{p}{,}\PYG{l+s+s1}{\PYGZsq{}}\PYG{l+s+s1}{PAKGGREVCO2CER PAKGGREVCO2GER PAKGGREVCO2OER}\PYG{l+s+s1}{\PYGZsq{}}\PYG{p}{)}
\end{sphinxVerbatim}

\end{sphinxuseclass}\end{sphinxVerbatimInput}
\begin{sphinxVerbatimOutput}

\begin{sphinxuseclass}{cell_output}
\begin{sphinxVerbatim}[commandchars=\\\{\}]
The folowing variables are fixed
PAKGGREVCO2CER
PAKGGREVCO2GER
PAKGGREVCO2OER
\end{sphinxVerbatim}

\end{sphinxuseclass}\end{sphinxVerbatimOutput}

\end{sphinxuseclass}
\sphinxAtStartPar
The fix command above effectively does in one line all of the following code.

\begin{sphinxVerbatim}[commandchars=\\\{\}]
\PYG{n}{baseline\PYGZus{}real}\PYG{o}{=}\PYG{n}{baselinereal}\PYG{o}{.}\PYG{n}{copy}\PYG{p}{(}\PYG{p}{)}


\PYG{c+c1}{\PYGZsh{}Create the \PYGZus{}X variables if we exogenize the equation}
\PYG{n}{baseline\PYGZus{}real} \PYG{o}{=} \PYG{n}{baseline\PYGZus{}real}\PYG{o}{.}\PYG{n}{mfcalc}\PYG{p}{(}\PYG{l+s+s1}{\PYGZsq{}\PYGZsq{}\PYGZsq{}}
\PYG{l+s+s1}{PAKGGREVCO2CER\PYGZus{}X = PAKGGREVCO2CER}
\PYG{l+s+s1}{PAKGGREVCO2GER\PYGZus{}X = PAKGGREVCO2GER}
\PYG{l+s+s1}{PAKGGREVCO2OER\PYGZus{}X = PAKGGREVCO2OER}
\PYG{l+s+s1}{\PYGZsq{}\PYGZsq{}\PYGZsq{}}\PYG{p}{)}

\PYG{c+c1}{\PYGZsh{}create the \PYGZus{}D varibale so we can exogenize the equations (set \PYGZus{}D=1)\PYGZhy{}\PYGZhy{} currently it is exogenized }
\PYG{n}{baseline\PYGZus{}real} \PYG{o}{=} \PYG{n}{baseline\PYGZus{}real}\PYG{o}{.}\PYG{n}{upd}\PYG{p}{(}\PYG{l+s+s1}{\PYGZsq{}\PYGZsq{}\PYGZsq{}}
\PYG{l+s+s1}{\PYGZlt{}\PYGZhy{}0 \PYGZhy{}1\PYGZgt{} }
\PYG{l+s+s1}{PAKGGREVCO2CER\PYGZus{}D PAKGGREVCO2GER\PYGZus{}D PAKGGREVCO2OER\PYGZus{}D = 1}
\PYG{l+s+s1}{\PYGZsq{}\PYGZsq{}\PYGZsq{}}\PYG{p}{)}
\end{sphinxVerbatim}

\sphinxAtStartPar
Finally the new model is solved, the result is kept in a new baseline and a quick check ensures that the model did indeed reproduce the data that it was originally fed, including the initial Carbon Tax levels.

\begin{sphinxuseclass}{cell}\begin{sphinxVerbatimInput}

\begin{sphinxuseclass}{cell_input}
\begin{sphinxVerbatim}[commandchars=\\\{\}]
\PYG{c+c1}{\PYGZsh{}Solve the model for the new baseline}
\PYG{n}{res} \PYG{o}{=} \PYG{n}{mpakreal}\PYG{p}{(}\PYG{n}{baseline\PYGZus{}real}\PYG{p}{,}\PYG{l+m+mi}{2021}\PYG{p}{,}\PYG{l+m+mi}{2100}\PYG{p}{,}\PYG{n}{alfa}\PYG{o}{=}\PYG{l+m+mf}{0.5}\PYG{p}{,}\PYG{n}{keep}\PYG{o}{=}\PYG{l+s+s1}{\PYGZsq{}}\PYG{l+s+s1}{Baseline \PYGZhy{} adjusted model}\PYG{l+s+s1}{\PYGZsq{}}\PYG{p}{)} 

\PYG{n}{mpakreal}\PYG{p}{[}\PYG{l+s+s1}{\PYGZsq{}}\PYG{l+s+s1}{PAKNYGDPMKTPKN PAKNECONPRVTXN PAKGGBALOVRL PAKGGREVCO2CER PAKCCEMISCO2TKN}\PYG{l+s+s1}{\PYGZsq{}}\PYG{p}{]}\PYG{o}{.}\PYG{n}{difpctlevel}\PYG{o}{.}\PYG{n}{mul100}\PYG{o}{.}\PYG{n}{df}
\end{sphinxVerbatim}

\end{sphinxuseclass}\end{sphinxVerbatimInput}
\begin{sphinxVerbatimOutput}

\begin{sphinxuseclass}{cell_output}
\begin{sphinxVerbatim}[commandchars=\\\{\}]
      PAKNYGDPMKTPKN  PAKNECONPRVTXN  PAKGGREVCO2CER  PAKCCEMISCO2TKN
2021             0.0             0.0            \PYGZhy{}0.0              0.0
2022             0.0             0.0            \PYGZhy{}0.0              0.0
2023             0.0             0.0            \PYGZhy{}0.0              0.0
2024             0.0             0.0            \PYGZhy{}0.0              0.0
2025             0.0             0.0            \PYGZhy{}0.0              0.0
...              ...             ...             ...              ...
2096             0.0             0.0            \PYGZhy{}0.0              0.0
2097             0.0             0.0            \PYGZhy{}0.0              0.0
2098             0.0             0.0            \PYGZhy{}0.0              0.0
2099             0.0             0.0            \PYGZhy{}0.0              0.0
2100             0.0             0.0            \PYGZhy{}0.0              0.0

[80 rows x 4 columns]
\end{sphinxVerbatim}

\end{sphinxuseclass}\end{sphinxVerbatimOutput}

\end{sphinxuseclass}

\subsection{Solving the revised model}
\label{\detokenize{content/05_WBModels/MoreComplexScenarios:solving-the-revised-model}}
\sphinxAtStartPar
With the new model generated, it can now be solves with the real tax rate endogenized in the forecast period.  This involves three steps.
\begin{enumerate}
\sphinxsetlistlabels{\arabic}{enumi}{enumii}{}{.}%
\item {} 
\sphinxAtStartPar
Set the nominal tax rate to 30 in 2024

\item {} 
\sphinxAtStartPar
Now Endogenize the equation for the rest of the period

\item {} 
\sphinxAtStartPar
Solve the model.

\end{enumerate}

\begin{sphinxuseclass}{cell}\begin{sphinxVerbatimInput}

\begin{sphinxuseclass}{cell_input}
\begin{sphinxVerbatim}[commandchars=\\\{\}]
\PYG{n}{scenario\PYGZus{}real\PYGZus{}CTax} \PYG{o}{=} \PYG{n}{baseline\PYGZus{}real}\PYG{o}{.}\PYG{n}{upd}\PYG{p}{(}\PYG{l+s+s1}{\PYGZsq{}\PYGZsq{}\PYGZsq{}}
\PYG{l+s+s1}{\PYGZlt{}2024 2024\PYGZgt{} }
\PYG{l+s+s1}{PAKGGREVCO2CER\PYGZus{}x PAKGGREVCO2GER\PYGZus{}x PAKGGREVCO2OER\PYGZus{}x = 30 \PYGZsh{} Sets the exogenous value to 29 in 2024}
\PYG{l+s+s1}{\PYGZlt{}2025 2100 \PYGZgt{} }
\PYG{l+s+s1}{PAKGGREVCO2CER\PYGZus{}D PAKGGREVCO2GER\PYGZus{}D PAKGGREVCO2OER\PYGZus{}d = 0   \PYGZsh{} Endogenizes the new equations for the rest of time so that the real\PYGZhy{}rate stays at 30USD}
\PYG{l+s+s1}{\PYGZsq{}\PYGZsq{}\PYGZsq{}}\PYG{p}{)}


\PYG{n}{\PYGZus{}} \PYG{o}{=} \PYG{n}{mpakreal}\PYG{p}{(}\PYG{n}{scenario\PYGZus{}real\PYGZus{}CTax}\PYG{p}{,}\PYG{l+m+mi}{2021}\PYG{p}{,}\PYG{l+m+mi}{2100}\PYG{p}{,}\PYG{n}{alfa}\PYG{o}{=}\PYG{l+m+mf}{0.5}\PYG{p}{,}\PYG{n}{keep}\PYG{o}{=}\PYG{l+s+s1}{\PYGZsq{}}\PYG{l+s+s1}{Real model real tax = 30 in 2022 currency units}\PYG{l+s+s1}{\PYGZsq{}}\PYG{p}{)}
\end{sphinxVerbatim}

\end{sphinxuseclass}\end{sphinxVerbatimInput}

\end{sphinxuseclass}
\sphinxAtStartPar
Initially the Carbon tax comes in at 30 but gradually its rate in USD rises in line with inflation such that it reaches \$1500 by 2100.

\begin{sphinxuseclass}{cell}\begin{sphinxVerbatimInput}

\begin{sphinxuseclass}{cell_input}
\begin{sphinxVerbatim}[commandchars=\\\{\}]
\PYG{n+nb}{round}\PYG{p}{(}\PYG{n}{mpakreal}\PYG{p}{[}\PYG{l+s+s1}{\PYGZsq{}}\PYG{l+s+s1}{PAKGGREVCO2?ER PAKNECONPRVTXN PAKPANUSATLS}\PYG{l+s+s1}{\PYGZsq{}}\PYG{p}{]}\PYG{o}{.}\PYG{n}{df}\PYG{p}{,}\PYG{l+m+mi}{1}\PYG{p}{)}
\end{sphinxVerbatim}

\end{sphinxuseclass}\end{sphinxVerbatimInput}
\begin{sphinxVerbatimOutput}

\begin{sphinxuseclass}{cell_output}
\begin{sphinxVerbatim}[commandchars=\\\{\}]
      PAKGGREVCO2CER  PAKGGREVCO2GER  PAKGGREVCO2OER  PAKNECONPRVTXN   
2021            \PYGZhy{}5.5           \PYGZhy{}41.0            \PYGZhy{}8.7             1.8  \PYGZbs{}
2022            \PYGZhy{}5.5           \PYGZhy{}41.0            \PYGZhy{}8.7             2.0   
2023            \PYGZhy{}5.5           \PYGZhy{}41.0            \PYGZhy{}8.7             2.1   
2024            30.0            30.0            30.0             2.4   
2025            32.2            32.2            32.2             2.5   
...              ...             ...             ...             ...   
2096          1174.0          1174.0          1174.0           106.7   
2097          1237.9          1237.9          1237.9           112.8   
2098          1305.4          1305.4          1305.4           119.2   
2099          1376.5          1376.5          1376.5           126.0   
2100          1451.4          1451.4          1451.4           133.2   

      PAKPANUSATLS  
2021         107.0  
2022         106.8  
2023         106.7  
2024         106.3  
2025         106.2  
...            ...  
2096          94.3  
2097          94.1  
2098          93.9  
2099          93.7  
2100          93.5  

[80 rows x 5 columns]
\end{sphinxVerbatim}

\end{sphinxuseclass}\end{sphinxVerbatimOutput}

\end{sphinxuseclass}
\sphinxAtStartPar
This seemingly very high level is just a reflection of the 75 years of inflation that compounded require a much higher nominal Carbon tax rate to have the same relative price effect. The cumulative effect of inflation in the range of 5.5 percent per annum causes the price level to increase 74 times (7400 percent increase  133/1.8 from fourth data column in the above table).

\sphinxAtStartPar
The table below shows the same data but in growth rate terms – indicating that the Carbon tax effective rate is gradually rising each year in line domestic inflation adjusted for the exchange rate.

\begin{sphinxuseclass}{cell}\begin{sphinxVerbatimInput}

\begin{sphinxuseclass}{cell_input}
\begin{sphinxVerbatim}[commandchars=\\\{\}]
\PYG{n}{mpakreal}\PYG{p}{[}\PYG{l+s+s1}{\PYGZsq{}}\PYG{l+s+s1}{PAKGGREVCO2?ER PAKNECONPRVTXN PAKPANUSATLS}\PYG{l+s+s1}{\PYGZsq{}}\PYG{p}{]}\PYG{o}{.}\PYG{n}{pct}\PYG{o}{.}\PYG{n}{mul100}\PYG{o}{.}\PYG{n}{df}
\end{sphinxVerbatim}

\end{sphinxuseclass}\end{sphinxVerbatimInput}
\begin{sphinxVerbatimOutput}

\begin{sphinxuseclass}{cell_output}
\begin{sphinxVerbatim}[commandchars=\\\{\}]
      PAKGGREVCO2CER  PAKGGREVCO2GER  PAKGGREVCO2OER  PAKNECONPRVTXN   
2021        0.000000        0.000000        0.000000        9.476828  \PYGZbs{}
2022        0.000000        0.000000        0.000000        8.776453   
2023        0.000000        0.000000        0.000000        7.978008   
2024     \PYGZhy{}640.556268     \PYGZhy{}173.169155     \PYGZhy{}444.405991       10.081669   
2025        7.388261        7.388261        7.388261        7.073469   
...              ...             ...             ...             ...   
2096        5.448370        5.448370        5.448370        5.695440   
2097        5.447880        5.447880        5.447880        5.695184   
2098        5.447324        5.447324        5.447324        5.694856   
2099        5.446708        5.446708        5.446708        5.694466   
2100        5.446042        5.446042        5.446042        5.694020   

      PAKPANUSATLS  
2021     \PYGZhy{}0.158806  
2022     \PYGZhy{}0.155332  
2023     \PYGZhy{}0.137261  
2024     \PYGZhy{}0.328810  
2025     \PYGZhy{}0.134969  
...            ...  
2096     \PYGZhy{}0.201701  
2097     \PYGZhy{}0.201682  
2098     \PYGZhy{}0.201658  
2099     \PYGZhy{}0.201629  
2100     \PYGZhy{}0.201596  

[80 rows x 5 columns]
\end{sphinxVerbatim}

\end{sphinxuseclass}\end{sphinxVerbatimOutput}

\end{sphinxuseclass}

\subsection{Results}
\label{\detokenize{content/05_WBModels/MoreComplexScenarios:results}}
\sphinxAtStartPar
The results from the simulation with the Carbon Tax rate endogenized so as to maintain its real value over time, are broadly consistent with the results from the ex ante real scenario performed above.

\begin{sphinxuseclass}{cell}\begin{sphinxVerbatimInput}

\begin{sphinxuseclass}{cell_input}
\begin{sphinxVerbatim}[commandchars=\\\{\}]
\PYG{n}{mpakreal}\PYG{p}{[}\PYG{l+s+s1}{\PYGZsq{}}\PYG{l+s+s1}{PAKCCEMISCO2?KN}\PYG{l+s+s1}{\PYGZsq{}}\PYG{p}{]}\PYG{o}{.}\PYG{n}{difpctlevel}\PYG{o}{.}\PYG{n}{mul100}\PYG{o}{.}\PYG{n}{df}
\end{sphinxVerbatim}

\end{sphinxuseclass}\end{sphinxVerbatimInput}
\begin{sphinxVerbatimOutput}

\begin{sphinxuseclass}{cell_output}
\begin{sphinxVerbatim}[commandchars=\\\{\}]
      PAKCCEMISCO2CKN  PAKCCEMISCO2GKN  PAKCCEMISCO2OKN  PAKCCEMISCO2TKN
2021         0.000000         0.000000         0.000000         0.000000
2022         0.000000         0.000000         0.000000         0.000000
2023         0.000000         0.000000         0.000000         0.000000
2024       \PYGZhy{}42.758320       \PYGZhy{}28.536509       \PYGZhy{}11.513706       \PYGZhy{}23.275423
2025       \PYGZhy{}43.091702       \PYGZhy{}27.682273       \PYGZhy{}12.205913       \PYGZhy{}23.439332
...               ...              ...              ...              ...
2096       \PYGZhy{}25.808151       \PYGZhy{}11.978413        \PYGZhy{}5.652815       \PYGZhy{}11.846362
2097       \PYGZhy{}25.660632       \PYGZhy{}11.894715        \PYGZhy{}5.601081       \PYGZhy{}11.764007
2098       \PYGZhy{}25.513640       \PYGZhy{}11.811486        \PYGZhy{}5.549756       \PYGZhy{}11.682112
2099       \PYGZhy{}25.367162       \PYGZhy{}11.728715        \PYGZhy{}5.498829       \PYGZhy{}11.600668
2100       \PYGZhy{}25.221185       \PYGZhy{}11.646395        \PYGZhy{}5.448290       \PYGZhy{}11.519664

[80 rows x 4 columns]
\end{sphinxVerbatim}

\end{sphinxuseclass}\end{sphinxVerbatimOutput}

\end{sphinxuseclass}
\begin{sphinxuseclass}{cell}\begin{sphinxVerbatimInput}

\begin{sphinxuseclass}{cell_input}
\begin{sphinxVerbatim}[commandchars=\\\{\}]
\PYG{n}{mpakreal}\PYG{p}{[}\PYG{l+s+s1}{\PYGZsq{}}\PYG{l+s+s1}{PAKCCEMISCO2?KN PAKNYGDPMKTPKN}\PYG{l+s+s1}{\PYGZsq{}}\PYG{p}{]}\PYG{o}{.}\PYG{n}{difpctlevel}\PYG{o}{.}\PYG{n}{mul100}\PYG{o}{.}\PYG{n}{rename}\PYG{p}{(}\PYG{p}{)}\PYG{o}{.}\PYG{n}{plot}\PYG{p}{(}
    \PYG{n}{title}\PYG{o}{=}\PYG{l+s+s2}{\PYGZdq{}}\PYG{l+s+s2}{Emissions impact of a \PYGZdl{}30 USD Carbon tax}\PYG{l+s+s2}{\PYGZdq{}}\PYG{p}{,}\PYG{n}{showfig}\PYG{o}{=}\PYG{k+kc}{True}\PYG{p}{)}
\end{sphinxVerbatim}

\end{sphinxuseclass}\end{sphinxVerbatimInput}
\begin{sphinxVerbatimOutput}

\begin{sphinxuseclass}{cell_output}
\noindent\sphinxincludegraphics{{33b50cb47f7b8dcde51371be48a791836b66cdccc19a99c5ed26d4fcc69e44be}.png}

\end{sphinxuseclass}\end{sphinxVerbatimOutput}

\end{sphinxuseclass}
\sphinxAtStartPar
The modified model, which preserves the real value of the carbon tax has a permanent and substantive negative effect on emissions. The impact is not at an unchanging level, reflecting in part adaptation within the economy. Of particular import is what is being done with the revenues from the Carbon tax, the structure of GDP (shifts to more carbon intensive activities) and in this particular scenario the fact that the level of activity is higher and therefore emission higher than they would have been had GDP remained unchanged.

\sphinxstepscope


\part{Model Analytics}

\sphinxstepscope


\chapter{Model analytics}
\label{\detokenize{content/06_ModelAnalytics/ModelStructure:model-analytics}}\label{\detokenize{content/06_ModelAnalytics/ModelStructure::doc}}
\sphinxAtStartPar
A model has a well defined logical and causal structure.\hyperlink{cite.content/99_BackMatter/References:id4}{Kogiku}provides an introduction to causal analysis of models can be found in (1968), while \hyperlink{cite.content/99_BackMatter/References:id2}{Berndsen} provides a more elaborate discussion.

\sphinxAtStartPar
At the simplest level, the equations of a model can be organized into blocks.
\begin{itemize}
\item {} 
\sphinxAtStartPar
\sphinxstylestrong{Simultaneous block} include equations that have are co\sphinxhyphen{}determined simultaneously. They contain feedback loops that mean they may require several iterations before a solution that satisfies them all is found. A classic simultaneous block would include GDP and Consumption. Consumption depends on income. Income depends on GDP, but COnsumption determines GDP.

\item {} 
\sphinxAtStartPar
\sphinxstylestrong{Recursive blocks} include equations that are a simple function of other variables. For example, the current account balance is just the difference between Export Revenues and Import Revenues.  These can be solved with just one pass once the values of the simultaneous blocks have been resolved.

\end{itemize}

\sphinxAtStartPar
At the equation level, each endogenous variable is a function of one or more variables, but because these variables are also dependent on other variables in the model, those right hand side variables that are endogenous can have their equations substituted into the first level equation to get an extended set of dependencies and he endogenous right hand side variables of these second level variables can also have their right hand sides substituted into the equation etc.

\sphinxAtStartPar
\sphinxcode{\sphinxupquote{Modelflow}} uses the \sphinxhref{https://networkx.org/}{networkx} python package to analyze the interrelation ships within the model and between equations and includes a number of methods and properties to present these interrelationships both in tabular and graphical form {[}\textasciicircum{}Graphvix{]}, a subset of which is exposed in this chapter.

\sphinxAtStartPar
Setting up the python environment and loading a pre\sphinxhyphen{}exisitng model

\begin{sphinxuseclass}{cell}\begin{sphinxVerbatimInput}

\begin{sphinxuseclass}{cell_input}
\begin{sphinxVerbatim}[commandchars=\\\{\}]
\PYG{k+kn}{from} \PYG{n+nn}{modelclass} \PYG{k+kn}{import} \PYG{n}{model}

\PYG{n}{latex} \PYG{o}{=} \PYG{k+kc}{True}    \PYG{c+c1}{\PYGZsh{} Setting to allow the diagrams to be incorporated in latex}

\PYG{n}{mpak}\PYG{p}{,}\PYG{n}{baseline} \PYG{o}{=} \PYG{n}{model}\PYG{o}{.}\PYG{n}{modelload}\PYG{p}{(}\PYG{l+s+s1}{\PYGZsq{}}\PYG{l+s+s1}{../models/pak.pcim}\PYG{l+s+s1}{\PYGZsq{}}\PYG{p}{,}\PYG{n}{alfa}\PYG{o}{=}\PYG{l+m+mf}{0.7}\PYG{p}{,}\PYG{n}{run}\PYG{o}{=}\PYG{l+m+mi}{1}\PYG{p}{)}

\PYG{n}{mpak}\PYG{o}{.}\PYG{n}{model\PYGZus{}description}\PYG{o}{=}\PYG{l+s+s2}{\PYGZdq{}}\PYG{l+s+s2}{World Bank climate aware model of Pakistan as described in Burns et al. (2019)}\PYG{l+s+s2}{\PYGZdq{}}
\PYG{n}{mpak}\PYG{o}{.}\PYG{n}{model\PYGZus{}description}
\PYG{n}{mpak}\PYG{o}{.}\PYG{n}{periode}\PYG{o}{=}\PYG{l+m+mi}{2100}
\end{sphinxVerbatim}

\end{sphinxuseclass}\end{sphinxVerbatimInput}
\begin{sphinxVerbatimOutput}

\begin{sphinxuseclass}{cell_output}
\begin{sphinxVerbatim}[commandchars=\\\{\}]
file read:  C:\PYGZbs{}modelflow manual\PYGZbs{}papers\PYGZbs{}mfbook\PYGZbs{}content\PYGZbs{}models\PYGZbs{}pak.pcim
\end{sphinxVerbatim}

\end{sphinxuseclass}\end{sphinxVerbatimOutput}

\end{sphinxuseclass}

\section{Model information}
\label{\detokenize{content/06_ModelAnalytics/ModelStructure:model-information}}
\sphinxAtStartPar
The model object contains information about the model itself, its name, its structure (does it contain simultaneous equations or is it recursive), the number of variables it contains and the number that are exogenous and endogenous (have associated equations).

\begin{sphinxuseclass}{cell}\begin{sphinxVerbatimInput}

\begin{sphinxuseclass}{cell_input}
\begin{sphinxVerbatim}[commandchars=\\\{\}]
\PYG{n}{mpak}
\end{sphinxVerbatim}

\end{sphinxuseclass}\end{sphinxVerbatimInput}
\begin{sphinxVerbatimOutput}

\begin{sphinxuseclass}{cell_output}
\begin{sphinxVerbatim}[commandchars=\\\{\}]
\PYGZlt{}
Model name                              :                  PAK 
Model structure                         :         Simultaneous 
Number of variables                     :                  839 
Number of exogeneous  variables         :                  461 
Number of endogeneous variables         :                  378 
\PYGZgt{}
\end{sphinxVerbatim}

\end{sphinxuseclass}\end{sphinxVerbatimOutput}

\end{sphinxuseclass}
\sphinxAtStartPar
The model work space also has a time dimension, its sample period. This can be retrieved and changed.

\sphinxAtStartPar
`mpak.per\_current’

\begin{sphinxuseclass}{cell}\begin{sphinxVerbatimInput}

\begin{sphinxuseclass}{cell_input}
\begin{sphinxVerbatim}[commandchars=\\\{\}]
\PYG{n}{mpak}\PYG{o}{.}\PYG{n}{current\PYGZus{}per}
\end{sphinxVerbatim}

\end{sphinxuseclass}\end{sphinxVerbatimInput}
\begin{sphinxVerbatimOutput}

\begin{sphinxuseclass}{cell_output}
\begin{sphinxVerbatim}[commandchars=\\\{\}]
Index([2016, 2017, 2018, 2019, 2020, 2021, 2022, 2023, 2024, 2025, 2026, 2027,
       2028, 2029, 2030],
      dtype=\PYGZsq{}int64\PYGZsq{})
\end{sphinxVerbatim}

\end{sphinxuseclass}\end{sphinxVerbatimOutput}

\end{sphinxuseclass}

\section{Model structure}
\label{\detokenize{content/06_ModelAnalytics/ModelStructure:model-structure}}
\sphinxAtStartPar
A quick way to visualize the structure of a model is to plot its \sphinxhref{https://en.wikipedia.org/wiki/Adjacency\_matrix}{adjacency matrix}.

\sphinxAtStartPar
The adjacency matrix plots the relationships between endogenous variables in the model, dividing them into one or more simultaneous blocks and one or more recursive blocks.

\sphinxAtStartPar
Below is the adjacency matrix for the Pakistan model. Variables in the red square block depend on one or more variables that in turn depends upon them, requiring the mode to solve for their values simultaneously.  The variables in the green triangles do not enter directly or indirectly as an argument in the variables that determine them and therefore can be solved in one iteration once the values for the simultaneous variables are determined.

\begin{sphinxuseclass}{cell}\begin{sphinxVerbatimInput}

\begin{sphinxuseclass}{cell_input}
\begin{sphinxVerbatim}[commandchars=\\\{\}]
\PYG{n}{mpak}\PYG{o}{.}\PYG{n}{plotadjacency}\PYG{p}{(}\PYG{n}{size}\PYG{o}{=}\PYG{p}{(}\PYG{l+m+mi}{20}\PYG{p}{,}\PYG{l+m+mi}{20}\PYG{p}{)}\PYG{p}{,}\PYG{n}{nolag}\PYG{o}{=}\PYG{l+m+mi}{0}\PYG{p}{)}\PYG{p}{;}
\end{sphinxVerbatim}

\end{sphinxuseclass}\end{sphinxVerbatimInput}
\begin{sphinxVerbatimOutput}

\begin{sphinxuseclass}{cell_output}
\noindent\sphinxincludegraphics{{937b07e997fb331c918a5fab34d0d34253c773235bf11b44a2cf1d0312342aa9}.png}

\end{sphinxuseclass}\end{sphinxVerbatimOutput}

\end{sphinxuseclass}

\chapter{The dependencies of individual endogenous variables (the \sphinxstyleliteralintitle{\sphinxupquote{.tracepre()}}  and \sphinxstyleliteralintitle{\sphinxupquote{.tracedep()}}methods)}
\label{\detokenize{content/06_ModelAnalytics/ModelStructure:the-dependencies-of-individual-endogenous-variables-the-tracepre-and-tracedep-methods}}
\sphinxAtStartPar
As noted above, every endogenous variables is directly dependent on the variables that occur on its right hand side (RHS), but is also indirectly dependent on the variables that determine its RHS variables and in turn those that determine the variables to the right of them \sphinxstyleemphasis{ad infinitum}.

\sphinxAtStartPar
\sphinxcode{\sphinxupquote{Modelflow}} includes several methods and properties that allow these dependencies to be explored.

\sphinxAtStartPar
The \sphinxcode{\sphinxupquote{.frml}} property returns the normalized formula of an equation, from which the right hand variables for the equation can be discerned.

\begin{sphinxuseclass}{cell}\begin{sphinxVerbatimInput}

\begin{sphinxuseclass}{cell_input}
\begin{sphinxVerbatim}[commandchars=\\\{\}]
\PYG{n}{mpak}\PYG{o}{.}\PYG{n}{PAKNECONPRVTKN}\PYG{o}{.}\PYG{n}{frml}
\end{sphinxVerbatim}

\end{sphinxuseclass}\end{sphinxVerbatimInput}
\begin{sphinxVerbatimOutput}

\begin{sphinxuseclass}{cell_output}
\begin{sphinxVerbatim}[commandchars=\\\{\}]
Endogeneous: PAKNECONPRVTKN: HH. Cons Real
Formular: FRML \PYGZlt{}DAMP,STOC\PYGZgt{} PAKNECONPRVTKN = (PAKNECONPRVTKN(\PYGZhy{}1)*EXP(PAKNECONPRVTKN\PYGZus{}A+ (\PYGZhy{}0.2*(LOG(PAKNECONPRVTKN(\PYGZhy{}1))\PYGZhy{}LOG(1.21203101101442)\PYGZhy{}LOG((((PAKBXFSTREMTCD(\PYGZhy{}1)\PYGZhy{}PAKBMFSTREMTCD(\PYGZhy{}1))*PAKPANUSATLS(\PYGZhy{}1))+PAKGGEXPTRNSCN(\PYGZhy{}1)+PAKNYYWBTOTLCN(\PYGZhy{}1)*(1\PYGZhy{}PAKGGREVDRCTXN(\PYGZhy{}1)/100))/PAKNECONPRVTXN(\PYGZhy{}1)))+0.763938860758873*((LOG((((PAKBXFSTREMTCD\PYGZhy{}PAKBMFSTREMTCD)*PAKPANUSATLS)+PAKGGEXPTRNSCN+PAKNYYWBTOTLCN*(1\PYGZhy{}PAKGGREVDRCTXN/100))/PAKNECONPRVTXN))\PYGZhy{}(LOG((((PAKBXFSTREMTCD(\PYGZhy{}1)\PYGZhy{}PAKBMFSTREMTCD(\PYGZhy{}1))*PAKPANUSATLS(\PYGZhy{}1))+PAKGGEXPTRNSCN(\PYGZhy{}1)+PAKNYYWBTOTLCN(\PYGZhy{}1)*(1\PYGZhy{}PAKGGREVDRCTXN(\PYGZhy{}1)/100))/PAKNECONPRVTXN(\PYGZhy{}1))))\PYGZhy{}0.0634474791568939*DURING\PYGZus{}2009\PYGZhy{}0.3*(PAKFMLBLPOLYXN/100\PYGZhy{}((LOG(PAKNECONPRVTXN))\PYGZhy{}(LOG(PAKNECONPRVTXN(\PYGZhy{}1)))))) )) * (1\PYGZhy{}PAKNECONPRVTKN\PYGZus{}D)+ PAKNECONPRVTKN\PYGZus{}X*PAKNECONPRVTKN\PYGZus{}D  \PYGZdl{}

PAKNECONPRVTKN  : HH. Cons Real
DURING\PYGZus{}2009     : 
PAKBMFSTREMTCD  : Imp., Remittances (BOP), US\PYGZdl{} mn
PAKBXFSTREMTCD  : Exp., Remittances (BOP), US\PYGZdl{} mn
PAKFMLBLPOLYXN  : Key Policy Interest Rate
PAKGGEXPTRNSCN  : Current Transfers
PAKGGREVDRCTXN  : Direct Revenue Tax Rate
PAKNECONPRVTKN\PYGZus{}A: Add factor:HH. Cons Real
PAKNECONPRVTKN\PYGZus{}D: Fix dummy:HH. Cons Real
PAKNECONPRVTKN\PYGZus{}X: Fix value:HH. Cons Real
PAKNECONPRVTXN  : Implicit LCU defl., Pvt. Cons., 2000 = 1
PAKNYYWBTOTLCN  : Total Wage Bill
PAKPANUSATLS    : Exchange rate LCU / US\PYGZdl{} \PYGZhy{} Pakistan
\end{sphinxVerbatim}

\end{sphinxuseclass}\end{sphinxVerbatimOutput}

\end{sphinxuseclass}
\sphinxAtStartPar
The method \sphinxcode{\sphinxupquote{.tracepre()}} provides a graphical representation of this relationship, showing all the variables that directly determine an endogenous variable (in this example real GDP), distinguishing between RHS variables that are endogenous (in blue) and those that are exogenous (yellow).

\begin{sphinxuseclass}{cell}\begin{sphinxVerbatimInput}

\begin{sphinxuseclass}{cell_input}
\begin{sphinxVerbatim}[commandchars=\\\{\}]
\PYG{n}{mpak}\PYG{o}{.}\PYG{n}{PAKNYGDPMKTPKN}\PYG{o}{.}\PYG{n}{tracepre}\PYG{p}{(}\PYG{n}{png}\PYG{o}{=}\PYG{n}{latex}\PYG{p}{)}
\end{sphinxVerbatim}

\end{sphinxuseclass}\end{sphinxVerbatimInput}
\begin{sphinxVerbatimOutput}

\begin{sphinxuseclass}{cell_output}
\noindent\sphinxincludegraphics{{343794ab8eccfe852a2f4fc3b6753a76097c4faa01c1ddf00a0257b23396180f}.png}

\end{sphinxuseclass}\end{sphinxVerbatimOutput}

\end{sphinxuseclass}
\sphinxAtStartPar
If the model has been solved, \sphinxcode{\sphinxupquote{.pretrc()}} goes one step further and reveals the relative imortance of each variable in the change of the dependent variable.


\section{Shock the model}
\label{\detokenize{content/06_ModelAnalytics/ModelStructure:shock-the-model}}
\sphinxAtStartPar
Below a \$30 nominal Carbon tax is applied beginning in 2025.

\begin{sphinxuseclass}{cell}\begin{sphinxVerbatimInput}

\begin{sphinxuseclass}{cell_input}
\begin{sphinxVerbatim}[commandchars=\\\{\}]
\PYG{n}{alternative}  \PYG{o}{=}  \PYG{n}{baseline}\PYG{o}{.}\PYG{n}{upd}\PYG{p}{(}\PYG{l+s+s2}{\PYGZdq{}}\PYG{l+s+s2}{\PYGZlt{}2025 2100\PYGZgt{} PAKGGREVCO2CER PAKGGREVCO2GER PAKGGREVCO2OER = 30}\PYG{l+s+s2}{\PYGZdq{}}\PYG{p}{)}
\PYG{n}{result} \PYG{o}{=} \PYG{n}{mpak}\PYG{p}{(}\PYG{n}{alternative}\PYG{p}{,}\PYG{l+m+mi}{2020}\PYG{p}{,}\PYG{l+m+mi}{2100}\PYG{p}{)} \PYG{c+c1}{\PYGZsh{} simulates the model }
\end{sphinxVerbatim}

\end{sphinxuseclass}\end{sphinxVerbatimInput}

\end{sphinxuseclass}
\sphinxAtStartPar
As a result GDP, consumption investment and most all variables in the model change, as illustrated in the below graphs that show the percent deviation of the main components of GDP from their baseline values.

\begin{sphinxuseclass}{cell}\begin{sphinxVerbatimInput}

\begin{sphinxuseclass}{cell_input}
\begin{sphinxVerbatim}[commandchars=\\\{\}]
\PYG{n}{mpak}\PYG{p}{[}\PYG{l+s+s1}{\PYGZsq{}}\PYG{l+s+s1}{PAKNYGDPMKTPKN PAKNECONPRVTKN PAKNEGDIFTOTKN PAKNEEXPGNFSKN PAKNEIMPGNFSKN}\PYG{l+s+s1}{\PYGZsq{}}\PYG{p}{]}\PYG{o}{.}\PYG{n}{difpctlevel}\PYG{o}{.}\PYG{n}{rename}\PYG{p}{(}\PYG{p}{)}\PYG{o}{.}\PYG{n}{plot}\PYG{p}{(}\PYG{p}{)}
\end{sphinxVerbatim}

\end{sphinxuseclass}\end{sphinxVerbatimInput}
\begin{sphinxVerbatimOutput}

\begin{sphinxuseclass}{cell_output}
\noindent\sphinxincludegraphics{{d181499adb4c6baeff5aa3687647f58ef8ccca4b337bb030f538b8ca5d10b91b}.png}

\end{sphinxuseclass}\end{sphinxVerbatimOutput}

\end{sphinxuseclass}

\section{Post shock \sphinxstyleliteralintitle{\sphinxupquote{.tracepre()}} indicates the relative importance of different variables in explaining the change in the dependent variable}
\label{\detokenize{content/06_ModelAnalytics/ModelStructure:post-shock-tracepre-indicates-the-relative-importance-of-different-variables-in-explaining-the-change-in-the-dependent-variable}}
\sphinxAtStartPar
Below the same command is executed, but because of the shock, the width of the lines indicating representing the causal links between variables is ticker the more important a given variable was in the previous simulation in explaining the change in the level of the dependent variable (GDP).

\begin{sphinxuseclass}{cell}\begin{sphinxVerbatimInput}

\begin{sphinxuseclass}{cell_input}
\begin{sphinxVerbatim}[commandchars=\\\{\}]
\PYG{n}{mpak}\PYG{o}{.}\PYG{n}{PAKNYGDPMKTPKN}\PYG{o}{.}\PYG{n}{tracepre}\PYG{p}{(}\PYG{n}{png}\PYG{o}{=}\PYG{n}{latex}\PYG{p}{)}
\end{sphinxVerbatim}

\end{sphinxuseclass}\end{sphinxVerbatimInput}
\begin{sphinxVerbatimOutput}

\begin{sphinxuseclass}{cell_output}
\noindent\sphinxincludegraphics{{8248f804f56e3c08c29d1bf44ae6d4534311efb6b025753d2ceef8303c0e2117}.png}

\end{sphinxuseclass}\end{sphinxVerbatimOutput}

\end{sphinxuseclass}
\begin{sphinxadmonition}{note}{Note:}
\sphinxAtStartPar
\sphinxstylestrong{png=latex}

\sphinxAtStartPar
The default behavior when displaying graphs in a \sphinxstyleemphasis{jupyter notebook} is to produce images in .svg format.
These images scale well and the mouseover feature can be used. That is: On mouseover of a node, the variable and the equation are displayed.  On mouseover on an joining line, the extent to which the variable contributed to the change in the dependent variable is displayed.

\sphinxAtStartPar
Unfortunately this \sphinxstyleemphasis{jupyter book} (that is not a notebook) requires images be in  jpg or PNG format so this functionality has been disabled, by specifying that the png format be used instead of svg.

\sphinxAtStartPar
For other purposes, the variable latex could be set equal to False in which case the same code will generate SVGs. In this case you can use the mouse to hover over elements to get descriptions
\end{sphinxadmonition}


\subsection{The filter option, restricting the output of \sphinxstyleliteralintitle{\sphinxupquote{.tracepre()}}}
\label{\detokenize{content/06_ModelAnalytics/ModelStructure:the-filter-option-restricting-the-output-of-tracepre}}
\sphinxAtStartPar
Using the filter option, the output of \sphinxcode{\sphinxupquote{.tracepre()}} can be restricted to RHS variables that have had large impacts on the dependent variable.v

\begin{sphinxuseclass}{cell}\begin{sphinxVerbatimInput}

\begin{sphinxuseclass}{cell_input}
\begin{sphinxVerbatim}[commandchars=\\\{\}]
\PYG{n}{gg}\PYG{o}{=}\PYG{n}{mpak}\PYG{o}{.}\PYG{n}{PAKNYGDPMKTPKN}\PYG{o}{.}\PYG{n}{tracepre}\PYG{p}{(}\PYG{n+nb}{filter}\PYG{o}{=}\PYG{l+m+mi}{20}\PYG{p}{,}\PYG{n}{png}\PYG{o}{=}\PYG{n}{latex}\PYG{p}{)}
\PYG{n}{gg}
\end{sphinxVerbatim}

\end{sphinxuseclass}\end{sphinxVerbatimInput}
\begin{sphinxVerbatimOutput}

\begin{sphinxuseclass}{cell_output}
\noindent\sphinxincludegraphics{{5425cb207e94d6dd59bd2662490eeb8698d717a17e221c5194a69aaf0695eac3}.png}

\end{sphinxuseclass}\end{sphinxVerbatimOutput}

\end{sphinxuseclass}

\subsection{The up option, extending the \sphinxstyleliteralintitle{\sphinxupquote{.tracepre}} beyond the first level causal variables}
\label{\detokenize{content/06_ModelAnalytics/ModelStructure:the-up-option-extending-the-tracepre-beyond-the-first-level-causal-variables}}
\sphinxAtStartPar
The up option allows \sphinxcode{\sphinxupquote{.tracepre}} dependencies to be followed through beyond the first level of causal variables.  Below it is extended to variables as much as three levels back, and restricted to those whose variation explains at least 20 percent of the change in GDP.

\begin{sphinxuseclass}{cell}\begin{sphinxVerbatimInput}

\begin{sphinxuseclass}{cell_input}
\begin{sphinxVerbatim}[commandchars=\\\{\}]
\PYG{n}{mpak}\PYG{o}{.}\PYG{n}{PAKNYGDPMKTPKN}\PYG{o}{.}\PYG{n}{tracepre}\PYG{p}{(}\PYG{n+nb}{filter} \PYG{o}{=} \PYG{l+m+mi}{20}\PYG{p}{,}\PYG{n}{up}\PYG{o}{=}\PYG{l+m+mi}{3}\PYG{p}{,}\PYG{n}{png}\PYG{o}{=}\PYG{n}{latex}\PYG{p}{,}\PYG{p}{)}
\end{sphinxVerbatim}

\end{sphinxuseclass}\end{sphinxVerbatimInput}
\begin{sphinxVerbatimOutput}

\begin{sphinxuseclass}{cell_output}
\noindent\sphinxincludegraphics{{48048c6419a01b5c71118c86e1f3e5d726e20c1d97c3dd5e122ffcfe4f7465b4}.png}

\end{sphinxuseclass}\end{sphinxVerbatimOutput}

\end{sphinxuseclass}

\subsection{The Focus2all=True  option  adds values to each node}
\label{\detokenize{content/06_ModelAnalytics/ModelStructure:the-focus2all-true-option-adds-values-to-each-node}}
\begin{sphinxuseclass}{cell}\begin{sphinxVerbatimInput}

\begin{sphinxuseclass}{cell_input}
\begin{sphinxVerbatim}[commandchars=\\\{\}]
\PYG{k}{with} \PYG{n}{mpak}\PYG{o}{.}\PYG{n}{set\PYGZus{}smpl}\PYG{p}{(}\PYG{l+m+mi}{2025}\PYG{p}{,}\PYG{l+m+mi}{2027}\PYG{p}{)}\PYG{p}{:}
    \PYG{n}{mpak}\PYG{o}{.}\PYG{n}{PAKNYGDPMKTPKN}\PYG{o}{.}\PYG{n}{tracepre}\PYG{p}{(}\PYG{n+nb}{filter} \PYG{o}{=} \PYG{l+m+mi}{20}\PYG{p}{,}\PYG{n}{fokus2all}\PYG{o}{=}\PYG{k+kc}{True}\PYG{p}{,}\PYG{n}{png}\PYG{o}{=}\PYG{n}{latex}\PYG{p}{)}
\end{sphinxVerbatim}

\end{sphinxuseclass}\end{sphinxVerbatimInput}
\begin{sphinxVerbatimOutput}

\begin{sphinxuseclass}{cell_output}
\noindent\sphinxincludegraphics{{8396256109d6e3b960c0a44a1f1299e33067eda08f4a2aa6aa942799857dd10a}.png}

\end{sphinxuseclass}\end{sphinxVerbatimOutput}

\end{sphinxuseclass}

\subsection{attshow=True and growthshow=True adds additional information to each node}
\label{\detokenize{content/06_ModelAnalytics/ModelStructure:attshow-true-and-growthshow-true-adds-additional-information-to-each-node}}
\begin{sphinxuseclass}{cell}\begin{sphinxVerbatimInput}

\begin{sphinxuseclass}{cell_input}
\begin{sphinxVerbatim}[commandchars=\\\{\}]
\PYG{k}{with} \PYG{n}{mpak}\PYG{o}{.}\PYG{n}{set\PYGZus{}smpl}\PYG{p}{(}\PYG{l+m+mi}{2025}\PYG{p}{,}\PYG{l+m+mi}{2027}\PYG{p}{)}\PYG{p}{:}
    \PYG{n}{mpak}\PYG{o}{.}\PYG{n}{PAKNYGDPMKTPKN}\PYG{o}{.}\PYG{n}{tracepre}\PYG{p}{(}\PYG{n+nb}{filter} \PYG{o}{=} \PYG{l+m+mi}{20}\PYG{p}{,}                                 
                                 \PYG{n}{fokus2all}\PYG{o}{=}\PYG{k+kc}{True}\PYG{p}{,}
                                 \PYG{n}{attshow}\PYG{o}{=}\PYG{k+kc}{True}\PYG{p}{,}
                                 \PYG{n}{growthshow} \PYG{o}{=} \PYG{k+kc}{True}\PYG{p}{,}
                                \PYG{n}{png}\PYG{o}{=}\PYG{n}{latex}\PYG{p}{,}\PYG{n}{size}\PYG{o}{=}\PYG{p}{(}\PYG{l+m+mi}{3}\PYG{p}{,}\PYG{l+m+mi}{3}\PYG{p}{)}\PYG{p}{)}
\end{sphinxVerbatim}

\end{sphinxuseclass}\end{sphinxVerbatimInput}
\begin{sphinxVerbatimOutput}

\begin{sphinxuseclass}{cell_output}
\noindent\sphinxincludegraphics{{3daa5dd10c68ea63bc89482fa20f33747e704255e2c04a22c58c7a6da7745c2d}.png}

\end{sphinxuseclass}\end{sphinxVerbatimOutput}

\end{sphinxuseclass}

\subsection{fokus2=’variables …’ allows narrowing the variables to display}
\label{\detokenize{content/06_ModelAnalytics/ModelStructure:fokus2-variables-allows-narrowing-the-variables-to-display}}
\begin{sphinxuseclass}{cell}\begin{sphinxVerbatimInput}

\begin{sphinxuseclass}{cell_input}
\begin{sphinxVerbatim}[commandchars=\\\{\}]
\PYG{k}{with} \PYG{n}{mpak}\PYG{o}{.}\PYG{n}{set\PYGZus{}smpl}\PYG{p}{(}\PYG{l+m+mi}{2025}\PYG{p}{,}\PYG{l+m+mi}{2027}\PYG{p}{)}\PYG{p}{:}
    \PYG{n}{mpak}\PYG{o}{.}\PYG{n}{PAKNYGDPMKTPKN}\PYG{o}{.}\PYG{n}{tracepre}\PYG{p}{(}\PYG{n+nb}{filter} \PYG{o}{=} \PYG{l+m+mi}{20}\PYG{p}{,}                                 
                                 \PYG{n}{fokus2}\PYG{o}{=}\PYG{l+s+s1}{\PYGZsq{}}\PYG{l+s+s1}{PAKNEGDIFTOTKN PAKNECONPRVTKN}\PYG{l+s+s1}{\PYGZsq{}}\PYG{p}{,}
                                \PYG{n}{png}\PYG{o}{=}\PYG{l+m+mi}{1}\PYG{p}{,}\PYG{n}{svg}\PYG{o}{=}\PYG{l+m+mi}{1}\PYG{p}{,}\PYG{n}{pdf}\PYG{o}{=}\PYG{l+m+mi}{1}\PYG{p}{,}\PYG{n}{attshow}\PYG{o}{=}\PYG{l+m+mi}{1}\PYG{p}{,}\PYG{n}{size}\PYG{o}{=}\PYG{p}{(}\PYG{l+m+mi}{4}\PYG{p}{,}\PYG{l+m+mi}{4}\PYG{p}{)}\PYG{p}{)}
\end{sphinxVerbatim}

\end{sphinxuseclass}\end{sphinxVerbatimInput}
\begin{sphinxVerbatimOutput}

\begin{sphinxuseclass}{cell_output}
\noindent\sphinxincludegraphics{{5678e12d81bb2f9ab59fae3cf606a095b35f84640412d950bb70ff344c1af1bb}.png}

\end{sphinxuseclass}\end{sphinxVerbatimOutput}

\end{sphinxuseclass}
\sphinxAtStartPar
In addition:
\begin{itemize}
\item {} 
\sphinxAtStartPar
\sphinxcode{\sphinxupquote{svg = True}} will output the graph into a svg file in the graph\textbackslash{} subfolder

\item {} 
\sphinxAtStartPar
\sphinxcode{\sphinxupquote{pdf = True}} will output the graph into a pdf file in the graph\textbackslash{} subfolder

\end{itemize}

\sphinxAtStartPar
Both these formats have the benefits of allowing zooming without becoming coarse. This is useful if the the output
becomes busy.

\sphinxAtStartPar
Also the svg format allows hovering with the mouse.


\section{Post shock \sphinxstyleliteralintitle{\sphinxupquote{.tracedep}} traces the impact of a variable on other variables}
\label{\detokenize{content/06_ModelAnalytics/ModelStructure:post-shock-tracedep-traces-the-impact-of-a-variable-on-other-variables}}
\sphinxAtStartPar
The same options can be used as for \sphinxcode{\sphinxupquote{.tracepre}}

\begin{sphinxuseclass}{cell}\begin{sphinxVerbatimInput}

\begin{sphinxuseclass}{cell_input}
\begin{sphinxVerbatim}[commandchars=\\\{\}]
\PYG{n}{mpak}\PYG{o}{.}\PYG{n}{PAKNECONPRVTKN}\PYG{o}{.}\PYG{n}{tracedep}\PYG{p}{(}\PYG{n}{down}\PYG{o}{=}\PYG{l+m+mi}{3}\PYG{p}{,}\PYG{n+nb}{filter}\PYG{o}{=}\PYG{l+m+mi}{20}\PYG{p}{)}
\end{sphinxVerbatim}

\end{sphinxuseclass}\end{sphinxVerbatimInput}
\begin{sphinxVerbatimOutput}

\begin{sphinxuseclass}{cell_output}
\begin{sphinxVerbatim}[commandchars=\\\{\}]
\PYGZlt{}IPython.core.display.SVG object\PYGZgt{}
\end{sphinxVerbatim}

\end{sphinxuseclass}\end{sphinxVerbatimOutput}

\end{sphinxuseclass}

\chapter{\sphinxstyleliteralintitle{\sphinxupquote{.modeldash()}} An interactive way to explore dependencies and attributions}
\label{\detokenize{content/06_ModelAnalytics/ModelStructure:modeldash-an-interactive-way-to-explore-dependencies-and-attributions}}
\sphinxAtStartPar
The \sphinxcode{\sphinxupquote{.modeldash()}} method (when executed in a Jupyter Notebook) generates a widget that allows you to dynamicaly adjust the arguments to the \sphinxcode{\sphinxupquote{tracepre()}} and \sphinxcode{\sphinxupquote{tracedep}} functions.

\begin{sphinxVerbatim}[commandchars=\\\{\}]
 \PYG{k}{with} \PYG{n}{mpak}\PYG{o}{.}\PYG{n}{set\PYGZus{}smpl}\PYG{p}{(}\PYG{l+m+mi}{2022}\PYG{p}{,}\PYG{l+m+mi}{2026}\PYG{p}{)}\PYG{p}{:}
        \PYG{n}{mpak}\PYG{o}{.}\PYG{n}{modeldash}\PYG{p}{(}\PYG{l+s+s1}{\PYGZsq{}}\PYG{l+s+s1}{PAKNYGDPMKTPKN}\PYG{l+s+s1}{\PYGZsq{}}\PYG{p}{,}\PYG{n}{jupyter}\PYG{o}{=}\PYG{k+kc}{True}\PYG{p}{,}\PYG{n}{inline}\PYG{o}{=}\PYG{k+kc}{False}\PYG{p}{)} 
\end{sphinxVerbatim}

\sphinxAtStartPar
The above commands generate a dashboard that looks a like the below, where the panel to the left allows the user to change options including the filter, the depth of the trace among other things.

\begin{figure}[htbp]
\centering
\capstart

\noindent\sphinxincludegraphics[width=0.800\linewidth]{{dash}.png}
\caption{The \sphinxstylestrong{Modeldash} method!}\label{\detokenize{content/06_ModelAnalytics/ModelStructure:dash}}\end{figure}

\sphinxstepscope


\chapter{Decomposing/attributing the impact of a shock}
\label{\detokenize{content/06_ModelAnalytics/Attribution:decomposing-attributing-the-impact-of-a-shock}}\label{\detokenize{content/06_ModelAnalytics/Attribution::doc}}
\sphinxAtStartPar
When working with a model it is often useful to have a better sense of the contribution of different channels to a final result.  For example, an increase in interest rates will tend to reduce investment and consumer demand – contributing to a reduction in GDP. At the same time, lower inflation as the higher interest rate takes effect will tend to work in the opposite direction.

\sphinxAtStartPar
The \sphinxcode{\sphinxupquote{tracedep()}} and \sphinxcode{\sphinxupquote{tracepre()}} methods introduced in the previous section give a sense of impacts, but the \sphinxcode{\sphinxupquote{modelflow}} methods \sphinxcode{\sphinxupquote{.dekomp()}} and \sphinxcode{\sphinxupquote{.totdif()}} take that one step further by calculating the  contribution of each channel to the overall result.


\section{Prepare the workspace}
\label{\detokenize{content/06_ModelAnalytics/Attribution:prepare-the-workspace}}
\sphinxAtStartPar
As always before running \sphinxcode{\sphinxupquote{modelflow}} the python environment needs to be initialized and libraries to be used imported.

\begin{sphinxuseclass}{cell}\begin{sphinxVerbatimInput}

\begin{sphinxuseclass}{cell_input}
\begin{sphinxVerbatim}[commandchars=\\\{\}]
\PYG{k+kn}{import} \PYG{n+nn}{pandas} \PYG{k}{as} \PYG{n+nn}{pd}

\PYG{c+c1}{\PYGZsh{} Modules from Modelflow }
\PYG{k+kn}{from} \PYG{n+nn}{modelclass} \PYG{k+kn}{import} \PYG{n}{model} 

\PYG{c+c1}{\PYGZsh{} optional functionalities }
\PYG{n}{model}\PYG{o}{.}\PYG{n}{widescreen}\PYG{p}{(}\PYG{p}{)}
\PYG{n}{model}\PYG{o}{.}\PYG{n}{scroll\PYGZus{}off}\PYG{p}{(}\PYG{p}{)}

\PYG{c+c1}{\PYGZsh{} Output compatabiltity with LaTeX }
\PYG{n}{latex}\PYG{o}{=}\PYG{k+kc}{True}
\end{sphinxVerbatim}

\end{sphinxuseclass}\end{sphinxVerbatimInput}
\begin{sphinxVerbatimOutput}

\begin{sphinxuseclass}{cell_output}
\begin{sphinxVerbatim}[commandchars=\\\{\}]
\PYGZlt{}IPython.core.display.HTML object\PYGZgt{}
\end{sphinxVerbatim}

\end{sphinxuseclass}\end{sphinxVerbatimOutput}

\end{sphinxuseclass}

\subsection{Load the pre\sphinxhyphen{}existing model, data and descriptions}
\label{\detokenize{content/06_ModelAnalytics/Attribution:load-the-pre-existing-model-data-and-descriptions}}
\sphinxAtStartPar
The file \sphinxcode{\sphinxupquote{pak.pcim}} contains a dump of model equations, dataframe, simulation options and variable descriptions:
\begin{itemize}
\item {} 
\sphinxAtStartPar
Loads model and simulates to establish a baseline.

\item {} 
\sphinxAtStartPar
Creates a dataframe with a tax rate of 29 USD/Ton for carbon emission for 3 sectors.

\item {} 
\sphinxAtStartPar
Simulates the new experiment.

\end{itemize}

\begin{sphinxuseclass}{cell}\begin{sphinxVerbatimInput}

\begin{sphinxuseclass}{cell_input}
\begin{sphinxVerbatim}[commandchars=\\\{\}]
\PYG{n}{mpak}\PYG{p}{,}\PYG{n}{baseline} \PYG{o}{=} \PYG{n}{model}\PYG{o}{.}\PYG{n}{modelload}\PYG{p}{(}\PYG{l+s+s1}{\PYGZsq{}}\PYG{l+s+s1}{../models/pak.pcim}\PYG{l+s+s1}{\PYGZsq{}}\PYG{p}{,}\PYG{n}{alfa}\PYG{o}{=}\PYG{l+m+mf}{0.7}\PYG{p}{,}\PYG{n}{run}\PYG{o}{=}\PYG{l+m+mi}{1}\PYG{p}{,}\PYG{n}{keep}\PYG{o}{=}\PYG{l+s+s1}{\PYGZsq{}}\PYG{l+s+s1}{Business as Usual}\PYG{l+s+s1}{\PYGZsq{}}\PYG{p}{)}
\PYG{n}{alternative}  \PYG{o}{=}  \PYG{n}{baseline}\PYG{o}{.}\PYG{n}{upd}\PYG{p}{(}\PYG{l+s+s2}{\PYGZdq{}}\PYG{l+s+s2}{\PYGZlt{}2020 2100\PYGZgt{} PAKGGREVCO2CER PAKGGREVCO2GER PAKGGREVCO2OER = 30}\PYG{l+s+s2}{\PYGZdq{}}\PYG{p}{)}
\PYG{n}{result} \PYG{o}{=} \PYG{n}{mpak}\PYG{p}{(}\PYG{n}{alternative}\PYG{p}{,}\PYG{l+m+mi}{2020}\PYG{p}{,}\PYG{l+m+mi}{2100}\PYG{p}{,}\PYG{n}{keep}\PYG{o}{=}\PYG{l+s+s1}{\PYGZsq{}}\PYG{l+s+s1}{Carbon tax nominal 30}\PYG{l+s+s1}{\PYGZsq{}}\PYG{p}{)} \PYG{c+c1}{\PYGZsh{} simulates the model }
\end{sphinxVerbatim}

\end{sphinxuseclass}\end{sphinxVerbatimInput}
\begin{sphinxVerbatimOutput}

\begin{sphinxuseclass}{cell_output}
\begin{sphinxVerbatim}[commandchars=\\\{\}]
file read:  C:\PYGZbs{}modelflow manual\PYGZbs{}papers\PYGZbs{}mfbook\PYGZbs{}content\PYGZbs{}models\PYGZbs{}pak.pcim
\end{sphinxVerbatim}

\end{sphinxuseclass}\end{sphinxVerbatimOutput}

\end{sphinxuseclass}

\section{The mathematics of attribution}
\label{\detokenize{content/06_ModelAnalytics/Attribution:the-mathematics-of-attribution}}
\sphinxAtStartPar
At its root the idea of attribution is simply taking the total derivative of the model to identify the sensitivity of the equation we are interested in to changes elsewhere in the model and then combine that with the changes in other variables.

\sphinxAtStartPar
Take a variable y that is a function of two other variables a and b.  In the model the relationship might be written as:

\sphinxAtStartPar
\(y = f(a,b)\)

\sphinxAtStartPar
If there are two observations
\label{equation:content/06_ModelAnalytics/Attribution:d704f437-36a8-4771-9c7d-deb693d4baca}\begin{eqnarray}
y_0 = f(a_0,b_0)\\
y_1 = f(a_1,b_1)
\end{eqnarray}
\sphinxAtStartPar
then we also have the change in all three variables \(\Delta y, \Delta a, \Delta b\) and the total derivative of y can be written as:

\sphinxAtStartPar
\(\Delta y = \underbrace{\Delta a \dfrac{\partial {f}}{\partial{a}}(a,b)}_{\Omega a} + 
\underbrace{\Delta b \dfrac{\partial {f}}{\partial{b}}(a,b)}_{\Omega b}+Residual\)

\sphinxAtStartPar
The first expresion can be called \(\Omega_a\) or the contribution of changes in a to changes in y, and the second \(\Omega_b\),  or the contribution of changes in b to changes in y.

\sphinxAtStartPar
\sphinxcode{\sphinxupquote{Modelflow}} performs a numerical approximation of \(\Omega_a\) and \(\Omega_b\) by performing two runs of the \(f()\):
\label{equation:content/06_ModelAnalytics/Attribution:675e70b5-c8ce-4a2f-ba07-1d815e73907d}\begin{eqnarray}  
y_0&=&f(a_{0},b_{0}) \\
y_1&=&f(a_0+\Delta a,b_{0}+ \Delta b)
\end{eqnarray}
\sphinxAtStartPar
and calculates \(\Omega_a\) and \(\Omega_b\) as:
\label{equation:content/06_ModelAnalytics/Attribution:f732d6f2-2e26-4d9d-9e61-ff51d60c34ef}\begin{eqnarray}  
\Omega a&=&f(a_1,b_1 )-f(a_1-\Delta a,b_1) \\
\Omega b&=&f(a_1,b_1 )-f(a_1,b_1-\Delta  b)
\end{eqnarray}
\sphinxAtStartPar
And:
\label{equation:content/06_ModelAnalytics/Attribution:c88de322-9fdb-4e40-bc5e-043ffe657777}\begin{eqnarray}
residual = \Omega a + \Omega b -(y_1 - y_0) 
\end{eqnarray}
\sphinxAtStartPar
If the model is fairly linear, the residual will be small.


\section{Model attribution or  single equation attribution?}
\label{\detokenize{content/06_ModelAnalytics/Attribution:model-attribution-or-single-equation-attribution}}
\sphinxAtStartPar
Above the relationship between y, a, and b was summarized by the function f().

\sphinxAtStartPar
\(f(a,b)\) could represent \sphinxstylestrong{a single equation} in the model or it could represent \sphinxstylestrong{the entire model}.

\sphinxAtStartPar
In the a \sphinxstylestrong{single equation} instance, \(\Delta a\) and \(\Delta b\) would be treated as exogenous variables in the attribution calculation as they are both on the right hand side of the equation. It does not matter if \(a\) and \(b\) are endogenous of exogenous variables in the complete model.

\sphinxAtStartPar
In the the \sphinxstylestrong{entire model} instance \(a\) and \(b\) are exogenous variables in the model, in addition it only makes sense to look at the exogenous variables which has changed between the experiments. That is \(\Delta a \neq 0\) and \(\Delta b \neq 0\)

\sphinxAtStartPar
Assume the simple equation example such that  \(a\) and \(b\) are simple variables. When \(\Delta y\), \(\Delta a\) and \(\Delta b\) reflect the difference across scenarios (say the value of the three variables in \sphinxcode{\sphinxupquote{.lastdf}} less the value in \sphinxcode{\sphinxupquote{.basedf}} then;

\sphinxAtStartPar
\(\Omega_a\), \(\Omega_b\) are the absolute contribution of a and b to the change in y, and
\(100*\bigg[\cfrac{\Omega_a}{\Delta y}\bigg]\)  \(100*\bigg[\cfrac{\Omega_b}{\Delta y}\bigg]\) are the share of the change in y explained by a and b respectively.

\sphinxAtStartPar
If \(\Delta y\), \(\Delta a\) and \(\Delta b\) are the changes over time (\(\Delta y_t=y_t-y_{t-1}\)), then \(\Omega_a\), \(\Omega_b\) are the contributions of a and b to the rate of growth of y, while \(100*\bigg[\cfrac{\Omega_a}{\Delta y_{t-1}}\bigg]\)  \(100*\bigg[\cfrac{\Omega_b}{\Delta y_{t-1}}\bigg]\) are are the contributions of a and b to the rate of growth of y.


\section{Decomposing the changes in a single endogenous variable \sphinxhyphen{} formula attribution}
\label{\detokenize{content/06_ModelAnalytics/Attribution:decomposing-the-changes-in-a-single-endogenous-variable-formula-attribution}}
\sphinxAtStartPar
The \sphinxcode{\sphinxupquote{modelflow}} method \sphinxcode{\sphinxupquote{.dekomp()}} isd used to calculate the contribution of different RHS variables to the change in an endogenous variable. Moreover the function is
to create input to more high level functions used for visualizing  the output from   \sphinxcode{\sphinxupquote{.dekomp()}}

\sphinxAtStartPar
This method utilizes that each model object always stores the initial and most recent simulation result in two dataframes called \sphinxcode{\sphinxupquote{.basedf}} and \sphinxcode{\sphinxupquote{.lastdf}}. The model object also all contains the equations of the model.


\subsection{The output of the \sphinxstyleliteralintitle{\sphinxupquote{dekomp()}} method}
\label{\detokenize{content/06_ModelAnalytics/Attribution:the-output-of-the-dekomp-method}}
\sphinxAtStartPar
The \sphinxcode{\sphinxupquote{dekomp()}} method calculates the contribution to changes in the level of the dependent variables in a given equation. It does not calculate the contributions of variables that determine the changes observed in the RHS variables themselves.

\sphinxAtStartPar
In the example below the contribution to the change in Total emissions is decomposed into the contribution from each of three sources in the model, the consumption of Crude Oil, Natural Gas and Coal.  As the equation for total emissions is just the sum of the three this is a fairly trivial decomposition, but it provides an easily understood illustration of the process at work.

\sphinxAtStartPar
The results of the \sphinxcode{\sphinxupquote{.dekomp}} command are divided into 4 separate tables.
\begin{enumerate}
\sphinxsetlistlabels{\arabic}{enumi}{enumii}{}{.}%
\item {} 
\sphinxAtStartPar
The first table of output shows the changes that are to be explained/ \sphinxstylestrong{The changes are always drawn from the most solution, i.e. from the \sphinxcode{\sphinxupquote{.basedf}} and \sphinxcode{\sphinxupquote{.lastdf}} DataFrames} of the model object.

\item {} 
\sphinxAtStartPar
The seond table shows the changes between the contributions of the LHS variables to the changes in the RHS variable.  Because this equation is an additive identity, these amount to the changes in each of the variables themselves.

\item {} 
\sphinxAtStartPar
the third table expresses these changes as a share of the total change.

\item {} 
\sphinxAtStartPar
The last shows the contributions of these changes to the change in the growth rate of the deopendent variable (these results would need to be multiplied by 100 to see that they add to the totals in table 1.

\end{enumerate}

\begin{sphinxuseclass}{cell}\begin{sphinxVerbatimInput}

\begin{sphinxuseclass}{cell_input}
\begin{sphinxVerbatim}[commandchars=\\\{\}]
\PYG{n}{mpak}\PYG{p}{[}\PYG{l+s+s1}{\PYGZsq{}}\PYG{l+s+s1}{PAKCCEMISCO2TKN}\PYG{l+s+s1}{\PYGZsq{}}\PYG{p}{]}\PYG{o}{.}\PYG{n}{frml}
\end{sphinxVerbatim}

\end{sphinxuseclass}\end{sphinxVerbatimInput}
\begin{sphinxVerbatimOutput}

\begin{sphinxuseclass}{cell_output}
\begin{sphinxVerbatim}[commandchars=\\\{\}]
PAKCCEMISCO2TKN : FRML \PYGZlt{}IDENT\PYGZgt{} PAKCCEMISCO2TKN = PAKCCEMISCO2CKN+PAKCCEMISCO2OKN+PAKCCEMISCO2GKN \PYGZdl{}
\end{sphinxVerbatim}

\end{sphinxuseclass}\end{sphinxVerbatimOutput}

\end{sphinxuseclass}
\sphinxAtStartPar
The results of the \sphinxcode{\sphinxupquote{.dekomp}} command are divided into 4 separate tables.
\begin{enumerate}
\sphinxsetlistlabels{\arabic}{enumi}{enumii}{}{.}%
\item {} 
\sphinxAtStartPar
The first table of output shows the changes that are to be explained/ \sphinxstylestrong{The changes are always drawn from the most solution, i.e. from the \sphinxcode{\sphinxupquote{.basedf}} and \sphinxcode{\sphinxupquote{.lastdf}} DataFrames} of the model object.

\item {} 
\sphinxAtStartPar
The seond table shows the changes between the contributions of the LHS variables to the changes in the RHS variable.  Because this equation is an additive identity, these amount to the changes in each of the variables themselves.

\item {} 
\sphinxAtStartPar
the third table expresses these changes as a share of the total change.

\item {} 
\sphinxAtStartPar
The last shows the contributions of these changes to the change in the growth rate of the deopendent variable (these results would need to be multiplied by 100 to see that they add to the totals in table 1.

\end{enumerate}

\begin{sphinxuseclass}{cell}\begin{sphinxVerbatimInput}

\begin{sphinxuseclass}{cell_input}
\begin{sphinxVerbatim}[commandchars=\\\{\}]
\PYG{k}{with} \PYG{n}{mpak}\PYG{o}{.}\PYG{n}{set\PYGZus{}smpl}\PYG{p}{(}\PYG{l+m+mi}{2021}\PYG{p}{,}\PYG{l+m+mi}{2025}\PYG{p}{)}\PYG{p}{:}
    \PYG{n}{mpak}\PYG{p}{[}\PYG{l+s+s1}{\PYGZsq{}}\PYG{l+s+s1}{PAKCCEMISCO2TKN}\PYG{l+s+s1}{\PYGZsq{}}\PYG{p}{]}\PYG{o}{.}\PYG{n}{dekomp}\PYG{p}{(}\PYG{p}{)}
\end{sphinxVerbatim}

\end{sphinxuseclass}\end{sphinxVerbatimInput}
\begin{sphinxVerbatimOutput}

\begin{sphinxuseclass}{cell_output}
\begin{sphinxVerbatim}[commandchars=\\\{\}]
Formula        : FRML \PYGZlt{}IDENT\PYGZgt{} PAKCCEMISCO2TKN = PAKCCEMISCO2CKN+PAKCCEMISCO2OKN+PAKCCEMISCO2GKN \PYGZdl{} 

                        2021         2022         2023         2024         2025
Variable    lag                                                                 
Base        0   217548186.56 221072469.97 225253519.79 230370294.27 236240658.85
Alternative 0   158136496.56 162262040.05 167116569.38 173010036.73 179772129.03
Difference  0   \PYGZhy{}59411690.00 \PYGZhy{}58810429.92 \PYGZhy{}58136950.40 \PYGZhy{}57360257.55 \PYGZhy{}56468529.82
Percent     0         \PYGZhy{}27.31       \PYGZhy{}26.60       \PYGZhy{}25.81       \PYGZhy{}24.90       \PYGZhy{}23.90

 Contributions to differende for  PAKCCEMISCO2TKN
                            2021         2022         2023         2024         2025
Variable        lag                                                                 
PAKCCEMISCO2CKN 0   \PYGZhy{}24251583.97 \PYGZhy{}24147394.08 \PYGZhy{}23999340.06 \PYGZhy{}23829476.98 \PYGZhy{}23631866.29
PAKCCEMISCO2OKN 0   \PYGZhy{}14968093.58 \PYGZhy{}15456795.05 \PYGZhy{}15753667.05 \PYGZhy{}15847991.86 \PYGZhy{}15778083.81
PAKCCEMISCO2GKN 0   \PYGZhy{}20192012.46 \PYGZhy{}19206240.79 \PYGZhy{}18383943.30 \PYGZhy{}17682788.70 \PYGZhy{}17058579.71

 Share of contributions to differende for  PAKCCEMISCO2TKN
                           2021        2022        2023        2024        2025
Variable        lag                                                            
PAKCCEMISCO2CKN 0           41\PYGZpc{}         41\PYGZpc{}         41\PYGZpc{}         42\PYGZpc{}         42\PYGZpc{}
PAKCCEMISCO2GKN 0           34\PYGZpc{}         33\PYGZpc{}         32\PYGZpc{}         31\PYGZpc{}         30\PYGZpc{}
PAKCCEMISCO2OKN 0           25\PYGZpc{}         26\PYGZpc{}         27\PYGZpc{}         28\PYGZpc{}         28\PYGZpc{}
Total           0          100\PYGZpc{}        100\PYGZpc{}        100\PYGZpc{}        100\PYGZpc{}        100\PYGZpc{}
Residual        0            0\PYGZpc{}          0\PYGZpc{}          0\PYGZpc{}          0\PYGZpc{}         \PYGZhy{}0\PYGZpc{}

 Contribution to growth rate PAKCCEMISCO2TKN
                           2021        2022        2023        2024        2025
Variable        lag                                                            
PAKCCEMISCO2CKN 0         \PYGZhy{}0.2\PYGZpc{}       \PYGZhy{}0.2\PYGZpc{}       \PYGZhy{}0.1\PYGZpc{}       \PYGZhy{}0.1\PYGZpc{}       \PYGZhy{}0.1\PYGZpc{}
PAKCCEMISCO2OKN 0         \PYGZhy{}0.1\PYGZpc{}       \PYGZhy{}0.1\PYGZpc{}       \PYGZhy{}0.1\PYGZpc{}       \PYGZhy{}0.1\PYGZpc{}       \PYGZhy{}0.1\PYGZpc{}
PAKCCEMISCO2GKN 0         \PYGZhy{}0.1\PYGZpc{}       \PYGZhy{}0.1\PYGZpc{}       \PYGZhy{}0.1\PYGZpc{}       \PYGZhy{}0.1\PYGZpc{}       \PYGZhy{}0.1\PYGZpc{}
\end{sphinxVerbatim}

\end{sphinxuseclass}\end{sphinxVerbatimOutput}

\end{sphinxuseclass}
\sphinxAtStartPar
The following single\sphinxhyphen{}equation decomposition looks to the impact on inflation.  The inflation equation is more complex and has more direct causal variables, so here the decomposition is more helpful.

\sphinxAtStartPar
Recall the inflation equation is given by the \sphinxcode{\sphinxupquote{.frml}} method for its normalized version and \sphinxcode{\sphinxupquote{.eviews}} for its original specification.  The equation for the consumer price level (PAKNECONPRVTXN) was originally specified in eviews as:

\begin{sphinxuseclass}{cell}\begin{sphinxVerbatimInput}

\begin{sphinxuseclass}{cell_input}
\begin{sphinxVerbatim}[commandchars=\\\{\}]
\PYG{n}{mpak}\PYG{p}{[}\PYG{l+s+s1}{\PYGZsq{}}\PYG{l+s+s1}{PAKNECONPRVTXN}\PYG{l+s+s1}{\PYGZsq{}}\PYG{p}{]}\PYG{o}{.}\PYG{n}{eviews}
\end{sphinxVerbatim}

\end{sphinxuseclass}\end{sphinxVerbatimInput}
\begin{sphinxVerbatimOutput}

\begin{sphinxuseclass}{cell_output}
\begin{sphinxVerbatim}[commandchars=\\\{\}]
PAKNECONPRVTXN : @IDENTITY PAKNECONPRVTXN  = ((PAKNECONENGYSH\PYGZca{}PAKCESENGYCON)  * PAKNECONENGYXN\PYGZca{}(1  \PYGZhy{} PAKCESENGYCON)  + (PAKNECONOTHRSH\PYGZca{}PAKCESENGYCON)  * PAKNECONOTHRXN\PYGZca{}(1  \PYGZhy{} PAKCESENGYCON))\PYGZca{}(1  / (1  \PYGZhy{} PAKCESENGYCON))
\end{sphinxVerbatim}

\end{sphinxuseclass}\end{sphinxVerbatimOutput}

\end{sphinxuseclass}
\sphinxAtStartPar
When normalized the equation solves for the \sphinxstylestrong{level} of the price deflator.  It is this normalized equation that is:

\begin{sphinxuseclass}{cell}\begin{sphinxVerbatimInput}

\begin{sphinxuseclass}{cell_input}
\begin{sphinxVerbatim}[commandchars=\\\{\}]
\PYG{n}{mpak}\PYG{p}{[}\PYG{l+s+s1}{\PYGZsq{}}\PYG{l+s+s1}{PAKNECONPRVTXN}\PYG{l+s+s1}{\PYGZsq{}}\PYG{p}{]}\PYG{o}{.}\PYG{n}{frml}
\end{sphinxVerbatim}

\end{sphinxuseclass}\end{sphinxVerbatimInput}
\begin{sphinxVerbatimOutput}

\begin{sphinxuseclass}{cell_output}
\begin{sphinxVerbatim}[commandchars=\\\{\}]
PAKNECONPRVTXN : FRML \PYGZlt{}IDENT\PYGZgt{} PAKNECONPRVTXN = ((PAKNECONENGYSH**PAKCESENGYCON)*PAKNECONENGYXN**(1\PYGZhy{}PAKCESENGYCON)+(PAKNECONOTHRSH**PAKCESENGYCON)*PAKNECONOTHRXN**(1\PYGZhy{}PAKCESENGYCON))**(1/(1\PYGZhy{}PAKCESENGYCON)) \PYGZdl{}
\end{sphinxVerbatim}

\end{sphinxuseclass}\end{sphinxVerbatimOutput}

\end{sphinxuseclass}
\sphinxAtStartPar
Because the normalized equation solves for the level of the price deflator, te decomposition will show the contributions of each explanatory variable to the increase in the price level (not that of the inflation rate).

\sphinxAtStartPar
Note in the Pakistan model, consumer inflation is derived as a CET aggregation of the price of energy goods(PAKNECONENGYXN) and non\sphinxhyphen{}energy goods (PAKNECONOTHRXN).

\begin{sphinxuseclass}{cell}\begin{sphinxVerbatimInput}

\begin{sphinxuseclass}{cell_input}
\begin{sphinxVerbatim}[commandchars=\\\{\}]
\PYG{k}{with} \PYG{n}{mpak}\PYG{o}{.}\PYG{n}{set\PYGZus{}smpl}\PYG{p}{(}\PYG{l+m+mi}{2021}\PYG{p}{,}\PYG{l+m+mi}{2025}\PYG{p}{)}\PYG{p}{:}
    \PYG{n}{mpak}\PYG{p}{[}\PYG{l+s+s1}{\PYGZsq{}}\PYG{l+s+s1}{PAKNECONPRVTXN}\PYG{l+s+s1}{\PYGZsq{}}\PYG{p}{]}\PYG{o}{.}\PYG{n}{dekomp}\PYG{p}{(}\PYG{p}{)}
\end{sphinxVerbatim}

\end{sphinxuseclass}\end{sphinxVerbatimInput}
\begin{sphinxVerbatimOutput}

\begin{sphinxuseclass}{cell_output}
\begin{sphinxVerbatim}[commandchars=\\\{\}]
Formula        : FRML \PYGZlt{}IDENT\PYGZgt{} PAKNECONPRVTXN = ((PAKNECONENGYSH**PAKCESENGYCON)*PAKNECONENGYXN**(1\PYGZhy{}PAKCESENGYCON)+(PAKNECONOTHRSH**PAKCESENGYCON)*PAKNECONOTHRXN**(1\PYGZhy{}PAKCESENGYCON))**(1/(1\PYGZhy{}PAKCESENGYCON)) \PYGZdl{} 

                      2021       2022       2023       2024       2025
Variable    lag                                                       
Base        0         1.82       1.98       2.14       2.30       2.45
Alternative 0         1.89       2.07       2.23       2.39       2.55
Difference  0         0.07       0.08       0.09       0.10       0.10
Percent     0         3.84       4.11       4.21       4.18       4.06

 Contributions to differende for  PAKNECONPRVTXN
                         2021       2022       2023       2024       2025
Variable       lag                                                       
PAKNECONENGYSH 0        \PYGZhy{}0.00      \PYGZhy{}0.00      \PYGZhy{}0.00      \PYGZhy{}0.00      \PYGZhy{}0.00
PAKCESENGYCON  0        \PYGZhy{}0.00      \PYGZhy{}0.00      \PYGZhy{}0.00      \PYGZhy{}0.00      \PYGZhy{}0.00
PAKNECONENGYXN 0         0.01       0.01       0.01       0.01       0.01
PAKNECONOTHRSH 0        \PYGZhy{}0.00      \PYGZhy{}0.00      \PYGZhy{}0.00      \PYGZhy{}0.00      \PYGZhy{}0.00
PAKNECONOTHRXN 0         0.06       0.07       0.08       0.08       0.09

 Share of contributions to differende for  PAKNECONPRVTXN
                          2021        2022        2023        2024        2025
Variable       lag                                                            
PAKNECONOTHRXN 0           81\PYGZpc{}         83\PYGZpc{}         85\PYGZpc{}         86\PYGZpc{}         86\PYGZpc{}
PAKNECONENGYXN 0           20\PYGZpc{}         17\PYGZpc{}         16\PYGZpc{}         15\PYGZpc{}         14\PYGZpc{}
PAKNECONENGYSH 0           \PYGZhy{}0\PYGZpc{}         \PYGZhy{}0\PYGZpc{}         \PYGZhy{}0\PYGZpc{}         \PYGZhy{}0\PYGZpc{}         \PYGZhy{}0\PYGZpc{}
PAKCESENGYCON  0           \PYGZhy{}0\PYGZpc{}         \PYGZhy{}0\PYGZpc{}         \PYGZhy{}0\PYGZpc{}         \PYGZhy{}0\PYGZpc{}         \PYGZhy{}0\PYGZpc{}
PAKNECONOTHRSH 0           \PYGZhy{}0\PYGZpc{}         \PYGZhy{}0\PYGZpc{}         \PYGZhy{}0\PYGZpc{}         \PYGZhy{}0\PYGZpc{}         \PYGZhy{}0\PYGZpc{}
Total          0          101\PYGZpc{}        101\PYGZpc{}        101\PYGZpc{}        101\PYGZpc{}        101\PYGZpc{}
Residual       0            1\PYGZpc{}          1\PYGZpc{}          1\PYGZpc{}          1\PYGZpc{}          1\PYGZpc{}

 Contribution to growth rate PAKNECONPRVTXN
                          2021        2022        2023        2024        2025
Variable       lag                                                            
PAKNECONENGYSH 0         \PYGZhy{}0.0\PYGZpc{}       \PYGZhy{}0.0\PYGZpc{}       \PYGZhy{}0.0\PYGZpc{}       \PYGZhy{}0.0\PYGZpc{}       \PYGZhy{}0.0\PYGZpc{}
PAKCESENGYCON  0         \PYGZhy{}0.0\PYGZpc{}       \PYGZhy{}0.0\PYGZpc{}       \PYGZhy{}0.0\PYGZpc{}       \PYGZhy{}0.0\PYGZpc{}       \PYGZhy{}0.0\PYGZpc{}
PAKNECONENGYXN 0          0.0\PYGZpc{}        0.0\PYGZpc{}        0.0\PYGZpc{}        0.0\PYGZpc{}        0.0\PYGZpc{}
PAKNECONOTHRSH 0         \PYGZhy{}0.0\PYGZpc{}       \PYGZhy{}0.0\PYGZpc{}       \PYGZhy{}0.0\PYGZpc{}       \PYGZhy{}0.0\PYGZpc{}       \PYGZhy{}0.0\PYGZpc{}
PAKNECONOTHRXN 0          0.0\PYGZpc{}        0.0\PYGZpc{}        0.0\PYGZpc{}        0.0\PYGZpc{}        0.0\PYGZpc{}
\end{sphinxVerbatim}

\end{sphinxuseclass}\end{sphinxVerbatimOutput}

\end{sphinxuseclass}
\sphinxAtStartPar
Interestingly only 25\% of the increase in the price level each period is due to the direct channel (the impact on energy prices), the bulk of the increase comes indirectly through other prices.  Indeed as time progresses this share rises from 75\% in the first year of the price change (2020) to 86\% by 2024.

\sphinxAtStartPar
Below is the formula for nonenergy consumer prices and its decomposition. This equation is written out as a more standard inflation equation reflecting changes in the cost of local goods production (PAKNYGDPFCSTXN), Government taxes on goods and services (PAKGGREVGNFSXN), the price of imports (PAKNEIMPGNGSXN) and the influence of the economic cycle (PAKNYGDPGAP\_).

\begin{sphinxuseclass}{cell}\begin{sphinxVerbatimInput}

\begin{sphinxuseclass}{cell_input}
\begin{sphinxVerbatim}[commandchars=\\\{\}]
\PYG{n}{mpak}\PYG{p}{[}\PYG{l+s+s1}{\PYGZsq{}}\PYG{l+s+s1}{PAKNECONOTHRXN}\PYG{l+s+s1}{\PYGZsq{}}\PYG{p}{]}\PYG{o}{.}\PYG{n}{eviews}
\end{sphinxVerbatim}

\end{sphinxuseclass}\end{sphinxVerbatimInput}
\begin{sphinxVerbatimOutput}

\begin{sphinxuseclass}{cell_output}
\begin{sphinxVerbatim}[commandchars=\\\{\}]
PAKNECONOTHRXN : DLOG(PAKNECONOTHRXN) = 0.590372627657176*DLOG(PAKNYGDPFCSTXN) + D(PAKGGREVGNFSXN/100) + (1 \PYGZhy{} 0.590372627657176)*DLOG(PAKNEIMPGNFSXN) + 0.2*PAKNYGDPGAP\PYGZus{}/100
\end{sphinxVerbatim}

\end{sphinxuseclass}\end{sphinxVerbatimOutput}

\end{sphinxuseclass}
\begin{sphinxuseclass}{cell}\begin{sphinxVerbatimInput}

\begin{sphinxuseclass}{cell_input}
\begin{sphinxVerbatim}[commandchars=\\\{\}]
\PYG{k}{with} \PYG{n}{mpak}\PYG{o}{.}\PYG{n}{set\PYGZus{}smpl}\PYG{p}{(}\PYG{l+m+mi}{2021}\PYG{p}{,}\PYG{l+m+mi}{2025}\PYG{p}{)}\PYG{p}{:}
    \PYG{n}{mpak}\PYG{p}{[}\PYG{l+s+s1}{\PYGZsq{}}\PYG{l+s+s1}{PAKNECONOTHRXN}\PYG{l+s+s1}{\PYGZsq{}}\PYG{p}{]}\PYG{o}{.}\PYG{n}{dekomp}\PYG{p}{(}\PYG{p}{)}
\end{sphinxVerbatim}

\end{sphinxuseclass}\end{sphinxVerbatimInput}
\begin{sphinxVerbatimOutput}

\begin{sphinxuseclass}{cell_output}
\begin{sphinxVerbatim}[commandchars=\\\{\}]
Formula        : FRML \PYGZlt{}DAMP,STOC\PYGZgt{} PAKNECONOTHRXN = (PAKNECONOTHRXN(\PYGZhy{}1)*EXP(PAKNECONOTHRXN\PYGZus{}A+ (0.590372627657176*((LOG(PAKNYGDPFCSTXN))\PYGZhy{}(LOG(PAKNYGDPFCSTXN(\PYGZhy{}1))))+((PAKGGREVGNFSXN/100)\PYGZhy{}(PAKGGREVGNFSXN(\PYGZhy{}1)/100))+(1\PYGZhy{}0.590372627657176)*((LOG(PAKNEIMPGNFSXN))\PYGZhy{}(LOG(PAKNEIMPGNFSXN(\PYGZhy{}1))))+0.2*PAKNYGDPGAP\PYGZus{}/100) )) * (1\PYGZhy{}PAKNECONOTHRXN\PYGZus{}D)+ PAKNECONOTHRXN\PYGZus{}X*PAKNECONOTHRXN\PYGZus{}D  \PYGZdl{} 

                      2021       2022       2023       2024       2025
Variable    lag                                                       
Base        0         1.86       2.02       2.18       2.34       2.50
Alternative 0         1.92       2.09       2.26       2.43       2.59
Difference  0         0.06       0.07       0.08       0.09       0.09
Percent     0         3.18       3.50       3.65       3.67       3.58

 Contributions to differende for  PAKNECONOTHRXN
                           2021       2022       2023       2024       2025
Variable         lag                                                       
PAKNECONOTHRXN   \PYGZhy{}1        0.05       0.06       0.08       0.09       0.09
PAKNECONOTHRXN\PYGZus{}A  0       \PYGZhy{}0.00      \PYGZhy{}0.00      \PYGZhy{}0.00      \PYGZhy{}0.00      \PYGZhy{}0.00
PAKNYGDPFCSTXN    0        0.01       0.01       0.02       0.02       0.02
                 \PYGZhy{}1       \PYGZhy{}0.00      \PYGZhy{}0.01      \PYGZhy{}0.01      \PYGZhy{}0.02      \PYGZhy{}0.02
PAKGGREVGNFSXN    0       \PYGZhy{}0.00      \PYGZhy{}0.00      \PYGZhy{}0.00      \PYGZhy{}0.00      \PYGZhy{}0.00
                 \PYGZhy{}1       \PYGZhy{}0.00      \PYGZhy{}0.00      \PYGZhy{}0.00      \PYGZhy{}0.00      \PYGZhy{}0.00
PAKNEIMPGNFSXN    0        0.04       0.05       0.05       0.05       0.05
                 \PYGZhy{}1       \PYGZhy{}0.05      \PYGZhy{}0.05      \PYGZhy{}0.05      \PYGZhy{}0.05      \PYGZhy{}0.05
PAKNYGDPGAP\PYGZus{}      0        0.00       0.00       0.00       0.00      \PYGZhy{}0.00
PAKNECONOTHRXN\PYGZus{}D  0       \PYGZhy{}0.00      \PYGZhy{}0.00      \PYGZhy{}0.00      \PYGZhy{}0.00      \PYGZhy{}0.00
PAKNECONOTHRXN\PYGZus{}X  0       \PYGZhy{}0.00      \PYGZhy{}0.00      \PYGZhy{}0.00      \PYGZhy{}0.00      \PYGZhy{}0.00

 Share of contributions to differende for  PAKNECONOTHRXN
                            2021        2022        2023        2024        2025
Variable         lag                                                            
PAKNECONOTHRXN   \PYGZhy{}1          85\PYGZpc{}         91\PYGZpc{}         96\PYGZpc{}        100\PYGZpc{}        102\PYGZpc{}
PAKNEIMPGNFSXN    0          75\PYGZpc{}         66\PYGZpc{}         61\PYGZpc{}         58\PYGZpc{}         56\PYGZpc{}
PAKNYGDPFCSTXN    0          13\PYGZpc{}         19\PYGZpc{}         23\PYGZpc{}         26\PYGZpc{}         28\PYGZpc{}
PAKNECONOTHRXN\PYGZus{}A  0          \PYGZhy{}0\PYGZpc{}         \PYGZhy{}0\PYGZpc{}         \PYGZhy{}0\PYGZpc{}         \PYGZhy{}0\PYGZpc{}         \PYGZhy{}0\PYGZpc{}
PAKGGREVGNFSXN    0          \PYGZhy{}0\PYGZpc{}         \PYGZhy{}0\PYGZpc{}         \PYGZhy{}0\PYGZpc{}         \PYGZhy{}0\PYGZpc{}         \PYGZhy{}0\PYGZpc{}
                 \PYGZhy{}1          \PYGZhy{}0\PYGZpc{}         \PYGZhy{}0\PYGZpc{}         \PYGZhy{}0\PYGZpc{}         \PYGZhy{}0\PYGZpc{}         \PYGZhy{}0\PYGZpc{}
PAKNECONOTHRXN\PYGZus{}D  0          \PYGZhy{}0\PYGZpc{}         \PYGZhy{}0\PYGZpc{}         \PYGZhy{}0\PYGZpc{}         \PYGZhy{}0\PYGZpc{}         \PYGZhy{}0\PYGZpc{}
PAKNECONOTHRXN\PYGZus{}X  0          \PYGZhy{}0\PYGZpc{}         \PYGZhy{}0\PYGZpc{}         \PYGZhy{}0\PYGZpc{}         \PYGZhy{}0\PYGZpc{}         \PYGZhy{}0\PYGZpc{}
PAKNYGDPGAP\PYGZus{}      0           7\PYGZpc{}          4\PYGZpc{}          2\PYGZpc{}          0\PYGZpc{}         \PYGZhy{}1\PYGZpc{}
PAKNYGDPFCSTXN   \PYGZhy{}1          \PYGZhy{}2\PYGZpc{}        \PYGZhy{}12\PYGZpc{}        \PYGZhy{}19\PYGZpc{}        \PYGZhy{}23\PYGZpc{}        \PYGZhy{}27\PYGZpc{}
PAKNEIMPGNFSXN   \PYGZhy{}1         \PYGZhy{}80\PYGZpc{}        \PYGZhy{}70\PYGZpc{}        \PYGZhy{}65\PYGZpc{}        \PYGZhy{}62\PYGZpc{}        \PYGZhy{}60\PYGZpc{}
Total             0          99\PYGZpc{}         99\PYGZpc{}         99\PYGZpc{}         99\PYGZpc{}         98\PYGZpc{}
Residual          0          \PYGZhy{}1\PYGZpc{}         \PYGZhy{}1\PYGZpc{}         \PYGZhy{}1\PYGZpc{}         \PYGZhy{}1\PYGZpc{}         \PYGZhy{}2\PYGZpc{}

 Contribution to growth rate PAKNECONOTHRXN
                            2021        2022        2023        2024        2025
Variable         lag                                                            
PAKNECONOTHRXN   \PYGZhy{}1        \PYGZhy{}0.0\PYGZpc{}       \PYGZhy{}0.0\PYGZpc{}       \PYGZhy{}0.0\PYGZpc{}       \PYGZhy{}0.0\PYGZpc{}       \PYGZhy{}0.0\PYGZpc{}
PAKNECONOTHRXN\PYGZus{}A  0        \PYGZhy{}0.0\PYGZpc{}       \PYGZhy{}0.0\PYGZpc{}       \PYGZhy{}0.0\PYGZpc{}       \PYGZhy{}0.0\PYGZpc{}       \PYGZhy{}0.0\PYGZpc{}
PAKNYGDPFCSTXN    0         0.0\PYGZpc{}        0.0\PYGZpc{}        0.0\PYGZpc{}        0.0\PYGZpc{}        0.0\PYGZpc{}
                 \PYGZhy{}1        \PYGZhy{}0.0\PYGZpc{}       \PYGZhy{}0.0\PYGZpc{}       \PYGZhy{}0.0\PYGZpc{}       \PYGZhy{}0.0\PYGZpc{}       \PYGZhy{}0.0\PYGZpc{}
PAKGGREVGNFSXN    0        \PYGZhy{}0.0\PYGZpc{}       \PYGZhy{}0.0\PYGZpc{}       \PYGZhy{}0.0\PYGZpc{}       \PYGZhy{}0.0\PYGZpc{}       \PYGZhy{}0.0\PYGZpc{}
                 \PYGZhy{}1        \PYGZhy{}0.0\PYGZpc{}       \PYGZhy{}0.0\PYGZpc{}       \PYGZhy{}0.0\PYGZpc{}       \PYGZhy{}0.0\PYGZpc{}       \PYGZhy{}0.0\PYGZpc{}
PAKNEIMPGNFSXN    0         0.0\PYGZpc{}        0.0\PYGZpc{}        0.0\PYGZpc{}        0.0\PYGZpc{}        0.0\PYGZpc{}
                 \PYGZhy{}1        \PYGZhy{}0.0\PYGZpc{}       \PYGZhy{}0.0\PYGZpc{}       \PYGZhy{}0.0\PYGZpc{}       \PYGZhy{}0.0\PYGZpc{}       \PYGZhy{}0.0\PYGZpc{}
PAKNYGDPGAP\PYGZus{}      0         0.0\PYGZpc{}        0.0\PYGZpc{}        0.0\PYGZpc{}        0.0\PYGZpc{}       \PYGZhy{}0.0\PYGZpc{}
PAKNECONOTHRXN\PYGZus{}D  0        \PYGZhy{}0.0\PYGZpc{}       \PYGZhy{}0.0\PYGZpc{}       \PYGZhy{}0.0\PYGZpc{}       \PYGZhy{}0.0\PYGZpc{}       \PYGZhy{}0.0\PYGZpc{}
PAKNECONOTHRXN\PYGZus{}X  0        \PYGZhy{}0.0\PYGZpc{}       \PYGZhy{}0.0\PYGZpc{}       \PYGZhy{}0.0\PYGZpc{}       \PYGZhy{}0.0\PYGZpc{}       \PYGZhy{}0.0\PYGZpc{}
\end{sphinxVerbatim}

\end{sphinxuseclass}\end{sphinxVerbatimOutput}

\end{sphinxuseclass}
\sphinxAtStartPar
These results indicate that much of the initial impact on prices is coming the increase in the price of imported goods (which includes a large fuel component). As time progresses, the imported inflation component declines and the lagged consumption price dominates.  Other factors such as the cost of domestically produced goods play a larger role and the net impact of imported prices (the total of the contemporaneous and lagged value) approaches zero. Cyclical pressure are initially adding to inflation before declining and eventually turning negative.

\begin{sphinxuseclass}{cell}\begin{sphinxVerbatimInput}

\begin{sphinxuseclass}{cell_input}
\begin{sphinxVerbatim}[commandchars=\\\{\}]
\PYG{n}{mpak}\PYG{p}{[}\PYG{l+s+s1}{\PYGZsq{}}\PYG{l+s+s1}{PAKNECONOTHRXN}\PYG{l+s+s1}{\PYGZsq{}}\PYG{p}{]}\PYG{o}{.}\PYG{n}{dif}\PYG{o}{.}\PYG{n}{df}
\end{sphinxVerbatim}

\end{sphinxuseclass}\end{sphinxVerbatimInput}
\begin{sphinxVerbatimOutput}

\begin{sphinxuseclass}{cell_output}
\begin{sphinxVerbatim}[commandchars=\\\{\}]
      PAKNECONOTHRXN
2020        0.045473
2021        0.059084
2022        0.070827
2023        0.079797
2024        0.085849
...              ...
2096        0.430279
2097        0.449908
2098        0.470503
2099        0.492105
2100        0.514761

[81 rows x 1 columns]
\end{sphinxVerbatim}

\end{sphinxuseclass}\end{sphinxVerbatimOutput}

\end{sphinxuseclass}

\subsection{Return values from \sphinxstyleliteralintitle{\sphinxupquote{.dekomp}}}
\label{\detokenize{content/06_ModelAnalytics/Attribution:return-values-from-dekomp}}
\sphinxAtStartPar
By specifying \sphinxcode{\sphinxupquote{lprint=False}} the output is suppressed, however \sphinxcode{\sphinxupquote{.dekomp}} returns 3 dataframes.

\sphinxAtStartPar
The first (named control below) shows the pre\sphinxhyphen{}shock and post shock levels of the dependent variable and their differences.

\begin{sphinxadmonition}{note}{Tip}

\sphinxAtStartPar
Notice that \sphinxcode{\sphinxupquote{pd.set\_option('display.float\_format', '\{:.2f\}'.format)}} is used to set the display format.
\end{sphinxadmonition}

\begin{sphinxuseclass}{cell}\begin{sphinxVerbatimInput}

\begin{sphinxuseclass}{cell_input}
\begin{sphinxVerbatim}[commandchars=\\\{\}]
\PYG{n}{control}\PYG{p}{,}\PYG{n}{delta}\PYG{p}{,}\PYG{n}{contributions}\PYG{o}{=}\PYG{n}{mpak}\PYG{o}{.}\PYG{n}{dekomp}\PYG{p}{(}\PYG{l+s+s1}{\PYGZsq{}}\PYG{l+s+s1}{PAKNECONOTHRXN}\PYG{l+s+s1}{\PYGZsq{}}\PYG{p}{,}\PYG{n}{lprint}\PYG{o}{=}\PYG{k+kc}{False}\PYG{p}{,}\PYG{n}{start}\PYG{o}{=}\PYG{l+m+mi}{2020}\PYG{p}{,}\PYG{n}{end}\PYG{o}{=}\PYG{l+m+mi}{2027}\PYG{p}{)}
\PYG{n}{pd}\PYG{o}{.}\PYG{n}{set\PYGZus{}option}\PYG{p}{(}\PYG{l+s+s1}{\PYGZsq{}}\PYG{l+s+s1}{display.float\PYGZus{}format}\PYG{l+s+s1}{\PYGZsq{}}\PYG{p}{,} \PYG{l+s+s1}{\PYGZsq{}}\PYG{l+s+si}{\PYGZob{}:.2f\PYGZcb{}}\PYG{l+s+s1}{\PYGZsq{}}\PYG{o}{.}\PYG{n}{format}\PYG{p}{)}
\PYG{n}{control}
\end{sphinxVerbatim}

\end{sphinxuseclass}\end{sphinxVerbatimInput}
\begin{sphinxVerbatimOutput}

\begin{sphinxuseclass}{cell_output}
\begin{sphinxVerbatim}[commandchars=\\\{\}]
                2020 2021 2022 2023 2024 2025 2026 2027
Variable    lag                                        
Base        0   1.70 1.86 2.02 2.18 2.34 2.50 2.65 2.81
Alternative 0   1.74 1.92 2.09 2.26 2.43 2.59 2.74 2.90
Difference  0   0.05 0.06 0.07 0.08 0.09 0.09 0.09 0.09
Percent     0   2.68 3.18 3.50 3.65 3.67 3.58 3.42 3.23
\end{sphinxVerbatim}

\end{sphinxuseclass}\end{sphinxVerbatimOutput}

\end{sphinxuseclass}
\sphinxAtStartPar
The second dataframe shows the changes in the various RHS variables, several of which occur with bot current and lagged values. In this case of the price variables, this is because they enter the original equation as inflation rates (the change in levels over time).

\begin{sphinxuseclass}{cell}\begin{sphinxVerbatimInput}

\begin{sphinxuseclass}{cell_input}
\begin{sphinxVerbatim}[commandchars=\\\{\}]
\PYG{n}{delta}
\end{sphinxVerbatim}

\end{sphinxuseclass}\end{sphinxVerbatimInput}
\begin{sphinxVerbatimOutput}

\begin{sphinxuseclass}{cell_output}
\begin{sphinxVerbatim}[commandchars=\\\{\}]
                      2020  2021  2022  2023  2024  2025  2026  2027
Variable         lag                                                
PAKNECONOTHRXN   \PYGZhy{}1  \PYGZhy{}0.00  0.05  0.06  0.08  0.09  0.09  0.09  0.10
PAKNECONOTHRXN\PYGZus{}A  0  \PYGZhy{}0.00 \PYGZhy{}0.00 \PYGZhy{}0.00 \PYGZhy{}0.00 \PYGZhy{}0.00 \PYGZhy{}0.00 \PYGZhy{}0.00 \PYGZhy{}0.00
PAKNYGDPFCSTXN    0   0.00  0.01  0.01  0.02  0.02  0.02  0.03  0.03
                 \PYGZhy{}1  \PYGZhy{}0.00 \PYGZhy{}0.00 \PYGZhy{}0.01 \PYGZhy{}0.01 \PYGZhy{}0.02 \PYGZhy{}0.02 \PYGZhy{}0.03 \PYGZhy{}0.03
PAKGGREVGNFSXN    0  \PYGZhy{}0.00 \PYGZhy{}0.00 \PYGZhy{}0.00 \PYGZhy{}0.00 \PYGZhy{}0.00 \PYGZhy{}0.00 \PYGZhy{}0.00 \PYGZhy{}0.00
                 \PYGZhy{}1  \PYGZhy{}0.00 \PYGZhy{}0.00 \PYGZhy{}0.00 \PYGZhy{}0.00 \PYGZhy{}0.00 \PYGZhy{}0.00 \PYGZhy{}0.00 \PYGZhy{}0.00
PAKNEIMPGNFSXN    0   0.04  0.04  0.05  0.05  0.05  0.05  0.05  0.05
                 \PYGZhy{}1  \PYGZhy{}0.00 \PYGZhy{}0.05 \PYGZhy{}0.05 \PYGZhy{}0.05 \PYGZhy{}0.05 \PYGZhy{}0.05 \PYGZhy{}0.05 \PYGZhy{}0.05
PAKNYGDPGAP\PYGZus{}      0   0.00  0.00  0.00  0.00  0.00 \PYGZhy{}0.00 \PYGZhy{}0.00 \PYGZhy{}0.00
PAKNECONOTHRXN\PYGZus{}D  0  \PYGZhy{}0.00 \PYGZhy{}0.00 \PYGZhy{}0.00 \PYGZhy{}0.00 \PYGZhy{}0.00 \PYGZhy{}0.00 \PYGZhy{}0.00 \PYGZhy{}0.00
PAKNECONOTHRXN\PYGZus{}X  0  \PYGZhy{}0.00 \PYGZhy{}0.00 \PYGZhy{}0.00 \PYGZhy{}0.00 \PYGZhy{}0.00 \PYGZhy{}0.00 \PYGZhy{}0.00 \PYGZhy{}0.00
\end{sphinxVerbatim}

\end{sphinxuseclass}\end{sphinxVerbatimOutput}

\end{sphinxuseclass}

\subsection{Aggregating the impact of each variables.}
\label{\detokenize{content/06_ModelAnalytics/Attribution:aggregating-the-impact-of-each-variables}}
\sphinxAtStartPar
The dataframe \sphinxcode{\sphinxupquote{delta}} has a multiindex index, consisting of variablele name, and lags. To display the total impact of each variable
no matter of the lag. The standard pandas function groupby can be used.

\begin{sphinxuseclass}{cell}\begin{sphinxVerbatimInput}

\begin{sphinxuseclass}{cell_input}
\begin{sphinxVerbatim}[commandchars=\\\{\}]
\PYG{n}{delta}\PYG{o}{.}\PYG{n}{groupby}\PYG{p}{(}\PYG{l+s+s1}{\PYGZsq{}}\PYG{l+s+s1}{Variable}\PYG{l+s+s1}{\PYGZsq{}}\PYG{p}{,}\PYG{n}{sort}\PYG{o}{=}\PYG{k+kc}{False}\PYG{p}{)}\PYG{o}{.}\PYG{n}{sum}\PYG{p}{(}\PYG{p}{)}
\end{sphinxVerbatim}

\end{sphinxuseclass}\end{sphinxVerbatimInput}
\begin{sphinxVerbatimOutput}

\begin{sphinxuseclass}{cell_output}
\begin{sphinxVerbatim}[commandchars=\\\{\}]
                  2020  2021  2022  2023  2024  2025  2026  2027
Variable                                                        
PAKNECONOTHRXN   \PYGZhy{}0.00  0.05  0.06  0.08  0.09  0.09  0.09  0.10
PAKNECONOTHRXN\PYGZus{}A \PYGZhy{}0.00 \PYGZhy{}0.00 \PYGZhy{}0.00 \PYGZhy{}0.00 \PYGZhy{}0.00 \PYGZhy{}0.00 \PYGZhy{}0.00 \PYGZhy{}0.00
PAKNYGDPFCSTXN    0.00  0.01  0.01  0.00  0.00  0.00 \PYGZhy{}0.00 \PYGZhy{}0.00
PAKGGREVGNFSXN   \PYGZhy{}0.00 \PYGZhy{}0.00 \PYGZhy{}0.00 \PYGZhy{}0.00 \PYGZhy{}0.00 \PYGZhy{}0.00 \PYGZhy{}0.00 \PYGZhy{}0.00
PAKNEIMPGNFSXN    0.04 \PYGZhy{}0.00 \PYGZhy{}0.00 \PYGZhy{}0.00 \PYGZhy{}0.00 \PYGZhy{}0.00 \PYGZhy{}0.00 \PYGZhy{}0.00
PAKNYGDPGAP\PYGZus{}      0.00  0.00  0.00  0.00  0.00 \PYGZhy{}0.00 \PYGZhy{}0.00 \PYGZhy{}0.00
PAKNECONOTHRXN\PYGZus{}D \PYGZhy{}0.00 \PYGZhy{}0.00 \PYGZhy{}0.00 \PYGZhy{}0.00 \PYGZhy{}0.00 \PYGZhy{}0.00 \PYGZhy{}0.00 \PYGZhy{}0.00
PAKNECONOTHRXN\PYGZus{}X \PYGZhy{}0.00 \PYGZhy{}0.00 \PYGZhy{}0.00 \PYGZhy{}0.00 \PYGZhy{}0.00 \PYGZhy{}0.00 \PYGZhy{}0.00 \PYGZhy{}0.00
\end{sphinxVerbatim}

\end{sphinxuseclass}\end{sphinxVerbatimOutput}

\end{sphinxuseclass}
\sphinxAtStartPar
The final dataframe is the omegas for each variable.  For the variables that appear as growth rates the Omegas would have to be added together to get a sense of the contribution of the variable.

\begin{sphinxuseclass}{cell}\begin{sphinxVerbatimInput}

\begin{sphinxuseclass}{cell_input}
\begin{sphinxVerbatim}[commandchars=\\\{\}]
\PYG{n}{contributions}\PYG{o}{.}\PYG{n}{groupby}\PYG{p}{(}\PYG{l+s+s1}{\PYGZsq{}}\PYG{l+s+s1}{Variable}\PYG{l+s+s1}{\PYGZsq{}}\PYG{p}{,}\PYG{n}{sort}\PYG{o}{=}\PYG{k+kc}{False}\PYG{p}{)}\PYG{o}{.}\PYG{n}{sum}\PYG{p}{(}\PYG{p}{)} \PYG{c+c1}{\PYGZsh{} as a percent of the total change (the omegas)}
\end{sphinxVerbatim}

\end{sphinxuseclass}\end{sphinxVerbatimInput}
\begin{sphinxVerbatimOutput}

\begin{sphinxuseclass}{cell_output}
\begin{sphinxVerbatim}[commandchars=\\\{\}]
                   2020  2021  2022  2023  2024   2025   2026   2027
Variable                                                            
PAKNECONOTHRXN    \PYGZhy{}0.00 84.72 91.07 96.01 99.72 102.42 104.35 105.74
PAKNEIMPGNFSXN    91.72 \PYGZhy{}4.17 \PYGZhy{}4.17 \PYGZhy{}4.17 \PYGZhy{}4.22  \PYGZhy{}4.25  \PYGZhy{}4.29  \PYGZhy{}4.33
PAKNYGDPFCSTXN     1.84 11.50  7.43  4.65  2.54   1.00  \PYGZhy{}0.13  \PYGZhy{}0.98
PAKNECONOTHRXN\PYGZus{}A  \PYGZhy{}0.00 \PYGZhy{}0.00 \PYGZhy{}0.00 \PYGZhy{}0.00 \PYGZhy{}0.00  \PYGZhy{}0.00  \PYGZhy{}0.00  \PYGZhy{}0.00
PAKGGREVGNFSXN    \PYGZhy{}0.00 \PYGZhy{}0.00 \PYGZhy{}0.00 \PYGZhy{}0.00 \PYGZhy{}0.00  \PYGZhy{}0.00  \PYGZhy{}0.00  \PYGZhy{}0.00
PAKNECONOTHRXN\PYGZus{}D  \PYGZhy{}0.00 \PYGZhy{}0.00 \PYGZhy{}0.00 \PYGZhy{}0.00 \PYGZhy{}0.00  \PYGZhy{}0.00  \PYGZhy{}0.00  \PYGZhy{}0.00
PAKNECONOTHRXN\PYGZus{}X  \PYGZhy{}0.00 \PYGZhy{}0.00 \PYGZhy{}0.00 \PYGZhy{}0.00 \PYGZhy{}0.00  \PYGZhy{}0.00  \PYGZhy{}0.00  \PYGZhy{}0.00
PAKNYGDPGAP\PYGZus{}       6.65  6.51  4.31  2.11  0.49  \PYGZhy{}0.68  \PYGZhy{}1.46  \PYGZhy{}1.95
Total            100.21 98.56 98.63 98.59 98.53  98.48  98.47  98.48
Residual           0.21 \PYGZhy{}1.44 \PYGZhy{}1.37 \PYGZhy{}1.41 \PYGZhy{}1.47  \PYGZhy{}1.52  \PYGZhy{}1.53  \PYGZhy{}1.52
\end{sphinxVerbatim}

\end{sphinxuseclass}\end{sphinxVerbatimOutput}

\end{sphinxuseclass}
\begin{sphinxadmonition}{note}{Tip}

\sphinxAtStartPar
Notice that \sphinxcode{\sphinxupquote{pd.reset\_option('display.float\_format')}} is used to reset the display format.
\end{sphinxadmonition}

\begin{sphinxuseclass}{cell}\begin{sphinxVerbatimInput}

\begin{sphinxuseclass}{cell_input}
\begin{sphinxVerbatim}[commandchars=\\\{\}]
\PYG{n}{pd}\PYG{o}{.}\PYG{n}{reset\PYGZus{}option}\PYG{p}{(}\PYG{l+s+s1}{\PYGZsq{}}\PYG{l+s+s1}{display.float\PYGZus{}format}\PYG{l+s+s1}{\PYGZsq{}}\PYG{p}{)}
\end{sphinxVerbatim}

\end{sphinxuseclass}\end{sphinxVerbatimInput}

\end{sphinxuseclass}

\subsection{Chart of the contributions over time}
\label{\detokenize{content/06_ModelAnalytics/Attribution:chart-of-the-contributions-over-time}}
\begin{sphinxadmonition}{note}{Truncate attribution}

\sphinxAtStartPar
Some equations have a lot of small contributions. These can be aggregated through the \sphinxcode{\sphinxupquote{threshold=<some number>}} parameter.
Variables for which all contributions are below the threshold will be lumped together in the \sphinxstylestrong{small} bin. Like below:
\end{sphinxadmonition}

\begin{sphinxuseclass}{cell}\begin{sphinxVerbatimInput}

\begin{sphinxuseclass}{cell_input}
\begin{sphinxVerbatim}[commandchars=\\\{\}]
\PYG{n}{fig}\PYG{o}{=}\PYG{n}{mpak}\PYG{o}{.}\PYG{n}{dekomp\PYGZus{}plot}\PYG{p}{(}\PYG{l+s+s1}{\PYGZsq{}}\PYG{l+s+s1}{PAKNECONOTHRXN}\PYG{l+s+s1}{\PYGZsq{}}\PYG{p}{,}\PYG{n}{pct}\PYG{o}{=}\PYG{k+kc}{False}\PYG{p}{,}\PYG{n}{rename}\PYG{o}{=}\PYG{k+kc}{True}\PYG{p}{,}\PYG{n}{threshold}\PYG{o}{=}\PYG{l+m+mf}{.0002}\PYG{p}{,}\PYG{n}{lag}\PYG{o}{=}\PYG{k+kc}{True}\PYG{p}{)}\PYG{p}{;} \PYG{c+c1}{\PYGZsh{}decomp of teh change in the level}
\end{sphinxVerbatim}

\end{sphinxuseclass}\end{sphinxVerbatimInput}
\begin{sphinxVerbatimOutput}

\begin{sphinxuseclass}{cell_output}
\noindent\sphinxincludegraphics{{d326f9e6e559b6c8dee0158aecf605f7541d67408078efd6a39d2a1e07c9164b}.png}

\end{sphinxuseclass}\end{sphinxVerbatimOutput}

\end{sphinxuseclass}

\subsubsection{The lags can be aggregated}
\label{\detokenize{content/06_ModelAnalytics/Attribution:the-lags-can-be-aggregated}}
\sphinxAtStartPar
Now it is obvious that most of the impact stems from one variable

\begin{sphinxuseclass}{cell}\begin{sphinxVerbatimInput}

\begin{sphinxuseclass}{cell_input}
\begin{sphinxVerbatim}[commandchars=\\\{\}]
\PYG{n}{fig}\PYG{o}{=}\PYG{n}{mpak}\PYG{o}{.}\PYG{n}{dekomp\PYGZus{}plot}\PYG{p}{(}\PYG{l+s+s1}{\PYGZsq{}}\PYG{l+s+s1}{PAKNECONOTHRXN}\PYG{l+s+s1}{\PYGZsq{}}\PYG{p}{,}\PYG{n}{pct}\PYG{o}{=}\PYG{k+kc}{True}\PYG{p}{,}\PYG{n}{rename}\PYG{o}{=}\PYG{k+kc}{True}\PYG{p}{,}\PYG{n}{threshold}\PYG{o}{=}\PYG{l+m+mf}{.02}\PYG{p}{,}\PYG{n}{lag}\PYG{o}{=}\PYG{k+kc}{False}\PYG{p}{)}\PYG{p}{;} \PYG{c+c1}{\PYGZsh{} expressed as a share (the conmtrinutions share)}
\end{sphinxVerbatim}

\end{sphinxuseclass}\end{sphinxVerbatimInput}
\begin{sphinxVerbatimOutput}

\begin{sphinxuseclass}{cell_output}
\noindent\sphinxincludegraphics{{5ab744a2e9105fb24b6617f341d060f25890eecfc0a8b3630b3dbbf119c66417}.png}

\end{sphinxuseclass}\end{sphinxVerbatimOutput}

\end{sphinxuseclass}

\subsubsection{The percent impact can be charted}
\label{\detokenize{content/06_ModelAnalytics/Attribution:the-percent-impact-can-be-charted}}
\begin{sphinxuseclass}{cell}\begin{sphinxVerbatimInput}

\begin{sphinxuseclass}{cell_input}
\begin{sphinxVerbatim}[commandchars=\\\{\}]
\PYG{n}{fig}\PYG{o}{=}\PYG{n}{mpak}\PYG{o}{.}\PYG{n}{dekomp\PYGZus{}plot}\PYG{p}{(}\PYG{l+s+s1}{\PYGZsq{}}\PYG{l+s+s1}{PAKNYGDPFCSTXN}\PYG{l+s+s1}{\PYGZsq{}}\PYG{p}{,}\PYG{n}{pct}\PYG{o}{=}\PYG{k+kc}{True}\PYG{p}{,}\PYG{n}{rename}\PYG{o}{=}\PYG{k+kc}{True}\PYG{p}{,}\PYG{n}{threshold}\PYG{o}{=}\PYG{l+m+mf}{.02}\PYG{p}{)}\PYG{p}{;}
\end{sphinxVerbatim}

\end{sphinxuseclass}\end{sphinxVerbatimInput}
\begin{sphinxVerbatimOutput}

\begin{sphinxuseclass}{cell_output}
\noindent\sphinxincludegraphics{{35b2b751bf26b344e507739ba02aa94531066bf23e42012ffd5be529fd72ecc3}.png}

\end{sphinxuseclass}\end{sphinxVerbatimOutput}

\end{sphinxuseclass}
\begin{sphinxuseclass}{cell}\begin{sphinxVerbatimInput}

\begin{sphinxuseclass}{cell_input}
\begin{sphinxVerbatim}[commandchars=\\\{\}]
\PYG{n}{help}\PYG{p}{(}\PYG{n}{mpak}\PYG{o}{.}\PYG{n}{dekomp\PYGZus{}plot}\PYG{p}{)}
\end{sphinxVerbatim}

\end{sphinxuseclass}\end{sphinxVerbatimInput}
\begin{sphinxVerbatimOutput}

\begin{sphinxuseclass}{cell_output}
\begin{sphinxVerbatim}[commandchars=\\\{\}]
Help on method dekomp\PYGZus{}plot in module modelclass:

dekomp\PYGZus{}plot(varnavn, sort=True, pct=True, per=\PYGZsq{}\PYGZsq{}, top=0.9, threshold=0.0, lag=True, rename=True, nametrans=\PYGZlt{}function Dekomp\PYGZus{}Mixin.\PYGZlt{}lambda\PYGZgt{} at 0x0000026A37FF5BD0\PYGZgt{}, time\PYGZus{}att=False) method of modelclass.model instance
    Returns  a chart with attribution for a variable over the smpl  
    
    Parameters
    \PYGZhy{}\PYGZhy{}\PYGZhy{}\PYGZhy{}\PYGZhy{}\PYGZhy{}\PYGZhy{}\PYGZhy{}\PYGZhy{}\PYGZhy{}
    varnavn : TYPE
        variable name.
    sort : TYPE, optional
        . The default is False.
    pct : TYPE, optional
        display pct contribution . The default is True.
    per : TYPE, optional
        DESCRIPTION. The default is \PYGZsq{}\PYGZsq{}.
    threshold : TYPE, optional
        cutoff. The default is 0.0.
    rename : TYPE, optional
        Use descriptions instead of variable names. The default is True.
    time\PYGZus{}att : TYPE, optional
        Do time attribution . The default is False.
    lag : TYPE, optional
       separete by lags The default is True.           
    top : TYPE, optional
      where to place the title 
       
    
    Returns
    \PYGZhy{}\PYGZhy{}\PYGZhy{}\PYGZhy{}\PYGZhy{}\PYGZhy{}\PYGZhy{}
    a matplotlib figure instance .
\end{sphinxVerbatim}

\end{sphinxuseclass}\end{sphinxVerbatimOutput}

\end{sphinxuseclass}

\subsection{Chart of the contributions for one year}
\label{\detokenize{content/06_ModelAnalytics/Attribution:chart-of-the-contributions-for-one-year}}
\sphinxAtStartPar
It can be useful to visualize the attribution as a waterfall chart for a single year

\begin{sphinxuseclass}{cell}\begin{sphinxVerbatimInput}

\begin{sphinxuseclass}{cell_input}
\begin{sphinxVerbatim}[commandchars=\\\{\}]
\PYG{n}{mpak}\PYG{o}{.}\PYG{n}{dekomp\PYGZus{}plot\PYGZus{}per}\PYG{p}{(}\PYG{l+s+s1}{\PYGZsq{}}\PYG{l+s+s1}{PAKNYGDPFCSTXN}\PYG{l+s+s1}{\PYGZsq{}}\PYG{p}{,}\PYG{n}{per}\PYG{o}{=}\PYG{l+m+mi}{2029}\PYG{p}{,}\PYG{n}{threshold}\PYG{o}{=}\PYG{l+m+mi}{5}\PYG{p}{)}  \PYG{c+c1}{\PYGZsh{} gives a waterfall of contributions}
\end{sphinxVerbatim}

\end{sphinxuseclass}\end{sphinxVerbatimInput}
\begin{sphinxVerbatimOutput}

\begin{sphinxuseclass}{cell_output}
\noindent\sphinxincludegraphics{{d63f7099730fcffa45c6ac8bd648553801055e0f5319ed27133fb09b3a8c9b38}.png}

\end{sphinxuseclass}\end{sphinxVerbatimOutput}

\end{sphinxuseclass}

\subsubsection{The waterfall can be sorted}
\label{\detokenize{content/06_ModelAnalytics/Attribution:the-waterfall-can-be-sorted}}
\begin{sphinxuseclass}{cell}\begin{sphinxVerbatimInput}

\begin{sphinxuseclass}{cell_input}
\begin{sphinxVerbatim}[commandchars=\\\{\}]
\PYG{n}{mpak}\PYG{o}{.}\PYG{n}{dekomp\PYGZus{}plot\PYGZus{}per}\PYG{p}{(}\PYG{l+s+s1}{\PYGZsq{}}\PYG{l+s+s1}{PAKNYGDPFCSTXN}\PYG{l+s+s1}{\PYGZsq{}}\PYG{p}{,}\PYG{n}{per}\PYG{o}{=}\PYG{l+m+mi}{2029}\PYG{p}{,}\PYG{n}{threshold}\PYG{o}{=}\PYG{l+m+mi}{5}\PYG{p}{,}\PYG{n}{sort}\PYG{o}{=}\PYG{k+kc}{True}\PYG{p}{)}  \PYG{c+c1}{\PYGZsh{} gives a waterfall of contributions}
\end{sphinxVerbatim}

\end{sphinxuseclass}\end{sphinxVerbatimInput}
\begin{sphinxVerbatimOutput}

\begin{sphinxuseclass}{cell_output}
\noindent\sphinxincludegraphics{{6bb394af30691737fe6e5681006dd423e96863248590c53ca76b6e8c6161b168}.png}

\end{sphinxuseclass}\end{sphinxVerbatimOutput}

\end{sphinxuseclass}

\section{Impacts at the model level: the \sphinxstyleliteralintitle{\sphinxupquote{.totdif()}} method}
\label{\detokenize{content/06_ModelAnalytics/Attribution:impacts-at-the-model-level-the-totdif-method}}
\sphinxAtStartPar
The method \sphinxcode{\sphinxupquote{.totdif()}} returns an instance  the totdif class, which provides a number of methods and properties to explore decomposition at the model level.

\sphinxAtStartPar
It works by solving the model numerous time, each time changing one of the right hand side variables and calculating the impact on the dependent variable. By default it uses the values from the \sphinxcode{\sphinxupquote{.lastdf}} Dataframe as the shock values and the values in \sphinxcode{\sphinxupquote{.basedf}} as the initial values.

\sphinxAtStartPar
For advanced users the RHS variables can be grouped into user defined blocks, which helps identify causal chains.


\subsection{Display the \protect\(\Delta\protect\)’s}
\label{\detokenize{content/06_ModelAnalytics/Attribution:display-the-delta-s}}
\sphinxAtStartPar
This is done using the \sphinxcode{\sphinxupquote{.exodif()}}Function. It will first find the exogenous variables where the values differ between
\sphinxcode{\sphinxupquote{.lastdf}}and \sphinxcode{\sphinxupquote{.basedf}} and then return a dataframe with all the values which differs.

\sphinxAtStartPar
In this case the dataframe contains the the effect of updating the \(CO^2\) tax to 30 for coal, gas and oil.

\begin{sphinxuseclass}{cell}\begin{sphinxVerbatimInput}

\begin{sphinxuseclass}{cell_input}
\begin{sphinxVerbatim}[commandchars=\\\{\}]
\PYG{n}{mpak}\PYG{o}{.}\PYG{n}{exodif}\PYG{p}{(}\PYG{p}{)}
\end{sphinxVerbatim}

\end{sphinxuseclass}\end{sphinxVerbatimInput}
\begin{sphinxVerbatimOutput}

\begin{sphinxuseclass}{cell_output}
\begin{sphinxVerbatim}[commandchars=\\\{\}]
      PAKGGREVCO2CER  PAKGGREVCO2GER  PAKGGREVCO2OER
2020       35.549839       71.000884        38.71065
2021       35.549839       71.000884        38.71065
2022       35.549839       71.000884        38.71065
2023       35.549839       71.000884        38.71065
2024       35.549839       71.000884        38.71065
...              ...             ...             ...
2096       35.549839       71.000884        38.71065
2097       35.549839       71.000884        38.71065
2098       35.549839       71.000884        38.71065
2099       35.549839       71.000884        38.71065
2100       35.549839       71.000884        38.71065

[81 rows x 3 columns]
\end{sphinxVerbatim}

\end{sphinxuseclass}\end{sphinxVerbatimOutput}

\end{sphinxuseclass}

\subsection{Now do the analyses}
\label{\detokenize{content/06_ModelAnalytics/Attribution:now-do-the-analyses}}
\sphinxAtStartPar
This involves solving the model a number of times, so

\begin{sphinxuseclass}{cell}\begin{sphinxVerbatimInput}

\begin{sphinxuseclass}{cell_input}
\begin{sphinxVerbatim}[commandchars=\\\{\}]
\PYG{n}{totdekomp} \PYG{o}{=} \PYG{n}{mpak}\PYG{o}{.}\PYG{n}{totdif}\PYG{p}{(}\PYG{p}{)} \PYG{c+c1}{\PYGZsh{} Calculate the total derivative½s of all equations in the model.}
\end{sphinxVerbatim}

\end{sphinxuseclass}\end{sphinxVerbatimInput}
\begin{sphinxVerbatimOutput}

\begin{sphinxuseclass}{cell_output}
\begin{sphinxVerbatim}[commandchars=\\\{\}]
Total dekomp took       :         3.379 Seconds
\end{sphinxVerbatim}

\end{sphinxuseclass}\end{sphinxVerbatimOutput}

\end{sphinxuseclass}

\subsection{.explain\_all will visualize the results}
\label{\detokenize{content/06_ModelAnalytics/Attribution:explain-all-will-visualize-the-results}}
\begin{sphinxuseclass}{cell}\begin{sphinxVerbatimInput}

\begin{sphinxuseclass}{cell_input}
\begin{sphinxVerbatim}[commandchars=\\\{\}]
\PYG{n}{showvar} \PYG{o}{=} \PYG{l+s+s1}{\PYGZsq{}}\PYG{l+s+s1}{PAKNYGDPMKTPKN}\PYG{l+s+s1}{\PYGZsq{}}
\PYG{n}{totdekomp}\PYG{o}{.}\PYG{n}{explain\PYGZus{}all}\PYG{p}{(}\PYG{n}{showvar}\PYG{p}{,}\PYG{n}{kind}\PYG{o}{=}\PYG{l+s+s1}{\PYGZsq{}}\PYG{l+s+s1}{area}\PYG{l+s+s1}{\PYGZsq{}}\PYG{p}{,}\PYG{n}{stacked}\PYG{o}{=}\PYG{k+kc}{True}\PYG{p}{)}\PYG{p}{;}    
\end{sphinxVerbatim}

\end{sphinxuseclass}\end{sphinxVerbatimInput}
\begin{sphinxVerbatimOutput}

\begin{sphinxuseclass}{cell_output}
\noindent\sphinxincludegraphics{{b711cf4b0679e8c40533dab1d5f657be07615aa691d7dd6cae77edd0b5926637}.png}

\end{sphinxuseclass}\end{sphinxVerbatimOutput}

\end{sphinxuseclass}

\subsection{Many variables}
\label{\detokenize{content/06_ModelAnalytics/Attribution:many-variables}}
\begin{sphinxuseclass}{cell}\begin{sphinxVerbatimInput}

\begin{sphinxuseclass}{cell_input}
\begin{sphinxVerbatim}[commandchars=\\\{\}]
\PYG{n}{showvar} \PYG{o}{=} \PYG{l+s+s1}{\PYGZsq{}}\PYG{l+s+s1}{PAKNYGDPMKTPKN PAKCCEMISCO2CKN PAKCCEMISCO2OKN PAKCCEMISCO2GKN PAKGGREVTOTLCN}\PYG{l+s+s1}{\PYGZsq{}}
\PYG{n}{totdekomp}\PYG{o}{.}\PYG{n}{explain\PYGZus{}all}\PYG{p}{(}\PYG{n}{showvar}\PYG{p}{,}\PYG{n}{kind}\PYG{o}{=}\PYG{l+s+s1}{\PYGZsq{}}\PYG{l+s+s1}{area}\PYG{l+s+s1}{\PYGZsq{}}\PYG{p}{,}\PYG{n}{stacked}\PYG{o}{=}\PYG{k+kc}{True}\PYG{p}{)}\PYG{p}{;}    
\end{sphinxVerbatim}

\end{sphinxuseclass}\end{sphinxVerbatimInput}
\begin{sphinxVerbatimOutput}

\begin{sphinxuseclass}{cell_output}
\noindent\sphinxincludegraphics{{ef59e7d20117c4502d187cc87f4388081a241d3d7fbddd4aa451c8d5de712bf4}.png}

\end{sphinxuseclass}\end{sphinxVerbatimOutput}

\end{sphinxuseclass}

\subsection{Charts for the contribution in one year}
\label{\detokenize{content/06_ModelAnalytics/Attribution:charts-for-the-contribution-in-one-year}}
\begin{sphinxuseclass}{cell}\begin{sphinxVerbatimInput}

\begin{sphinxuseclass}{cell_input}
\begin{sphinxVerbatim}[commandchars=\\\{\}]
\PYG{n}{showvar} \PYG{o}{=} \PYG{l+s+s1}{\PYGZsq{}}\PYG{l+s+s1}{PAKNYGDPMKTPKN}\PYG{l+s+s1}{\PYGZsq{}}

\PYG{n}{totdekomp}\PYG{o}{.}\PYG{n}{explain\PYGZus{}per}\PYG{p}{(}\PYG{n}{showvar}\PYG{p}{,}\PYG{n}{per}\PYG{o}{=}\PYG{l+m+mi}{2023}\PYG{p}{,}\PYG{n}{ysize}\PYG{o}{=}\PYG{l+m+mi}{8}\PYG{p}{)}
\end{sphinxVerbatim}

\end{sphinxuseclass}\end{sphinxVerbatimInput}
\begin{sphinxVerbatimOutput}

\begin{sphinxuseclass}{cell_output}
\noindent\sphinxincludegraphics{{5d2235e558172de0e98474c33c4c700319190bc2cdd0959bb37236b9375b489a}.png}

\end{sphinxuseclass}\end{sphinxVerbatimOutput}

\end{sphinxuseclass}

\subsection{Or we can use interactive widgets}
\label{\detokenize{content/06_ModelAnalytics/Attribution:or-we-can-use-interactive-widgets}}
\sphinxAtStartPar
This allows the user to select the specific variable of interest and what to display:

\begin{sphinxadmonition}{note}{Note:}
\sphinxAtStartPar
If this is read in a manual the widget is not live.

\sphinxAtStartPar
In a notebook the selection widgets are live.
\end{sphinxadmonition}

\begin{sphinxuseclass}{cell}\begin{sphinxVerbatimInput}

\begin{sphinxuseclass}{cell_input}
\begin{sphinxVerbatim}[commandchars=\\\{\}]
\PYG{n}{display}\PYG{p}{(}\PYG{n}{mpak}\PYG{o}{.}\PYG{n}{get\PYGZus{}att\PYGZus{}gui}\PYG{p}{(}\PYG{n}{var}\PYG{o}{=}\PYG{l+s+s1}{\PYGZsq{}}\PYG{l+s+s1}{PAKGGREVTOTLCN}\PYG{l+s+s1}{\PYGZsq{}}\PYG{p}{,}\PYG{n}{ysize}\PYG{o}{=}\PYG{l+m+mi}{7}\PYG{p}{)}\PYG{p}{)}\PYG{p}{;}
\end{sphinxVerbatim}

\end{sphinxuseclass}\end{sphinxVerbatimInput}
\begin{sphinxVerbatimOutput}

\begin{sphinxuseclass}{cell_output}
\begin{sphinxVerbatim}[commandchars=\\\{\}]
interactive(children=(Dropdown(description=\PYGZsq{}Variable\PYGZsq{}, index=108, options=(\PYGZsq{}CHNEXR05\PYGZsq{}, \PYGZsq{}CHNPCEXN05\PYGZsq{}, \PYGZsq{}DEUEXR05…
\end{sphinxVerbatim}

\begin{sphinxVerbatim}[commandchars=\\\{\}]
None
\end{sphinxVerbatim}

\end{sphinxuseclass}\end{sphinxVerbatimOutput}

\end{sphinxuseclass}

\subsection{Attribution of the last year}
\label{\detokenize{content/06_ModelAnalytics/Attribution:attribution-of-the-last-year}}
\begin{sphinxuseclass}{cell}\begin{sphinxVerbatimInput}

\begin{sphinxuseclass}{cell_input}
\begin{sphinxVerbatim}[commandchars=\\\{\}]
\PYG{n}{totdekomp}\PYG{o}{.}\PYG{n}{explain\PYGZus{}last}\PYG{p}{(}\PYG{n}{showvar}\PYG{p}{,}\PYG{n}{ysize}\PYG{o}{=}\PYG{l+m+mi}{8}\PYG{p}{)}
\end{sphinxVerbatim}

\end{sphinxuseclass}\end{sphinxVerbatimInput}
\begin{sphinxVerbatimOutput}

\begin{sphinxuseclass}{cell_output}
\noindent\sphinxincludegraphics{{c033e04a7ff87cb4a07536689ee9574d7682b52a255e0b5817c334de01dc4a18}.png}

\end{sphinxuseclass}\end{sphinxVerbatimOutput}

\end{sphinxuseclass}

\subsection{Attribution of accumulated effects}
\label{\detokenize{content/06_ModelAnalytics/Attribution:attribution-of-accumulated-effects}}
\begin{sphinxuseclass}{cell}\begin{sphinxVerbatimInput}

\begin{sphinxuseclass}{cell_input}
\begin{sphinxVerbatim}[commandchars=\\\{\}]
\PYG{n}{totdekomp}\PYG{o}{.}\PYG{n}{explain\PYGZus{}sum}\PYG{p}{(}\PYG{n}{showvar}\PYG{p}{,}\PYG{n}{ysize}\PYG{o}{=}\PYG{l+m+mi}{8}\PYG{p}{)}
\end{sphinxVerbatim}

\end{sphinxuseclass}\end{sphinxVerbatimInput}
\begin{sphinxVerbatimOutput}

\begin{sphinxuseclass}{cell_output}
\noindent\sphinxincludegraphics{{f61ca7e6b2b80abc4284322ec3882ec8b2baf04403a94f5fbb3bee21a6010fb9}.png}

\end{sphinxuseclass}\end{sphinxVerbatimOutput}

\end{sphinxuseclass}

\section{More advanced model attribution}
\label{\detokenize{content/06_ModelAnalytics/Attribution:more-advanced-model-attribution}}
\sphinxAtStartPar
For some  models (like the EBA bank stress test model) the number of changed exogenous variables can be large. Using a dictionary to contain the experiments allows us to create experiments where all variables for each country are analyzed, or each macro variable for all countries are analyzed.

\sphinxAtStartPar
Also it is possible to use aggregated sums \sphinxhyphen{} useful for looking at impact on PD’s. Or just the last time period \sphinxhyphen{} useful for looking at CET1 ratios.

\sphinxAtStartPar
If there are many experiments, data can be filtered in order to look only at the variables with an impact above a certain threshold.

\sphinxAtStartPar
The is also the possibility to   anonymize  the row and column names and to randomize
the order of rows and/or columns \sphinxhyphen{} useful for bank names.


\subsection{Grouping variables}
\label{\detokenize{content/06_ModelAnalytics/Attribution:grouping-variables}}
\sphinxAtStartPar
If two experiments differ by many exogenous variables it can make sense to group the variables into experiments. This allows the user to
slice and dice the impact along different dimensions.

\sphinxAtStartPar
To illustrate this below you will find a model attribution where the impact of gas and coal tax is grouped together and the oil tax is in its own group.

\begin{sphinxuseclass}{cell}\begin{sphinxVerbatimInput}

\begin{sphinxuseclass}{cell_input}
\begin{sphinxVerbatim}[commandchars=\\\{\}]
\PYG{n}{experiments} \PYG{o}{=} \PYG{p}{\PYGZob{}}\PYG{l+s+s1}{\PYGZsq{}}\PYG{l+s+s1}{gas and coal}\PYG{l+s+s1}{\PYGZsq{}}\PYG{p}{:}\PYG{p}{[}\PYG{l+s+s1}{\PYGZsq{}}\PYG{l+s+s1}{PAKGGREVCO2CER}\PYG{l+s+s1}{\PYGZsq{}}\PYG{p}{,} \PYG{l+s+s1}{\PYGZsq{}}\PYG{l+s+s1}{PAKGGREVCO2GER}\PYG{l+s+s1}{\PYGZsq{}}\PYG{p}{]}\PYG{p}{,}\PYG{l+s+s1}{\PYGZsq{}}\PYG{l+s+s1}{Oil}\PYG{l+s+s1}{\PYGZsq{}}\PYG{p}{:}\PYG{p}{[}\PYG{l+s+s1}{\PYGZsq{}}\PYG{l+s+s1}{PAKGGREVCO2OER}\PYG{l+s+s1}{\PYGZsq{}}\PYG{p}{]}\PYG{p}{\PYGZcb{}}
\PYG{n}{totdekomp\PYGZus{}group} \PYG{o}{=} \PYG{n}{mpak}\PYG{o}{.}\PYG{n}{totdif}\PYG{p}{(}\PYG{n}{experiments} \PYG{o}{=} \PYG{n}{experiments}\PYG{p}{)} \PYG{c+c1}{\PYGZsh{} Calculate the total derivative½s of all equations in the model.}
\end{sphinxVerbatim}

\end{sphinxuseclass}\end{sphinxVerbatimInput}
\begin{sphinxVerbatimOutput}

\begin{sphinxuseclass}{cell_output}
\begin{sphinxVerbatim}[commandchars=\\\{\}]
Total dekomp took       :         2.392 Seconds
\end{sphinxVerbatim}

\end{sphinxuseclass}\end{sphinxVerbatimOutput}

\end{sphinxuseclass}
\begin{sphinxuseclass}{cell}\begin{sphinxVerbatimInput}

\begin{sphinxuseclass}{cell_input}
\begin{sphinxVerbatim}[commandchars=\\\{\}]
\PYG{n}{showvar} \PYG{o}{=} \PYG{l+s+s1}{\PYGZsq{}}\PYG{l+s+s1}{PAKNYGDPMKTPKN}\PYG{l+s+s1}{\PYGZsq{}}
\PYG{n}{totdekomp\PYGZus{}group}\PYG{o}{.}\PYG{n}{explain\PYGZus{}all}\PYG{p}{(}\PYG{n}{showvar}\PYG{p}{,}\PYG{n}{kind}\PYG{o}{=}\PYG{l+s+s1}{\PYGZsq{}}\PYG{l+s+s1}{area}\PYG{l+s+s1}{\PYGZsq{}}\PYG{p}{,}\PYG{n}{stacked}\PYG{o}{=}\PYG{k+kc}{True}\PYG{p}{)}\PYG{p}{;}    
\end{sphinxVerbatim}

\end{sphinxuseclass}\end{sphinxVerbatimInput}
\begin{sphinxVerbatimOutput}

\begin{sphinxuseclass}{cell_output}
\noindent\sphinxincludegraphics{{1293a8fa1a10b1492cce34db04672d46f21a13f9a5000d06de1e896d161d95af}.png}

\end{sphinxuseclass}\end{sphinxVerbatimOutput}

\end{sphinxuseclass}

\subsection{Single equation attribution chart}
\label{\detokenize{content/06_ModelAnalytics/Attribution:single-equation-attribution-chart}}
\sphinxAtStartPar
The results can be visualized in different ways.

\begin{sphinxuseclass}{cell}\begin{sphinxVerbatimInput}

\begin{sphinxuseclass}{cell_input}
\begin{sphinxVerbatim}[commandchars=\\\{\}]
\PYG{n}{mpak}\PYG{o}{.}\PYG{n}{dekomp\PYGZus{}plot\PYGZus{}per}\PYG{p}{(}\PYG{l+s+s1}{\PYGZsq{}}\PYG{l+s+s1}{PAKNYGDPMKTPKN}\PYG{l+s+s1}{\PYGZsq{}}\PYG{p}{,}\PYG{n}{per}\PYG{o}{=}\PYG{l+m+mi}{2040}\PYG{p}{,}\PYG{n}{pct}\PYG{o}{=}\PYG{l+m+mi}{0}\PYG{p}{,}\PYG{n}{rename}\PYG{o}{=}\PYG{l+m+mi}{1}\PYG{p}{,}\PYG{n}{sort}\PYG{o}{=}\PYG{l+m+mi}{1}\PYG{p}{,}\PYG{n}{threshold} \PYG{o}{=}\PYG{l+m+mi}{200000}\PYG{p}{,}\PYG{n}{ysize}\PYG{o}{=}\PYG{l+m+mi}{7}\PYG{p}{)}\PYG{p}{;}
\end{sphinxVerbatim}

\end{sphinxuseclass}\end{sphinxVerbatimInput}

\end{sphinxuseclass}

\subsection{Attribution when comparing time frames}
\label{\detokenize{content/06_ModelAnalytics/Attribution:attribution-when-comparing-time-frames}}
\sphinxAtStartPar
In this case we seek to find out which variables explains the development from year to year. This is done only for the .lastdf dateframe.

\begin{sphinxuseclass}{cell}\begin{sphinxVerbatimInput}

\begin{sphinxuseclass}{cell_input}
\begin{sphinxVerbatim}[commandchars=\\\{\}]
\PYG{k}{with} \PYG{n}{mpak}\PYG{o}{.}\PYG{n}{set\PYGZus{}smpl}\PYG{p}{(}\PYG{l+m+mi}{2020}\PYG{p}{,}\PYG{l+m+mi}{2024}\PYG{p}{)}\PYG{p}{:}
    \PYG{n}{mpak}\PYG{p}{[}\PYG{l+s+s1}{\PYGZsq{}}\PYG{l+s+s1}{PAKNYGDPMKTPKN}\PYG{l+s+s1}{\PYGZsq{}}\PYG{p}{]}\PYG{o}{.}\PYG{n}{dekomp}\PYG{p}{(}\PYG{n}{time\PYGZus{}att}\PYG{o}{=}\PYG{k+kc}{True}\PYG{p}{)}
\end{sphinxVerbatim}

\end{sphinxuseclass}\end{sphinxVerbatimInput}
\begin{sphinxVerbatimOutput}

\begin{sphinxuseclass}{cell_output}
\begin{sphinxVerbatim}[commandchars=\\\{\}]
Formula        : FRML \PYGZlt{}IDENT\PYGZgt{} PAKNYGDPMKTPKN = PAKNECONPRVTKN+PAKNECONGOVTKN+PAKNEGDIFTOTKN+PAKNEGDISTKBKN+PAKNEEXPGNFSKN\PYGZhy{}PAKNEIMPGNFSKN+PAKNYGDPDISCKN+PAKADAP*PAKDISPREPKN \PYGZdl{} 

                      2020        2021        2022        2023        2024
Variable   lag                                                            
t\PYGZhy{}1        0   25760579.27 26471666.00 26767616.83 26891916.75 27090476.06
t          0   26471666.00 26767616.83 26891916.75 27090476.06 27455212.23
Difference 0     711086.73   295950.83   124299.92   198559.31   364736.17
Percent    0          2.76        1.12        0.46        0.74        1.35

 Contributions to differende for  PAKNYGDPMKTPKN
                         2020       2021       2022       2023       2024
Variable       lag                                                       
PAKNECONPRVTKN 0    416507.93  218812.97  107677.34  182518.76  332666.98
PAKNECONGOVTKN 0    348503.70   56821.70   \PYGZhy{}8433.94   15765.70   61617.96
PAKNEGDIFTOTKN 0    224846.18   63575.21   21086.24    8250.65   10770.78
PAKNEGDISTKBKN 0      9896.74   10138.31   10385.77   10639.25   10898.93
PAKNEEXPGNFSKN 0     95590.65  108489.15  115927.86  120153.92  122600.63
PAKNEIMPGNFSKN 0   \PYGZhy{}385555.01 \PYGZhy{}163214.66 \PYGZhy{}123703.85 \PYGZhy{}140162.72 \PYGZhy{}175246.89
PAKNYGDPDISCKN 0      1296.36    1328.02    1360.44    1393.63    1427.65
PAKADAP        0        \PYGZhy{}0.03      \PYGZhy{}0.02      \PYGZhy{}0.01      \PYGZhy{}0.02      \PYGZhy{}0.02
PAKDISPREPKN   0        \PYGZhy{}0.03      \PYGZhy{}0.02      \PYGZhy{}0.01      \PYGZhy{}0.02      \PYGZhy{}0.02

 Share of contributions to differende for  PAKNYGDPMKTPKN
                          2020        2021        2022        2023        2024
Variable       lag                                                            
PAKNECONPRVTKN 0           59\PYGZpc{}         74\PYGZpc{}         87\PYGZpc{}         92\PYGZpc{}         91\PYGZpc{}
PAKNEEXPGNFSKN 0           13\PYGZpc{}         37\PYGZpc{}         93\PYGZpc{}         61\PYGZpc{}         34\PYGZpc{}
PAKNECONGOVTKN 0           49\PYGZpc{}         19\PYGZpc{}         \PYGZhy{}7\PYGZpc{}          8\PYGZpc{}         17\PYGZpc{}
PAKNEGDISTKBKN 0            1\PYGZpc{}          3\PYGZpc{}          8\PYGZpc{}          5\PYGZpc{}          3\PYGZpc{}
PAKNEGDIFTOTKN 0           32\PYGZpc{}         21\PYGZpc{}         17\PYGZpc{}          4\PYGZpc{}          3\PYGZpc{}
PAKNYGDPDISCKN 0            0\PYGZpc{}          0\PYGZpc{}          1\PYGZpc{}          1\PYGZpc{}          0\PYGZpc{}
PAKADAP        0           \PYGZhy{}0\PYGZpc{}         \PYGZhy{}0\PYGZpc{}         \PYGZhy{}0\PYGZpc{}         \PYGZhy{}0\PYGZpc{}         \PYGZhy{}0\PYGZpc{}
PAKDISPREPKN   0           \PYGZhy{}0\PYGZpc{}         \PYGZhy{}0\PYGZpc{}         \PYGZhy{}0\PYGZpc{}         \PYGZhy{}0\PYGZpc{}         \PYGZhy{}0\PYGZpc{}
PAKNEIMPGNFSKN 0          \PYGZhy{}54\PYGZpc{}        \PYGZhy{}55\PYGZpc{}       \PYGZhy{}100\PYGZpc{}        \PYGZhy{}71\PYGZpc{}        \PYGZhy{}48\PYGZpc{}
Total          0          100\PYGZpc{}        100\PYGZpc{}        100\PYGZpc{}        100\PYGZpc{}        100\PYGZpc{}
Residual       0           \PYGZhy{}0\PYGZpc{}         \PYGZhy{}0\PYGZpc{}         \PYGZhy{}0\PYGZpc{}         \PYGZhy{}0\PYGZpc{}         \PYGZhy{}0\PYGZpc{}

 Contribution to growth rate PAKNYGDPMKTPKN
                          2020        2021        2022        2023        2024
Variable       lag                                                            
PAKNECONPRVTKN 0          0.0\PYGZpc{}        0.0\PYGZpc{}        0.0\PYGZpc{}        0.0\PYGZpc{}        0.0\PYGZpc{}
PAKNECONGOVTKN 0          0.0\PYGZpc{}        0.0\PYGZpc{}       \PYGZhy{}0.0\PYGZpc{}        0.0\PYGZpc{}        0.0\PYGZpc{}
PAKNEGDIFTOTKN 0          0.0\PYGZpc{}        0.0\PYGZpc{}        0.0\PYGZpc{}        0.0\PYGZpc{}        0.0\PYGZpc{}
PAKNEGDISTKBKN 0          0.0\PYGZpc{}        0.0\PYGZpc{}        0.0\PYGZpc{}        0.0\PYGZpc{}        0.0\PYGZpc{}
PAKNEEXPGNFSKN 0          0.0\PYGZpc{}        0.0\PYGZpc{}        0.0\PYGZpc{}        0.0\PYGZpc{}        0.0\PYGZpc{}
PAKNEIMPGNFSKN 0         \PYGZhy{}0.0\PYGZpc{}       \PYGZhy{}0.0\PYGZpc{}       \PYGZhy{}0.0\PYGZpc{}       \PYGZhy{}0.0\PYGZpc{}       \PYGZhy{}0.0\PYGZpc{}
PAKNYGDPDISCKN 0          0.0\PYGZpc{}        0.0\PYGZpc{}        0.0\PYGZpc{}        0.0\PYGZpc{}        0.0\PYGZpc{}
PAKADAP        0         \PYGZhy{}0.0\PYGZpc{}       \PYGZhy{}0.0\PYGZpc{}       \PYGZhy{}0.0\PYGZpc{}       \PYGZhy{}0.0\PYGZpc{}       \PYGZhy{}0.0\PYGZpc{}
PAKDISPREPKN   0         \PYGZhy{}0.0\PYGZpc{}       \PYGZhy{}0.0\PYGZpc{}       \PYGZhy{}0.0\PYGZpc{}       \PYGZhy{}0.0\PYGZpc{}       \PYGZhy{}0.0\PYGZpc{}
\end{sphinxVerbatim}

\end{sphinxuseclass}\end{sphinxVerbatimOutput}

\end{sphinxuseclass}
\begin{sphinxuseclass}{cell}\begin{sphinxVerbatimInput}

\begin{sphinxuseclass}{cell_input}
\begin{sphinxVerbatim}[commandchars=\\\{\}]
\PYG{n}{mpak}\PYG{o}{.}\PYG{n}{dekomp\PYGZus{}plot}\PYG{p}{(}\PYG{l+s+s1}{\PYGZsq{}}\PYG{l+s+s1}{PAKNYGDPMKTPKN}\PYG{l+s+s1}{\PYGZsq{}}\PYG{p}{,}\PYG{n}{pct}\PYG{o}{=}\PYG{l+m+mi}{0}\PYG{p}{,}\PYG{n}{rename}\PYG{o}{=}\PYG{l+m+mi}{1}\PYG{p}{,}\PYG{n}{sort}\PYG{o}{=}\PYG{l+m+mi}{1}\PYG{p}{,}\PYG{n}{threshold} \PYG{o}{=}\PYG{l+m+mi}{0}\PYG{p}{,}\PYG{n}{time\PYGZus{}att} \PYG{o}{=} \PYG{k+kc}{True}\PYG{p}{)}\PYG{p}{;}
\end{sphinxVerbatim}

\end{sphinxuseclass}\end{sphinxVerbatimInput}
\begin{sphinxVerbatimOutput}

\begin{sphinxuseclass}{cell_output}
\noindent\sphinxincludegraphics{{57a71b9cc93c8ce0f7108f4c0eec17b7bb017ffc28bda32c0a6445899795ac3c}.png}

\end{sphinxuseclass}\end{sphinxVerbatimOutput}

\end{sphinxuseclass}

\subsection{Visualizing attribution in dependency graphs}
\label{\detokenize{content/06_ModelAnalytics/Attribution:visualizing-attribution-in-dependency-graphs}}
\sphinxAtStartPar
The logical graph of the model can be used to show the upstream and downstream variable for a specific variable. More on this \DUrole{xref,myst}{here}
When drawing the logical graph for a variable the model attribution will be used to guide the thickness of edges between nodes (variables). This enables a visual impression of which
variables drives the impact.

\begin{sphinxadmonition}{note}{Note:}
\sphinxAtStartPar
If png == 0 the graph below will be rendered in SVG format. This enables tooltips with additional information when the mouse hovers
over an edge or an node.

\sphinxAtStartPar
Unfortunately svg can’t be displayed in the manual, so png has to be True for the manual. In a live jupyter notebook set latex=0. This will
enable svg format.
\end{sphinxadmonition}

\begin{sphinxuseclass}{cell}\begin{sphinxVerbatimInput}

\begin{sphinxuseclass}{cell_input}
\begin{sphinxVerbatim}[commandchars=\\\{\}]
\PYG{c+c1}{\PYGZsh{}mpak[\PYGZsq{}PAKNYGDPMKTPKN PAKNECONPRVTKN\PYGZsq{}].draw(up=3,down=0,png=latex,filter=20) \PYGZsh{} For book}
\PYG{n}{mpak}\PYG{p}{[}\PYG{l+s+s1}{\PYGZsq{}}\PYG{l+s+s1}{PAKNYGDPMKTPKN}\PYG{l+s+s1}{\PYGZsq{}}\PYG{p}{]}\PYG{o}{.}\PYG{n}{draw}\PYG{p}{(}\PYG{n}{up}\PYG{o}{=}\PYG{l+m+mi}{3}\PYG{p}{,}\PYG{n}{down}\PYG{o}{=}\PYG{l+m+mi}{0}\PYG{p}{,}\PYG{n}{png}\PYG{o}{=}\PYG{n}{latex}\PYG{p}{,}\PYG{n+nb}{filter}\PYG{o}{=}\PYG{l+m+mi}{400}\PYG{p}{,}\PYG{n}{svg}\PYG{o}{=}\PYG{k+kc}{True}\PYG{p}{,}\PYG{n}{size}\PYG{o}{=}\PYG{p}{(}\PYG{l+m+mi}{8}\PYG{p}{,}\PYG{l+m+mi}{40}\PYG{p}{)}\PYG{p}{)}\PYG{c+c1}{\PYGZsh{}3 for interactice}
\end{sphinxVerbatim}

\end{sphinxuseclass}\end{sphinxVerbatimInput}
\begin{sphinxVerbatimOutput}

\begin{sphinxuseclass}{cell_output}
\noindent\sphinxincludegraphics{{73b533584aeb5a2dd9bc8baed9a6f102b861dc5b5e48b9daff00480396f0daec}.png}

\end{sphinxuseclass}\end{sphinxVerbatimOutput}

\end{sphinxuseclass}
\begin{sphinxuseclass}{cell}\begin{sphinxVerbatimInput}

\begin{sphinxuseclass}{cell_input}
\begin{sphinxVerbatim}[commandchars=\\\{\}]
\PYG{n}{mpak}\PYG{p}{[}\PYG{l+s+s1}{\PYGZsq{}}\PYG{l+s+s1}{PAKNYGDPMKTPKN PAKNECONPRVTKN}\PYG{l+s+s1}{\PYGZsq{}}\PYG{p}{]}\PYG{o}{.}\PYG{n}{draw}\PYG{p}{(}\PYG{n}{up}\PYG{o}{=}\PYG{l+m+mi}{1}\PYG{p}{,}\PYG{n}{down}\PYG{o}{=}\PYG{l+m+mi}{1}\PYG{p}{,}\PYG{n}{png}\PYG{o}{=}\PYG{n}{latex}\PYG{p}{)}  \PYG{c+c1}{\PYGZsh{} diagram all direct dependencies }
\end{sphinxVerbatim}

\end{sphinxuseclass}\end{sphinxVerbatimInput}
\begin{sphinxVerbatimOutput}

\begin{sphinxuseclass}{cell_output}
\noindent\sphinxincludegraphics{{839bda5ae490fd1957c0f20dd10c45526b68a5ef25f6bb210f65a9958fb52d82}.png}

\noindent\sphinxincludegraphics{{3d1ea30c3c9589c3d1f73cbd36826aa53fa53f53032587bbae4a7a9fcf92dd64}.png}

\end{sphinxuseclass}\end{sphinxVerbatimOutput}

\end{sphinxuseclass}

\subsection{The attribution can be filtered and more levels can be displayed.}
\label{\detokenize{content/06_ModelAnalytics/Attribution:the-attribution-can-be-filtered-and-more-levels-can-be-displayed}}
\begin{sphinxuseclass}{cell}\begin{sphinxVerbatimInput}

\begin{sphinxuseclass}{cell_input}
\begin{sphinxVerbatim}[commandchars=\\\{\}]
\PYG{n}{mpak}\PYG{p}{[}\PYG{l+s+s1}{\PYGZsq{}}\PYG{l+s+s1}{PAKNYGDPMKTPKN}\PYG{l+s+s1}{\PYGZsq{}}\PYG{p}{]}\PYG{o}{.}\PYG{n}{draw}\PYG{p}{(}\PYG{n}{up}\PYG{o}{=}\PYG{l+m+mi}{2}\PYG{p}{,}\PYG{n}{down}\PYG{o}{=}\PYG{l+m+mi}{1}\PYG{p}{,}\PYG{n}{png}\PYG{o}{=}\PYG{n}{latex}\PYG{p}{,}\PYG{n+nb}{filter}\PYG{o}{=}\PYG{l+m+mi}{20}\PYG{p}{)} 
\end{sphinxVerbatim}

\end{sphinxuseclass}\end{sphinxVerbatimInput}
\begin{sphinxVerbatimOutput}

\begin{sphinxuseclass}{cell_output}
\noindent\sphinxincludegraphics{{eb99e9dd018c1770e09b138ed65d2f57d577d5cadd9d63dfe31bfcec8318c2ce}.png}

\end{sphinxuseclass}\end{sphinxVerbatimOutput}

\end{sphinxuseclass}

\subsection{Or it can be used in a dashboard (not avaiable in the offline manual)}
\label{\detokenize{content/06_ModelAnalytics/Attribution:or-it-can-be-used-in-a-dashboard-not-avaiable-in-the-offline-manual}}
\begin{sphinxuseclass}{cell}\begin{sphinxVerbatimInput}

\begin{sphinxuseclass}{cell_input}
\begin{sphinxVerbatim}[commandchars=\\\{\}]
\PYG{k}{try}\PYG{p}{:}
    \PYG{n}{mpak}\PYG{o}{.}\PYG{n}{modeldash}\PYG{p}{(}\PYG{l+s+s1}{\PYGZsq{}}\PYG{l+s+s1}{PAKNYGDPMKTPKN}\PYG{l+s+s1}{\PYGZsq{}}\PYG{p}{,}\PYG{n}{jupyter}\PYG{o}{=}\PYG{l+m+mi}{1}\PYG{p}{,}\PYG{n}{inline}\PYG{o}{=}\PYG{k+kc}{False}\PYG{p}{)}
\PYG{k}{except}\PYG{p}{:} 
    \PYG{n+nb}{print}\PYG{p}{(}\PYG{l+s+s1}{\PYGZsq{}}\PYG{l+s+s1}{No Dashboard installed}\PYG{l+s+s1}{\PYGZsq{}}\PYG{p}{)}
\end{sphinxVerbatim}

\end{sphinxuseclass}\end{sphinxVerbatimInput}
\begin{sphinxVerbatimOutput}

\begin{sphinxuseclass}{cell_output}
\begin{sphinxVerbatim}[commandchars=\\\{\}]
apprun
Dash is running on http://127.0.0.1:5001/
\end{sphinxVerbatim}

\begin{sphinxVerbatim}[commandchars=\\\{\}]
Dash app running on http://127.0.0.1:5001/
\end{sphinxVerbatim}

\end{sphinxuseclass}\end{sphinxVerbatimOutput}

\end{sphinxuseclass}
\sphinxstepscope

\begin{sphinxuseclass}{cell}\begin{sphinxVerbatimInput}

\begin{sphinxuseclass}{cell_input}
\begin{sphinxVerbatim}[commandchars=\\\{\}]
\PYG{o}{\PYGZpc{}}\PYG{k}{matplotlib} inline
\end{sphinxVerbatim}

\end{sphinxuseclass}\end{sphinxVerbatimInput}

\end{sphinxuseclass}

\chapter{Model eigenvalues}
\label{\detokenize{content/06_ModelAnalytics/EigenValues:model-eigenvalues}}\label{\detokenize{content/06_ModelAnalytics/EigenValues::doc}}
\sphinxAtStartPar
Eigenvalues are a fundamental concept in dynamic models. In simple terms, they summarize the adjustment process within a model. In the context of dynamic models, the eigenvalues of the model describe the behavior of the system over time. The sign and magnitude of the eigenvalues determine whether a system of equations will converge to a stable equilibrium, oscillate, or diverge. For macro models they determine whether the model is stable, marginally stable, or unstable.

\sphinxAtStartPar
In the case of a macromodel, which is effectively a system of differential equations, the eigenvalues of the coefficient matrix determine whether the system is stable or unstable. If all the eigenvalues have negative real parts, then the system is stable and will converge to a steady state over time. If at least one eigenvalue has a positive real part, then the system is unstable, and the solutions will diverge over time.

\sphinxAtStartPar
This Notebook uses a  model for Pakistan described here: 


\section{Imports}
\label{\detokenize{content/06_ModelAnalytics/EigenValues:imports}}
\sphinxAtStartPar
Modelflow’s modelclass includes most of the methods needed to manage a model in Modelflow.

\begin{sphinxuseclass}{cell}\begin{sphinxVerbatimInput}

\begin{sphinxuseclass}{cell_input}
\begin{sphinxVerbatim}[commandchars=\\\{\}]
\PYG{k+kn}{from} \PYG{n+nn}{modelclass} \PYG{k+kn}{import} \PYG{n}{model} 
\PYG{k+kn}{from} \PYG{n+nn}{modelnewton} \PYG{k+kn}{import} \PYG{n}{newton\PYGZus{}diff}
\PYG{k+kn}{import} \PYG{n+nn}{modelmf}
\PYG{n}{model}\PYG{o}{.}\PYG{n}{widescreen}\PYG{p}{(}\PYG{p}{)}
\PYG{n}{model}\PYG{o}{.}\PYG{n}{scroll\PYGZus{}off}\PYG{p}{(}\PYG{p}{)}
\end{sphinxVerbatim}

\end{sphinxuseclass}\end{sphinxVerbatimInput}
\begin{sphinxVerbatimOutput}

\begin{sphinxuseclass}{cell_output}
\begin{sphinxVerbatim}[commandchars=\\\{\}]
\PYGZlt{}IPython.core.display.HTML object\PYGZgt{}
\end{sphinxVerbatim}

\end{sphinxuseclass}\end{sphinxVerbatimOutput}

\end{sphinxuseclass}

\section{Load a pre\sphinxhyphen{}existing model, data and descriptions}
\label{\detokenize{content/06_ModelAnalytics/EigenValues:load-a-pre-existing-model-data-and-descriptions}}
\sphinxAtStartPar
The file \sphinxcode{\sphinxupquote{pak.pcim}} contains a dump of model equations, dataframe, simulation options and variable descriptions. The file has been created when onboarding the model.
Examples can be found \DUrole{xref,myst}{here}

\begin{sphinxuseclass}{cell}\begin{sphinxVerbatimInput}

\begin{sphinxuseclass}{cell_input}
\begin{sphinxVerbatim}[commandchars=\\\{\}]
\PYG{n}{mpak}\PYG{p}{,}\PYG{n}{baseline} \PYG{o}{=} \PYG{n}{model}\PYG{o}{.}\PYG{n}{modelload}\PYG{p}{(}\PYG{l+s+s1}{\PYGZsq{}}\PYG{l+s+s1}{../../models/pak.pcim}\PYG{l+s+s1}{\PYGZsq{}}\PYG{p}{,}\PYG{n}{alfa}\PYG{o}{=}\PYG{l+m+mf}{0.7}\PYG{p}{,}\PYG{n}{run}\PYG{o}{=}\PYG{l+m+mi}{1}\PYG{p}{)}
\end{sphinxVerbatim}

\end{sphinxuseclass}\end{sphinxVerbatimInput}
\begin{sphinxVerbatimOutput}

\begin{sphinxuseclass}{cell_output}
\begin{sphinxVerbatim}[commandchars=\\\{\}]
Open file from URL:  https://raw.githubusercontent.com/IbHansen/modelflow\PYGZhy{}manual/main/model\PYGZus{}repo/pak.pcim
\end{sphinxVerbatim}

\end{sphinxuseclass}\end{sphinxVerbatimOutput}

\end{sphinxuseclass}
\begin{sphinxuseclass}{cell}\begin{sphinxVerbatimInput}

\begin{sphinxuseclass}{cell_input}
\begin{sphinxVerbatim}[commandchars=\\\{\}]
\PYG{n}{mpak\PYGZus{}newton} \PYG{o}{=} \PYG{n}{newton\PYGZus{}diff}\PYG{p}{(}\PYG{n}{mpak}\PYG{p}{,}\PYG{n}{forcenum}\PYG{o}{=}\PYG{l+m+mi}{0}\PYG{p}{,}\PYG{n}{ljit}\PYG{o}{=}\PYG{k+kc}{False}\PYG{p}{)}      \PYG{c+c1}{\PYGZsh{} create a newton\PYGZus{}diff instance which contains derivatives }
\end{sphinxVerbatim}

\end{sphinxuseclass}\end{sphinxVerbatimInput}

\end{sphinxuseclass}
\begin{sphinxuseclass}{cell}\begin{sphinxVerbatimInput}

\begin{sphinxuseclass}{cell_input}
\begin{sphinxVerbatim}[commandchars=\\\{\}]
\PYG{n}{eig\PYGZus{}dic} \PYG{o}{=} \PYG{n}{mpak\PYGZus{}newton}\PYG{o}{.}\PYG{n}{get\PYGZus{}eigenvectors}\PYG{p}{(}\PYG{n}{filnan} \PYG{o}{=} \PYG{k+kc}{True}\PYG{p}{,}\PYG{n}{periode}\PYG{o}{=} \PYG{p}{(}\PYG{l+m+mi}{2023}\PYG{p}{,}\PYG{l+m+mi}{2024}\PYG{p}{,}\PYG{l+m+mi}{2027}\PYG{p}{,}\PYG{l+m+mi}{2099}\PYG{p}{)}\PYG{p}{,}\PYG{n}{silent}\PYG{o}{=}\PYG{k+kc}{True}\PYG{p}{)} \PYG{c+c1}{\PYGZsh{}  }
\PYG{n}{mpak\PYGZus{}newton}\PYG{o}{.}\PYG{n}{eigplot\PYGZus{}all}\PYG{p}{(}\PYG{n}{eig\PYGZus{}dic}\PYG{p}{,}\PYG{n}{size}\PYG{o}{=}\PYG{p}{(}\PYG{l+m+mi}{3}\PYG{p}{,}\PYG{l+m+mi}{3}\PYG{p}{)}\PYG{p}{)}\PYG{p}{;}
\end{sphinxVerbatim}

\end{sphinxuseclass}\end{sphinxVerbatimInput}
\begin{sphinxVerbatimOutput}

\begin{sphinxuseclass}{cell_output}
\begin{sphinxVerbatim}[commandchars=\\\{\}]
C:\PYGZbs{}modelflow2\PYGZbs{}modelflow\PYGZbs{}modelnewton.py:785: UserWarning: The figure layout has changed to tight
  fig.tight\PYGZus{}layout()
\end{sphinxVerbatim}

\noindent\sphinxincludegraphics{{bbaa0fd72aed589b2df94ce1f5d04c65896a1ca9a4ca0412a7c430ab273e8d6a}.png}

\end{sphinxuseclass}\end{sphinxVerbatimOutput}

\end{sphinxuseclass}
\begin{sphinxuseclass}{cell}\begin{sphinxVerbatimInput}

\begin{sphinxuseclass}{cell_input}
\begin{sphinxVerbatim}[commandchars=\\\{\}]
\PYG{n}{help}\PYG{p}{(}\PYG{n}{mpak\PYGZus{}newton}\PYG{p}{)}
\end{sphinxVerbatim}

\end{sphinxuseclass}\end{sphinxVerbatimInput}
\begin{sphinxVerbatimOutput}

\begin{sphinxuseclass}{cell_output}
\begin{sphinxVerbatim}[commandchars=\\\{\}]
Help on newton\PYGZus{}diff in module modelnewton object:

class newton\PYGZus{}diff(builtins.object)
 |  newton\PYGZus{}diff(mmodel, df=None, endovar=None, onlyendocur=False, timeit=False, silent=True, forcenum=False, per=\PYGZsq{}\PYGZsq{}, ljit=0, nchunk=None, endoandexo=False)
 |  
 |  Class to handle newron solving 
 |  this is for un\PYGZhy{}nomalized or normalized models ie models of the forrm 
 |  
 |  0 = G(y,x)
 |  y = F(y,x)
 |  
 |  Methods defined here:
 |  
 |  \PYGZus{}\PYGZus{}init\PYGZus{}\PYGZus{}(self, mmodel, df=None, endovar=None, onlyendocur=False, timeit=False, silent=True, forcenum=False, per=\PYGZsq{}\PYGZsq{}, ljit=0, nchunk=None, endoandexo=False)
 |      Args:
 |          mmodel (TYPE): Model to analyze.
 |          df (TYPE, optional): Dataframe. if None mmodel.lastdf will be used 
 |          endovar (TYPE, optional): if set defines which endogeneous to include . Defaults to None.
 |          onlyendocur (TYPE, optional): Only calculate for the curren endogeneous variables. Defaults to False.
 |          timeit (TYPE, optional): writeout time informations . Defaults to False.
 |          silent (TYPE, optional):  Defaults to True.
 |          forcenum (TYPE, optional): Force differentiation to be numeric else try sumbolic (slower)  Defaults to False.
 |          per (TYPE, optional): Period for which to calculte the jacobi . Defaults to \PYGZsq{}\PYGZsq{}.
 |          ljit (TYPE, optional): Trigger just in time compilation of the differential coiefficient. Defaults to 0.
 |          nchunk (TYPE, optional): Chunks for which the model is written  \PYGZhy{} relevant if ljit == True. Defaults to None.
 |          endoandexo (TYPE, optional): Calculate for both endogeneous and exogeneous . Defaults to False.
 |      
 |      Returns:
 |          None.
 |  
 |  eigenvector\PYGZus{}plot(self, per=None, size=(4, 3), top=0.9)
 |  
 |  eigplot(self, eig\PYGZus{}dic=None, per=None, size=(4, 3), top=0.9)
 |  
 |  eigplot\PYGZus{}all(self, eig\PYGZus{}dic, size=(4, 3), maxfig=6)
 |  
 |  eigplot\PYGZus{}all0(self, eig\PYGZus{}dic, size=(4, 3))
 |  
 |  get\PYGZus{}df\PYGZus{}eigen\PYGZus{}dict(self)
 |  
 |  get\PYGZus{}diff\PYGZus{}df\PYGZus{}1per(self, df=None, periode=None)
 |  
 |  get\PYGZus{}diff\PYGZus{}df\PYGZus{}tot(self, periode=None, df=None)
 |  
 |  get\PYGZus{}diff\PYGZus{}mat\PYGZus{}1per(self, periode=None, df=None)
 |      fetch a dict of one periode sparse jacobimatrices
 |  
 |  get\PYGZus{}diff\PYGZus{}mat\PYGZus{}all\PYGZus{}1per(self, periode=None, df=None, asdf=False)
 |  
 |  get\PYGZus{}diff\PYGZus{}mat\PYGZus{}tot(self, df=None)
 |      Fetch a stacked jacobimatrix for the whole model.current\PYGZus{}per
 |      
 |      Returns a sparse matrix.
 |  
 |  get\PYGZus{}diff\PYGZus{}melted(self, periode=None, df=None)
 |      returns a tall matrix with all values to construct jacobimatrix(es)
 |  
 |  get\PYGZus{}diff\PYGZus{}melted\PYGZus{}var(self, periode=None, df=None)
 |      makes dict with all  derivative matrices for all lags
 |  
 |  get\PYGZus{}diff\PYGZus{}values\PYGZus{}all(self, periode=None, df=None, asdf=False)
 |      stuff the values of derivatives into nested dic
 |  
 |  get\PYGZus{}diffmodel(self)
 |      Returns a model which calculates the partial derivatives of a model
 |  
 |  get\PYGZus{}eigenvectors(self, periode=None, asdf=True, filnan=False, silent=False)
 |  
 |  get\PYGZus{}solve1per(self, df=None, periode=None)
 |  
 |  get\PYGZus{}solve1perlu(self, df=\PYGZsq{}\PYGZsq{}, periode=\PYGZsq{}\PYGZsq{})
 |  
 |  get\PYGZus{}solvestacked(self, df=\PYGZsq{}\PYGZsq{})
 |  
 |  get\PYGZus{}solvestacked\PYGZus{}it(self, df=\PYGZsq{}\PYGZsq{}, solver=\PYGZlt{}function bicg at 0x00000248B6EDCB80\PYGZgt{})
 |  
 |  modeldiff(self)
 |      Differentiate relations for self.enovar with respect to endogeneous variable 
 |      The result is placed in a dictory in the model instanse: model.diffendocur
 |  
 |  show\PYGZus{}diff(self, pat=\PYGZsq{}*\PYGZsq{})
 |      Displays espressions for differential koifficients for a variable
 |      if var ends with * all matchning variables are displayes
 |  
 |  show\PYGZus{}diff\PYGZus{}latex(self, pat=\PYGZsq{}*\PYGZsq{}, show\PYGZus{}expression=True, show\PYGZus{}values=True, maxper=5)
 |  
 |  show\PYGZus{}stacked\PYGZus{}diff(self, time=None, lhs=\PYGZsq{}\PYGZsq{}, rhs=\PYGZsq{}\PYGZsq{}, dec=2, show=True)
 |      Parameters
 |      \PYGZhy{}\PYGZhy{}\PYGZhy{}\PYGZhy{}\PYGZhy{}\PYGZhy{}\PYGZhy{}\PYGZhy{}\PYGZhy{}\PYGZhy{}
 |      time : list, optional
 |          DESCRIPTION. The default is None. Time for which to retrieve stacked jacobi
 |      lhs : string, optional
 |          DESCRIPTION. The default is \PYGZsq{}\PYGZsq{}. Left hand side variables 
 |      rhs : TYPE, optional
 |          DESCRIPTION. The default is \PYGZsq{}\PYGZsq{}. Right hand side variabnles 
 |      dec : TYPE, optional
 |          DESCRIPTION. The default is 2.
 |      show : TYPE, optional
 |          DESCRIPTION. The default is True.
 |      
 |      Returns
 |      \PYGZhy{}\PYGZhy{}\PYGZhy{}\PYGZhy{}\PYGZhy{}\PYGZhy{}\PYGZhy{}
 |      selected rows and columns of stacked jacobi as dataframe .
 |  
 |  \PYGZhy{}\PYGZhy{}\PYGZhy{}\PYGZhy{}\PYGZhy{}\PYGZhy{}\PYGZhy{}\PYGZhy{}\PYGZhy{}\PYGZhy{}\PYGZhy{}\PYGZhy{}\PYGZhy{}\PYGZhy{}\PYGZhy{}\PYGZhy{}\PYGZhy{}\PYGZhy{}\PYGZhy{}\PYGZhy{}\PYGZhy{}\PYGZhy{}\PYGZhy{}\PYGZhy{}\PYGZhy{}\PYGZhy{}\PYGZhy{}\PYGZhy{}\PYGZhy{}\PYGZhy{}\PYGZhy{}\PYGZhy{}\PYGZhy{}\PYGZhy{}\PYGZhy{}\PYGZhy{}\PYGZhy{}\PYGZhy{}\PYGZhy{}\PYGZhy{}\PYGZhy{}\PYGZhy{}\PYGZhy{}\PYGZhy{}\PYGZhy{}\PYGZhy{}\PYGZhy{}\PYGZhy{}\PYGZhy{}\PYGZhy{}\PYGZhy{}\PYGZhy{}\PYGZhy{}\PYGZhy{}\PYGZhy{}\PYGZhy{}\PYGZhy{}\PYGZhy{}\PYGZhy{}\PYGZhy{}\PYGZhy{}\PYGZhy{}\PYGZhy{}\PYGZhy{}\PYGZhy{}\PYGZhy{}\PYGZhy{}\PYGZhy{}\PYGZhy{}\PYGZhy{}
 |  Static methods defined here:
 |  
 |  get\PYGZus{}feedback(eig\PYGZus{}dic, per=None)
 |      Returns a dict of max abs eigenvector and the sign
 |  
 |  \PYGZhy{}\PYGZhy{}\PYGZhy{}\PYGZhy{}\PYGZhy{}\PYGZhy{}\PYGZhy{}\PYGZhy{}\PYGZhy{}\PYGZhy{}\PYGZhy{}\PYGZhy{}\PYGZhy{}\PYGZhy{}\PYGZhy{}\PYGZhy{}\PYGZhy{}\PYGZhy{}\PYGZhy{}\PYGZhy{}\PYGZhy{}\PYGZhy{}\PYGZhy{}\PYGZhy{}\PYGZhy{}\PYGZhy{}\PYGZhy{}\PYGZhy{}\PYGZhy{}\PYGZhy{}\PYGZhy{}\PYGZhy{}\PYGZhy{}\PYGZhy{}\PYGZhy{}\PYGZhy{}\PYGZhy{}\PYGZhy{}\PYGZhy{}\PYGZhy{}\PYGZhy{}\PYGZhy{}\PYGZhy{}\PYGZhy{}\PYGZhy{}\PYGZhy{}\PYGZhy{}\PYGZhy{}\PYGZhy{}\PYGZhy{}\PYGZhy{}\PYGZhy{}\PYGZhy{}\PYGZhy{}\PYGZhy{}\PYGZhy{}\PYGZhy{}\PYGZhy{}\PYGZhy{}\PYGZhy{}\PYGZhy{}\PYGZhy{}\PYGZhy{}\PYGZhy{}\PYGZhy{}\PYGZhy{}\PYGZhy{}\PYGZhy{}\PYGZhy{}\PYGZhy{}
 |  Data descriptors defined here:
 |  
 |  \PYGZus{}\PYGZus{}dict\PYGZus{}\PYGZus{}
 |      dictionary for instance variables (if defined)
 |  
 |  \PYGZus{}\PYGZus{}weakref\PYGZus{}\PYGZus{}
 |      list of weak references to the object (if defined)
\end{sphinxVerbatim}

\end{sphinxuseclass}\end{sphinxVerbatimOutput}

\end{sphinxuseclass}
\begin{sphinxuseclass}{cell}\begin{sphinxVerbatimInput}

\begin{sphinxuseclass}{cell_input}
\begin{sphinxVerbatim}[commandchars=\\\{\}]
mpak\PYGZus{}newton.get\PYGZus{}eigenvectors\PYG{o}{?}
\end{sphinxVerbatim}

\end{sphinxuseclass}\end{sphinxVerbatimInput}

\end{sphinxuseclass}
\begin{sphinxuseclass}{cell}\begin{sphinxVerbatimInput}

\begin{sphinxuseclass}{cell_input}
\begin{sphinxVerbatim}[commandchars=\\\{\}]
\PYG{n}{eig\PYGZus{}dic}
\end{sphinxVerbatim}

\end{sphinxuseclass}\end{sphinxVerbatimInput}
\begin{sphinxVerbatimOutput}

\begin{sphinxuseclass}{cell_output}
\begin{sphinxVerbatim}[commandchars=\\\{\}]
\PYGZob{}2023: array([0.+0.j, 0.+0.j, 0.+0.j, ..., 0.+0.j, 0.+0.j, 0.+0.j]),
 2024: array([0.+0.j, 0.+0.j, 0.+0.j, ..., 0.+0.j, 0.+0.j, 0.+0.j]),
 2027: array([0.+0.j, 0.+0.j, 0.+0.j, ..., 0.+0.j, 0.+0.j, 0.+0.j]),
 2099: array([0.+0.j, 0.+0.j, 0.+0.j, ..., 0.+0.j, 0.+0.j, 0.+0.j])\PYGZcb{}
\end{sphinxVerbatim}

\end{sphinxuseclass}\end{sphinxVerbatimOutput}

\end{sphinxuseclass}
\begin{sphinxuseclass}{cell}\begin{sphinxVerbatimInput}

\begin{sphinxuseclass}{cell_input}
\begin{sphinxVerbatim}[commandchars=\\\{\}]
\PYG{c+c1}{\PYGZsh{}e2027=eig\PYGZus{}dic[2023]}
\PYG{c+c1}{\PYGZsh{}e2027[5]}
\PYG{c+c1}{\PYGZsh{}}
\PYG{c+c1}{\PYGZsh{}import pandas as pd}
\PYG{c+c1}{\PYGZsh{}df=pd.DataFrame(e2027)}
\PYG{c+c1}{\PYGZsh{}}
\PYG{c+c1}{\PYGZsh{}df.tail(10)}
\end{sphinxVerbatim}

\end{sphinxuseclass}\end{sphinxVerbatimInput}

\end{sphinxuseclass}
\sphinxstepscope


\part{Technical how tos}

\sphinxstepscope

\begin{sphinxuseclass}{cell}\begin{sphinxVerbatimInput}

\begin{sphinxuseclass}{cell_input}
\begin{sphinxVerbatim}[commandchars=\\\{\}]
\PYG{o}{\PYGZpc{}}\PYG{k}{matplotlib} inline
\end{sphinxVerbatim}

\end{sphinxuseclass}\end{sphinxVerbatimInput}

\end{sphinxuseclass}

\chapter{Getting Help}
\label{\detokenize{content/07_MoreFeatures/GettingHelp:getting-help}}\label{\detokenize{content/07_MoreFeatures/GettingHelp::doc}}
\sphinxAtStartPar
mpak.fix?
mpak.fix??

\sphinxstepscope


\part{Features}

\sphinxstepscope


\chapter{Useful model instance properties and methods}
\label{\detokenize{content/notebooks/modelflow_features:useful-model-instance-properties-and-methods}}\label{\detokenize{content/notebooks/modelflow_features::doc}}
\sphinxAtStartPar
The focus of this chapter is to introduce some properties and methods of the model instance.

\sphinxAtStartPar
First a model and data is loaded, then a scenario is run. Then we have some content to use.

\sphinxAtStartPar
A model instance gives the user access to a number of properties and methods which helps in managing the model and its results.

\sphinxAtStartPar
If \sphinxcode{\sphinxupquote{mmodel}} is a model instance \sphinxcode{\sphinxupquote{mmodel.<property>}} will return a property. Some properties can also be assigned by the user just by:
\begin{quote}

\sphinxAtStartPar
mmodel.property = something
\end{quote}

\sphinxAtStartPar
The model class itself also have a few properties. These are simple accessed by  \sphinxcode{\sphinxupquote{model.<property>}}.

\sphinxAtStartPar
Enjoy


\section{Import the model class}
\label{\detokenize{content/notebooks/modelflow_features:import-the-model-class}}
\sphinxAtStartPar
This class incorporates most of the methods used to manage a model.

\sphinxAtStartPar
Assuming the ModelFlow library has been installed on your machine, the following imports set up your notebook so that you can run the cells in this notebook.

\sphinxAtStartPar
In order to manipulate plots later on matplotlib.pyplot is also imported.

\begin{sphinxuseclass}{cell}\begin{sphinxVerbatimInput}

\begin{sphinxuseclass}{cell_input}
\begin{sphinxVerbatim}[commandchars=\\\{\}]
\PYG{c+c1}{\PYGZsh{}\PYGZpc{}matplotlib notebook}
\PYG{o}{\PYGZpc{}}\PYG{k}{matplotlib} inline
\end{sphinxVerbatim}

\end{sphinxuseclass}\end{sphinxVerbatimInput}

\end{sphinxuseclass}
\begin{sphinxuseclass}{cell}\begin{sphinxVerbatimInput}

\begin{sphinxuseclass}{cell_input}
\begin{sphinxVerbatim}[commandchars=\\\{\}]
\PYG{k+kn}{from} \PYG{n+nn}{modelclass} \PYG{k+kn}{import} \PYG{n}{model} 
\end{sphinxVerbatim}

\end{sphinxuseclass}\end{sphinxVerbatimInput}

\end{sphinxuseclass}
\begin{sphinxuseclass}{cell}\begin{sphinxVerbatimInput}

\begin{sphinxuseclass}{cell_input}
\begin{sphinxVerbatim}[commandchars=\\\{\}]
\PYG{k+kn}{import} \PYG{n+nn}{matplotlib}\PYG{n+nn}{.}\PYG{n+nn}{pyplot} \PYG{k}{as} \PYG{n+nn}{plt} \PYG{c+c1}{\PYGZsh{} To manipulate plots }
\end{sphinxVerbatim}

\end{sphinxuseclass}\end{sphinxVerbatimInput}

\end{sphinxuseclass}

\section{Class methods to help in Jupyter Notebook}
\label{\detokenize{content/notebooks/modelflow_features:class-methods-to-help-in-jupyter-notebook}}

\subsection{.widescreen() use Jupyter Notebook in widescreen}
\label{\detokenize{content/notebooks/modelflow_features:widescreen-use-jupyter-notebook-in-widescreen}}
\sphinxAtStartPar
Enables the whole viewing area of the browser.

\begin{sphinxuseclass}{cell}\begin{sphinxVerbatimInput}

\begin{sphinxuseclass}{cell_input}
\begin{sphinxVerbatim}[commandchars=\\\{\}]
\PYG{n}{model}\PYG{o}{.}\PYG{n}{widescreen}\PYG{p}{(}\PYG{p}{)} 
\end{sphinxVerbatim}

\end{sphinxuseclass}\end{sphinxVerbatimInput}
\begin{sphinxVerbatimOutput}

\begin{sphinxuseclass}{cell_output}
\begin{sphinxVerbatim}[commandchars=\\\{\}]
\PYGZlt{}IPython.core.display.HTML object\PYGZgt{}
\end{sphinxVerbatim}

\end{sphinxuseclass}\end{sphinxVerbatimOutput}

\end{sphinxuseclass}

\subsection{.scroll\_off() Turn off scroll cells in Jupyter Notebook}
\label{\detokenize{content/notebooks/modelflow_features:scroll-off-turn-off-scroll-cells-in-jupyter-notebook}}
\sphinxAtStartPar
Can be useful

\begin{sphinxuseclass}{cell}\begin{sphinxVerbatimInput}

\begin{sphinxuseclass}{cell_input}
\begin{sphinxVerbatim}[commandchars=\\\{\}]
\PYG{n}{model}\PYG{o}{.}\PYG{n}{scroll\PYGZus{}off}\PYG{p}{(}\PYG{p}{)}
\end{sphinxVerbatim}

\end{sphinxuseclass}\end{sphinxVerbatimInput}

\end{sphinxuseclass}

\section{.modelload Load a pre\sphinxhyphen{}cooked model, data and descriptions}
\label{\detokenize{content/notebooks/modelflow_features:modelload-load-a-pre-cooked-model-data-and-descriptions}}
\sphinxAtStartPar
In this notebook, we will be using a pre\sphinxhyphen{}existing  model of Pakistan.

\sphinxAtStartPar
The file ‘pak.pcim’ has been created from a Eviews workspace. It contains all that is needed to run the model:
\begin{itemize}
\item {} 
\sphinxAtStartPar
Model equations

\item {} 
\sphinxAtStartPar
Data

\item {} 
\sphinxAtStartPar
Simulation options

\item {} 
\sphinxAtStartPar
Variable descriptions

\end{itemize}

\sphinxAtStartPar
Using the ‘modelload’ method of the  ‘model’ class, a model instance ‘mpak’ and a ‘result’ DataFrame is created.

\begin{sphinxuseclass}{cell}\begin{sphinxVerbatimInput}

\begin{sphinxuseclass}{cell_input}
\begin{sphinxVerbatim}[commandchars=\\\{\}]
\PYG{n}{mpak}\PYG{p}{,}\PYG{n}{baseline} \PYG{o}{=} \PYG{n}{model}\PYG{o}{.}\PYG{n}{modelload}\PYG{p}{(}\PYG{l+s+s1}{\PYGZsq{}}\PYG{l+s+s1}{../models/pak.pcim}\PYG{l+s+s1}{\PYGZsq{}}\PYG{p}{,}\PYG{n}{run}\PYG{o}{=}\PYG{l+m+mi}{1}\PYG{p}{,}\PYG{n}{silent}\PYG{o}{=}\PYG{l+m+mi}{1}\PYG{p}{,}\PYG{n}{keep}\PYG{o}{=}\PYG{l+s+s1}{\PYGZsq{}}\PYG{l+s+s1}{Baseline}\PYG{l+s+s1}{\PYGZsq{}}\PYG{p}{)}
\end{sphinxVerbatim}

\end{sphinxuseclass}\end{sphinxVerbatimInput}
\begin{sphinxVerbatimOutput}

\begin{sphinxuseclass}{cell_output}
\begin{sphinxVerbatim}[commandchars=\\\{\}]
file read:  C:\PYGZbs{}modelflow manual\PYGZbs{}papers\PYGZbs{}mfbook\PYGZbs{}content\PYGZbs{}models\PYGZbs{}pak.pcim
\end{sphinxVerbatim}

\end{sphinxuseclass}\end{sphinxVerbatimOutput}

\end{sphinxuseclass}
\sphinxAtStartPar
\sphinxstylestrong{mpak} 
The \sphinxstyleemphasis{modelload} method processes the file and initiates the model, that we call ‘mpak’ (m for model and pak for Pakistan) with both equations and the data.

\sphinxAtStartPar
‘mpak’ is an \sphinxstylestrong{instance}  of the  model object with which we will work.

\sphinxAtStartPar
\sphinxstylestrong{baseline}  
‘result’ is a Pandas dataframe containing the data that was loaded.

\sphinxAtStartPar
\sphinxstylestrong{run=1} the model is simulated. The simulation timeframe  and options from the time the file where dumped will be used. The two objects \sphinxstylestrong{mpak.basedf} and \sphinxstylestrong{mpak.lastdf} will contain the simulation result. If run=0 the model will not be simulated.

\sphinxAtStartPar
\sphinxstylestrong{silent=1} if silent is set to 0  information regarding the simulation will be displayed.

\sphinxAtStartPar
\sphinxstylestrong{keep=’Baseline’} This saves the result in a dictionary mpak.keep\_solutions.


\section{Create a scenario}
\label{\detokenize{content/notebooks/modelflow_features:create-a-scenario}}
\sphinxAtStartPar
Many objects relates to comparison of different scenarios. So first a scenario is created by updating some exogenous variables.
In this case the carbon tax rates for gas, oil and coal are all set to 29 from 2023 to 2100. Then the scenario is simulated.
Now the mpak object contains a number of useful properties and methods.

\sphinxAtStartPar
You can find more on this experiment \DUrole{xref,myst}{here}

\begin{sphinxuseclass}{cell}\begin{sphinxVerbatimInput}

\begin{sphinxuseclass}{cell_input}
\begin{sphinxVerbatim}[commandchars=\\\{\}]
\PYG{n}{scenario\PYGZus{}exo}  \PYG{o}{=}  \PYG{n}{baseline}\PYG{o}{.}\PYG{n}{upd}\PYG{p}{(}\PYG{l+s+s2}{\PYGZdq{}}\PYG{l+s+s2}{\PYGZlt{}2020 2100\PYGZgt{} PAKGGREVCO2CER PAKGGREVCO2GER PAKGGREVCO2OER = 29}\PYG{l+s+s2}{\PYGZdq{}}\PYG{p}{)}
\end{sphinxVerbatim}

\end{sphinxuseclass}\end{sphinxVerbatimInput}

\end{sphinxuseclass}

\section{() Simulate on a dataframe}
\label{\detokenize{content/notebooks/modelflow_features:simulate-on-a-dataframe}}
\sphinxAtStartPar
When calling the model instance like \sphinxcode{\sphinxupquote{mpak(dataframe,start, end)}} the model will be simulated for the time frame \sphinxcode{\sphinxupquote{start to end}} using the dataframe.  
Just above we created a dataframe \sphinxcode{\sphinxupquote{scenario\_exo}} where the tax variables are updated. Now the \sphinxcode{\sphinxupquote{mpak}} can be simulated. We simulate from 2020 to 2100.

\begin{sphinxuseclass}{cell}\begin{sphinxVerbatimInput}

\begin{sphinxuseclass}{cell_input}
\begin{sphinxVerbatim}[commandchars=\\\{\}]
\PYG{n}{scenario} \PYG{o}{=} \PYG{n}{mpak}\PYG{p}{(}\PYG{n}{scenario\PYGZus{}exo}\PYG{p}{,}\PYG{l+m+mi}{2020}\PYG{p}{,}\PYG{l+m+mi}{2100}\PYG{p}{,}\PYG{n}{keep}\PYG{o}{=}\PYG{l+s+sa}{f}\PYG{l+s+s1}{\PYGZsq{}}\PYG{l+s+s1}{Coal, Oil and Gastax : 29}\PYG{l+s+s1}{\PYGZsq{}}\PYG{p}{)} \PYG{c+c1}{\PYGZsh{} runs the simulation}
\end{sphinxVerbatim}

\end{sphinxuseclass}\end{sphinxVerbatimInput}

\end{sphinxuseclass}

\section{Access results}
\label{\detokenize{content/notebooks/modelflow_features:access-results}}
\sphinxAtStartPar
Now we have two dataframes with results \sphinxcode{\sphinxupquote{baseline}} and \sphinxcode{\sphinxupquote{scenario}}. These dataframes can be manipulated and visualized
with the tools provided by the \sphinxstylestrong{pandas} library and other like \sphinxstylestrong{Matplotlib} and \sphinxstylestrong{Plotly}. However to make things easy the first and
latest simulation result is also in the mpak object:
\begin{itemize}
\item {} 
\sphinxAtStartPar
\sphinxstylestrong{mpak.basedf}: Dataframe with the values for baseline

\item {} 
\sphinxAtStartPar
\sphinxstylestrong{mpak.lastdf}: Dataframe with the values for alternative

\end{itemize}

\sphinxAtStartPar
This means that .basedf and .lastdf will contain the same result after the first simulation. 
If new scenarios are simulated the data in .lastdf will then be replaced with the latest results.

\sphinxAtStartPar
These dataframes are used by a number of model instance methods as you will see later.

\sphinxAtStartPar
The user can assign dataframes to both .basedf and .lastdf. This is useful for comparing simulations which are not the first and last.

\begin{sphinxuseclass}{cell}\begin{sphinxVerbatimInput}

\begin{sphinxuseclass}{cell_input}
\begin{sphinxVerbatim}[commandchars=\\\{\}]
\PYG{n+nb}{print}\PYG{p}{(}\PYG{l+s+sa}{f}\PYG{l+s+s1}{\PYGZsq{}}\PYG{l+s+s1}{mpak.basedf: Dataframe: with }\PYG{l+s+si}{\PYGZob{}}\PYG{n}{mpak}\PYG{o}{.}\PYG{n}{basedf}\PYG{o}{.}\PYG{n}{shape}\PYG{p}{[}\PYG{l+m+mi}{0}\PYG{p}{]}\PYG{l+s+si}{\PYGZcb{}}\PYG{l+s+s1}{ years and }\PYG{l+s+si}{\PYGZob{}}\PYG{n}{mpak}\PYG{o}{.}\PYG{n}{basedf}\PYG{o}{.}\PYG{n}{shape}\PYG{p}{[}\PYG{l+m+mi}{1}\PYG{p}{]}\PYG{l+s+si}{\PYGZcb{}}\PYG{l+s+s1}{ variables}\PYG{l+s+s1}{\PYGZsq{}}\PYG{p}{)}
\PYG{n+nb}{print}\PYG{p}{(}\PYG{l+s+sa}{f}\PYG{l+s+s1}{\PYGZsq{}}\PYG{l+s+s1}{mpak.lastdf: Dataframe: with }\PYG{l+s+si}{\PYGZob{}}\PYG{n}{mpak}\PYG{o}{.}\PYG{n}{lastdf}\PYG{o}{.}\PYG{n}{shape}\PYG{p}{[}\PYG{l+m+mi}{0}\PYG{p}{]}\PYG{l+s+si}{\PYGZcb{}}\PYG{l+s+s1}{ years and }\PYG{l+s+si}{\PYGZob{}}\PYG{n}{mpak}\PYG{o}{.}\PYG{n}{lastdf}\PYG{o}{.}\PYG{n}{shape}\PYG{p}{[}\PYG{l+m+mi}{1}\PYG{p}{]}\PYG{l+s+si}{\PYGZcb{}}\PYG{l+s+s1}{ variables}\PYG{l+s+s1}{\PYGZsq{}}\PYG{p}{)}
\end{sphinxVerbatim}

\end{sphinxuseclass}\end{sphinxVerbatimInput}
\begin{sphinxVerbatimOutput}

\begin{sphinxuseclass}{cell_output}
\begin{sphinxVerbatim}[commandchars=\\\{\}]
mpak.basedf: Dataframe: with 121 years and 1290 variables
mpak.lastdf: Dataframe: with 121 years and 1290 variables
\end{sphinxVerbatim}

\end{sphinxuseclass}\end{sphinxVerbatimOutput}

\end{sphinxuseclass}

\subsection{.keep\_solutions, A dictionary of dataframes with results}
\label{\detokenize{content/notebooks/modelflow_features:keep-solutions-a-dictionary-of-dataframes-with-results}}
\sphinxAtStartPar
Create a dictionary of dataframes with .keep\_solutions. Sometimes we want to be able to compare more than two scenarios. Using \sphinxcode{\sphinxupquote{keep='some description'}} the dataframe with results can be saved into a dictionary with the description as key and the dataframe as value.

\sphinxAtStartPar
In our example we have created two scenarios. A baseline and a scenario with the tax set to 29. So mpak.keep\_solutions looks like this:

\begin{sphinxuseclass}{cell}\begin{sphinxVerbatimInput}

\begin{sphinxuseclass}{cell_input}
\begin{sphinxVerbatim}[commandchars=\\\{\}]
\PYG{n+nb}{print}\PYG{p}{(}\PYG{l+s+s1}{\PYGZsq{}}\PYG{l+s+s1}{mpak.keep\PYGZus{}solutions contains:}\PYG{l+s+s1}{\PYGZsq{}}\PYG{p}{)}
\PYG{k}{for} \PYG{n}{key}\PYG{p}{,}\PYG{n}{value} \PYG{o+ow}{in} \PYG{n}{mpak}\PYG{o}{.}\PYG{n}{keep\PYGZus{}solutions}\PYG{o}{.}\PYG{n}{items}\PYG{p}{(}\PYG{p}{)}\PYG{p}{:} 
    \PYG{n+nb}{print}\PYG{p}{(}\PYG{l+s+sa}{f}\PYG{l+s+s1}{\PYGZsq{}}\PYG{l+s+s1}{key = }\PYG{l+s+si}{\PYGZob{}}\PYG{n}{key}\PYG{l+s+si}{:}\PYG{l+s+s1}{25}\PYG{l+s+si}{\PYGZcb{}}\PYG{l+s+s1}{|Dataframe: }\PYG{l+s+si}{\PYGZob{}}\PYG{n}{value}\PYG{o}{.}\PYG{n}{shape}\PYG{p}{[}\PYG{l+m+mi}{0}\PYG{p}{]}\PYG{l+s+si}{\PYGZcb{}}\PYG{l+s+s1}{ years and }\PYG{l+s+si}{\PYGZob{}}\PYG{n}{value}\PYG{o}{.}\PYG{n}{shape}\PYG{p}{[}\PYG{l+m+mi}{1}\PYG{p}{]}\PYG{l+s+si}{\PYGZcb{}}\PYG{l+s+s1}{ variables}\PYG{l+s+s1}{\PYGZsq{}}\PYG{p}{)}
\end{sphinxVerbatim}

\end{sphinxuseclass}\end{sphinxVerbatimInput}
\begin{sphinxVerbatimOutput}

\begin{sphinxuseclass}{cell_output}
\begin{sphinxVerbatim}[commandchars=\\\{\}]
mpak.keep\PYGZus{}solutions contains:
key = Baseline                 |Dataframe: 121 years and 1290 variables
key = Coal, Oil and Gastax : 29|Dataframe: 121 years and 1290 variables
\end{sphinxVerbatim}

\end{sphinxuseclass}\end{sphinxVerbatimOutput}

\end{sphinxuseclass}
\sphinxAtStartPar
Sometime it can be useful to reset the \sphinxcode{\sphinxupquote{.keep\_solutions}}, so that a new set of solutions can be inspected. This is done by replacing it with an empty dictionary. Two methods can be used:
\begin{quote}

\sphinxAtStartPar
mpak.keep\_solutions = \{\}
\end{quote}

\sphinxAtStartPar
or in the simulation call:
\begin{quote}

\sphinxAtStartPar
mpak(,,keep=’’)
\end{quote}


\subsection{More on manipulating keep\_solution:}
\label{\detokenize{content/notebooks/modelflow_features:more-on-manipulating-keep-solution}}
\sphinxAtStartPar
\DUrole{xref,myst}{Here}


\subsection{.oldkwargs, Options in the simulation call is persistent between calls}
\label{\detokenize{content/notebooks/modelflow_features:oldkwargs-options-in-the-simulation-call-is-persistent-between-calls}}
\sphinxAtStartPar
When simulating a model the parameters are persistent. So the user just have to provide the
solution options once. These persistent parameters are located in the property .oldkwargs.

\sphinxAtStartPar
In this case the persistent parameters are:

\begin{sphinxuseclass}{cell}\begin{sphinxVerbatimInput}

\begin{sphinxuseclass}{cell_input}
\begin{sphinxVerbatim}[commandchars=\\\{\}]
\PYG{n}{mpak}\PYG{o}{.}\PYG{n}{oldkwargs}
\end{sphinxVerbatim}

\end{sphinxuseclass}\end{sphinxVerbatimInput}
\begin{sphinxVerbatimOutput}

\begin{sphinxuseclass}{cell_output}
\begin{sphinxVerbatim}[commandchars=\\\{\}]
\PYGZob{}\PYGZsq{}silent\PYGZsq{}: 1, \PYGZsq{}alfa\PYGZsq{}: 0.7, \PYGZsq{}ldumpvar\PYGZsq{}: 0, \PYGZsq{}keep\PYGZsq{}: \PYGZsq{}Coal, Oil and Gastax : 29\PYGZsq{}\PYGZcb{}
\end{sphinxVerbatim}

\end{sphinxuseclass}\end{sphinxVerbatimOutput}

\end{sphinxuseclass}
\sphinxAtStartPar
The user may have to reset the parameters, this is done like this:

\sphinxAtStartPar
To reset the options just do:
\begin{quote}

\sphinxAtStartPar
mpak.oldkwargs = \{\}
\end{quote}


\section{.current\_per, The time frame operations are performed on}
\label{\detokenize{content/notebooks/modelflow_features:current-per-the-time-frame-operations-are-performed-on}}
\sphinxAtStartPar
Most operations on a model class instance operates on the current time frame.
It is a subset of the row index of the dataframe which is simulated.

\sphinxAtStartPar
In this case it is:

\begin{sphinxuseclass}{cell}\begin{sphinxVerbatimInput}

\begin{sphinxuseclass}{cell_input}
\begin{sphinxVerbatim}[commandchars=\\\{\}]
\PYG{n}{mpak}\PYG{o}{.}\PYG{n}{current\PYGZus{}per}
\end{sphinxVerbatim}

\end{sphinxuseclass}\end{sphinxVerbatimInput}
\begin{sphinxVerbatimOutput}

\begin{sphinxuseclass}{cell_output}
\begin{sphinxVerbatim}[commandchars=\\\{\}]
Index([2020, 2021, 2022, 2023, 2024, 2025, 2026, 2027, 2028, 2029, 2030, 2031,
       2032, 2033, 2034, 2035, 2036, 2037, 2038, 2039, 2040, 2041, 2042, 2043,
       2044, 2045, 2046, 2047, 2048, 2049, 2050, 2051, 2052, 2053, 2054, 2055,
       2056, 2057, 2058, 2059, 2060, 2061, 2062, 2063, 2064, 2065, 2066, 2067,
       2068, 2069, 2070, 2071, 2072, 2073, 2074, 2075, 2076, 2077, 2078, 2079,
       2080, 2081, 2082, 2083, 2084, 2085, 2086, 2087, 2088, 2089, 2090, 2091,
       2092, 2093, 2094, 2095, 2096, 2097, 2098, 2099, 2100],
      dtype=\PYGZsq{}int64\PYGZsq{})
\end{sphinxVerbatim}

\end{sphinxuseclass}\end{sphinxVerbatimOutput}

\end{sphinxuseclass}
\sphinxAtStartPar
The possible times in the dataframe is contained in the \sphinxcode{\sphinxupquote{<dataframe>.index}} property.

\begin{sphinxuseclass}{cell}\begin{sphinxVerbatimInput}

\begin{sphinxuseclass}{cell_input}
\begin{sphinxVerbatim}[commandchars=\\\{\}]
\PYG{n}{scenario}\PYG{o}{.}\PYG{n}{index}  \PYG{c+c1}{\PYGZsh{} the index of the dataframe}
\end{sphinxVerbatim}

\end{sphinxuseclass}\end{sphinxVerbatimInput}
\begin{sphinxVerbatimOutput}

\begin{sphinxuseclass}{cell_output}
\begin{sphinxVerbatim}[commandchars=\\\{\}]
Index([1980, 1981, 1982, 1983, 1984, 1985, 1986, 1987, 1988, 1989,
       ...
       2091, 2092, 2093, 2094, 2095, 2096, 2097, 2098, 2099, 2100],
      dtype=\PYGZsq{}int64\PYGZsq{}, length=121)
\end{sphinxVerbatim}

\end{sphinxuseclass}\end{sphinxVerbatimOutput}

\end{sphinxuseclass}

\subsection{.smpl, Set time frame}
\label{\detokenize{content/notebooks/modelflow_features:smpl-set-time-frame}}
\sphinxAtStartPar
The time frame can be set like this:

\begin{sphinxuseclass}{cell}\begin{sphinxVerbatimInput}

\begin{sphinxuseclass}{cell_input}
\begin{sphinxVerbatim}[commandchars=\\\{\}]
\PYG{n}{mpak}\PYG{o}{.}\PYG{n}{smpl}\PYG{p}{(}\PYG{l+m+mi}{2020}\PYG{p}{,}\PYG{l+m+mi}{2025}\PYG{p}{)}
\PYG{n}{mpak}\PYG{o}{.}\PYG{n}{current\PYGZus{}per}
\end{sphinxVerbatim}

\end{sphinxuseclass}\end{sphinxVerbatimInput}
\begin{sphinxVerbatimOutput}

\begin{sphinxuseclass}{cell_output}
\begin{sphinxVerbatim}[commandchars=\\\{\}]
Index([2020, 2021, 2022, 2023, 2024, 2025], dtype=\PYGZsq{}int64\PYGZsq{})
\end{sphinxVerbatim}

\end{sphinxuseclass}\end{sphinxVerbatimOutput}

\end{sphinxuseclass}

\subsection{.set\_smpl, Set timeframe for a local scope}
\label{\detokenize{content/notebooks/modelflow_features:set-smpl-set-timeframe-for-a-local-scope}}
\sphinxAtStartPar
For many operations it can be useful to apply the operations for a shorter time frame, but retain the global time frame after the operation. 
This can be done  with a \sphinxcode{\sphinxupquote{with}} statement like this.

\begin{sphinxuseclass}{cell}\begin{sphinxVerbatimInput}

\begin{sphinxuseclass}{cell_input}
\begin{sphinxVerbatim}[commandchars=\\\{\}]
\PYG{n+nb}{print}\PYG{p}{(}\PYG{l+s+sa}{f}\PYG{l+s+s1}{\PYGZsq{}}\PYG{l+s+s1}{Global time  before   }\PYG{l+s+si}{\PYGZob{}}\PYG{n}{mpak}\PYG{o}{.}\PYG{n}{current\PYGZus{}per}\PYG{l+s+si}{\PYGZcb{}}\PYG{l+s+s1}{\PYGZsq{}}\PYG{p}{)}
\PYG{k}{with} \PYG{n}{mpak}\PYG{o}{.}\PYG{n}{set\PYGZus{}smpl}\PYG{p}{(}\PYG{l+m+mi}{2022}\PYG{p}{,}\PYG{l+m+mi}{2023}\PYG{p}{)}\PYG{p}{:}
    \PYG{n+nb}{print}\PYG{p}{(}\PYG{l+s+sa}{f}\PYG{l+s+s1}{\PYGZsq{}}\PYG{l+s+s1}{Local time frame      }\PYG{l+s+si}{\PYGZob{}}\PYG{n}{mpak}\PYG{o}{.}\PYG{n}{current\PYGZus{}per}\PYG{l+s+si}{\PYGZcb{}}\PYG{l+s+s1}{\PYGZsq{}}\PYG{p}{)}
\PYG{n+nb}{print}\PYG{p}{(}\PYG{l+s+sa}{f}\PYG{l+s+s1}{\PYGZsq{}}\PYG{l+s+s1}{Unchanged global time }\PYG{l+s+si}{\PYGZob{}}\PYG{n}{mpak}\PYG{o}{.}\PYG{n}{current\PYGZus{}per}\PYG{l+s+si}{\PYGZcb{}}\PYG{l+s+s1}{\PYGZsq{}}\PYG{p}{)}
\end{sphinxVerbatim}

\end{sphinxuseclass}\end{sphinxVerbatimInput}
\begin{sphinxVerbatimOutput}

\begin{sphinxuseclass}{cell_output}
\begin{sphinxVerbatim}[commandchars=\\\{\}]
Global time  before   Index([2020, 2021, 2022, 2023, 2024, 2025], dtype=\PYGZsq{}int64\PYGZsq{})
Local time frame      Index([2022, 2023], dtype=\PYGZsq{}int64\PYGZsq{})
Unchanged global time Index([2020, 2021, 2022, 2023, 2024, 2025], dtype=\PYGZsq{}int64\PYGZsq{})
\end{sphinxVerbatim}

\end{sphinxuseclass}\end{sphinxVerbatimOutput}

\end{sphinxuseclass}

\subsection{.set\_smpl\_relative Set relative timeframe for a local scope}
\label{\detokenize{content/notebooks/modelflow_features:set-smpl-relative-set-relative-timeframe-for-a-local-scope}}
\sphinxAtStartPar
When creating a script it can be useful to set the time frame relative to the
current time.

\sphinxAtStartPar
Like this:

\begin{sphinxuseclass}{cell}\begin{sphinxVerbatimInput}

\begin{sphinxuseclass}{cell_input}
\begin{sphinxVerbatim}[commandchars=\\\{\}]
\PYG{n+nb}{print}\PYG{p}{(}\PYG{l+s+sa}{f}\PYG{l+s+s1}{\PYGZsq{}}\PYG{l+s+s1}{Global time  before   }\PYG{l+s+si}{\PYGZob{}}\PYG{n}{mpak}\PYG{o}{.}\PYG{n}{current\PYGZus{}per}\PYG{l+s+si}{\PYGZcb{}}\PYG{l+s+s1}{\PYGZsq{}}\PYG{p}{)}
\PYG{k}{with} \PYG{n}{mpak}\PYG{o}{.}\PYG{n}{set\PYGZus{}smpl\PYGZus{}relative} \PYG{p}{(}\PYG{o}{\PYGZhy{}}\PYG{l+m+mi}{1}\PYG{p}{,}\PYG{l+m+mi}{0}\PYG{p}{)}\PYG{p}{:}
    \PYG{n+nb}{print}\PYG{p}{(}\PYG{l+s+sa}{f}\PYG{l+s+s1}{\PYGZsq{}}\PYG{l+s+s1}{Local time frame      }\PYG{l+s+si}{\PYGZob{}}\PYG{n}{mpak}\PYG{o}{.}\PYG{n}{current\PYGZus{}per}\PYG{l+s+si}{\PYGZcb{}}\PYG{l+s+s1}{\PYGZsq{}}\PYG{p}{)}
\PYG{n+nb}{print}\PYG{p}{(}\PYG{l+s+sa}{f}\PYG{l+s+s1}{\PYGZsq{}}\PYG{l+s+s1}{Unchanged global time }\PYG{l+s+si}{\PYGZob{}}\PYG{n}{mpak}\PYG{o}{.}\PYG{n}{current\PYGZus{}per}\PYG{l+s+si}{\PYGZcb{}}\PYG{l+s+s1}{\PYGZsq{}}\PYG{p}{)}
\end{sphinxVerbatim}

\end{sphinxuseclass}\end{sphinxVerbatimInput}
\begin{sphinxVerbatimOutput}

\begin{sphinxuseclass}{cell_output}
\begin{sphinxVerbatim}[commandchars=\\\{\}]
Global time  before   Index([2020, 2021, 2022, 2023, 2024, 2025], dtype=\PYGZsq{}int64\PYGZsq{})
Local time frame      Index([2019, 2020, 2021, 2022, 2023, 2024, 2025], dtype=\PYGZsq{}int64\PYGZsq{})
Unchanged global time Index([2020, 2021, 2022, 2023, 2024, 2025], dtype=\PYGZsq{}int64\PYGZsq{})
\end{sphinxVerbatim}

\end{sphinxuseclass}\end{sphinxVerbatimOutput}

\end{sphinxuseclass}

\section{Using the index operator {[} {]} to select and visualize variables.}
\label{\detokenize{content/notebooks/modelflow_features:using-the-index-operator-to-select-and-visualize-variables}}\label{\detokenize{content/notebooks/modelflow_features:index-operator}}
\sphinxAtStartPar
The index operator {[} {]} can be used to select variables and then process the values for quick analysis.

\sphinxAtStartPar
To select variables the method accept patterns which defines variable names. Wildcards:
\begin{itemize}
\item {} 
\sphinxAtStartPar
\sphinxcode{\sphinxupquote{\textbackslash{}*}} matches everything

\item {} 
\sphinxAtStartPar
\sphinxcode{\sphinxupquote{?}} matches any single character

\item {} 
\sphinxAtStartPar
\sphinxcode{\sphinxupquote{\textbackslash{}{[}seq{]}}} matches any character in seq

\item {} 
\sphinxAtStartPar
\sphinxcode{\sphinxupquote{\textbackslash{}{[}!seq{]}}} matches any character not in seq

\end{itemize}

\sphinxAtStartPar
For more how wildcards can be used, the specification can be found here (\sphinxurl{https://docs.python.org/3/library/fnmatch.html})

\sphinxAtStartPar
In the following example we are selecting the results of mpak{[}‘PAKNYGDPMKTPKN’{]}

\sphinxAtStartPar
This call will return a special class (called \sphinxcode{\sphinxupquote{vis}}). It implements a number
of methods and properties which comes in handy for quick analyses.

\sphinxAtStartPar
Several properties and methods can be chained. An example:

\begin{sphinxuseclass}{cell}\begin{sphinxVerbatimInput}

\begin{sphinxuseclass}{cell_input}
\begin{sphinxVerbatim}[commandchars=\\\{\}]
\PYG{k}{with} \PYG{n}{mpak}\PYG{o}{.}\PYG{n}{set\PYGZus{}smpl}\PYG{p}{(}\PYG{l+m+mi}{2020}\PYG{p}{,}\PYG{l+m+mi}{2100}\PYG{p}{)}\PYG{p}{:}
    \PYG{n}{mpak}\PYG{p}{[}\PYG{l+s+s1}{\PYGZsq{}}\PYG{l+s+s1}{PAKNYGDPMKTPKN}\PYG{l+s+s1}{\PYGZsq{}}\PYG{p}{]}\PYG{o}{.}\PYG{n}{difpctlevel}\PYG{o}{.}\PYG{n}{mul100}\PYG{o}{.}\PYG{n}{rename}\PYG{p}{(}\PYG{p}{)}\PYG{o}{.}\PYG{n}{plot}\PYG{p}{(}\PYG{n}{colrow}\PYG{o}{=}\PYG{l+m+mi}{1}\PYG{p}{,}
                \PYG{n}{title}\PYG{o}{=}\PYG{l+s+s1}{\PYGZsq{}}\PYG{l+s+s1}{Difference to baseline in percent}\PYG{l+s+s1}{\PYGZsq{}}\PYG{p}{,}\PYG{n}{top}\PYG{o}{=}\PYG{l+m+mf}{0.8}\PYG{p}{)}\PYG{p}{;}
\end{sphinxVerbatim}

\end{sphinxuseclass}\end{sphinxVerbatimInput}

\end{sphinxuseclass}
\sphinxAtStartPar
But first some basic information


\subsection{model{[}‘\#ENDO’{]}}
\label{\detokenize{content/notebooks/modelflow_features:model-endo}}
\sphinxAtStartPar
Use ‘\#ENDO’ to access all endogenous variables in your model instance.

\sphinxAtStartPar
For the sake of space, the result is saved in the variable ‘allendo’ and not printed.

\begin{sphinxuseclass}{cell}\begin{sphinxVerbatimInput}

\begin{sphinxuseclass}{cell_input}
\begin{sphinxVerbatim}[commandchars=\\\{\}]
\PYG{n}{allendo} \PYG{o}{=} \PYG{n}{mpak}\PYG{p}{[}\PYG{l+s+s1}{\PYGZsq{}}\PYG{l+s+s1}{\PYGZsh{}ENDO}\PYG{l+s+s1}{\PYGZsq{}}\PYG{p}{]}
\PYG{c+c1}{\PYGZsh{} allendo.show}
\end{sphinxVerbatim}

\end{sphinxuseclass}\end{sphinxVerbatimInput}

\end{sphinxuseclass}

\subsection{Access values in .lastdf and .basedf}
\label{\detokenize{content/notebooks/modelflow_features:access-values-in-lastdf-and-basedf}}
\sphinxAtStartPar
To limit the output printed, we set the time frame to 2020 to 2023.

\begin{sphinxuseclass}{cell}\begin{sphinxVerbatimInput}

\begin{sphinxuseclass}{cell_input}
\begin{sphinxVerbatim}[commandchars=\\\{\}]
\PYG{n}{mpak}\PYG{o}{.}\PYG{n}{smpl}\PYG{p}{(}\PYG{l+m+mi}{2020}\PYG{p}{,}\PYG{l+m+mi}{2023}\PYG{p}{)}\PYG{p}{;}
\end{sphinxVerbatim}

\end{sphinxuseclass}\end{sphinxVerbatimInput}

\end{sphinxuseclass}
\sphinxAtStartPar
To access the values of ‘PAKNYGDPMKTPKN’ and ‘PAKNECONPRVTKN’ from the latest simulation a small widget is displayed.

\begin{sphinxuseclass}{cell}\begin{sphinxVerbatimInput}

\begin{sphinxuseclass}{cell_input}
\begin{sphinxVerbatim}[commandchars=\\\{\}]
\PYG{n}{mpak}\PYG{p}{[}\PYG{l+s+s1}{\PYGZsq{}}\PYG{l+s+s1}{PAKNYGDPMKTPKN PAKNECONPRVTKN}\PYG{l+s+s1}{\PYGZsq{}}\PYG{p}{]} 
\end{sphinxVerbatim}

\end{sphinxuseclass}\end{sphinxVerbatimInput}
\begin{sphinxVerbatimOutput}

\begin{sphinxuseclass}{cell_output}
\begin{sphinxVerbatim}[commandchars=\\\{\}]
Tab(children=(Tab(children=(HTML(value=\PYGZsq{}\PYGZlt{}?xml version=\PYGZdq{}1.0\PYGZdq{} encoding=\PYGZdq{}utf\PYGZhy{}8\PYGZdq{} standalone=\PYGZdq{}no\PYGZdq{}?\PYGZgt{}\PYGZbs{}n\PYGZlt{}!DOCTYPE svg …
\end{sphinxVerbatim}

\begin{sphinxVerbatim}[commandchars=\\\{\}]

\end{sphinxVerbatim}

\end{sphinxuseclass}\end{sphinxVerbatimOutput}

\end{sphinxuseclass}
\sphinxAtStartPar
To access the values of ‘PAKNYGDPMKTPKN’ and ‘PAKNECONPRVTKN’ from the base dataframe, specify .base

\begin{sphinxuseclass}{cell}\begin{sphinxVerbatimInput}

\begin{sphinxuseclass}{cell_input}
\begin{sphinxVerbatim}[commandchars=\\\{\}]
\PYG{n}{mpak}\PYG{p}{[}\PYG{l+s+s1}{\PYGZsq{}}\PYG{l+s+s1}{PAKNYGDPMKTPKN PAKNECONPRVTKN}\PYG{l+s+s1}{\PYGZsq{}}\PYG{p}{]}\PYG{o}{.}\PYG{n}{base}\PYG{o}{.}\PYG{n}{df} 
\end{sphinxVerbatim}

\end{sphinxuseclass}\end{sphinxVerbatimInput}
\begin{sphinxVerbatimOutput}

\begin{sphinxuseclass}{cell_output}
\begin{sphinxVerbatim}[commandchars=\\\{\}]
      PAKNYGDPMKTPKN  PAKNECONPRVTKN
2020    2.627394e+07    2.367289e+07
2021    2.651137e+07    2.397282e+07
2022    2.668514e+07    2.416413e+07
2023    2.696308e+07    2.442786e+07
\end{sphinxVerbatim}

\end{sphinxuseclass}\end{sphinxVerbatimOutput}

\end{sphinxuseclass}

\subsection{.df  Pandas dataframe}
\label{\detokenize{content/notebooks/modelflow_features:df-pandas-dataframe}}
\sphinxAtStartPar
Sometime you need to perform additional operations on the values. Therefor the .df will return a dataframe with the selected variables.

\begin{sphinxuseclass}{cell}\begin{sphinxVerbatimInput}

\begin{sphinxuseclass}{cell_input}
\begin{sphinxVerbatim}[commandchars=\\\{\}]
\PYG{n}{mpak}\PYG{p}{[}\PYG{l+s+s1}{\PYGZsq{}}\PYG{l+s+s1}{PAKNYGDPMKTPKN PAKNECONPRVTKN}\PYG{l+s+s1}{\PYGZsq{}}\PYG{p}{]}\PYG{o}{.}\PYG{n}{df}
\end{sphinxVerbatim}

\end{sphinxuseclass}\end{sphinxVerbatimInput}
\begin{sphinxVerbatimOutput}

\begin{sphinxuseclass}{cell_output}
\begin{sphinxVerbatim}[commandchars=\\\{\}]
      PAKNYGDPMKTPKN  PAKNECONPRVTKN
2020    2.647002e+07    2.344055e+07
2021    2.676493e+07    2.366076e+07
2022    2.688965e+07    2.376966e+07
2023    2.708904e+07    2.395330e+07
\end{sphinxVerbatim}

\end{sphinxuseclass}\end{sphinxVerbatimOutput}

\end{sphinxuseclass}

\subsection{.show  as a html table with tooltips}
\label{\detokenize{content/notebooks/modelflow_features:show-as-a-html-table-with-tooltips}}
\sphinxAtStartPar
If you want the variable descriptions use this

\begin{sphinxuseclass}{cell}\begin{sphinxVerbatimInput}

\begin{sphinxuseclass}{cell_input}
\begin{sphinxVerbatim}[commandchars=\\\{\}]
\PYG{n}{mpak}\PYG{p}{[}\PYG{l+s+s1}{\PYGZsq{}}\PYG{l+s+s1}{PAKNYGDPMKTPKN PAKNECONPRVTKN}\PYG{l+s+s1}{\PYGZsq{}}\PYG{p}{]}\PYG{o}{.}\PYG{n}{show}
\end{sphinxVerbatim}

\end{sphinxuseclass}\end{sphinxVerbatimInput}
\begin{sphinxVerbatimOutput}

\begin{sphinxuseclass}{cell_output}
\begin{sphinxVerbatim}[commandchars=\\\{\}]
Tab(children=(Tab(children=(HTML(value=\PYGZsq{}\PYGZlt{}?xml version=\PYGZdq{}1.0\PYGZdq{} encoding=\PYGZdq{}utf\PYGZhy{}8\PYGZdq{} standalone=\PYGZdq{}no\PYGZdq{}?\PYGZgt{}\PYGZbs{}n\PYGZlt{}!DOCTYPE svg …
\end{sphinxVerbatim}

\end{sphinxuseclass}\end{sphinxVerbatimOutput}

\end{sphinxuseclass}

\subsection{.names Variable names}
\label{\detokenize{content/notebooks/modelflow_features:names-variable-names}}
\sphinxAtStartPar
If you select variables using wildcards, then you can access the names that correspond to your query.

\begin{sphinxuseclass}{cell}\begin{sphinxVerbatimInput}

\begin{sphinxuseclass}{cell_input}
\begin{sphinxVerbatim}[commandchars=\\\{\}]
\PYG{n}{mpak}\PYG{p}{[}\PYG{l+s+s1}{\PYGZsq{}}\PYG{l+s+s1}{PAKNYGDP??????}\PYG{l+s+s1}{\PYGZsq{}}\PYG{p}{]}\PYG{o}{.}\PYG{n}{names}
\end{sphinxVerbatim}

\end{sphinxuseclass}\end{sphinxVerbatimInput}
\begin{sphinxVerbatimOutput}

\begin{sphinxuseclass}{cell_output}
\begin{sphinxVerbatim}[commandchars=\\\{\}]
[\PYGZsq{}PAKNYGDPDISCCN\PYGZsq{},
 \PYGZsq{}PAKNYGDPDISCKN\PYGZsq{},
 \PYGZsq{}PAKNYGDPFCSTCN\PYGZsq{},
 \PYGZsq{}PAKNYGDPFCSTKN\PYGZsq{},
 \PYGZsq{}PAKNYGDPFCSTXN\PYGZsq{},
 \PYGZsq{}PAKNYGDPMKTPCD\PYGZsq{},
 \PYGZsq{}PAKNYGDPMKTPCN\PYGZsq{},
 \PYGZsq{}PAKNYGDPMKTPKD\PYGZsq{},
 \PYGZsq{}PAKNYGDPMKTPKN\PYGZsq{},
 \PYGZsq{}PAKNYGDPMKTPXN\PYGZsq{},
 \PYGZsq{}PAKNYGDPPOTLKN\PYGZsq{}]
\end{sphinxVerbatim}

\end{sphinxuseclass}\end{sphinxVerbatimOutput}

\end{sphinxuseclass}

\subsection{.frml The formulas}
\label{\detokenize{content/notebooks/modelflow_features:frml-the-formulas}}
\sphinxAtStartPar
Use .frml to access all the equations for the endogenous variables.

\begin{sphinxuseclass}{cell}\begin{sphinxVerbatimInput}

\begin{sphinxuseclass}{cell_input}
\begin{sphinxVerbatim}[commandchars=\\\{\}]
\PYG{n}{mpak}\PYG{p}{[}\PYG{l+s+s1}{\PYGZsq{}}\PYG{l+s+s1}{PAKNYGDPMKTPKN PAKNECONPRVTKN}\PYG{l+s+s1}{\PYGZsq{}}\PYG{p}{]}\PYG{o}{.}\PYG{n}{frml}
\end{sphinxVerbatim}

\end{sphinxuseclass}\end{sphinxVerbatimInput}
\begin{sphinxVerbatimOutput}

\begin{sphinxuseclass}{cell_output}
\begin{sphinxVerbatim}[commandchars=\\\{\}]
PAKNYGDPMKTPKN : FRML \PYGZlt{}IDENT\PYGZgt{} PAKNYGDPMKTPKN = PAKNECONPRVTKN+PAKNECONGOVTKN+PAKNEGDIFTOTKN+PAKNEGDISTKBKN+PAKNEEXPGNFSKN\PYGZhy{}PAKNEIMPGNFSKN+PAKNYGDPDISCKN+PAKADAP*PAKDISPREPKN \PYGZdl{}
PAKNECONPRVTKN : FRML \PYGZlt{}DAMP,STOC\PYGZgt{} PAKNECONPRVTKN = (PAKNECONPRVTKN(\PYGZhy{}1)*EXP(PAKNECONPRVTKN\PYGZus{}A+ (\PYGZhy{}0.2*(LOG(PAKNECONPRVTKN(\PYGZhy{}1))\PYGZhy{}LOG(1.21203101101442)\PYGZhy{}LOG((((PAKBXFSTREMTCD(\PYGZhy{}1)\PYGZhy{}PAKBMFSTREMTCD(\PYGZhy{}1))*PAKPANUSATLS(\PYGZhy{}1))+PAKGGEXPTRNSCN(\PYGZhy{}1)+PAKNYYWBTOTLCN(\PYGZhy{}1)*(1\PYGZhy{}PAKGGREVDRCTXN(\PYGZhy{}1)/100))/PAKNECONPRVTXN(\PYGZhy{}1)))+0.763938860758873*((LOG((((PAKBXFSTREMTCD\PYGZhy{}PAKBMFSTREMTCD)*PAKPANUSATLS)+PAKGGEXPTRNSCN+PAKNYYWBTOTLCN*(1\PYGZhy{}PAKGGREVDRCTXN/100))/PAKNECONPRVTXN))\PYGZhy{}(LOG((((PAKBXFSTREMTCD(\PYGZhy{}1)\PYGZhy{}PAKBMFSTREMTCD(\PYGZhy{}1))*PAKPANUSATLS(\PYGZhy{}1))+PAKGGEXPTRNSCN(\PYGZhy{}1)+PAKNYYWBTOTLCN(\PYGZhy{}1)*(1\PYGZhy{}PAKGGREVDRCTXN(\PYGZhy{}1)/100))/PAKNECONPRVTXN(\PYGZhy{}1))))\PYGZhy{}0.0634474791568939*DURING\PYGZus{}2009\PYGZhy{}0.3*(PAKFMLBLPOLYXN/100\PYGZhy{}((LOG(PAKNECONPRVTXN))\PYGZhy{}(LOG(PAKNECONPRVTXN(\PYGZhy{}1)))))) )) * (1\PYGZhy{}PAKNECONPRVTKN\PYGZus{}D)+ PAKNECONPRVTKN\PYGZus{}X*PAKNECONPRVTKN\PYGZus{}D \PYGZdl{}
\end{sphinxVerbatim}

\end{sphinxuseclass}\end{sphinxVerbatimOutput}

\end{sphinxuseclass}

\subsection{.rename() Rename variables to descriptions}
\label{\detokenize{content/notebooks/modelflow_features:rename-rename-variables-to-descriptions}}
\sphinxAtStartPar
Use .rename() to assign variable descriptions as variable names.

\sphinxAtStartPar
Handy when plotting!

\begin{sphinxuseclass}{cell}\begin{sphinxVerbatimInput}

\begin{sphinxuseclass}{cell_input}
\begin{sphinxVerbatim}[commandchars=\\\{\}]
\PYG{n}{mpak}\PYG{p}{[}\PYG{l+s+s1}{\PYGZsq{}}\PYG{l+s+s1}{PAKNYGDPMKTPKN PAKNECONPRVTKN}\PYG{l+s+s1}{\PYGZsq{}}\PYG{p}{]}\PYG{o}{.}\PYG{n}{rename}\PYG{p}{(}\PYG{p}{)}\PYG{o}{.}\PYG{n}{df}
\end{sphinxVerbatim}

\end{sphinxuseclass}\end{sphinxVerbatimInput}
\begin{sphinxVerbatimOutput}

\begin{sphinxuseclass}{cell_output}
\begin{sphinxVerbatim}[commandchars=\\\{\}]
          Real GDP  HH. Cons Real
2020  2.647002e+07   2.344055e+07
2021  2.676493e+07   2.366076e+07
2022  2.688965e+07   2.376966e+07
2023  2.708904e+07   2.395330e+07
\end{sphinxVerbatim}

\end{sphinxuseclass}\end{sphinxVerbatimOutput}

\end{sphinxuseclass}

\subsection{Transformations of solution results}
\label{\detokenize{content/notebooks/modelflow_features:transformations-of-solution-results}}
\sphinxAtStartPar
When the variables has been selected through the index operator a number of standard data transformations can be performed.


\begin{savenotes}\sphinxattablestart
\centering
\begin{tabulary}{\linewidth}[t]{|T|T|T|}
\hline
\sphinxstyletheadfamily 
\sphinxAtStartPar
Transfomation
&\sphinxstyletheadfamily 
\sphinxAtStartPar
Meaning
&\sphinxstyletheadfamily 
\sphinxAtStartPar
expression
\\
\hline
\sphinxAtStartPar
pct
&
\sphinxAtStartPar
Growth rates
&
\sphinxAtStartPar
\(\left(\cfrac{this_t}{this_{t-1}}-1\right )\)
\\
\hline
\sphinxAtStartPar
dif
&
\sphinxAtStartPar
Difference in level
&
\sphinxAtStartPar
\(l-b\)
\\
\hline
\sphinxAtStartPar
difpct
&
\sphinxAtStartPar
Differens in growth rate
&
\sphinxAtStartPar
\(\left( \cfrac{l_t}{l_{t-1}}-1 \right) - \left(\cfrac{b_t}{b_{t-1}}-1 \right)\)
\\
\hline
\sphinxAtStartPar
difpctlevel
&
\sphinxAtStartPar
differens in level in pct of baseline
&
\sphinxAtStartPar
\(\left( \cfrac{l_t-b_t}{b_{t}} \right) \)
\\
\hline
\sphinxAtStartPar
mul100
&
\sphinxAtStartPar
multiply by 100
&
\sphinxAtStartPar
\(this_t \times 100\)
\\
\hline
\end{tabulary}
\par
\sphinxattableend\end{savenotes}
\begin{itemize}
\item {} 
\sphinxAtStartPar
\(this\) is the chained value. Default lastdf but if preseeded by .base the values from .basedf will be used

\item {} 
\sphinxAtStartPar
\(b\) is the values from .basedf

\item {} 
\sphinxAtStartPar
\(l\) is the values from .lastdf

\end{itemize}


\subsection{.dif Difference in level}
\label{\detokenize{content/notebooks/modelflow_features:dif-difference-in-level}}
\sphinxAtStartPar
The ‘dif’ command displays the difference in levels of the latest and previous solutions.

\sphinxAtStartPar
\(l-b\)

\sphinxAtStartPar
where l is the variable from the .lastdf and b is the variable from .basedf.

\begin{sphinxuseclass}{cell}\begin{sphinxVerbatimInput}

\begin{sphinxuseclass}{cell_input}
\begin{sphinxVerbatim}[commandchars=\\\{\}]
\PYG{n}{mpak}\PYG{p}{[}\PYG{l+s+s1}{\PYGZsq{}}\PYG{l+s+s1}{PAKNYGDPMKTPKN PAKNECONPRVTKN}\PYG{l+s+s1}{\PYGZsq{}}\PYG{p}{]}\PYG{o}{.}\PYG{n}{dif}\PYG{o}{.}\PYG{n}{plot}\PYG{p}{(}\PYG{p}{)}
\end{sphinxVerbatim}

\end{sphinxuseclass}\end{sphinxVerbatimInput}
\begin{sphinxVerbatimOutput}

\begin{sphinxuseclass}{cell_output}
\noindent\sphinxincludegraphics{{2e15cd4f122be77877a2d23a38f9b477fc6a33d7c897686afad1b442eb6a18cc}.png}

\end{sphinxuseclass}\end{sphinxVerbatimOutput}

\end{sphinxuseclass}

\subsection{.pct  Growthrates}
\label{\detokenize{content/notebooks/modelflow_features:pct-growthrates}}
\sphinxAtStartPar
Display growth rates

\sphinxAtStartPar
\(\left(\cfrac{l_t}{l_{t-1}}-1\right )\)

\begin{sphinxuseclass}{cell}\begin{sphinxVerbatimInput}

\begin{sphinxuseclass}{cell_input}
\begin{sphinxVerbatim}[commandchars=\\\{\}]
\PYG{n}{mpak}\PYG{p}{[}\PYG{l+s+s1}{\PYGZsq{}}\PYG{l+s+s1}{PAKNYGDPMKTPKN PAKNECONPRVTKN}\PYG{l+s+s1}{\PYGZsq{}}\PYG{p}{]}\PYG{o}{.}\PYG{n}{pct}\PYG{o}{.}\PYG{n}{plot}\PYG{p}{(}\PYG{p}{)}\PYG{p}{;}
\end{sphinxVerbatim}

\end{sphinxuseclass}\end{sphinxVerbatimInput}

\end{sphinxuseclass}

\subsection{.difpct property difference in growthrate}
\label{\detokenize{content/notebooks/modelflow_features:difpct-property-difference-in-growthrate}}
\sphinxAtStartPar
The difference in the growth rates  between the last and base dataframe.

\sphinxAtStartPar
\(\left( \cfrac{l_t}{l_{t-1}}-1 \right) - \left(\cfrac{b_t}{b_{t-1}}-1 \right)\)

\begin{sphinxuseclass}{cell}\begin{sphinxVerbatimInput}

\begin{sphinxuseclass}{cell_input}
\begin{sphinxVerbatim}[commandchars=\\\{\}]
\PYG{n}{mpak}\PYG{p}{[}\PYG{l+s+s1}{\PYGZsq{}}\PYG{l+s+s1}{PAKNYGDPMKTPKN PAKNECONPRVTKN}\PYG{l+s+s1}{\PYGZsq{}}\PYG{p}{]}\PYG{o}{.}\PYG{n}{difpct}\PYG{o}{.}\PYG{n}{plot}\PYG{p}{(}\PYG{p}{)} \PYG{p}{;} 
\end{sphinxVerbatim}

\end{sphinxuseclass}\end{sphinxVerbatimInput}

\end{sphinxuseclass}

\subsection{.difpctlevel percent difference of  levels}
\label{\detokenize{content/notebooks/modelflow_features:difpctlevel-percent-difference-of-levels}}
\sphinxAtStartPar
\(\left( \cfrac{l_t-b_t}{b_{t}} \right) \)

\begin{sphinxuseclass}{cell}\begin{sphinxVerbatimInput}

\begin{sphinxuseclass}{cell_input}
\begin{sphinxVerbatim}[commandchars=\\\{\}]
\PYG{n}{mpak}\PYG{p}{[}\PYG{l+s+s1}{\PYGZsq{}}\PYG{l+s+s1}{PAKNYGDPMKTPKN PAKNECONPRVTKN}\PYG{l+s+s1}{\PYGZsq{}}\PYG{p}{]}\PYG{o}{.}\PYG{n}{difpctlevel}\PYG{o}{.}\PYG{n}{plot}\PYG{p}{(}\PYG{p}{)}\PYG{p}{;}  
\end{sphinxVerbatim}

\end{sphinxuseclass}\end{sphinxVerbatimInput}

\end{sphinxuseclass}

\subsection{mul100 multiply by 100}
\label{\detokenize{content/notebooks/modelflow_features:mul100-multiply-by-100}}
\sphinxAtStartPar
multiply growth rate by 100.

\begin{sphinxuseclass}{cell}\begin{sphinxVerbatimInput}

\begin{sphinxuseclass}{cell_input}
\begin{sphinxVerbatim}[commandchars=\\\{\}]
\PYG{n}{mpak}\PYG{p}{[}\PYG{l+s+s1}{\PYGZsq{}}\PYG{l+s+s1}{PAKNYGDPMKTPKN PAKNECONPRVTKN}\PYG{l+s+s1}{\PYGZsq{}}\PYG{p}{]}\PYG{o}{.}\PYG{n}{pct}\PYG{o}{.}\PYG{n}{mul100}\PYG{o}{.}\PYG{n}{plot}\PYG{p}{(}\PYG{p}{)} 
\end{sphinxVerbatim}

\end{sphinxuseclass}\end{sphinxVerbatimInput}
\begin{sphinxVerbatimOutput}

\begin{sphinxuseclass}{cell_output}
\noindent\sphinxincludegraphics{{1c86c834aebde599612b932557233fdb805ad5fc1e020793dcee56ccbcc9dfcd}.png}

\end{sphinxuseclass}\end{sphinxVerbatimOutput}

\end{sphinxuseclass}

\section{.plot chart the selected and transformed variables}
\label{\detokenize{content/notebooks/modelflow_features:plot-chart-the-selected-and-transformed-variables}}
\sphinxAtStartPar
After the varaibles has been selected and transformed, they can  be plotted. The .plot() method plots the selected variables separately

\begin{sphinxuseclass}{cell}\begin{sphinxVerbatimInput}

\begin{sphinxuseclass}{cell_input}
\begin{sphinxVerbatim}[commandchars=\\\{\}]
\PYG{n}{mpak}\PYG{o}{.}\PYG{n}{smpl}\PYG{p}{(}\PYG{l+m+mi}{2020}\PYG{p}{,}\PYG{l+m+mi}{2100}\PYG{p}{)}\PYG{p}{;}

\PYG{n}{mpak}\PYG{p}{[}\PYG{l+s+s1}{\PYGZsq{}}\PYG{l+s+s1}{PAKNYGDP??????}\PYG{l+s+s1}{\PYGZsq{}}\PYG{p}{]}\PYG{o}{.}\PYG{n}{rename}\PYG{p}{(}\PYG{p}{)}\PYG{o}{.}\PYG{n}{plot}\PYG{p}{(}\PYG{p}{)}\PYG{p}{;}
\end{sphinxVerbatim}

\end{sphinxuseclass}\end{sphinxVerbatimInput}

\end{sphinxuseclass}

\subsection{Options to plot()}
\label{\detokenize{content/notebooks/modelflow_features:options-to-plot}}
\sphinxAtStartPar
Common:
\begin{itemize}
\item {} 
\sphinxAtStartPar
title (optional): title. Defaults to ‘’.

\item {} 
\sphinxAtStartPar
colrow (TYPE, optional): Columns per row . Defaults to 2.

\item {} 
\sphinxAtStartPar
sharey (TYPE, optional): Share y axis between plots. Defaults to False.

\item {} 
\sphinxAtStartPar
top (TYPE, optional): Relative position of the title. Defaults to 0.90.

\end{itemize}

\sphinxAtStartPar
More excotic:
\begin{itemize}
\item {} 
\sphinxAtStartPar
splitchar (TYPE, optional): If the name should be split . Defaults to ‘\_\_’.

\item {} 
\sphinxAtStartPar
savefig (TYPE, optional): Save figure. Defaults to ‘’.

\item {} 
\sphinxAtStartPar
xsize  (TYPE, optional): x size default to 10

\item {} 
\sphinxAtStartPar
ysize  (TYPE, optional): y size per row, defaults to 2

\item {} 
\sphinxAtStartPar
ppos (optional): \# of position to use if split. Defaults to \sphinxhyphen{}1.

\item {} 
\sphinxAtStartPar
kind (TYPE, optional): Matplotlib kind . Defaults to ‘line’.

\end{itemize}

\begin{sphinxuseclass}{cell}\begin{sphinxVerbatimInput}

\begin{sphinxuseclass}{cell_input}
\begin{sphinxVerbatim}[commandchars=\\\{\}]
\PYG{n}{mpak}\PYG{p}{[}\PYG{l+s+s1}{\PYGZsq{}}\PYG{l+s+s1}{PAKNYGDP??????}\PYG{l+s+s1}{\PYGZsq{}}\PYG{p}{]}\PYG{o}{.}\PYG{n}{difpct}\PYG{o}{.}\PYG{n}{mul100}\PYG{o}{.}\PYG{n}{rename}\PYG{p}{(}\PYG{p}{)}\PYG{o}{.}\PYG{n}{plot}\PYG{p}{(}\PYG{n}{title}\PYG{o}{=}\PYG{l+s+s1}{\PYGZsq{}}\PYG{l+s+s1}{GDP growth }\PYG{l+s+s1}{\PYGZsq{}}\PYG{p}{,}\PYG{n}{top} \PYG{o}{=} \PYG{l+m+mf}{0.92}\PYG{p}{)}\PYG{p}{;}
\end{sphinxVerbatim}

\end{sphinxuseclass}\end{sphinxVerbatimInput}

\end{sphinxuseclass}

\section{Plotting inspiration}
\label{\detokenize{content/notebooks/modelflow_features:plotting-inspiration}}
\sphinxAtStartPar
The following graph shows the components of GDP using the values of the baseline dataframe.

\begin{sphinxuseclass}{cell}\begin{sphinxVerbatimInput}

\begin{sphinxuseclass}{cell_input}
\begin{sphinxVerbatim}[commandchars=\\\{\}]
\PYG{n}{mpak}\PYG{p}{[}\PYG{l+s+s1}{\PYGZsq{}}\PYG{l+s+s1}{PAKNYGDPMKTPKN PAKNECONPRVTKN PAKNEGDIFTOTKN}\PYG{l+s+s1}{\PYGZsq{}}\PYG{p}{]}\PYG{o}{.}\PYGZbs{}
\PYG{n}{difpctlevel}\PYG{o}{.}\PYG{n}{mul100}\PYG{o}{.}\PYG{n}{rename}\PYG{p}{(}\PYG{p}{)}\PYG{o}{.}\PYGZbs{}
\PYG{n}{plot}\PYG{p}{(}\PYG{n}{title}\PYG{o}{=}\PYG{l+s+s1}{\PYGZsq{}}\PYG{l+s+s1}{Components of GDP in pct of baseline}\PYG{l+s+s1}{\PYGZsq{}}\PYG{p}{,}\PYG{n}{colrow}\PYG{o}{=}\PYG{l+m+mi}{1}\PYG{p}{,}\PYG{n}{top}\PYG{o}{=}\PYG{l+m+mf}{0.90}\PYG{p}{,}\PYG{n}{kind}\PYG{o}{=}\PYG{l+s+s1}{\PYGZsq{}}\PYG{l+s+s1}{bar}\PYG{l+s+s1}{\PYGZsq{}}\PYG{p}{)} \PYG{p}{;}
\end{sphinxVerbatim}

\end{sphinxuseclass}\end{sphinxVerbatimInput}

\end{sphinxuseclass}

\subsection{Heatmaps}
\label{\detokenize{content/notebooks/modelflow_features:heatmaps}}
\sphinxAtStartPar
For some model types heatmaps can be helpful, and they come out of the box. This feature was developed for use by bank stress test models.

\begin{sphinxuseclass}{cell}\begin{sphinxVerbatimInput}

\begin{sphinxuseclass}{cell_input}
\begin{sphinxVerbatim}[commandchars=\\\{\}]
\PYG{k}{with} \PYG{n}{mpak}\PYG{o}{.}\PYG{n}{set\PYGZus{}smpl}\PYG{p}{(}\PYG{l+m+mi}{2020}\PYG{p}{,}\PYG{l+m+mi}{2030}\PYG{p}{)}\PYG{p}{:}
    \PYG{n}{heatmap} \PYG{o}{=} \PYG{n}{mpak}\PYG{p}{[}\PYG{l+s+s1}{\PYGZsq{}}\PYG{l+s+s1}{PAKNYGDPMKTPKN PAKNECONPRVTKN}\PYG{l+s+s1}{\PYGZsq{}}\PYG{p}{]}\PYG{o}{.}\PYG{n}{pct}\PYG{o}{.}\PYG{n}{rename}\PYG{p}{(}\PYG{p}{)}\PYG{o}{.}\PYG{n}{mul100}\PYG{o}{.}\PYG{n}{heat}\PYG{p}{(}\PYG{n}{title}\PYG{o}{=}\PYG{l+s+s1}{\PYGZsq{}}\PYG{l+s+s1}{Growth rates}\PYG{l+s+s1}{\PYGZsq{}}\PYG{p}{,}\PYG{n}{annot}\PYG{o}{=}\PYG{k+kc}{True}\PYG{p}{,}\PYG{n}{dec}\PYG{o}{=}\PYG{l+m+mi}{1}\PYG{p}{,}\PYG{n}{size}\PYG{o}{=}\PYG{p}{(}\PYG{l+m+mi}{10}\PYG{p}{,}\PYG{l+m+mi}{3}\PYG{p}{)}\PYG{p}{)}  
\end{sphinxVerbatim}

\end{sphinxuseclass}\end{sphinxVerbatimInput}

\end{sphinxuseclass}
\sphinxAtStartPar



\subsection{Violin and boxplots,}
\label{\detokenize{content/notebooks/modelflow_features:violin-and-boxplots}}
\sphinxAtStartPar
Not obvious for macro models, but useful for stress test  models with many banks.

\begin{sphinxuseclass}{cell}\begin{sphinxVerbatimInput}

\begin{sphinxuseclass}{cell_input}
\begin{sphinxVerbatim}[commandchars=\\\{\}]
\PYG{k}{with} \PYG{n}{mpak}\PYG{o}{.}\PYG{n}{set\PYGZus{}smpl}\PYG{p}{(}\PYG{l+m+mi}{2020}\PYG{p}{,}\PYG{l+m+mi}{2030}\PYG{p}{)}\PYG{p}{:} 
    \PYG{n}{mpak}\PYG{p}{[}\PYG{l+s+s1}{\PYGZsq{}}\PYG{l+s+s1}{PAKNYGDPMKTPKN PAKNECONPRVTKN}\PYG{l+s+s1}{\PYGZsq{}}\PYG{p}{]}\PYG{o}{.}\PYG{n}{difpct}\PYG{o}{.}\PYG{n}{box}\PYG{p}{(}\PYG{p}{)}  
    \PYG{n}{mpak}\PYG{p}{[}\PYG{l+s+s1}{\PYGZsq{}}\PYG{l+s+s1}{PAKNYGDPMKTPKN PAKNECONPRVTKN}\PYG{l+s+s1}{\PYGZsq{}}\PYG{p}{]}\PYG{o}{.}\PYG{n}{difpct}\PYG{o}{.}\PYG{n}{violin}\PYG{p}{(}\PYG{p}{)}  
\end{sphinxVerbatim}

\end{sphinxuseclass}\end{sphinxVerbatimInput}
\begin{sphinxVerbatimOutput}

\begin{sphinxuseclass}{cell_output}
\noindent\sphinxincludegraphics{{a21628d50e960ebc93a04bcb0e594b4caaff83d39e9b1fa0cdad38b4488288ae}.png}

\noindent\sphinxincludegraphics{{11e8415c630bebd221bcb2ff134bb18262a7c36d952bc21b9035faa32ba22e2e}.png}

\end{sphinxuseclass}\end{sphinxVerbatimOutput}

\end{sphinxuseclass}

\subsection{Plot baseline vs alternative}
\label{\detokenize{content/notebooks/modelflow_features:plot-baseline-vs-alternative}}
\sphinxAtStartPar
A raw routine, only showing levels.
To make it really useful it should be expanded.

\begin{sphinxuseclass}{cell}\begin{sphinxVerbatimInput}

\begin{sphinxuseclass}{cell_input}
\begin{sphinxVerbatim}[commandchars=\\\{\}]
\PYG{n}{mpak}\PYG{p}{[}\PYG{l+s+s1}{\PYGZsq{}}\PYG{l+s+s1}{PAKNYGDPMKTPKN PAKNECONPRVTKN}\PYG{l+s+s1}{\PYGZsq{}}\PYG{p}{]}\PYG{o}{.}\PYG{n}{plot\PYGZus{}alt}\PYG{p}{(}\PYG{p}{)} \PYG{p}{;}
\end{sphinxVerbatim}

\end{sphinxuseclass}\end{sphinxVerbatimInput}
\begin{sphinxVerbatimOutput}

\begin{sphinxuseclass}{cell_output}
\noindent\sphinxincludegraphics{{7daeeef1986ee736b023364941580cb93cb9b6a6761bd47c11370ba566749a49}.png}

\end{sphinxuseclass}\end{sphinxVerbatimOutput}

\end{sphinxuseclass}

\section{.draw() Graphical presentation of relationships between variables}
\label{\detokenize{content/notebooks/modelflow_features:draw-graphical-presentation-of-relationships-between-variables}}
\sphinxAtStartPar
.draw() helps you understand the relationship between variables in your model better.

\sphinxAtStartPar
The thickness the arrow reflect the attribution of the the upstream variable to the impact on the downstream variable.


\subsection{.draw(up = level, down = level)}
\label{\detokenize{content/notebooks/modelflow_features:draw-up-level-down-level}}
\sphinxAtStartPar
You can specify how many levels up and down you want in your graphical presentation (Needs more explanation).

\sphinxAtStartPar
In this example all variables that depend directly upon GDP and consumption as well as those that are determined by them, are displayed. This means one step upstream in the model logic and one step downstream.

\sphinxAtStartPar
More on the how to visualize the logic structure \DUrole{xref,myst}{here}

\begin{sphinxuseclass}{cell}\begin{sphinxVerbatimInput}

\begin{sphinxuseclass}{cell_input}
\begin{sphinxVerbatim}[commandchars=\\\{\}]
\PYG{n}{mpak}\PYG{p}{[}\PYG{l+s+s1}{\PYGZsq{}}\PYG{l+s+s1}{PAKNYGDPMKTPKN PAKNECONPRVTKN}\PYG{l+s+s1}{\PYGZsq{}}\PYG{p}{]}\PYG{o}{.}\PYG{n}{draw}\PYG{p}{(}\PYG{n}{up}\PYG{o}{=}\PYG{l+m+mi}{1}\PYG{p}{,}\PYG{n}{down}\PYG{o}{=}\PYG{l+m+mi}{1}\PYG{p}{)}  \PYG{c+c1}{\PYGZsh{} diagram of all direct dependencies }
\end{sphinxVerbatim}

\end{sphinxuseclass}\end{sphinxVerbatimInput}
\begin{sphinxVerbatimOutput}

\begin{sphinxuseclass}{cell_output}
\begin{sphinxVerbatim}[commandchars=\\\{\}]
\PYGZlt{}IPython.core.display.SVG object\PYGZgt{}
\end{sphinxVerbatim}

\begin{sphinxVerbatim}[commandchars=\\\{\}]
\PYGZlt{}IPython.core.display.SVG object\PYGZgt{}
\end{sphinxVerbatim}

\end{sphinxuseclass}\end{sphinxVerbatimOutput}

\end{sphinxuseclass}

\subsection{.draw(filter =<minimal impact>)}
\label{\detokenize{content/notebooks/modelflow_features:draw-filter-minimal-impact}}
\sphinxAtStartPar
By specifying filter=  only links where the minimal impact is more than <minimal impact> are show. In this case 20\%

\begin{sphinxuseclass}{cell}\begin{sphinxVerbatimInput}

\begin{sphinxuseclass}{cell_input}
\begin{sphinxVerbatim}[commandchars=\\\{\}]
\PYG{n}{mpak}\PYG{p}{[}\PYG{l+s+s1}{\PYGZsq{}}\PYG{l+s+s1}{PAKNECONPRVTKN}\PYG{l+s+s1}{\PYGZsq{}}\PYG{p}{]}\PYG{o}{.}\PYG{n}{draw}\PYG{p}{(}\PYG{n}{up}\PYG{o}{=}\PYG{l+m+mi}{3}\PYG{p}{,}\PYG{n}{down}\PYG{o}{=}\PYG{l+m+mi}{1}\PYG{p}{,}\PYG{n+nb}{filter}\PYG{o}{=}\PYG{l+m+mi}{20}\PYG{p}{)}  
\end{sphinxVerbatim}

\end{sphinxuseclass}\end{sphinxVerbatimInput}
\begin{sphinxVerbatimOutput}

\begin{sphinxuseclass}{cell_output}
\begin{sphinxVerbatim}[commandchars=\\\{\}]
\PYGZlt{}IPython.core.display.SVG object\PYGZgt{}
\end{sphinxVerbatim}

\end{sphinxuseclass}\end{sphinxVerbatimOutput}

\end{sphinxuseclass}

\section{dekomp() Attrribution of right hand side variables to change in result.}
\label{\detokenize{content/notebooks/modelflow_features:dekomp-attrribution-of-right-hand-side-variables-to-change-in-result}}
\sphinxAtStartPar
For more information on attribution look \DUrole{xref,myst}{here}

\sphinxAtStartPar
The dekomp command decomposes the contributions of the right hand side variables to the observed change in the left hand side variables.

\begin{sphinxuseclass}{cell}\begin{sphinxVerbatimInput}

\begin{sphinxuseclass}{cell_input}
\begin{sphinxVerbatim}[commandchars=\\\{\}]
\PYG{k}{with} \PYG{n}{mpak}\PYG{o}{.}\PYG{n}{set\PYGZus{}smpl}\PYG{p}{(}\PYG{l+m+mi}{2021}\PYG{p}{,}\PYG{l+m+mi}{2025}\PYG{p}{)}\PYG{p}{:}
    \PYG{n}{mpak}\PYG{p}{[}\PYG{l+s+s1}{\PYGZsq{}}\PYG{l+s+s1}{PAKNYGDPMKTPKN PAKNECONPRVTKN}\PYG{l+s+s1}{\PYGZsq{}}\PYG{p}{]}\PYG{o}{.}\PYG{n}{dekomp}\PYG{p}{(}\PYG{p}{)}  \PYG{c+c1}{\PYGZsh{} frml attribution }
\end{sphinxVerbatim}

\end{sphinxuseclass}\end{sphinxVerbatimInput}
\begin{sphinxVerbatimOutput}

\begin{sphinxuseclass}{cell_output}
\begin{sphinxVerbatim}[commandchars=\\\{\}]
Formula        : FRML \PYGZlt{}IDENT\PYGZgt{} PAKNYGDPMKTPKN = PAKNECONPRVTKN+PAKNECONGOVTKN+PAKNEGDIFTOTKN+PAKNEGDISTKBKN+PAKNEEXPGNFSKN\PYGZhy{}PAKNEIMPGNFSKN+PAKNYGDPDISCKN+PAKADAP*PAKDISPREPKN \PYGZdl{} 

                       2021        2022        2023        2024        2025
Variable    lag                                                            
Base        0   26511370.41 26685141.87 26963077.57 27393200.36 27963231.53
Alternative 0   26764926.87 26889649.52 27089036.50 27454422.35 27979057.19
Difference  0     253556.46   204507.65   125958.93    61221.99    15825.66
Percent     0          0.96        0.77        0.47        0.22        0.06

 Contributions to differende for  PAKNYGDPMKTPKN
                         2021       2022       2023       2024       2025
Variable       lag                                                       
PAKNECONPRVTKN 0   \PYGZhy{}312052.97 \PYGZhy{}394466.14 \PYGZhy{}474558.93 \PYGZhy{}531755.17 \PYGZhy{}563616.01
PAKNECONGOVTKN 0    303335.99  268694.45  232506.87  209988.19  197439.80
PAKNEGDIFTOTKN 0    188565.48  188222.74  177226.36  163571.78  148739.93
PAKNEGDISTKBKN 0        \PYGZhy{}0.02      \PYGZhy{}0.01      \PYGZhy{}0.02      \PYGZhy{}0.02      \PYGZhy{}0.05
PAKNEEXPGNFSKN 0     \PYGZhy{}2911.23   \PYGZhy{}5414.50   \PYGZhy{}7960.34  \PYGZhy{}10272.64  \PYGZhy{}12204.84
PAKNEIMPGNFSKN 0     76619.12  147471.06  198744.89  229689.74  245466.56
PAKNYGDPDISCKN 0        \PYGZhy{}0.02      \PYGZhy{}0.01      \PYGZhy{}0.02      \PYGZhy{}0.02      \PYGZhy{}0.05
PAKADAP        0        \PYGZhy{}0.02      \PYGZhy{}0.01      \PYGZhy{}0.02      \PYGZhy{}0.02      \PYGZhy{}0.05
PAKDISPREPKN   0        \PYGZhy{}0.02      \PYGZhy{}0.01      \PYGZhy{}0.02      \PYGZhy{}0.02      \PYGZhy{}0.05

 Share of contributions to differende for  PAKNYGDPMKTPKN
                          2021        2022        2023        2024        2025
Variable       lag                                                            
PAKNEIMPGNFSKN 0           30\PYGZpc{}         72\PYGZpc{}        158\PYGZpc{}        375\PYGZpc{}       1551\PYGZpc{}
PAKNECONGOVTKN 0          120\PYGZpc{}        131\PYGZpc{}        185\PYGZpc{}        343\PYGZpc{}       1248\PYGZpc{}
PAKNEGDIFTOTKN 0           74\PYGZpc{}         92\PYGZpc{}        141\PYGZpc{}        267\PYGZpc{}        940\PYGZpc{}
PAKNEGDISTKBKN 0           \PYGZhy{}0\PYGZpc{}         \PYGZhy{}0\PYGZpc{}         \PYGZhy{}0\PYGZpc{}         \PYGZhy{}0\PYGZpc{}         \PYGZhy{}0\PYGZpc{}
PAKNYGDPDISCKN 0           \PYGZhy{}0\PYGZpc{}         \PYGZhy{}0\PYGZpc{}         \PYGZhy{}0\PYGZpc{}         \PYGZhy{}0\PYGZpc{}         \PYGZhy{}0\PYGZpc{}
PAKADAP        0           \PYGZhy{}0\PYGZpc{}         \PYGZhy{}0\PYGZpc{}         \PYGZhy{}0\PYGZpc{}         \PYGZhy{}0\PYGZpc{}         \PYGZhy{}0\PYGZpc{}
PAKDISPREPKN   0           \PYGZhy{}0\PYGZpc{}         \PYGZhy{}0\PYGZpc{}         \PYGZhy{}0\PYGZpc{}         \PYGZhy{}0\PYGZpc{}         \PYGZhy{}0\PYGZpc{}
PAKNEEXPGNFSKN 0           \PYGZhy{}1\PYGZpc{}         \PYGZhy{}3\PYGZpc{}         \PYGZhy{}6\PYGZpc{}        \PYGZhy{}17\PYGZpc{}        \PYGZhy{}77\PYGZpc{}
PAKNECONPRVTKN 0         \PYGZhy{}123\PYGZpc{}       \PYGZhy{}193\PYGZpc{}       \PYGZhy{}377\PYGZpc{}       \PYGZhy{}869\PYGZpc{}      \PYGZhy{}3561\PYGZpc{}
Total          0          100\PYGZpc{}        100\PYGZpc{}        100\PYGZpc{}        100\PYGZpc{}        100\PYGZpc{}
Residual       0           \PYGZhy{}0\PYGZpc{}         \PYGZhy{}0\PYGZpc{}         \PYGZhy{}0\PYGZpc{}         \PYGZhy{}0\PYGZpc{}         \PYGZhy{}0\PYGZpc{}

 Contribution to growth rate PAKNYGDPMKTPKN
                          2021        2022        2023        2024        2025
Variable       lag                                                            
PAKNECONPRVTKN 0         \PYGZhy{}0.0\PYGZpc{}       \PYGZhy{}0.0\PYGZpc{}       \PYGZhy{}0.0\PYGZpc{}       \PYGZhy{}0.0\PYGZpc{}       \PYGZhy{}0.0\PYGZpc{}
PAKNECONGOVTKN 0          0.0\PYGZpc{}        0.0\PYGZpc{}        0.0\PYGZpc{}        0.0\PYGZpc{}        0.0\PYGZpc{}
PAKNEGDIFTOTKN 0          0.0\PYGZpc{}        0.0\PYGZpc{}        0.0\PYGZpc{}        0.0\PYGZpc{}        0.0\PYGZpc{}
PAKNEGDISTKBKN 0         \PYGZhy{}0.0\PYGZpc{}       \PYGZhy{}0.0\PYGZpc{}       \PYGZhy{}0.0\PYGZpc{}       \PYGZhy{}0.0\PYGZpc{}       \PYGZhy{}0.0\PYGZpc{}
PAKNEEXPGNFSKN 0         \PYGZhy{}0.0\PYGZpc{}       \PYGZhy{}0.0\PYGZpc{}       \PYGZhy{}0.0\PYGZpc{}       \PYGZhy{}0.0\PYGZpc{}       \PYGZhy{}0.0\PYGZpc{}
PAKNEIMPGNFSKN 0          0.0\PYGZpc{}        0.0\PYGZpc{}        0.0\PYGZpc{}        0.0\PYGZpc{}        0.0\PYGZpc{}
PAKNYGDPDISCKN 0         \PYGZhy{}0.0\PYGZpc{}       \PYGZhy{}0.0\PYGZpc{}       \PYGZhy{}0.0\PYGZpc{}       \PYGZhy{}0.0\PYGZpc{}       \PYGZhy{}0.0\PYGZpc{}
PAKADAP        0         \PYGZhy{}0.0\PYGZpc{}       \PYGZhy{}0.0\PYGZpc{}       \PYGZhy{}0.0\PYGZpc{}       \PYGZhy{}0.0\PYGZpc{}       \PYGZhy{}0.0\PYGZpc{}
PAKDISPREPKN   0         \PYGZhy{}0.0\PYGZpc{}       \PYGZhy{}0.0\PYGZpc{}       \PYGZhy{}0.0\PYGZpc{}       \PYGZhy{}0.0\PYGZpc{}       \PYGZhy{}0.0\PYGZpc{}

Formula        : FRML \PYGZlt{}DAMP,STOC\PYGZgt{} PAKNECONPRVTKN = (PAKNECONPRVTKN(\PYGZhy{}1)*EXP(PAKNECONPRVTKN\PYGZus{}A+ (\PYGZhy{}0.2*(LOG(PAKNECONPRVTKN(\PYGZhy{}1))\PYGZhy{}LOG(1.21203101101442)\PYGZhy{}LOG((((PAKBXFSTREMTCD(\PYGZhy{}1)\PYGZhy{}PAKBMFSTREMTCD(\PYGZhy{}1))*PAKPANUSATLS(\PYGZhy{}1))+PAKGGEXPTRNSCN(\PYGZhy{}1)+PAKNYYWBTOTLCN(\PYGZhy{}1)*(1\PYGZhy{}PAKGGREVDRCTXN(\PYGZhy{}1)/100))/PAKNECONPRVTXN(\PYGZhy{}1)))+0.763938860758873*((LOG((((PAKBXFSTREMTCD\PYGZhy{}PAKBMFSTREMTCD)*PAKPANUSATLS)+PAKGGEXPTRNSCN+PAKNYYWBTOTLCN*(1\PYGZhy{}PAKGGREVDRCTXN/100))/PAKNECONPRVTXN))\PYGZhy{}(LOG((((PAKBXFSTREMTCD(\PYGZhy{}1)\PYGZhy{}PAKBMFSTREMTCD(\PYGZhy{}1))*PAKPANUSATLS(\PYGZhy{}1))+PAKGGEXPTRNSCN(\PYGZhy{}1)+PAKNYYWBTOTLCN(\PYGZhy{}1)*(1\PYGZhy{}PAKGGREVDRCTXN(\PYGZhy{}1)/100))/PAKNECONPRVTXN(\PYGZhy{}1))))\PYGZhy{}0.0634474791568939*DURING\PYGZus{}2009\PYGZhy{}0.3*(PAKFMLBLPOLYXN/100\PYGZhy{}((LOG(PAKNECONPRVTXN))\PYGZhy{}(LOG(PAKNECONPRVTXN(\PYGZhy{}1)))))) )) * (1\PYGZhy{}PAKNECONPRVTKN\PYGZus{}D)+ PAKNECONPRVTKN\PYGZus{}X*PAKNECONPRVTKN\PYGZus{}D  \PYGZdl{} 

                       2021        2022        2023        2024        2025
Variable    lag                                                            
Base        0   23972815.36 24164128.02 24427863.05 24818524.47 25323255.17
Alternative 0   23660762.40 23769661.89 23953304.14 24286769.32 24759639.22
Difference  0    \PYGZhy{}312052.95  \PYGZhy{}394466.13  \PYGZhy{}474558.91  \PYGZhy{}531755.15  \PYGZhy{}563615.95
Percent     0         \PYGZhy{}1.30       \PYGZhy{}1.63       \PYGZhy{}1.94       \PYGZhy{}2.14       \PYGZhy{}2.23

 Contributions to differende for  PAKNECONPRVTKN
                           2021       2022       2023       2024       2025
Variable         lag                                                       
PAKNECONPRVTKN   \PYGZhy{}1  \PYGZhy{}187434.07 \PYGZhy{}250462.33 \PYGZhy{}317486.72 \PYGZhy{}384175.74 \PYGZhy{}432745.55
PAKNECONPRVTKN\PYGZus{}A  0       \PYGZhy{}0.01      \PYGZhy{}0.01      \PYGZhy{}0.01      \PYGZhy{}0.01      \PYGZhy{}0.04
PAKBXFSTREMTCD   \PYGZhy{}1   \PYGZhy{}38694.42  \PYGZhy{}49412.27  \PYGZhy{}52084.76  \PYGZhy{}50817.15  \PYGZhy{}48170.52
PAKBMFSTREMTCD   \PYGZhy{}1      120.58     140.33     135.37     121.42     106.31
PAKPANUSATLS     \PYGZhy{}1     3137.57    3566.27    3817.26    3916.63    3901.04
PAKGGEXPTRNSCN   \PYGZhy{}1    \PYGZhy{}2382.89   \PYGZhy{}4372.94   \PYGZhy{}5966.91   \PYGZhy{}7223.04   \PYGZhy{}8206.86
PAKNYYWBTOTLCN   \PYGZhy{}1   \PYGZhy{}78794.18 \PYGZhy{}120093.04 \PYGZhy{}145773.43 \PYGZhy{}156461.75 \PYGZhy{}167189.37
PAKGGREVDRCTXN   \PYGZhy{}1       \PYGZhy{}0.01      \PYGZhy{}0.01      \PYGZhy{}0.01      \PYGZhy{}0.01      \PYGZhy{}0.04
PAKNECONPRVTXN   \PYGZhy{}1   204247.65  231199.67  249025.22  258789.48  262005.59
PAKBXFSTREMTCD    0    66466.87   69836.78   67727.29   63861.51   59511.44
PAKBMFSTREMTCD    0     \PYGZhy{}189.25    \PYGZhy{}182.00    \PYGZhy{}162.26    \PYGZhy{}141.32    \PYGZhy{}122.29
PAKPANUSATLS      0    \PYGZhy{}4809.78   \PYGZhy{}5132.40   \PYGZhy{}5233.84   \PYGZhy{}5184.63   \PYGZhy{}5043.02
PAKGGEXPTRNSCN    0     5895.35    8018.71    9646.92   10900.77   11850.22
PAKNYYWBTOTLCN    0   160980.65  194563.25  207466.32  220404.39  237003.52
PAKGGREVDRCTXN    0       \PYGZhy{}0.01      \PYGZhy{}0.01      \PYGZhy{}0.01      \PYGZhy{}0.01      \PYGZhy{}0.04
PAKNECONPRVTXN    0  \PYGZhy{}410022.04 \PYGZhy{}440677.37 \PYGZhy{}455322.70 \PYGZhy{}458425.12 \PYGZhy{}453478.88
DURING\PYGZus{}2009       0       \PYGZhy{}0.01      \PYGZhy{}0.01      \PYGZhy{}0.01      \PYGZhy{}0.01      \PYGZhy{}0.04
PAKFMLBLPOLYXN    0   \PYGZhy{}34994.35  \PYGZhy{}36409.85  \PYGZhy{}35203.57  \PYGZhy{}32189.52  \PYGZhy{}28049.75
PAKNECONPRVTKN\PYGZus{}D  0       \PYGZhy{}0.01      \PYGZhy{}0.01      \PYGZhy{}0.01      \PYGZhy{}0.01      \PYGZhy{}0.04
PAKNECONPRVTKN\PYGZus{}X  0       \PYGZhy{}0.01      \PYGZhy{}0.01      \PYGZhy{}0.01      \PYGZhy{}0.01      \PYGZhy{}0.04

 Share of contributions to differende for  PAKNECONPRVTKN
                            2021        2022        2023        2024        2025
Variable         lag                                                            
PAKNECONPRVTXN    0         131\PYGZpc{}        112\PYGZpc{}         96\PYGZpc{}         86\PYGZpc{}         80\PYGZpc{}
PAKNECONPRVTKN   \PYGZhy{}1          60\PYGZpc{}         63\PYGZpc{}         67\PYGZpc{}         72\PYGZpc{}         77\PYGZpc{}
PAKNYYWBTOTLCN   \PYGZhy{}1          25\PYGZpc{}         30\PYGZpc{}         31\PYGZpc{}         29\PYGZpc{}         30\PYGZpc{}
PAKBXFSTREMTCD   \PYGZhy{}1          12\PYGZpc{}         13\PYGZpc{}         11\PYGZpc{}         10\PYGZpc{}          9\PYGZpc{}
PAKFMLBLPOLYXN    0          11\PYGZpc{}          9\PYGZpc{}          7\PYGZpc{}          6\PYGZpc{}          5\PYGZpc{}
PAKGGEXPTRNSCN   \PYGZhy{}1           1\PYGZpc{}          1\PYGZpc{}          1\PYGZpc{}          1\PYGZpc{}          1\PYGZpc{}
PAKPANUSATLS      0           2\PYGZpc{}          1\PYGZpc{}          1\PYGZpc{}          1\PYGZpc{}          1\PYGZpc{}
PAKBMFSTREMTCD    0           0\PYGZpc{}          0\PYGZpc{}          0\PYGZpc{}          0\PYGZpc{}          0\PYGZpc{}
PAKNECONPRVTKN\PYGZus{}A  0           0\PYGZpc{}          0\PYGZpc{}          0\PYGZpc{}          0\PYGZpc{}          0\PYGZpc{}
PAKGGREVDRCTXN   \PYGZhy{}1           0\PYGZpc{}          0\PYGZpc{}          0\PYGZpc{}          0\PYGZpc{}          0\PYGZpc{}
                  0           0\PYGZpc{}          0\PYGZpc{}          0\PYGZpc{}          0\PYGZpc{}          0\PYGZpc{}
DURING\PYGZus{}2009       0           0\PYGZpc{}          0\PYGZpc{}          0\PYGZpc{}          0\PYGZpc{}          0\PYGZpc{}
PAKNECONPRVTKN\PYGZus{}D  0           0\PYGZpc{}          0\PYGZpc{}          0\PYGZpc{}          0\PYGZpc{}          0\PYGZpc{}
PAKNECONPRVTKN\PYGZus{}X  0           0\PYGZpc{}          0\PYGZpc{}          0\PYGZpc{}          0\PYGZpc{}          0\PYGZpc{}
PAKBMFSTREMTCD   \PYGZhy{}1          \PYGZhy{}0\PYGZpc{}         \PYGZhy{}0\PYGZpc{}         \PYGZhy{}0\PYGZpc{}         \PYGZhy{}0\PYGZpc{}         \PYGZhy{}0\PYGZpc{}
PAKPANUSATLS     \PYGZhy{}1          \PYGZhy{}1\PYGZpc{}         \PYGZhy{}1\PYGZpc{}         \PYGZhy{}1\PYGZpc{}         \PYGZhy{}1\PYGZpc{}         \PYGZhy{}1\PYGZpc{}
PAKGGEXPTRNSCN    0          \PYGZhy{}2\PYGZpc{}         \PYGZhy{}2\PYGZpc{}         \PYGZhy{}2\PYGZpc{}         \PYGZhy{}2\PYGZpc{}         \PYGZhy{}2\PYGZpc{}
PAKBXFSTREMTCD    0         \PYGZhy{}21\PYGZpc{}        \PYGZhy{}18\PYGZpc{}        \PYGZhy{}14\PYGZpc{}        \PYGZhy{}12\PYGZpc{}        \PYGZhy{}11\PYGZpc{}
PAKNYYWBTOTLCN    0         \PYGZhy{}52\PYGZpc{}        \PYGZhy{}49\PYGZpc{}        \PYGZhy{}44\PYGZpc{}        \PYGZhy{}41\PYGZpc{}        \PYGZhy{}42\PYGZpc{}
PAKNECONPRVTXN   \PYGZhy{}1         \PYGZhy{}65\PYGZpc{}        \PYGZhy{}59\PYGZpc{}        \PYGZhy{}52\PYGZpc{}        \PYGZhy{}49\PYGZpc{}        \PYGZhy{}46\PYGZpc{}
Total             0         101\PYGZpc{}        101\PYGZpc{}        101\PYGZpc{}        101\PYGZpc{}        101\PYGZpc{}
Residual          0           1\PYGZpc{}          1\PYGZpc{}          1\PYGZpc{}          1\PYGZpc{}          1\PYGZpc{}

 Contribution to growth rate PAKNECONPRVTKN
                            2021        2022        2023        2024        2025
Variable         lag                                                            
PAKNECONPRVTKN   \PYGZhy{}1         0.0\PYGZpc{}        0.0\PYGZpc{}        0.0\PYGZpc{}        0.0\PYGZpc{}        0.0\PYGZpc{}
PAKNECONPRVTKN\PYGZus{}A  0        \PYGZhy{}0.0\PYGZpc{}       \PYGZhy{}0.0\PYGZpc{}       \PYGZhy{}0.0\PYGZpc{}       \PYGZhy{}0.0\PYGZpc{}       \PYGZhy{}0.0\PYGZpc{}
PAKBXFSTREMTCD   \PYGZhy{}1        \PYGZhy{}0.0\PYGZpc{}       \PYGZhy{}0.0\PYGZpc{}       \PYGZhy{}0.0\PYGZpc{}       \PYGZhy{}0.0\PYGZpc{}       \PYGZhy{}0.0\PYGZpc{}
PAKBMFSTREMTCD   \PYGZhy{}1         0.0\PYGZpc{}        0.0\PYGZpc{}        0.0\PYGZpc{}        0.0\PYGZpc{}        0.0\PYGZpc{}
PAKPANUSATLS     \PYGZhy{}1         0.0\PYGZpc{}        0.0\PYGZpc{}        0.0\PYGZpc{}        0.0\PYGZpc{}        0.0\PYGZpc{}
PAKGGEXPTRNSCN   \PYGZhy{}1        \PYGZhy{}0.0\PYGZpc{}       \PYGZhy{}0.0\PYGZpc{}       \PYGZhy{}0.0\PYGZpc{}       \PYGZhy{}0.0\PYGZpc{}       \PYGZhy{}0.0\PYGZpc{}
PAKNYYWBTOTLCN   \PYGZhy{}1        \PYGZhy{}0.0\PYGZpc{}       \PYGZhy{}0.0\PYGZpc{}       \PYGZhy{}0.0\PYGZpc{}       \PYGZhy{}0.0\PYGZpc{}       \PYGZhy{}0.0\PYGZpc{}
PAKGGREVDRCTXN   \PYGZhy{}1        \PYGZhy{}0.0\PYGZpc{}       \PYGZhy{}0.0\PYGZpc{}       \PYGZhy{}0.0\PYGZpc{}       \PYGZhy{}0.0\PYGZpc{}       \PYGZhy{}0.0\PYGZpc{}
PAKNECONPRVTXN   \PYGZhy{}1         0.0\PYGZpc{}        0.0\PYGZpc{}        0.0\PYGZpc{}        0.0\PYGZpc{}        0.0\PYGZpc{}
PAKBXFSTREMTCD    0         0.0\PYGZpc{}        0.0\PYGZpc{}        0.0\PYGZpc{}        0.0\PYGZpc{}        0.0\PYGZpc{}
PAKBMFSTREMTCD    0        \PYGZhy{}0.0\PYGZpc{}       \PYGZhy{}0.0\PYGZpc{}       \PYGZhy{}0.0\PYGZpc{}       \PYGZhy{}0.0\PYGZpc{}       \PYGZhy{}0.0\PYGZpc{}
PAKPANUSATLS      0        \PYGZhy{}0.0\PYGZpc{}       \PYGZhy{}0.0\PYGZpc{}       \PYGZhy{}0.0\PYGZpc{}       \PYGZhy{}0.0\PYGZpc{}       \PYGZhy{}0.0\PYGZpc{}
PAKGGEXPTRNSCN    0         0.0\PYGZpc{}        0.0\PYGZpc{}        0.0\PYGZpc{}        0.0\PYGZpc{}        0.0\PYGZpc{}
PAKNYYWBTOTLCN    0         0.0\PYGZpc{}        0.0\PYGZpc{}        0.0\PYGZpc{}        0.0\PYGZpc{}        0.0\PYGZpc{}
PAKGGREVDRCTXN    0        \PYGZhy{}0.0\PYGZpc{}       \PYGZhy{}0.0\PYGZpc{}       \PYGZhy{}0.0\PYGZpc{}       \PYGZhy{}0.0\PYGZpc{}       \PYGZhy{}0.0\PYGZpc{}
PAKNECONPRVTXN    0        \PYGZhy{}0.0\PYGZpc{}       \PYGZhy{}0.0\PYGZpc{}       \PYGZhy{}0.0\PYGZpc{}       \PYGZhy{}0.0\PYGZpc{}       \PYGZhy{}0.0\PYGZpc{}
DURING\PYGZus{}2009       0        \PYGZhy{}0.0\PYGZpc{}       \PYGZhy{}0.0\PYGZpc{}       \PYGZhy{}0.0\PYGZpc{}       \PYGZhy{}0.0\PYGZpc{}       \PYGZhy{}0.0\PYGZpc{}
PAKFMLBLPOLYXN    0        \PYGZhy{}0.0\PYGZpc{}       \PYGZhy{}0.0\PYGZpc{}       \PYGZhy{}0.0\PYGZpc{}       \PYGZhy{}0.0\PYGZpc{}       \PYGZhy{}0.0\PYGZpc{}
PAKNECONPRVTKN\PYGZus{}D  0        \PYGZhy{}0.0\PYGZpc{}       \PYGZhy{}0.0\PYGZpc{}       \PYGZhy{}0.0\PYGZpc{}       \PYGZhy{}0.0\PYGZpc{}       \PYGZhy{}0.0\PYGZpc{}
PAKNECONPRVTKN\PYGZus{}X  0        \PYGZhy{}0.0\PYGZpc{}       \PYGZhy{}0.0\PYGZpc{}       \PYGZhy{}0.0\PYGZpc{}       \PYGZhy{}0.0\PYGZpc{}       \PYGZhy{}0.0\PYGZpc{}
\end{sphinxVerbatim}

\end{sphinxuseclass}\end{sphinxVerbatimOutput}

\end{sphinxuseclass}

\section{Bespoken plots using matplotlib  (or plotly \sphinxhyphen{}later) (should go to a separate plot book}
\label{\detokenize{content/notebooks/modelflow_features:bespoken-plots-using-matplotlib-or-plotly-later-should-go-to-a-separate-plot-book}}
\sphinxAtStartPar
The predefined plots are not necessary created for presentation purpose. To create  bespoken plots the they can be
constructed directly in python scripts. The two main libraries are matplotlib, plotly but any ther python plotting library can be used. Here is an example using matplotlib.


\section{Plot four separate plots of multiple series in grid}
\label{\detokenize{content/notebooks/modelflow_features:plot-four-separate-plots-of-multiple-series-in-grid}}
\begin{sphinxuseclass}{cell}\begin{sphinxVerbatimInput}

\begin{sphinxuseclass}{cell_input}
\begin{sphinxVerbatim}[commandchars=\\\{\}]
\PYG{n}{figure}\PYG{p}{,}\PYG{n}{axs}\PYG{o}{=} \PYG{n}{plt}\PYG{o}{.}\PYG{n}{subplots}\PYG{p}{(}\PYG{l+m+mi}{2}\PYG{p}{,}\PYG{l+m+mi}{2}\PYG{p}{,}\PYG{n}{figsize}\PYG{o}{=}\PYG{p}{(}\PYG{l+m+mi}{11}\PYG{p}{,} \PYG{l+m+mi}{7}\PYG{p}{)}\PYG{p}{)}
\PYG{n}{axs}\PYG{p}{[}\PYG{l+m+mi}{0}\PYG{p}{,}\PYG{l+m+mi}{0}\PYG{p}{]}\PYG{o}{.}\PYG{n}{plot}\PYG{p}{(}\PYG{n}{mpak}\PYG{o}{.}\PYG{n}{basedf}\PYG{o}{.}\PYG{n}{loc}\PYG{p}{[}\PYG{l+m+mi}{2020}\PYG{p}{:}\PYG{l+m+mi}{2099}\PYG{p}{,}\PYG{l+s+s1}{\PYGZsq{}}\PYG{l+s+s1}{PAKGGBALOVRLCN\PYGZus{}}\PYG{l+s+s1}{\PYGZsq{}}\PYG{p}{]}\PYG{p}{,}\PYG{n}{label}\PYG{o}{=}\PYG{l+s+s1}{\PYGZsq{}}\PYG{l+s+s1}{Baseline}\PYG{l+s+s1}{\PYGZsq{}}\PYG{p}{)}
\PYG{n}{axs}\PYG{p}{[}\PYG{l+m+mi}{0}\PYG{p}{,}\PYG{l+m+mi}{0}\PYG{p}{]}\PYG{o}{.}\PYG{n}{plot}\PYG{p}{(}\PYG{n}{mpak}\PYG{o}{.}\PYG{n}{lastdf}\PYG{o}{.}\PYG{n}{loc}\PYG{p}{[}\PYG{l+m+mi}{2020}\PYG{p}{:}\PYG{l+m+mi}{2099}\PYG{p}{,}\PYG{l+s+s1}{\PYGZsq{}}\PYG{l+s+s1}{PAKGGBALOVRLCN\PYGZus{}}\PYG{l+s+s1}{\PYGZsq{}}\PYG{p}{]}\PYG{p}{,}\PYG{n}{label}\PYG{o}{=}\PYG{l+s+s1}{\PYGZsq{}}\PYG{l+s+s1}{Scenario}\PYG{l+s+s1}{\PYGZsq{}}\PYG{p}{)}
\PYG{c+c1}{\PYGZsh{}axs[0,0].legend()}

\PYG{n}{axs}\PYG{p}{[}\PYG{l+m+mi}{0}\PYG{p}{,}\PYG{l+m+mi}{1}\PYG{p}{]}\PYG{o}{.}\PYG{n}{plot}\PYG{p}{(}\PYG{n}{mpak}\PYG{o}{.}\PYG{n}{basedf}\PYG{o}{.}\PYG{n}{loc}\PYG{p}{[}\PYG{l+m+mi}{2020}\PYG{p}{:}\PYG{l+m+mi}{2099}\PYG{p}{,}\PYG{l+s+s1}{\PYGZsq{}}\PYG{l+s+s1}{PAKGGDBTTOTLCN\PYGZus{}}\PYG{l+s+s1}{\PYGZsq{}}\PYG{p}{]}\PYG{p}{,}\PYG{n}{label}\PYG{o}{=}\PYG{l+s+s1}{\PYGZsq{}}\PYG{l+s+s1}{Baseline}\PYG{l+s+s1}{\PYGZsq{}}\PYG{p}{)}
\PYG{n}{axs}\PYG{p}{[}\PYG{l+m+mi}{0}\PYG{p}{,}\PYG{l+m+mi}{1}\PYG{p}{]}\PYG{o}{.}\PYG{n}{plot}\PYG{p}{(}\PYG{n}{mpak}\PYG{o}{.}\PYG{n}{lastdf}\PYG{o}{.}\PYG{n}{loc}\PYG{p}{[}\PYG{l+m+mi}{2020}\PYG{p}{:}\PYG{l+m+mi}{2099}\PYG{p}{,}\PYG{l+s+s1}{\PYGZsq{}}\PYG{l+s+s1}{PAKGGDBTTOTLCN\PYGZus{}}\PYG{l+s+s1}{\PYGZsq{}}\PYG{p}{]}\PYG{p}{,}\PYG{n}{label}\PYG{o}{=}\PYG{l+s+s1}{\PYGZsq{}}\PYG{l+s+s1}{Scenario}\PYG{l+s+s1}{\PYGZsq{}}\PYG{p}{)}

\PYG{n}{axs}\PYG{p}{[}\PYG{l+m+mi}{1}\PYG{p}{,}\PYG{l+m+mi}{0}\PYG{p}{]}\PYG{o}{.}\PYG{n}{plot}\PYG{p}{(}\PYG{n}{mpak}\PYG{o}{.}\PYG{n}{basedf}\PYG{o}{.}\PYG{n}{loc}\PYG{p}{[}\PYG{l+m+mi}{2020}\PYG{p}{:}\PYG{l+m+mi}{2099}\PYG{p}{,}\PYG{l+s+s1}{\PYGZsq{}}\PYG{l+s+s1}{PAKGGREVTOTLCN}\PYG{l+s+s1}{\PYGZsq{}}\PYG{p}{]}\PYG{o}{/}\PYG{n}{mpak}\PYG{o}{.}\PYG{n}{basedf}\PYG{o}{.}\PYG{n}{loc}\PYG{p}{[}\PYG{l+m+mi}{2020}\PYG{p}{:}\PYG{l+m+mi}{2099}\PYG{p}{,}\PYG{l+s+s1}{\PYGZsq{}}\PYG{l+s+s1}{PAKNYGDPMKTPCN}\PYG{l+s+s1}{\PYGZsq{}}\PYG{p}{]}\PYG{o}{*}\PYG{l+m+mi}{100}\PYG{p}{,}\PYG{n}{label}\PYG{o}{=}\PYG{l+s+s1}{\PYGZsq{}}\PYG{l+s+s1}{Baseline}\PYG{l+s+s1}{\PYGZsq{}}\PYG{p}{)}
\PYG{n}{axs}\PYG{p}{[}\PYG{l+m+mi}{1}\PYG{p}{,}\PYG{l+m+mi}{0}\PYG{p}{]}\PYG{o}{.}\PYG{n}{plot}\PYG{p}{(}\PYG{n}{mpak}\PYG{o}{.}\PYG{n}{lastdf}\PYG{o}{.}\PYG{n}{loc}\PYG{p}{[}\PYG{l+m+mi}{2020}\PYG{p}{:}\PYG{l+m+mi}{2099}\PYG{p}{,}\PYG{l+s+s1}{\PYGZsq{}}\PYG{l+s+s1}{PAKGGREVTOTLCN}\PYG{l+s+s1}{\PYGZsq{}}\PYG{p}{]}\PYG{o}{/}\PYG{n}{mpak}\PYG{o}{.}\PYG{n}{lastdf}\PYG{o}{.}\PYG{n}{loc}\PYG{p}{[}\PYG{l+m+mi}{2020}\PYG{p}{:}\PYG{l+m+mi}{2099}\PYG{p}{,}\PYG{l+s+s1}{\PYGZsq{}}\PYG{l+s+s1}{PAKNYGDPMKTPCN}\PYG{l+s+s1}{\PYGZsq{}}\PYG{p}{]}\PYG{o}{*}\PYG{l+m+mi}{100}\PYG{p}{,}\PYG{n}{label}\PYG{o}{=}\PYG{l+s+s1}{\PYGZsq{}}\PYG{l+s+s1}{Scenario}\PYG{l+s+s1}{\PYGZsq{}}\PYG{p}{)}

\PYG{n}{axs}\PYG{p}{[}\PYG{l+m+mi}{1}\PYG{p}{,}\PYG{l+m+mi}{1}\PYG{p}{]}\PYG{o}{.}\PYG{n}{plot}\PYG{p}{(}\PYG{n}{mpak}\PYG{o}{.}\PYG{n}{basedf}\PYG{o}{.}\PYG{n}{loc}\PYG{p}{[}\PYG{l+m+mi}{2020}\PYG{p}{:}\PYG{l+m+mi}{2099}\PYG{p}{,}\PYG{l+s+s1}{\PYGZsq{}}\PYG{l+s+s1}{PAKGGREVGRNTCN}\PYG{l+s+s1}{\PYGZsq{}}\PYG{p}{]}\PYG{o}{/}\PYG{n}{mpak}\PYG{o}{.}\PYG{n}{basedf}\PYG{o}{.}\PYG{n}{loc}\PYG{p}{[}\PYG{l+m+mi}{2020}\PYG{p}{:}\PYG{l+m+mi}{2099}\PYG{p}{,}\PYG{l+s+s1}{\PYGZsq{}}\PYG{l+s+s1}{PAKNYGDPMKTPCN}\PYG{l+s+s1}{\PYGZsq{}}\PYG{p}{]}\PYG{o}{*}\PYG{l+m+mi}{100}\PYG{p}{,}\PYG{n}{label}\PYG{o}{=}\PYG{l+s+s1}{\PYGZsq{}}\PYG{l+s+s1}{Baseline}\PYG{l+s+s1}{\PYGZsq{}}\PYG{p}{)}
\PYG{n}{axs}\PYG{p}{[}\PYG{l+m+mi}{1}\PYG{p}{,}\PYG{l+m+mi}{1}\PYG{p}{]}\PYG{o}{.}\PYG{n}{plot}\PYG{p}{(}\PYG{n}{mpak}\PYG{o}{.}\PYG{n}{lastdf}\PYG{o}{.}\PYG{n}{loc}\PYG{p}{[}\PYG{l+m+mi}{2020}\PYG{p}{:}\PYG{l+m+mi}{2099}\PYG{p}{,}\PYG{l+s+s1}{\PYGZsq{}}\PYG{l+s+s1}{PAKGGREVGRNTCN}\PYG{l+s+s1}{\PYGZsq{}}\PYG{p}{]}\PYG{o}{/}\PYG{n}{mpak}\PYG{o}{.}\PYG{n}{lastdf}\PYG{o}{.}\PYG{n}{loc}\PYG{p}{[}\PYG{l+m+mi}{2020}\PYG{p}{:}\PYG{l+m+mi}{2099}\PYG{p}{,}\PYG{l+s+s1}{\PYGZsq{}}\PYG{l+s+s1}{PAKNYGDPMKTPCN}\PYG{l+s+s1}{\PYGZsq{}}\PYG{p}{]}\PYG{o}{*}\PYG{l+m+mi}{100}\PYG{p}{,}\PYG{n}{label}\PYG{o}{=}\PYG{l+s+s1}{\PYGZsq{}}\PYG{l+s+s1}{Scenario}\PYG{l+s+s1}{\PYGZsq{}}\PYG{p}{)}
\PYG{c+c1}{\PYGZsh{}axs2[4].plot(mpak.lastdf.loc[2000:2099,\PYGZsq{}PAKGGREVGRNTCN\PYGZsq{}]/mpak.basedf.loc[2000:2099,\PYGZsq{}PAKNYGDPMKTPCN\PYGZsq{}]*100,label=\PYGZsq{}Scenario\PYGZsq{})}

\PYG{n}{axs}\PYG{p}{[}\PYG{l+m+mi}{0}\PYG{p}{,}\PYG{l+m+mi}{0}\PYG{p}{]}\PYG{o}{.}\PYG{n}{title}\PYG{o}{.}\PYG{n}{set\PYGZus{}text}\PYG{p}{(}\PYG{l+s+s2}{\PYGZdq{}}\PYG{l+s+s2}{Fiscal balance (}\PYG{l+s+si}{\PYGZpc{} o}\PYG{l+s+s2}{f GDP)}\PYG{l+s+s2}{\PYGZdq{}}\PYG{p}{)}
\PYG{n}{axs}\PYG{p}{[}\PYG{l+m+mi}{0}\PYG{p}{,}\PYG{l+m+mi}{1}\PYG{p}{]}\PYG{o}{.}\PYG{n}{title}\PYG{o}{.}\PYG{n}{set\PYGZus{}text}\PYG{p}{(}\PYG{l+s+s2}{\PYGZdq{}}\PYG{l+s+s2}{Gov}\PYG{l+s+s2}{\PYGZsq{}}\PYG{l+s+s2}{t Debt (}\PYG{l+s+si}{\PYGZpc{} o}\PYG{l+s+s2}{f GDP)}\PYG{l+s+s2}{\PYGZdq{}}\PYG{p}{)}
\PYG{n}{axs}\PYG{p}{[}\PYG{l+m+mi}{1}\PYG{p}{,}\PYG{l+m+mi}{0}\PYG{p}{]}\PYG{o}{.}\PYG{n}{title}\PYG{o}{.}\PYG{n}{set\PYGZus{}text}\PYG{p}{(}\PYG{l+s+s2}{\PYGZdq{}}\PYG{l+s+s2}{Total revenues (}\PYG{l+s+si}{\PYGZpc{} o}\PYG{l+s+s2}{f GDP)}\PYG{l+s+s2}{\PYGZdq{}}\PYG{p}{)}
\PYG{n}{axs}\PYG{p}{[}\PYG{l+m+mi}{1}\PYG{p}{,}\PYG{l+m+mi}{1}\PYG{p}{]}\PYG{o}{.}\PYG{n}{title}\PYG{o}{.}\PYG{n}{set\PYGZus{}text}\PYG{p}{(}\PYG{l+s+s2}{\PYGZdq{}}\PYG{l+s+s2}{Grant Revenues (}\PYG{l+s+si}{\PYGZpc{} o}\PYG{l+s+s2}{f GDP)}\PYG{l+s+s2}{\PYGZdq{}}\PYG{p}{)}
\PYG{n}{figure}\PYG{o}{.}\PYG{n}{suptitle}\PYG{p}{(}\PYG{l+s+s2}{\PYGZdq{}}\PYG{l+s+s2}{Fiscal outcomes}\PYG{l+s+s2}{\PYGZdq{}}\PYG{p}{)}

\PYG{n}{plt}\PYG{o}{.}\PYG{n}{figlegend}\PYG{p}{(}\PYG{p}{[}\PYG{l+s+s1}{\PYGZsq{}}\PYG{l+s+s1}{Baseline}\PYG{l+s+s1}{\PYGZsq{}}\PYG{p}{,}\PYG{l+s+s1}{\PYGZsq{}}\PYG{l+s+s1}{Scenario}\PYG{l+s+s1}{\PYGZsq{}}\PYG{p}{]}\PYG{p}{,}\PYG{n}{loc}\PYG{o}{=}\PYG{l+s+s1}{\PYGZsq{}}\PYG{l+s+s1}{lower left}\PYG{l+s+s1}{\PYGZsq{}}\PYG{p}{,}\PYG{n}{ncol}\PYG{o}{=}\PYG{l+m+mi}{5}\PYG{p}{)}  
\PYG{n}{figure}\PYG{o}{.}\PYG{n}{tight\PYGZus{}layout}\PYG{p}{(}\PYG{n}{pad}\PYG{o}{=}\PYG{l+m+mf}{2.3}\PYG{p}{)} \PYG{c+c1}{\PYGZsh{}Ensures legend does not overlap dates}
\PYG{n}{figure}
\end{sphinxVerbatim}

\end{sphinxuseclass}\end{sphinxVerbatimInput}
\begin{sphinxVerbatimOutput}

\begin{sphinxuseclass}{cell_output}
\noindent\sphinxincludegraphics{{af263b8b7718467e4adcce4f07a1db6ce46b66ee3c78d4e8505248794bb2d008}.png}

\noindent\sphinxincludegraphics{{1240ed8e6a959d1fd107938c26fed5bd58cac2184ed1d07447db8b2f65dcc06a}.png}

\end{sphinxuseclass}\end{sphinxVerbatimOutput}

\end{sphinxuseclass}
\sphinxstepscope


\part{Target and instruments}

\sphinxstepscope


\chapter{One target one instrument with 3 instrument variables}
\label{\detokenize{content/howto/targetinstruments/One target one instrument with 3 instrument variables:one-target-one-instrument-with-3-instrument-variables}}\label{\detokenize{content/howto/targetinstruments/One target one instrument with 3 instrument variables::doc}}
\sphinxAtStartPar
This notebook shows how to make a single simple experiment
\begin{enumerate}
\sphinxsetlistlabels{\arabic}{enumi}{enumii}{}{.}%
\item {} 
\sphinxAtStartPar
Loading a pre\sphinxhyphen{}existing model in Modelflow

\item {} 
\sphinxAtStartPar
Creating an experimet by updating some variables

\item {} 
\sphinxAtStartPar
Simulating the model

\item {} 
\sphinxAtStartPar
Visualizing the results

\end{enumerate}

\sphinxAtStartPar
This Notebook uses a  model for Pakistan described here: 

\begin{sphinxuseclass}{cell}\begin{sphinxVerbatimInput}

\begin{sphinxuseclass}{cell_input}
\begin{sphinxVerbatim}[commandchars=\\\{\}]
\PYG{o}{\PYGZpc{}}\PYG{k}{matplotlib} inline
\end{sphinxVerbatim}

\end{sphinxuseclass}\end{sphinxVerbatimInput}

\end{sphinxuseclass}

\section{Imports}
\label{\detokenize{content/howto/targetinstruments/One target one instrument with 3 instrument variables:imports}}
\sphinxAtStartPar
Modelflow’s modelclass includes most of the methods needed to manage a model in Modelflow.

\begin{sphinxuseclass}{cell}\begin{sphinxVerbatimInput}

\begin{sphinxuseclass}{cell_input}
\begin{sphinxVerbatim}[commandchars=\\\{\}]
\PYG{k+kn}{from} \PYG{n+nn}{modelclass} \PYG{k+kn}{import} \PYG{n}{model} 
\PYG{k+kn}{import} \PYG{n+nn}{modelmf}
\PYG{n}{model}\PYG{o}{.}\PYG{n}{widescreen}\PYG{p}{(}\PYG{p}{)}
\PYG{n}{model}\PYG{o}{.}\PYG{n}{scroll\PYGZus{}off}\PYG{p}{(}\PYG{p}{)}
\end{sphinxVerbatim}

\end{sphinxuseclass}\end{sphinxVerbatimInput}
\begin{sphinxVerbatimOutput}

\begin{sphinxuseclass}{cell_output}
\begin{sphinxVerbatim}[commandchars=\\\{\}]
\PYGZlt{}IPython.core.display.HTML object\PYGZgt{}
\end{sphinxVerbatim}

\end{sphinxuseclass}\end{sphinxVerbatimOutput}

\end{sphinxuseclass}
\begin{sphinxuseclass}{cell}\begin{sphinxVerbatimInput}

\begin{sphinxuseclass}{cell_input}
\begin{sphinxVerbatim}[commandchars=\\\{\}]
\PYG{k+kn}{import} \PYG{n+nn}{modelmf}

\PYG{k+kn}{from} \PYG{n+nn}{modelinvert} \PYG{k+kn}{import} \PYG{n}{targets\PYGZus{}instruments}
\PYG{k+kn}{from} \PYG{n+nn}{modelclass} \PYG{k+kn}{import} \PYG{n}{model}
\end{sphinxVerbatim}

\end{sphinxuseclass}\end{sphinxVerbatimInput}

\end{sphinxuseclass}

\section{Load a pre\sphinxhyphen{}existing model, data and descriptions}
\label{\detokenize{content/howto/targetinstruments/One target one instrument with 3 instrument variables:load-a-pre-existing-model-data-and-descriptions}}
\sphinxAtStartPar
The file \sphinxcode{\sphinxupquote{pak.pcim}} contains a dump of model equations, dataframe, simulation options and variable descriptions. The file has been created when onboarding the model.
Examples can be found \DUrole{xref,myst}{here}

\begin{sphinxuseclass}{cell}\begin{sphinxVerbatimInput}

\begin{sphinxuseclass}{cell_input}
\begin{sphinxVerbatim}[commandchars=\\\{\}]
\PYG{n}{mpak}\PYG{p}{,}\PYG{n}{baseline} \PYG{o}{=} \PYG{n}{model}\PYG{o}{.}\PYG{n}{modelload}\PYG{p}{(}\PYG{l+s+s1}{\PYGZsq{}}\PYG{l+s+s1}{../../models/pak.pcim}\PYG{l+s+s1}{\PYGZsq{}}\PYG{p}{,}\PYG{n}{alfa}\PYG{o}{=}\PYG{l+m+mf}{0.7}\PYG{p}{,}\PYG{n}{run}\PYG{o}{=}\PYG{l+m+mi}{1}\PYG{p}{,}\PYG{n}{ljit}\PYG{o}{=}\PYG{l+m+mi}{0}\PYG{p}{)}
\end{sphinxVerbatim}

\end{sphinxuseclass}\end{sphinxVerbatimInput}
\begin{sphinxVerbatimOutput}

\begin{sphinxuseclass}{cell_output}
\begin{sphinxVerbatim}[commandchars=\\\{\}]
file read:  C:\PYGZbs{}modelflow manual\PYGZbs{}papers\PYGZbs{}mfbook\PYGZbs{}content\PYGZbs{}models\PYGZbs{}pak.pcim
\end{sphinxVerbatim}

\end{sphinxuseclass}\end{sphinxVerbatimOutput}

\end{sphinxuseclass}

\section{Define targets and instruments}
\label{\detokenize{content/howto/targetinstruments/One target one instrument with 3 instrument variables:define-targets-and-instruments}}
\sphinxAtStartPar
In year 2100 we want to reduce the emission \sphinxstylestrong{PAKCCEMISCO2TKN} to a percent of the baseline (business as usual)

\sphinxAtStartPar
There is a tax rate for 3 different emission fuel types:

\begin{sphinxuseclass}{cell}\begin{sphinxVerbatimInput}

\begin{sphinxuseclass}{cell_input}
\begin{sphinxVerbatim}[commandchars=\\\{\}]
\PYG{k}{for} \PYG{n}{variable} \PYG{o+ow}{in} \PYG{p}{[}\PYG{l+s+s1}{\PYGZsq{}}\PYG{l+s+s1}{PAKGGREVCO2CER}\PYG{l+s+s1}{\PYGZsq{}}\PYG{p}{,} \PYG{l+s+s1}{\PYGZsq{}}\PYG{l+s+s1}{PAKGGREVCO2GER}\PYG{l+s+s1}{\PYGZsq{}}\PYG{p}{,} \PYG{l+s+s1}{\PYGZsq{}}\PYG{l+s+s1}{PAKGGREVCO2OER}\PYG{l+s+s1}{\PYGZsq{}}\PYG{p}{,} \PYG{p}{]}\PYG{p}{:}
    \PYG{n+nb}{print}\PYG{p}{(}\PYG{n}{variable}\PYG{p}{,}\PYG{l+s+s1}{\PYGZsq{}}\PYG{l+s+s1}{:}\PYG{l+s+s1}{\PYGZsq{}}\PYG{p}{,}\PYG{n}{mpak}\PYG{o}{.}\PYG{n}{var\PYGZus{}description}\PYG{p}{[}\PYG{n}{variable}\PYG{p}{]}\PYG{p}{)}
    
\end{sphinxVerbatim}

\end{sphinxuseclass}\end{sphinxVerbatimInput}
\begin{sphinxVerbatimOutput}

\begin{sphinxuseclass}{cell_output}
\begin{sphinxVerbatim}[commandchars=\\\{\}]
PAKGGREVCO2CER : Carbon tax on coal (USD/t)
PAKGGREVCO2GER : Carbon tax on gas (USD/t)
PAKGGREVCO2OER : Carbon tax on oil (USD/t)
\end{sphinxVerbatim}

\end{sphinxuseclass}\end{sphinxVerbatimOutput}

\end{sphinxuseclass}
\sphinxAtStartPar
So the tax instrument consists of 3 variables.

\begin{sphinxuseclass}{cell}\begin{sphinxVerbatimInput}

\begin{sphinxuseclass}{cell_input}
\begin{sphinxVerbatim}[commandchars=\\\{\}]
\PYG{n}{target\PYGZus{}var} \PYG{o}{=} \PYG{p}{[}\PYG{l+s+s1}{\PYGZsq{}}\PYG{l+s+s1}{PAKCCEMISCO2TKN}\PYG{l+s+s1}{\PYGZsq{}}\PYG{p}{]}
\PYG{n}{instruments} \PYG{o}{=} \PYG{p}{[}\PYG{p}{[}\PYG{l+s+s1}{\PYGZsq{}}\PYG{l+s+s1}{PAKGGREVCO2CER}\PYG{l+s+s1}{\PYGZsq{}}\PYG{p}{,}\PYG{l+s+s1}{\PYGZsq{}}\PYG{l+s+s1}{PAKGGREVCO2GER}\PYG{l+s+s1}{\PYGZsq{}}\PYG{p}{,} \PYG{l+s+s1}{\PYGZsq{}}\PYG{l+s+s1}{PAKGGREVCO2OER}\PYG{l+s+s1}{\PYGZsq{}}\PYG{p}{]}\PYG{p}{]}
\end{sphinxVerbatim}

\end{sphinxuseclass}\end{sphinxVerbatimInput}

\end{sphinxuseclass}

\section{Set the target reduction}
\label{\detokenize{content/howto/targetinstruments/One target one instrument with 3 instrument variables:set-the-target-reduction}}
\begin{sphinxuseclass}{cell}\begin{sphinxVerbatimInput}

\begin{sphinxuseclass}{cell_input}
\begin{sphinxVerbatim}[commandchars=\\\{\}]
\PYG{n}{reduction\PYGZus{}percent} \PYG{o}{=} \PYG{l+m+mi}{30}   \PYG{c+c1}{\PYGZsh{} Input the desired reduction }
\end{sphinxVerbatim}

\end{sphinxuseclass}\end{sphinxVerbatimInput}

\end{sphinxuseclass}
\begin{sphinxuseclass}{cell}\begin{sphinxVerbatimInput}

\begin{sphinxuseclass}{cell_input}
\begin{sphinxVerbatim}[commandchars=\\\{\}]
\PYG{n}{bau\PYGZus{}2100} \PYG{o}{=}\PYG{n}{baseline}\PYG{o}{.}\PYG{n}{loc}\PYG{p}{[}\PYG{l+m+mi}{2100}\PYG{p}{,}\PYG{l+s+s1}{\PYGZsq{}}\PYG{l+s+s1}{PAKCCEMISCO2TKN}\PYG{l+s+s1}{\PYGZsq{}}\PYG{p}{]}
\PYG{n}{bau\PYGZus{}2022} \PYG{o}{=} \PYG{n}{baseline}\PYG{o}{.}\PYG{n}{loc}\PYG{p}{[}\PYG{l+m+mi}{2022}\PYG{p}{,}\PYG{l+s+s1}{\PYGZsq{}}\PYG{l+s+s1}{PAKCCEMISCO2TKN}\PYG{l+s+s1}{\PYGZsq{}}\PYG{p}{]}
\PYG{n}{bau\PYGZus{}growth\PYGZus{}rate} \PYG{o}{=} \PYG{p}{(}\PYG{n}{bau\PYGZus{}2100}\PYG{o}{/}\PYG{n}{bau\PYGZus{}2022}\PYG{p}{)}\PYG{o}{*}\PYG{o}{*}\PYG{p}{(}\PYG{l+m+mi}{1}\PYG{o}{/}\PYG{p}{(}\PYG{l+m+mi}{2100}\PYG{o}{\PYGZhy{}}\PYG{l+m+mi}{2022}\PYG{p}{)}\PYG{p}{)}

\PYG{n}{target\PYGZus{}2100} \PYG{o}{=} \PYG{n}{baseline}\PYG{o}{.}\PYG{n}{loc}\PYG{p}{[}\PYG{l+m+mi}{2100}\PYG{p}{,}\PYG{l+s+s1}{\PYGZsq{}}\PYG{l+s+s1}{PAKCCEMISCO2TKN}\PYG{l+s+s1}{\PYGZsq{}}\PYG{p}{]}\PYG{o}{*}\PYG{p}{(}\PYG{l+m+mi}{1}\PYG{o}{\PYGZhy{}}\PYG{n}{reduction\PYGZus{}percent}\PYG{o}{/}\PYG{l+m+mi}{100}\PYG{p}{)}
\PYG{n}{target\PYGZus{}growth\PYGZus{}rate} \PYG{o}{=} \PYG{p}{(}\PYG{n}{target\PYGZus{}2100}\PYG{o}{/}\PYG{n}{bau\PYGZus{}2022}\PYG{p}{)}\PYG{o}{*}\PYG{o}{*}\PYG{p}{(}\PYG{l+m+mi}{1}\PYG{o}{/}\PYG{p}{(}\PYG{l+m+mi}{2100}\PYG{o}{\PYGZhy{}}\PYG{l+m+mi}{2022}\PYG{p}{)}\PYG{p}{)}

\PYG{n+nb}{print}\PYG{p}{(}\PYG{l+s+sa}{f}\PYG{l+s+s2}{\PYGZdq{}}\PYG{l+s+s2}{Business as usual Emission value in 2100: }\PYG{l+s+si}{\PYGZob{}}\PYG{n}{bau\PYGZus{}2100}\PYG{l+s+si}{:}\PYG{l+s+s2}{13,.0f}\PYG{l+s+si}{\PYGZcb{}}\PYG{l+s+s2}{\PYGZdq{}}\PYG{p}{)}
\PYG{n+nb}{print}\PYG{p}{(}\PYG{l+s+sa}{f}\PYG{l+s+s2}{\PYGZdq{}}\PYG{l+s+s2}{Business as usual Emission value in 2100: }\PYG{l+s+si}{\PYGZob{}}\PYG{n}{target\PYGZus{}2100}\PYG{l+s+si}{:}\PYG{l+s+s2}{13,.0f}\PYG{l+s+si}{\PYGZcb{}}\PYG{l+s+s2}{\PYGZdq{}}\PYG{p}{)}
\PYG{n+nb}{print}\PYG{p}{(}\PYG{l+s+sa}{f}\PYG{l+s+s2}{\PYGZdq{}}\PYG{l+s+s2}{Business as usual growth rate in percent: }\PYG{l+s+si}{\PYGZob{}}\PYG{n}{bau\PYGZus{}growth\PYGZus{}rate}\PYG{o}{\PYGZhy{}}\PYG{l+m+mi}{1}\PYG{l+s+si}{:}\PYG{l+s+s2}{13,.1\PYGZpc{}}\PYG{l+s+si}{\PYGZcb{}}\PYG{l+s+s2}{\PYGZdq{}}\PYG{p}{)}
\PYG{n+nb}{print}\PYG{p}{(}\PYG{l+s+sa}{f}\PYG{l+s+s2}{\PYGZdq{}}\PYG{l+s+s2}{Target growth rate in percent           : }\PYG{l+s+si}{\PYGZob{}}\PYG{n}{target\PYGZus{}growth\PYGZus{}rate}\PYG{o}{\PYGZhy{}}\PYG{l+m+mi}{1}\PYG{l+s+si}{:}\PYG{l+s+s2}{13,.1\PYGZpc{}}\PYG{l+s+si}{\PYGZcb{}}\PYG{l+s+s2}{\PYGZdq{}}\PYG{p}{)}
\end{sphinxVerbatim}

\end{sphinxuseclass}\end{sphinxVerbatimInput}
\begin{sphinxVerbatimOutput}

\begin{sphinxuseclass}{cell_output}
\begin{sphinxVerbatim}[commandchars=\\\{\}]
Business as usual Emission value in 2100: 1,949,241,066
Business as usual Emission value in 2100: 1,364,468,747
Business as usual growth rate in percent:          2.8\PYGZpc{}
Target growth rate in percent           :          2.4\PYGZpc{}
\end{sphinxVerbatim}

\end{sphinxuseclass}\end{sphinxVerbatimOutput}

\end{sphinxuseclass}

\section{Create a dataframe with the target}
\label{\detokenize{content/howto/targetinstruments/One target one instrument with 3 instrument variables:create-a-dataframe-with-the-target}}
\begin{sphinxuseclass}{cell}\begin{sphinxVerbatimInput}

\begin{sphinxuseclass}{cell_input}
\begin{sphinxVerbatim}[commandchars=\\\{\}]
\PYG{n}{target\PYGZus{}before} \PYG{o}{=} \PYG{n}{baseline}\PYG{o}{.}\PYG{n}{loc}\PYG{p}{[}\PYG{l+m+mi}{2022}\PYG{p}{:}\PYG{p}{,}\PYG{p}{[}\PYG{l+s+s1}{\PYGZsq{}}\PYG{l+s+s1}{PAKCCEMISCO2TKN}\PYG{l+s+s1}{\PYGZsq{}}\PYG{p}{]}\PYG{p}{]}     \PYG{c+c1}{\PYGZsh{} Create dataframe with only the target variable }
\PYG{c+c1}{\PYGZsh{} create a target dataframe with a projection of the target variable }
\PYG{n}{target} \PYG{o}{=} \PYG{n}{target\PYGZus{}before}\PYG{o}{.}\PYG{n}{mfcalc}\PYG{p}{(}\PYG{l+s+sa}{f}\PYG{l+s+s1}{\PYGZsq{}}\PYG{l+s+s1}{PAKCCEMISCO2TKN = PAKCCEMISCO2TKN(\PYGZhy{}1) * }\PYG{l+s+si}{\PYGZob{}}\PYG{n}{target\PYGZus{}growth\PYGZus{}rate}\PYG{l+s+si}{\PYGZcb{}}\PYG{l+s+s1}{\PYGZsq{}}\PYG{p}{,}\PYG{l+m+mi}{2023}\PYG{p}{,}\PYG{l+m+mi}{2100} \PYG{p}{)}
\end{sphinxVerbatim}

\end{sphinxuseclass}\end{sphinxVerbatimInput}
\begin{sphinxVerbatimOutput}

\begin{sphinxuseclass}{cell_output}
\begin{sphinxVerbatim}[commandchars=\\\{\}]
* Take care. Lags or leads in the equations, mfcalc run for 2023 to 2024
\end{sphinxVerbatim}

\end{sphinxuseclass}\end{sphinxVerbatimOutput}

\end{sphinxuseclass}

\section{Now solve the problem:}
\label{\detokenize{content/howto/targetinstruments/One target one instrument with 3 instrument variables:now-solve-the-problem}}
\begin{sphinxuseclass}{cell}\begin{sphinxVerbatimInput}

\begin{sphinxuseclass}{cell_input}
\begin{sphinxVerbatim}[commandchars=\\\{\}]
\PYG{n}{\PYGZus{}} \PYG{o}{=} \PYG{n}{mpak}\PYG{o}{.}\PYG{n}{invert}\PYG{p}{(}\PYG{n}{mpak}\PYG{o}{.}\PYG{n}{basedf}\PYG{p}{,}                  \PYG{c+c1}{\PYGZsh{} Invert calls the target instrument device                   }
                \PYG{n}{targets} \PYG{o}{=} \PYG{n}{target}\PYG{p}{,}                   
                \PYG{n}{instruments}\PYG{o}{=}\PYG{n}{instruments}\PYG{p}{,}
                \PYG{n}{DefaultImpuls}\PYG{o}{=}\PYG{l+m+mi}{2}\PYG{p}{,}              \PYG{c+c1}{\PYGZsh{} The default impulse instrument variables }
                \PYG{n}{defaultconv}\PYG{o}{=}\PYG{l+m+mf}{2.0}\PYG{p}{,}              \PYG{c+c1}{\PYGZsh{} Convergergence criteria for targets}
                \PYG{n}{varimpulse}\PYG{o}{=}\PYG{k+kc}{True}\PYG{p}{,}              \PYG{c+c1}{\PYGZsh{} Only change the variable for current period}
                \PYG{n}{nonlin}\PYG{o}{=}\PYG{l+m+mi}{15}\PYG{p}{,}                    \PYG{c+c1}{\PYGZsh{} If no convergence in 15 iteration recalculate jacobi }
                \PYG{n}{silent}\PYG{o}{=}\PYG{l+m+mi}{1}                      \PYG{c+c1}{\PYGZsh{} Don\PYGZsq{}t show iteration output (try 1 for showing)}
                \PYG{p}{)}
\end{sphinxVerbatim}

\end{sphinxuseclass}\end{sphinxVerbatimInput}
\begin{sphinxVerbatimOutput}

\begin{sphinxuseclass}{cell_output}
\begin{sphinxVerbatim}[commandchars=\\\{\}]
Finding instruments :   0\PYGZpc{}|          | 0/79
\end{sphinxVerbatim}

\end{sphinxuseclass}\end{sphinxVerbatimOutput}

\end{sphinxuseclass}

\section{Make a graph and decorate with a line and an annotation}
\label{\detokenize{content/howto/targetinstruments/One target one instrument with 3 instrument variables:make-a-graph-and-decorate-with-a-line-and-an-annotation}}
\sphinxAtStartPar
Also show the tax rate

\begin{sphinxuseclass}{cell}\begin{sphinxVerbatimInput}

\begin{sphinxuseclass}{cell_input}
\begin{sphinxVerbatim}[commandchars=\\\{\}]
\PYG{k}{with} \PYG{n}{mpak}\PYG{o}{.}\PYG{n}{set\PYGZus{}smpl}\PYG{p}{(}\PYG{l+m+mi}{2020}\PYG{p}{,}\PYG{l+m+mi}{2100}\PYG{p}{)}\PYG{p}{:}    \PYG{c+c1}{\PYGZsh{} change if you want another  timeframe }
    \PYG{n}{fig} \PYG{o}{=} \PYG{n}{mpak}\PYG{p}{[}\PYG{l+s+sa}{f}\PYG{l+s+s1}{\PYGZsq{}}\PYG{l+s+s1}{PAKCCEMISCO2TKN}\PYG{l+s+s1}{\PYGZsq{}} \PYG{p}{]}\PYG{o}{.}\PYG{n}{plot\PYGZus{}alt}\PYG{p}{(}\PYG{n}{title}\PYG{o}{=}\PYG{l+s+s1}{\PYGZsq{}}\PYG{l+s+s1}{Pakistan}\PYG{l+s+s1}{\PYGZsq{}}\PYG{p}{)}
    \PYG{n}{fig}\PYG{o}{.}\PYG{n}{axes}\PYG{p}{[}\PYG{l+m+mi}{0}\PYG{p}{]}\PYG{o}{.}\PYG{n}{axhline}\PYG{p}{(} \PYG{n}{target\PYGZus{}2100}\PYG{p}{,}
                                  \PYG{n}{xmin}\PYG{o}{=}\PYG{l+m+mf}{0.6}\PYG{p}{,}
                                  \PYG{n}{xmax} \PYG{o}{=} \PYG{l+m+mf}{0.99}\PYG{p}{,}
                                  \PYG{n}{linewidth}\PYG{o}{=}\PYG{l+m+mi}{3}\PYG{p}{,} 
                                  \PYG{n}{color}\PYG{o}{=}\PYG{l+s+s1}{\PYGZsq{}}\PYG{l+s+s1}{r}\PYG{l+s+s1}{\PYGZsq{}}\PYG{p}{,} \PYG{n}{ls}\PYG{o}{=}\PYG{l+s+s1}{\PYGZsq{}}\PYG{l+s+s1}{dashed}\PYG{l+s+s1}{\PYGZsq{}}\PYG{p}{)}

    \PYG{n}{fig}\PYG{o}{.}\PYG{n}{axes}\PYG{p}{[}\PYG{l+m+mi}{0}\PYG{p}{]}\PYG{o}{.}\PYG{n}{annotate}\PYG{p}{(}\PYG{l+s+sa}{f}\PYG{l+s+s1}{\PYGZsq{}}\PYG{l+s+s1}{BAU 2050 reduced by }\PYG{l+s+si}{\PYGZob{}}\PYG{n}{reduction\PYGZus{}percent}\PYG{l+s+si}{\PYGZcb{}}\PYG{l+s+s1}{\PYGZpc{}}\PYG{l+s+s1}{\PYGZsq{}}\PYG{p}{,} \PYG{n}{xy}\PYG{o}{=}\PYG{p}{(}\PYG{l+m+mi}{2050}\PYG{p}{,}\PYG{n}{target\PYGZus{}2100} \PYG{p}{)}\PYG{p}{)}
    \PYG{n}{fig2} \PYG{o}{=} \PYG{n}{mpak}\PYG{p}{[}\PYG{l+s+sa}{f}\PYG{l+s+s1}{\PYGZsq{}}\PYG{l+s+s1}{PAKGGREVCO2CER}\PYG{l+s+s1}{\PYGZsq{}} \PYG{p}{]}\PYG{o}{.}\PYG{n}{plot\PYGZus{}alt}\PYG{p}{(}\PYG{n}{title}\PYG{o}{=}\PYG{l+s+sa}{f}\PYG{l+s+s1}{\PYGZsq{}}\PYG{l+s+s1}{Pakistan}\PYG{l+s+s1}{\PYGZsq{}}\PYG{p}{)}\PYG{p}{;} 
\end{sphinxVerbatim}

\end{sphinxuseclass}\end{sphinxVerbatimInput}
\begin{sphinxVerbatimOutput}

\begin{sphinxuseclass}{cell_output}
\noindent\sphinxincludegraphics{{19152bec338121b31bb559a4a2bbfa62875833cae141352e74b98cd6728ec57d}.png}

\noindent\sphinxincludegraphics{{bcaf02fe183eef5a0af2e104f6ef19d0cc71837cfa2277ae520fc44b273a0507}.png}

\end{sphinxuseclass}\end{sphinxVerbatimOutput}

\end{sphinxuseclass}

\subsection{Look at selected variables with the {[}{]} operator}
\label{\detokenize{content/howto/targetinstruments/One target one instrument with 3 instrument variables:look-at-selected-variables-with-the-operator}}
\sphinxAtStartPar
If you want to look at multiple variables the index {[}{]} operator can be used to select the variables to analyze/visualize. Here only a few operations will be shown. There is more {\hyperref[\detokenize{content/notebooks/modelflow_features:index-operator}]{\sphinxcrossref{\DUrole{std,std-ref}{here}}}}

\begin{sphinxuseclass}{cell}\begin{sphinxVerbatimInput}

\begin{sphinxuseclass}{cell_input}
\begin{sphinxVerbatim}[commandchars=\\\{\}]
\PYG{n}{mpak}\PYG{p}{[}\PYG{l+s+s1}{\PYGZsq{}}\PYG{l+s+s1}{PAKNYGDPMKTPKN PAKNECONGOVTKN PAKNEGDIFTOTKN PAKNEIMPGNFSKN PAKCCEMISCO2TKN}\PYG{l+s+s1}{\PYGZsq{}}\PYG{p}{]}
\end{sphinxVerbatim}

\end{sphinxuseclass}\end{sphinxVerbatimInput}
\begin{sphinxVerbatimOutput}

\begin{sphinxuseclass}{cell_output}
\begin{sphinxVerbatim}[commandchars=\\\{\}]
Tab(children=(Tab(children=(HTML(value=\PYGZsq{}\PYGZlt{}?xml version=\PYGZdq{}1.0\PYGZdq{} encoding=\PYGZdq{}utf\PYGZhy{}8\PYGZdq{} standalone=\PYGZdq{}no\PYGZdq{}?\PYGZgt{}\PYGZbs{}n\PYGZlt{}!DOCTYPE svg …
\end{sphinxVerbatim}

\begin{sphinxVerbatim}[commandchars=\\\{\}]

\end{sphinxVerbatim}

\end{sphinxuseclass}\end{sphinxVerbatimOutput}

\end{sphinxuseclass}

\section{Now define instruments so they don’t get the same shock.}
\label{\detokenize{content/howto/targetinstruments/One target one instrument with 3 instrument variables:now-define-instruments-so-they-don-t-get-the-same-shock}}
\sphinxAtStartPar
Here the coal emission gets twice the shock as the two other.

\begin{sphinxuseclass}{cell}\begin{sphinxVerbatimInput}

\begin{sphinxuseclass}{cell_input}
\begin{sphinxVerbatim}[commandchars=\\\{\}]
\PYG{n}{new\PYGZus{}instruments} \PYG{o}{=}\PYG{p}{[}\PYG{p}{[}\PYG{p}{(}\PYG{l+s+s1}{\PYGZsq{}}\PYG{l+s+s1}{PAKGGREVCO2CER}\PYG{l+s+s1}{\PYGZsq{}}\PYG{p}{,}\PYG{l+m+mi}{10}\PYG{p}{)}\PYG{p}{,}
                   \PYG{p}{(}\PYG{l+s+s1}{\PYGZsq{}}\PYG{l+s+s1}{PAKGGREVCO2GER}\PYG{l+s+s1}{\PYGZsq{}}\PYG{p}{,} \PYG{l+m+mi}{5}\PYG{p}{)}\PYG{p}{,}
                   \PYG{p}{(}\PYG{l+s+s1}{\PYGZsq{}}\PYG{l+s+s1}{PAKGGREVCO2OER}\PYG{l+s+s1}{\PYGZsq{}}\PYG{p}{,}\PYG{l+m+mi}{5}\PYG{p}{)}\PYG{p}{]}\PYG{p}{]}


\PYG{n}{\PYGZus{}} \PYG{o}{=} \PYG{n}{mpak}\PYG{o}{.}\PYG{n}{invert}\PYG{p}{(}\PYG{n}{mpak}\PYG{o}{.}\PYG{n}{basedf}\PYG{p}{,}\PYG{n}{targets} \PYG{o}{=} \PYG{n}{target}\PYG{p}{,}
                            \PYG{n}{instruments}\PYG{o}{=}\PYG{n}{new\PYGZus{}instruments}\PYG{p}{,}
                          \PYG{n}{DefaultImpuls}\PYG{o}{=}\PYG{l+m+mi}{1}\PYG{p}{,}
                                \PYG{n}{defaultconv}\PYG{o}{=}\PYG{l+m+mf}{2.0}\PYG{p}{,}\PYG{n}{varimpulse}\PYG{o}{=}\PYG{k+kc}{True}\PYG{p}{,}\PYG{n}{nonlin}\PYG{o}{=}\PYG{l+m+mi}{15}\PYG{p}{,}\PYG{n}{silent}\PYG{o}{=}\PYG{l+m+mi}{1}\PYG{p}{)}
\end{sphinxVerbatim}

\end{sphinxuseclass}\end{sphinxVerbatimInput}
\begin{sphinxVerbatimOutput}

\begin{sphinxuseclass}{cell_output}
\begin{sphinxVerbatim}[commandchars=\\\{\}]
Finding instruments :   0\PYGZpc{}|          | 0/79
\end{sphinxVerbatim}

\end{sphinxuseclass}\end{sphinxVerbatimOutput}

\end{sphinxuseclass}
\begin{sphinxuseclass}{cell}\begin{sphinxVerbatimInput}

\begin{sphinxuseclass}{cell_input}
\begin{sphinxVerbatim}[commandchars=\\\{\}]
\PYG{k}{with} \PYG{n}{mpak}\PYG{o}{.}\PYG{n}{set\PYGZus{}smpl}\PYG{p}{(}\PYG{l+m+mi}{2020}\PYG{p}{,}\PYG{l+m+mi}{2100}\PYG{p}{)}\PYG{p}{:}    \PYG{c+c1}{\PYGZsh{} change if you want another  timeframe }
    \PYG{n}{fig} \PYG{o}{=} \PYG{n}{mpak}\PYG{p}{[}\PYG{l+s+sa}{f}\PYG{l+s+s1}{\PYGZsq{}}\PYG{l+s+s1}{PAKCCEMISCO2TKN}\PYG{l+s+s1}{\PYGZsq{}} \PYG{p}{]}\PYG{o}{.}\PYG{n}{plot\PYGZus{}alt}\PYG{p}{(}\PYG{n}{title}\PYG{o}{=}\PYG{l+s+s1}{\PYGZsq{}}\PYG{l+s+s1}{Pakistan}\PYG{l+s+s1}{\PYGZsq{}}\PYG{p}{)}
    \PYG{n}{fig}\PYG{o}{.}\PYG{n}{axes}\PYG{p}{[}\PYG{l+m+mi}{0}\PYG{p}{]}\PYG{o}{.}\PYG{n}{axhline}\PYG{p}{(} \PYG{n}{target\PYGZus{}2100}\PYG{p}{,}
                                  \PYG{n}{xmin}\PYG{o}{=}\PYG{l+m+mf}{0.6}\PYG{p}{,}
                                  \PYG{n}{xmax} \PYG{o}{=} \PYG{l+m+mf}{0.99}\PYG{p}{,}
                                  \PYG{n}{linewidth}\PYG{o}{=}\PYG{l+m+mi}{3}\PYG{p}{,} 
                                  \PYG{n}{color}\PYG{o}{=}\PYG{l+s+s1}{\PYGZsq{}}\PYG{l+s+s1}{r}\PYG{l+s+s1}{\PYGZsq{}}\PYG{p}{,} \PYG{n}{ls}\PYG{o}{=}\PYG{l+s+s1}{\PYGZsq{}}\PYG{l+s+s1}{dashed}\PYG{l+s+s1}{\PYGZsq{}}\PYG{p}{)}

    \PYG{n}{fig}\PYG{o}{.}\PYG{n}{axes}\PYG{p}{[}\PYG{l+m+mi}{0}\PYG{p}{]}\PYG{o}{.}\PYG{n}{annotate}\PYG{p}{(}\PYG{l+s+sa}{f}\PYG{l+s+s1}{\PYGZsq{}}\PYG{l+s+s1}{BAU 2050 reduced by }\PYG{l+s+si}{\PYGZob{}}\PYG{n}{reduction\PYGZus{}percent}\PYG{l+s+si}{\PYGZcb{}}\PYG{l+s+s1}{\PYGZpc{}}\PYG{l+s+s1}{\PYGZsq{}}\PYG{p}{,} \PYG{n}{xy}\PYG{o}{=}\PYG{p}{(}\PYG{l+m+mi}{2050}\PYG{p}{,}\PYG{n}{target\PYGZus{}2100}\PYG{o}{*}\PYG{l+m+mf}{1.015} \PYG{p}{)}\PYG{p}{)}
\end{sphinxVerbatim}

\end{sphinxuseclass}\end{sphinxVerbatimInput}
\begin{sphinxVerbatimOutput}

\begin{sphinxuseclass}{cell_output}
\noindent\sphinxincludegraphics{{9303ce3c3e2125eaefdec7c2c48a225447cd6af2eabbfe5ef949258a3ae4e886}.png}

\end{sphinxuseclass}\end{sphinxVerbatimOutput}

\end{sphinxuseclass}
\begin{sphinxuseclass}{cell}\begin{sphinxVerbatimInput}

\begin{sphinxuseclass}{cell_input}
\begin{sphinxVerbatim}[commandchars=\\\{\}]
    \PYG{n}{mpak}\PYG{p}{[}\PYG{l+s+s1}{\PYGZsq{}}\PYG{l+s+s1}{PAKGGREVCO2CER PAKGGREVCO2GER PAKGGREVCO2OER}\PYG{l+s+s1}{\PYGZsq{}} \PYG{p}{]}
\end{sphinxVerbatim}

\end{sphinxuseclass}\end{sphinxVerbatimInput}
\begin{sphinxVerbatimOutput}

\begin{sphinxuseclass}{cell_output}
\begin{sphinxVerbatim}[commandchars=\\\{\}]
Tab(children=(Tab(children=(HTML(value=\PYGZsq{}\PYGZlt{}?xml version=\PYGZdq{}1.0\PYGZdq{} encoding=\PYGZdq{}utf\PYGZhy{}8\PYGZdq{} standalone=\PYGZdq{}no\PYGZdq{}?\PYGZgt{}\PYGZbs{}n\PYGZlt{}!DOCTYPE svg …
\end{sphinxVerbatim}

\begin{sphinxVerbatim}[commandchars=\\\{\}]

\end{sphinxVerbatim}

\end{sphinxuseclass}\end{sphinxVerbatimOutput}

\end{sphinxuseclass}
\begin{sphinxuseclass}{cell}\begin{sphinxVerbatimInput}

\begin{sphinxuseclass}{cell_input}
\begin{sphinxVerbatim}[commandchars=\\\{\}]
\PYG{n}{help}\PYG{p}{(}\PYG{n}{mpak}\PYG{o}{.}\PYG{n}{invert}\PYG{p}{)}
\end{sphinxVerbatim}

\end{sphinxuseclass}\end{sphinxVerbatimInput}
\begin{sphinxVerbatimOutput}

\begin{sphinxuseclass}{cell_output}
\begin{sphinxVerbatim}[commandchars=\\\{\}]
Help on method invert in module modelclass:

invert(databank, targets, instruments, silent=1, DefaultImpuls=0.01, defaultconv=0.001, nonlin=False, maxiter=30, delay=0, **kwargs) method of modelclass.model instance
    Solves instruments for targets
\end{sphinxVerbatim}

\end{sphinxuseclass}\end{sphinxVerbatimOutput}

\end{sphinxuseclass}
\begin{sphinxuseclass}{cell}\begin{sphinxVerbatimInput}

\begin{sphinxuseclass}{cell_input}
\begin{sphinxVerbatim}[commandchars=\\\{\}]
\PYG{k+kn}{import} \PYG{n+nn}{modelinvert}
\PYG{n}{help}\PYG{p}{(}\PYG{n}{modelinvert}\PYG{o}{.}\PYG{n}{targets\PYGZus{}instruments}\PYG{p}{)}
\end{sphinxVerbatim}

\end{sphinxuseclass}\end{sphinxVerbatimInput}
\begin{sphinxVerbatimOutput}

\begin{sphinxuseclass}{cell_output}
\begin{sphinxVerbatim}[commandchars=\\\{\}]
Help on class targets\PYGZus{}instruments in module modelinvert:

class targets\PYGZus{}instruments(builtins.object)
 |  targets\PYGZus{}instruments(databank, targets, instruments, model, DefaultImpuls=0.01, defaultconv=0.01, delay=0, nonlin=False, silent=True, maxiter=30, solveopt=\PYGZob{}\PYGZcb{}, varimpulse=False)
 |  
 |  Class to handle general target/instrument problems. 
 |  Where the response is delayed specify this with delay.
 |  
 |  Number of targets should be equal to number of instruments 
 |  
 |  An instrument can comprice of severeral variables
 |  
 |  **Instruments** are inputtet as a list of instruments
 |  
 |  To calculate the jacobian each instrument variable has a impuls, 
 |  which is used as delta when evaluating the jacobi matrix:: 
 |      
 |    [ \PYGZsq{}QO\PYGZus{}J\PYGZsq{},\PYGZsq{}TG\PYGZsq{}]   Simple list each variable are shocked by the default impulse 
 |    [ (\PYGZsq{}QO\PYGZus{}J\PYGZsq{},0.5), \PYGZsq{}TG\PYGZsq{}]  Here QO\PYGZus{}J is getting its own impuls (0.5)
 |    [ [(\PYGZsq{}QO\PYGZus{}J\PYGZsq{},0.5),(\PYGZsq{}ORLOV\PYGZsq{},1.)] , (\PYGZsq{}TG\PYGZsq{},0.01)] here an impuls is given for each variable, and the first instrument consiste of two variables 
 |  
 |  **Targets** are list of variables
 |  
 |  Convergence is achieved when all targets are within convergens distance from the target value
 |  
 |  Convergencedistance can be set individual for a target variable by setting a value in \PYGZlt{}modelinstance\PYGZgt{}.targetconv 
 |  
 |  Targets and target values are provided by a dataframe.
 |  
 |  Methods defined here:
 |  
 |  \PYGZus{}\PYGZus{}call\PYGZus{}\PYGZus{}(self, *args, **kwargs)
 |      Uses :any:`targetseek`
 |  
 |  \PYGZus{}\PYGZus{}init\PYGZus{}\PYGZus{}(self, databank, targets, instruments, model, DefaultImpuls=0.01, defaultconv=0.01, delay=0, nonlin=False, silent=True, maxiter=30, solveopt=\PYGZob{}\PYGZcb{}, varimpulse=False)
 |      Initialize self.  See help(type(self)) for accurate signature.
 |  
 |  invjacobi(self, per, diag=False, delay=0)
 |      Calculates the inverted jacobi matrix
 |      
 |      returns a dataframe
 |  
 |  jacobi(self, per, delay=None)
 |      Calculates a jecobi matrix of derivatives based on the instruments and targets 
 |      
 |      returns a dataframe
 |  
 |  targetseek(self, databank=None, shortfall=False, ti\PYGZus{}damp=1.0, delay=0, progressbar=True, **kwargs)
 |      Calculates the instruments as a function of targets
 |  
 |  \PYGZhy{}\PYGZhy{}\PYGZhy{}\PYGZhy{}\PYGZhy{}\PYGZhy{}\PYGZhy{}\PYGZhy{}\PYGZhy{}\PYGZhy{}\PYGZhy{}\PYGZhy{}\PYGZhy{}\PYGZhy{}\PYGZhy{}\PYGZhy{}\PYGZhy{}\PYGZhy{}\PYGZhy{}\PYGZhy{}\PYGZhy{}\PYGZhy{}\PYGZhy{}\PYGZhy{}\PYGZhy{}\PYGZhy{}\PYGZhy{}\PYGZhy{}\PYGZhy{}\PYGZhy{}\PYGZhy{}\PYGZhy{}\PYGZhy{}\PYGZhy{}\PYGZhy{}\PYGZhy{}\PYGZhy{}\PYGZhy{}\PYGZhy{}\PYGZhy{}\PYGZhy{}\PYGZhy{}\PYGZhy{}\PYGZhy{}\PYGZhy{}\PYGZhy{}\PYGZhy{}\PYGZhy{}\PYGZhy{}\PYGZhy{}\PYGZhy{}\PYGZhy{}\PYGZhy{}\PYGZhy{}\PYGZhy{}\PYGZhy{}\PYGZhy{}\PYGZhy{}\PYGZhy{}\PYGZhy{}\PYGZhy{}\PYGZhy{}\PYGZhy{}\PYGZhy{}\PYGZhy{}\PYGZhy{}\PYGZhy{}\PYGZhy{}\PYGZhy{}\PYGZhy{}
 |  Data descriptors defined here:
 |  
 |  \PYGZus{}\PYGZus{}dict\PYGZus{}\PYGZus{}
 |      dictionary for instance variables (if defined)
 |  
 |  \PYGZus{}\PYGZus{}weakref\PYGZus{}\PYGZus{}
 |      list of weak references to the object (if defined)
\end{sphinxVerbatim}

\end{sphinxuseclass}\end{sphinxVerbatimOutput}

\end{sphinxuseclass}
\sphinxstepscope


\part{References}

\sphinxstepscope


\chapter{References}
\label{\detokenize{content/99_BackMatter/References:references}}\label{\detokenize{content/99_BackMatter/References::doc}}
\begin{sphinxadmonition}{warning}{Warning:}
\sphinxAtStartPar
IB: heading to awoid errormessage in jupyterbook
\end{sphinxadmonition}

\begin{sphinxthebibliography}{1}
\bibitem[1]{content/99_BackMatter/References:id20}
\sphinxAtStartPar
Doug Addison. \sphinxstyleemphasis{The World Bank revised minimum standard model (RMSM) : concepts and issues}. Number WPS231 in Policy Research Working Papers. World Bank, Washington DC., 1989. URL: \sphinxurl{https://documents.worldbank.org/en/publication/documents-reports/documentdetail/997721468765042532/the-world-bank-revised-minimum-standard-model-rmsm-concepts-and-issues}.
\bibitem[2]{content/99_BackMatter/References:id2}
\sphinxAtStartPar
Ron Berndsen. Causal ordering in economic models. \sphinxstyleemphasis{Decision Support Systems}, 1995. ISBN: 0167\sphinxhyphen{}9236. \sphinxhref{https://doi.org/10.1016/0167-9236(94)00034-P}{doi:10.1016/0167\sphinxhyphen{}9236(94)00034\sphinxhyphen{}P}.
\bibitem[3]{content/99_BackMatter/References:id17}
\sphinxAtStartPar
Olivier Blanchard. On the future of Macroeconomic models. \sphinxstyleemphasis{Oxford Review of Economic Policy}, 34(1\sphinxhyphen{}2):43–54, 2018. URL: \sphinxurl{https://academic.oup.com/oxrep/article/34/1-2/43/4781808}, \sphinxhref{https://doi.org/https://doi.org/10.1093/oxrep/grx045}{doi:https://doi.org/10.1093/oxrep/grx045}.
\bibitem[4]{content/99_BackMatter/References:id15}
\sphinxAtStartPar
Andrew Burns, Benoit Campagne, Charl Jooste, David Stephan, and Thi Thanh Bui. \sphinxstyleemphasis{The World Bank Macro\sphinxhyphen{}Fiscal Model Technical Description}. Number 8965 in Policy Research Working Papers. World Bank, Washington DC., 2019. URL: \sphinxurl{https://openknowledge.worldbank.org/handle/10986/32217}.
\bibitem[5]{content/99_BackMatter/References:id14}
\sphinxAtStartPar
Andrew Burns, Charl Jooste, and Gregor Schwerhoff. \sphinxstyleemphasis{Climate Modeling for Macroeconomic Policy : A Case Study for Pakistan}. Number 9780 in Policy Research Working Papers. World Bank, Washington, DC, 2021. URL: \sphinxurl{https://openknowledge.worldbank.org/bitstream/handle/10986/36307/Climate-Modeling-for-Macroeconomic-Policy-A-Case-Study-for-Pakistan.pdf?sequence=1\&isAllowed=y}.
\bibitem[6]{content/99_BackMatter/References:id18}
\sphinxAtStartPar
Andrew Burns, Charl Jooste, and Gregor Schwerhoff. \sphinxstyleemphasis{Macroeconomic Modeling of Managing Hurricane Damage in the Caribbean: The Case of Jamaica}. Volume 9505 of Policy Research Working Paper. World Bank, Washington DC., 2021. URL: \sphinxurl{https://documents1.worldbank.org/curated/en/593351609776234361/pdf/Macroeconomic-Modeling-of-Managing-Hurricane-Damage-in-the-Caribbean-The-Case-of-Jamaica.pdf}.
\bibitem[7]{content/99_BackMatter/References:id21}
\sphinxAtStartPar
Hollis Chenery. \sphinxstyleemphasis{Studies in Development Planning.} Harvard University Press,, Cambridge, MA., 1971.
\bibitem[8]{content/99_BackMatter/References:id4}
\sphinxAtStartPar
K.C. Kogiku. \sphinxstyleemphasis{An Introduction to Macroeconomic Models}. McGwaw\sphinxhyphen{}Hill, 1968. URL: \sphinxurl{https://books.google.de/books?id=jp4LzQEACAAJ}.
\bibitem[9]{content/99_BackMatter/References:id16}
\sphinxAtStartPar
M. R. Wickens and T. S. Breusch. Dynamic Specification, the Long\sphinxhyphen{}Run and the Estimation of Transformed Regression Models. \sphinxstyleemphasis{The Economic Journal}, 98:189–205, April 1988.
\end{sphinxthebibliography}







\renewcommand{\indexname}{Index}
\printindex
\end{document}