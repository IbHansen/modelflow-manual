%% Generated by Sphinx.
\def\sphinxdocclass{jupyterBook}
\documentclass[letterpaper,10pt,english]{jupyterBook}
\ifdefined\pdfpxdimen
   \let\sphinxpxdimen\pdfpxdimen\else\newdimen\sphinxpxdimen
\fi \sphinxpxdimen=.75bp\relax
\ifdefined\pdfimageresolution
    \pdfimageresolution= \numexpr \dimexpr1in\relax/\sphinxpxdimen\relax
\fi
%% let collapsible pdf bookmarks panel have high depth per default
\PassOptionsToPackage{bookmarksdepth=5}{hyperref}
%% turn off hyperref patch of \index as sphinx.xdy xindy module takes care of
%% suitable \hyperpage mark-up, working around hyperref-xindy incompatibility
\PassOptionsToPackage{hyperindex=false}{hyperref}
%% memoir class requires extra handling
\makeatletter\@ifclassloaded{memoir}
{\ifdefined\memhyperindexfalse\memhyperindexfalse\fi}{}\makeatother

\PassOptionsToPackage{warn}{textcomp}

\catcode`^^^^00a0\active\protected\def^^^^00a0{\leavevmode\nobreak\ }
\usepackage{cmap}
\usepackage{fontspec}
\defaultfontfeatures[\rmfamily,\sffamily,\ttfamily]{}
\usepackage{amsmath,amssymb,amstext}
\usepackage{polyglossia}
\setmainlanguage{english}



\setmainfont{FreeSerif}[
  Extension      = .otf,
  UprightFont    = *,
  ItalicFont     = *Italic,
  BoldFont       = *Bold,
  BoldItalicFont = *BoldItalic
]
\setsansfont{FreeSans}[
  Extension      = .otf,
  UprightFont    = *,
  ItalicFont     = *Oblique,
  BoldFont       = *Bold,
  BoldItalicFont = *BoldOblique,
]
\setmonofont{FreeMono}[
  Extension      = .otf,
  UprightFont    = *,
  ItalicFont     = *Oblique,
  BoldFont       = *Bold,
  BoldItalicFont = *BoldOblique,
]



\usepackage[Bjarne]{fncychap}
\usepackage[,numfigreset=1,mathnumfig]{sphinx}

\fvset{fontsize=\small}
\usepackage{geometry}


% Include hyperref last.
\usepackage{hyperref}
% Fix anchor placement for figures with captions.
\usepackage{hypcap}% it must be loaded after hyperref.
% Set up styles of URL: it should be placed after hyperref.
\urlstyle{same}

\addto\captionsenglish{\renewcommand{\contentsname}{The World Bank's MFMod Framework and Modelflow}}

\usepackage{sphinxmessages}



        % Start of preamble defined in sphinx-jupyterbook-latex %
         \usepackage[Latin,Greek]{ucharclasses}
        \usepackage{unicode-math}
        % fixing title of the toc
        \addto\captionsenglish{\renewcommand{\contentsname}{Contents}}
        \hypersetup{
            pdfencoding=auto,
            psdextra
        }
        % End of preamble defined in sphinx-jupyterbook-latex %
        

\title{The World Bank's MFMod Framework in Python with Modelflow}
\date{Apr 15, 2023}
\release{}
\author{Andrew Burns and Ib Hansen}
\newcommand{\sphinxlogo}{\vbox{}}
\renewcommand{\releasename}{}
\makeindex
\begin{document}

\pagestyle{empty}
\sphinxmaketitle
\pagestyle{plain}
\sphinxtableofcontents
\pagestyle{normal}
\phantomsection\label{\detokenize{content/introduction::doc}}


\sphinxAtStartPar
\(\Large{Foreword}\)

\sphinxAtStartPar
Lorem Ipsum
“Neque porro quisquam est qui dolorem ipsum quia dolor sit amet, consectetur, adipisci velit…”
“There is no one who loves pain itself, who seeks after it and wants to have it, simply because it is pain…”

\sphinxAtStartPar
freestar

\sphinxAtStartPar
freestar
Lorem ipsum dolor sit amet, consectetur adipiscing elit. Vestibulum aliquam varius mi. Suspendisse pharetra egestas viverra. Aenean viverra hendrerit sagittis. Curabitur vel lectus at arcu mattis blandit. Quisque aliquet erat nunc, vitae consequat eros venenatis eu. Vivamus ut arcu eget ipsum mollis iaculis. Aliquam rhoncus bibendum orci. Donec lacinia, mauris placerat auctor vehicula, odio eros efficitur leo, et porttitor est urna vitae erat. Cras tempor nec purus at tincidunt. Maecenas viverra massa diam, sit amet tristique mi scelerisque non. Etiam scelerisque, risus ac mollis hendrerit, ex velit vehicula tortor, quis accumsan leo enim sed leo. Suspendisse potenti. Nulla libero diam, eleifend nec sollicitudin ut, varius non eros.

\sphinxAtStartPar
Ut sit amet mollis ipsum. Donec tempor magna ac blandit gravida. Phasellus viverra, arcu at euismod auctor, lectus justo vehicula eros, sit amet posuere felis mi ac purus. Nullam gravida lacinia bibendum. Vivamus ultrices justo sed aliquam feugiat. Mauris vulputate sapien in tempus posuere. Morbi nec purus eget ipsum fermentum congue. Vivamus auctor, mi sit amet lacinia suscipit, ipsum lectus pulvinar risus, non condimentum eros felis sed quam. Pellentesque consectetur leo sit amet condimentum commodo. Orci varius natoque penatibus et magnis dis parturient montes, nascetur ridiculus mus. Duis risus mi, elementum ac leo ut, ultrices scelerisque dui.

\sphinxAtStartPar
Sed quis arcu et dui viverra interdum vitae id enim. In euismod diam quis eleifend viverra. Nullam sodales dictum turpis, vestibulum sodales erat. Morbi quis orci dictum mauris volutpat porttitor at at sapien. Maecenas nec metus ut felis malesuada dapibus. Duis semper lacus eget hendrerit congue. Aenean condimentum, ligula ac sagittis rutrum, turpis elit pulvinar libero, eget tristique sapien sem eget lacus. Curabitur egestas velit quis eros volutpat rhoncus. Nunc quam nibh, commodo ac egestas non, tristique sit amet nisl. Ut vitae lacinia justo.

\sphinxAtStartPar
Indermit Gil
World Bank Chief Economist

\sphinxstepscope


\part{The World Bank's MFMod Framework and Modelflow}

\sphinxstepscope


\chapter{Introduction}
\label{\detokenize{content/01_Introduction/Introduction:introduction}}\label{\detokenize{content/01_Introduction/Introduction::doc}}
\begin{sphinxadmonition}{warning}{Warning:}
\sphinxAtStartPar
This Jupyter Book is work in progress.
\end{sphinxadmonition}

\sphinxAtStartPar
This paper describes the implementation of the World Bank’s MacroFiscalModel (MFMod, see Burns \sphinxstyleemphasis{et al.} {[}\hyperlink{cite.content/99_BackMatter/References:id15}{2019}{]}) using the open source solution program ModelFlow (\sphinxhref{https://ibhansen.github.io/doc/index}{Hansen, 2023}).

\sphinxAtStartPar
The impetus for this paper and the work that it summarizes was to make available to a wider constituency the work that the Bank has done over the past several decades to disseminate Macro\sphinxhyphen{}structural models%
\begin{footnote}[1]\sphinxAtStartFootnote
Economic modelling has a long tradition at the World Bank.  The Bank has had a long\sphinxhyphen{}standing involvement in macroeconomic modelling, initally with linear programming polanning models {[}\hyperlink{cite.content/99_BackMatter/References:id21}{Chenery, 1971}{]}, and then CGE models {[}{]}. Indeed, the popular modelling package GAMS, which is widely used to solve CGE and Linear Programming models, \sphinxhref{https://www.gams.com/about/company/}{started out} as a project begun at the World Bank in the 1976 {[}\hyperlink{cite.content/99_BackMatter/References:id20}{Addison, 1989}{]}.
%
\end{footnote} – notably those that form part of its MFMod (MacroFiscalModel) framework.


\section{The MFMod Framework at the World Bank}
\label{\detokenize{content/01_Introduction/Introduction:the-mfmod-framework-at-the-world-bank}}
\sphinxAtStartPar
MFMod is the World Bank’s work\sphinxhyphen{}horse macro\sphinxhyphen{}structural economic modelling framework. It exists both a linked system of 184 country specific models that can be solved either independently or as a larger system (\sphinxcode{\sphinxupquote{MFMod}}), and as a series of  standalone customized models, known collectively as MFMod Standalones (MFMod SAs) that have been developed from the central model to the fit the specific needs of individual countries. Both modelling systems can be solved using the EViews modelling languiage, or through the intermediation of an easy\sphinxhyphen{}to\sphinxhyphen{}use excel front end developed by the Bank.

\sphinxAtStartPar
The main \sphinxcode{\sphinxupquote{MFMod}} global model evolved from earlier macro\sphinxhyphen{}structural models developed during the 2000s to strengthen the basis for the forecasts produced by the World Bank. Some examples of these models were released on the World Bank’s isimulate platform early in 2010 along with several CGE models dating from this period. These earlier models were substantially extended into what has become the main MFMod (MacroFiscalModel) model during 2014. Since 2015, MFMod replaced the Bank’s RMSIM\sphinxhyphen{}X model ({[}\hyperlink{cite.content/99_BackMatter/References:id20}{Addison, 1989}{]}), as the Bank’s main tool for forecasting and economic analysis, and is used for the World Bank’s twice annual forecasting exercise \sphinxhref{https://www.worldbank.org/en/publication/macro-poverty-outlook}{The Macro Poverty Outlook}.

\sphinxAtStartPar
The main documentation for \sphinxcode{\sphinxupquote{MFMod}} are Burns \sphinxstyleemphasis{et al.} {[}\hyperlink{cite.content/99_BackMatter/References:id15}{2019}{]}.


\subsection{Climate aware version of MFMod}
\label{\detokenize{content/01_Introduction/Introduction:climate-aware-version-of-mfmod}}
\sphinxAtStartPar
Most recently, the Bank has extended the standard MFMod framework to incorporate the main features of climate change {[}\hyperlink{cite.content/99_BackMatter/References:id14}{Burns \sphinxstyleemphasis{et al.}, 2021}{]}– both in terms of the impact of the economy on climate (principally through green\sphinxhyphen{}house gas emissions, like \(CO_2, N_{2}O, CH_4, ...\)) and the impact of the changing climate on the economy (higher temperatures, changes in rainfall quantity and variability, increased incidence of extreme weather) and their impacts on the economy (agricultural output, labor productivity, physical damages due to extreme weather events, sea\sphinxhyphen{}level rises etc.).

\sphinxAtStartPar
Variants on the model initially described in  Burns \sphinxstyleemphasis{et al.} {[}\hyperlink{cite.content/99_BackMatter/References:id14}{2021}{]}, have been developed for {[}xx{]} countries and underpin the economic analysis contained in many of the World Bank’s  \sphinxhref{https://www.worldbank.org/en/publication/country-climate-development-reports}{Country Climate Development Reports}.


\section{Early steps to bring the MFMod system to the broader economics community}
\label{\detokenize{content/01_Introduction/Introduction:early-steps-to-bring-the-mfmod-system-to-the-broader-economics-community}}
\sphinxAtStartPar
Bank staff were quick to recognize that the models built for its own needs could be of use to the broader economics community. An initial project \sphinxcode{\sphinxupquote{isimulate}} made several versions of this earlier model available for simulation on the \sphinxhref{https://isimulate.worldbank.org}{isimulate platform} in 2007, and these models continue to be available there.  The \sphinxcode{\sphinxupquote{isimulate}} platform housed (and continues to house) public access to earlier versions of the MFMod system, and allows simulation of these and other models – but does not give researchers access to the code or the ability to construct complex simulations.

\sphinxAtStartPar
In another effort to make models widely available a large number (more than 60 as of June 2023) customized stand\sphinxhyphen{}alone models (collectively known as called MFModSA \sphinxhyphen{} MacroFiscalModel StandAlones)  have been developed from the main model. Typically developed for a country\sphinxhyphen{}client (Ministry of Finance, Economy or Planning or Central Bank), these Stand Alones extend the standard model by incorporating additional details not in the standard model that are of specific import to different economies and the country\sphinxhyphen{}clients for whom they were built, including: a more detailed breakdown of the sectoral make up of an economy, more detailed fiscal and monetary accounts, and other economically important features of the economy that may exist only inside the aggregates of the standard model.

\sphinxAtStartPar
Training and dissemination around these customized versions of MFMod have been ongoing since 2013. In addition to making customized models available to client governments, Bank teams have run technical assistance program designed to train government officials in the use of these models, their maintenance, modification and revision.


\section{Moving the framework to an open\sphinxhyphen{}source footing}
\label{\detokenize{content/01_Introduction/Introduction:moving-the-framework-to-an-open-source-footing}}
\sphinxAtStartPar
Models in the MFMod family are normally built using the proprietary \DUrole{xref,myst}{EViews} econometric and modelling package. While offering many advantages for model development and maintenance, its cost may be a barrier to clients in developing countries.  As a result, the World Bank joined with Ib Hansen, a Danish economist formerly with the European Central Bank and the Danish Central Bank, who over the years has developed \sphinxcode{\sphinxupquote{modelflow}} a generalized solution engine written in Python for economic models. Together with World Bank, Hansen has worked to extend \sphinxcode{\sphinxupquote{modelflow}} so that MFMod models can be ported and run in the framework.

\sphinxAtStartPar
This paper reports on the results of these efforts. In particular, it provides step by step instructions on how to install the \sphinxcode{\sphinxupquote{modelflow}} framework, import a World Bank macrostructural model,  perform simulations with that model and report results using the many analytical and reporting tools that have been built into \sphinxcode{\sphinxupquote{modelflow}}.  It is not a manual for \sphinxcode{\sphinxupquote{modelflow}}, such a manual can be found \sphinxhref{https://ibhansen.github.io/doc/index}{here} nor is it documentation for the MFMod system, such documentation can be found here {[}\hyperlink{cite.content/99_BackMatter/References:id15}{Burns \sphinxstyleemphasis{et al.}, 2019}{]} and here {[}\hyperlink{cite.content/99_BackMatter/References:id18}{Burns \sphinxstyleemphasis{et al.}, 2021}{]}, {[}\hyperlink{cite.content/99_BackMatter/References:id14}{Burns \sphinxstyleemphasis{et al.}, 2021}{]}). Nor is it documentation for the specific models described and worked with below.


\bigskip\hrule\bigskip


\sphinxstepscope


\section{Macrostructural models}
\label{\detokenize{content/02_MacrostructuralModels/MacroStructuralModels:macrostructural-models}}\label{\detokenize{content/02_MacrostructuralModels/MacroStructuralModels::doc}}
\sphinxAtStartPar
The economics profession uses a wide range of models for different purposes.  Macro\sphinxhyphen{}structural models (also known as semi\sphinxhyphen{}structural or Macro\sphinxhyphen{}econometric models) are a class of models that seek to summarize the most important interconnections and determinants of economic activity in an economy. Computable General Equilibrium (CGE), and Dynamic Stochastic General Equilibrium (DSGE) models are other classes of models that also seek, using somewhat different methodologies, to capture the main economic channels by which the actions of agents (firms, households, governments) interact and help determine the structure, level and rate of growth of economic activity in an economy.

\sphinxAtStartPar
Olivier Blanchard, former Chief Economist at the International Monetary Fund, in a series of articles published between 2016 and 2018 that were summarized in Blanchard {[}\hyperlink{cite.content/99_BackMatter/References:id17}{2018}{]}, lays out his views on the relative strengths and weaknesses of each of these systems, concluding that each has a role to play in helping economists analyze the macro\sphinxhyphen{}economy. Typically, organizations, including the World Bank, use all of these tools, privileging one or the other for specific purposes. Macrostructural models like the MFMod framework are widely used by Central Banks, Ministries of Finance; and professional forecasters both for the purposes of generating forecasts and policy analysis.


\subsection{A system of equations}
\label{\detokenize{content/02_MacrostructuralModels/MacroStructuralModels:a-system-of-equations}}
\sphinxAtStartPar
Mathematically, macro\sphinxhyphen{}structural models are a system of equations comprised of two kinds of equations and three kinds of variables.

\sphinxAtStartPar
\sphinxstylestrong{Types of variables in macro\sphinxhyphen{}structural models}
\begin{itemize}
\item {} 
\sphinxAtStartPar
\sphinxcode{\sphinxupquote{Identities}} are variables that are determined by a well defined accounting rule that always holds. The famous GDP Identity Y=C+I+G+(X\sphinxhyphen{}M) is one such identity, that indicates that GDP at market prices is definitionally equal to Consumption plus Investment plus Government spending plus Exports less Imports.

\item {} 
\sphinxAtStartPar
\sphinxcode{\sphinxupquote{Behavioural}} variables are determined by equations that typically attempt to summarize an economic (vs accounting) relationship. Thus, the equation that says Real Consumption = f(Disposable Income,the price level, and animal spirits) is a behavioural equation – where the relationship is drawn from economic theory. Because these equations do not fully explain the variation in the dependent variable and the sensitivities of variables to the changes in other variables are uncertain, these equations and their parameters are  typically estimated econometrically and are subject to error.

\item {} 
\sphinxAtStartPar
\sphinxcode{\sphinxupquote{Exogenous}} variables are not determined by the model. Typically there are set either by assumption or from data external to the model.  For an individual country model, the exogenous variables would often include the global price of crude oil  because the level of activity of the economy itself is unlikely to affect the world price of oil.

\end{itemize}

\sphinxAtStartPar
In a fully general form it can be written as:
\label{equation:content/02_MacrostructuralModels/MacroStructuralModels:34a990c2-2bc2-4fc2-a4c3-d4691d40f6a9}\begin{align}
y_t^1  &=  f^1(y_{t+u}^1...,y_{t+u}^n...,y_t^2...,y_{t}^n...y_{t-r}^1...,y_{t-r}^n,x_t^1...x_{t}^k,...x_{t-s}^1...,x_{t-s}^k) \\
y_t^2  &=  f^2(y_{t+u}^1...,y_{t+u}^n...,y_t^1...,y_{t}^n...y_{t-r}^1...,y_{t-r}^n,x_t^1...x_{t}^k,...x_{t-s}^1...,x_{t-s}^k) \\
\vdots \\
y_t^n  &=  f^n(y_{t+u}^1...,y_{t+u}^n...,y_t^1...,y_{t}^{n-1}...y_{t-r}^1...,y_{t-r}^n,x_t^1...x_{t}^r,x..._{t-s}^1...,x_{t-s}^k)
\end{align}
\sphinxAtStartPar
where \( y_t^1 \) is one of n endogenous variables and \(x_t^1\) is an exogenous variable and there are as many equations as there are unknown (endogenous variables).

\sphinxAtStartPar
Substituting the variable mnemonics Y,C,I,G,X,M for the simple model the above can be rewritten as as a system of 6 equations in 6 unknowns:
\label{equation:content/02_MacrostructuralModels/MacroStructuralModels:ee99f0d0-3cf1-46f2-b304-6d7a9e9ee0fa}\begin{align}
Y_t  &=  C_t+I_t+G+t+ (X_t-M_t) \\
C_t &= c_t(C_{t-1},C_{t-2},I_t,G_t,X_t,M_t,P_t)\\
I_t &= c_t(I_{t-1},I_{t-2},C_t,G_t,X_t,M_t,P_t)\\
G_t &= c_t(G_{t-1},G_{t-2},C_t,I_t,X_t,M_t,P_t)\\
X_t &= c_t(X_{t-1},X_{t-2},C_t,I_t,G_t,M_t,P_t,P^f_t)\\
M_t &= c_t(M_{t-1},M_{t-2},C_t,I_t,G_t,X_t,P_t,P^f_t)
\end{align}
\sphinxAtStartPar
and where \(P_t, P^f_t\) (domestic and foreign prices, respectively) are exogenous in this simple model.

\sphinxstepscope


\chapter{Modelflow and the MFMod models of the World Bank}
\label{\detokenize{content/02_MacrostructuralModels/MFModAndModelFlow:modelflow-and-the-mfmod-models-of-the-world-bank}}\label{\detokenize{content/02_MacrostructuralModels/MFModAndModelFlow::doc}}
\sphinxAtStartPar
At the World Bank models built using the MFMod framework are developed in \DUrole{xref,myst}{EViews}. When disseminated to clients, the models are operated in a World Bank customized EViews environment. But as a systems of equations and associated data the models can be solved, and operated under any system capable of solving a system of simultaneous equations – as long as the equations and data can be transferred from EViews to the secondary system. \sphinxcode{\sphinxupquote{Modelflow}} is such a system and offers a wide range of features that permit not only solving the model, but also provide a rich and powerful suite of tools for analyzing the model and reporting results.


\section{A brief history of ModelFlow}
\label{\detokenize{content/02_MacrostructuralModels/MFModAndModelFlow:a-brief-history-of-modelflow}}
\sphinxAtStartPar
Modelflow is a python library that was developed by Ib Hansen over several years while working at the Danish Central Bank and the European Central Bank. The framework has been used both to port the U.S. Federal Reserve’s macro\sphinxhyphen{}structural  model to python, but also been used to bring several stress\sphinxhyphen{}testing models developed by European Central Banks and the European Central Bank into a python environment.

\sphinxAtStartPar
Beginning in 2019, Hansen has worked with the World Bank to develop additional features that facilitate working with models built using the Bank’s MFMod Framework, with the objective of creating an open source platform through which the Bank’s models can be made available to the public.

\sphinxAtStartPar
This paper, and the models that accompany it, are the product of this collaboration.

\begin{sphinxShadowBox}
\sphinxstylesidebartitle{}
\end{sphinxShadowBox}

\sphinxstepscope


\section{Installation of Modelflow}
\label{\detokenize{content/03_Installation/InstallingPython:installation-of-modelflow}}\label{\detokenize{content/03_Installation/InstallingPython::doc}}
\sphinxAtStartPar
Modelflow is a python package that defines the \sphinxcode{\sphinxupquote{model}} class, its methods and a number of other functions that extend and combine pre\sphinxhyphen{}existing python functions to allow the easy solution of complex systems of equations including macro\sphinxhyphen{}structural models like MFMod.  To work with \sphinxcode{\sphinxupquote{modelflow}}, a user needs to first install python (preferably the Anaconda variant), several supporting packages, and of course the \sphinxcode{\sphinxupquote{modelflow}} package itself.  While \sphinxcode{\sphinxupquote{modelflow}} can be run directly from the python command\sphinxhyphen{}line or IDEs (Interactive Development Environments) like \sphinxcode{\sphinxupquote{Spyder}} or Microsoft’s \sphinxcode{\sphinxupquote{Visual Code}}, it is suggested that users also install the Jupyter notebook system. Jupyter Notebook facilitates an interactive approach to building python programs, annotating them and ultimately doing simulations using MFMod under \sphinxcode{\sphinxupquote{modelflow}}. This entire manual and the examples in it were all written and executed in the Jupyter Notebook environment.


\subsection{Installation of Python}
\label{\detokenize{content/03_Installation/InstallingPython:installation-of-python}}
\sphinxAtStartPar
\sphinxcode{\sphinxupquote{Python}} is an extremely powerful, versatile and extensible open\sphinxhyphen{}source language. It is widely used for artificial intelligence application, interactive web sites, and scientific processing. As of 14 November 2022, the \sphinxcode{\sphinxupquote{Python Package Index}} (PyPI), the official repository for third\sphinxhyphen{}party Python software, contained over 415,000 packages that extend its functionality %
\begin{footnote}[1]\sphinxAtStartFootnote
\sphinxhref{https://en.wikipedia.org/wiki/Python\_(programming\_language)}{Wikipedia article on python}
%
\end{footnote}. Modelflow is one of these packages.

\sphinxAtStartPar
Python comes in many flavors and \sphinxcode{\sphinxupquote{modelflow}} will work with any of them.  However, \sphinxstylestrong{users are strongly advised to use the Anaconda version of Python}.

\sphinxAtStartPar
The remainder of this section points to instructions on how to install the Anaconda version of python (under Windows, MacOS and under Linux). \sphinxcode{\sphinxupquote{Modelflow}} works equally well under all three. This is followed by section that describes the steps necessary to create an anaconda environment with all the necessary packages to run \sphinxcode{\sphinxupquote{modelflow}}.


\subsubsection{Installation of Anaconda under Windows}
\label{\detokenize{content/03_Installation/InstallingPython:installation-of-anaconda-under-windows}}
\sphinxAtStartPar
The definitive source for installing Anaconda under windows can be found \sphinxhref{https://docs.anaconda.com/anaconda/install/windows/}{here}.

\begin{sphinxadmonition}{warning}{Warning:}
\sphinxAtStartPar
\sphinxstylestrong{It is strongly advised that Anaconda be installed for a single user (Just Me)}  This is much easier to maintain over time.  Installing “For all users on this computer” the other option offered by the \sphinxcode{\sphinxupquote{anaconda}} installer   will substantially increase the complexity of maintaining python on your computer.
\end{sphinxadmonition}


\subsubsection{Installation of Python under macOS}
\label{\detokenize{content/03_Installation/InstallingPython:installation-of-python-under-macos}}
\sphinxAtStartPar
The definitive source for installing Anaconda under macOS can be found \sphinxhref{https://docs.anaconda.com/anaconda/install/mac-os/}{here}.


\subsubsection{Installation of Python under Linux}
\label{\detokenize{content/03_Installation/InstallingPython:installation-of-python-under-linux}}
\sphinxAtStartPar
The definitive source for installing Anaconda under Linux can be found \sphinxhref{https://docs.anaconda.com/anaconda/install/linux/}{here}.


\bigskip\hrule\bigskip


\sphinxstepscope


\section{Installation of Modelflow}
\label{\detokenize{content/03_Installation/InstallingModelFlow:installation-of-modelflow}}\label{\detokenize{content/03_Installation/InstallingModelFlow::doc}}
\sphinxAtStartPar
\sphinxcode{\sphinxupquote{Modelflow}} is a python package that defines the modelflow class \sphinxcode{\sphinxupquote{model}} among others.  \sphinxcode{\sphinxupquote{Modelflow}} has many dependencies. Installing the class the first time can take some time depending on your internet connection and computer speed.  It is essential that you follow all of the steps outlined below to ensure that your version of \sphinxcode{\sphinxupquote{modelflow}} operates as expected.

\begin{sphinxadmonition}{warning}{Warning:}
\sphinxAtStartPar
The following instructions concern the installation of \sphinxcode{\sphinxupquote{modelflow}} within an Anaconda installation of python.  Different flavors of Python may require slight changes to this recipe, but are not covered here.

\sphinxAtStartPar
\sphinxcode{\sphinxupquote{Modelflow}} is built and tested using the anaconda python environment.  It is strongly recommended to use Anaconda with \sphinxcode{\sphinxupquote{modelflow}}.

\sphinxAtStartPar
If you have not already installed Anaconda following the instructions in the preceding section, please do so \sphinxstylestrong{Now}.
\end{sphinxadmonition}


\subsection{Installation of \sphinxstyleliteralintitle{\sphinxupquote{modelflow}} under Anaconda}
\label{\detokenize{content/03_Installation/InstallingModelFlow:installation-of-modelflow-under-anaconda}}\begin{enumerate}
\sphinxsetlistlabels{\arabic}{enumi}{enumii}{}{.}%
\item {} 
\sphinxAtStartPar
Open the anaconda command prompt

\item {} 
\sphinxAtStartPar
Execute the following commands by copying and pasting them – either line by line or as a single mult\sphinxhyphen{}line step

\item {} 
\sphinxAtStartPar
Press enter

\end{enumerate}

\begin{sphinxVerbatim}[commandchars=\\\{\}]
\PYG{n}{conda} \PYG{n}{create} \PYG{o}{\PYGZhy{}}\PYG{n}{n} \PYG{n}{ModelFlow} \PYG{o}{\PYGZhy{}}\PYG{n}{c} \PYG{n}{ibh} \PYG{o}{\PYGZhy{}}\PYG{n}{c}  \PYG{n}{conda}\PYG{o}{\PYGZhy{}}\PYG{n}{forge} \PYG{n}{modelflow\PYGZus{}pinned\PYGZus{}developement\PYGZus{}test} \PYG{o}{\PYGZhy{}}\PYG{n}{y}
\PYG{n}{conda} \PYG{n}{activate} \PYG{n}{ModelFlow}
\PYG{n}{pip} \PYG{n}{install} \PYG{n}{dash\PYGZus{}interactive\PYGZus{}graphviz}
\PYG{n}{conda} \PYG{n}{install} \PYG{n}{pyeviews} \PYG{o}{\PYGZhy{}}\PYG{n}{c} \PYG{n}{conda}\PYG{o}{\PYGZhy{}}\PYG{n}{forge} \PYG{o}{\PYGZhy{}}\PYG{n}{y}
\PYG{n}{jupyter} \PYG{n}{contrib} \PYG{n}{nbextension} \PYG{n}{install} \PYG{o}{\PYGZhy{}}\PYG{o}{\PYGZhy{}}\PYG{n}{user}
\PYG{n}{jupyter} \PYG{n}{nbextension} \PYG{n}{enable} \PYG{n}{hide\PYGZus{}input\PYGZus{}all}\PYG{o}{/}\PYG{n}{main}
\PYG{n}{jupyter} \PYG{n}{nbextension} \PYG{n}{enable} \PYG{n}{splitcell}\PYG{o}{/}\PYG{n}{splitcellcd}
\PYG{n}{jupyter} \PYG{n}{nbextension} \PYG{n}{enable} \PYG{n}{toc2}\PYG{o}{/}\PYG{n}{main}

\end{sphinxVerbatim}

\sphinxAtStartPar
Depending on the speed of your computer and of your internet connection installation could take as little as 10 minutes or more than 1/2 an hour.

\sphinxAtStartPar
At the end of the process you will have a new conda environment called \sphinxcode{\sphinxupquote{modelflow}}, and this will have been activated. The computer set up is complete and the user is ready to work with \sphinxcode{\sphinxupquote{modelflow}}.

\sphinxAtStartPar
The following sections give a brief introduction to Jupyter notebook, which is a flexible tool that allows us to execute python code, interact with the modelflow class and World Bank Models and annotate what we have done for future replication.

\sphinxstepscope


\part{Some python essentials for using WorldBank models with modelflow}

\sphinxstepscope


\chapter{Introduction to  Jupyter Notebook}
\label{\detokenize{content/04_PythonEssentials/Intro_Jupyter_notebook:introduction-to-jupyter-notebook}}\label{\detokenize{content/04_PythonEssentials/Intro_Jupyter_notebook::doc}}
\sphinxAtStartPar
Jupyter Notebook is a web application for creating, annotating, simulating and working with computational documents.  Originally developed for python, the latest versions of EViews also support Jupyter Notebooks. Jupyter Notebook offers a simple, streamlined, document\sphinxhyphen{}centric experience and can be a great environment for documenting the work you are doing, and trying alternative methods of achieving desirable results.  Many of the methods in \sphinxcode{\sphinxupquote{modelflow}} have been developed to work well with Jupyter Notebook. Indeed this documentation was written as a series of Jupyter Notebooks bound together with Jupyter Book.

\sphinxAtStartPar
Jupyter Notebook is not the only way to work with modelflow or Python.  As users become more advanced they are likely to migrate to a more program\sphinxhyphen{}centric IDE (Interactive Development Environment) like Spyder or Microsoft Visual Code.

\sphinxAtStartPar
However, to start Jupyter Notebooks are a great way to learn, follow work done by others and tweak them to fit your own needs.

\sphinxAtStartPar
There are many fine tutorials on Jupyter Notebook on the web, and \sphinxhref{https://docs.jupyter.org/en/latest/}{The official Jupyter site} is a good starting point. The following aims to provide enough information to get a user started.  Another good reference is \sphinxhref{https://jupyter.brynmawr.edu/services/public/dblank/Jupyter\%20Notebook\%20Users\%20Manual.ipynb}{here.}


\section{Starting Jupyter Notebook}
\label{\detokenize{content/04_PythonEssentials/Intro_Jupyter_notebook:starting-jupyter-notebook}}
\sphinxAtStartPar
Each time, a user wants to work with \sphinxcode{\sphinxupquote{modelflow}}, they will need to activate the \sphinxcode{\sphinxupquote{modelflow}} environment by
\begin{enumerate}
\sphinxsetlistlabels{\arabic}{enumi}{enumii}{}{.}%
\item {} 
\sphinxAtStartPar
Opening the Anaconda command prompt window

\item {} 
\sphinxAtStartPar
Activate the ModelFlow environment we just created by executing the following command

\end{enumerate}

\sphinxAtStartPar
\sphinxcode{\sphinxupquote{conda activate modelflow}}

\sphinxAtStartPar
From here, any number of mechanisms can be used to interact with \sphinxcode{\sphinxupquote{modelflow}} and World Bank models.

\sphinxAtStartPar
\sphinxstylestrong{To use Jupyter Notebook} the Jupyter notebook, must be first started.  Following steps 1\sphinxhyphen{}2 above, a user would need to execute from the conda command line:

\sphinxAtStartPar
\sphinxcode{\sphinxupquote{jupyter notebook}}

\sphinxAtStartPar
This will launch the Jupyter environment in your default web browser, which should look something like this:

\sphinxAtStartPar
\sphinxincludegraphics{{NewJNSession}.png}

\sphinxAtStartPar
where the directory structure presented is that of the directory from the \sphinxcode{\sphinxupquote{jupyter notebook}} command was executed.

\begin{sphinxadmonition}{warning}{Warning:}
\sphinxAtStartPar
Note the directory from which you execute the \sphinxcode{\sphinxupquote{jupyter notebook}} \sphinxstylestrong{mfbook} in the example above will be the \sphinxstylestrong{root} directory for the jupyter session, and only directories and files below this root directory will be accessible by jupyter.
\end{sphinxadmonition}


\section{Creating a notebook}
\label{\detokenize{content/04_PythonEssentials/Intro_Jupyter_notebook:creating-a-notebook}}
\sphinxAtStartPar
The idea behind jupyter notebook was to create an interactive version of the notebooks that scientists use(d) to:
\begin{itemize}
\item {} 
\sphinxAtStartPar
record what they have done

\item {} 
\sphinxAtStartPar
perhaps explain why

\item {} 
\sphinxAtStartPar
document how data was generated, and

\item {} 
\sphinxAtStartPar
record the results of their experiments

\end{itemize}

\sphinxAtStartPar
The motivation for these notebooks and Jupyter notebook is to record the precise steps taken to produce a set of results, which if followed by others would allow the to generate the same results.

\sphinxAtStartPar
To create a notebook you must select from the Jupyter Notebook menu

\sphinxAtStartPar
File\sphinxhyphen{}> New Notebook

\begin{figure}[htbp]
\centering
\capstart

\noindent\sphinxincludegraphics[height=150\sphinxpxdimen]{{NewNotebook}.png}
\caption{A newly created Jupyter Notebook session}\label{\detokenize{content/04_PythonEssentials/Intro_Jupyter_notebook:new-notebook}}\end{figure}

\sphinxAtStartPar
This will generate a blank unnamed notebook with one empty cell, that looks something like this:

\sphinxAtStartPar
\sphinxcode{\sphinxupquote{!{[}NewCell{]}(./Newcell.png)}}

\begin{figure}[htbp]
\centering
\capstart

\noindent\sphinxincludegraphics[height=150\sphinxpxdimen]{{Newcell}.png}
\caption{A newly created Jupyter Notebook}\label{\detokenize{content/04_PythonEssentials/Intro_Jupyter_notebook:id1}}\end{figure}

\begin{sphinxadmonition}{warning}{Warning:}
\sphinxAtStartPar
Each notebook has associated with it a “Kernel”, which is an instance of the computing environment in which code will be executed. For Jupyter Notebooks that work with \sphinxcode{\sphinxupquote{modelflow}} this will be a Python Kernel. If your computer has more than one “kernel’s” installed on it, you may be prompted when creating a new notebook for the kernel with which to associate it.  Typically this should be the Python Kernel under which your modelflow was built – currently python 3.9 in April 2023.
\end{sphinxadmonition}


\section{Jupter Notebook cells}
\label{\detokenize{content/04_PythonEssentials/Intro_Jupyter_notebook:jupter-notebook-cells}}
\sphinxAtStartPar
A Jupyter Notebook is comprised of a series of cells.

\sphinxAtStartPar
\sphinxstyleemphasis{\sphinxstylestrong{Jupyter Notebook cells can contain:}}
\begin{itemize}
\item {} 
\sphinxAtStartPar
\sphinxstylestrong{computer code} (typically python code, but as noted other kernels – like Eviews – can be used with jupyter).

\item {} 
\sphinxAtStartPar
\sphinxstylestrong{markdown text}: plain text that can include special characters that make some text appear as bold, or indicate the text is a header, or instruct Jupyter Notebook to render the text as a mathematical formula.  All of the text in this document was entered using Jupyter Notebook’s markdown language

\item {} 
\sphinxAtStartPar
Results (in the form of tables or graphs) from the execution of computer code specified in a code cell

\end{itemize}

\sphinxAtStartPar
\sphinxstylestrong{Every cell has two modes:}
\begin{enumerate}
\sphinxsetlistlabels{\arabic}{enumi}{enumii}{}{.}%
\item {} 
\sphinxAtStartPar
Edit mode – indicated by a green vertical bar. In edit mode the user can change the code, or the markdown.

\item {} 
\sphinxAtStartPar
Select/Copy mode – indicated by a blue vertical bar.  This will be teh state of the cell when its content has been executed.  For markdown cells this means that the text and special characters have been rendered into formatted text.  For code cells, this means the code has been executed and its output (if any) displayed in an output cell.

\end{enumerate}

\sphinxAtStartPar
\sphinxstylestrong{Users can switch between Edit and Select/Copy Mode by hitting Enter}

\sphinxAtStartPar
This entire book was generated using markdown cells, code cells and output cells from Jupyter Notebooks.

\begin{sphinxadmonition}{note}{Note:}
\sphinxAtStartPar
Jupyter Notebooks were designed to facilitate \sphinxstyleemphasis{replicability}: the idea that a scientific analysis should contain \sphinxhyphen{} in addition to the final output (text, graphs, tables) \sphinxhyphen{} all the computational steps needed to get from raw input data to the results.
\end{sphinxadmonition}


\subsection{How to add, delete and move cells}
\label{\detokenize{content/04_PythonEssentials/Intro_Jupyter_notebook:how-to-add-delete-and-move-cells}}
\sphinxAtStartPar
The newly created Jupyter Notebook will have a code cell by default.  Cells can be added, deleted and moved either via mouse using the toolbar or by keyboard shortcut.

\sphinxAtStartPar
\sphinxstylestrong{Using the Toolbar}
\begin{itemize}
\item {} 
\sphinxAtStartPar
\sphinxstylestrong{+ button}: add a cell below the current cell

\item {} 
\sphinxAtStartPar
\sphinxstylestrong{scissors}: cut  current cell (can be undone from “Edit” tab)

\item {} 
\sphinxAtStartPar
\sphinxstylestrong{clipboard}: paste a previously cut cell to the current location

\item {} 
\sphinxAtStartPar
\sphinxstylestrong{up\sphinxhyphen{} and down arrows}: move cells (cell must be in Select/Copy mode – vertical side bar must be blue)

\item {} 
\sphinxAtStartPar
\sphinxstylestrong{hold shift + click cells in left margin}: select multiple cells (vertical bar must be blue)

\end{itemize}

\sphinxAtStartPar
\sphinxstylestrong{Using keyboard short cuts}
\begin{itemize}
\item {} 
\sphinxAtStartPar
\sphinxstylestrong{esc + a}: add a cell above the current cell

\item {} 
\sphinxAtStartPar
\sphinxstylestrong{esc + b}: add a cell below the current cell

\item {} 
\sphinxAtStartPar
\sphinxstylestrong{esc + d+d}: delete the current cell

\end{itemize}


\subsection{Change the type of a cell}
\label{\detokenize{content/04_PythonEssentials/Intro_Jupyter_notebook:change-the-type-of-a-cell}}
\sphinxAtStartPar
You can also change the type of a cell. New cells are by default “code” cells.

\sphinxAtStartPar
\sphinxstylestrong{Using the Toolbar}
\begin{itemize}
\item {} 
\sphinxAtStartPar
Select the desired type from the drop down.  options include
\begin{itemize}
\item {} 
\sphinxAtStartPar
Markdown

\item {} 
\sphinxAtStartPar
Code

\item {} 
\sphinxAtStartPar
Raw NBConvert

\item {} 
\sphinxAtStartPar
Heading

\end{itemize}

\end{itemize}

\sphinxAtStartPar
\sphinxstylestrong{Using keyboard short cuts}
\begin{itemize}
\item {} 
\sphinxAtStartPar
\sphinxstylestrong{esc + m}: make the current cell a markdown cell

\item {} 
\sphinxAtStartPar
\sphinxstylestrong{esc + y}: make the current cell a code  cell

\end{itemize}

\sphinxAtStartPar
\sphinxstylestrong{Auto\sphinxhyphen{}complete and context\sphinxhyphen{}sensitive help}

\sphinxAtStartPar
When editing a code cell, you can use these short\sphinxhyphen{}cuts to autocomplete and or call up documentation for a command.
\begin{itemize}
\item {} 
\sphinxAtStartPar
\sphinxstylestrong{tab}: autocomplete and  method selection

\item {} 
\sphinxAtStartPar
\sphinxstylestrong{double tab}: documention (double tab for full doc)

\end{itemize}


\section{Execution of cells}
\label{\detokenize{content/04_PythonEssentials/Intro_Jupyter_notebook:execution-of-cells}}
\sphinxAtStartPar
Every cell in a Jupyter Notebook can be executed, either by using the Run button on the Jupyter Notebook menu, or by using one of \sphinxstylestrong{two keyboard shortcuts}:
\begin{itemize}
\item {} 
\sphinxAtStartPar
\sphinxstylestrong{ctrl + Enter}: Executes the code in the cell or formats the markdown of a cell.  The current cell retains the focus – cursor stays on cell executed.

\item {} 
\sphinxAtStartPar
\sphinxstylestrong{shift + enter}: Executes the code in the cell or formats the markdown of a cell. Focus (cursor) jumps to the next cell

\end{itemize}

\sphinxAtStartPar
For other useful shortcuts see “Help” => “Keyboard Shortcuts” or simply press keyboard icon in the toolbar.


\subsection{Executing python code}
\label{\detokenize{content/04_PythonEssentials/Intro_Jupyter_notebook:executing-python-code}}
\sphinxAtStartPar
Below is a code with some standard python that declares a variable “x”, assigns it the value 10, declares a second variable “y” and assigns it the value 45.  The final line of y alone, instructs python to display the value of the variable y.  The results of the operation appear in Jupyter Notebook as an output cell Out{[}\#{]}.  By pressing \sphinxstylestrong{Ctrl\sphinxhyphen{}Enter} the code will be executed and the output displayed below.

\begin{sphinxuseclass}{cell}\begin{sphinxVerbatimInput}

\begin{sphinxuseclass}{cell_input}
\begin{sphinxVerbatim}[commandchars=\\\{\}]
\PYG{n}{x} \PYG{o}{=} \PYG{l+m+mi}{10}
\PYG{n}{y} \PYG{o}{=} \PYG{l+m+mi}{45}
\PYG{n}{y}
\end{sphinxVerbatim}

\end{sphinxuseclass}\end{sphinxVerbatimInput}
\begin{sphinxVerbatimOutput}

\begin{sphinxuseclass}{cell_output}
\begin{sphinxVerbatim}[commandchars=\\\{\}]
45
\end{sphinxVerbatim}

\end{sphinxuseclass}\end{sphinxVerbatimOutput}

\end{sphinxuseclass}
\sphinxAtStartPar
\sphinxstylestrong{The semi\sphinxhyphen{}colon “;” suppresses output in Jupyter Notebook}

\sphinxAtStartPar
In the example below, a semi\sphinxhyphen{}colon “;” has been appended to the final line.  This suppresses the display of the value contained by y;  As a result there is no output cell.

\begin{sphinxuseclass}{cell}\begin{sphinxVerbatimInput}

\begin{sphinxuseclass}{cell_input}
\begin{sphinxVerbatim}[commandchars=\\\{\}]
\PYG{n}{x} \PYG{o}{=} \PYG{l+m+mi}{10}
\PYG{n}{y} \PYG{o}{=} \PYG{l+m+mi}{45}
\PYG{n}{y}\PYG{p}{;}
\end{sphinxVerbatim}

\end{sphinxuseclass}\end{sphinxVerbatimInput}

\end{sphinxuseclass}
\sphinxAtStartPar
Another way to display results is to use the print function.

\begin{sphinxuseclass}{cell}\begin{sphinxVerbatimInput}

\begin{sphinxuseclass}{cell_input}
\begin{sphinxVerbatim}[commandchars=\\\{\}]
\PYG{n}{x} \PYG{o}{=} \PYG{l+m+mi}{10}
\PYG{n+nb}{print}\PYG{p}{(}\PYG{n}{x}\PYG{p}{)}
\end{sphinxVerbatim}

\end{sphinxuseclass}\end{sphinxVerbatimInput}
\begin{sphinxVerbatimOutput}

\begin{sphinxuseclass}{cell_output}
\begin{sphinxVerbatim}[commandchars=\\\{\}]
10
\end{sphinxVerbatim}

\end{sphinxuseclass}\end{sphinxVerbatimOutput}

\end{sphinxuseclass}
\sphinxAtStartPar
Variables in a Jupyter Notebook session are persistent, as a result in the subsequent cell, we can declare a variable ‘z’ equal to 2*y and it will have the value 90.

\begin{sphinxuseclass}{cell}\begin{sphinxVerbatimInput}

\begin{sphinxuseclass}{cell_input}
\begin{sphinxVerbatim}[commandchars=\\\{\}]
\PYG{n}{z}\PYG{o}{=}\PYG{n}{y}\PYG{o}{*}\PYG{l+m+mi}{2}
\PYG{n}{z}
\end{sphinxVerbatim}

\end{sphinxuseclass}\end{sphinxVerbatimInput}
\begin{sphinxVerbatimOutput}

\begin{sphinxuseclass}{cell_output}
\begin{sphinxVerbatim}[commandchars=\\\{\}]
90
\end{sphinxVerbatim}

\end{sphinxuseclass}\end{sphinxVerbatimOutput}

\end{sphinxuseclass}

\section{Markdown cells and the markdown scripting language in Jupyter Notebook}
\label{\detokenize{content/04_PythonEssentials/Intro_Jupyter_notebook:markdown-cells-and-the-markdown-scripting-language-in-jupyter-notebook}}
\sphinxAtStartPar
Text cells in a notebook can be made more interesting by using markdown.

\sphinxAtStartPar
Cells designated as markdown cells when executed are rendered in a rich text format (html).

\sphinxAtStartPar
Markdown is a lightweight markup language for creating formatted text using a plain\sphinxhyphen{}text editor.  Used in a markdown cell of Jupyter Notebook it can be used to produce nicely formatted text that mixes text, mathematical formulae, code and outputs from executed python code.

\sphinxAtStartPar
Rather than the relatively complex commands of html <h1></h1>, markdown uses a simplified set of commands to control how text elements should be rendered.


\subsection{Common markdown commands}
\label{\detokenize{content/04_PythonEssentials/Intro_Jupyter_notebook:common-markdown-commands}}
\sphinxAtStartPar
Some of the most common of these include:


\begin{savenotes}\sphinxattablestart
\centering
\begin{tabulary}{\linewidth}[t]{|T|T|}
\hline
\sphinxstyletheadfamily 
\sphinxAtStartPar
symbol
&\sphinxstyletheadfamily 
\sphinxAtStartPar
Effect
\\
\hline
\sphinxAtStartPar
\#
&
\sphinxAtStartPar
Header
\\
\hline
\sphinxAtStartPar
\#\#
&
\sphinxAtStartPar
second level
\\
\hline
\sphinxAtStartPar
\#\#\#
&
\sphinxAtStartPar
third level etc.
\\
\hline
\sphinxAtStartPar
**Bold text**
&
\sphinxAtStartPar
\sphinxstylestrong{Bold text}
\\
\hline
\sphinxAtStartPar
*Italics text*
&
\sphinxAtStartPar
\sphinxstyleemphasis{Italics text}
\\
\hline
\sphinxAtStartPar
* text
&
\sphinxAtStartPar
Bulleted text or dot notes
\\
\hline
\sphinxAtStartPar
1. text
&
\sphinxAtStartPar
1. Numbered bullets
\\
\hline
\end{tabulary}
\par
\sphinxattableend\end{savenotes}


\subsection{Tables in markdown}
\label{\detokenize{content/04_PythonEssentials/Intro_Jupyter_notebook:tables-in-markdown}}
\sphinxAtStartPar
Tables like the one above can be constructed using | as separators.

\sphinxAtStartPar
The |:–|:—————–| on the second line tells the Table generator how to justifythe cintents of columns.  :– means left justify :–: means center justify and –: means right justify.

\sphinxAtStartPar
Below is the markdown code that generated the above table:

\begin{sphinxVerbatim}[commandchars=\\\{\}]
\PYG{o}{|} \PYG{n}{symbol}           \PYG{o}{|} \PYG{n}{Effect}          \PYG{o}{|}
\PYG{o}{|}\PYG{p}{:}\PYG{o}{\PYGZhy{}}\PYG{o}{\PYGZhy{}}\PYG{o}{|}\PYG{p}{:}\PYG{o}{\PYGZhy{}}\PYG{o}{\PYGZhy{}}\PYG{o}{\PYGZhy{}}\PYG{o}{\PYGZhy{}}\PYG{o}{\PYGZhy{}}\PYG{o}{\PYGZhy{}}\PYG{o}{|}                                \PYG{c+c1}{\PYGZsh{} Specifies the justification for the columns of the table.}
\PYG{o}{|} \PYGZbs{}\PYG{c+c1}{\PYGZsh{}               | Header        |}
\PYG{o}{|} \PYGZbs{}\PYG{c+c1}{\PYGZsh{}\PYGZbs{}\PYGZsh{}             | second level |}
\PYG{o}{|} \PYGZbs{}\PYG{o}{*}\PYGZbs{}\PYG{o}{*}\PYG{n}{Bold} \PYG{n}{text}\PYGZbs{}\PYG{o}{*}\PYGZbs{}\PYG{o}{*} \PYG{o}{|} \PYG{o}{*}\PYG{o}{*}\PYG{n}{Bold} \PYG{n}{text}\PYG{o}{*}\PYG{o}{*}   \PYG{o}{|}
\PYG{o}{|} \PYGZbs{}\PYG{o}{*}\PYG{n}{Italics} \PYG{n}{text}\PYGZbs{}\PYG{o}{*} \PYG{o}{|} \PYG{o}{*}\PYG{n}{Italics} \PYG{n}{text}\PYG{o}{*}   \PYG{o}{|}
\PYG{o}{|} 
\PYG{o}{|} \PYG{l+m+mi}{1}\PYGZbs{}\PYG{o}{.} \PYG{n}{text}  \PYG{o}{|} \PYG{l+m+mf}{1.} \PYG{n}{Numbered} \PYG{n}{bullets}   \PYG{o}{|}

\end{sphinxVerbatim}


\subsection{Displaying code}
\label{\detokenize{content/04_PythonEssentials/Intro_Jupyter_notebook:displaying-code}}
\sphinxAtStartPar
To display a (unexecutable)  block of code within a markdown cell, encapsulate it (surround it) with backticks `.

\sphinxAtStartPar
For a multiline section of code use three backticks at the beginning and end.

\sphinxAtStartPar
```
Multi line
text to be rendered as code
```.

\sphinxAtStartPar
will be rendered as: \sphinxcode{\sphinxupquote{text to be rendered as code}}.

\begin{sphinxVerbatim}[commandchars=\\\{\}]

\PYG{n}{Multi} \PYG{n}{line} 
\PYG{n}{text} \PYG{n}{to} \PYG{n}{be} \PYG{n}{rendered} \PYG{k}{as} \PYG{n}{code} 
\end{sphinxVerbatim}

\sphinxAtStartPar
For inline code references ‘ a sigle back tick at the beginning and end suffices.

\sphinxAtStartPar
\sphinxstylestrong{This sentence:}

\sphinxAtStartPar
An example sentence with some back\sphinxhyphen{}ticked `text as code` in the middle.

\sphinxAtStartPar
\sphinxstylestrong{will render as:}

\sphinxAtStartPar
An example sentence with some back\sphinxhyphen{}ticked \sphinxcode{\sphinxupquote{text as code}} in the middle.


\subsection{Rendering mathematics in markdown}
\label{\detokenize{content/04_PythonEssentials/Intro_Jupyter_notebook:rendering-mathematics-in-markdown}}
\sphinxAtStartPar
Jupyter Notebook’s implementation of Markdown supports \sphinxcode{\sphinxupquote{latex}} mathematical notation.

\sphinxAtStartPar
Inline enclose the \sphinxcode{\sphinxupquote{latex}} code in \sphinxcode{\sphinxupquote{\$}}:

\sphinxAtStartPar
An Equation: \sphinxcode{\sphinxupquote{\$y\_t = \textbackslash{}beta\_0 + \textbackslash{}beta\_1 x\_t + u\_t\textbackslash{}\$}} will renders as: \(y_t = \beta_0 + \beta_1 x_t + u_t\)

\sphinxAtStartPar
if enclosed in \sphinxcode{\sphinxupquote{\$\$}} \sphinxcode{\sphinxupquote{\$\$}} it will be centered on its own line.
\begin{equation*}
\begin{split}y_t = \beta_0 + \beta_1 x_t + u_t\end{split}
\end{equation*}

\subsubsection{Complex and multi\sphinxhyphen{}line math}
\label{\detokenize{content/04_PythonEssentials/Intro_Jupyter_notebook:complex-and-multi-line-math}}
\begin{sphinxVerbatim}[commandchars=\\\{\}]
\PYGZbs{}\PYG{n}{begin}\PYG{p}{\PYGZob{}}\PYG{n}{align}\PYG{p}{\PYGZcb{}}
\PYG{n}{Y\PYGZus{}t}  \PYG{o}{\PYGZam{}}\PYG{o}{=}  \PYG{n}{C\PYGZus{}t}\PYG{o}{+}\PYG{n}{I\PYGZus{}t}\PYG{o}{+}\PYG{n}{G}\PYG{o}{+}\PYG{n}{t}\PYG{o}{+} \PYG{p}{(}\PYG{n}{X\PYGZus{}t}\PYG{o}{\PYGZhy{}}\PYG{n}{M\PYGZus{}t}\PYG{p}{)} \PYGZbs{}\PYGZbs{}
\PYG{n}{C\PYGZus{}t} \PYG{o}{\PYGZam{}}\PYG{o}{=} \PYG{n}{c\PYGZus{}t}\PYG{p}{(}\PYG{n}{C\PYGZus{}}\PYG{p}{\PYGZob{}}\PYG{n}{t}\PYG{o}{\PYGZhy{}}\PYG{l+m+mi}{1}\PYG{p}{\PYGZcb{}}\PYG{p}{,}\PYG{n}{C\PYGZus{}}\PYG{p}{\PYGZob{}}\PYG{n}{t}\PYG{o}{\PYGZhy{}}\PYG{l+m+mi}{2}\PYG{p}{\PYGZcb{}}\PYG{p}{,}\PYG{n}{I\PYGZus{}t}\PYG{p}{,}\PYG{n}{G\PYGZus{}t}\PYG{p}{,}\PYG{n}{X\PYGZus{}t}\PYG{p}{,}\PYG{n}{M\PYGZus{}t}\PYG{p}{,}\PYG{n}{P\PYGZus{}t}\PYG{p}{)}\PYGZbs{}\PYGZbs{}
\PYG{n}{I\PYGZus{}t} \PYG{o}{\PYGZam{}}\PYG{o}{=} \PYG{n}{c\PYGZus{}t}\PYG{p}{(}\PYG{n}{I\PYGZus{}}\PYG{p}{\PYGZob{}}\PYG{n}{t}\PYG{o}{\PYGZhy{}}\PYG{l+m+mi}{1}\PYG{p}{\PYGZcb{}}\PYG{p}{,}\PYG{n}{I\PYGZus{}}\PYG{p}{\PYGZob{}}\PYG{n}{t}\PYG{o}{\PYGZhy{}}\PYG{l+m+mi}{2}\PYG{p}{\PYGZcb{}}\PYG{p}{,}\PYG{n}{C\PYGZus{}t}\PYG{p}{,}\PYG{n}{G\PYGZus{}t}\PYG{p}{,}\PYG{n}{X\PYGZus{}t}\PYG{p}{,}\PYG{n}{M\PYGZus{}t}\PYG{p}{,}\PYG{n}{P\PYGZus{}t}\PYG{p}{)}\PYGZbs{}\PYGZbs{}
\PYG{n}{G\PYGZus{}t} \PYG{o}{\PYGZam{}}\PYG{o}{=} \PYG{n}{c\PYGZus{}t}\PYG{p}{(}\PYG{n}{G\PYGZus{}}\PYG{p}{\PYGZob{}}\PYG{n}{t}\PYG{o}{\PYGZhy{}}\PYG{l+m+mi}{1}\PYG{p}{\PYGZcb{}}\PYG{p}{,}\PYG{n}{G\PYGZus{}}\PYG{p}{\PYGZob{}}\PYG{n}{t}\PYG{o}{\PYGZhy{}}\PYG{l+m+mi}{2}\PYG{p}{\PYGZcb{}}\PYG{p}{,}\PYG{n}{C\PYGZus{}t}\PYG{p}{,}\PYG{n}{I\PYGZus{}t}\PYG{p}{,}\PYG{n}{X\PYGZus{}t}\PYG{p}{,}\PYG{n}{M\PYGZus{}t}\PYG{p}{,}\PYG{n}{P\PYGZus{}t}\PYG{p}{)}\PYGZbs{}\PYGZbs{}
\PYG{n}{X\PYGZus{}t} \PYG{o}{\PYGZam{}}\PYG{o}{=} \PYG{n}{c\PYGZus{}t}\PYG{p}{(}\PYG{n}{X\PYGZus{}}\PYG{p}{\PYGZob{}}\PYG{n}{t}\PYG{o}{\PYGZhy{}}\PYG{l+m+mi}{1}\PYG{p}{\PYGZcb{}}\PYG{p}{,}\PYG{n}{X\PYGZus{}}\PYG{p}{\PYGZob{}}\PYG{n}{t}\PYG{o}{\PYGZhy{}}\PYG{l+m+mi}{2}\PYG{p}{\PYGZcb{}}\PYG{p}{,}\PYG{n}{C\PYGZus{}t}\PYG{p}{,}\PYG{n}{I\PYGZus{}t}\PYG{p}{,}\PYG{n}{G\PYGZus{}t}\PYG{p}{,}\PYG{n}{M\PYGZus{}t}\PYG{p}{,}\PYG{n}{P\PYGZus{}t}\PYG{p}{,}\PYG{n}{P}\PYG{o}{\PYGZca{}}\PYG{n}{f\PYGZus{}t}\PYG{p}{)}\PYGZbs{}\PYGZbs{}
\PYG{n}{M\PYGZus{}t} \PYG{o}{\PYGZam{}}\PYG{o}{=} \PYG{n}{c\PYGZus{}t}\PYG{p}{(}\PYG{n}{M\PYGZus{}}\PYG{p}{\PYGZob{}}\PYG{n}{t}\PYG{o}{\PYGZhy{}}\PYG{l+m+mi}{1}\PYG{p}{\PYGZcb{}}\PYG{p}{,}\PYG{n}{M\PYGZus{}}\PYG{p}{\PYGZob{}}\PYG{n}{t}\PYG{o}{\PYGZhy{}}\PYG{l+m+mi}{2}\PYG{p}{\PYGZcb{}}\PYG{p}{,}\PYG{n}{C\PYGZus{}t}\PYG{p}{,}\PYG{n}{I\PYGZus{}t}\PYG{p}{,}\PYG{n}{G\PYGZus{}t}\PYG{p}{,}\PYG{n}{X\PYGZus{}t}\PYG{p}{,}\PYG{n}{P\PYGZus{}t}\PYG{p}{,}\PYG{n}{P}\PYG{o}{\PYGZca{}}\PYG{n}{f\PYGZus{}t}\PYG{p}{)}
\PYGZbs{}\PYG{n}{end}\PYG{p}{\PYGZob{}}\PYG{n}{align}\PYG{p}{\PYGZcb{}}
\end{sphinxVerbatim}

\sphinxAtStartPar
The above \sphinxcode{\sphinxupquote{latex}} mathematics code uses the  \sphinxcode{\sphinxupquote{\&}} symbol to tell \sphinxcode{\sphinxupquote{latex}} to align the different lines (separated by \sphinxcode{\sphinxupquote{\textbackslash{}\textbackslash{}}}) on the character immediately after the \sphinxcode{\sphinxupquote{\&}}. In this instance the equals “=” sign.
\label{equation:content/04_PythonEssentials/Intro_Jupyter_notebook:1e91bedf-4003-45af-a21a-e0744585338d}\begin{align}
Y_t  &=  C_t+I_t+G+t+ (X_t-M_t) \\
C_t &= c_t(C_{t-1},C_{t-2},I_t,G_t,X_t,M_t,P_t)\\
I_t &= c_t(I_{t-1},I_{t-2},C_t,G_t,X_t,M_t,P_t)\\
G_t &= c_t(G_{t-1},G_{t-2},C_t,I_t,X_t,M_t,P_t)\\
X_t &= c_t(X_{t-1},X_{t-2},C_t,I_t,G_t,M_t,P_t,P^f_t)\\
M_t &= c_t(M_{t-1},M_{t-2},C_t,I_t,G_t,X_t,P_t,P^f_t)
\end{align}

\subsection{links to more info on markdown}
\label{\detokenize{content/04_PythonEssentials/Intro_Jupyter_notebook:links-to-more-info-on-markdown}}
\sphinxAtStartPar
There are several very good markdown cheatsheets on the internet, one of these is \sphinxhref{https://www.markdownguide.org/cheat-sheet/}{here}

\sphinxstepscope


\chapter{Some Python basics}
\label{\detokenize{content/04_PythonEssentials/PythonPandasDataframes:some-python-basics}}\label{\detokenize{content/04_PythonEssentials/PythonPandasDataframes::doc}}
\sphinxAtStartPar
Before using \sphinxcode{\sphinxupquote{modelflow}} with the World Bank’s MFMod models, users  will have to understand at least some basic elements of \sphinxcode{\sphinxupquote{python}} syntax and usage.  Notably they will need to understand about packages, libraries and classes, how to access them.


\section{Starting python in windows}
\label{\detokenize{content/04_PythonEssentials/PythonPandasDataframes:starting-python-in-windows}}
\sphinxAtStartPar
To begin using \sphinxcode{\sphinxupquote{modelflow}}, python itself needs to be started.  This can be done either using the \sphinxcode{\sphinxupquote{Anaconda}} navigator or from the command line shell. In either case, the user will need to start python and select the \sphinxcode{\sphinxupquote{modelflow}} environment.


\section{Anaconda navigator}
\label{\detokenize{content/04_PythonEssentials/PythonPandasDataframes:anaconda-navigator}}\begin{enumerate}
\sphinxsetlistlabels{\arabic}{enumi}{enumii}{}{.}%
\item {} 
\sphinxAtStartPar
Start Anaconda Navigator by typing Anaconda in the Start window and opening the Navigator (see Figure).

\item {} 
\sphinxAtStartPar
From Anaconda Navigator select the \sphinxcode{\sphinxupquote{Modelflow}} environment (see figure)

\end{enumerate}

\begin{figure}[htbp]
\centering
\capstart

\noindent\sphinxincludegraphics[height=180\sphinxpxdimen]{{AnacondaNav1}.png}
\caption{A newly created Jupyter Notebook session}\label{\detokenize{content/04_PythonEssentials/PythonPandasDataframes:start-anaconda-navigator}}\end{figure}
\begin{enumerate}
\sphinxsetlistlabels{\arabic}{enumi}{enumii}{}{.}%
\item {} 
\sphinxAtStartPar
Once the environment is selected the user can either select a command line environment or start jupyter notebook by clicking on either the
\begin{enumerate}
\sphinxsetlistlabels{\arabic}{enumii}{enumiii}{}{.}%
\item {} 
\sphinxAtStartPar
Jupyter Notebook environment

\item {} 
\sphinxAtStartPar
The command line environment

\item {} 
\sphinxAtStartPar
A programming IDE environment

\end{enumerate}

\end{enumerate}

\begin{figure}[htbp]
\centering
\capstart

\noindent\sphinxincludegraphics[height=180\sphinxpxdimen]{{NavigatorChoices}.png}
\caption{A newly created Jupyter Notebook session}\label{\detokenize{content/04_PythonEssentials/PythonPandasDataframes:id1}}\end{figure}


\section{Python  packages, libraries and classes}
\label{\detokenize{content/04_PythonEssentials/PythonPandasDataframes:python-packages-libraries-and-classes}}
\sphinxAtStartPar
Some features of \sphinxcode{\sphinxupquote{python}} are built\sphinxhyphen{}in out\sphinxhyphen{}of\sphinxhyphen{}the\sphinxhyphen{}box.  Others build up on these basic features.

\sphinxAtStartPar
A \sphinxstylestrong{python class} is a code template that defines a python object. Classes can have properties {[}variables or data{]} associated with them and methods (behaviours or functions) associated with them. In python a class is created by the keyword class. An object of type class is created (instantiated) using the class’s “constructor” – a special method that creates an object that is an instance of a class.

\sphinxAtStartPar
A \sphinxstylestrong{module} is a Python object consisting of Python code. A module can define functions, classes and variables. A module can also include runnable code.

\sphinxAtStartPar
A \sphinxstylestrong{python package} is a collection of modules that are related to each other. When a module from an external package is required by a program, that package (or module in the package) must  be \sphinxstylestrong{imported} into the current session for its modules can be put to use.

\sphinxAtStartPar
A \sphinxstylestrong{python library} is a collection of related modules or packages.

\sphinxAtStartPar
\sphinxcode{\sphinxupquote{Modelflow}} is a python package that \sphinxstyleemphasis{inherits} (build on or adds to) the methods and properties of other \sphinxcode{\sphinxupquote{python}} classes like \sphinxcode{\sphinxupquote{pandas}}, \sphinxcode{\sphinxupquote{numpy}} and \sphinxcode{\sphinxupquote{mathplotlib}}.

\begin{sphinxadmonition}{note}{Note:}
\sphinxAtStartPar
In modelflow the model is a class and we can create an instance of a model (an object filled with the characteristics of the class) by executing the code \sphinxcode{\sphinxupquote{mymodel = model(myformulas)}} see below for a working example.
\end{sphinxadmonition}


\section{Importing packages, libraries, modules and classes}
\label{\detokenize{content/04_PythonEssentials/PythonPandasDataframes:importing-packages-libraries-modules-and-classes}}
\sphinxAtStartPar
Some libraries, packages, and modules are part of the core python package and will be available (importable) from the get\sphinxhyphen{}go.  Others are not, and need to be installed before importing them into a session.

\sphinxAtStartPar
If you followed the modelflow installation instructions you have already downloaded and installed on your computer all the packages necessary for running World Bank models under modelflow.  But to work with them in a given Jupyter Notebook session or in a program context, you will also need to \sphinxcode{\sphinxupquote{import}} them into your session before you call them.

\begin{sphinxadmonition}{note}{Note:}
\sphinxAtStartPar
\sphinxstylestrong{Installation} of a package is not the same as \sphinxstylestrong{import}ing a package. To be imported a package must be installed once on the computer that wishes to use it.  Once it has been installed, the package must be imported into each python session where it is to be used.
\end{sphinxadmonition}

\sphinxAtStartPar
Typically a python program will start with the importation of the libraries, classes and modules that will be used.  Because a Jupyter Notebook is essentially a heavily annotated program, it also requires that packages used be imported.

\sphinxAtStartPar
As described above packages, libraries and modules are containers that can include other elements.  Take for example the package Math.

\sphinxAtStartPar
To import the Math Package we execute the command \sphinxcode{\sphinxupquote{ import math}}.  Having done that we can can call the functions and data that are defined in it.

\begin{sphinxuseclass}{cell}\begin{sphinxVerbatimInput}

\begin{sphinxuseclass}{cell_input}
\begin{sphinxVerbatim}[commandchars=\\\{\}]
\PYG{c+c1}{\PYGZsh{} the \PYGZdq{}\PYGZsh{}\PYGZdq{}\PYGZdq{} in a code cell indicates a comment, test after the \PYGZsh{} will not be executed}
\PYG{k+kn}{import} \PYG{n+nn}{math}

\PYG{c+c1}{\PYGZsh{} Now that we have imported math we can access some of the elements identified in the package, }
\PYG{c+c1}{\PYGZsh{} For example math contains a definition for pi, we can access that by executing the pi method }
\PYG{c+c1}{\PYGZsh{} of the library math}
\PYG{n}{math}\PYG{o}{.}\PYG{n}{pi}
\end{sphinxVerbatim}

\end{sphinxuseclass}\end{sphinxVerbatimInput}
\begin{sphinxVerbatimOutput}

\begin{sphinxuseclass}{cell_output}
\begin{sphinxVerbatim}[commandchars=\\\{\}]
3.141592653589793
\end{sphinxVerbatim}

\end{sphinxuseclass}\end{sphinxVerbatimOutput}

\end{sphinxuseclass}

\subsection{Import specific elements or classes from a module or library}
\label{\detokenize{content/04_PythonEssentials/PythonPandasDataframes:import-specific-elements-or-classes-from-a-module-or-library}}
\sphinxAtStartPar
The python package \sphinxcode{\sphinxupquote{math}} contains several functions and classes.

\sphinxAtStartPar
If I want I can import them directly. Then when I call them I will not have to precede them with the name of their libary. to do this I use the \sphinxstylestrong{from} syntax.  \sphinxcode{\sphinxupquote{from math import pi,cos,sin}} will import the pi constant and the two functions cos and sin and allow me to call them directly.

\sphinxAtStartPar
Compared these calls with the one in the preceding section – there the call to the method pi has to be preceded by its namespace designator math.  i.e. \sphinxcode{\sphinxupquote{math.pi}}. Below we import pi directly and can just call it with pi.

\begin{sphinxuseclass}{cell}\begin{sphinxVerbatimInput}

\begin{sphinxuseclass}{cell_input}
\begin{sphinxVerbatim}[commandchars=\\\{\}]
\PYG{k+kn}{from} \PYG{n+nn}{math} \PYG{k+kn}{import} \PYG{n}{pi}\PYG{p}{,}\PYG{n}{cos}\PYG{p}{,}\PYG{n}{sin}

\PYG{n+nb}{print}\PYG{p}{(}\PYG{n}{pi}\PYG{p}{)}
\PYG{n+nb}{print}\PYG{p}{(}\PYG{n}{cos}\PYG{p}{(}\PYG{l+m+mi}{3}\PYG{p}{)}\PYG{p}{)}
\end{sphinxVerbatim}

\end{sphinxuseclass}\end{sphinxVerbatimInput}
\begin{sphinxVerbatimOutput}

\begin{sphinxuseclass}{cell_output}
\begin{sphinxVerbatim}[commandchars=\\\{\}]
3.141592653589793
\PYGZhy{}0.9899924966004454
\end{sphinxVerbatim}

\end{sphinxuseclass}\end{sphinxVerbatimOutput}

\end{sphinxuseclass}

\subsection{import a class but give it an alias}
\label{\detokenize{content/04_PythonEssentials/PythonPandasDataframes:import-a-class-but-give-it-an-alias}}
\sphinxAtStartPar
A class and instead of using its full name as above or it can be given an alias, that is hopefully shorter but still obvious enough that the user knows what class is being referred to.

\sphinxAtStartPar
For example  \sphinxcode{\sphinxupquote{import math as m}} allows a call to pi using the more succint syntax \sphinxcode{\sphinxupquote{m.py}}.

\begin{sphinxuseclass}{cell}\begin{sphinxVerbatimInput}

\begin{sphinxuseclass}{cell_input}
\begin{sphinxVerbatim}[commandchars=\\\{\}]
\PYG{k+kn}{import} \PYG{n+nn}{math} \PYG{k}{as} \PYG{n+nn}{m}
\PYG{n+nb}{print}\PYG{p}{(}\PYG{n}{m}\PYG{o}{.}\PYG{n}{pi}\PYG{p}{)}
\PYG{n+nb}{print}\PYG{p}{(}\PYG{n}{m}\PYG{o}{.}\PYG{n}{cos}\PYG{p}{(}\PYG{l+m+mi}{3}\PYG{p}{)}\PYG{p}{)}
\end{sphinxVerbatim}

\end{sphinxuseclass}\end{sphinxVerbatimInput}
\begin{sphinxVerbatimOutput}

\begin{sphinxuseclass}{cell_output}
\begin{sphinxVerbatim}[commandchars=\\\{\}]
3.141592653589793
\PYGZhy{}0.9899924966004454
\end{sphinxVerbatim}

\end{sphinxuseclass}\end{sphinxVerbatimOutput}

\end{sphinxuseclass}

\subsection{Standard aliases}
\label{\detokenize{content/04_PythonEssentials/PythonPandasDataframes:standard-aliases}}
\sphinxAtStartPar
Some packages are so frequently used that by convention they have been “assigned” specific aliases.

\sphinxAtStartPar
For example:

\sphinxAtStartPar
\sphinxstylestrong{Common aliases}


\begin{savenotes}\sphinxattablestart
\centering
\begin{tabulary}{\linewidth}[t]{|T|T|T|T|}
\hline
\sphinxstyletheadfamily 
\sphinxAtStartPar
Alias
&\sphinxstyletheadfamily 
\sphinxAtStartPar
aliased package
&\sphinxstyletheadfamily 
\sphinxAtStartPar
example
&\sphinxstyletheadfamily 
\sphinxAtStartPar
functionalty
\\
\hline
\sphinxAtStartPar
pd
&
\sphinxAtStartPar
pandas
&
\sphinxAtStartPar
import pandas as pd
&
\sphinxAtStartPar
Pandas are used for storing and retriveing data
\\
\hline
\sphinxAtStartPar
np
&
\sphinxAtStartPar
numpy
&
\sphinxAtStartPar
import numpy as np
&
\sphinxAtStartPar
Numpy gives access to some advanced mathematical features
\\
\hline
\end{tabulary}
\par
\sphinxattableend\end{savenotes}

\sphinxAtStartPar
You don’t have to use those conventions but it will make your code easier to read by others who are familiar with it.


\chapter{Introduction to Pandas dataframes}
\label{\detokenize{content/04_PythonEssentials/PythonPandasDataframes:introduction-to-pandas-dataframes}}
\sphinxAtStartPar
Modelflow is built on top of the Pandas library. Pandas is the Swiss knife of data science and can perform an impressing array of date oriented tasks.

\sphinxAtStartPar
This tutorial is a very short introduction to how pandas dataframes are used with Modelflow. For a more complete discussion see any of the many tutorials on the internet, notably:
\begin{itemize}
\item {} 
\sphinxAtStartPar
\sphinxhref{https://pandas.pydata.org/}{Pandas homepage}

\item {} 
\sphinxAtStartPar
\sphinxhref{https://pandas.pydata.org/pandas-docs/stable/getting\_started/tutorials.html}{Pandas community tutorials}

\end{itemize}


\section{Import the pandas library}
\label{\detokenize{content/04_PythonEssentials/PythonPandasDataframes:import-the-pandas-library}}
\sphinxAtStartPar
As with any python program, in order to use a package or library it must first be imported into the session. As noted above, by  convention pandas is imported as pd

\begin{sphinxuseclass}{cell}\begin{sphinxVerbatimInput}

\begin{sphinxuseclass}{cell_input}
\begin{sphinxVerbatim}[commandchars=\\\{\}]
\PYG{k+kn}{import} \PYG{n+nn}{pandas} \PYG{k}{as} \PYG{n+nn}{pd} 
\end{sphinxVerbatim}

\end{sphinxuseclass}\end{sphinxVerbatimInput}

\end{sphinxuseclass}
\sphinxAtStartPar
Pandas, like any library, contains many classes and methods.  The discussion below focuses on \sphinxstylestrong{Series} and \sphinxstylestrong{DataFrames} two classes that are part of the pandas library.  Both \sphinxcode{\sphinxupquote{series}} and \sphinxcode{\sphinxupquote{dataframes}} are containers that can be used to store time\sphinxhyphen{}series data and that have associated with them a number of very useful methods for displaying and manipulating time\sphinxhyphen{}series data.Unlike other statistical packages neither \sphinxcode{\sphinxupquote{series}} nor \sphinxcode{\sphinxupquote{dataframes}} are inherently or exclusively time\sphinxhyphen{}series in nature.  \sphinxcode{\sphinxupquote{Modelflow}} and macro\sphinxhyphen{}economists use them in this way, but the classes themselves are not dated in anyway out\sphinxhyphen{}of\sphinxhyphen{}the\sphinxhyphen{}box.


\section{The \sphinxstyleliteralintitle{\sphinxupquote{Pandas}} class \sphinxstyleliteralintitle{\sphinxupquote{series}}}
\label{\detokenize{content/04_PythonEssentials/PythonPandasDataframes:the-pandas-class-series}}
\sphinxAtStartPar
A pandas series is class that can be used to instantiate an object that holds a two dimensional array comprised of values and an index.

\sphinxAtStartPar
The constructor for a \sphinxcode{\sphinxupquote{Series}} object is \sphinxcode{\sphinxupquote{pandas.Series()}}.  The content inside the parentheses will determine the nature of the series\sphinxhyphen{}object generated.  As an object\sphinxhyphen{}oriented language Python supports \sphinxcode{\sphinxupquote{overrides}} (which is to say a method can have more than one way in which it can be called). Specifically there can be different constructors defined for a class, depending on how the data that is to be used to initialize it is organized.


\subsection{Series declared from a list}
\label{\detokenize{content/04_PythonEssentials/PythonPandasDataframes:series-declared-from-a-list}}
\sphinxAtStartPar
The simplest way to create a Series is to pass an array of values as a Python list to the Series constructor.

\begin{sphinxadmonition}{note}{Note:}
\sphinxAtStartPar
A list in python is a comma delimited collection of items.  It could be text, numbers or even more complex objects.  When declared (and returned) list are enclosed in square brackets.

\sphinxAtStartPar
For example both of the following two lines are perfectly good examples of lists.

\sphinxAtStartPar
mylist={[}2,7,8,9{]}
mylist2={[}“Some text”,”Some more Text”,2,3{]}

\sphinxAtStartPar
The list is entirely agnostic about the type of data it contains.
\end{sphinxadmonition}

\sphinxAtStartPar
In the examples below Simplest, Simple and simple3 are all series – although series3 which is derived from a list mixing text and numeric values would be hard to interpret as an economic series.

\begin{sphinxuseclass}{cell}\begin{sphinxVerbatimInput}

\begin{sphinxuseclass}{cell_input}
\begin{sphinxVerbatim}[commandchars=\\\{\}]
\PYG{n}{values}\PYG{o}{=}\PYG{p}{[}\PYG{l+m+mi}{7}\PYG{p}{,}\PYG{l+m+mi}{8}\PYG{p}{,}\PYG{l+m+mi}{9}\PYG{p}{,}\PYG{l+m+mi}{10}\PYG{p}{,}\PYG{l+m+mi}{11}\PYG{p}{]}
\PYG{n}{weird}\PYG{o}{=}\PYG{p}{[}\PYG{l+s+s2}{\PYGZdq{}}\PYG{l+s+s2}{Some text}\PYG{l+s+s2}{\PYGZdq{}}\PYG{p}{,}\PYG{l+s+s2}{\PYGZdq{}}\PYG{l+s+s2}{Some more Text}\PYG{l+s+s2}{\PYGZdq{}}\PYG{p}{,}\PYG{l+m+mi}{2}\PYG{p}{,}\PYG{l+m+mi}{3}\PYG{p}{]}

\PYG{c+c1}{\PYGZsh{} Here the constructor is passed a numeric list}
\PYG{n}{Simplest}\PYG{o}{=}\PYG{n}{pd}\PYG{o}{.}\PYG{n}{Series}\PYG{p}{(}\PYG{p}{[}\PYG{l+m+mi}{2}\PYG{p}{,}\PYG{l+m+mi}{3}\PYG{p}{,}\PYG{l+m+mi}{4}\PYG{p}{,}\PYG{l+m+mi}{5}\PYG{p}{,}\PYG{l+m+mi}{6}\PYG{p}{]}\PYG{p}{)}
\PYG{n}{Simplest}
\end{sphinxVerbatim}

\end{sphinxuseclass}\end{sphinxVerbatimInput}
\begin{sphinxVerbatimOutput}

\begin{sphinxuseclass}{cell_output}
\begin{sphinxVerbatim}[commandchars=\\\{\}]
0    2
1    3
2    4
3    5
4    6
dtype: int64
\end{sphinxVerbatim}

\end{sphinxuseclass}\end{sphinxVerbatimOutput}

\end{sphinxuseclass}
\begin{sphinxuseclass}{cell}\begin{sphinxVerbatimInput}

\begin{sphinxuseclass}{cell_input}
\begin{sphinxVerbatim}[commandchars=\\\{\}]
\PYG{c+c1}{\PYGZsh{} In this case the constructor is passed a variable that contains a list}
\PYG{n}{simple2}\PYG{o}{=}\PYG{n}{pd}\PYG{o}{.}\PYG{n}{Series}\PYG{p}{(}\PYG{n}{values}\PYG{p}{)}
\PYG{n}{simple2}
\end{sphinxVerbatim}

\end{sphinxuseclass}\end{sphinxVerbatimInput}
\begin{sphinxVerbatimOutput}

\begin{sphinxuseclass}{cell_output}
\begin{sphinxVerbatim}[commandchars=\\\{\}]
0     7
1     8
2     9
3    10
4    11
dtype: int64
\end{sphinxVerbatim}

\end{sphinxuseclass}\end{sphinxVerbatimOutput}

\end{sphinxuseclass}
\begin{sphinxuseclass}{cell}\begin{sphinxVerbatimInput}

\begin{sphinxuseclass}{cell_input}
\begin{sphinxVerbatim}[commandchars=\\\{\}]
\PYG{c+c1}{\PYGZsh{} Here the constructor is passed a variable containing a list that is a mix of }
\PYG{c+c1}{\PYGZsh{} alphanumerics and numerical values}
\PYG{n}{simple3}\PYG{o}{=}\PYG{n}{pd}\PYG{o}{.}\PYG{n}{Series}\PYG{p}{(}\PYG{n}{weird}\PYG{p}{)}
\PYG{n}{simple3}
\end{sphinxVerbatim}

\end{sphinxuseclass}\end{sphinxVerbatimInput}
\begin{sphinxVerbatimOutput}

\begin{sphinxuseclass}{cell_output}
\begin{sphinxVerbatim}[commandchars=\\\{\}]
0         Some text
1    Some more Text
2                 2
3                 3
dtype: object
\end{sphinxVerbatim}

\end{sphinxuseclass}\end{sphinxVerbatimOutput}

\end{sphinxuseclass}
\sphinxAtStartPar
Note that all three series have different length.

\sphinxAtStartPar
Moreover, constructed in this way (by passing a list to the constructor) each of these \sphinxcode{\sphinxupquote{Series}} are automatically assigned a zero\sphinxhyphen{}based index (a numerial index that starts with 0).


\subsection{Series declared using a specific index}
\label{\detokenize{content/04_PythonEssentials/PythonPandasDataframes:series-declared-using-a-specific-index}}
\sphinxAtStartPar
In this example the series Simple and Simple2 are recreated (overwritten), but this time an index is specified. Here the index is declared as a(nother) list.

\begin{sphinxuseclass}{cell}\begin{sphinxVerbatimInput}

\begin{sphinxuseclass}{cell_input}
\begin{sphinxVerbatim}[commandchars=\\\{\}]
\PYG{c+c1}{\PYGZsh{} In this example the constructor is given both the values }
\PYG{c+c1}{\PYGZsh{} and specific values for the index}
\PYG{n}{Simplest}\PYG{o}{=}\PYG{n}{pd}\PYG{o}{.}\PYG{n}{Series}\PYG{p}{(}\PYG{p}{[}\PYG{l+m+mi}{2}\PYG{p}{,}\PYG{l+m+mi}{3}\PYG{p}{,}\PYG{l+m+mi}{4}\PYG{p}{,}\PYG{l+m+mi}{5}\PYG{p}{,}\PYG{l+m+mi}{6}\PYG{p}{]}\PYG{p}{,}\PYG{n}{index}\PYG{o}{=}\PYG{p}{[}\PYG{l+m+mi}{1966}\PYG{p}{,}\PYG{l+m+mi}{1967}\PYG{p}{,}\PYG{l+m+mi}{1996}\PYG{p}{,}\PYG{l+m+mi}{1999}\PYG{p}{,}\PYG{l+m+mi}{2000}\PYG{p}{]}\PYG{p}{)}
\PYG{n}{Simplest}
\end{sphinxVerbatim}

\end{sphinxuseclass}\end{sphinxVerbatimInput}
\begin{sphinxVerbatimOutput}

\begin{sphinxuseclass}{cell_output}
\begin{sphinxVerbatim}[commandchars=\\\{\}]
1966    2
1967    3
1996    4
1999    5
2000    6
dtype: int64
\end{sphinxVerbatim}

\end{sphinxuseclass}\end{sphinxVerbatimOutput}

\end{sphinxuseclass}
\begin{sphinxuseclass}{cell}\begin{sphinxVerbatimInput}

\begin{sphinxuseclass}{cell_input}
\begin{sphinxVerbatim}[commandchars=\\\{\}]
\PYG{n}{simple2}\PYG{o}{=}\PYG{n}{pd}\PYG{o}{.}\PYG{n}{Series}\PYG{p}{(}\PYG{n}{values}\PYG{p}{,}\PYG{n}{index}\PYG{o}{=}\PYG{p}{[}\PYG{l+m+mi}{1966}\PYG{p}{,}\PYG{l+m+mi}{1967}\PYG{p}{,}\PYG{l+m+mi}{1996}\PYG{p}{,}\PYG{l+m+mi}{1999}\PYG{p}{,}\PYG{l+m+mi}{2000}\PYG{p}{]}\PYG{p}{)}
\PYG{n}{simple2}
\end{sphinxVerbatim}

\end{sphinxuseclass}\end{sphinxVerbatimInput}
\begin{sphinxVerbatimOutput}

\begin{sphinxuseclass}{cell_output}
\begin{sphinxVerbatim}[commandchars=\\\{\}]
1966     7
1967     8
1996     9
1999    10
2000    11
dtype: int64
\end{sphinxVerbatim}

\end{sphinxuseclass}\end{sphinxVerbatimOutput}

\end{sphinxuseclass}
\sphinxAtStartPar
Now the Series look more like time series data!


\subsection{Create Series from a dictionary}
\label{\detokenize{content/04_PythonEssentials/PythonPandasDataframes:create-series-from-a-dictionary}}
\sphinxAtStartPar
In python a dictionary is a data structure that is more generally known in computer science as an associative array. A dictionary consists of a collection of key\sphinxhyphen{}value pairs, where each key\sphinxhyphen{}value pair \sphinxstyleemphasis{maps} or \sphinxstyleemphasis{links} the key to its associated value.

\begin{sphinxadmonition}{note}{Note:}
\sphinxAtStartPar
A dictionary is enclosed in curly brackets \{\}, versus a list which is enclosed in square brackets{[}{]}.
\end{sphinxadmonition}

\sphinxAtStartPar
Thus mydict=\{“1966”:2,”1967”:3,”1968”:4,”1969”:5,”2000”:\sphinxhyphen{}15\} creates an object called mydict.   \sphinxcode{\sphinxupquote{mydict}}maps (or links) the key “1966” links to the value 2.

\begin{sphinxadmonition}{note}{Note:}
\sphinxAtStartPar
In this example the Key was a string but we could just as easily made it a numerical value:
\end{sphinxadmonition}

\sphinxAtStartPar
mydict2=\{1966:2,1967:3,1968:4,1969:5,2000:\sphinxhyphen{}15\} creates an object called mydict2 that links (maps) the key “1966” to the value 2.

\sphinxAtStartPar
The series constructor also accepts a dictionary, and maps the key to the index of the Series.

\begin{sphinxuseclass}{cell}\begin{sphinxVerbatimInput}

\begin{sphinxuseclass}{cell_input}
\begin{sphinxVerbatim}[commandchars=\\\{\}]
\PYG{n}{mydict2}\PYG{o}{=}\PYG{p}{\PYGZob{}}\PYG{l+m+mi}{1966}\PYG{p}{:}\PYG{l+m+mi}{2}\PYG{p}{,}\PYG{l+m+mi}{1967}\PYG{p}{:}\PYG{l+m+mi}{3}\PYG{p}{,}\PYG{l+m+mi}{1968}\PYG{p}{:}\PYG{l+m+mi}{4}\PYG{p}{,}\PYG{l+m+mi}{1969}\PYG{p}{:}\PYG{l+m+mi}{5}\PYG{p}{,}\PYG{l+m+mi}{2000}\PYG{p}{:}\PYG{l+m+mi}{6}\PYG{p}{\PYGZcb{}}
\PYG{n}{simple2}\PYG{o}{=}\PYG{n}{pd}\PYG{o}{.}\PYG{n}{Series}\PYG{p}{(}\PYG{n}{mydict2}\PYG{p}{)}
\PYG{n}{simple2}
\end{sphinxVerbatim}

\end{sphinxuseclass}\end{sphinxVerbatimInput}
\begin{sphinxVerbatimOutput}

\begin{sphinxuseclass}{cell_output}
\begin{sphinxVerbatim}[commandchars=\\\{\}]
1966    2
1967    3
1968    4
1969    5
2000    6
dtype: int64
\end{sphinxVerbatim}

\end{sphinxuseclass}\end{sphinxVerbatimOutput}

\end{sphinxuseclass}

\section{Properties and methods of dataframes in modelflow}
\label{\detokenize{content/04_PythonEssentials/PythonPandasDataframes:properties-and-methods-of-dataframes-in-modelflow}}
\sphinxAtStartPar
Any class can have both properties (data) and methods (functions that operate on the data of the particular instance of the class). With object\sphinxhyphen{}oriented programming languages like python, classes can be built as supersets of existing classes. The Modelflow class \sphinxcode{\sphinxupquote{model}} inherits or encapsulates all of the features of the pandas dataframe and extends it in many important ways.  Some of the methods below are standard pandas methods, others have been added to it by \sphinxcode{\sphinxupquote{modelflow}} features

\sphinxAtStartPar
Much more detail on standard pandas dataframes can be found on the \sphinxhref{https://pandas.pydata.org/docs/reference/frame.html}{official pandas website}.


\subsection{dataframes}
\label{\detokenize{content/04_PythonEssentials/PythonPandasDataframes:dataframes}}
\sphinxAtStartPar
The dataframe is the primary structure of pandas and is a two\sphinxhyphen{}dimensional data structure with named rows and columns.  Each columns can have different data types (numeric, string, etc).

\sphinxAtStartPar
By convention, a dataframe if often called df or some other modifier followed by df, to assist in reading the code.


\subsection{Creating or instantiating a dataframe}
\label{\detokenize{content/04_PythonEssentials/PythonPandasDataframes:creating-or-instantiating-a-dataframe}}
\sphinxAtStartPar
Like any object, a dataframe can be created by calling the constructor of the pandas class \sphinxcode{\sphinxupquote{dataframe}}.

\sphinxAtStartPar
Each class has many constructors, so there are very many ways to create a \sphinxcode{\sphinxupquote{dataframe}}. The \sphinxcode{\sphinxupquote{pandas.DataFrame()}} method is constructor for the \sphinxcode{\sphinxupquote{DataFrame}} class. It takes several forms (as with \sphinxcode{\sphinxupquote{Series}}), but always returns an instance of a (instantiates) \sphinxcode{\sphinxupquote{DataFrame}} object – i.e. a variable that is a \sphinxcode{\sphinxupquote{DataFrame}}.

\sphinxAtStartPar
The code example below creates a \sphinxcode{\sphinxupquote{DataFrame}} of three columns A,B,C; indexed between 2019 and 2021.  Macroeconomists may interpret the index as dates, but for pandas they are just numbers.

\sphinxAtStartPar
Below a \sphinxcode{\sphinxupquote{DataFrame}} named \sphinxcode{\sphinxupquote{df}} is instantiated from a dictionary and assigned a specific index by passing a list of years as the index.

\begin{sphinxuseclass}{cell}\begin{sphinxVerbatimInput}

\begin{sphinxuseclass}{cell_input}
\begin{sphinxVerbatim}[commandchars=\\\{\}]
\PYG{n}{df} \PYG{o}{=} \PYG{n}{pd}\PYG{o}{.}\PYG{n}{DataFrame}\PYG{p}{(}\PYG{p}{\PYGZob{}}\PYG{l+s+s1}{\PYGZsq{}}\PYG{l+s+s1}{B}\PYG{l+s+s1}{\PYGZsq{}}\PYG{p}{:} \PYG{p}{[}\PYG{l+m+mi}{1}\PYG{p}{,}\PYG{l+m+mi}{1}\PYG{p}{,}\PYG{l+m+mi}{1}\PYG{p}{,}\PYG{l+m+mi}{1}\PYG{p}{]}\PYG{p}{,}\PYG{l+s+s1}{\PYGZsq{}}\PYG{l+s+s1}{C}\PYG{l+s+s1}{\PYGZsq{}}\PYG{p}{:}\PYG{p}{[}\PYG{l+m+mi}{1}\PYG{p}{,}\PYG{l+m+mi}{2}\PYG{p}{,}\PYG{l+m+mi}{3}\PYG{p}{,}\PYG{l+m+mi}{6}\PYG{p}{]}\PYG{p}{,}\PYG{l+s+s1}{\PYGZsq{}}\PYG{l+s+s1}{E}\PYG{l+s+s1}{\PYGZsq{}}\PYG{p}{:}\PYG{p}{[}\PYG{l+m+mi}{4}\PYG{p}{,}\PYG{l+m+mi}{4}\PYG{p}{,}\PYG{l+m+mi}{4}\PYG{p}{,}\PYG{l+m+mi}{4}\PYG{p}{]}\PYG{p}{\PYGZcb{}}\PYG{p}{,}\PYG{n}{index}\PYG{o}{=}\PYG{p}{[}\PYG{l+m+mi}{2018}\PYG{p}{,}\PYG{l+m+mi}{2019}\PYG{p}{,}\PYG{l+m+mi}{2020}\PYG{p}{,}\PYG{l+m+mi}{2021}\PYG{p}{]}\PYG{p}{)}
\PYG{n}{df} 
\end{sphinxVerbatim}

\end{sphinxuseclass}\end{sphinxVerbatimInput}
\begin{sphinxVerbatimOutput}

\begin{sphinxuseclass}{cell_output}
\begin{sphinxVerbatim}[commandchars=\\\{\}]
      B  C  E
2018  1  1  4
2019  1  2  4
2020  1  3  4
2021  1  6  4
\end{sphinxVerbatim}

\end{sphinxuseclass}\end{sphinxVerbatimOutput}

\end{sphinxuseclass}
\begin{sphinxadmonition}{note}{Note:}
\sphinxAtStartPar
In the \sphinxcode{\sphinxupquote{DataFrames}} that are used in macrostructural models like MFMod, each  column can be interpreted as a time\sphinxhyphen{}series of an economic variable. So in this dataframe,  A, B and C woudl normally be interpreted as economic time series.

\sphinxAtStartPar
There is nothing in the \sphinxcode{\sphinxupquote{DataFrame}} class that suggests that the data it stores must be time\sphinxhyphen{}series or even numeric in nature.
\end{sphinxadmonition}


\subsection{Adding a column to a dataframe}
\label{\detokenize{content/04_PythonEssentials/PythonPandasDataframes:adding-a-column-to-a-dataframe}}
\sphinxAtStartPar
If a value is assigned to a column that does not exist, pandas will add a column with that name and fill it with values resulting from  the calculation.

\begin{sphinxadmonition}{note}{Note:}
\sphinxAtStartPar
The size of the object assigned to the new column must match the size (number of rows) of the pre\sphinxhyphen{}existing \sphinxcode{\sphinxupquote{DataFarame}}.
\end{sphinxadmonition}

\begin{sphinxuseclass}{cell}\begin{sphinxVerbatimInput}

\begin{sphinxuseclass}{cell_input}
\begin{sphinxVerbatim}[commandchars=\\\{\}]
\PYG{n}{df}\PYG{p}{[}\PYG{l+s+s2}{\PYGZdq{}}\PYG{l+s+s2}{NEW}\PYG{l+s+s2}{\PYGZdq{}}\PYG{p}{]}\PYG{o}{=}\PYG{p}{[}\PYG{l+m+mi}{10}\PYG{p}{,}\PYG{l+m+mi}{12}\PYG{p}{,}\PYG{l+m+mi}{10}\PYG{p}{,}\PYG{l+m+mi}{13}\PYG{p}{]}
\PYG{n}{df}
\end{sphinxVerbatim}

\end{sphinxuseclass}\end{sphinxVerbatimInput}
\begin{sphinxVerbatimOutput}

\begin{sphinxuseclass}{cell_output}
\begin{sphinxVerbatim}[commandchars=\\\{\}]
      B  C  E  NEW
2018  1  1  4   10
2019  1  2  4   12
2020  1  3  4   10
2021  1  6  4   13
\end{sphinxVerbatim}

\end{sphinxuseclass}\end{sphinxVerbatimOutput}

\end{sphinxuseclass}

\subsection{Revising values}
\label{\detokenize{content/04_PythonEssentials/PythonPandasDataframes:revising-values}}
\sphinxAtStartPar
If the column exists than the = method will revise the values of the rows with the values assigned in the statement.

\begin{sphinxadmonition}{warning}{Warning:}
\sphinxAtStartPar
The dimensions of the list assigned via the \sphinxcode{\sphinxupquote{=}} method must be the same as the \sphinxcode{\sphinxupquote{DataFrame}} (i.e. there must be exactly as many values as there are rows).  Alternatively if only one value is provided, then that value will replace all of the values in the specified column (be broadcast to the other rows in the column).
\end{sphinxadmonition}

\begin{sphinxuseclass}{cell}\begin{sphinxVerbatimInput}

\begin{sphinxuseclass}{cell_input}
\begin{sphinxVerbatim}[commandchars=\\\{\}]
\PYG{n}{df}\PYG{p}{[}\PYG{l+s+s2}{\PYGZdq{}}\PYG{l+s+s2}{NEW}\PYG{l+s+s2}{\PYGZdq{}}\PYG{p}{]}\PYG{o}{=}\PYG{p}{[}\PYG{l+m+mi}{11}\PYG{p}{,}\PYG{l+m+mi}{12}\PYG{p}{,}\PYG{l+m+mi}{10}\PYG{p}{,}\PYG{l+m+mi}{14}\PYG{p}{]}

\PYG{n}{df}
\end{sphinxVerbatim}

\end{sphinxuseclass}\end{sphinxVerbatimInput}
\begin{sphinxVerbatimOutput}

\begin{sphinxuseclass}{cell_output}
\begin{sphinxVerbatim}[commandchars=\\\{\}]
      B  C  E  NEW
2018  1  1  4   11
2019  1  2  4   12
2020  1  3  4   10
2021  1  6  4   14
\end{sphinxVerbatim}

\end{sphinxuseclass}\end{sphinxVerbatimOutput}

\end{sphinxuseclass}
\begin{sphinxuseclass}{cell}\begin{sphinxVerbatimInput}

\begin{sphinxuseclass}{cell_input}
\begin{sphinxVerbatim}[commandchars=\\\{\}]
\PYG{c+c1}{\PYGZsh{} replace all of the rows of column B with the same value}
\PYG{n}{df}\PYG{p}{[}\PYG{l+s+s1}{\PYGZsq{}}\PYG{l+s+s1}{B}\PYG{l+s+s1}{\PYGZsq{}}\PYG{p}{]}\PYG{o}{=}\PYG{l+m+mi}{17}
\PYG{n}{df}
\end{sphinxVerbatim}

\end{sphinxuseclass}\end{sphinxVerbatimInput}
\begin{sphinxVerbatimOutput}

\begin{sphinxuseclass}{cell_output}
\begin{sphinxVerbatim}[commandchars=\\\{\}]
       B  C  E  NEW
2018  17  1  4   11
2019  17  2  4   12
2020  17  3  4   10
2021  17  6  4   14
\end{sphinxVerbatim}

\end{sphinxuseclass}\end{sphinxVerbatimOutput}

\end{sphinxuseclass}

\section{Column names in  Modelflow}
\label{\detokenize{content/04_PythonEssentials/PythonPandasDataframes:column-names-in-modelflow}}
\begin{sphinxShadowBox}
\sphinxstylesidebartitle{Modelflow variable names}

\sphinxAtStartPar
Modelflow places more restrictions on column names than do pandas \sphinxstyleemphasis{per se}.
\end{sphinxShadowBox}

\sphinxAtStartPar
While pandas dataframes are very liberal in what names can be given to columns, \sphinxcode{\sphinxupquote{modelflow}} is more restrictive.

\sphinxAtStartPar
Specifically, in modelflow a variable name must:
\begin{itemize}
\item {} 
\sphinxAtStartPar
start with a letter

\item {} 
\sphinxAtStartPar
be upper case

\end{itemize}

\sphinxAtStartPar
Thus while all these are legal column names in pandas, some are illegal in modelflow.


\begin{savenotes}\sphinxattablestart
\centering
\begin{tabulary}{\linewidth}[t]{|T|T|T|}
\hline
\sphinxstyletheadfamily 
\sphinxAtStartPar
Variable Name
&\sphinxstyletheadfamily 
\sphinxAtStartPar
Legal in modelfow?
&\sphinxstyletheadfamily 
\sphinxAtStartPar
Reason
\\
\hline
\sphinxAtStartPar
IB
&
\sphinxAtStartPar
yes
&
\sphinxAtStartPar
Starts with a letter and is uppercase
\\
\hline
\sphinxAtStartPar
ib
&
\sphinxAtStartPar
no
&
\sphinxAtStartPar
 lowercase letters are not allowed
\\
\hline
\sphinxAtStartPar
42ANSWER
&
\sphinxAtStartPar
No
&
\sphinxAtStartPar
 does not start with a letter 
\\
\hline
\sphinxAtStartPar
\_HORSE1
&
\sphinxAtStartPar
No
&
\sphinxAtStartPar
does not start with a letter 
\\
\hline
\sphinxAtStartPar
A\_VERY\_LONG\_NAME\_THAT\_IS\_LEGAL\_3
&
\sphinxAtStartPar
Yes
&
\sphinxAtStartPar
 Starts with a letter and is uppercase 
\\
\hline
\end{tabulary}
\par
\sphinxattableend\end{savenotes}


\section{.index and time dimensions in Modelflow}
\label{\detokenize{content/04_PythonEssentials/PythonPandasDataframes:index-and-time-dimensions-in-modelflow}}
\sphinxAtStartPar
As we saw above, series have indices.  Dataframes also have indices, which are the row names of the dataframe.

\sphinxAtStartPar
In \sphinxcode{\sphinxupquote{modelflow}} the index series is typically understood to represent a date.

\sphinxAtStartPar
For yearly models a list of integers like in the above example works fine.

\sphinxAtStartPar
For higher frequency models the index can be one of pandas datatypes.

\begin{sphinxadmonition}{warning}{Warning:}
\sphinxAtStartPar
Not all datetypes work well with the graphics routines of modelflow.  Users are advised to use use the \sphinxcode{\sphinxupquote{pd.period\_range()}} method to generate date indexes.

\sphinxAtStartPar
For example:

\begin{sphinxVerbatim}[commandchars=\\\{\}]
\PYG{n}{dates} \PYG{o}{=} \PYG{n}{pd}\PYG{o}{.}\PYG{n}{period\PYGZus{}range}\PYG{p}{(}\PYG{n}{start}\PYG{o}{=}\PYG{l+s+s1}{\PYGZsq{}}\PYG{l+s+s1}{1975q1}\PYG{l+s+s1}{\PYGZsq{}}\PYG{p}{,}\PYG{n}{end}\PYG{o}{=}\PYG{l+s+s1}{\PYGZsq{}}\PYG{l+s+s1}{2125q4}\PYG{l+s+s1}{\PYGZsq{}}\PYG{p}{,}\PYG{n}{freq}\PYG{o}{=}\PYG{l+s+s1}{\PYGZsq{}}\PYG{l+s+s1}{Q}\PYG{l+s+s1}{\PYGZsq{}}\PYG{p}{)}
\PYG{n}{df}\PYG{o}{.}\PYG{n}{index}\PYG{o}{=}\PYG{n}{dates}
\end{sphinxVerbatim}
\end{sphinxadmonition}


\subsection{Leads and lags}
\label{\detokenize{content/04_PythonEssentials/PythonPandasDataframes:leads-and-lags}}
\sphinxAtStartPar
In modelflow leads and lags can be indicated by following the variable with a parenthesis and either \sphinxhyphen{}1 or \sphinxhyphen{}2 two for one or two period lags (where the number following the negative sign indicates the number of time periods that are lagged). Positive numbers are used for forward leads (no +sign required).

\sphinxAtStartPar
When a method defined by the \sphinxcode{\sphinxupquote{modelflow}} class encounters something like \sphinxcode{\sphinxupquote{A(\sphinxhyphen{}1)}}, it will take the value from the row above the current row. No matter if the index is an integer, a year, quarter or a millisecond. The same goes for leads, \sphinxcode{\sphinxupquote{A(+1)}} will return the value of \sphinxcode{\sphinxupquote{A}} in the next row.

\sphinxAtStartPar
As a result in a quarterly model \sphinxcode{\sphinxupquote{B=A(\sphinxhyphen{}4)}} would assign B the value of A from the same quarter in the previous year.


\subsection{.columns lists the column names of a dataframe}
\label{\detokenize{content/04_PythonEssentials/PythonPandasDataframes:columns-lists-the-column-names-of-a-dataframe}}
\sphinxAtStartPar
The method \sphinxcode{\sphinxupquote{.columns}} returns the names of the columns in the dataframe.

\begin{sphinxuseclass}{cell}\begin{sphinxVerbatimInput}

\begin{sphinxuseclass}{cell_input}
\begin{sphinxVerbatim}[commandchars=\\\{\}]
\PYG{n}{df}\PYG{o}{.}\PYG{n}{columns}
\end{sphinxVerbatim}

\end{sphinxuseclass}\end{sphinxVerbatimInput}
\begin{sphinxVerbatimOutput}

\begin{sphinxuseclass}{cell_output}
\begin{sphinxVerbatim}[commandchars=\\\{\}]
Index([\PYGZsq{}B\PYGZsq{}, \PYGZsq{}C\PYGZsq{}, \PYGZsq{}E\PYGZsq{}, \PYGZsq{}NEW\PYGZsq{}], dtype=\PYGZsq{}object\PYGZsq{})
\end{sphinxVerbatim}

\end{sphinxuseclass}\end{sphinxVerbatimOutput}

\end{sphinxuseclass}

\subsection{.size indicates the dimension of a list}
\label{\detokenize{content/04_PythonEssentials/PythonPandasDataframes:size-indicates-the-dimension-of-a-list}}
\sphinxAtStartPar
so \sphinxcode{\sphinxupquote{df.columns.size}} returns the number of columns in a dataframe.

\begin{sphinxuseclass}{cell}\begin{sphinxVerbatimInput}

\begin{sphinxuseclass}{cell_input}
\begin{sphinxVerbatim}[commandchars=\\\{\}]
\PYG{n}{df}\PYG{o}{.}\PYG{n}{columns}\PYG{o}{.}\PYG{n}{size}
\end{sphinxVerbatim}

\end{sphinxuseclass}\end{sphinxVerbatimInput}
\begin{sphinxVerbatimOutput}

\begin{sphinxuseclass}{cell_output}
\begin{sphinxVerbatim}[commandchars=\\\{\}]
4
\end{sphinxVerbatim}

\end{sphinxuseclass}\end{sphinxVerbatimOutput}

\end{sphinxuseclass}
\sphinxAtStartPar
The dataframe df has 4 columns.


\subsection{.eval() evaluates calculates an expression on the data of a dataframe}
\label{\detokenize{content/04_PythonEssentials/PythonPandasDataframes:eval-evaluates-calculates-an-expression-on-the-data-of-a-dataframe}}
\sphinxAtStartPar
\sphinxcode{\sphinxupquote{.eval}} is a native dataframe method, which does calculations on a \sphinxcode{\sphinxupquote{dataframe}} and returns a revised \sphinxcode{\sphinxupquote{datafame}}. With this method expressions can be evaluated and new columns created.

\begin{sphinxuseclass}{cell}\begin{sphinxVerbatimInput}

\begin{sphinxuseclass}{cell_input}
\begin{sphinxVerbatim}[commandchars=\\\{\}]
\PYG{n}{df}\PYG{o}{.}\PYG{n}{eval}\PYG{p}{(}\PYG{l+s+s1}{\PYGZsq{}\PYGZsq{}\PYGZsq{}}\PYG{l+s+s1}{X = B*C}
\PYG{l+s+s1}{           THE\PYGZus{}ANSWER = 42}\PYG{l+s+s1}{\PYGZsq{}\PYGZsq{}\PYGZsq{}}\PYG{p}{)}
\end{sphinxVerbatim}

\end{sphinxuseclass}\end{sphinxVerbatimInput}
\begin{sphinxVerbatimOutput}

\begin{sphinxuseclass}{cell_output}
\begin{sphinxVerbatim}[commandchars=\\\{\}]
       B  C  E  NEW    X  THE\PYGZus{}ANSWER
2018  17  1  4   11   17          42
2019  17  2  4   12   34          42
2020  17  3  4   10   51          42
2021  17  6  4   14  102          42
\end{sphinxVerbatim}

\end{sphinxuseclass}\end{sphinxVerbatimOutput}

\end{sphinxuseclass}
\begin{sphinxuseclass}{cell}\begin{sphinxVerbatimInput}

\begin{sphinxuseclass}{cell_input}
\begin{sphinxVerbatim}[commandchars=\\\{\}]
\PYG{n}{df}
\end{sphinxVerbatim}

\end{sphinxuseclass}\end{sphinxVerbatimInput}
\begin{sphinxVerbatimOutput}

\begin{sphinxuseclass}{cell_output}
\begin{sphinxVerbatim}[commandchars=\\\{\}]
       B  C  E  NEW
2018  17  1  4   11
2019  17  2  4   12
2020  17  3  4   10
2021  17  6  4   14
\end{sphinxVerbatim}

\end{sphinxuseclass}\end{sphinxVerbatimOutput}

\end{sphinxuseclass}
\sphinxAtStartPar
In the above example the resulting dataframe is displayed but is not stored.

\sphinxAtStartPar
To store it, the results of the calculation must be assigned to a variable.  The pre\sphinxhyphen{}existing dataframe can be overwritten by assigning it the result of the eval statement.

\begin{sphinxuseclass}{cell}\begin{sphinxVerbatimInput}

\begin{sphinxuseclass}{cell_input}
\begin{sphinxVerbatim}[commandchars=\\\{\}]
\PYG{n}{df}\PYG{o}{=}\PYG{n}{df}\PYG{o}{.}\PYG{n}{eval}\PYG{p}{(}\PYG{l+s+s1}{\PYGZsq{}\PYGZsq{}\PYGZsq{}}\PYG{l+s+s1}{X = B*C}
\PYG{l+s+s1}{           THE\PYGZus{}ANSWER = 42}\PYG{l+s+s1}{\PYGZsq{}\PYGZsq{}\PYGZsq{}}\PYG{p}{)}
\PYG{n}{df}
\end{sphinxVerbatim}

\end{sphinxuseclass}\end{sphinxVerbatimInput}
\begin{sphinxVerbatimOutput}

\begin{sphinxuseclass}{cell_output}
\begin{sphinxVerbatim}[commandchars=\\\{\}]
       B  C  E  NEW    X  THE\PYGZus{}ANSWER
2018  17  1  4   11   17          42
2019  17  2  4   12   34          42
2020  17  3  4   10   51          42
2021  17  6  4   14  102          42
\end{sphinxVerbatim}

\end{sphinxuseclass}\end{sphinxVerbatimOutput}

\end{sphinxuseclass}
\sphinxAtStartPar
With this operation the new columns, x and THE\_ANSWER have been appended to the dataframe df.

\begin{sphinxadmonition}{note}{Note:}
\sphinxAtStartPar
The \sphinxcode{\sphinxupquote{.eval()}} method is a native pandas method.  As such it cannot handle lagged variables (because pandas do not support the idea of a lagged variable.

\sphinxAtStartPar
The \sphinxcode{\sphinxupquote{.mfcalc()}} and the \sphinxcode{\sphinxupquote{.upd()}} methods discussed below are \sphinxcode{\sphinxupquote{modelflow}} features that extend the functionalities native to \sphinxcode{\sphinxupquote{dataframe}} that allows such calculations to be performed.
\end{sphinxadmonition}


\subsection{.loc{[}{]} selects a portion (slice) of a dataframe}
\label{\detokenize{content/04_PythonEssentials/PythonPandasDataframes:loc-selects-a-portion-slice-of-a-dataframe}}
\sphinxAtStartPar
The \sphinxcode{\sphinxupquote{.loc{[}{]}}} method allows you to display and/or revise specific sub\sphinxhyphen{}sections of a column or row in a dataframe.


\subsubsection{.loc{[}row,column{]} A single element}
\label{\detokenize{content/04_PythonEssentials/PythonPandasDataframes:loc-row-column-a-single-element}}
\sphinxAtStartPar
\sphinxcode{\sphinxupquote{.loc{[}row,column{]}}} operates on a single cell in the dataframe.  Thus the below displays the value of the cell with index=2019 observation from the  column C.

\begin{sphinxuseclass}{cell}\begin{sphinxVerbatimInput}

\begin{sphinxuseclass}{cell_input}
\begin{sphinxVerbatim}[commandchars=\\\{\}]
\PYG{n}{df}\PYG{o}{.}\PYG{n}{loc}\PYG{p}{[}\PYG{l+m+mi}{2019}\PYG{p}{,}\PYG{l+s+s1}{\PYGZsq{}}\PYG{l+s+s1}{C}\PYG{l+s+s1}{\PYGZsq{}}\PYG{p}{]}
\end{sphinxVerbatim}

\end{sphinxuseclass}\end{sphinxVerbatimInput}
\begin{sphinxVerbatimOutput}

\begin{sphinxuseclass}{cell_output}
\begin{sphinxVerbatim}[commandchars=\\\{\}]
2
\end{sphinxVerbatim}

\end{sphinxuseclass}\end{sphinxVerbatimOutput}

\end{sphinxuseclass}

\subsubsection{.loc{[}:,column{]} A single column}
\label{\detokenize{content/04_PythonEssentials/PythonPandasDataframes:loc-column-a-single-column}}
\sphinxAtStartPar
The lone colon in a loc statement indicates all the rows or columns.  Here all of the rows.

\begin{sphinxuseclass}{cell}\begin{sphinxVerbatimInput}

\begin{sphinxuseclass}{cell_input}
\begin{sphinxVerbatim}[commandchars=\\\{\}]
\PYG{n}{df}\PYG{o}{.}\PYG{n}{loc}\PYG{p}{[}\PYG{p}{:}\PYG{p}{,}\PYG{l+s+s1}{\PYGZsq{}}\PYG{l+s+s1}{C}\PYG{l+s+s1}{\PYGZsq{}}\PYG{p}{]}
\end{sphinxVerbatim}

\end{sphinxuseclass}\end{sphinxVerbatimInput}
\begin{sphinxVerbatimOutput}

\begin{sphinxuseclass}{cell_output}
\begin{sphinxVerbatim}[commandchars=\\\{\}]
2018    1
2019    2
2020    3
2021    6
Name: C, dtype: int64
\end{sphinxVerbatim}

\end{sphinxuseclass}\end{sphinxVerbatimOutput}

\end{sphinxuseclass}

\subsubsection{.loc{[}row,:{]} A single row}
\label{\detokenize{content/04_PythonEssentials/PythonPandasDataframes:loc-row-a-single-row}}
\sphinxAtStartPar
Here all of the columns, for the selected row.

\begin{sphinxuseclass}{cell}\begin{sphinxVerbatimInput}

\begin{sphinxuseclass}{cell_input}
\begin{sphinxVerbatim}[commandchars=\\\{\}]
\PYG{n}{df}\PYG{o}{.}\PYG{n}{loc}\PYG{p}{[}\PYG{l+m+mi}{2019}\PYG{p}{,}\PYG{p}{:}\PYG{p}{]}
\end{sphinxVerbatim}

\end{sphinxuseclass}\end{sphinxVerbatimInput}
\begin{sphinxVerbatimOutput}

\begin{sphinxuseclass}{cell_output}
\begin{sphinxVerbatim}[commandchars=\\\{\}]
B             17
C              2
E              4
NEW           12
X             34
THE\PYGZus{}ANSWER    42
Name: 2019, dtype: int64
\end{sphinxVerbatim}

\end{sphinxuseclass}\end{sphinxVerbatimOutput}

\end{sphinxuseclass}

\subsubsection{.loc{[}:,{[}names…{]}{]} Several columns}
\label{\detokenize{content/04_PythonEssentials/PythonPandasDataframes:loc-names-several-columns}}
\sphinxAtStartPar
Passing a list in either the rows or columns portion of the loc statement will allow multiple rows or columns to be displayed.

\begin{sphinxuseclass}{cell}\begin{sphinxVerbatimInput}

\begin{sphinxuseclass}{cell_input}
\begin{sphinxVerbatim}[commandchars=\\\{\}]
\PYG{n}{df}\PYG{o}{.}\PYG{n}{loc}\PYG{p}{[}\PYG{p}{[}\PYG{l+m+mi}{2018}\PYG{p}{,}\PYG{l+m+mi}{2021}\PYG{p}{]}\PYG{p}{,}\PYG{p}{[}\PYG{l+s+s1}{\PYGZsq{}}\PYG{l+s+s1}{B}\PYG{l+s+s1}{\PYGZsq{}}\PYG{p}{,}\PYG{l+s+s1}{\PYGZsq{}}\PYG{l+s+s1}{C}\PYG{l+s+s1}{\PYGZsq{}}\PYG{p}{]}\PYG{p}{]}
\end{sphinxVerbatim}

\end{sphinxuseclass}\end{sphinxVerbatimInput}
\begin{sphinxVerbatimOutput}

\begin{sphinxuseclass}{cell_output}
\begin{sphinxVerbatim}[commandchars=\\\{\}]
       B  C
2018  17  1
2021  17  6
\end{sphinxVerbatim}

\end{sphinxuseclass}\end{sphinxVerbatimOutput}

\end{sphinxuseclass}

\subsubsection{.loc using the colon to select a range}
\label{\detokenize{content/04_PythonEssentials/PythonPandasDataframes:loc-using-the-colon-to-select-a-range}}
\sphinxAtStartPar
with the colon operator we can also select a range of results.

\sphinxAtStartPar
Here from 2018 to 2019.

\begin{sphinxuseclass}{cell}\begin{sphinxVerbatimInput}

\begin{sphinxuseclass}{cell_input}
\begin{sphinxVerbatim}[commandchars=\\\{\}]
\PYG{n}{df}\PYG{o}{.}\PYG{n}{loc}\PYG{p}{[}\PYG{l+m+mi}{2018}\PYG{p}{:}\PYG{l+m+mi}{2020}\PYG{p}{,}\PYG{p}{[}\PYG{l+s+s1}{\PYGZsq{}}\PYG{l+s+s1}{B}\PYG{l+s+s1}{\PYGZsq{}}\PYG{p}{,}\PYG{l+s+s1}{\PYGZsq{}}\PYG{l+s+s1}{C}\PYG{l+s+s1}{\PYGZsq{}}\PYG{p}{]}\PYG{p}{]}
\end{sphinxVerbatim}

\end{sphinxuseclass}\end{sphinxVerbatimInput}
\begin{sphinxVerbatimOutput}

\begin{sphinxuseclass}{cell_output}
\begin{sphinxVerbatim}[commandchars=\\\{\}]
       B  C
2018  17  1
2019  17  2
2020  17  3
\end{sphinxVerbatim}

\end{sphinxuseclass}\end{sphinxVerbatimOutput}

\end{sphinxuseclass}

\subsubsection{.loc{[}{]} can also be used on the left hand side to assign values to specific cells}
\label{\detokenize{content/04_PythonEssentials/PythonPandasDataframes:loc-can-also-be-used-on-the-left-hand-side-to-assign-values-to-specific-cells}}
\sphinxAtStartPar
This can be very handy when updating scenarios.

\begin{sphinxuseclass}{cell}\begin{sphinxVerbatimInput}

\begin{sphinxuseclass}{cell_input}
\begin{sphinxVerbatim}[commandchars=\\\{\}]
\PYG{n}{df}\PYG{o}{.}\PYG{n}{loc}\PYG{p}{[}\PYG{l+m+mi}{2019}\PYG{p}{:}\PYG{l+m+mi}{2020}\PYG{p}{,}\PYG{l+s+s1}{\PYGZsq{}}\PYG{l+s+s1}{C}\PYG{l+s+s1}{\PYGZsq{}}\PYG{p}{]} \PYG{o}{=} \PYG{l+m+mi}{17}
\PYG{n}{df}
\end{sphinxVerbatim}

\end{sphinxuseclass}\end{sphinxVerbatimInput}
\begin{sphinxVerbatimOutput}

\begin{sphinxuseclass}{cell_output}
\begin{sphinxVerbatim}[commandchars=\\\{\}]
       B   C  E  NEW    X  THE\PYGZus{}ANSWER
2018  17   1  4   11   17          42
2019  17  17  4   12   34          42
2020  17  17  4   10   51          42
2021  17   6  4   14  102          42
\end{sphinxVerbatim}

\end{sphinxuseclass}\end{sphinxVerbatimOutput}

\end{sphinxuseclass}
\begin{sphinxadmonition}{warning}{Warning:}
\sphinxAtStartPar
The dimensions on the right hand side of = and the left hand side should match. That is: either the dimensions should be the same, or the right hand side should be \sphinxcode{\sphinxupquote{broadcasted}} into the left hand slice.

\sphinxAtStartPar
For more on broadcasting \sphinxhref{https://jakevdp.github.io/PythonDataScienceHandbook/02.05-computation-on-arrays-broadcasting.html}{see here}
\end{sphinxadmonition}

\sphinxAtStartPar
\sphinxstylestrong{For more info on the .loc{[}{]} method}
\begin{itemize}
\item {} 
\sphinxAtStartPar
\sphinxhref{https://pandas.pydata.org/docs/reference/api/pandas.DataFrame.loc.html}{Description}

\item {} 
\sphinxAtStartPar
\sphinxhref{https://www.google.com/search?q=pandas+dataframe+loc\&newwindow=1}{Search}

\end{itemize}

\sphinxAtStartPar
\sphinxstylestrong{For more info on pandas:}
\begin{itemize}
\item {} 
\sphinxAtStartPar
\sphinxhref{https://pandas.pydata.org/}{Pandas homepage}

\item {} 
\sphinxAtStartPar
\sphinxhref{https://pandas.pydata.org/pandas-docs/stable/getting\_started/tutorials.html}{Pandas community tutorials}

\end{itemize}

\sphinxstepscope


\chapter{Modelflow extensions to pandas}
\label{\detokenize{content/04_PythonEssentials/UpdateCommand:modelflow-extensions-to-pandas}}\label{\detokenize{content/04_PythonEssentials/UpdateCommand::doc}}
\sphinxAtStartPar
Modeflow inherits all the capabilities of pandas and extends some as well.

\sphinxAtStartPar
Data in a dataframe can be modified directly with built\sphinxhyphen{}in pandas functionalities like \sphinxcode{\sphinxupquote{.loc{[}{]}}}, but \sphinxcode{\sphinxupquote{modelflow}} extends these capabilities with in important ways with the \sphinxcode{\sphinxupquote{.upd()}} and \sphinxcode{\sphinxupquote{.mfcalc()}} methods.


\section{.upd() method of modelflow}
\label{\detokenize{content/04_PythonEssentials/UpdateCommand:upd-method-of-modelflow}}
\sphinxAtStartPar
The \sphinxcode{\sphinxupquote{.upd()}} method extends pandas by giving the user a concise and expressive way to modify data in a dataframe using a syntax that a database\sphinxhyphen{}manager or macroeconomic modeler might find more natural.

\sphinxAtStartPar
Notably it allows the user to employ formula’s to do updates, and supports both lags and leads on variables.

\sphinxAtStartPar
\sphinxcode{\sphinxupquote{.upd()}} can be used to:
\begin{itemize}
\item {} 
\sphinxAtStartPar
Perform different types of  updates

\item {} 
\sphinxAtStartPar
Perform multiple updates each on a new line

\item {} 
\sphinxAtStartPar
Perform changes over specific periods

\item {} 
\sphinxAtStartPar
Use one input which is used for all time frames, or a separate input for each time

\item {} 
\sphinxAtStartPar
Preserve pre\sphinxhyphen{}shock growth rates for out of sample time\sphinxhyphen{}periods

\item {} 
\sphinxAtStartPar
Display results

\end{itemize}


\subsection{\sphinxstyleliteralintitle{\sphinxupquote{.upd()}} method operators}
\label{\detokenize{content/04_PythonEssentials/UpdateCommand:upd-method-operators}}
\sphinxAtStartPar
Below are some of the operators that can be used in the \sphinxcode{\sphinxupquote{.upd()}} method

\sphinxAtStartPar
\sphinxstylestrong{Types of update:}


\begin{savenotes}\sphinxattablestart
\centering
\begin{tabulary}{\linewidth}[t]{|T|T|}
\hline
\sphinxstyletheadfamily 
\sphinxAtStartPar
Update to perform
&\sphinxstyletheadfamily 
\sphinxAtStartPar
Use this operator
\\
\hline
\sphinxAtStartPar
Set a variable equal to the input
&
\sphinxAtStartPar
=
\\
\hline
\sphinxAtStartPar
Add the input to the input
&
\sphinxAtStartPar
+
\\
\hline
\sphinxAtStartPar
Set the variable to itself multiplied by the input
&
\sphinxAtStartPar
*
\\
\hline
\sphinxAtStartPar
Increase/Decrease the variable by a percent of itself (1+input/100)
&
\sphinxAtStartPar
\%
\\
\hline
\sphinxAtStartPar
Set the growth rate of the variable to the input
&
\sphinxAtStartPar
=growth
\\
\hline
\sphinxAtStartPar
Change the growth rate of the variable to its current growth rate plus the input value in percentage points
&
\sphinxAtStartPar
+growth
\\
\hline
\sphinxAtStartPar
Specify the amount by which the variable should increase from its previous period level (\(\Delta = var_t - var_{t-1}\))
&
\sphinxAtStartPar
=diff
\\
\hline
\end{tabulary}
\par
\sphinxattableend\end{savenotes}

\begin{sphinxadmonition}{danger}{Danger:}
\sphinxAtStartPar
Note: the syntax of an update command requires that there be a space between variable names and the operators.

\sphinxAtStartPar
Thus \sphinxcode{\sphinxupquote{df.upd("A = 7")}} is fine, but \sphinxcode{\sphinxupquote{df.upd("A =7")}} will generate an error.

\sphinxAtStartPar
Similarly  \sphinxcode{\sphinxupquote{df.upd("A * 1.1")}} is fine, but \sphinxcode{\sphinxupquote{df.upd("A* 1.1")}} will generate an error.
\end{sphinxadmonition}


\subsection{\sphinxstyleliteralintitle{\sphinxupquote{.upd()}} some examples}
\label{\detokenize{content/04_PythonEssentials/UpdateCommand:upd-some-examples}}

\subsection{Setting up the python environment}
\label{\detokenize{content/04_PythonEssentials/UpdateCommand:setting-up-the-python-environment}}
\sphinxAtStartPar
In order to use \sphinxcode{\sphinxupquote{.upd()}} all of the necessary libraries must be \sphinxstylestrong{imported} into the python session.

\begin{sphinxuseclass}{cell}\begin{sphinxVerbatimInput}

\begin{sphinxuseclass}{cell_input}
\begin{sphinxVerbatim}[commandchars=\\\{\}]
\PYG{o}{\PYGZpc{}}\PYG{k}{load\PYGZus{}ext} autoreload
\PYG{o}{\PYGZpc{}}\PYG{k}{autoreload} 2

\PYG{c+c1}{\PYGZsh{} First import pandas and the model into the  workspace}
\PYG{c+c1}{\PYGZsh{} There is no problem importing multiple times, though it is not very efficient.}
\PYG{k+kn}{import} \PYG{n+nn}{pandas} \PYG{k}{as} \PYG{n+nn}{pd}

\PYG{k+kn}{from} \PYG{n+nn}{modelclass} \PYG{k+kn}{import} \PYG{n}{model} 
\PYG{c+c1}{\PYGZsh{} functions that improve rendering of modelflow outputs under Jupyter Notebook}
\PYG{n}{model}\PYG{o}{.}\PYG{n}{widescreen}\PYG{p}{(}\PYG{p}{)}
\PYG{n}{model}\PYG{o}{.}\PYG{n}{scroll\PYGZus{}off}\PYG{p}{(}\PYG{p}{)}
\end{sphinxVerbatim}

\end{sphinxuseclass}\end{sphinxVerbatimInput}
\begin{sphinxVerbatimOutput}

\begin{sphinxuseclass}{cell_output}
\begin{sphinxVerbatim}[commandchars=\\\{\}]
\PYGZlt{}IPython.core.display.HTML object\PYGZgt{}
\end{sphinxVerbatim}

\end{sphinxuseclass}\end{sphinxVerbatimOutput}

\end{sphinxuseclass}
\sphinxAtStartPar
Now create a dataframe using standard pandas syntax.  In this instance with years as the index and a dictionary defining the variables and their data.

\begin{sphinxuseclass}{cell}\begin{sphinxVerbatimInput}

\begin{sphinxuseclass}{cell_input}
\begin{sphinxVerbatim}[commandchars=\\\{\}]
\PYG{c+c1}{\PYGZsh{} Create a dataframe using standard pandas}

\PYG{n}{df} \PYG{o}{=} \PYG{n}{pd}\PYG{o}{.}\PYG{n}{DataFrame}\PYG{p}{(}\PYG{p}{\PYGZob{}}\PYG{l+s+s1}{\PYGZsq{}}\PYG{l+s+s1}{B}\PYG{l+s+s1}{\PYGZsq{}}\PYG{p}{:} \PYG{p}{[}\PYG{l+m+mi}{1}\PYG{p}{,}\PYG{l+m+mi}{1}\PYG{p}{,}\PYG{l+m+mi}{1}\PYG{p}{,}\PYG{l+m+mi}{1}\PYG{p}{]}\PYG{p}{,}\PYG{l+s+s1}{\PYGZsq{}}\PYG{l+s+s1}{C}\PYG{l+s+s1}{\PYGZsq{}}\PYG{p}{:}\PYG{p}{[}\PYG{l+m+mi}{1}\PYG{p}{,}\PYG{l+m+mi}{2}\PYG{p}{,}\PYG{l+m+mi}{3}\PYG{p}{,}\PYG{l+m+mi}{6}\PYG{p}{]}\PYG{p}{,}\PYG{l+s+s1}{\PYGZsq{}}\PYG{l+s+s1}{E}\PYG{l+s+s1}{\PYGZsq{}}\PYG{p}{:}\PYG{p}{[}\PYG{l+m+mi}{4}\PYG{p}{,}\PYG{l+m+mi}{4}\PYG{p}{,}\PYG{l+m+mi}{4}\PYG{p}{,}\PYG{l+m+mi}{4}\PYG{p}{]}\PYG{p}{\PYGZcb{}}\PYG{p}{,}\PYG{n}{index}\PYG{o}{=}\PYG{p}{[}\PYG{l+m+mi}{2018}\PYG{p}{,}\PYG{l+m+mi}{2019}\PYG{p}{,}\PYG{l+m+mi}{2020}\PYG{p}{,}\PYG{l+m+mi}{2021}\PYG{p}{]}\PYG{p}{)}
\PYG{n}{df} 
\end{sphinxVerbatim}

\end{sphinxuseclass}\end{sphinxVerbatimInput}
\begin{sphinxVerbatimOutput}

\begin{sphinxuseclass}{cell_output}
\begin{sphinxVerbatim}[commandchars=\\\{\}]
      B  C  E
2018  1  1  4
2019  1  2  4
2020  1  3  4
2021  1  6  4
\end{sphinxVerbatim}

\end{sphinxuseclass}\end{sphinxVerbatimOutput}

\end{sphinxuseclass}
\sphinxAtStartPar
A somewhat more creative way to initialize the dataframe for dates would use a loop to specify the dates that get passed to the constructor as an argument.

\sphinxAtStartPar
Below a dataframe df with two Series (A and B), is initialized with the values 100 for all data points.

\sphinxAtStartPar
The index is defined dynamically by a loop \sphinxcode{\sphinxupquote{index={[}2020+v for v in range(number\_of\_rows){]}}} that runs for number\_of\_rows times (6 times in this example) setting v equal to 2020+0, 2020+1,…,202+5. The resulting list whose values are assigned to index is {[}2020,2021,2022,2023,2024,2025{]}.

\sphinxAtStartPar
The big advantage of this method is that if the user wanted to have data created for the period 1990 to 2030, they would only have to change number\_of\_rows from 6 to 41 and 2020 in the loop to 1990.

\sphinxAtStartPar
The second example simplifies further by just specifying the begin and end point of the range.

\begin{sphinxuseclass}{cell}\begin{sphinxVerbatimInput}

\begin{sphinxuseclass}{cell_input}
\begin{sphinxVerbatim}[commandchars=\\\{\}]
\PYG{c+c1}{\PYGZsh{}define the number of years for which the data is to be created.}
\PYG{n}{number\PYGZus{}of\PYGZus{}rows} \PYG{o}{=} \PYG{l+m+mi}{6} 

\PYG{c+c1}{\PYGZsh{} call the dataframe constructor}
\PYG{n}{df} \PYG{o}{=} \PYG{n}{pd}\PYG{o}{.}\PYG{n}{DataFrame}\PYG{p}{(}\PYG{l+m+mi}{100}\PYG{p}{,}
       \PYG{n}{index}\PYG{o}{=}\PYG{p}{[}\PYG{l+m+mi}{2020}\PYG{o}{+}\PYG{n}{v} \PYG{k}{for} \PYG{n}{v} \PYG{o+ow}{in} \PYG{n+nb}{range}\PYG{p}{(}\PYG{n}{number\PYGZus{}of\PYGZus{}rows}\PYG{p}{)}\PYG{p}{]}\PYG{p}{,} \PYG{c+c1}{\PYGZsh{} create row index}
       \PYG{c+c1}{\PYGZsh{} equivalent to index=[2020,2021,2022,2023,2024,2025] }
       \PYG{n}{columns}\PYG{o}{=}\PYG{p}{[}\PYG{l+s+s1}{\PYGZsq{}}\PYG{l+s+s1}{A}\PYG{l+s+s1}{\PYGZsq{}}\PYG{p}{,}\PYG{l+s+s1}{\PYGZsq{}}\PYG{l+s+s1}{B}\PYG{l+s+s1}{\PYGZsq{}}\PYG{p}{]}\PYG{p}{)}                                 \PYG{c+c1}{\PYGZsh{} create column name }
\PYG{n}{df}

\PYG{n}{df1} \PYG{o}{=} \PYG{n}{pd}\PYG{o}{.}\PYG{n}{DataFrame}\PYG{p}{(}\PYG{l+m+mi}{200}\PYG{p}{,}
       \PYG{n}{index}\PYG{o}{=}\PYG{p}{[}\PYG{n}{v} \PYG{k}{for} \PYG{n}{v} \PYG{o+ow}{in} \PYG{n+nb}{range}\PYG{p}{(}\PYG{l+m+mi}{2020}\PYG{p}{,}\PYG{l+m+mi}{2030}\PYG{p}{)}\PYG{p}{]}\PYG{p}{,} \PYG{c+c1}{\PYGZsh{} create row index}
       \PYG{c+c1}{\PYGZsh{} equivalent to index=[2020,2021,...,2030] }
       \PYG{n}{columns}\PYG{o}{=}\PYG{p}{[}\PYG{l+s+s1}{\PYGZsq{}}\PYG{l+s+s1}{A1}\PYG{l+s+s1}{\PYGZsq{}}\PYG{p}{,}\PYG{l+s+s1}{\PYGZsq{}}\PYG{l+s+s1}{B1}\PYG{l+s+s1}{\PYGZsq{}}\PYG{p}{]}\PYG{p}{)}                                 \PYG{c+c1}{\PYGZsh{} create column name }
\PYG{n}{df1}
\end{sphinxVerbatim}

\end{sphinxuseclass}\end{sphinxVerbatimInput}
\begin{sphinxVerbatimOutput}

\begin{sphinxuseclass}{cell_output}
\begin{sphinxVerbatim}[commandchars=\\\{\}]
       A1   B1
2020  200  200
2021  200  200
2022  200  200
2023  200  200
2024  200  200
2025  200  200
2026  200  200
2027  200  200
2028  200  200
2029  200  200
\end{sphinxVerbatim}

\end{sphinxuseclass}\end{sphinxVerbatimOutput}

\end{sphinxuseclass}

\subsection{Use .upd to create a new variable (= operator)}
\label{\detokenize{content/04_PythonEssentials/UpdateCommand:use-upd-to-create-a-new-variable-operator}}
\sphinxAtStartPar
With standard pandas a user can add a column (series) to a dataframe simply by assigning a adding to a dataframe.  For example:

\sphinxAtStartPar
\sphinxcode{\sphinxupquote{df{[}'NEW2'{]}={[}17,12,14,15{]}}}

\sphinxAtStartPar
\sphinxcode{\sphinxupquote{.upd()}} provides this functionality as well.

\begin{sphinxuseclass}{cell}\begin{sphinxVerbatimInput}

\begin{sphinxuseclass}{cell_input}
\begin{sphinxVerbatim}[commandchars=\\\{\}]
\PYG{n}{df2}\PYG{o}{=}\PYG{n}{df}\PYG{o}{.}\PYG{n}{upd}\PYG{p}{(}\PYG{l+s+s1}{\PYGZsq{}}\PYG{l+s+s1}{c = 142}\PYG{l+s+s1}{\PYGZsq{}}\PYG{p}{)} 
\PYG{n}{df2}
\end{sphinxVerbatim}

\end{sphinxuseclass}\end{sphinxVerbatimInput}
\begin{sphinxVerbatimOutput}

\begin{sphinxuseclass}{cell_output}
\begin{sphinxVerbatim}[commandchars=\\\{\}]
        A    B      C
2020  100  100  142.0
2021  100  100  142.0
2022  100  100  142.0
2023  100  100  142.0
2024  100  100  142.0
2025  100  100  142.0
\end{sphinxVerbatim}

\end{sphinxuseclass}\end{sphinxVerbatimOutput}

\end{sphinxuseclass}
\begin{sphinxadmonition}{note}{Note:}
\sphinxAtStartPar
Note that the new variable name was entered as a lower case ‘c’ here.  Lowercase letters are not legal \sphinxcode{\sphinxupquote{modelflow}} variable names.  The \sphinxcode{\sphinxupquote{.upd()}} method knows is part of modelflow and knows this rule, so it automatically translates lowercase entries into upper case so that the statement works.
\end{sphinxadmonition}


\subsection{Multiple updates and specific time periods}
\label{\detokenize{content/04_PythonEssentials/UpdateCommand:multiple-updates-and-specific-time-periods}}
\sphinxAtStartPar
The modelflow method \sphinxcode{\sphinxupquote{.upd()}} takes a string as an argument.  That string can contain a single update command or can contain multiple commands.

\sphinxAtStartPar
Moreover by including a <Begin End> date clause in a given update command, the update will be restricted to the associated time period.

\sphinxAtStartPar
The below illustrates this, modifying two existing variables A, B over different time periods and creating a new variable.

\begin{sphinxadmonition}{danger}{Danger:}
\sphinxAtStartPar
Note that the third line inherits the time period of the previous line.

\sphinxAtStartPar
Note also the submitted string can include comments as well (denoted with the standard python \# indicator).
\end{sphinxadmonition}

\begin{sphinxuseclass}{cell}\begin{sphinxVerbatimInput}

\begin{sphinxuseclass}{cell_input}
\begin{sphinxVerbatim}[commandchars=\\\{\}]
\PYG{n}{df}\PYG{o}{.}\PYG{n}{upd}\PYG{p}{(}\PYG{l+s+s2}{\PYGZdq{}\PYGZdq{}\PYGZdq{}}
\PYG{l+s+s2}{\PYGZsh{} Same number of values as years}
\PYG{l+s+s2}{\PYGZlt{}2021 2024\PYGZgt{} A = 42 44 45 46    \PYGZsh{} 4 years}
\PYG{l+s+s2}{\PYGZlt{}2020     \PYGZgt{} B = 200            \PYGZsh{} 1 year }
\PYG{l+s+s2}{c = 500                        \PYGZsh{} Same period as previous line}
\PYG{l+s+s2}{\PYGZlt{}\PYGZhy{}0 \PYGZhy{}1\PYGZgt{} D = 33                   \PYGZsh{} All years }
\PYG{l+s+s2}{\PYGZdq{}\PYGZdq{}\PYGZdq{}}\PYG{p}{)}
\end{sphinxVerbatim}

\end{sphinxuseclass}\end{sphinxVerbatimInput}
\begin{sphinxVerbatimOutput}

\begin{sphinxuseclass}{cell_output}
\begin{sphinxVerbatim}[commandchars=\\\{\}]
        A    B      C     D
2020  100  200  500.0  33.0
2021   42  100    0.0  33.0
2022   44  100    0.0  33.0
2023   45  100    0.0  33.0
2024   46  100    0.0  33.0
2025  100  100    0.0  33.0
\end{sphinxVerbatim}

\end{sphinxuseclass}\end{sphinxVerbatimOutput}

\end{sphinxuseclass}
\begin{sphinxShadowBox}
\sphinxstylesidebartitle{}

\sphinxAtStartPar
\sphinxstylestrong{Time scope of .upd()}

\sphinxAtStartPar
Made this a margin just to see

\sphinxAtStartPar
The update command takes a variety of mathematical operators \sphinxcode{\sphinxupquote{=, +, *, \% =GROWTH, +GROWTH, =DIFF}} and applies them to data for the period set in the leading <>.

\sphinxAtStartPar
If the user wants to modify a series or group of series for only a specific point in time or a period of time, she can indicate the period in the command line.
\begin{itemize}
\item {} 
\sphinxAtStartPar
If \sphinxstylestrong{one date} is specified the operation is applied to a single point in time

\item {} 
\sphinxAtStartPar
If \sphinxstylestrong{two dates}  are specifies the operation is applied over a period of time.

\end{itemize}

\sphinxAtStartPar
The selected time period will persist until re\sphinxhyphen{}set with a new time specification. Useful to avoid visual noise if several variables are going to be updated for the same time period.

\sphinxAtStartPar
The time period can be rest to the full time\sphinxhyphen{}period by using the special <\sphinxhyphen{}0 \sphinxhyphen{}1> time period.  More generally:
\begin{itemize}
\item {} 
\sphinxAtStartPar
Indicates the start of the dataframe use \sphinxhyphen{}0

\item {} 
\sphinxAtStartPar
Indicates the end of the dataframe use \sphinxhyphen{}1

\end{itemize}

\sphinxAtStartPar
If no time is provided the dataframe start and end period will be used.
\end{sphinxShadowBox}


\subsection{Setting specific datapoints to specific values}
\label{\detokenize{content/04_PythonEssentials/UpdateCommand:setting-specific-datapoints-to-specific-values}}
\sphinxAtStartPar
This example, demonstrates the equals operator.  The \sphinxcode{\sphinxupquote{=}} operator indicates that the variable a should be set equal to the indicated values following the \sphinxcode{\sphinxupquote{=}} operator (42 44 45 46 in the first line, 200 in the second and 500 inthe third). The dates enclosed in <> indicate the period over which the change should be applied.

\sphinxAtStartPar
Either:
\begin{itemize}
\item {} 
\sphinxAtStartPar
The number of data points provided must match the number of dates in the period, Or

\item {} 
\sphinxAtStartPar
Only one data point is provided, it is applied to all dates in the period.

\end{itemize}

\sphinxAtStartPar
If only one period is to be modified then it can be followed by just one date.

\sphinxAtStartPar
Note that the final line inherited the time period set in the second line.

\begin{sphinxuseclass}{cell}\begin{sphinxVerbatimInput}

\begin{sphinxuseclass}{cell_input}
\begin{sphinxVerbatim}[commandchars=\\\{\}]
\PYG{n}{df}\PYG{o}{.}\PYG{n}{upd}\PYG{p}{(}\PYG{l+s+s2}{\PYGZdq{}\PYGZdq{}\PYGZdq{}}
\PYG{l+s+s2}{\PYGZsh{} Same number of values as years}
\PYG{l+s+s2}{\PYGZlt{}2021 2024\PYGZgt{} A = 42 44 45 46    \PYGZsh{} 4 years}
\PYG{l+s+s2}{\PYGZlt{}2023     \PYGZgt{} B = 200            \PYGZsh{} 1 year }
\PYG{l+s+s2}{c = 500}
\PYG{l+s+s2}{\PYGZdq{}\PYGZdq{}\PYGZdq{}}\PYG{p}{)}
\end{sphinxVerbatim}

\end{sphinxuseclass}\end{sphinxVerbatimInput}
\begin{sphinxVerbatimOutput}

\begin{sphinxuseclass}{cell_output}
\begin{sphinxVerbatim}[commandchars=\\\{\}]
        A    B      C
2020  100  100    0.0
2021   42  100    0.0
2022   44  100    0.0
2023   45  200  500.0
2024   46  100    0.0
2025  100  100    0.0
\end{sphinxVerbatim}

\end{sphinxuseclass}\end{sphinxVerbatimOutput}

\end{sphinxuseclass}

\subsection{Adding  the specified  values to all values in a range (the + operator)}
\label{\detokenize{content/04_PythonEssentials/UpdateCommand:adding-the-specified-values-to-all-values-in-a-range-the-operator}}
\sphinxAtStartPar
NB: Here upd with the  + operator indicates that we are adding 42.

\begin{sphinxuseclass}{cell}\begin{sphinxVerbatimInput}

\begin{sphinxuseclass}{cell_input}
\begin{sphinxVerbatim}[commandchars=\\\{\}]
\PYG{n}{df}\PYG{o}{.}\PYG{n}{upd}\PYG{p}{(}\PYG{l+s+s1}{\PYGZsq{}\PYGZsq{}\PYGZsq{}}
\PYG{l+s+s1}{\PYGZsh{} Or one number to all years in between start and end }
\PYG{l+s+s1}{\PYGZlt{}2022 2024\PYGZgt{} B  +  42    \PYGZsh{} one value broadcast to 3 years }
\PYG{l+s+s1}{\PYGZsq{}\PYGZsq{}\PYGZsq{}}\PYG{p}{)}
\end{sphinxVerbatim}

\end{sphinxuseclass}\end{sphinxVerbatimInput}
\begin{sphinxVerbatimOutput}

\begin{sphinxuseclass}{cell_output}
\begin{sphinxVerbatim}[commandchars=\\\{\}]
        A    B
2020  100  100
2021  100  100
2022  100  142
2023  100  142
2024  100  142
2025  100  100
\end{sphinxVerbatim}

\end{sphinxuseclass}\end{sphinxVerbatimOutput}

\end{sphinxuseclass}

\subsection{Multiplying all values in a range by the specified values (the * operator)}
\label{\detokenize{content/04_PythonEssentials/UpdateCommand:multiplying-all-values-in-a-range-by-the-specified-values-the-operator}}
\begin{sphinxuseclass}{cell}\begin{sphinxVerbatimInput}

\begin{sphinxuseclass}{cell_input}
\begin{sphinxVerbatim}[commandchars=\\\{\}]
\PYG{n}{df}\PYG{o}{.}\PYG{n}{upd}\PYG{p}{(}\PYG{l+s+s1}{\PYGZsq{}\PYGZsq{}\PYGZsq{}}
\PYG{l+s+s1}{\PYGZsh{} Same number of values as years}
\PYG{l+s+s1}{\PYGZlt{}2021 2023\PYGZgt{} A *  42 44 55}
\PYG{l+s+s1}{\PYGZsq{}\PYGZsq{}\PYGZsq{}}\PYG{p}{)}
\end{sphinxVerbatim}

\end{sphinxuseclass}\end{sphinxVerbatimInput}
\begin{sphinxVerbatimOutput}

\begin{sphinxuseclass}{cell_output}
\begin{sphinxVerbatim}[commandchars=\\\{\}]
         A    B
2020   100  100
2021  4200  100
2022  4400  100
2023  5500  100
2024   100  100
2025   100  100
\end{sphinxVerbatim}

\end{sphinxuseclass}\end{sphinxVerbatimOutput}

\end{sphinxuseclass}

\subsection{Increasing all  values in a range by a  specified percent amount (the \% operator)}
\label{\detokenize{content/04_PythonEssentials/UpdateCommand:increasing-all-values-in-a-range-by-a-specified-percent-amount-the-operator}}
\sphinxAtStartPar
In this example:
\begin{itemize}
\item {} 
\sphinxAtStartPar
A is increased by 42 and 44\% over the range 2021 through 2022.

\item {} 
\sphinxAtStartPar
B is increased by 10 percent in all years

\item {} 
\sphinxAtStartPar
C is a new variable, is created and set to 100 for the whole range

\item {} 
\sphinxAtStartPar
C is decreased by 12 percent over the range 2023 through 2025.

\end{itemize}

\begin{sphinxuseclass}{cell}\begin{sphinxVerbatimInput}

\begin{sphinxuseclass}{cell_input}
\begin{sphinxVerbatim}[commandchars=\\\{\}]
\PYG{n}{df}\PYG{o}{.}\PYG{n}{upd}\PYG{p}{(}\PYG{l+s+s1}{\PYGZsq{}\PYGZsq{}\PYGZsq{}}
\PYG{l+s+s1}{\PYGZlt{}2021 2022 \PYGZgt{} A }\PYG{l+s+s1}{\PYGZpc{}}\PYG{l+s+s1}{  42 44   }
\PYG{l+s+s1}{\PYGZlt{}\PYGZhy{}0 \PYGZhy{}1\PYGZgt{} B }\PYG{l+s+s1}{\PYGZpc{}}\PYG{l+s+s1}{ 10            \PYGZsh{} all rows }
\PYG{l+s+s1}{C = 100                   \PYGZsh{} all rows persist }
\PYG{l+s+s1}{\PYGZlt{}2023 2025\PYGZgt{} C }\PYG{l+s+s1}{\PYGZpc{}}\PYG{l+s+s1}{ \PYGZhy{}12       \PYGZsh{} now only for 3 years }
\PYG{l+s+s1}{\PYGZsq{}\PYGZsq{}\PYGZsq{}}\PYG{p}{)}
\end{sphinxVerbatim}

\end{sphinxuseclass}\end{sphinxVerbatimInput}
\begin{sphinxVerbatimOutput}

\begin{sphinxuseclass}{cell_output}
\begin{sphinxVerbatim}[commandchars=\\\{\}]
        A      B      C
2020  100  110.0  100.0
2021  142  110.0  100.0
2022  144  110.0  100.0
2023  100  110.0   88.0
2024  100  110.0   88.0
2025  100  110.0   88.0
\end{sphinxVerbatim}

\end{sphinxuseclass}\end{sphinxVerbatimOutput}

\end{sphinxuseclass}

\subsection{Set the percent growth rate to specified values (=GROWTH)}
\label{\detokenize{content/04_PythonEssentials/UpdateCommand:set-the-percent-growth-rate-to-specified-values-growth}}
\begin{sphinxuseclass}{cell}\begin{sphinxVerbatimInput}

\begin{sphinxuseclass}{cell_input}
\begin{sphinxVerbatim}[commandchars=\\\{\}]
\PYG{n}{res} \PYG{o}{=} \PYG{n}{df}\PYG{o}{.}\PYG{n}{upd}\PYG{p}{(}\PYG{l+s+s1}{\PYGZsq{}\PYGZsq{}\PYGZsq{}}
\PYG{l+s+s1}{\PYGZsh{} Same number of values as years}
\PYG{l+s+s1}{\PYGZlt{}2021 2022\PYGZgt{} A =GROWTH  1 5  }
\PYG{l+s+s1}{\PYGZlt{}2020\PYGZgt{} c = 100 }
\PYG{l+s+s1}{\PYGZlt{}2021 2025\PYGZgt{} c =GROWTH 2 }
\PYG{l+s+s1}{\PYGZsq{}\PYGZsq{}\PYGZsq{}}\PYG{p}{)}
\PYG{n+nb}{print}\PYG{p}{(}\PYG{l+s+sa}{f}\PYG{l+s+s1}{\PYGZsq{}}\PYG{l+s+s1}{Dataframe:}\PYG{l+s+se}{\PYGZbs{}n}\PYG{l+s+si}{\PYGZob{}}\PYG{n}{res}\PYG{l+s+si}{\PYGZcb{}}\PYG{l+s+se}{\PYGZbs{}n}\PYG{l+s+se}{\PYGZbs{}n}\PYG{l+s+s1}{Growth:}\PYG{l+s+se}{\PYGZbs{}n}\PYG{l+s+si}{\PYGZob{}}\PYG{n}{res}\PYG{o}{.}\PYG{n}{pct\PYGZus{}change}\PYG{p}{(}\PYG{p}{)}\PYG{o}{*}\PYG{l+m+mi}{100}\PYG{l+s+si}{\PYGZcb{}}\PYG{l+s+se}{\PYGZbs{}n}\PYG{l+s+s1}{\PYGZsq{}}\PYG{p}{)} \PYG{c+c1}{\PYGZsh{} Explained b}
\end{sphinxVerbatim}

\end{sphinxuseclass}\end{sphinxVerbatimInput}
\begin{sphinxVerbatimOutput}

\begin{sphinxuseclass}{cell_output}
\begin{sphinxVerbatim}[commandchars=\\\{\}]
Dataframe:
           A    B           C
2020  100.00  100  100.000000
2021  101.00  100  102.000000
2022  106.05  100  104.040000
2023  100.00  100  106.120800
2024  100.00  100  108.243216
2025  100.00  100  110.408080

Growth:
             A    B    C
2020       NaN  NaN  NaN
2021  1.000000  0.0  2.0
2022  5.000000  0.0  2.0
2023 \PYGZhy{}5.704856  0.0  2.0
2024  0.000000  0.0  2.0
2025  0.000000  0.0  2.0
\end{sphinxVerbatim}

\end{sphinxuseclass}\end{sphinxVerbatimOutput}

\end{sphinxuseclass}

\subsection{Add or subtract from the existing percent growth rate (+GROWTH operator)}
\label{\detokenize{content/04_PythonEssentials/UpdateCommand:add-or-subtract-from-the-existing-percent-growth-rate-growth-operator}}
\sphinxAtStartPar
The below example is a bit more complicated.

\sphinxAtStartPar
The first line sets the growth rate of A to 1\% in all periods beginning in  2021

\sphinxAtStartPar
The second command adds 2 3 4 5 6 to the growth rates in each period after 2021, resulting in growth rates of 3,4,5,6,7.

\begin{sphinxuseclass}{cell}\begin{sphinxVerbatimInput}

\begin{sphinxuseclass}{cell_input}
\begin{sphinxVerbatim}[commandchars=\\\{\}]
\PYG{n}{res} \PYG{o}{=}\PYG{n}{df}\PYG{o}{.}\PYG{n}{upd}\PYG{p}{(}\PYG{l+s+s1}{\PYGZsq{}\PYGZsq{}\PYGZsq{}}
\PYG{l+s+s1}{\PYGZlt{}2021 \PYGZgt{} A =GROWTH  1  \PYGZsh{} All selected years set to the same growth rate}
\PYG{l+s+s1}{a +growth   2  \PYGZsh{} Add to the existing growth rate these numbers  }
\PYG{l+s+s1}{\PYGZsq{}\PYGZsq{}\PYGZsq{}}\PYG{p}{)}
\PYG{n+nb}{print}\PYG{p}{(}\PYG{l+s+sa}{f}\PYG{l+s+s1}{\PYGZsq{}}\PYG{l+s+s1}{Dataframe:}\PYG{l+s+se}{\PYGZbs{}n}\PYG{l+s+si}{\PYGZob{}}\PYG{n}{res}\PYG{l+s+si}{\PYGZcb{}}\PYG{l+s+se}{\PYGZbs{}n}\PYG{l+s+se}{\PYGZbs{}n}\PYG{l+s+s1}{Growth:}\PYG{l+s+se}{\PYGZbs{}n}\PYG{l+s+si}{\PYGZob{}}\PYG{n}{res}\PYG{o}{.}\PYG{n}{pct\PYGZus{}change}\PYG{p}{(}\PYG{p}{)}\PYG{o}{*}\PYG{l+m+mi}{100}\PYG{l+s+si}{\PYGZcb{}}\PYG{l+s+se}{\PYGZbs{}n}\PYG{l+s+s1}{\PYGZsq{}}\PYG{p}{)}
\end{sphinxVerbatim}

\end{sphinxuseclass}\end{sphinxVerbatimInput}
\begin{sphinxVerbatimOutput}

\begin{sphinxuseclass}{cell_output}
\begin{sphinxVerbatim}[commandchars=\\\{\}]
Dataframe:
        A    B
2020  100  100
2021  103  100
2022  100  100
2023  100  100
2024  100  100
2025  100  100

Growth:
             A    B
2020       NaN  NaN
2021  3.000000  0.0
2022 \PYGZhy{}2.912621  0.0
2023  0.000000  0.0
2024  0.000000  0.0
2025  0.000000  0.0
\end{sphinxVerbatim}

\end{sphinxuseclass}\end{sphinxVerbatimOutput}

\end{sphinxuseclass}

\subsection{Set the change in a variable to specific values (=diff operator)}
\label{\detokenize{content/04_PythonEssentials/UpdateCommand:set-the-change-in-a-variable-to-specific-values-diff-operator}}
\sphinxAtStartPar
\(\Delta = var_t - var_{t-1} = some number\)

\sphinxAtStartPar
Here sets the value of A in 2021 to 2 more than the value of 2020, and the 2022 value as 4 more than the \sphinxstylestrong{revised} value of 2021.

\sphinxAtStartPar
The second line creates a new variable “UPBY2” to the data frame and sets it equal to 100 for all periods,

\sphinxAtStartPar
The third line adds 2 to the previous periods value UPBY2.

\begin{sphinxuseclass}{cell}\begin{sphinxVerbatimInput}

\begin{sphinxuseclass}{cell_input}
\begin{sphinxVerbatim}[commandchars=\\\{\}]
\PYG{n}{df}\PYG{o}{.}\PYG{n}{upd}\PYG{p}{(}\PYG{l+s+s1}{\PYGZsq{}\PYGZsq{}\PYGZsq{}}
\PYG{l+s+s1}{\PYGZlt{} 2021 2022\PYGZgt{} A =diff  2 4   \PYGZsh{} Same number of values as years}
\PYG{l+s+s1}{\PYGZlt{}2020 \PYGZgt{} UpBy2 = 100 \PYGZsh{} sets rows equal to the same  number for all years in between start and end }
\PYG{l+s+s1}{\PYGZlt{}2021 2025\PYGZgt{} UpBy2 =diff  2  }

\PYG{l+s+s1}{\PYGZsq{}\PYGZsq{}\PYGZsq{}}\PYG{p}{)}
\end{sphinxVerbatim}

\end{sphinxuseclass}\end{sphinxVerbatimInput}
\begin{sphinxVerbatimOutput}

\begin{sphinxuseclass}{cell_output}
\begin{sphinxVerbatim}[commandchars=\\\{\}]
        A    B  UPBY2
2020  100  100  100.0
2021  102  100  102.0
2022  106  100  104.0
2023  100  100  106.0
2024  100  100  108.0
2025  100  100  110.0
\end{sphinxVerbatim}

\end{sphinxuseclass}\end{sphinxVerbatimOutput}

\end{sphinxuseclass}

\subsection{Recall  that we have not overwritten df, so the df dataframe is unchanged.}
\label{\detokenize{content/04_PythonEssentials/UpdateCommand:recall-that-we-have-not-overwritten-df-so-the-df-dataframe-is-unchanged}}
\begin{sphinxuseclass}{cell}\begin{sphinxVerbatimInput}

\begin{sphinxuseclass}{cell_input}
\begin{sphinxVerbatim}[commandchars=\\\{\}]
\PYG{n}{df}
\end{sphinxVerbatim}

\end{sphinxuseclass}\end{sphinxVerbatimInput}
\begin{sphinxVerbatimOutput}

\begin{sphinxuseclass}{cell_output}
\begin{sphinxVerbatim}[commandchars=\\\{\}]
        A    B
2020  100  100
2021  100  100
2022  100  100
2023  100  100
2024  100  100
2025  100  100
\end{sphinxVerbatim}

\end{sphinxuseclass}\end{sphinxVerbatimOutput}

\end{sphinxuseclass}
\begin{sphinxadmonition}{note}{Note:}
\sphinxAtStartPar
The method \sphinxcode{\sphinxupquote{.upd()}} only operates on on variable.  A command like \sphinxcode{\sphinxupquote{.upd('A = B')}} would not work. For these kind of functions, use \sphinxcode{\sphinxupquote{.mfcalc()}} (see next section).
\end{sphinxadmonition}


\subsection{Keep growth rates after the update time – the –kg option}
\label{\detokenize{content/04_PythonEssentials/UpdateCommand:keep-growth-rates-after-the-update-time-the-kg-option}}
\sphinxAtStartPar
In a long projection it can sometime be useful to be able to update variables for which new information is available, but for the subsequent periods keep the growth rate the same as before the update. In database management this is frequently done when two time\sphinxhyphen{}series with different levels are spliced together.

\sphinxAtStartPar
The \sphinxhyphen{}kg or –keep\_growth option instructs modelview to calculate the growth rate of the existing pre\sphinxhyphen{}change series, and then use it to preserve the pre\sphinxhyphen{}change growth rates of the series for the periods that were \sphinxstylestrong{not} changed.

\sphinxAtStartPar
This allows to update variables for which new information is available, but keep the growth rate the same as before the update in the period after the update time.


\subsubsection{The default keep\_growth behaviour}
\label{\detokenize{content/04_PythonEssentials/UpdateCommand:the-default-keep-growth-behaviour}}
\sphinxAtStartPar
The \sphinxcode{\sphinxupquote{upd()}} method has a parameter \sphinxcode{\sphinxupquote{keep\_growth}}, which by default is equal to \sphinxcode{\sphinxupquote{False}}.

\sphinxAtStartPar
\sphinxcode{\sphinxupquote{keep\_growth}} determines how data in  the time periods after those where an update is executed are treated.

\sphinxAtStartPar
If \sphinxcode{\sphinxupquote{keep\_growth}} is \sphinxcode{\sphinxupquote{False}} then data in the sub\sphinxhyphen{}period after a change is left unchanged.

\sphinxAtStartPar
if \sphinxcode{\sphinxupquote{keep\_growth}} is set to “\sphinxcode{\sphinxupquote{True}}” then the system will preserve the pre\sphinxhyphen{}change growth rate of the affected variable in the time period \sphinxstyleemphasis{after the change}.

\begin{sphinxadmonition}{note}{Note:}
\sphinxAtStartPar
At the line level:
\begin{itemize}
\item {} 
\sphinxAtStartPar
\sphinxcode{\sphinxupquote{keep\_growth=True}} can be expressed as –kg

\item {} 
\sphinxAtStartPar
\sphinxcode{\sphinxupquote{keep\_growth=False}} can be expressed as –nkg

\end{itemize}
\end{sphinxadmonition}

\sphinxAtStartPar
Let’s see this in a concrete example.  Consider the following \sphinxcode{\sphinxupquote{dataframe}} df with two variables A and B, that each grow by 2\% per period, with A initialized at a level of 100 and B at a level of 110 so that we can see each separately on a graph.

\begin{sphinxuseclass}{cell}\begin{sphinxVerbatimInput}

\begin{sphinxuseclass}{cell_input}
\begin{sphinxVerbatim}[commandchars=\\\{\}]
\PYG{n}{df} \PYG{o}{=} \PYG{n}{pd}\PYG{o}{.}\PYG{n}{DataFrame}\PYG{p}{(}\PYG{l+m+mi}{100}\PYG{p}{,}
       \PYG{n}{index}\PYG{o}{=}\PYG{p}{[}\PYG{l+m+mi}{2020}\PYG{o}{+}\PYG{n}{v} \PYG{k}{for} \PYG{n}{v} \PYG{o+ow}{in} \PYG{n+nb}{range}\PYG{p}{(}\PYG{n}{number\PYGZus{}of\PYGZus{}rows}\PYG{p}{)}\PYG{p}{]}\PYG{p}{,} \PYG{c+c1}{\PYGZsh{} create row index}
       \PYG{c+c1}{\PYGZsh{} equivalent to index=[2020,2021,2022,2023,2024,2025] }
       \PYG{n}{columns}\PYG{o}{=}\PYG{p}{[}\PYG{l+s+s1}{\PYGZsq{}}\PYG{l+s+s1}{A}\PYG{l+s+s1}{\PYGZsq{}}\PYG{p}{,}\PYG{l+s+s1}{\PYGZsq{}}\PYG{l+s+s1}{B}\PYG{l+s+s1}{\PYGZsq{}}\PYG{p}{]}\PYG{p}{)} 

\PYG{n}{df}\PYG{o}{=}\PYG{n}{df}\PYG{o}{.}\PYG{n}{upd}\PYG{p}{(}\PYG{l+s+s2}{\PYGZdq{}\PYGZdq{}\PYGZdq{}}\PYG{l+s+s2}{\PYGZlt{}2021 \PYGZhy{}1\PYGZgt{} A =growth 2}
\PYG{l+s+s2}{           \PYGZlt{}2020 \PYGZhy{}1\PYGZgt{}   B = 110}
\PYG{l+s+s2}{          \PYGZlt{}2021 \PYGZhy{}1\PYGZgt{}    B =growth 2}
\PYG{l+s+s2}{          }\PYG{l+s+s2}{\PYGZdq{}\PYGZdq{}\PYGZdq{}}\PYG{p}{)}
\PYG{c+c1}{\PYGZsh{} Store these variables for later use in comparisons}
\PYG{n}{df}\PYG{p}{[}\PYG{l+s+s1}{\PYGZsq{}}\PYG{l+s+s1}{A\PYGZus{}ORIG}\PYG{l+s+s1}{\PYGZsq{}}\PYG{p}{]}\PYG{o}{=}\PYG{n}{df}\PYG{p}{[}\PYG{l+s+s1}{\PYGZsq{}}\PYG{l+s+s1}{A}\PYG{l+s+s1}{\PYGZsq{}}\PYG{p}{]}
\PYG{n}{df}\PYG{p}{[}\PYG{l+s+s1}{\PYGZsq{}}\PYG{l+s+s1}{B\PYGZus{}ORIG}\PYG{l+s+s1}{\PYGZsq{}}\PYG{p}{]}\PYG{o}{=}\PYG{n}{df}\PYG{p}{[}\PYG{l+s+s1}{\PYGZsq{}}\PYG{l+s+s1}{B}\PYG{l+s+s1}{\PYGZsq{}}\PYG{p}{]}
\PYG{n}{df}
\end{sphinxVerbatim}

\end{sphinxuseclass}\end{sphinxVerbatimInput}
\begin{sphinxVerbatimOutput}

\begin{sphinxuseclass}{cell_output}
\begin{sphinxVerbatim}[commandchars=\\\{\}]
               A           B      A\PYGZus{}ORIG      B\PYGZus{}ORIG
2020  100.000000  110.000000  100.000000  110.000000
2021  102.000000  112.200000  102.000000  112.200000
2022  104.040000  114.444000  104.040000  114.444000
2023  106.120800  116.732880  106.120800  116.732880
2024  108.243216  119.067538  108.243216  119.067538
2025  110.408080  121.448888  110.408080  121.448888
\end{sphinxVerbatim}

\end{sphinxuseclass}\end{sphinxVerbatimOutput}

\end{sphinxuseclass}
\begin{sphinxuseclass}{cell}\begin{sphinxVerbatimInput}

\begin{sphinxuseclass}{cell_input}
\begin{sphinxVerbatim}[commandchars=\\\{\}]
\PYG{n}{df}\PYG{p}{[}\PYG{p}{[}\PYG{l+s+s1}{\PYGZsq{}}\PYG{l+s+s1}{A}\PYG{l+s+s1}{\PYGZsq{}}\PYG{p}{,}\PYG{l+s+s1}{\PYGZsq{}}\PYG{l+s+s1}{B}\PYG{l+s+s1}{\PYGZsq{}}\PYG{p}{]}\PYG{p}{]}\PYG{o}{.}\PYG{n}{plot}\PYG{p}{(}\PYG{p}{)}
\end{sphinxVerbatim}

\end{sphinxuseclass}\end{sphinxVerbatimInput}
\begin{sphinxVerbatimOutput}

\begin{sphinxuseclass}{cell_output}
\begin{sphinxVerbatim}[commandchars=\\\{\}]
\PYGZlt{}AxesSubplot: \PYGZgt{}
\end{sphinxVerbatim}

\noindent\sphinxincludegraphics{{UpdateCommand_34_1}.png}

\end{sphinxuseclass}\end{sphinxVerbatimOutput}

\end{sphinxuseclass}
\sphinxAtStartPar
Now lets modify each by adding 5 to the level in 2022 and 2023.  For B we will do setting the keep\_growth option as False and for ‘B’ keep\_growth positive.  While the keep\_growth is a global variable it can be set at the line level also using the –kg option (\sphinxcode{\sphinxupquote{keep\_growth=True}}) and –nkg option (\sphinxcode{\sphinxupquote{keep\_growth=False}}).

\begin{sphinxuseclass}{cell}\begin{sphinxVerbatimInput}

\begin{sphinxuseclass}{cell_input}
\begin{sphinxVerbatim}[commandchars=\\\{\}]
\PYG{n}{df}\PYG{o}{=}\PYG{n}{df}\PYG{o}{.}\PYG{n}{upd}\PYG{p}{(}\PYG{l+s+s2}{\PYGZdq{}\PYGZdq{}\PYGZdq{}}
\PYG{l+s+s2}{            \PYGZlt{}2022 2023\PYGZgt{} A + 5 \PYGZhy{}\PYGZhy{}kg}
\PYG{l+s+s2}{            \PYGZlt{}2022 2023\PYGZgt{} B + 5 \PYGZhy{}\PYGZhy{}nkg}
\PYG{l+s+s2}{            }\PYG{l+s+s2}{\PYGZdq{}\PYGZdq{}\PYGZdq{}}\PYG{p}{)}

\PYG{n}{df}\PYG{p}{[}\PYG{p}{[}\PYG{l+s+s1}{\PYGZsq{}}\PYG{l+s+s1}{A}\PYG{l+s+s1}{\PYGZsq{}}\PYG{p}{,}\PYG{l+s+s1}{\PYGZsq{}}\PYG{l+s+s1}{B}\PYG{l+s+s1}{\PYGZsq{}}\PYG{p}{,}\PYG{l+s+s1}{\PYGZsq{}}\PYG{l+s+s1}{A\PYGZus{}ORIG}\PYG{l+s+s1}{\PYGZsq{}}\PYG{p}{,}\PYG{l+s+s1}{\PYGZsq{}}\PYG{l+s+s1}{B\PYGZus{}ORIG}\PYG{l+s+s1}{\PYGZsq{}}\PYG{p}{]}\PYG{p}{]}\PYG{o}{.}\PYG{n}{plot}\PYG{p}{(}\PYG{p}{)}
    
\end{sphinxVerbatim}

\end{sphinxuseclass}\end{sphinxVerbatimInput}
\begin{sphinxVerbatimOutput}

\begin{sphinxuseclass}{cell_output}
\begin{sphinxVerbatim}[commandchars=\\\{\}]
\PYGZlt{}AxesSubplot: \PYGZgt{}
\end{sphinxVerbatim}

\noindent\sphinxincludegraphics{{UpdateCommand_36_1}.png}

\end{sphinxuseclass}\end{sphinxVerbatimOutput}

\end{sphinxuseclass}
\sphinxAtStartPar
In the first example ‘A’ (the green and blue lines) the level of A is increased by 5 for two periods (2021\sphinxhyphen{}2022). The subsequent values are also increased and they were calculated to maintain the growth rate of the original series.

\sphinxAtStartPar
For the ‘B’ variable the same level change was input but because of the \sphinxcode{\sphinxupquote{\sphinxhyphen{}\sphinxhyphen{}nkg}} (equivalent to \sphinxcode{\sphinxupquote{keep\_growth=False}}) the periods after the change were unaffected and retained their old values.

\sphinxAtStartPar
Below are plots the growth rates of the two transformed series.

\sphinxAtStartPar
Here the growth in both series accelerates in 2022, by slightly less than 5 percentage points because a) the base of each is more than 100, with the base of B being higher (it was initialized at 110). In 2023 the growth rate of A returns to 2 percent, while the growth rate of B is actually negative because the level (see earlier graph) has fallen back to its original level.

\begin{sphinxuseclass}{cell}\begin{sphinxVerbatimInput}

\begin{sphinxuseclass}{cell_input}
\begin{sphinxVerbatim}[commandchars=\\\{\}]
\PYG{n}{dfg}\PYG{o}{=}\PYG{n}{df}\PYG{p}{[}\PYG{p}{[}\PYG{l+s+s1}{\PYGZsq{}}\PYG{l+s+s1}{A}\PYG{l+s+s1}{\PYGZsq{}}\PYG{p}{,}\PYG{l+s+s1}{\PYGZsq{}}\PYG{l+s+s1}{B}\PYG{l+s+s1}{\PYGZsq{}}\PYG{p}{]}\PYG{p}{]}\PYG{o}{.}\PYG{n}{pct\PYGZus{}change}\PYG{p}{(}\PYG{p}{)}\PYG{o}{*}\PYG{l+m+mi}{100}
\PYG{n}{dfg}\PYG{o}{.}\PYG{n}{plot}\PYG{p}{(}\PYG{p}{)}
\end{sphinxVerbatim}

\end{sphinxuseclass}\end{sphinxVerbatimInput}
\begin{sphinxVerbatimOutput}

\begin{sphinxuseclass}{cell_output}
\begin{sphinxVerbatim}[commandchars=\\\{\}]
\PYGZlt{}AxesSubplot: \PYGZgt{}
\end{sphinxVerbatim}

\noindent\sphinxincludegraphics{{UpdateCommand_38_1}.png}

\end{sphinxuseclass}\end{sphinxVerbatimOutput}

\end{sphinxuseclass}

\subsection{.upd(,,,keep\_growth) some more examples}
\label{\detokenize{content/04_PythonEssentials/UpdateCommand:upd-keep-growth-some-more-examples}}

\subsection{Initialize a new dataframe First make a dataframe with some growth rate}
\label{\detokenize{content/04_PythonEssentials/UpdateCommand:initialize-a-new-dataframe-first-make-a-dataframe-with-some-growth-rate}}
\begin{sphinxuseclass}{cell}\begin{sphinxVerbatimInput}

\begin{sphinxuseclass}{cell_input}
\begin{sphinxVerbatim}[commandchars=\\\{\}]
\PYG{c+c1}{\PYGZsh{} instantiate a new dataframe with one column \PYGZsq{}A\PYGZsq{} with avlue 100 everywhere and index 2020\PYGZhy{}2025}
\PYG{n}{dftest} \PYG{o}{=} \PYG{n}{pd}\PYG{o}{.}\PYG{n}{DataFrame}\PYG{p}{(}\PYG{l+m+mi}{100}\PYG{p}{,}
       \PYG{n}{index}\PYG{o}{=}\PYG{p}{[}\PYG{l+m+mi}{2020}\PYG{o}{+}\PYG{n}{v} \PYG{k}{for} \PYG{n}{v} \PYG{o+ow}{in} \PYG{n+nb}{range}\PYG{p}{(}\PYG{n}{number\PYGZus{}of\PYGZus{}rows}\PYG{p}{)}\PYG{p}{]}\PYG{p}{,} \PYG{c+c1}{\PYGZsh{} create row index}
       \PYG{c+c1}{\PYGZsh{} equivalent to index=[2020,2021,2022,2023,2024,2025] }
       \PYG{n}{columns}\PYG{o}{=}\PYG{p}{[}\PYG{l+s+s1}{\PYGZsq{}}\PYG{l+s+s1}{A}\PYG{l+s+s1}{\PYGZsq{}}\PYG{p}{]}\PYG{p}{)}                                 \PYG{c+c1}{\PYGZsh{} create column name }

\PYG{c+c1}{\PYGZsh{} Update a to have growth rate accelerationg linearly by 1 from 1 oercent to 5 percent}
\PYG{n}{original} \PYG{o}{=} \PYG{n}{dftest}\PYG{o}{.}\PYG{n}{upd}\PYG{p}{(}\PYG{l+s+s1}{\PYGZsq{}}\PYG{l+s+s1}{\PYGZlt{}2021 2025\PYGZgt{} a =growth 1 2 3 4 5}\PYG{l+s+s1}{\PYGZsq{}}\PYG{p}{)}  
\PYG{n+nb}{print}\PYG{p}{(}\PYG{l+s+sa}{f}\PYG{l+s+s1}{\PYGZsq{}}\PYG{l+s+s1}{Levels:}\PYG{l+s+se}{\PYGZbs{}n}\PYG{l+s+si}{\PYGZob{}}\PYG{n}{original}\PYG{l+s+si}{\PYGZcb{}}\PYG{l+s+se}{\PYGZbs{}n}\PYG{l+s+se}{\PYGZbs{}n}\PYG{l+s+s1}{Growth:}\PYG{l+s+se}{\PYGZbs{}n}\PYG{l+s+si}{\PYGZob{}}\PYG{n}{original}\PYG{o}{.}\PYG{n}{pct\PYGZus{}change}\PYG{p}{(}\PYG{p}{)}\PYG{o}{*}\PYG{l+m+mi}{100}\PYG{l+s+si}{\PYGZcb{}}\PYG{l+s+se}{\PYGZbs{}n}\PYG{l+s+s1}{\PYGZsq{}}\PYG{p}{)}
\end{sphinxVerbatim}

\end{sphinxuseclass}\end{sphinxVerbatimInput}
\begin{sphinxVerbatimOutput}

\begin{sphinxuseclass}{cell_output}
\begin{sphinxVerbatim}[commandchars=\\\{\}]
Levels:
               A
2020  100.000000
2021  101.000000
2022  103.020000
2023  106.110600
2024  110.355024
2025  115.872775

Growth:
        A
2020  NaN
2021  1.0
2022  2.0
2023  3.0
2024  4.0
2025  5.0
\end{sphinxVerbatim}

\end{sphinxuseclass}\end{sphinxVerbatimOutput}

\end{sphinxuseclass}

\subsection{now update A in 2021 to 2023 to a new value}
\label{\detokenize{content/04_PythonEssentials/UpdateCommand:now-update-a-in-2021-to-2023-to-a-new-value}}
\sphinxAtStartPar
Below performs the same operation, the first time the updated value is assigned to the \sphinxcode{\sphinxupquote{dataframe}} \sphinxcode{\sphinxupquote{nkg}} and the default behaviour of \sphinxcode{\sphinxupquote{keep\_growth}} is \sphinxcode{\sphinxupquote{False}}

\sphinxAtStartPar
In the second example the \sphinxcode{\sphinxupquote{\sphinxhyphen{}kg}} line option is specified, telling modelflow to maintain the growth rates of the dependent variable in the periods after the update is executed.

\begin{sphinxuseclass}{cell}\begin{sphinxVerbatimInput}

\begin{sphinxuseclass}{cell_input}
\begin{sphinxVerbatim}[commandchars=\\\{\}]
\PYG{n}{nokg} \PYG{o}{=} \PYG{n}{original}\PYG{o}{.}\PYG{n}{upd}\PYG{p}{(}\PYG{l+s+s1}{\PYGZsq{}\PYGZsq{}\PYGZsq{}}
\PYG{l+s+s1}{\PYGZlt{}2021 2025\PYGZgt{}  a =growth 1 2 3 4 5 }
\PYG{l+s+s1}{\PYGZlt{}2021 2023\PYGZgt{}  a = 120  }
\PYG{l+s+s1}{\PYGZsq{}\PYGZsq{}\PYGZsq{}}\PYG{p}{,}\PYG{n}{lprint}\PYG{o}{=}\PYG{l+m+mi}{0}\PYG{p}{)}

\PYG{n}{kg} \PYG{o}{=} \PYG{n}{original}\PYG{o}{.}\PYG{n}{upd}\PYG{p}{(}\PYG{l+s+s1}{\PYGZsq{}\PYGZsq{}\PYGZsq{}}
\PYG{l+s+s1}{\PYGZlt{}2021 2025\PYGZgt{}  a =growth 1 2 3 4 5 }
\PYG{l+s+s1}{\PYGZlt{}2021 2023\PYGZgt{}  a = 120  \PYGZhy{}\PYGZhy{}kg}
\PYG{l+s+s1}{\PYGZsq{}\PYGZsq{}\PYGZsq{}}\PYG{p}{,}\PYG{n}{lprint}\PYG{o}{=}\PYG{l+m+mi}{0}\PYG{p}{)}


\PYG{n}{kg}\PYG{o}{=}\PYG{n}{kg}\PYG{o}{.}\PYG{n}{rename}\PYG{p}{(}\PYG{n}{columns}\PYG{o}{=}\PYG{p}{\PYGZob{}}\PYG{l+s+s2}{\PYGZdq{}}\PYG{l+s+s2}{A}\PYG{l+s+s2}{\PYGZdq{}}\PYG{p}{:}\PYG{l+s+s2}{\PYGZdq{}}\PYG{l+s+s2}{KG}\PYG{l+s+s2}{\PYGZdq{}}\PYG{p}{\PYGZcb{}}\PYG{p}{)}       \PYG{c+c1}{\PYGZsh{}rename cols to facilitate display}
\PYG{n}{nokg}\PYG{o}{=}\PYG{n}{nokg}\PYG{o}{.}\PYG{n}{rename}\PYG{p}{(}\PYG{n}{columns}\PYG{o}{=}\PYG{p}{\PYGZob{}}\PYG{l+s+s2}{\PYGZdq{}}\PYG{l+s+s2}{A}\PYG{l+s+s2}{\PYGZdq{}}\PYG{p}{:}\PYG{l+s+s2}{\PYGZdq{}}\PYG{l+s+s2}{NOKG}\PYG{l+s+s2}{\PYGZdq{}}\PYG{p}{\PYGZcb{}}\PYG{p}{)} \PYG{c+c1}{\PYGZsh{}rename cols to facilitate display}

\PYG{n}{combo}\PYG{o}{=}\PYG{n}{pd}\PYG{o}{.}\PYG{n}{concat}\PYG{p}{(}\PYG{p}{[}\PYG{n}{kg}\PYG{p}{,}\PYG{n}{nokg}\PYG{p}{]}\PYG{p}{,} \PYG{n}{axis}\PYG{o}{=}\PYG{l+m+mi}{1}\PYG{p}{)}
\PYG{n}{combo}


\PYG{n+nb}{print}\PYG{p}{(}\PYG{l+s+sa}{f}\PYG{l+s+s1}{\PYGZsq{}}\PYG{l+s+s1}{Levels}\PYG{l+s+se}{\PYGZbs{}n}\PYG{l+s+si}{\PYGZob{}}\PYG{n}{combo}\PYG{l+s+si}{\PYGZcb{}}\PYG{l+s+se}{\PYGZbs{}n}\PYG{l+s+se}{\PYGZbs{}n}\PYG{l+s+s1}{Growth}\PYG{l+s+se}{\PYGZbs{}n}\PYG{l+s+si}{\PYGZob{}}\PYG{n}{combo}\PYG{o}{.}\PYG{n}{pct\PYGZus{}change}\PYG{p}{(}\PYG{p}{)}\PYG{o}{*}\PYG{l+m+mi}{100}\PYG{l+s+si}{\PYGZcb{}}\PYG{l+s+s1}{\PYGZsq{}}\PYG{p}{)}
\end{sphinxVerbatim}

\end{sphinxuseclass}\end{sphinxVerbatimInput}
\begin{sphinxVerbatimOutput}

\begin{sphinxuseclass}{cell_output}
\begin{sphinxVerbatim}[commandchars=\\\{\}]
Levels
          KG        NOKG
2020  100.00  100.000000
2021  120.00  120.000000
2022  120.00  120.000000
2023  120.00  120.000000
2024  124.80  110.355024
2025  131.04  115.872775

Growth
        KG      NOKG
2020   NaN       NaN
2021  20.0  20.00000
2022   0.0   0.00000
2023   0.0   0.00000
2024   4.0  \PYGZhy{}8.03748
2025   5.0   5.00000
\end{sphinxVerbatim}

\end{sphinxuseclass}\end{sphinxVerbatimOutput}

\end{sphinxuseclass}
\begin{sphinxadmonition}{note}{Note:}
\sphinxAtStartPar
In the first example where KG (keep\_growth) \sphinxstylestrong{was not set}, because the level was set constant for three periods at 120 the rate of growth was 0 for the final two years of the set period.  But following this update, the level of A in 2023 is 120. With \sphinxcode{\sphinxupquote{keep\_Growth=False}} (its default value)m the level of A in 2024 remains at its unchanged  unchanged (lower) level of 100.35. As a result, the growth rate in 2024 is negative.

\sphinxAtStartPar
In the \sphinxstylestrong{–kg} example, the pre\sphinxhyphen{}exsting growth rate (of 4\%) is applied to the new value of 120 and so the level in 2024 is (120*1.04)=124.8 and 2025 is 131.04.
\end{sphinxadmonition}


\subsubsection{.upd() with the option keep\_growth set globally}
\label{\detokenize{content/04_PythonEssentials/UpdateCommand:upd-with-the-option-keep-growth-set-globally}}
\sphinxAtStartPar
Above the line level option \sphinxcode{\sphinxupquote{\sphinxhyphen{}\sphinxhyphen{}keep\_growth}} or \sphinxcode{\sphinxupquote{\sphinxhyphen{}\sphinxhyphen{}kg}} was used to keep the growth rate(or not) for a given operation.

\sphinxAtStartPar
This works because by default the option \sphinxcode{\sphinxupquote{Keep\_growth}} is set to false, implementing \sphinxcode{\sphinxupquote{\sphinxhyphen{}\sphinxhyphen{}kg}} at the line level temporarily set the keep\_growth flag to  true for the specific line (and those following).

\sphinxAtStartPar
The \sphinxcode{\sphinxupquote{keep\_growth}} flag can also be set globally for all the lines by setting the option in the command line.

\sphinxAtStartPar
\sphinxcode{\sphinxupquote{keep\_growth=True}}.

\sphinxAtStartPar
Now as default, all lines will keep the growth rate (unless overridden at the line level with \sphinxcode{\sphinxupquote{\sphinxhyphen{}\sphinxhyphen{}nkg}} or \sphinxcode{\sphinxupquote{\sphinxhyphen{}\sphinxhyphen{}no\_keep\_growth}}.
\begin{itemize}
\item {} 
\sphinxAtStartPar
c,d are updated in 2022 and 2023 and keep the growth rates afterwards

\item {} 
\sphinxAtStartPar
e the \sphinxcode{\sphinxupquote{\sphinxhyphen{}\sphinxhyphen{}no\_keep\_growth}} in this line prevents the updating 2024\sphinxhyphen{}2025

\end{itemize}

\begin{sphinxuseclass}{cell}\begin{sphinxVerbatimInput}

\begin{sphinxuseclass}{cell_input}
\begin{sphinxVerbatim}[commandchars=\\\{\}]
\PYG{c+c1}{\PYGZsh{} Create a data frame}
\PYG{n}{dftest} \PYG{o}{=} \PYG{n}{pd}\PYG{o}{.}\PYG{n}{DataFrame}\PYG{p}{(}\PYG{l+m+mi}{100}\PYG{p}{,}
       \PYG{n}{index}\PYG{o}{=}\PYG{p}{[}\PYG{l+m+mi}{2020}\PYG{o}{+}\PYG{n}{v} \PYG{k}{for} \PYG{n}{v} \PYG{o+ow}{in} \PYG{n+nb}{range}\PYG{p}{(}\PYG{n}{number\PYGZus{}of\PYGZus{}rows}\PYG{p}{)}\PYG{p}{]}\PYG{p}{,} \PYG{c+c1}{\PYGZsh{} create row index}
       \PYG{c+c1}{\PYGZsh{} equivalent to index=[2020,2021,2022,2023,2024,2025] }
       \PYG{n}{columns}\PYG{o}{=}\PYG{p}{[}\PYG{l+s+s1}{\PYGZsq{}}\PYG{l+s+s1}{A}\PYG{l+s+s1}{\PYGZsq{}}\PYG{p}{,}\PYG{l+s+s1}{\PYGZsq{}}\PYG{l+s+s1}{B}\PYG{l+s+s1}{\PYGZsq{}}\PYG{p}{,}\PYG{l+s+s1}{\PYGZsq{}}\PYG{l+s+s1}{C}\PYG{l+s+s1}{\PYGZsq{}}\PYG{p}{,}\PYG{l+s+s1}{\PYGZsq{}}\PYG{l+s+s1}{D}\PYG{l+s+s1}{\PYGZsq{}}\PYG{p}{,}\PYG{l+s+s1}{\PYGZsq{}}\PYG{l+s+s1}{E}\PYG{l+s+s1}{\PYGZsq{}}\PYG{p}{]}\PYG{p}{)}                                 \PYG{c+c1}{\PYGZsh{} create column name }
\PYG{n}{df}
\end{sphinxVerbatim}

\end{sphinxuseclass}\end{sphinxVerbatimInput}
\begin{sphinxVerbatimOutput}

\begin{sphinxuseclass}{cell_output}
\begin{sphinxVerbatim}[commandchars=\\\{\}]
               A           B      A\PYGZus{}ORIG      B\PYGZus{}ORIG
2020  100.000000  110.000000  100.000000  110.000000
2021  102.000000  112.200000  102.000000  112.200000
2022  109.040000  119.444000  104.040000  114.444000
2023  111.120800  121.732880  106.120800  116.732880
2024  113.343216  119.067538  108.243216  119.067538
2025  115.610080  121.448888  110.408080  121.448888
\end{sphinxVerbatim}

\end{sphinxuseclass}\end{sphinxVerbatimOutput}

\end{sphinxuseclass}
\begin{sphinxuseclass}{cell}\begin{sphinxVerbatimInput}

\begin{sphinxuseclass}{cell_input}
\begin{sphinxVerbatim}[commandchars=\\\{\}]
\PYG{n}{dfres} \PYG{o}{=} \PYG{n}{dftest}\PYG{o}{.}\PYG{n}{upd}\PYG{p}{(}\PYG{l+s+s1}{\PYGZsq{}\PYGZsq{}\PYGZsq{}}
\PYG{l+s+s1}{\PYGZlt{}2022 2023\PYGZgt{} c = 200 }
\PYG{l+s+s1}{\PYGZlt{}2022 2023\PYGZgt{} d = 300  }
\PYG{l+s+s1}{\PYGZlt{}2022 2023\PYGZgt{} e = 400  \PYGZhy{}\PYGZhy{}no\PYGZus{}keep\PYGZus{}growth }
\PYG{l+s+s1}{\PYGZsq{}\PYGZsq{}\PYGZsq{}}\PYG{p}{,}\PYG{n}{keep\PYGZus{}growth}\PYG{o}{=}\PYG{k+kc}{True}\PYG{p}{)}  \PYG{c+c1}{\PYGZsh{} \PYGZlt{}=  Set keep\PYGZus{}growth to True for the entirety of the command, }
                       \PYG{c+c1}{\PYGZsh{} except for e where it is overridden by the \PYGZhy{}\PYGZhy{}no\PYGZus{}keep\PYGZus{}growth flag}
\PYG{n+nb}{print}\PYG{p}{(}\PYG{l+s+sa}{f}\PYG{l+s+s1}{\PYGZsq{}}\PYG{l+s+s1}{Dataframe:}\PYG{l+s+se}{\PYGZbs{}n}\PYG{l+s+si}{\PYGZob{}}\PYG{n}{dfres}\PYG{l+s+si}{\PYGZcb{}}\PYG{l+s+se}{\PYGZbs{}n}\PYG{l+s+se}{\PYGZbs{}n}\PYG{l+s+s1}{Growth:}\PYG{l+s+se}{\PYGZbs{}n}\PYG{l+s+si}{\PYGZob{}}\PYG{n}{dfres}\PYG{o}{.}\PYG{n}{pct\PYGZus{}change}\PYG{p}{(}\PYG{p}{)}\PYG{o}{*}\PYG{l+m+mi}{100}\PYG{l+s+si}{\PYGZcb{}}\PYG{l+s+se}{\PYGZbs{}n}\PYG{l+s+s1}{\PYGZsq{}}\PYG{p}{)}
\end{sphinxVerbatim}

\end{sphinxuseclass}\end{sphinxVerbatimInput}
\begin{sphinxVerbatimOutput}

\begin{sphinxuseclass}{cell_output}
\begin{sphinxVerbatim}[commandchars=\\\{\}]
Dataframe:
        A    B      C      D    E
2020  100  100  100.0  100.0  100
2021  100  100  100.0  100.0  100
2022  100  100  200.0  300.0  400
2023  100  100  200.0  300.0  400
2024  100  100  200.0  300.0  100
2025  100  100  200.0  300.0  100

Growth:
        A    B      C      D      E
2020  NaN  NaN    NaN    NaN    NaN
2021  0.0  0.0    0.0    0.0    0.0
2022  0.0  0.0  100.0  200.0  300.0
2023  0.0  0.0    0.0    0.0    0.0
2024  0.0  0.0    0.0    0.0  \PYGZhy{}75.0
2025  0.0  0.0    0.0    0.0    0.0
\end{sphinxVerbatim}

\end{sphinxuseclass}\end{sphinxVerbatimOutput}

\end{sphinxuseclass}

\subsection{Update several variable in one line}
\label{\detokenize{content/04_PythonEssentials/UpdateCommand:update-several-variable-in-one-line}}
\sphinxAtStartPar
Sometime there is a need to update several variable with the same value over the same time frame. To ease this case .update can accept several variables in one line

\begin{sphinxuseclass}{cell}\begin{sphinxVerbatimInput}

\begin{sphinxuseclass}{cell_input}
\begin{sphinxVerbatim}[commandchars=\\\{\}]
\PYG{n}{df}\PYG{o}{.}\PYG{n}{upd}\PYG{p}{(}\PYG{l+s+s1}{\PYGZsq{}\PYGZsq{}\PYGZsq{}}
\PYG{l+s+s1}{\PYGZlt{}2022 2024\PYGZgt{} h i j k =      40      \PYGZsh{} earlier values are set to zero by default}
\PYG{l+s+s1}{\PYGZlt{}2020\PYGZgt{}      p q r s =       1000   \PYGZsh{} All values beginning in 2020 set to 1000}
\PYG{l+s+s1}{\PYGZlt{}2021 \PYGZhy{}1\PYGZgt{}   p q r s =growth 2      \PYGZsh{} \PYGZhy{}1 indicates the last year of dataframe}
\PYG{l+s+s1}{\PYGZsq{}\PYGZsq{}\PYGZsq{}}\PYG{p}{)}
\end{sphinxVerbatim}

\end{sphinxuseclass}\end{sphinxVerbatimInput}
\begin{sphinxVerbatimOutput}

\begin{sphinxuseclass}{cell_output}
\begin{sphinxVerbatim}[commandchars=\\\{\}]
               A           B      A\PYGZus{}ORIG      B\PYGZus{}ORIG     H     I     J     K  \PYGZbs{}
2020  100.000000  110.000000  100.000000  110.000000   0.0   0.0   0.0   0.0   
2021  102.000000  112.200000  102.000000  112.200000   0.0   0.0   0.0   0.0   
2022  109.040000  119.444000  104.040000  114.444000  40.0  40.0  40.0  40.0   
2023  111.120800  121.732880  106.120800  116.732880  40.0  40.0  40.0  40.0   
2024  113.343216  119.067538  108.243216  119.067538  40.0  40.0  40.0  40.0   
2025  115.610080  121.448888  110.408080  121.448888   0.0   0.0   0.0   0.0   

                P            Q            R            S  
2020  1000.000000  1000.000000  1000.000000  1000.000000  
2021  1020.000000  1020.000000  1020.000000  1020.000000  
2022  1040.400000  1040.400000  1040.400000  1040.400000  
2023  1061.208000  1061.208000  1061.208000  1061.208000  
2024  1082.432160  1082.432160  1082.432160  1082.432160  
2025  1104.080803  1104.080803  1104.080803  1104.080803  
\end{sphinxVerbatim}

\end{sphinxuseclass}\end{sphinxVerbatimOutput}

\end{sphinxuseclass}

\subsection{.upd(,,scale=<number, default=1>) Scale the updates}
\label{\detokenize{content/04_PythonEssentials/UpdateCommand:upd-scale-number-default-1-scale-the-updates}}
\sphinxAtStartPar
When running a scenario it can be useful to be able to create a number of scenarios based on one update but with different scale.

\sphinxAtStartPar
This can be particularly useful when we want to do sensitivity analyses of model results, depending on how heavily a shocked variable is hit

\sphinxAtStartPar
When using the scale option, scale=0  the baseline while scale=0.5 is a scenario half
the severity.

\sphinxAtStartPar
In the example below the values of the dataframes are printed. We use the scale option (setting to to 0, 0.5 and 1) to run three scenarios using the same code but where the update in each case is multiplied by either 0, 0.5 or 1.

\begin{sphinxadmonition}{note}{Note:}
\sphinxAtStartPar
Here we are just printing the outputs, a more interesting example would involve the solving a  model using different levels of a given shock.
\end{sphinxadmonition}

\begin{sphinxuseclass}{cell}\begin{sphinxVerbatimInput}

\begin{sphinxuseclass}{cell_input}
\begin{sphinxVerbatim}[commandchars=\\\{\}]
\PYG{n+nb}{print}\PYG{p}{(}\PYG{l+s+sa}{f}\PYG{l+s+s1}{\PYGZsq{}}\PYG{l+s+s1}{input dataframe: }\PYG{l+s+se}{\PYGZbs{}n}\PYG{l+s+si}{\PYGZob{}}\PYG{n}{df}\PYG{l+s+si}{\PYGZcb{}}\PYG{l+s+se}{\PYGZbs{}n}\PYG{l+s+se}{\PYGZbs{}n}\PYG{l+s+s1}{\PYGZsq{}}\PYG{p}{)}
\PYG{k}{for} \PYG{n}{severity} \PYG{o+ow}{in} \PYG{p}{[}\PYG{l+m+mi}{0}\PYG{p}{,}\PYG{l+m+mf}{0.5}\PYG{p}{,}\PYG{l+m+mi}{1}\PYG{p}{]}\PYG{p}{:} 
    \PYG{c+c1}{\PYGZsh{} First make a dataframe with some growth rate }
    \PYG{n}{res} \PYG{o}{=} \PYG{n}{df}\PYG{o}{.}\PYG{n}{upd}\PYG{p}{(}\PYG{l+s+s1}{\PYGZsq{}\PYGZsq{}\PYGZsq{}}
\PYG{l+s+s1}{    \PYGZlt{}2021 2025\PYGZgt{}}
\PYG{l+s+s1}{    a =growth 1 2 3 4 5 }
\PYG{l+s+s1}{    b + 10}
\PYG{l+s+s1}{    }\PYG{l+s+s1}{\PYGZsq{}\PYGZsq{}\PYGZsq{}}\PYG{p}{,}\PYG{n}{scale}\PYG{o}{=}\PYG{n}{severity}\PYG{p}{)}
    \PYG{n+nb}{print}\PYG{p}{(}\PYG{l+s+sa}{f}\PYG{l+s+s1}{\PYGZsq{}}\PYG{l+s+si}{\PYGZob{}}\PYG{n}{severity}\PYG{l+s+si}{=\PYGZcb{}}\PYG{l+s+se}{\PYGZbs{}n}\PYG{l+s+s1}{Dataframe:}\PYG{l+s+se}{\PYGZbs{}n}\PYG{l+s+si}{\PYGZob{}}\PYG{n}{res}\PYG{l+s+si}{\PYGZcb{}}\PYG{l+s+se}{\PYGZbs{}n}\PYG{l+s+se}{\PYGZbs{}n}\PYG{l+s+s1}{Growth:}\PYG{l+s+se}{\PYGZbs{}n}\PYG{l+s+si}{\PYGZob{}}\PYG{n}{res}\PYG{o}{.}\PYG{n}{pct\PYGZus{}change}\PYG{p}{(}\PYG{p}{)}\PYG{o}{*}\PYG{l+m+mi}{100}\PYG{l+s+si}{\PYGZcb{}}\PYG{l+s+se}{\PYGZbs{}n}\PYG{l+s+se}{\PYGZbs{}n}\PYG{l+s+s1}{\PYGZsq{}}\PYG{p}{)}
    \PYG{c+c1}{\PYGZsh{}  }
    \PYG{c+c1}{\PYGZsh{} Here the updated dataframe is only printed. }
    \PYG{c+c1}{\PYGZsh{} A more realistic use case is to simulate a model like this: }
    \PYG{c+c1}{\PYGZsh{} dummy\PYGZus{} = mpak(res,keep=\PYGZsq{}Severity \PYGZob{}serverity\PYGZcb{}\PYGZsq{})    \PYGZsh{} more realistic }
\end{sphinxVerbatim}

\end{sphinxuseclass}\end{sphinxVerbatimInput}
\begin{sphinxVerbatimOutput}

\begin{sphinxuseclass}{cell_output}
\begin{sphinxVerbatim}[commandchars=\\\{\}]
input dataframe: 
               A           B      A\PYGZus{}ORIG      B\PYGZus{}ORIG
2020  100.000000  110.000000  100.000000  110.000000
2021  102.000000  112.200000  102.000000  112.200000
2022  109.040000  119.444000  104.040000  114.444000
2023  111.120800  121.732880  106.120800  116.732880
2024  113.343216  119.067538  108.243216  119.067538
2025  115.610080  121.448888  110.408080  121.448888


severity=0
Dataframe:
          A           B      A\PYGZus{}ORIG      B\PYGZus{}ORIG
2020  100.0  110.000000  100.000000  110.000000
2021  100.0  112.200000  102.000000  112.200000
2022  100.0  119.444000  104.040000  114.444000
2023  100.0  121.732880  106.120800  116.732880
2024  100.0  119.067538  108.243216  119.067538
2025  100.0  121.448888  110.408080  121.448888

Growth:
        A         B  A\PYGZus{}ORIG  B\PYGZus{}ORIG
2020  NaN       NaN     NaN     NaN
2021  0.0  2.000000     2.0     2.0
2022  0.0  6.456328     2.0     2.0
2023  0.0  1.916279     2.0     2.0
2024  0.0 \PYGZhy{}2.189501     2.0     2.0
2025  0.0  2.000000     2.0     2.0


severity=0.5
Dataframe:
               A           B      A\PYGZus{}ORIG      B\PYGZus{}ORIG
2020  100.000000  110.000000  100.000000  110.000000
2021  100.500000  117.200000  102.000000  112.200000
2022  101.505000  124.444000  104.040000  114.444000
2023  103.027575  126.732880  106.120800  116.732880
2024  105.088126  124.067538  108.243216  119.067538
2025  107.715330  126.448888  110.408080  121.448888

Growth:
        A         B  A\PYGZus{}ORIG  B\PYGZus{}ORIG
2020  NaN       NaN     NaN     NaN
2021  0.5  6.545455     2.0     2.0
2022  1.0  6.180887     2.0     2.0
2023  1.5  1.839285     2.0     2.0
2024  2.0 \PYGZhy{}2.103118     2.0     2.0
2025  2.5  1.919399     2.0     2.0


severity=1
Dataframe:
               A           B      A\PYGZus{}ORIG      B\PYGZus{}ORIG
2020  100.000000  110.000000  100.000000  110.000000
2021  101.000000  122.200000  102.000000  112.200000
2022  103.020000  129.444000  104.040000  114.444000
2023  106.110600  131.732880  106.120800  116.732880
2024  110.355024  129.067538  108.243216  119.067538
2025  115.872775  131.448888  110.408080  121.448888

Growth:
        A          B  A\PYGZus{}ORIG  B\PYGZus{}ORIG
2020  NaN        NaN     NaN     NaN
2021  1.0  11.090909     2.0     2.0
2022  2.0   5.927987     2.0     2.0
2023  3.0   1.768240     2.0     2.0
2024  4.0  \PYGZhy{}2.023293     2.0     2.0
2025  5.0   1.845042     2.0     2.0
\end{sphinxVerbatim}

\end{sphinxuseclass}\end{sphinxVerbatimOutput}

\end{sphinxuseclass}

\subsection{.upd(,,lprint=True ) prints values the before and after update}
\label{\detokenize{content/04_PythonEssentials/UpdateCommand:upd-lprint-true-prints-values-the-before-and-after-update}}
\sphinxAtStartPar
The \sphinxcode{\sphinxupquote{lPrint}} option of the method \sphinxcode{\sphinxupquote{upd()}} is by defualt \sphinxcode{\sphinxupquote{= False}}.  By setting it true an update command will output the results of the calculation comapriong the values of the dataframe (over the impacted period) before, after and the difference between the two.

\begin{sphinxuseclass}{cell}\begin{sphinxVerbatimInput}

\begin{sphinxuseclass}{cell_input}
\begin{sphinxVerbatim}[commandchars=\\\{\}]
\PYG{n}{df}\PYG{o}{.}\PYG{n}{upd}\PYG{p}{(}\PYG{l+s+s1}{\PYGZsq{}\PYGZsq{}\PYGZsq{}}
\PYG{l+s+s1}{\PYGZsh{} Same number of values as years}
\PYG{l+s+s1}{\PYGZlt{}2021 2022\PYGZgt{} A *  42 44}
\PYG{l+s+s1}{\PYGZsq{}\PYGZsq{}\PYGZsq{}}\PYG{p}{,}\PYG{n}{lprint}\PYG{o}{=}\PYG{l+m+mi}{1}\PYG{p}{)}
\end{sphinxVerbatim}

\end{sphinxuseclass}\end{sphinxVerbatimInput}
\begin{sphinxVerbatimOutput}

\begin{sphinxuseclass}{cell_output}
\begin{sphinxVerbatim}[commandchars=\\\{\}]
Update * [42.0, 44.0] 2021 2022
A                    Before                After                 Diff
2021               102.0000            4284.0000            4182.0000
2022               109.0400            4797.7600            4688.7200
\end{sphinxVerbatim}

\begin{sphinxVerbatim}[commandchars=\\\{\}]
                A           B      A\PYGZus{}ORIG      B\PYGZus{}ORIG
2020   100.000000  110.000000  100.000000  110.000000
2021  4284.000000  112.200000  102.000000  112.200000
2022  4797.760000  119.444000  104.040000  114.444000
2023   111.120800  121.732880  106.120800  116.732880
2024   113.343216  119.067538  108.243216  119.067538
2025   115.610080  121.448888  110.408080  121.448888
\end{sphinxVerbatim}

\end{sphinxuseclass}\end{sphinxVerbatimOutput}

\end{sphinxuseclass}

\subsection{.upd(,,create=True ) Requires the variable to exist}
\label{\detokenize{content/04_PythonEssentials/UpdateCommand:upd-create-true-requires-the-variable-to-exist}}
\sphinxAtStartPar
Until now .upd has created variables if they did not exist in the input dataframe.

\sphinxAtStartPar
To catch misspellings the parameter \sphinxcode{\sphinxupquote{create}} can be set to False.
New variables will not be created, and an exception will be raised.

\sphinxAtStartPar
Here Pythons exception handling is used, so the notebook will continue to run the cells below.

\begin{sphinxuseclass}{cell}\begin{sphinxVerbatimInput}

\begin{sphinxuseclass}{cell_input}
\begin{sphinxVerbatim}[commandchars=\\\{\}]
\PYG{k}{try}\PYG{p}{:}
    \PYG{n}{xx} \PYG{o}{=} \PYG{n}{df}\PYG{o}{.}\PYG{n}{upd}\PYG{p}{(}\PYG{l+s+s1}{\PYGZsq{}\PYGZsq{}\PYGZsq{}}
\PYG{l+s+s1}{    \PYGZsh{} Same number of values as years}
\PYG{l+s+s1}{    \PYGZlt{}2021 2022\PYGZgt{} Aa *  42 44}
\PYG{l+s+s1}{    }\PYG{l+s+s1}{\PYGZsq{}\PYGZsq{}\PYGZsq{}}\PYG{p}{,}\PYG{n}{create}\PYG{o}{=}\PYG{k+kc}{False}\PYG{p}{)}
    \PYG{n+nb}{print}\PYG{p}{(}\PYG{n}{xx}\PYG{p}{)}
\PYG{k}{except} \PYG{n+ne}{Exception} \PYG{k}{as} \PYG{n}{inst}\PYG{p}{:}
    \PYG{n}{xx} \PYG{o}{=} \PYG{k+kc}{None}
    \PYG{n+nb}{print}\PYG{p}{(}\PYG{n}{inst}\PYG{p}{)} 
\end{sphinxVerbatim}

\end{sphinxuseclass}\end{sphinxVerbatimInput}
\begin{sphinxVerbatimOutput}

\begin{sphinxuseclass}{cell_output}
\begin{sphinxVerbatim}[commandchars=\\\{\}]
Variable to update not found:AA, timespan = [2021 2022] 
Set create=True if you want the variable created: 
\end{sphinxVerbatim}

\end{sphinxuseclass}\end{sphinxVerbatimOutput}

\end{sphinxuseclass}
\sphinxstepscope


\part{References}

\sphinxstepscope

\begin{sphinxthebibliography}{BCJ+19}
\bibitem[Add89]{content/99_BackMatter/References:id20}
\sphinxAtStartPar
Doug Addison. \sphinxstyleemphasis{The World Bank revised minimum standard model (RMSM) : concepts and issues}. Number WPS231 in Policy Research Working Papers. World Bank, Washington DC., 1989. URL: \sphinxurl{https://documents.worldbank.org/en/publication/documents-reports/documentdetail/997721468765042532/the-world-bank-revised-minimum-standard-model-rmsm-concepts-and-issues}.
\bibitem[Bla18]{content/99_BackMatter/References:id17}
\sphinxAtStartPar
Olivier Blanchard. On the future of Macroeconomic models. \sphinxstyleemphasis{Oxford Review of Economic Policy}, 34(1\sphinxhyphen{}2):43–54, 2018. URL: \sphinxurl{https://academic.oup.com/oxrep/article/34/1-2/43/4781808}, \sphinxhref{https://doi.org/https://doi.org/10.1093/oxrep/grx045}{doi:https://doi.org/10.1093/oxrep/grx045}.
\bibitem[BCJ+19]{content/99_BackMatter/References:id15}
\sphinxAtStartPar
Andrew Burns, Benoit Campagne, Charl Jooste, David Stephan, and Thi Thanh Bui. \sphinxstyleemphasis{The World Bank Macro\sphinxhyphen{}Fiscal Model Technical Description}. Number 8965 in Policy Research Working Papers. World Bank, Washington DC., 2019. URL: \sphinxurl{https://openknowledge.worldbank.org/handle/10986/32217}.
\bibitem[BJS21a]{content/99_BackMatter/References:id14}
\sphinxAtStartPar
Andrew Burns, Charl Jooste, and Gregor Schwerhoff. \sphinxstyleemphasis{Climate Modeling for Macroeconomic Policy : A Case Study for Pakistan}. Number 9780 in Policy Research Working Papers. World Bank, Washington, DC, 2021. URL: \sphinxurl{https://openknowledge.worldbank.org/bitstream/handle/10986/36307/Climate-Modeling-for-Macroeconomic-Policy-A-Case-Study-for-Pakistan.pdf?sequence=1\&isAllowed=y}.
\bibitem[BJS21b]{content/99_BackMatter/References:id18}
\sphinxAtStartPar
Andrew Burns, Charl Jooste, and Gregor Schwerhoff. \sphinxstyleemphasis{Macroeconomic Modeling of Managing Hurricane Damage in the Caribbean: The Case of Jamaica}. Volume 9505 of Policy Research Working Paper. World Bank, Washington DC., 2021. URL: \sphinxurl{https://documents1.worldbank.org/curated/en/593351609776234361/pdf/Macroeconomic-Modeling-of-Managing-Hurricane-Damage-in-the-Caribbean-The-Case-of-Jamaica.pdf}.
\bibitem[Che71]{content/99_BackMatter/References:id21}
\sphinxAtStartPar
Hollis Chenery. \sphinxstyleemphasis{Studies in Development Planning.} Harvard University Press,, Cambridge, MA., 1971.
\end{sphinxthebibliography}







\renewcommand{\indexname}{Index}
\printindex
\end{document}