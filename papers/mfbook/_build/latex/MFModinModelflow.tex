%% Generated by Sphinx.
\def\sphinxdocclass{jupyterBook}
\documentclass[letterpaper,10pt,english]{jupyterBook}
\ifdefined\pdfpxdimen
   \let\sphinxpxdimen\pdfpxdimen\else\newdimen\sphinxpxdimen
\fi \sphinxpxdimen=.75bp\relax
\ifdefined\pdfimageresolution
    \pdfimageresolution= \numexpr \dimexpr1in\relax/\sphinxpxdimen\relax
\fi
%% let collapsible pdf bookmarks panel have high depth per default
\PassOptionsToPackage{bookmarksdepth=5}{hyperref}
%% turn off hyperref patch of \index as sphinx.xdy xindy module takes care of
%% suitable \hyperpage mark-up, working around hyperref-xindy incompatibility
\PassOptionsToPackage{hyperindex=false}{hyperref}
%% memoir class requires extra handling
\makeatletter\@ifclassloaded{memoir}
{\ifdefined\memhyperindexfalse\memhyperindexfalse\fi}{}\makeatother

\PassOptionsToPackage{warn}{textcomp}

\catcode`^^^^00a0\active\protected\def^^^^00a0{\leavevmode\nobreak\ }
\usepackage{cmap}
\usepackage{fontspec}
\defaultfontfeatures[\rmfamily,\sffamily,\ttfamily]{}
\usepackage{amsmath,amssymb,amstext}
\usepackage{polyglossia}
\setmainlanguage{english}



\setmainfont{FreeSerif}[
  Extension      = .otf,
  UprightFont    = *,
  ItalicFont     = *Italic,
  BoldFont       = *Bold,
  BoldItalicFont = *BoldItalic
]
\setsansfont{FreeSans}[
  Extension      = .otf,
  UprightFont    = *,
  ItalicFont     = *Oblique,
  BoldFont       = *Bold,
  BoldItalicFont = *BoldOblique,
]
\setmonofont{FreeMono}[
  Extension      = .otf,
  UprightFont    = *,
  ItalicFont     = *Oblique,
  BoldFont       = *Bold,
  BoldItalicFont = *BoldOblique,
]



\usepackage[Bjarne]{fncychap}
\usepackage[,numfigreset=1,mathnumfig]{sphinx}

\fvset{fontsize=\small}
\usepackage{geometry}


% Include hyperref last.
\usepackage{hyperref}
% Fix anchor placement for figures with captions.
\usepackage{hypcap}% it must be loaded after hyperref.
% Set up styles of URL: it should be placed after hyperref.
\urlstyle{same}

\addto\captionsenglish{\renewcommand{\contentsname}{The World Bank's MFMod Framework and Modelflow}}

\usepackage{sphinxmessages}



        % Start of preamble defined in sphinx-jupyterbook-latex %
         \usepackage[Latin,Greek]{ucharclasses}
        \usepackage{unicode-math}
        % fixing title of the toc
        \addto\captionsenglish{\renewcommand{\contentsname}{Contents}}
        \hypersetup{
            pdfencoding=auto,
            psdextra
        }
        % End of preamble defined in sphinx-jupyterbook-latex %
        

\title{The World Bank's MFMod Framework in Python with Modelflow}
\date{Jun 20, 2023}
\release{}
\author{Andrew Burns and Ib Hansen}
\newcommand{\sphinxlogo}{\vbox{}}
\renewcommand{\releasename}{}
\makeindex
\begin{document}

\pagestyle{empty}
\sphinxmaketitle
\pagestyle{plain}
\sphinxtableofcontents
\pagestyle{normal}
\phantomsection\label{\detokenize{content/introduction::doc}}


\sphinxAtStartPar
Over the decades, the World Bank has invested heavily in the tools available to its country economists to analyze forecast and monitor economic activity.  The fruit of those efforts is the \sphinxcode{\sphinxupquote{MFMod}} framework, a system of 184 macro\sphinxhyphen{}structural models of nearly all the economies in the world.

\sphinxAtStartPar
\sphinxcode{\sphinxupquote{MFMod}} is the work horse tool used by World Bank economists to produce the twice annual compendium \sphinxhref{https://www.worldbank.org/en/publication/macro-poverty-outlook}{Macro Poverty Outlook} which presents concise statements of the Bank’s views on the major challenges, outlook, and forecasts for almost all of the developing countries in the world.

\sphinxAtStartPar
Making the tool that underpins this publication available in an open source form to a broader audience has been a major focus of the Bank’s modelling team over the past several years.  This publication is the culmination of this effort.

\sphinxAtStartPar
While destined for a relatively small audience of macroeconomic modellers, it is hoped that it will generate significant benefits both for the Bank which will benefit from feedback on the models and from clients who will have for the first time access to the Bank’s models in a costless form.

\sphinxAtStartPar
Pablo Saavedra

\sphinxAtStartPar
Vice President
Equitable Growth, Finance and Institutions
The World Bank

\sphinxstepscope


\part{The World Bank's MFMod Framework and Modelflow}

\sphinxstepscope


\chapter{Macrostructural models}
\label{\detokenize{content/02_MacrostructuralModels/MacroStructuralModels:macrostructural-models}}\label{\detokenize{content/02_MacrostructuralModels/MacroStructuralModels::doc}}
\sphinxAtStartPar
The economics profession uses a wide range of models for different purposes.  Macro\sphinxhyphen{}structural models (also known as semi\sphinxhyphen{}structural or Macro\sphinxhyphen{}econometric models) are a class of models that seek to summarize the most important interconnections and determinants of economic activity in an economy. Computable General Equilibrium (CGE), and Dynamic Stochastic General Equilibrium (DSGE) models are other classes of models that also seek, using somewhat different methodologies, to capture the main economic channels by which the actions of agents (firms, households, governments) interact and help determine the structure, level and rate of growth of economic activity in an economy.

\sphinxAtStartPar
Typically, organizations, including the World Bank, use all of these tools, privileging one or the other for specific purposes. Macrostructural models like the MFMod framework are widely used by Central Banks, Ministries of Finance; and professional forecasters both for the purposes of generating forecasts and policy analysis. While macrostructural models fell out of favor with academic economists, they remain central tools in policy making and forecasting circles. Olivier Blanchard, former Chief Economist at the International Monetary Fund, summarized his conclusions of a recent debate between several leading academics in a recent paper Blanchard {[}\hyperlink{cite.content/99_BackMatter/References:id17}{2018}{]}, that lays out his views on the relative strengths and weaknesses of each of these systems. He concludes that academic economists are wrong to discard out of hand policy models such as macro\sphinxhyphen{}structural models, concluding until academic models improve, such models will continue to play a central role in helping economists analyze the macro\sphinxhyphen{}economy.


\section{A system of equations}
\label{\detokenize{content/02_MacrostructuralModels/MacroStructuralModels:a-system-of-equations}}
\sphinxAtStartPar
Mathematically, macro\sphinxhyphen{}structural models are a system of equations comprised of two kinds of equations and three kinds of variables.

\sphinxAtStartPar
\sphinxstylestrong{Types of variables in macro\sphinxhyphen{}structural models}
\begin{itemize}
\item {} 
\sphinxAtStartPar
\sphinxcode{\sphinxupquote{Identities}} are variables that are determined by a well defined accounting rule that always holds. The famous GDP formula Y=C+I+G+(X\sphinxhyphen{}M) is one such identity, it indicates that GDP at market prices is definitionally equal to Consumption plus Investment plus Government spending plus Exports less Imports.

\item {} 
\sphinxAtStartPar
\sphinxcode{\sphinxupquote{Behavioural}} variables are determined by equations that attempt to summarize an economic (vs accounting) relationship between variables, where the relationship is derived from economic theory. Thus, the equation that says Real Consumption =  is a function of Disposable Income, the price level, and animal spirits) is a behavioural equation. Because these behavioural equations do not fully explain the variation in the variable they seek to explain, and because the sensitivities of variables to the changes in other variables are uncertain, these equations and their parameters are  typically estimated econometrically and are subject to error.

\item {} 
\sphinxAtStartPar
\sphinxcode{\sphinxupquote{Exogenous}} variables are not determined by the model. Typically there are set either by assumption or from data external to the model.  For an individual country model, the exogenous variables might include the global price of crude oil  because the level of activity of the economy itself is unlikely to affect the world price of oil. Similarily, the rate of growth of GDP in other economies may be treated as an exogenosu variable, important to determining exports in the modelled developing country but following the small\sphinxhyphen{}country assumption deemed to be largely unaffected by activity in the modelled country.

\end{itemize}

\sphinxAtStartPar
In a fully general form, the system of equations that comprise a mode might be written as:
\begin{align*}
y_t^1  &=  f^1(y_{t+u}^1...,y_{t+u}^n...,y_t^2...,y_{t}^n...y_{t-r}^1...,y_{t-r}^n,x_t^1...x_{t}^k,...x_{t-s}^1...,x_{t-s}^k) \\
y_t^2  &=  f^2(y_{t+u}^1...,y_{t+u}^n...,y_t^1...,y_{t}^n...y_{t-r}^1...,y_{t-r}^n,x_t^1...x_{t}^k,...x_{t-s}^1...,x_{t-s}^k) \\
\vdots \\
y_t^n  &=  f^n(y_{t+u}^1...,y_{t+u}^n...,y_t^1...,y_{t}^{n-1}...y_{t-r}^1...,y_{t-r}^n,x_t^1...x_{t}^r,x..._{t-s}^1...,x_{t-s}^k)
\end{align*}
\sphinxAtStartPar
where \( y_t^1 \) is one of n endogenous variables and \(x_t^1\) is an exogenous variable and there are as many equations as there are unknown (endogenous variables).

\sphinxAtStartPar
Substituting the variable mnemonics Y,C,I,G,X,M for the y’s above, a simple macrostructural model can be written as as a system of 6 equations in 6 unknowns:
\begin{align*}
Y_t  &=  C_t+I_t+G+t+ (X_t-M_t) \\
C_t &= c_t(C_{t-1},C_{t-2},I_t,G_t,X_t,M_t,P_t)\\
I_t &= c_t(I_{t-1},I_{t-2},C_t,G_t,X_t,M_t,P_t)\\
G_t &= c_t(G_{t-1},G_{t-2},C_t,I_t,X_t,M_t,P_t)\\
X_t &= c_t(X_{t-1},X_{t-2},C_t,I_t,G_t,M_t,P_t,P^f_t)\\
M_t &= c_t(M_{t-1},M_{t-2},C_t,I_t,G_t,X_t,P_t,P^f_t)
\end{align*}
\sphinxAtStartPar
Where \(P_t, P^f_t\) (domestic and foreign prices, respectively) are exogenous in this simple model.

\sphinxstepscope


\chapter{Modelflow and the MFMod models of the World Bank}
\label{\detokenize{content/02_MacrostructuralModels/MFModAndModelFlow:modelflow-and-the-mfmod-models-of-the-world-bank}}\label{\detokenize{content/02_MacrostructuralModels/MFModAndModelFlow::doc}}
\sphinxAtStartPar
At the World Bank models built using the MFMod framework are developed in \DUrole{xref,myst}{EViews}. When disseminated to clients, the models are solved and simulated using EViews, although often through the intermediary of an easy\sphinxhyphen{}to\sphinxhyphen{}use customized excel EViews environment developed by the World Bank. That said, as a systems of equations and associated data, the models can be solved and operated under any system capable of solving a system of simultaneous equations. \sphinxcode{\sphinxupquote{Modelflow}} is such a system and offers a wide range of features that permit not only solving the model, but also provide a rich and powerful suite of tools for analyzing the model and reporting results.


\section{A brief history of ModelFlow\}}
\label{\detokenize{content/02_MacrostructuralModels/MFModAndModelFlow:a-brief-history-of-modelflow}}
\sphinxAtStartPar
Modelflow is a python library that was developed by Ib Hansen over several years while working at the Danish Central Bank and the European Central Bank. The framework has been used both to port the U.S. Federal Reserve’s macro\sphinxhyphen{}structural  model to python, but also been used to bring several stress\sphinxhyphen{}testing models developed by European Central Banks and the European Central Bank into a python environment.

\sphinxAtStartPar
Beginning in 2019, Hansen has worked with the World Bank to develop additional features that facilitate working with models built using the Bank’s MFMod Framework, with the objective of creating an open source platform through which the Bank’s models can be made available to the public.

\sphinxAtStartPar
This paper, and the models that accompany it, are the product of this collaboration.

\sphinxstepscope


\chapter{Installation}
\label{\detokenize{content/03_Installation/Installing:installation}}\label{\detokenize{content/03_Installation/Installing::doc}}
\begin{sphinxadmonition}{warning}{Warning:}
\sphinxAtStartPar
IB:Changed to Miniconda.
consolidated  install python and modelflow for ease of use
\end{sphinxadmonition}

\sphinxAtStartPar
Modelflow is a python package that defines the \sphinxcode{\sphinxupquote{model}} class, its methods and a number of other functions that extend and combine pre\sphinxhyphen{}existing python functions to allow the easy solution of complex systems of equations including macro\sphinxhyphen{}structural models like MFMod.  To work with \sphinxcode{\sphinxupquote{modelflow}}, a user needs to first install python (preferably the Anaconda variant), several supporting packages, and of course the \sphinxcode{\sphinxupquote{modelflow}} package itself.

\sphinxAtStartPar
\sphinxcode{\sphinxupquote{modelflow}} can be run through the Jupyter notebook system, directly from the python command\sphinxhyphen{}line or from IDEs (Interactive Development Environments) like \sphinxcode{\sphinxupquote{Spyder}} or Microsoft’s \sphinxcode{\sphinxupquote{Visual Code}}.

\sphinxAtStartPar
The Jupyter Notebook facilitates an interactive approach to building python programs, annotating them and ultimately doing simulations using MFMod under \sphinxcode{\sphinxupquote{modelflow}}. This entire manual and the examples in it were all written and executed in the Jupyter Notebook environment.


\section{Installation of Python}
\label{\detokenize{content/03_Installation/Installing:installation-of-python}}
\sphinxAtStartPar
\sphinxcode{\sphinxupquote{Python}} is an extremely powerful, versatile and extensible open\sphinxhyphen{}source language. It is widely used for artificial intelligence application, interactive web sites, and scientific processing. As of 14 November 2022, the \sphinxcode{\sphinxupquote{Python Package Index}} (PyPI), the official repository for third\sphinxhyphen{}party Python software, contained over 415,000 packages that extend its functionality %
\begin{footnote}[1]\sphinxAtStartFootnote
\sphinxhref{https://en.wikipedia.org/wiki/Python\_(programming\_language)}{Wikipedia article on python}
%
\end{footnote}. Modelflow is one of these packages.

\sphinxAtStartPar
Python comes in many flavors and \sphinxcode{\sphinxupquote{modelflow}} will work with any of them.  However, \sphinxstylestrong{users are strongly advised to use the Anaconda version of Python}. It is advised to use the \sphinxstylestrong{Miniconda} version of the package manager, this is will only install the minimum needed packages. However the full \sphinxstylestrong{Anaconda} version can also be used.

\sphinxAtStartPar
The remainder of this section points to instructions on how to install the Miniconda version of python under Windows. \sphinxcode{\sphinxupquote{Modelflow}} works equally well under linux and should work under MacOS.

\sphinxAtStartPar
This is followed by section that describes the steps necessary to create an anaconda environment with all the necessary packages to run \sphinxcode{\sphinxupquote{modelflow}}.


\section{Installation of Miniconda under Windows:}
\label{\detokenize{content/03_Installation/Installing:installation-of-miniconda-under-windows}}
\sphinxAtStartPar
The easy way to install is to use the \sphinxstylestrong{Miniconda} package manager from Anaconda. You can find a link \sphinxhref{https://docs.conda.io/en/latest/miniconda.html}{here}. The latest version can be downloaded and installed by execution these command in a command window.

\sphinxAtStartPar
Open the a command prompt

\sphinxAtStartPar
\sphinxstylestrong{First} copy/paste these lines to the command prompt

\begin{sphinxVerbatim}[commandchars=\\\{\}]
\PYG{n}{curl} \PYG{o}{\PYGZhy{}}\PYG{n}{L} \PYG{l+s+s2}{\PYGZdq{}}\PYG{l+s+s2}{https://repo.anaconda.com/miniconda/Miniconda3\PYGZhy{}latest\PYGZhy{}Windows\PYGZhy{}x86\PYGZus{}64.exe}\PYG{l+s+s2}{\PYGZdq{}} \PYG{o}{\PYGZhy{}}\PYG{o}{\PYGZhy{}}\PYG{n}{output} \PYG{o}{\PYGZpc{}}\PYG{n}{TEMP}\PYG{o}{\PYGZpc{}}\PYGZbs{}\PYG{n}{miniconda}\PYG{o}{.}\PYG{n}{exe}
\PYG{o}{\PYGZpc{}}\PYG{n}{temp}\PYG{o}{\PYGZpc{}}\PYGZbs{}\PYG{n}{miniconda}\PYG{o}{.}\PYG{n}{exe} \PYG{o}{/}\PYG{n}{S} \PYG{o}{/}\PYG{n}{D}\PYG{o}{=}\PYG{o}{\PYGZpc{}}\PYG{n}{USERPROFILE}\PYG{o}{\PYGZpc{}}\PYGZbs{}\PYG{n}{miniconda3}

\end{sphinxVerbatim}

\sphinxAtStartPar
These lines will download miniconda and install it in the users folder

\sphinxAtStartPar
\sphinxstylestrong{Then} copy/paste these lines to the command prompt

\begin{sphinxVerbatim}[commandchars=\\\{\}]
\PYG{n}{call} \PYG{o}{\PYGZpc{}}\PYG{n}{USERPROFILE}\PYG{o}{\PYGZpc{}}\PYGZbs{}\PYG{n}{miniconda3}\PYGZbs{}\PYG{n}{Scripts}\PYGZbs{}\PYG{n}{activate}\PYG{o}{.}\PYG{n}{bat} \PYG{o}{\PYGZpc{}}\PYG{n}{USERPROFILE}\PYG{o}{\PYGZpc{}}\PYGZbs{}\PYG{n}{miniconda3}
\PYG{n}{call} \PYG{n}{conda} \PYG{n}{install} \PYG{n}{mamba} \PYG{o}{\PYGZhy{}}\PYG{n}{c} \PYG{n}{conda}\PYG{o}{\PYGZhy{}}\PYG{n}{forge} \PYG{o}{\PYGZhy{}}\PYG{n}{y}

\end{sphinxVerbatim}

\sphinxAtStartPar
These lines will install the mamba package manager. Mamba is a faster alternative to the conda default package manager in miniconda.


\subsection{Installation of Python under macOS}
\label{\detokenize{content/03_Installation/Installing:installation-of-python-under-macos}}
\sphinxAtStartPar
The definitive source for installing Anaconda under macOS can be found \sphinxhref{https://docs.anaconda.com/anaconda/install/mac-os/}{here}.


\subsection{Installation of Python under Linux}
\label{\detokenize{content/03_Installation/Installing:installation-of-python-under-linux}}
\sphinxAtStartPar
The definitive source for installing Anaconda under Linux can be found \sphinxhref{https://docs.anaconda.com/anaconda/install/linux/}{here}.


\section{Installation of Modelflow}
\label{\detokenize{content/03_Installation/Installing:installation-of-modelflow}}
\sphinxAtStartPar
\sphinxcode{\sphinxupquote{Modelflow}} is a python package that defines the modelflow class \sphinxcode{\sphinxupquote{model}} among others.  \sphinxcode{\sphinxupquote{Modelflow}} has many dependencies. Installing the class the first time can take some time depending on your internet connection and computer speed.  It is essential that you follow all of the steps outlined below to ensure that your version of \sphinxcode{\sphinxupquote{modelflow}} operates as expected.

\begin{sphinxadmonition}{warning}{Warning:}
\sphinxAtStartPar
The following instructions concern the installation of \sphinxcode{\sphinxupquote{modelflow}} within an Anaconda installation of python.  Different flavors of Python may require slight changes to this recipe, but are not covered here.

\sphinxAtStartPar
\sphinxcode{\sphinxupquote{Modelflow}} is built and tested using the anaconda python environment.  It is strongly recommended to use Anaconda with \sphinxcode{\sphinxupquote{modelflow}}.

\sphinxAtStartPar
If you have not already installed Anaconda following the instructions in the preceding section, please do so \sphinxstylestrong{Now}.
\end{sphinxadmonition}


\section{Installation of \sphinxstyleliteralintitle{\sphinxupquote{modelflow}} using Miniconda}
\label{\detokenize{content/03_Installation/Installing:installation-of-modelflow-using-miniconda}}
\sphinxAtStartPar
Open the a command prompt or continue in the command prompt used to install python.

\sphinxAtStartPar
copy/paste these lines to the command prompt

\begin{sphinxVerbatim}[commandchars=\\\{\}]
\PYG{o}{\PYGZpc{}}\PYG{n}{USERPROFILE}\PYG{o}{\PYGZpc{}}\PYGZbs{}\PYG{n}{miniconda3}\PYGZbs{}\PYG{n}{Scripts}\PYGZbs{}\PYG{n}{activate}\PYG{o}{.}\PYG{n}{bat} \PYG{o}{\PYGZpc{}}\PYG{n}{USERPROFILE}\PYG{o}{\PYGZpc{}}\PYGZbs{}\PYG{n}{miniconda3} 

\PYG{n}{mamba} \PYG{n}{create} \PYG{o}{\PYGZhy{}}\PYG{n}{n} \PYG{n}{modelflow} \PYG{o}{\PYGZhy{}}\PYG{n}{c} \PYG{n}{ibh} \PYG{o}{\PYGZhy{}}\PYG{n}{c}  \PYG{n}{conda}\PYG{o}{\PYGZhy{}}\PYG{n}{forge} \PYG{n}{modelflow\PYGZus{}stable} \PYG{o}{\PYGZhy{}}\PYG{n}{y}

\PYG{n}{conda} \PYG{n}{activate} \PYG{n}{modelflow}
\PYG{n}{pip} \PYG{n}{install} \PYG{n}{dash\PYGZus{}interactive\PYGZus{}graphviz}
\PYG{n}{jupyter} \PYG{n}{contrib} \PYG{n}{nbextension} \PYG{n}{install} \PYG{o}{\PYGZhy{}}\PYG{o}{\PYGZhy{}}\PYG{n}{user}
\PYG{n}{jupyter} \PYG{n}{nbextension} \PYG{n}{enable} \PYG{n}{hide\PYGZus{}input\PYGZus{}all}\PYG{o}{/}\PYG{n}{main}
\PYG{n}{jupyter} \PYG{n}{nbextension} \PYG{n}{enable} \PYG{n}{splitcell}\PYG{o}{/}\PYG{n}{splitcell}
\PYG{n}{jupyter} \PYG{n}{nbextension} \PYG{n}{enable} \PYG{n}{toc2}\PYG{o}{/}\PYG{n}{main}
\PYG{n}{jupyter} \PYG{n}{nbextension} \PYG{n}{enable} \PYG{n}{varInspector}\PYG{o}{/}\PYG{n}{main}


\end{sphinxVerbatim}

\sphinxAtStartPar
Depending on the speed of your computer and of your internet connection installation could take as little as 10 minutes or more than 1/2 an hour.

\sphinxAtStartPar
At the end of the process you will have a new conda environment called \sphinxcode{\sphinxupquote{modelflow}}, and this will have been activated. The computer set up is complete and the user is ready to work with \sphinxcode{\sphinxupquote{modelflow}}.

\sphinxAtStartPar
The following sections give a brief introduction to Jupyter notebook, which is a flexible tool that allows us to execute python code, interact with the modelflow class and World Bank Models and annotate what we have done for future replication.


\section{Updating Modelflow}
\label{\detokenize{content/03_Installation/Installing:updating-modelflow}}
\sphinxAtStartPar
Once installed, a \sphinxcode{\sphinxupquote{modelflow}} environment can be updated. This is done first by activating the \sphinxcode{\sphinxupquote{modelflow}} environment created above.

\begin{sphinxVerbatim}[commandchars=\\\{\}]
\PYG{o}{\PYGZpc{}}\PYG{n}{USERPROFILE}\PYG{o}{\PYGZpc{}}\PYGZbs{}\PYG{n}{miniconda3}\PYGZbs{}\PYG{n}{Scripts}\PYGZbs{}\PYG{n}{activate}\PYG{o}{.}\PYG{n}{bat} \PYG{o}{\PYGZpc{}}\PYG{n}{USERPROFILE}\PYG{o}{\PYGZpc{}}\PYGZbs{}\PYG{n}{miniconda3} 
\PYG{n}{conda} \PYG{n}{activate} \PYG{n}{modelflow}

\end{sphinxVerbatim}

\sphinxAtStartPar
And then executing the command:

\begin{sphinxVerbatim}[commandchars=\\\{\}]
\PYG{n}{mamba} \PYG{n}{install} \PYG{n}{modelflow} \PYG{o}{\PYGZhy{}}\PYG{n}{c} \PYG{n}{ibh} \PYG{o}{\PYGZhy{}}\PYG{o}{\PYGZhy{}}\PYG{n}{no}\PYG{o}{\PYGZhy{}}\PYG{n}{deps}



\end{sphinxVerbatim}


\section{Starting a Python session with modelflow.}
\label{\detokenize{content/03_Installation/Installing:starting-a-python-session-with-modelflow}}
\sphinxAtStartPar
This can be done by starting the \sphinxcode{\sphinxupquote{Anaconda Prompt (miniconda3)}} app. This will create a command window with a base python environment. Now you want to change the environment to the modelflow environment. This is done like this:

\begin{sphinxVerbatim}[commandchars=\\\{\}]
\PYG{n}{conda} \PYG{n}{activate} \PYG{n}{modelflow}

\end{sphinxVerbatim}

\sphinxAtStartPar
Now we are ready to start jupyter by:

\begin{sphinxVerbatim}[commandchars=\\\{\}]
\PYG{n}{cd} \PYG{o}{\PYGZlt{}}\PYG{n}{path} \PYG{n}{where} \PYG{n}{you} \PYG{n}{want} \PYG{n}{to} \PYG{n}{start}\PYG{o}{\PYGZgt{}} 
\PYG{n}{jupyter} \PYG{n}{notebook}
\end{sphinxVerbatim}


\section{Creating a .lnk file to start Jupyter with modelflow.}
\label{\detokenize{content/03_Installation/Installing:creating-a-lnk-file-to-start-jupyter-with-modelflow}}
\sphinxAtStartPar
It can be convenient to create a icon on the desktop which start up jupyter in a modelflow environment. This is done by creating a new shortcut where the \sphinxstylestrong{destination/target} is:

\begin{sphinxVerbatim}[commandchars=\\\{\}]
\PYG{o}{\PYGZpc{}}\PYG{n}{windir}\PYG{o}{\PYGZpc{}}\PYGZbs{}\PYG{n}{System32}\PYGZbs{}\PYG{n}{cmd}\PYG{o}{.}\PYG{n}{exe} \PYG{l+s+s2}{\PYGZdq{}}\PYG{l+s+s2}{/K}\PYG{l+s+s2}{\PYGZdq{}} \PYG{o}{\PYGZpc{}}\PYG{n}{USERPROFILE}\PYG{o}{\PYGZpc{}}\PYGZbs{}\PYG{n}{miniconda3}\PYGZbs{}\PYG{n}{Scripts}\PYGZbs{}\PYG{n}{activate}\PYG{o}{.}\PYG{n}{bat} \PYG{o}{\PYGZpc{}}\PYG{n}{USERPROFILE}\PYG{o}{\PYGZpc{}}\PYGZbs{}\PYG{n}{miniconda3}\PYGZbs{}\PYG{n}{envs}\PYGZbs{}\PYG{n}{modelflow}\PYG{o}{\PYGZam{}}\PYG{o}{\PYGZam{}}\PYG{n}{jupyter} \PYG{n}{notebook}
\end{sphinxVerbatim}

\sphinxAtStartPar
This will create a command window, activate modelflow and start jupyter notebook

\sphinxAtStartPar
and the \sphinxstylestrong{Start in}  is set to the folder where you want jupyter to start from. 
Be aware that jupyter can’t access notebooks “before” the \sphinxcode{\sphinxupquote{Start in}} location. \sphinxcode{\sphinxupquote{C:\textbackslash{}}} could be a choice.


\bigskip\hrule\bigskip


\sphinxstepscope


\chapter{Testing your installation of modelflow}
\label{\detokenize{content/03_Installation/TestingModelFlow:testing-your-installation-of-modelflow}}\label{\detokenize{content/03_Installation/TestingModelFlow::doc}}
\sphinxAtStartPar
To test that the installation of modelflow has worked properly, we will build a model using the modelflow framework and then simulate it.  A simple model that illustrates many of the functions of modelflow is the Solow growth model.

\sphinxAtStartPar
The code below first sets up the python environment by importing the modelflow  and pandas classes.  The initial two lines of code and the final two lines just set up the environment for optimal display and are not required.

\sphinxAtStartPar
To test the installation on your system you can copy this code into a Jupyter notebook and execute it.


\section{Specifying the model}
\label{\detokenize{content/03_Installation/TestingModelFlow:specifying-the-model}}
\sphinxAtStartPar
Having loaded the model class from the modelflow library, we can start constructing the model.

\sphinxAtStartPar
The first step is to define the equations of the model, using \sphinxcode{\sphinxupquote{modelflow}}’s Business Logic Language.

\begin{sphinxShadowBox}
\sphinxstylesidebartitle{\sphinxstylestrong{Business Logic Language}}

\sphinxAtStartPar
More on how to specify models \DUrole{xref,myst}{here}
\end{sphinxShadowBox}

\sphinxAtStartPar
The below code segment defines a string fsolow that contains the equations for the solow model, where:
\begin{itemize}
\item {} 
\sphinxAtStartPar
GDP is defined as a simple Cobb\sphinxhyphen{}Douglas production function as the product of TFP, Capital (raised to the share of capital in total income) and Labour (raised to the share of labor in total income)

\item {} 
\sphinxAtStartPar
Investment is equal to GDP less consumption

\item {} 
\sphinxAtStartPar
The change in capital is equal to investment this period less the depreciation of the capital stock from the previous period

\item {} 
\sphinxAtStartPar
Labor grows at the rate of growth of the variable \sphinxcode{\sphinxupquote{Labor\_growth}}

\item {} 
\sphinxAtStartPar
a pure reporting identity Capital\_intensity the ratio of the Capital Stock to the Labor input

\end{itemize}

\sphinxAtStartPar
We thus have a system of 6 equations with 6 unknowns (GDP, Consumption, Investment, Change in the Capital stock, and change in Labor supply, and the capital\_intensity) and exogenous variables (TFP, alfa,savings\_rate,Depreciation\_rate and Labor\_growth).

\begin{sphinxadmonition}{note}{Note:}
\sphinxAtStartPar
The equations for Labor and Capital have been entered as difference equations. The \sphinxcode{\sphinxupquote{modelflow}} object will automatically normalize them, generating an internal representation of \sphinxcode{\sphinxupquote{Labour=Labour(t\sphinxhyphen{}1)*(1+Labor\_growth)}} and \sphinxcode{\sphinxupquote{Capital=Capital(t\sphinxhyphen{}1)*(1\sphinxhyphen{}Depreciation\_rate)+Investment}}
\end{sphinxadmonition}

\begin{sphinxuseclass}{cell}\begin{sphinxVerbatimInput}

\begin{sphinxuseclass}{cell_input}
\begin{sphinxVerbatim}[commandchars=\\\{\}]
\PYG{n}{fsolow} \PYG{o}{=} \PYG{l+s+s1}{\PYGZsq{}\PYGZsq{}\PYGZsq{}}\PYG{l+s+se}{\PYGZbs{}}
\PYG{l+s+s1}{GDP          = TFP  * Capital**alfa * Labor **(1\PYGZhy{}alfa) }
\PYG{l+s+s1}{Consumption     = (1\PYGZhy{}saving\PYGZus{}rate)  * GDP }
\PYG{l+s+s1}{Investment      = GDP \PYGZhy{} Consumption   }
\PYG{l+s+s1}{diff(Capital)   = Investment\PYGZhy{}Depreciation\PYGZus{}rate * Capital(\PYGZhy{}1)}
\PYG{l+s+s1}{diff(Labor)     = Labor\PYGZus{}growth * Labor(\PYGZhy{}1)  }
\PYG{l+s+s1}{Capital\PYGZus{}intensity = Capital/Labor }
\PYG{l+s+s1}{\PYGZsq{}\PYGZsq{}\PYGZsq{}}
\end{sphinxVerbatim}

\end{sphinxuseclass}\end{sphinxVerbatimInput}

\end{sphinxuseclass}
\sphinxAtStartPar
To create the model we instantiate (create) a variable \sphinxcode{\sphinxupquote{msolow}} (which will ultimately contain both the equations and data for the model) using the \sphinxcode{\sphinxupquote{.from\_eq()}} method of the \sphinxcode{\sphinxupquote{modelflow}} class – submitting to it the equations in string form, and giving it the name “Solow model”.

\begin{sphinxuseclass}{cell}\begin{sphinxVerbatimInput}

\begin{sphinxuseclass}{cell_input}
\begin{sphinxVerbatim}[commandchars=\\\{\}]
\PYG{n}{msolow} \PYG{o}{=} \PYG{n}{model}\PYG{o}{.}\PYG{n}{from\PYGZus{}eq}\PYG{p}{(}\PYG{n}{fsolow}\PYG{p}{,}\PYG{n}{modelname}\PYG{o}{=}\PYG{l+s+s1}{\PYGZsq{}}\PYG{l+s+s1}{Solow model}\PYG{l+s+s1}{\PYGZsq{}}\PYG{p}{)}
\end{sphinxVerbatim}

\end{sphinxuseclass}\end{sphinxVerbatimInput}

\end{sphinxuseclass}
\sphinxAtStartPar
The internal representation of the normalized equations can be displayed in normalized business language with the \sphinxcode{\sphinxupquote{modelflow}} method \sphinxcode{\sphinxupquote{.print\_model}}:

\begin{sphinxuseclass}{cell}\begin{sphinxVerbatimInput}

\begin{sphinxuseclass}{cell_input}
\begin{sphinxVerbatim}[commandchars=\\\{\}]
\PYG{n}{msolow}\PYG{o}{.}\PYG{n}{print\PYGZus{}model}
\end{sphinxVerbatim}

\end{sphinxuseclass}\end{sphinxVerbatimInput}
\begin{sphinxVerbatimOutput}

\begin{sphinxuseclass}{cell_output}
\begin{sphinxVerbatim}[commandchars=\\\{\}]
FRML \PYGZlt{}\PYGZgt{} GDP          = TFP  * CAPITAL**ALFA * LABOR **(1\PYGZhy{}ALFA)  \PYGZdl{}
FRML \PYGZlt{}\PYGZgt{} CONSUMPTION     = (1\PYGZhy{}SAVING\PYGZus{}RATE)  * GDP  \PYGZdl{}
FRML \PYGZlt{}\PYGZgt{} INVESTMENT      = GDP \PYGZhy{} CONSUMPTION    \PYGZdl{}
FRML \PYGZlt{}\PYGZgt{} CAPITAL=CAPITAL(\PYGZhy{}1)+(INVESTMENT\PYGZhy{}DEPRECIATION\PYGZus{}RATE * CAPITAL(\PYGZhy{}1))\PYGZdl{}
FRML \PYGZlt{}\PYGZgt{} LABOR=LABOR(\PYGZhy{}1)+(LABOR\PYGZus{}GROWTH * LABOR(\PYGZhy{}1))\PYGZdl{}
FRML \PYGZlt{}\PYGZgt{} CAPITAL\PYGZus{}INTENSITY = CAPITAL/LABOR  \PYGZdl{}
\end{sphinxVerbatim}

\end{sphinxuseclass}\end{sphinxVerbatimOutput}

\end{sphinxuseclass}

\section{Create some data}
\label{\detokenize{content/03_Installation/TestingModelFlow:create-some-data}}
\sphinxAtStartPar
For the moment \sphinxcode{\sphinxupquote{msolow}} has a mathematical representation of a system of equations but no data.

\sphinxAtStartPar
To add data we  create a pandas dataframe with initial values for our exogenous variables. Technically capital and labor are endogenous in the Solow model, but because they are specified as change equations their initial values are exogenous and need to be initialized.

\sphinxAtStartPar
The code below  instantiates (creates) a panda dataframe \sphinxcode{\sphinxupquote{df}} and fills it with the variables for our model, initializing these with a series of values over 300 datapoints.  The final command displays the first ten rows of the dataframe.

\begin{sphinxadmonition}{note}{Note:}
\begin{sphinxVerbatim}[commandchars=\\\{\}]
Pandas data frames is a foundational class of python.  There are thousands of web sites dedicated to understanding pandas.  Some notable ones include:
\end{sphinxVerbatim}
\end{sphinxadmonition}

\begin{sphinxuseclass}{cell}\begin{sphinxVerbatimInput}

\begin{sphinxuseclass}{cell_input}
\begin{sphinxVerbatim}[commandchars=\\\{\}]
\PYG{n}{N} \PYG{o}{=} \PYG{l+m+mi}{300}  
\PYG{n}{df} \PYG{o}{=} \PYG{n}{pd}\PYG{o}{.}\PYG{n}{DataFrame}\PYG{p}{(}\PYG{p}{\PYGZob{}}\PYG{l+s+s1}{\PYGZsq{}}\PYG{l+s+s1}{LABOR}\PYG{l+s+s1}{\PYGZsq{}}\PYG{p}{:}\PYG{p}{[}\PYG{l+m+mi}{100}\PYG{p}{]}\PYG{o}{*}\PYG{n}{N}\PYG{p}{,}
                   \PYG{l+s+s1}{\PYGZsq{}}\PYG{l+s+s1}{CAPITAL}\PYG{l+s+s1}{\PYGZsq{}}\PYG{p}{:}\PYG{p}{[}\PYG{l+m+mi}{100}\PYG{p}{]}\PYG{o}{*}\PYG{n}{N}\PYG{p}{,} 
                   \PYG{l+s+s1}{\PYGZsq{}}\PYG{l+s+s1}{ALFA}\PYG{l+s+s1}{\PYGZsq{}}\PYG{p}{:}\PYG{p}{[}\PYG{l+m+mf}{0.5}\PYG{p}{]}\PYG{o}{*}\PYG{n}{N}\PYG{p}{,} 
                   \PYG{l+s+s1}{\PYGZsq{}}\PYG{l+s+s1}{TFP}\PYG{l+s+s1}{\PYGZsq{}}\PYG{p}{:} \PYG{p}{[}\PYG{l+m+mi}{1}\PYG{p}{]}\PYG{o}{*}\PYG{n}{N}\PYG{p}{,} 
                   \PYG{l+s+s1}{\PYGZsq{}}\PYG{l+s+s1}{DEPRECIATION\PYGZus{}RATE}\PYG{l+s+s1}{\PYGZsq{}}\PYG{p}{:} \PYG{p}{[}\PYG{l+m+mf}{0.05}\PYG{p}{]}\PYG{o}{*}\PYG{n}{N}\PYG{p}{,} 
                   \PYG{l+s+s1}{\PYGZsq{}}\PYG{l+s+s1}{LABOR\PYGZus{}GROWTH}\PYG{l+s+s1}{\PYGZsq{}}\PYG{p}{:} \PYG{p}{[}\PYG{l+m+mf}{0.01}\PYG{p}{]}\PYG{o}{*}\PYG{n}{N}\PYG{p}{,} 
                   \PYG{l+s+s1}{\PYGZsq{}}\PYG{l+s+s1}{SAVING\PYGZus{}RATE}\PYG{l+s+s1}{\PYGZsq{}}\PYG{p}{:}\PYG{p}{[}\PYG{l+m+mf}{0.05}\PYG{p}{]}\PYG{o}{*}\PYG{n}{N}\PYG{p}{\PYGZcb{}}\PYG{p}{,}\PYG{n}{index}\PYG{o}{=}\PYG{p}{[}\PYG{n}{v} \PYG{k}{for} \PYG{n}{v} \PYG{o+ow}{in} \PYG{n+nb}{range}\PYG{p}{(}\PYG{l+m+mi}{2000}\PYG{p}{,}\PYG{l+m+mi}{2300}\PYG{p}{)}\PYG{p}{]}\PYG{p}{)}
\PYG{n}{df}\PYG{o}{.}\PYG{n}{head}\PYG{p}{(}\PYG{p}{)} \PYG{c+c1}{\PYGZsh{}this prints out the first 5 rows of the dataframe}
\end{sphinxVerbatim}

\end{sphinxuseclass}\end{sphinxVerbatimInput}
\begin{sphinxVerbatimOutput}

\begin{sphinxuseclass}{cell_output}
\begin{sphinxVerbatim}[commandchars=\\\{\}]
      LABOR  CAPITAL  ALFA  TFP  DEPRECIATION\PYGZus{}RATE  LABOR\PYGZus{}GROWTH  SAVING\PYGZus{}RATE
2000    100      100   0.5    1               0.05          0.01         0.05
2001    100      100   0.5    1               0.05          0.01         0.05
2002    100      100   0.5    1               0.05          0.01         0.05
2003    100      100   0.5    1               0.05          0.01         0.05
2004    100      100   0.5    1               0.05          0.01         0.05
\end{sphinxVerbatim}

\end{sphinxuseclass}\end{sphinxVerbatimOutput}

\end{sphinxuseclass}

\section{Putting it together}
\label{\detokenize{content/03_Installation/TestingModelFlow:putting-it-together}}
\sphinxAtStartPar
Having defined an initial data set for all the exogenous variables, we can combine these with the equations and solve the model.

\sphinxAtStartPar
The command below solves the model \sphinxcode{\sphinxupquote{msolow}} on the data contained in the dataframe \sphinxcode{\sphinxupquote{df}} and stores the output in a new dataframe called \sphinxcode{\sphinxupquote{result}}.

\sphinxAtStartPar
The last line displays the values of the simulated model, which now includes results for the endogenous variables, and different values for the Labor and Capital variables reflecting their endogeneity for periods 2 through 300.

\begin{sphinxuseclass}{cell}\begin{sphinxVerbatimInput}

\begin{sphinxuseclass}{cell_input}
\begin{sphinxVerbatim}[commandchars=\\\{\}]
\PYG{n}{result} \PYG{o}{=} \PYG{n}{msolow}\PYG{p}{(}\PYG{n}{df}\PYG{p}{,}\PYG{n}{keep}\PYG{o}{=}\PYG{l+s+s1}{\PYGZsq{}}\PYG{l+s+s1}{Baseline}\PYG{l+s+s1}{\PYGZsq{}}\PYG{p}{)} 
\PYG{c+c1}{\PYGZsh{} The model is simulated for all years possible }

\PYG{n}{result}\PYG{o}{.}\PYG{n}{head}\PYG{p}{(}\PYG{l+m+mi}{10}\PYG{p}{)}
\end{sphinxVerbatim}

\end{sphinxuseclass}\end{sphinxVerbatimInput}
\begin{sphinxVerbatimOutput}

\begin{sphinxuseclass}{cell_output}
\begin{sphinxVerbatim}[commandchars=\\\{\}]
           LABOR     CAPITAL  ALFA  TFP  DEPRECIATION\PYGZus{}RATE  LABOR\PYGZus{}GROWTH   
2000  100.000000  100.000000   0.5  1.0               0.05          0.01  \PYGZbs{}
2001  101.000000  100.025580   0.5  1.0               0.05          0.01   
2002  102.010000  100.076226   0.5  1.0               0.05          0.01   
2003  103.030100  100.151443   0.5  1.0               0.05          0.01   
2004  104.060401  100.250762   0.5  1.0               0.05          0.01   
2005  105.101005  100.373733   0.5  1.0               0.05          0.01   
2006  106.152015  100.519926   0.5  1.0               0.05          0.01   
2007  107.213535  100.688931   0.5  1.0               0.05          0.01   
2008  108.285671  100.880357   0.5  1.0               0.05          0.01   
2009  109.368527  101.093830   0.5  1.0               0.05          0.01   

      SAVING\PYGZus{}RATE  CAPITAL\PYGZus{}INTENSITY  INVESTMENT  CONSUMPTION         GDP  
2000         0.05           0.000000    0.000000     0.000000    0.000000  
2001         0.05           0.990352    5.025580    95.486029  100.511609  
2002         0.05           0.981043    5.051924    95.986562  101.038487  
2003         0.05           0.972060    5.079029    96.501546  101.580575  
2004         0.05           0.963390    5.106891    97.030930  102.137821  
2005         0.05           0.955022    5.135509    97.574667  102.710176  
2006         0.05           0.946943    5.164880    98.132713  103.297593  
2007         0.05           0.939144    5.195002    98.705029  103.900030  
2008         0.05           0.931613    5.225872    99.291576  104.517449  
2009         0.05           0.924341    5.257491    99.892323  105.149813  
\end{sphinxVerbatim}

\end{sphinxuseclass}\end{sphinxVerbatimOutput}

\end{sphinxuseclass}

\section{Create a scenario and run again}
\label{\detokenize{content/03_Installation/TestingModelFlow:create-a-scenario-and-run-again}}
\begin{sphinxShadowBox}
\sphinxstylesidebartitle{\sphinxstylestrong{dataframe.upd}}

\sphinxAtStartPar
When importing modelclass all pandas dataframes are enriched with a a handy way to create a new pandas dataframe as a copy of an existing one but with one or more series updated.

\sphinxAtStartPar
In this case df.upd will create a a new dataframe \sphinxcode{\sphinxupquote{dfscenaario}} with updated LABOR\_GROWTH

\sphinxAtStartPar
For more detail on the \sphinxcode{\sphinxupquote{.upd}} method look here \DUrole{xref,myst}{here}
\end{sphinxShadowBox}

\begin{sphinxuseclass}{cell}\begin{sphinxVerbatimInput}

\begin{sphinxuseclass}{cell_input}
\begin{sphinxVerbatim}[commandchars=\\\{\}]
\PYG{n}{dfscenario} \PYG{o}{=} \PYG{n}{df}\PYG{o}{.}\PYG{n}{mfcalc}\PYG{p}{(}\PYG{l+s+s1}{\PYGZsq{}}\PYG{l+s+s1}{\PYGZlt{}2023 2200\PYGZgt{} LABOR\PYGZus{}GROWTH = LABOR\PYGZus{}GROWTH + 0.002}\PYG{l+s+s1}{\PYGZsq{}}\PYG{p}{)}  \PYG{c+c1}{\PYGZsh{} create a new dataframe, increase LABOR\PYGZus{}GROWTH by 0.002}
\PYG{n}{scenario}   \PYG{o}{=} \PYG{n}{msolow}\PYG{p}{(}\PYG{n}{dfscenario}\PYG{p}{,}\PYG{n}{keep}\PYG{o}{=}\PYG{l+s+s1}{\PYGZsq{}}\PYG{l+s+s1}{Higher labor growth }\PYG{l+s+s1}{\PYGZsq{}}\PYG{p}{)} \PYG{c+c1}{\PYGZsh{} simulate the model }
\end{sphinxVerbatim}

\end{sphinxuseclass}\end{sphinxVerbatimInput}

\end{sphinxuseclass}

\section{Inspect results}
\label{\detokenize{content/03_Installation/TestingModelFlow:inspect-results}}
\sphinxAtStartPar
\sphinxcode{\sphinxupquote{Modelflow}} includes a range of methods to view data and results, either as graphs or as tables.  Some of these are part of standard python, others are additional features that \sphinxcode{\sphinxupquote{modelflow}} makes available.

\sphinxAtStartPar
Scenario results can be inspected either by referring to the scenario name given in the (optional) \sphinxcode{\sphinxupquote{keep}} statement when the model was solved, by referring to the \sphinxcode{\sphinxupquote{basedf}} and the \sphinxcode{\sphinxupquote{lastdf}}.
\begin{itemize}
\item {} 
\sphinxAtStartPar
\sphinxcode{\sphinxupquote{basedf}} is a dataframe that is automatically generated when the model is solved and contains a copy of the initial conditions of the model prior to the shock.

\item {} 
\sphinxAtStartPar
\sphinxcode{\sphinxupquote{lastdf}}is a dataframe that is automatically generated when the model is solved and contains a copy of the results from the simulation. Several built in display functions use these functions to display results.

\end{itemize}

\sphinxAtStartPar
Finally one could also look at the dataframe to which the results of the simulation were assigned \sphinxcode{\sphinxupquote{scenario}} in the example above.

\sphinxAtStartPar
Below is a small sub\sphinxhyphen{}set of the visualization options available.


\subsection{Graphical representations of results}
\label{\detokenize{content/03_Installation/TestingModelFlow:graphical-representations-of-results}}

\subsubsection{The .dif.plot() method}
\label{\detokenize{content/03_Installation/TestingModelFlow:the-dif-plot-method}}
\sphinxAtStartPar
The \sphinxcode{\sphinxupquote{.dif.plot}} method will plot the change in the level of requested variables.  Requested variables can be selected either directly by name or using wildcards.

\sphinxAtStartPar
In this example, a wild card specification is used, requesting the display of all variables that begin with the text ‘labor’.  Note that the selector is not case sensitive.

\sphinxAtStartPar
In this case we are displaying changes into the labor and labor growth variables due to the shock when we increased the growth rate of labor by .0002

\begin{sphinxuseclass}{cell}\begin{sphinxVerbatimInput}

\begin{sphinxuseclass}{cell_input}
\begin{sphinxVerbatim}[commandchars=\\\{\}]
\PYG{n}{msolow}\PYG{p}{[}\PYG{l+s+s1}{\PYGZsq{}}\PYG{l+s+s1}{labor*}\PYG{l+s+s1}{\PYGZsq{}}\PYG{p}{]}\PYG{o}{.}\PYG{n}{dif}\PYG{o}{.}\PYG{n}{plot}\PYG{p}{(}\PYG{p}{)} 
\end{sphinxVerbatim}

\end{sphinxuseclass}\end{sphinxVerbatimInput}
\begin{sphinxVerbatimOutput}

\begin{sphinxuseclass}{cell_output}
\noindent\sphinxincludegraphics{{28ae05d62fafe5fd35db4868226962fccd6269bec73310fece8437ae7f551422}.png}

\end{sphinxuseclass}\end{sphinxVerbatimOutput}

\end{sphinxuseclass}
\sphinxAtStartPar
In this example, instead of using a wild card selector we requested a variable explicitly by name.

\begin{sphinxuseclass}{cell}\begin{sphinxVerbatimInput}

\begin{sphinxuseclass}{cell_input}
\begin{sphinxVerbatim}[commandchars=\\\{\}]
\PYG{n}{msolow}\PYG{p}{[}\PYG{l+s+s1}{\PYGZsq{}}\PYG{l+s+s1}{GDP LABOR\PYGZus{}GROWTH}\PYG{l+s+s1}{\PYGZsq{}}\PYG{p}{]}\PYG{o}{.}\PYG{n}{pct}\PYG{o}{.}\PYG{n}{plot}\PYG{p}{(}\PYG{p}{)} 
\end{sphinxVerbatim}

\end{sphinxuseclass}\end{sphinxVerbatimInput}
\begin{sphinxVerbatimOutput}

\begin{sphinxuseclass}{cell_output}
\noindent\sphinxincludegraphics{{49f43ba057063db290c43f05c27a589984c46ab42fe9078b7976044d7baec37f}.png}

\end{sphinxuseclass}\end{sphinxVerbatimOutput}

\end{sphinxuseclass}

\subsubsection{Using the kept solutions}
\label{\detokenize{content/03_Installation/TestingModelFlow:using-the-kept-solutions}}
\sphinxAtStartPar
Because the keyword \sphinxcode{\sphinxupquote{keep}} was used when running the simulations, we can refer to the scenarios by their names – or produce graphs from multiple scenarios – not just the first and last.

\begin{sphinxuseclass}{cell}\begin{sphinxVerbatimInput}

\begin{sphinxuseclass}{cell_input}
\begin{sphinxVerbatim}[commandchars=\\\{\}]
\PYG{n}{msolow}\PYG{o}{.}\PYG{n}{keep\PYGZus{}plot}\PYG{p}{(}\PYG{l+s+s1}{\PYGZsq{}}\PYG{l+s+s1}{GDP}\PYG{l+s+s1}{\PYGZsq{}}\PYG{p}{)}
    
\end{sphinxVerbatim}

\end{sphinxuseclass}\end{sphinxVerbatimInput}
\begin{sphinxVerbatimOutput}

\begin{sphinxuseclass}{cell_output}
\noindent\sphinxincludegraphics{{d226200c19295f5ac6c2bcdb7b0d7f79658e204e0b397db9e49da9baa34f83ba}.png}

\begin{sphinxVerbatim}[commandchars=\\\{\}]
\PYGZob{}\PYGZsq{}GDP\PYGZsq{}: \PYGZlt{}Figure size 1000x600 with 1 Axes\PYGZgt{}\PYGZcb{}
\end{sphinxVerbatim}

\end{sphinxuseclass}\end{sphinxVerbatimOutput}

\end{sphinxuseclass}

\subsection{Textual and tabular display of results}
\label{\detokenize{content/03_Installation/TestingModelFlow:textual-and-tabular-display-of-results}}
\sphinxAtStartPar
Standard pandas syntax can be used to display data in the results dataframes.

\sphinxAtStartPar
Here we use the standard pandas \sphinxcode{\sphinxupquote{.loc}} method to display every 10th data point for consumption from the results dataframe, beginning from observation 50 through 100.

\begin{sphinxuseclass}{cell}\begin{sphinxVerbatimInput}

\begin{sphinxuseclass}{cell_input}
\begin{sphinxVerbatim}[commandchars=\\\{\}]
\PYG{n}{msolow}\PYG{o}{.}\PYG{n}{lastdf}\PYG{o}{.}\PYG{n}{loc}\PYG{p}{[}\PYG{l+m+mi}{50}\PYG{p}{:}\PYG{l+m+mi}{100}\PYG{p}{:}\PYG{l+m+mi}{10}\PYG{p}{,}\PYG{l+s+s1}{\PYGZsq{}}\PYG{l+s+s1}{CONSUMPTION}\PYG{l+s+s1}{\PYGZsq{}}\PYG{p}{]}
\end{sphinxVerbatim}

\end{sphinxuseclass}\end{sphinxVerbatimInput}
\begin{sphinxVerbatimOutput}

\begin{sphinxuseclass}{cell_output}
\begin{sphinxVerbatim}[commandchars=\\\{\}]
Series([], Name: CONSUMPTION, dtype: float64)
\end{sphinxVerbatim}

\end{sphinxuseclass}\end{sphinxVerbatimOutput}

\end{sphinxuseclass}

\subsubsection{The \sphinxstyleliteralintitle{\sphinxupquote{.dif.df}} method}
\label{\detokenize{content/03_Installation/TestingModelFlow:the-dif-df-method}}
\sphinxAtStartPar
The \sphinxcode{\sphinxupquote{.dif.df}} method prints out the changes in variables, i.e. eh difference between the level of specified  variables in the \sphinxcode{\sphinxupquote{lastdf}} dataframe vs the \sphinxcode{\sphinxupquote{basedf}} dataframe.

\begin{sphinxuseclass}{cell}\begin{sphinxVerbatimInput}

\begin{sphinxuseclass}{cell_input}
\begin{sphinxVerbatim}[commandchars=\\\{\}]
\PYG{n}{msolow}\PYG{p}{[}\PYG{l+s+s1}{\PYGZsq{}}\PYG{l+s+s1}{GDP CONSUMPTION}\PYG{l+s+s1}{\PYGZsq{}}\PYG{p}{]}\PYG{o}{.}\PYG{n}{dif}\PYG{o}{.}\PYG{n}{df}
\end{sphinxVerbatim}

\end{sphinxuseclass}\end{sphinxVerbatimInput}
\begin{sphinxVerbatimOutput}

\begin{sphinxuseclass}{cell_output}
\begin{sphinxVerbatim}[commandchars=\\\{\}]
             GDP  CONSUMPTION
2001    0.000000     0.000000
2002    0.000000     0.000000
2003    0.000000     0.000000
2004    0.000000     0.000000
2005    0.000000     0.000000
...          ...          ...
2295  665.334581   632.067852
2296  672.097592   638.492713
2297  678.925939   644.979642
2298  685.820324   651.529308
2299  692.781453   658.142380

[299 rows x 2 columns]
\end{sphinxVerbatim}

\end{sphinxuseclass}\end{sphinxVerbatimOutput}

\end{sphinxuseclass}

\subsubsection{The \sphinxstyleliteralintitle{\sphinxupquote{.difpct.df}} method}
\label{\detokenize{content/03_Installation/TestingModelFlow:the-difpct-df-method}}
\sphinxAtStartPar
The \sphinxcode{\sphinxupquote{.dif.pct.df}} method express the changes between the last simulation and base simulation results as a percent differences in the level (\({\Delta X_t \over X^{basedf}_{t-1}} \) ).  In the example below the mul100 method multiplies the result by 100.

\begin{sphinxuseclass}{cell}\begin{sphinxVerbatimInput}

\begin{sphinxuseclass}{cell_input}
\begin{sphinxVerbatim}[commandchars=\\\{\}]
\PYG{n}{msolow}\PYG{p}{[}\PYG{l+s+s1}{\PYGZsq{}}\PYG{l+s+s1}{GDP CONSUMPTION}\PYG{l+s+s1}{\PYGZsq{}}\PYG{p}{]}\PYG{o}{.}\PYG{n}{difpct}\PYG{o}{.}\PYG{n}{mul100}\PYG{o}{.}\PYG{n}{df}
\end{sphinxVerbatim}

\end{sphinxuseclass}\end{sphinxVerbatimInput}
\begin{sphinxVerbatimOutput}

\begin{sphinxuseclass}{cell_output}
\begin{sphinxVerbatim}[commandchars=\\\{\}]
           GDP  CONSUMPTION
2001       NaN          NaN
2002  0.000000     0.000000
2003  0.000000     0.000000
2004  0.000000     0.000000
2005  0.000000     0.000000
...        ...          ...
2295  0.005047     0.005047
2296  0.004892     0.004892
2297  0.004742     0.004742
2298  0.004596     0.004596
2299  0.004456     0.004456

[299 rows x 2 columns]
\end{sphinxVerbatim}

\end{sphinxuseclass}\end{sphinxVerbatimOutput}

\end{sphinxuseclass}

\subsection{Interactive display of impacts}
\label{\detokenize{content/03_Installation/TestingModelFlow:interactive-display-of-impacts}}
\sphinxAtStartPar
When working within Jupyter notebook the dif command will produce (without the .df termination) will generate a widget with the results expressed as level differences, percent differences, differences in the growth rate – both graphically and in table form.

\sphinxAtStartPar
Please consult \DUrole{xref,myst}{here} for a fuller presentation of the display routines built into \sphinxcode{\sphinxupquote{modelflfow}}.

\begin{sphinxuseclass}{cell}\begin{sphinxVerbatimInput}

\begin{sphinxuseclass}{cell_input}
\begin{sphinxVerbatim}[commandchars=\\\{\}]
\PYG{n}{msolow}\PYG{p}{[}\PYG{l+s+s1}{\PYGZsq{}}\PYG{l+s+s1}{GDP CONSUMPTION}\PYG{l+s+s1}{\PYGZsq{}}\PYG{p}{]}\PYG{o}{.}\PYG{n}{dif}
\end{sphinxVerbatim}

\end{sphinxuseclass}\end{sphinxVerbatimInput}
\begin{sphinxVerbatimOutput}

\begin{sphinxuseclass}{cell_output}
\begin{sphinxVerbatim}[commandchars=\\\{\}]
Tab(children=(Tab(children=(HTML(value=\PYGZsq{}\PYGZlt{}?xml version=\PYGZdq{}1.0\PYGZdq{} encoding=\PYGZdq{}utf\PYGZhy{}8\PYGZdq{} standalone=\PYGZdq{}no\PYGZdq{}?\PYGZgt{}\PYGZbs{}n\PYGZlt{}!DOCTYPE svg …
\end{sphinxVerbatim}

\begin{sphinxVerbatim}[commandchars=\\\{\}]

\end{sphinxVerbatim}

\end{sphinxuseclass}\end{sphinxVerbatimOutput}

\end{sphinxuseclass}
\sphinxstepscope


\part{Some python essentials for using World Bank models with modelflow}

\sphinxstepscope


\chapter{Introduction to  Jupyter Notebook}
\label{\detokenize{content/04_PythonEssentials/Intro_Jupyter_notebook:introduction-to-jupyter-notebook}}\label{\detokenize{content/04_PythonEssentials/Intro_Jupyter_notebook::doc}}
\sphinxAtStartPar
Jupyter Notebook is a web application for creating, annotating, simulating and working with computational documents.  Originally developed for python, the latest versions of EViews also support Jupyter Notebooks. Jupyter Notebook offers a simple, streamlined, document\sphinxhyphen{}centric experience and can be a great environment for documenting the work you are doing, and trying alternative methods of achieving desirable results.  Many of the methods in \sphinxcode{\sphinxupquote{modelflow}} have been developed to work well with Jupyter Notebook. Indeed this documentation was written as a series of Jupyter Notebooks bound together with \sphinxhref{https://jupyterbook.org/en/stable/intro.html}{Jupyter Book}.

\sphinxAtStartPar
Jupyter Notebook is not the only way to work with modelflow or Python.  As users become more advanced they are likely to migrate to a more program\sphinxhyphen{}centric IDE (Interactive Development Environment) like Spyder or Microsoft Visual Code.

\sphinxAtStartPar
However, to start Jupyter Notebooks are a great way to learn, follow work done by others, and tweak them to fit your own needs.

\sphinxAtStartPar
There are many fine tutorials on Jupyter Notebook on the web, and \sphinxhref{https://docs.jupyter.org/en/latest/}{The official Jupyter site} is a good starting point. The following aims to provide enough information to get a user started.  Another good reference is \sphinxhref{https://jupyter.brynmawr.edu/services/public/dblank/Jupyter\%20Notebook\%20Users\%20Manual.ipynb}{here.}


\section{Starting Jupyter Notebook}
\label{\detokenize{content/04_PythonEssentials/Intro_Jupyter_notebook:starting-jupyter-notebook}}
\sphinxAtStartPar
Each time, a user wants to work with \sphinxcode{\sphinxupquote{modelflow}}, they will need to activate the \sphinxcode{\sphinxupquote{modelflow}} environment by
\begin{enumerate}
\sphinxsetlistlabels{\arabic}{enumi}{enumii}{}{)}%
\item {} 
\sphinxAtStartPar
Opening the Anaconda command prompt window

\item {} 
\sphinxAtStartPar
Activate the ModelFlow environment we just created by executing the following command: \sphinxcode{\sphinxupquote{conda activate modelflow}}

\item {} 
\sphinxAtStartPar
Navigate to a position on his/her computer’s directory structure where they want to store their jupyter notebooks.

\end{enumerate}

\begin{sphinxadmonition}{warning}{Warning:}
\sphinxAtStartPar
Note the directory from which you execute the \sphinxcode{\sphinxupquote{jupyter notebook}} \sphinxstylestrong{mfbook} in the example above will be the \sphinxstylestrong{root} directory for the jupyter session.  \sphinxstylestrong{Only directories and files below this root directory will be accessible by jupyter}.
\end{sphinxadmonition}

\sphinxAtStartPar
From here, any number of mechanisms can be used to interact with \sphinxcode{\sphinxupquote{modelflow}} and World Bank models.

\sphinxAtStartPar
\sphinxstylestrong{To use Jupyter Notebook} the Jupyter notebook, must be first started.  Following steps 1\sphinxhyphen{}3 above, a user would need to execute from the conda command line:

\sphinxAtStartPar
\sphinxcode{\sphinxupquote{jupyter notebook}}

\sphinxAtStartPar
This will launch the Jupyter environment in your default web browser, which should look something like the image below, where the directory structure presented is that of the directory from the \sphinxcode{\sphinxupquote{jupyter notebook}} command was executed.

\sphinxAtStartPar
\sphinxincludegraphics{{NewJNSession}.png}

\sphinxAtStartPar
Thismay need to ne junked if we cant make it more user friendly


\section{Creating a .lnk file to start Jupyter with modelflow.}
\label{\detokenize{content/04_PythonEssentials/Intro_Jupyter_notebook:creating-a-lnk-file-to-start-jupyter-with-modelflow}}
\sphinxAtStartPar
It can be convenient to create a icon on the desktop which start up jupyter in a modelflow environment.

\sphinxAtStartPar
First right click on teh desktop background, this will bring up the following menu.  Select New and then Shortcut.

\sphinxAtStartPar
This is done by creating a new shortcut where the \sphinxstylestrong{destination/target} is:


\section{for Anaconda}
\label{\detokenize{content/04_PythonEssentials/Intro_Jupyter_notebook:for-anaconda}}
\begin{sphinxVerbatim}[commandchars=\\\{\}]
\PYG{o}{\PYGZpc{}}\PYG{n}{windir}\PYG{o}{\PYGZpc{}}\PYGZbs{}\PYG{n}{System32}\PYGZbs{}\PYG{n}{cmd}\PYG{o}{.}\PYG{n}{exe} \PYG{l+s+s2}{\PYGZdq{}}\PYG{l+s+s2}{/K}\PYG{l+s+s2}{\PYGZdq{}} \PYG{o}{\PYGZpc{}}\PYG{n}{USERPROFILE}\PYG{o}{\PYGZpc{}}\PYGZbs{}\PYG{n}{miniconda3}\PYGZbs{}\PYG{n}{Scripts}\PYGZbs{}\PYG{n}{activate}\PYG{o}{.}\PYG{n}{bat} \PYG{o}{\PYGZpc{}}\PYG{n}{USERPROFILE}\PYG{o}{\PYGZpc{}}\PYGZbs{}\PYG{n}{miniconda3}\PYGZbs{}\PYG{n}{envs}\PYGZbs{}\PYG{n}{modelflow}\PYG{o}{\PYGZam{}}\PYG{o}{\PYGZam{}}\PYG{n}{jupyter} \PYG{n}{notebook}
\end{sphinxVerbatim}

\sphinxAtStartPar
This will create a command window, activate modelflow and start jupyter notebook

\sphinxAtStartPar
and the \sphinxstylestrong{Start in}  is set to the folder where you want jupyter to start from. 
Be aware that jupyter can’t access notebooks “before” the \sphinxcode{\sphinxupquote{Start in}} location. \sphinxcode{\sphinxupquote{C:\textbackslash{}}} could be a choice.


\section{for Miniconda}
\label{\detokenize{content/04_PythonEssentials/Intro_Jupyter_notebook:for-miniconda}}
\begin{sphinxVerbatim}[commandchars=\\\{\}]
\PYG{o}{\PYGZpc{}}\PYG{n}{windir}\PYG{o}{\PYGZpc{}}\PYGZbs{}\PYG{n}{System32}\PYGZbs{}\PYG{n}{cmd}\PYG{o}{.}\PYG{n}{exe} \PYG{l+s+s2}{\PYGZdq{}}\PYG{l+s+s2}{/K}\PYG{l+s+s2}{\PYGZdq{}} \PYG{o}{\PYGZpc{}}\PYG{n}{USERPROFILE}\PYG{o}{\PYGZpc{}}\PYGZbs{}\PYG{n}{miniconda3}\PYGZbs{}\PYG{n}{Scripts}\PYGZbs{}\PYG{n}{activate}\PYG{o}{.}\PYG{n}{bat} \PYG{o}{\PYGZpc{}}\PYG{n}{USERPROFILE}\PYG{o}{\PYGZpc{}}\PYGZbs{}\PYG{n}{miniconda3}\PYGZbs{}\PYG{n}{envs}\PYGZbs{}\PYG{n}{modelflow}\PYG{o}{\PYGZam{}}\PYG{o}{\PYGZam{}}\PYG{n}{jupyter} \PYG{n}{notebook}
\end{sphinxVerbatim}

\sphinxAtStartPar
This will create a command window, activate modelflow and start jupyter notebook

\sphinxAtStartPar
and the \sphinxstylestrong{Start in}  is set to the folder where you want jupyter to start from. 
Be aware that jupyter can’t access notebooks “before” the \sphinxcode{\sphinxupquote{Start in}} location. \sphinxcode{\sphinxupquote{C:\textbackslash{}}} could be a choice.


\section{Creating a notebook}
\label{\detokenize{content/04_PythonEssentials/Intro_Jupyter_notebook:creating-a-notebook}}
\sphinxAtStartPar
The idea behind jupyter notebook was to create an interactive version of the notebooks that scientists use(d) to:
\begin{itemize}
\item {} 
\sphinxAtStartPar
record what they have done

\item {} 
\sphinxAtStartPar
perhaps explain why

\item {} 
\sphinxAtStartPar
document how data was generated, and

\item {} 
\sphinxAtStartPar
record the results of their experiments

\end{itemize}

\sphinxAtStartPar
The motivation for these notebooks and Jupyter notebook is to record the precise steps taken to produce a set of results, which if followed by others would allow the to generate the same results.

\sphinxAtStartPar
To create a notebook you must select from the Jupyter Notebook menu

\sphinxAtStartPar
File\sphinxhyphen{}> New Notebook

\begin{figure}[htbp]
\centering
\capstart

\noindent\sphinxincludegraphics[height=150\sphinxpxdimen]{{NewNotebook}.png}
\caption{A newly created Jupyter Notebook session}\label{\detokenize{content/04_PythonEssentials/Intro_Jupyter_notebook:new-notebook}}\end{figure}

\sphinxAtStartPar
This will generate a blank unnamed notebook with one empty cell, that looks something like this:

\begin{figure}[htbp]
\centering
\capstart

\noindent\sphinxincludegraphics[height=225\sphinxpxdimen]{{Newcell}.png}
\caption{A newly created Jupyter Notebook}\label{\detokenize{content/04_PythonEssentials/Intro_Jupyter_notebook:id1}}\end{figure}

\begin{sphinxadmonition}{warning}{Warning:}
\sphinxAtStartPar
Each notebook has associated with it a “Kernel”, which is an instance of the computing environment in which code will be executed. For Jupyter Notebooks that work with \sphinxcode{\sphinxupquote{modelflow}} this will be a Python Kernel. If your computer has more than one “kernel’s” installed on it, you may be prompted when creating a new notebook for the kernel with which to associate it.  Typically this should be the Python Kernel under which your modelflow was built – currently python 3.9 in April 2023.
\end{sphinxadmonition}


\section{Jupter Notebook cells}
\label{\detokenize{content/04_PythonEssentials/Intro_Jupyter_notebook:jupter-notebook-cells}}
\sphinxAtStartPar
A Jupyter Notebook is comprised of a series of cells.

\sphinxAtStartPar
\sphinxstyleemphasis{\sphinxstylestrong{Jupyter Notebook cells can contain:}}
\begin{itemize}
\item {} 
\sphinxAtStartPar
\sphinxstylestrong{computer code} (typically python code, but as noted other kernels – like Eviews – can be used with jupyter).

\item {} 
\sphinxAtStartPar
\sphinxstylestrong{markdown text}: plain text that can include special characters that make some text appear as bold, or indicate the text is a header, or instruct Jupyter Notebook to render the text as a mathematical formula.  All of the text in this document was entered using Jupyter Notebook’s markdown language (more on this below).

\item {} 
\sphinxAtStartPar
Results (in the form of tables or graphs) from the execution of computer code specified in a code cell.

\end{itemize}

\sphinxAtStartPar
\sphinxstylestrong{Every cell has two modes:}
\begin{enumerate}
\sphinxsetlistlabels{\arabic}{enumi}{enumii}{}{.}%
\item {} 
\sphinxAtStartPar
Edit mode – indicated by a green vertical bar. In edit mode, the user can change the code, or the markdown.

\item {} 
\sphinxAtStartPar
Select/Copy mode – indicated by a blue vertical bar.  This will be the state of the cell when its content has been executed.  For markdown cells this means that the text and special characters have been rendered into formatted text.  For code cells, this means the code has been executed and its output (if any)is displayed in an output cell that will be generated in the space immediately below the code cell that generated the output.

\end{enumerate}

\sphinxAtStartPar
\sphinxstylestrong{Users can switch between Edit and Select/Copy Mode by hitting Enter}

\begin{sphinxadmonition}{note}{Note:}
\sphinxAtStartPar
Jupyter Notebooks were designed to facilitate \sphinxstyleemphasis{replicability}: the idea that a scientific analysis should contain \sphinxhyphen{} in addition to the final output (text, graphs, tables) \sphinxhyphen{} all the computational steps needed to get from raw input data to the results.
\end{sphinxadmonition}


\subsection{How to add, delete and move cells}
\label{\detokenize{content/04_PythonEssentials/Intro_Jupyter_notebook:how-to-add-delete-and-move-cells}}
\sphinxAtStartPar
The newly created Jupyter Notebook will have a code cell by default.  Cells can be added, deleted and moved either via mouse using the toolbar or by keyboard shortcut.

\sphinxAtStartPar
\sphinxstylestrong{Using the Toolbar}
\begin{itemize}
\item {} 
\sphinxAtStartPar
\sphinxstylestrong{+ button}: add a cell below the current cell

\item {} 
\sphinxAtStartPar
\sphinxstylestrong{scissors}: cut  current cell (can be undone from “Edit” tab)

\item {} 
\sphinxAtStartPar
\sphinxstylestrong{clipboard}: paste a previously cut cell to the current location

\item {} 
\sphinxAtStartPar
\sphinxstylestrong{up\sphinxhyphen{} and down arrows}: move cells (cell must be in Select/Copy mode – vertical side bar must be blue)

\item {} 
\sphinxAtStartPar
\sphinxstylestrong{hold shift + click cells in left margin}: select multiple cells (vertical bar must be blue)

\end{itemize}

\sphinxAtStartPar
\sphinxstylestrong{Using keyboard short cuts}
\begin{itemize}
\item {} 
\sphinxAtStartPar
\sphinxstylestrong{esc + a}: add a cell above the current cell

\item {} 
\sphinxAtStartPar
\sphinxstylestrong{esc + b}: add a cell below the current cell

\item {} 
\sphinxAtStartPar
\sphinxstylestrong{esc + d+d}: delete the current cell

\end{itemize}


\subsection{Change the type of a cell}
\label{\detokenize{content/04_PythonEssentials/Intro_Jupyter_notebook:change-the-type-of-a-cell}}
\sphinxAtStartPar
You can also change the type of a cell. New cells are by default “code” cells.

\sphinxAtStartPar
\sphinxstylestrong{Using the Toolbar}
\begin{itemize}
\item {} 
\sphinxAtStartPar
Select the desired type from the drop down.  options include
\begin{itemize}
\item {} 
\sphinxAtStartPar
Markdown

\item {} 
\sphinxAtStartPar
Code

\item {} 
\sphinxAtStartPar
Raw NBConvert

\item {} 
\sphinxAtStartPar
Heading

\end{itemize}

\end{itemize}

\sphinxAtStartPar
\sphinxstylestrong{Using keyboard short cuts}
\begin{itemize}
\item {} 
\sphinxAtStartPar
\sphinxstylestrong{esc + m}: make the current cell a markdown cell

\item {} 
\sphinxAtStartPar
\sphinxstylestrong{esc + y}: make the current cell a code  cell

\end{itemize}

\sphinxAtStartPar
\sphinxstylestrong{Auto\sphinxhyphen{}complete and context\sphinxhyphen{}sensitive help}

\sphinxAtStartPar
When editing a code cell, you can use these short\sphinxhyphen{}cuts to autocomplete and or call up documentation for a command.
\begin{itemize}
\item {} 
\sphinxAtStartPar
\sphinxstylestrong{tab}: autocomplete and  method selection

\item {} 
\sphinxAtStartPar
\sphinxstylestrong{double tab}: documention (double tab for full doc)

\end{itemize}


\section{Execution of cells}
\label{\detokenize{content/04_PythonEssentials/Intro_Jupyter_notebook:execution-of-cells}}
\sphinxAtStartPar
Every cell in a Jupyter Notebook can be executed. Executing a markdown cell will cause the cell’s content to be rendered as html.  Executing a python code cell, will cause its content to be executed. Cells can be executed either by using the Run button on the Jupyter Notebook menu, or by using one of \sphinxstylestrong{two keyboard shortcuts}:
\begin{itemize}
\item {} 
\sphinxAtStartPar
\sphinxstylestrong{ctrl + Enter}: Executes the code in the cell or formats the markdown of a cell.  The current cell retains the focus – cursor stays on cell executed.

\item {} 
\sphinxAtStartPar
\sphinxstylestrong{shift + enter}: Executes the code in the cell or formats the markdown of a cell. Focus (cursor) jumps to the next cell

\end{itemize}

\sphinxAtStartPar
For other useful shortcuts see “Help” => “Keyboard Shortcuts” or simply press keyboard icon in the toolbar.


\subsection{Executing python code}
\label{\detokenize{content/04_PythonEssentials/Intro_Jupyter_notebook:executing-python-code}}
\sphinxAtStartPar
Below is a code cel with some standard python that declares a variable “x”, assigns it the value 10, declares a second variable “y” and assigns it the value 45.  The final line of y alone, instructs python to display the value of the variable y.  The results of the operation appear in Jupyter Notebook as an output cell Out{[}\#{]}.  By pressing \sphinxstylestrong{Ctrl\sphinxhyphen{}Enter} the code will be executed and the output displayed below.

\begin{sphinxuseclass}{cell}\begin{sphinxVerbatimInput}

\begin{sphinxuseclass}{cell_input}
\begin{sphinxVerbatim}[commandchars=\\\{\}]
\PYG{n}{x} \PYG{o}{=} \PYG{l+m+mi}{10}
\PYG{n}{y} \PYG{o}{=} \PYG{l+m+mi}{45}
\PYG{n}{y}
\end{sphinxVerbatim}

\end{sphinxuseclass}\end{sphinxVerbatimInput}
\begin{sphinxVerbatimOutput}

\begin{sphinxuseclass}{cell_output}
\begin{sphinxVerbatim}[commandchars=\\\{\}]
45
\end{sphinxVerbatim}

\end{sphinxuseclass}\end{sphinxVerbatimOutput}

\end{sphinxuseclass}
\sphinxAtStartPar
\sphinxstylestrong{The semi\sphinxhyphen{}colon “;” suppresses output in Jupyter Notebook}

\sphinxAtStartPar
In the example below, a semi\sphinxhyphen{}colon “;” has been appended to the final line.  This suppresses the display of the value contained by y;  As a result there is no output cell.

\begin{sphinxuseclass}{cell}\begin{sphinxVerbatimInput}

\begin{sphinxuseclass}{cell_input}
\begin{sphinxVerbatim}[commandchars=\\\{\}]
\PYG{n}{x} \PYG{o}{=} \PYG{l+m+mi}{10}
\PYG{n}{y} \PYG{o}{=} \PYG{l+m+mi}{45}
\PYG{n}{y}\PYG{p}{;}
\end{sphinxVerbatim}

\end{sphinxuseclass}\end{sphinxVerbatimInput}

\end{sphinxuseclass}
\sphinxAtStartPar
Another way to display results is to use the print function.

\begin{sphinxuseclass}{cell}\begin{sphinxVerbatimInput}

\begin{sphinxuseclass}{cell_input}
\begin{sphinxVerbatim}[commandchars=\\\{\}]
\PYG{n}{x} \PYG{o}{=} \PYG{l+m+mi}{10}
\PYG{n+nb}{print}\PYG{p}{(}\PYG{n}{x}\PYG{p}{)}
\end{sphinxVerbatim}

\end{sphinxuseclass}\end{sphinxVerbatimInput}
\begin{sphinxVerbatimOutput}

\begin{sphinxuseclass}{cell_output}
\begin{sphinxVerbatim}[commandchars=\\\{\}]
10
\end{sphinxVerbatim}

\end{sphinxuseclass}\end{sphinxVerbatimOutput}

\end{sphinxuseclass}
\sphinxAtStartPar
Variables in a Jupyter Notebook session are persistent, as a result in the subsequent cell, we can declare a variable ‘z’ equal to 2*y and it will have the value 90.

\begin{sphinxuseclass}{cell}\begin{sphinxVerbatimInput}

\begin{sphinxuseclass}{cell_input}
\begin{sphinxVerbatim}[commandchars=\\\{\}]
\PYG{n}{z}\PYG{o}{=}\PYG{n}{y}\PYG{o}{*}\PYG{l+m+mi}{2}
\PYG{n}{z}
\end{sphinxVerbatim}

\end{sphinxuseclass}\end{sphinxVerbatimInput}
\begin{sphinxVerbatimOutput}

\begin{sphinxuseclass}{cell_output}
\begin{sphinxVerbatim}[commandchars=\\\{\}]
90
\end{sphinxVerbatim}

\end{sphinxuseclass}\end{sphinxVerbatimOutput}

\end{sphinxuseclass}
\begin{sphinxadmonition}{note}{Note:}
\sphinxAtStartPar
Cells can be executed in any order. The natural order is to execute them sequentially in the order they appear in the Jupyter notebook.  However, when debugging or developing code it may be useful to execute cells out of their natural order.\\
The persistence of data (the fact that a variable \sphinxcode{\sphinxupquote{y}} defined in one cell can be used in another cell) depends on the order in which cells were executed, not the order they appear in the notebook.
\end{sphinxadmonition}


\subsection{Markdown cells and the markdown scripting language in Jupyter Notebook}
\label{\detokenize{content/04_PythonEssentials/Intro_Jupyter_notebook:markdown-cells-and-the-markdown-scripting-language-in-jupyter-notebook}}
\sphinxAtStartPar
Text cells in a notebook can be made more interesting by using markdown.

\sphinxAtStartPar
Cells designated as markdown cells when executed are rendered in a rich text format (html).

\sphinxAtStartPar
Markdown is a lightweight markup language for creating formatted text using a plain\sphinxhyphen{}text editor.  Used in a markdown cell of Jupyter Notebook it can be used to produce nicely formatted text that mixes text, mathematical formulae, code and outputs from executed python code.

\sphinxAtStartPar
Rather than the relatively complex commands of html <h1></h1>, markdown uses a simplified set of commands to control how text elements should be rendered.


\subsubsection{Common markdown commands}
\label{\detokenize{content/04_PythonEssentials/Intro_Jupyter_notebook:common-markdown-commands}}
\sphinxAtStartPar
Some of the most common of these include:


\begin{savenotes}\sphinxattablestart
\centering
\begin{tabulary}{\linewidth}[t]{|T|T|}
\hline
\sphinxstyletheadfamily 
\sphinxAtStartPar
symbol
&\sphinxstyletheadfamily 
\sphinxAtStartPar
Effect
\\
\hline
\sphinxAtStartPar
\#
&
\sphinxAtStartPar
Header
\\
\hline
\sphinxAtStartPar
\#\#
&
\sphinxAtStartPar
second level
\\
\hline
\sphinxAtStartPar
\#\#\#
&
\sphinxAtStartPar
third level etc.
\\
\hline
\sphinxAtStartPar
**Bold text**
&
\sphinxAtStartPar
\sphinxstylestrong{Bold text}
\\
\hline
\sphinxAtStartPar
*Italics text*
&
\sphinxAtStartPar
\sphinxstyleemphasis{Italics text}
\\
\hline
\sphinxAtStartPar
* text
&
\sphinxAtStartPar
Bulleted text or dot notes
\\
\hline
\sphinxAtStartPar
1. text
&
\sphinxAtStartPar
1. Numbered bullets
\\
\hline
\end{tabulary}
\par
\sphinxattableend\end{savenotes}


\subsubsection{Tables in markdown}
\label{\detokenize{content/04_PythonEssentials/Intro_Jupyter_notebook:tables-in-markdown}}
\sphinxAtStartPar
Tables like the one above can be constructed using | as separators.

\sphinxAtStartPar
Below is the markdown code that generated the above table:

\begin{sphinxVerbatim}[commandchars=\\\{\}]
\PYG{o}{|} \PYG{n}{symbol}           \PYG{o}{|} \PYG{n}{Effect}          \PYG{o}{|}
\PYG{o}{|}\PYG{p}{:}\PYG{o}{\PYGZhy{}}\PYG{o}{\PYGZhy{}}\PYG{o}{|}\PYG{p}{:}\PYG{o}{\PYGZhy{}}\PYG{o}{\PYGZhy{}}\PYG{o}{|}                                \PYG{c+c1}{\PYGZsh{} Specifies the justification for the columns of the table.}
\PYG{o}{|} \PYGZbs{}\PYG{c+c1}{\PYGZsh{}               | Header        |}
\PYG{o}{|} \PYGZbs{}\PYG{c+c1}{\PYGZsh{}\PYGZbs{}\PYGZsh{}             | second level |}
\PYG{o}{|} \PYGZbs{}\PYG{o}{*}\PYGZbs{}\PYG{o}{*}\PYG{n}{Bold} \PYG{n}{text}\PYGZbs{}\PYG{o}{*}\PYGZbs{}\PYG{o}{*} \PYG{o}{|} \PYG{o}{*}\PYG{o}{*}\PYG{n}{Bold} \PYG{n}{text}\PYG{o}{*}\PYG{o}{*}   \PYG{o}{|}
\PYG{o}{|} \PYGZbs{}\PYG{o}{*}\PYG{n}{Italics} \PYG{n}{text}\PYGZbs{}\PYG{o}{*} \PYG{o}{|} \PYG{o}{*}\PYG{n}{Italics} \PYG{n}{text}\PYG{o}{*}   \PYG{o}{|}
\PYG{o}{|} 
\PYG{o}{|} \PYG{l+m+mi}{1}\PYGZbs{}\PYG{o}{.} \PYG{n}{text}  \PYG{o}{|} \PYG{l+m+mf}{1.} \PYG{n}{Numbered} \PYG{n}{bullets}   \PYG{o}{|}

\end{sphinxVerbatim}

\sphinxAtStartPar
The |:–|:–| on the second line tells the Table generator how to justify the contents of columns.


\begin{savenotes}\sphinxattablestart
\centering
\begin{tabulary}{\linewidth}[t]{|T|T|}
\hline
\sphinxstyletheadfamily 
\sphinxAtStartPar
Symbol
&\sphinxstyletheadfamily 
\sphinxAtStartPar
Meaning
\\
\hline
\sphinxAtStartPar
\sphinxcode{\sphinxupquote{:\sphinxhyphen{}\sphinxhyphen{}}}
&
\sphinxAtStartPar
left justify
\\
\hline
\sphinxAtStartPar
\sphinxcode{\sphinxupquote{:\sphinxhyphen{}\sphinxhyphen{}:}}
&
\sphinxAtStartPar
center justify
\\
\hline
\sphinxAtStartPar
\sphinxcode{\sphinxupquote{\sphinxhyphen{}\sphinxhyphen{}:}}
&
\sphinxAtStartPar
right justify.
\\
\hline
\end{tabulary}
\par
\sphinxattableend\end{savenotes}


\subsection{Displaying code}
\label{\detokenize{content/04_PythonEssentials/Intro_Jupyter_notebook:displaying-code}}
\sphinxAtStartPar
To display a  block of (unexecutable) code within a markdown cell, encapsulate it (surround it) with backticks `.


\subsubsection{Inine code blocks}
\label{\detokenize{content/04_PythonEssentials/Intro_Jupyter_notebook:inine-code-blocks}}
\sphinxAtStartPar
For inline code references ‘ a single back tick at the beginning and end suffices.  For example, the below line

\sphinxAtStartPar
An example sentence with some back\sphinxhyphen{}ticked `text as code` in the middle, \sphinxstylestrong{will render as:} An example sentence with some back\sphinxhyphen{}ticked \sphinxcode{\sphinxupquote{text as code}} in the middle.


\subsubsection{Multiline code block}
\label{\detokenize{content/04_PythonEssentials/Intro_Jupyter_notebook:multiline-code-block}}
\sphinxAtStartPar
For a multiline section of code use three backticks at the beginning and end.

\sphinxAtStartPar
The below block of code:

\sphinxAtStartPar
\sphinxcode{\sphinxupquote{```}}

\sphinxAtStartPar
\sphinxcode{\sphinxupquote{Multi line }}

\sphinxAtStartPar
\sphinxcode{\sphinxupquote{text to be rendered as code }}

\sphinxAtStartPar
\sphinxcode{\sphinxupquote{```}}

\sphinxAtStartPar
will render as:

\begin{sphinxVerbatim}[commandchars=\\\{\}]
\PYG{n}{Multi} \PYG{n}{line} 
\PYG{n}{text} \PYG{n}{to} \PYG{n}{be} \PYG{n}{rendered} \PYG{k}{as} \PYG{n}{code} 
\end{sphinxVerbatim}


\subsection{Rendering mathematics in markdown}
\label{\detokenize{content/04_PythonEssentials/Intro_Jupyter_notebook:rendering-mathematics-in-markdown}}
\sphinxAtStartPar
Jupyter Notebook’s implementation of Markdown supports \sphinxcode{\sphinxupquote{latex}} mathematical notation.

\sphinxAtStartPar
Inline enclose the \sphinxcode{\sphinxupquote{latex}} code in \sphinxcode{\sphinxupquote{\$}}:

\sphinxAtStartPar
An Equation: \sphinxcode{\sphinxupquote{\$y\_t = \textbackslash{}beta\_0 + \textbackslash{}beta\_1 x\_t + u\_t\textbackslash{}\$}} will renders as: \(y_t = \beta_0 + \beta_1 x_t + u_t\)

\sphinxAtStartPar
if enclosed in \sphinxcode{\sphinxupquote{\$\$}} \sphinxcode{\sphinxupquote{\$\$}}, i.e.  as

\begin{sphinxVerbatim}[commandchars=\\\{\}]
\PYGZdl{}\PYGZdl{}y\PYGZus{}t = \PYGZbs{}beta\PYGZus{}0 + \PYGZbs{}beta\PYGZus{}1 x\PYGZus{}t + u\PYGZus{}t\PYGZdl{}\PYGZdl{}
\end{sphinxVerbatim}

\sphinxAtStartPar
it will be rendered centered on its own line – as here.
\begin{equation*}
\begin{split}y_t = \beta_0 + \beta_1 x_t + u_t\end{split}
\end{equation*}

\subsubsection{Complex and multi\sphinxhyphen{}line math}
\label{\detokenize{content/04_PythonEssentials/Intro_Jupyter_notebook:complex-and-multi-line-math}}
\begin{sphinxVerbatim}[commandchars=\\\{\}]
\PYGZbs{}\PYG{n}{begin}\PYG{p}{\PYGZob{}}\PYG{n}{align}\PYG{o}{*}\PYG{p}{\PYGZcb{}}
\PYG{n}{Y\PYGZus{}t}  \PYG{o}{\PYGZam{}}\PYG{o}{=}  \PYG{n}{C\PYGZus{}t}\PYG{o}{+}\PYG{n}{I\PYGZus{}t}\PYG{o}{+}\PYG{n}{G}\PYG{o}{+}\PYG{n}{t}\PYG{o}{+} \PYG{p}{(}\PYG{n}{X\PYGZus{}t}\PYG{o}{\PYGZhy{}}\PYG{n}{M\PYGZus{}t}\PYG{p}{)} \PYGZbs{}\PYGZbs{}
\PYG{n}{C\PYGZus{}t} \PYG{o}{\PYGZam{}}\PYG{o}{=} \PYG{n}{c\PYGZus{}t}\PYG{p}{(}\PYG{n}{C\PYGZus{}}\PYG{p}{\PYGZob{}}\PYG{n}{t}\PYG{o}{\PYGZhy{}}\PYG{l+m+mi}{1}\PYG{p}{\PYGZcb{}}\PYG{p}{,}\PYG{n}{C\PYGZus{}}\PYG{p}{\PYGZob{}}\PYG{n}{t}\PYG{o}{\PYGZhy{}}\PYG{l+m+mi}{2}\PYG{p}{\PYGZcb{}}\PYG{p}{,}\PYG{n}{I\PYGZus{}t}\PYG{p}{,}\PYG{n}{G\PYGZus{}t}\PYG{p}{,}\PYG{n}{X\PYGZus{}t}\PYG{p}{,}\PYG{n}{M\PYGZus{}t}\PYG{p}{,}\PYG{n}{P\PYGZus{}t}\PYG{p}{)}\PYGZbs{}\PYGZbs{}
\PYG{n}{I\PYGZus{}t} \PYG{o}{\PYGZam{}}\PYG{o}{=} \PYG{n}{c\PYGZus{}t}\PYG{p}{(}\PYG{n}{I\PYGZus{}}\PYG{p}{\PYGZob{}}\PYG{n}{t}\PYG{o}{\PYGZhy{}}\PYG{l+m+mi}{1}\PYG{p}{\PYGZcb{}}\PYG{p}{,}\PYG{n}{I\PYGZus{}}\PYG{p}{\PYGZob{}}\PYG{n}{t}\PYG{o}{\PYGZhy{}}\PYG{l+m+mi}{2}\PYG{p}{\PYGZcb{}}\PYG{p}{,}\PYG{n}{C\PYGZus{}t}\PYG{p}{,}\PYG{n}{G\PYGZus{}t}\PYG{p}{,}\PYG{n}{X\PYGZus{}t}\PYG{p}{,}\PYG{n}{M\PYGZus{}t}\PYG{p}{,}\PYG{n}{P\PYGZus{}t}\PYG{p}{)}\PYGZbs{}\PYGZbs{}
\PYG{n}{G\PYGZus{}t} \PYG{o}{\PYGZam{}}\PYG{o}{=} \PYG{n}{c\PYGZus{}t}\PYG{p}{(}\PYG{n}{G\PYGZus{}}\PYG{p}{\PYGZob{}}\PYG{n}{t}\PYG{o}{\PYGZhy{}}\PYG{l+m+mi}{1}\PYG{p}{\PYGZcb{}}\PYG{p}{,}\PYG{n}{G\PYGZus{}}\PYG{p}{\PYGZob{}}\PYG{n}{t}\PYG{o}{\PYGZhy{}}\PYG{l+m+mi}{2}\PYG{p}{\PYGZcb{}}\PYG{p}{,}\PYG{n}{C\PYGZus{}t}\PYG{p}{,}\PYG{n}{I\PYGZus{}t}\PYG{p}{,}\PYG{n}{X\PYGZus{}t}\PYG{p}{,}\PYG{n}{M\PYGZus{}t}\PYG{p}{,}\PYG{n}{P\PYGZus{}t}\PYG{p}{)}\PYGZbs{}\PYGZbs{}
\PYG{n}{X\PYGZus{}t} \PYG{o}{\PYGZam{}}\PYG{o}{=} \PYG{n}{c\PYGZus{}t}\PYG{p}{(}\PYG{n}{X\PYGZus{}}\PYG{p}{\PYGZob{}}\PYG{n}{t}\PYG{o}{\PYGZhy{}}\PYG{l+m+mi}{1}\PYG{p}{\PYGZcb{}}\PYG{p}{,}\PYG{n}{X\PYGZus{}}\PYG{p}{\PYGZob{}}\PYG{n}{t}\PYG{o}{\PYGZhy{}}\PYG{l+m+mi}{2}\PYG{p}{\PYGZcb{}}\PYG{p}{,}\PYG{n}{C\PYGZus{}t}\PYG{p}{,}\PYG{n}{I\PYGZus{}t}\PYG{p}{,}\PYG{n}{G\PYGZus{}t}\PYG{p}{,}\PYG{n}{M\PYGZus{}t}\PYG{p}{,}\PYG{n}{P\PYGZus{}t}\PYG{p}{,}\PYG{n}{P}\PYG{o}{\PYGZca{}}\PYG{n}{f\PYGZus{}t}\PYG{p}{)}\PYGZbs{}\PYGZbs{}
\PYG{n}{M\PYGZus{}t} \PYG{o}{\PYGZam{}}\PYG{o}{=} \PYG{n}{c\PYGZus{}t}\PYG{p}{(}\PYG{n}{M\PYGZus{}}\PYG{p}{\PYGZob{}}\PYG{n}{t}\PYG{o}{\PYGZhy{}}\PYG{l+m+mi}{1}\PYG{p}{\PYGZcb{}}\PYG{p}{,}\PYG{n}{M\PYGZus{}}\PYG{p}{\PYGZob{}}\PYG{n}{t}\PYG{o}{\PYGZhy{}}\PYG{l+m+mi}{2}\PYG{p}{\PYGZcb{}}\PYG{p}{,}\PYG{n}{C\PYGZus{}t}\PYG{p}{,}\PYG{n}{I\PYGZus{}t}\PYG{p}{,}\PYG{n}{G\PYGZus{}t}\PYG{p}{,}\PYG{n}{X\PYGZus{}t}\PYG{p}{,}\PYG{n}{P\PYGZus{}t}\PYG{p}{,}\PYG{n}{P}\PYG{o}{\PYGZca{}}\PYG{n}{f\PYGZus{}t}\PYG{p}{)}
\PYGZbs{}\PYG{n}{end}\PYG{p}{\PYGZob{}}\PYG{n}{align}\PYG{o}{*}\PYG{p}{\PYGZcb{}}
\end{sphinxVerbatim}

\sphinxAtStartPar
The above \sphinxcode{\sphinxupquote{latex}} mathematics code uses the  \sphinxcode{\sphinxupquote{\&}} symbol to tell \sphinxcode{\sphinxupquote{latex}} to align the different lines (separated by \sphinxcode{\sphinxupquote{\textbackslash{}\textbackslash{}}}) on the character immediately after the \sphinxcode{\sphinxupquote{\&}}. In this instance the equals “=” sign.  The \sphinxcode{\sphinxupquote{*}} after align supresses equation numbering.
\begin{align*}
Y_t  &=  C_t+I_t+G+t+ (X_t-M_t) \\
C_t &= c_t(C_{t-1},C_{t-2},I_t,G_t,X_t,M_t,P_t)\\
I_t &= c_t(I_{t-1},I_{t-2},C_t,G_t,X_t,M_t,P_t)\\
G_t &= c_t(G_{t-1},G_{t-2},C_t,I_t,X_t,M_t,P_t)\\
X_t &= c_t(X_{t-1},X_{t-2},C_t,I_t,G_t,M_t,P_t,P^f_t)\\
M_t &= c_t(M_{t-1},M_{t-2},C_t,I_t,G_t,X_t,P_t,P^f_t)
\end{align*}

\subsection{links to more info on markdown}
\label{\detokenize{content/04_PythonEssentials/Intro_Jupyter_notebook:links-to-more-info-on-markdown}}
\sphinxAtStartPar
There are many very good markdown cheatsheets and tutorials on the internet, one cheetsheet is \sphinxhref{https://www.markdownguide.org/cheat-sheet/}{here}

\sphinxstepscope


\chapter{Some Python basics}
\label{\detokenize{content/04_PythonEssentials/PythonPandasDataframes:some-python-basics}}\label{\detokenize{content/04_PythonEssentials/PythonPandasDataframes::doc}}
\sphinxAtStartPar
Before using \sphinxcode{\sphinxupquote{modelflow}} with the World Bank’s MFMod models, users  will have to understand at least some basic elements of \sphinxcode{\sphinxupquote{python}} syntax and usage.  Notably they will need to understand about packages, libraries and classes, and how to access them.


\section{Starting python in windows}
\label{\detokenize{content/04_PythonEssentials/PythonPandasDataframes:starting-python-in-windows}}
\sphinxAtStartPar
To begin using \sphinxcode{\sphinxupquote{modelflow}}, python itself needs to be started.  This can be done either using the \sphinxcode{\sphinxupquote{Anaconda}} navigator or from the command line shell. In either case, the user will need to start python and select the \sphinxcode{\sphinxupquote{modelflow}} environment.


\section{Anaconda navigator}
\label{\detokenize{content/04_PythonEssentials/PythonPandasDataframes:anaconda-navigator}}\begin{enumerate}
\sphinxsetlistlabels{\arabic}{enumi}{enumii}{}{.}%
\item {} 
\sphinxAtStartPar
Start Anaconda Navigator by typing Anaconda in the Start window and opening the Navigator (see Figure).

\item {} 
\sphinxAtStartPar
From Anaconda Navigator select the \sphinxcode{\sphinxupquote{Modelflow}} environment (see figure)

\end{enumerate}

\begin{figure}[htbp]
\centering
\capstart

\noindent\sphinxincludegraphics[height=225\sphinxpxdimen]{{AnacondaNav1}.png}
\caption{A newly created Jupyter Notebook session}\label{\detokenize{content/04_PythonEssentials/PythonPandasDataframes:start-anaconda-navigator}}\end{figure}
\begin{enumerate}
\sphinxsetlistlabels{\arabic}{enumi}{enumii}{}{.}%
\setcounter{enumi}{2}
\item {} 
\sphinxAtStartPar
Once the environment is selected the user can either select a command line environment or start jupyter notebook by clicking on either the
\begin{enumerate}
\sphinxsetlistlabels{\arabic}{enumii}{enumiii}{}{.}%
\item {} 
\sphinxAtStartPar
Jupyter Notebook environment

\item {} 
\sphinxAtStartPar
The command line environment

\item {} 
\sphinxAtStartPar
A programming IDE environment

\end{enumerate}

\end{enumerate}

\begin{figure}[htbp]
\centering
\capstart

\noindent\sphinxincludegraphics[height=225\sphinxpxdimen]{{NavigatorChoices}.png}
\caption{A newly created Jupyter Notebook session}\label{\detokenize{content/04_PythonEssentials/PythonPandasDataframes:id1}}\end{figure}


\section{Python  packages, libraries and classes}
\label{\detokenize{content/04_PythonEssentials/PythonPandasDataframes:python-packages-libraries-and-classes}}
\sphinxAtStartPar
Some features of \sphinxcode{\sphinxupquote{python}} are built\sphinxhyphen{}in out\sphinxhyphen{}of\sphinxhyphen{}the\sphinxhyphen{}box.  Others build up on these basic features.

\sphinxAtStartPar
A \sphinxstylestrong{python class} is a code template that defines a python object. Classes can have properties {[}variables or data{]} associated with them and methods (behaviours or functions) associated with them. In python, a class is created by the keyword class. An object of type class is created (instantiated) using the class’s “constructor” – a special method that creates an object that is an instance of a class.

\sphinxAtStartPar
A \sphinxstylestrong{module} is a Python object consisting of Python code. A module can define functions, classes and variables. A module can also include runnable code.

\sphinxAtStartPar
A \sphinxstylestrong{python package} is a collection of modules that are related to each other. When a module from an external package is required by a program, that package (or module in the package) must  be \sphinxstylestrong{imported} into the current session in order for its modules to be accessible.

\sphinxAtStartPar
A \sphinxstylestrong{python library} is a collection of related modules or packages.

\sphinxAtStartPar
\sphinxcode{\sphinxupquote{Modelflow}} is a python package that \sphinxstyleemphasis{inherits} (build on or adds to) the methods and properties of other \sphinxcode{\sphinxupquote{python}} classes like \sphinxcode{\sphinxupquote{pandas}}, \sphinxcode{\sphinxupquote{numpy}} and \sphinxcode{\sphinxupquote{mathplotlib}}.


\section{Importing packages, libraries, modules and classes}
\label{\detokenize{content/04_PythonEssentials/PythonPandasDataframes:importing-packages-libraries-modules-and-classes}}
\sphinxAtStartPar
Some libraries, packages, and modules are part of the core python package and will be available (importable) from the get\sphinxhyphen{}go.  Others are not, and need to be installed before importing them into a session.

\sphinxAtStartPar
If you followed the modelflow installation instructions you have already downloaded and installed on your computer all the packages necessary for running World Bank models under modelflow.  But to work with them in a given Jupyter Notebook session or in a program context, you will also need to \sphinxcode{\sphinxupquote{import}} them into your session before you call them.

\begin{sphinxadmonition}{note}{Note:}
\sphinxAtStartPar
\sphinxstylestrong{Installation} of a package is not the same as \sphinxstylestrong{import}ing a package. Installation downloads a package’s programs from the internet onto the user’s machine, making them available to be imported when required.  To be imported a package must be installed once on the computer that wishes to use it.  Once it has been installed, the package must be imported into each python session where it is to be used.
\end{sphinxadmonition}

\sphinxAtStartPar
Typically a python program will start with the importation of the libraries, classes and modules that will be used.  Because a Jupyter Notebook is essentially a heavily annotated program, it also requires that packages used be imported.

\sphinxAtStartPar
As described above: packages, libraries and modules are containers that can include other elements.  Take for example the package Math.

\sphinxAtStartPar
To import the Math Package we execute the command \sphinxcode{\sphinxupquote{ import math}}.  Having done that we can can call the functions and data that are defined in it.

\begin{sphinxuseclass}{cell}\begin{sphinxVerbatimInput}

\begin{sphinxuseclass}{cell_input}
\begin{sphinxVerbatim}[commandchars=\\\{\}]
\PYG{c+c1}{\PYGZsh{} the \PYGZdq{}\PYGZsh{}\PYGZdq{} in a code cell indicates a comment, test after the \PYGZsh{} will not be executed}
\PYG{k+kn}{import} \PYG{n+nn}{math}

\PYG{c+c1}{\PYGZsh{} Now that we have imported math we can access some of the elements identified in }
\PYG{c+c1}{\PYGZsh{} the package.}
\PYG{c+c1}{\PYGZsh{} For example math contains a definition for pi, we can access that by executing }
\PYG{c+c1}{\PYGZsh{} the pi method of the library math}

\PYG{n}{math}\PYG{o}{.}\PYG{n}{pi}
\end{sphinxVerbatim}

\end{sphinxuseclass}\end{sphinxVerbatimInput}
\begin{sphinxVerbatimOutput}

\begin{sphinxuseclass}{cell_output}
\begin{sphinxVerbatim}[commandchars=\\\{\}]
3.141592653589793
\end{sphinxVerbatim}

\end{sphinxuseclass}\end{sphinxVerbatimOutput}

\end{sphinxuseclass}

\subsection{Import specific elements or classes from a module or library}
\label{\detokenize{content/04_PythonEssentials/PythonPandasDataframes:import-specific-elements-or-classes-from-a-module-or-library}}
\sphinxAtStartPar
The python package \sphinxcode{\sphinxupquote{math}} contains several functions and classes.

\sphinxAtStartPar
Rather than importing the whole package (as above), these classes can be imported directly using the \sphinxstylestrong{from syntax}.

\sphinxAtStartPar
\sphinxcode{\sphinxupquote{from math import pi,cos,sin}}

\sphinxAtStartPar
When imported in this fashion, the user does not have to precede the class or method with the name of their libary. The above \sphinxcode{\sphinxupquote{from math import pi,cos,sin}} command imports the pi constant and the two functions cos and sin from the math package directly and allow  the user to call them using their names without preceding them iwth \sphinxcode{\sphinxupquote{math.}}.

\sphinxAtStartPar
Compare these calls with the one in the preceding section – there the call to the method pi has to be preceded by its namespace designator math.  i.e. \sphinxcode{\sphinxupquote{math.pi}}. Below we import pi directly and can just call it with pi.

\begin{sphinxuseclass}{cell}\begin{sphinxVerbatimInput}

\begin{sphinxuseclass}{cell_input}
\begin{sphinxVerbatim}[commandchars=\\\{\}]
\PYG{k+kn}{from} \PYG{n+nn}{math} \PYG{k+kn}{import} \PYG{n}{pi}\PYG{p}{,}\PYG{n}{cos}\PYG{p}{,}\PYG{n}{sin}

\PYG{n+nb}{print}\PYG{p}{(}\PYG{n}{pi}\PYG{p}{)}
\PYG{n+nb}{print}\PYG{p}{(}\PYG{n}{cos}\PYG{p}{(}\PYG{l+m+mi}{3}\PYG{p}{)}\PYG{p}{)}
\end{sphinxVerbatim}

\end{sphinxuseclass}\end{sphinxVerbatimInput}
\begin{sphinxVerbatimOutput}

\begin{sphinxuseclass}{cell_output}
\begin{sphinxVerbatim}[commandchars=\\\{\}]
3.141592653589793
\PYGZhy{}0.9899924966004454
\end{sphinxVerbatim}

\end{sphinxuseclass}\end{sphinxVerbatimOutput}

\end{sphinxuseclass}

\subsection{import a class but give it an alias}
\label{\detokenize{content/04_PythonEssentials/PythonPandasDataframes:import-a-class-but-give-it-an-alias}}
\sphinxAtStartPar
An imported class can also be given an alias, that is hopefully shorter than its official name but still obvious enough that the user knows what class is being referred to.

\sphinxAtStartPar
For example  \sphinxcode{\sphinxupquote{import math as m}} allows a call to pi using the more succinct syntax \sphinxcode{\sphinxupquote{m.py}}.

\begin{sphinxuseclass}{cell}\begin{sphinxVerbatimInput}

\begin{sphinxuseclass}{cell_input}
\begin{sphinxVerbatim}[commandchars=\\\{\}]
\PYG{k+kn}{import} \PYG{n+nn}{math} \PYG{k}{as} \PYG{n+nn}{m}
\PYG{n+nb}{print}\PYG{p}{(}\PYG{n}{m}\PYG{o}{.}\PYG{n}{pi}\PYG{p}{)}
\PYG{n+nb}{print}\PYG{p}{(}\PYG{n}{m}\PYG{o}{.}\PYG{n}{cos}\PYG{p}{(}\PYG{l+m+mi}{3}\PYG{p}{)}\PYG{p}{)}
\end{sphinxVerbatim}

\end{sphinxuseclass}\end{sphinxVerbatimInput}
\begin{sphinxVerbatimOutput}

\begin{sphinxuseclass}{cell_output}
\begin{sphinxVerbatim}[commandchars=\\\{\}]
3.141592653589793
\PYGZhy{}0.9899924966004454
\end{sphinxVerbatim}

\end{sphinxuseclass}\end{sphinxVerbatimOutput}

\end{sphinxuseclass}

\subsection{Standard aliases}
\label{\detokenize{content/04_PythonEssentials/PythonPandasDataframes:standard-aliases}}
\sphinxAtStartPar
Some packages are so frequently used that by convention they have been “assigned” specific aliases.

\sphinxAtStartPar
For example:

\sphinxAtStartPar
\sphinxstylestrong{Common aliases}


\begin{savenotes}\sphinxattablestart
\centering
\begin{tabulary}{\linewidth}[t]{|T|T|T|T|}
\hline
\sphinxstyletheadfamily 
\sphinxAtStartPar
Alias
&\sphinxstyletheadfamily 
\sphinxAtStartPar
aliased package
&\sphinxstyletheadfamily 
\sphinxAtStartPar
example
&\sphinxstyletheadfamily 
\sphinxAtStartPar
functionalty
\\
\hline
\sphinxAtStartPar
pd
&
\sphinxAtStartPar
pandas
&
\sphinxAtStartPar
import pandas as pd
&
\sphinxAtStartPar
Pandas are used for storing and retriveing data
\\
\hline
\sphinxAtStartPar
np
&
\sphinxAtStartPar
numpy
&
\sphinxAtStartPar
import numpy as np
&
\sphinxAtStartPar
Numpy gives access to some advanced mathematical features
\\
\hline
\end{tabulary}
\par
\sphinxattableend\end{savenotes}

\sphinxAtStartPar
You don’t have to use those conventions but it will make your code easier to read by others who are familiar with it.


\chapter{Introduction to Pandas, Pandas Series and Pandas dataframes}
\label{\detokenize{content/04_PythonEssentials/PythonPandasDataframes:introduction-to-pandas-pandas-series-and-pandas-dataframes}}
\sphinxAtStartPar
Modelflow is built on top of the Pandas library. Pandas is the Swiss knife of data science and can perform an impressive array of data oriented tasks.

\sphinxAtStartPar
This tutorial is a very short introduction to how pandas series and dataframes are used with Modelflow. For a more complete discussion see any of the many tutorials on the internet, notably:
\begin{itemize}
\item {} 
\sphinxAtStartPar
\sphinxhref{https://pandas.pydata.org/}{Pandas homepage}

\item {} 
\sphinxAtStartPar
\sphinxhref{https://pandas.pydata.org/pandas-docs/stable/getting\_started/tutorials.html}{Pandas community tutorials}

\end{itemize}


\section{Import the pandas library}
\label{\detokenize{content/04_PythonEssentials/PythonPandasDataframes:import-the-pandas-library}}
\sphinxAtStartPar
As with any python program, in order to use a package or library it must first be imported into the session. As noted above, by  convention pandas is imported as pd

\begin{sphinxuseclass}{cell}\begin{sphinxVerbatimInput}

\begin{sphinxuseclass}{cell_input}
\begin{sphinxVerbatim}[commandchars=\\\{\}]
\PYG{k+kn}{import} \PYG{n+nn}{pandas} \PYG{k}{as} \PYG{n+nn}{pd} 
\end{sphinxVerbatim}

\end{sphinxuseclass}\end{sphinxVerbatimInput}

\end{sphinxuseclass}
\sphinxAtStartPar
Pandas, like any library, contains many classes and methods.  The discussion below focuses on \sphinxstylestrong{Series} and \sphinxstylestrong{DataFrames} two classes that are part of the pandas library.  Both \sphinxcode{\sphinxupquote{series}} and \sphinxcode{\sphinxupquote{dataframes}} are containers that can be used to store time\sphinxhyphen{}series data and that have associated with them a number of very useful methods for displaying and manipulating time\sphinxhyphen{}series data.

\sphinxAtStartPar
Unlike other statistical packages neither \sphinxcode{\sphinxupquote{series}} nor \sphinxcode{\sphinxupquote{dataframes}} are inherently or exclusively time\sphinxhyphen{}series in nature. \sphinxcode{\sphinxupquote{Modelflow}} and macro\sphinxhyphen{}economists use them in this way, but the classes themselves are not dated or exclusively numerical out\sphinxhyphen{}of\sphinxhyphen{}the\sphinxhyphen{}box.


\section{The \sphinxstyleliteralintitle{\sphinxupquote{series}} class in \sphinxstyleliteralintitle{\sphinxupquote{Pandas}}}
\label{\detokenize{content/04_PythonEssentials/PythonPandasDataframes:the-series-class-in-pandas}}
\sphinxAtStartPar
\sphinxcode{\sphinxupquote{Series}} is a class that is part of the pandas package and can be used to instantiate an object that holds a two dimensional array comprised of values and an index.

\sphinxAtStartPar
The constructor for a \sphinxcode{\sphinxupquote{Series}} object is \sphinxcode{\sphinxupquote{pandas.Series()}}.  The content inside the parentheses will determine the nature of the series\sphinxhyphen{}object generated.  As an object\sphinxhyphen{}oriented language, Python supports \sphinxcode{\sphinxupquote{overrides}} (which is to say a method can have more than one way in which it can be called). Specifically there can be different constructors defined for a class, depending on how the data used to initialize the object it is organized.


\subsection{Series declared from a list}
\label{\detokenize{content/04_PythonEssentials/PythonPandasDataframes:series-declared-from-a-list}}
\sphinxAtStartPar
The simplest way to create a Series is to pass an array of values as a Python list to the Series constructor.

\begin{sphinxadmonition}{note}{Note:}
\sphinxAtStartPar
A list in python is a comma delimited collection of items.  It could be text, numbers or even more complex objects.  When declared (and returned) lists are enclosed in square brackets.

\sphinxAtStartPar
For example both of the following two lines are perfectly good examples of lists.

\sphinxAtStartPar
mylist={[}2,7,8,9{]}

\sphinxAtStartPar
mylist2={[}“Some text”,”Some more Text”,2,3{]}

\sphinxAtStartPar
The list is entirely agnostic about the type of data it contains.
\end{sphinxadmonition}

\sphinxAtStartPar
In the examples below Simplest, Simple and simple3 are all series – although series3 which is derived from a list mixing text and numeric values would be hard to interpret as an economic series.

\begin{sphinxuseclass}{cell}\begin{sphinxVerbatimInput}

\begin{sphinxuseclass}{cell_input}
\begin{sphinxVerbatim}[commandchars=\\\{\}]
\PYG{n}{values}\PYG{o}{=}\PYG{p}{[}\PYG{l+m+mi}{7}\PYG{p}{,}\PYG{l+m+mi}{8}\PYG{p}{,}\PYG{l+m+mi}{9}\PYG{p}{,}\PYG{l+m+mi}{10}\PYG{p}{,}\PYG{l+m+mi}{11}\PYG{p}{]}
\PYG{n}{weird}\PYG{o}{=}\PYG{p}{[}\PYG{l+s+s2}{\PYGZdq{}}\PYG{l+s+s2}{Some text}\PYG{l+s+s2}{\PYGZdq{}}\PYG{p}{,}\PYG{l+s+s2}{\PYGZdq{}}\PYG{l+s+s2}{Some more Text}\PYG{l+s+s2}{\PYGZdq{}}\PYG{p}{,}\PYG{l+m+mi}{2}\PYG{p}{,}\PYG{l+m+mi}{3}\PYG{p}{]}

\PYG{c+c1}{\PYGZsh{} Here the constructor is passed a numeric list}
\PYG{n}{Simplest}\PYG{o}{=}\PYG{n}{pd}\PYG{o}{.}\PYG{n}{Series}\PYG{p}{(}\PYG{p}{[}\PYG{l+m+mi}{2}\PYG{p}{,}\PYG{l+m+mi}{3}\PYG{p}{,}\PYG{l+m+mi}{4}\PYG{p}{,}\PYG{l+m+mi}{5}\PYG{p}{,}\PYG{l+m+mi}{6}\PYG{p}{,}\PYG{l+m+mi}{7}\PYG{p}{]}\PYG{p}{)}
\PYG{n}{Simplest}
\end{sphinxVerbatim}

\end{sphinxuseclass}\end{sphinxVerbatimInput}
\begin{sphinxVerbatimOutput}

\begin{sphinxuseclass}{cell_output}
\begin{sphinxVerbatim}[commandchars=\\\{\}]
0    2
1    3
2    4
3    5
4    6
5    7
dtype: int64
\end{sphinxVerbatim}

\end{sphinxuseclass}\end{sphinxVerbatimOutput}

\end{sphinxuseclass}
\begin{sphinxuseclass}{cell}\begin{sphinxVerbatimInput}

\begin{sphinxuseclass}{cell_input}
\begin{sphinxVerbatim}[commandchars=\\\{\}]
\PYG{c+c1}{\PYGZsh{} In this case the constructor is passed a variable that contains a list}
\PYG{n}{simple2}\PYG{o}{=}\PYG{n}{pd}\PYG{o}{.}\PYG{n}{Series}\PYG{p}{(}\PYG{n}{values}\PYG{p}{)}
\PYG{n}{simple2}
\end{sphinxVerbatim}

\end{sphinxuseclass}\end{sphinxVerbatimInput}
\begin{sphinxVerbatimOutput}

\begin{sphinxuseclass}{cell_output}
\begin{sphinxVerbatim}[commandchars=\\\{\}]
0     7
1     8
2     9
3    10
4    11
dtype: int64
\end{sphinxVerbatim}

\end{sphinxuseclass}\end{sphinxVerbatimOutput}

\end{sphinxuseclass}
\begin{sphinxuseclass}{cell}\begin{sphinxVerbatimInput}

\begin{sphinxuseclass}{cell_input}
\begin{sphinxVerbatim}[commandchars=\\\{\}]
\PYG{c+c1}{\PYGZsh{} Here the constructor is passed a variable containing a list that is a mix of }
\PYG{c+c1}{\PYGZsh{} alphanumerics and numerical values}
\PYG{n}{simple3}\PYG{o}{=}\PYG{n}{pd}\PYG{o}{.}\PYG{n}{Series}\PYG{p}{(}\PYG{n}{weird}\PYG{p}{)}
\PYG{n}{simple3}
\end{sphinxVerbatim}

\end{sphinxuseclass}\end{sphinxVerbatimInput}
\begin{sphinxVerbatimOutput}

\begin{sphinxuseclass}{cell_output}
\begin{sphinxVerbatim}[commandchars=\\\{\}]
0         Some text
1    Some more Text
2                 2
3                 3
dtype: object
\end{sphinxVerbatim}

\end{sphinxuseclass}\end{sphinxVerbatimOutput}

\end{sphinxuseclass}
\sphinxAtStartPar
Note that all three series have different length.

\sphinxAtStartPar
Moreover, constructed in this way (by passing a list to the constructor) each of these \sphinxcode{\sphinxupquote{Series}} are automatically assigned a zero\sphinxhyphen{}based index (a numerial index that starts with 0).


\subsection{Series declared using a specific index}
\label{\detokenize{content/04_PythonEssentials/PythonPandasDataframes:series-declared-using-a-specific-index}}
\sphinxAtStartPar
In this example the series Simple and Simple2 are recreated (overwritten), but this time an index is specified. Here the index is declared as a(nother) list.

\begin{sphinxuseclass}{cell}\begin{sphinxVerbatimInput}

\begin{sphinxuseclass}{cell_input}
\begin{sphinxVerbatim}[commandchars=\\\{\}]
\PYG{c+c1}{\PYGZsh{} In this example the constructor is given both the values }
\PYG{c+c1}{\PYGZsh{} and specific values for the index}
\PYG{n}{Simplest}\PYG{o}{=}\PYG{n}{pd}\PYG{o}{.}\PYG{n}{Series}\PYG{p}{(}\PYG{p}{[}\PYG{l+m+mi}{2}\PYG{p}{,}\PYG{l+m+mi}{3}\PYG{p}{,}\PYG{l+m+mi}{4}\PYG{p}{,}\PYG{l+m+mi}{5}\PYG{p}{,}\PYG{l+m+mi}{6}\PYG{p}{]}\PYG{p}{,}\PYG{n}{index}\PYG{o}{=}\PYG{p}{[}\PYG{l+m+mi}{1966}\PYG{p}{,}\PYG{l+m+mi}{1967}\PYG{p}{,}\PYG{l+m+mi}{1996}\PYG{p}{,}\PYG{l+m+mi}{1999}\PYG{p}{,}\PYG{l+m+mi}{2000}\PYG{p}{]}\PYG{p}{)}
\PYG{n}{Simplest}
\end{sphinxVerbatim}

\end{sphinxuseclass}\end{sphinxVerbatimInput}
\begin{sphinxVerbatimOutput}

\begin{sphinxuseclass}{cell_output}
\begin{sphinxVerbatim}[commandchars=\\\{\}]
1966    2
1967    3
1996    4
1999    5
2000    6
dtype: int64
\end{sphinxVerbatim}

\end{sphinxuseclass}\end{sphinxVerbatimOutput}

\end{sphinxuseclass}
\begin{sphinxuseclass}{cell}\begin{sphinxVerbatimInput}

\begin{sphinxuseclass}{cell_input}
\begin{sphinxVerbatim}[commandchars=\\\{\}]
\PYG{n}{simple2}\PYG{o}{=}\PYG{n}{pd}\PYG{o}{.}\PYG{n}{Series}\PYG{p}{(}\PYG{n}{values}\PYG{p}{,}\PYG{n}{index}\PYG{o}{=}\PYG{p}{[}\PYG{l+m+mi}{1966}\PYG{p}{,}\PYG{l+m+mi}{1967}\PYG{p}{,}\PYG{l+m+mi}{1996}\PYG{p}{,}\PYG{l+m+mi}{1999}\PYG{p}{,}\PYG{l+m+mi}{2000}\PYG{p}{]}\PYG{p}{)}
\PYG{n}{simple2}
\end{sphinxVerbatim}

\end{sphinxuseclass}\end{sphinxVerbatimInput}
\begin{sphinxVerbatimOutput}

\begin{sphinxuseclass}{cell_output}
\begin{sphinxVerbatim}[commandchars=\\\{\}]
1966     7
1967     8
1996     9
1999    10
2000    11
dtype: int64
\end{sphinxVerbatim}

\end{sphinxuseclass}\end{sphinxVerbatimOutput}

\end{sphinxuseclass}
\sphinxAtStartPar
Now these Series look more like time\sphinxhyphen{}series data!


\subsection{Create Series from a dictionary}
\label{\detokenize{content/04_PythonEssentials/PythonPandasDataframes:create-series-from-a-dictionary}}
\sphinxAtStartPar
In python, a dictionary is a data structure that is more generally known in computer science as an associative array. A dictionary consists of a collection of key\sphinxhyphen{}value pairs, where each key\sphinxhyphen{}value pair \sphinxstyleemphasis{maps} or \sphinxstyleemphasis{links} the key to its associated value.

\begin{sphinxadmonition}{note}{Note:}
\sphinxAtStartPar
A dictionary is enclosed in curly brackets \{\}, versus a list which is enclosed in square brackets{[}{]}.
\end{sphinxadmonition}

\sphinxAtStartPar
Thus mydict=\{“1966”:2,”1967”:3,”1968”:4,”1969”:5,”2000”:\sphinxhyphen{}15\} creates a dictionary object called mydict.   \sphinxcode{\sphinxupquote{mydict}}maps (or links) the key “1966” links to the value 2.

\begin{sphinxadmonition}{note}{Note:}
\sphinxAtStartPar
In this example the Key was a string but we could just as easily made it a numerical value:
\end{sphinxadmonition}

\sphinxAtStartPar
mydict2=\{1966:2,1967:3,1968:4,1969:5,2000:\sphinxhyphen{}15\} creates an object called mydict2 that links (maps) the key “1966” to the value 2.

\sphinxAtStartPar
The series constructor also accepts a dictionary, and maps the key to the index of the Series.

\begin{sphinxuseclass}{cell}\begin{sphinxVerbatimInput}

\begin{sphinxuseclass}{cell_input}
\begin{sphinxVerbatim}[commandchars=\\\{\}]
\PYG{n}{mydict2}\PYG{o}{=}\PYG{p}{\PYGZob{}}\PYG{l+m+mi}{1966}\PYG{p}{:}\PYG{l+m+mi}{2}\PYG{p}{,}\PYG{l+m+mi}{1967}\PYG{p}{:}\PYG{l+m+mi}{3}\PYG{p}{,}\PYG{l+m+mi}{1968}\PYG{p}{:}\PYG{l+m+mi}{4}\PYG{p}{,}\PYG{l+m+mi}{1969}\PYG{p}{:}\PYG{l+m+mi}{5}\PYG{p}{,}\PYG{l+m+mi}{2000}\PYG{p}{:}\PYG{l+m+mi}{6}\PYG{p}{\PYGZcb{}}
\PYG{n}{simple2}\PYG{o}{=}\PYG{n}{pd}\PYG{o}{.}\PYG{n}{Series}\PYG{p}{(}\PYG{n}{mydict2}\PYG{p}{)}
\PYG{n}{simple2}
\end{sphinxVerbatim}

\end{sphinxuseclass}\end{sphinxVerbatimInput}
\begin{sphinxVerbatimOutput}

\begin{sphinxuseclass}{cell_output}
\begin{sphinxVerbatim}[commandchars=\\\{\}]
1966    2
1967    3
1968    4
1969    5
2000    6
dtype: int64
\end{sphinxVerbatim}

\end{sphinxuseclass}\end{sphinxVerbatimOutput}

\end{sphinxuseclass}

\section{The \sphinxstyleliteralintitle{\sphinxupquote{DataFrame}} class in \sphinxstyleliteralintitle{\sphinxupquote{Pandas}}}
\label{\detokenize{content/04_PythonEssentials/PythonPandasDataframes:the-dataframe-class-in-pandas}}
\sphinxAtStartPar
The \sphinxcode{\sphinxupquote{DataFrame}} is the primary structure of pandas. It is a two\sphinxhyphen{}dimensional data structure with named rows and columns.  Each columns can have different data types (numeric, string, etc).

\sphinxAtStartPar
By convention, a dataframe if often called df or some other modifier followed by df, to assist in reading the code.

\sphinxAtStartPar
Much more detail on standard pandas dataframes can be found on the \sphinxhref{https://pandas.pydata.org/docs/reference/frame.html}{official pandas website}.


\subsection{Creating or instantiating a dataframe}
\label{\detokenize{content/04_PythonEssentials/PythonPandasDataframes:creating-or-instantiating-a-dataframe}}
\sphinxAtStartPar
Like any object, a \sphinxcode{\sphinxupquote{DataFrame}} can be created by calling the constructor of the pandas class \sphinxcode{\sphinxupquote{DataFrame}}.

\sphinxAtStartPar
Each class has many constructors, so there are very many ways to create a \sphinxcode{\sphinxupquote{dataframe}}. The \sphinxcode{\sphinxupquote{pandas.DataFrame()}} method is constructor for the \sphinxcode{\sphinxupquote{DataFrame}} class. It takes several forms (as with \sphinxcode{\sphinxupquote{Series}}), but always returns an instance (instantiates) of a \sphinxcode{\sphinxupquote{DataFrame}} object – i.e. a variable whose contents are a \sphinxcode{\sphinxupquote{DataFrame}}.

\sphinxAtStartPar
The code example below creates a \sphinxcode{\sphinxupquote{DataFrame}} of three columns A,B,C; indexed between 2019 and 2021.  Macroeconomists may interpret the index as dates, but for pandas they are just numbers.

\sphinxAtStartPar
Below a \sphinxcode{\sphinxupquote{DataFrame}} named \sphinxcode{\sphinxupquote{df}} is instantiated from a dictionary and assigned a specific index by passing a list of years as the index.

\begin{sphinxuseclass}{cell}\begin{sphinxVerbatimInput}

\begin{sphinxuseclass}{cell_input}
\begin{sphinxVerbatim}[commandchars=\\\{\}]
\PYG{n}{df} \PYG{o}{=} \PYG{n}{pd}\PYG{o}{.}\PYG{n}{DataFrame}\PYG{p}{(}\PYG{p}{\PYGZob{}}\PYG{l+s+s1}{\PYGZsq{}}\PYG{l+s+s1}{B}\PYG{l+s+s1}{\PYGZsq{}}\PYG{p}{:} \PYG{p}{[}\PYG{l+m+mi}{1}\PYG{p}{,}\PYG{l+m+mi}{1}\PYG{p}{,}\PYG{l+m+mi}{1}\PYG{p}{,}\PYG{l+m+mi}{1}\PYG{p}{]}\PYG{p}{,}\PYG{l+s+s1}{\PYGZsq{}}\PYG{l+s+s1}{C}\PYG{l+s+s1}{\PYGZsq{}}\PYG{p}{:}\PYG{p}{[}\PYG{l+m+mi}{1}\PYG{p}{,}\PYG{l+m+mi}{2}\PYG{p}{,}\PYG{l+m+mi}{3}\PYG{p}{,}\PYG{l+m+mi}{6}\PYG{p}{]}\PYG{p}{,}\PYG{l+s+s1}{\PYGZsq{}}\PYG{l+s+s1}{E}\PYG{l+s+s1}{\PYGZsq{}}\PYG{p}{:}\PYG{p}{[}\PYG{l+m+mi}{4}\PYG{p}{,}\PYG{l+m+mi}{4}\PYG{p}{,}\PYG{l+m+mi}{4}\PYG{p}{,}\PYG{l+m+mi}{4}\PYG{p}{]}\PYG{p}{\PYGZcb{}}\PYG{p}{,}\PYG{n}{index}\PYG{o}{=}\PYG{p}{[}\PYG{l+m+mi}{2018}\PYG{p}{,}\PYG{l+m+mi}{2019}\PYG{p}{,}\PYG{l+m+mi}{2020}\PYG{p}{,}\PYG{l+m+mi}{2021}\PYG{p}{]}\PYG{p}{)}
\PYG{n}{df} 
\end{sphinxVerbatim}

\end{sphinxuseclass}\end{sphinxVerbatimInput}
\begin{sphinxVerbatimOutput}

\begin{sphinxuseclass}{cell_output}
\begin{sphinxVerbatim}[commandchars=\\\{\}]
      B  C  E
2018  1  1  4
2019  1  2  4
2020  1  3  4
2021  1  6  4
\end{sphinxVerbatim}

\end{sphinxuseclass}\end{sphinxVerbatimOutput}

\end{sphinxuseclass}
\begin{sphinxadmonition}{note}{Note:}
\sphinxAtStartPar
In the \sphinxcode{\sphinxupquote{DataFrame}}s that are used in macrostructural models like MFMod, each  column is often interpreted as a time\sphinxhyphen{}series of an economic variable. So in this dataframe,  normally A, B and C each be interpreted as economic time series.

\sphinxAtStartPar
That said, there is nothing in the \sphinxcode{\sphinxupquote{DataFrame}} class that suggests that the data it stores must be time\sphinxhyphen{}series or even numeric in nature.
\end{sphinxadmonition}


\subsection{Alternative ways to set the time period of a dated index}
\label{\detokenize{content/04_PythonEssentials/PythonPandasDataframes:alternative-ways-to-set-the-time-period-of-a-dated-index}}
\sphinxAtStartPar
A somewhat more creative way to initialize the dataframe for dates would use a loop to specify the dates that get passed to the constructor as an argument.

\sphinxAtStartPar
Below a dataframe df with two Series (A and B), is initialized with the values 100 for all data points.

\sphinxAtStartPar
The index is defined dynamically by a loop \sphinxcode{\sphinxupquote{index={[}2020+v for v in range(number\_of\_rows){]}}} that runs for number\_of\_rows times (6 times in this example) setting v equal to 2020+0, 2020+1,…,202+5. The resulting list whose values are assigned to index is {[}2020,2021,2022,2023,2024,2025{]}.

\sphinxAtStartPar
The big advantage of this method is that if the user wanted to have data created for the period 1990 to 2030, they would only have to change number\_of\_rows from 6 to 41, and the change the staring date in the loop from 2020 to 1990.

\begin{sphinxuseclass}{cell}\begin{sphinxVerbatimInput}

\begin{sphinxuseclass}{cell_input}
\begin{sphinxVerbatim}[commandchars=\\\{\}]
\PYG{c+c1}{\PYGZsh{}define the number of years for which the data is to be created.}
\PYG{n}{number\PYGZus{}of\PYGZus{}rows} \PYG{o}{=} \PYG{l+m+mi}{6} 

\PYG{c+c1}{\PYGZsh{} call the dataframe constructor}
\PYG{n}{df} \PYG{o}{=} \PYG{n}{pd}\PYG{o}{.}\PYG{n}{DataFrame}\PYG{p}{(}\PYG{l+m+mi}{100}\PYG{p}{,}
       \PYG{n}{index}\PYG{o}{=}\PYG{p}{[}\PYG{l+m+mi}{2020}\PYG{o}{+}\PYG{n}{v} \PYG{k}{for} \PYG{n}{v} \PYG{o+ow}{in} \PYG{n+nb}{range}\PYG{p}{(}\PYG{n}{number\PYGZus{}of\PYGZus{}rows}\PYG{p}{)}\PYG{p}{]}\PYG{p}{,} \PYG{c+c1}{\PYGZsh{} create row index}
       \PYG{c+c1}{\PYGZsh{} equivalent to index=[2020,2021,2022,2023,2024,2025] }
       \PYG{n}{columns}\PYG{o}{=}\PYG{p}{[}\PYG{l+s+s1}{\PYGZsq{}}\PYG{l+s+s1}{A}\PYG{l+s+s1}{\PYGZsq{}}\PYG{p}{,}\PYG{l+s+s1}{\PYGZsq{}}\PYG{l+s+s1}{B}\PYG{l+s+s1}{\PYGZsq{}}\PYG{p}{]}\PYG{p}{)}                                 \PYG{c+c1}{\PYGZsh{} create column name }
\PYG{n}{df}
\end{sphinxVerbatim}

\end{sphinxuseclass}\end{sphinxVerbatimInput}
\begin{sphinxVerbatimOutput}

\begin{sphinxuseclass}{cell_output}
\begin{sphinxVerbatim}[commandchars=\\\{\}]
        A    B
2020  100  100
2021  100  100
2022  100  100
2023  100  100
2024  100  100
2025  100  100
\end{sphinxVerbatim}

\end{sphinxuseclass}\end{sphinxVerbatimOutput}

\end{sphinxuseclass}
\sphinxAtStartPar
This second example simplifies the creation even further by  specifying the begin and end point as a range.

\begin{sphinxuseclass}{cell}\begin{sphinxVerbatimInput}

\begin{sphinxuseclass}{cell_input}
\begin{sphinxVerbatim}[commandchars=\\\{\}]
\PYG{n}{df1} \PYG{o}{=} \PYG{n}{pd}\PYG{o}{.}\PYG{n}{DataFrame}\PYG{p}{(}\PYG{l+m+mi}{200}\PYG{p}{,}
       \PYG{n}{index}\PYG{o}{=}\PYG{p}{[}\PYG{n}{v} \PYG{k}{for} \PYG{n}{v} \PYG{o+ow}{in} \PYG{n+nb}{range}\PYG{p}{(}\PYG{l+m+mi}{2020}\PYG{p}{,}\PYG{l+m+mi}{2030}\PYG{p}{)}\PYG{p}{]}\PYG{p}{,} \PYG{c+c1}{\PYGZsh{} create row index}
       \PYG{c+c1}{\PYGZsh{} equivalent to index=[2020,2021,...,2030] }
       \PYG{n}{columns}\PYG{o}{=}\PYG{p}{[}\PYG{l+s+s1}{\PYGZsq{}}\PYG{l+s+s1}{A1}\PYG{l+s+s1}{\PYGZsq{}}\PYG{p}{,}\PYG{l+s+s1}{\PYGZsq{}}\PYG{l+s+s1}{B1}\PYG{l+s+s1}{\PYGZsq{}}\PYG{p}{]}\PYG{p}{)}                                 \PYG{c+c1}{\PYGZsh{} create column name }
\PYG{n}{df1}
\end{sphinxVerbatim}

\end{sphinxuseclass}\end{sphinxVerbatimInput}
\begin{sphinxVerbatimOutput}

\begin{sphinxuseclass}{cell_output}
\begin{sphinxVerbatim}[commandchars=\\\{\}]
       A1   B1
2020  200  200
2021  200  200
2022  200  200
2023  200  200
2024  200  200
2025  200  200
2026  200  200
2027  200  200
2028  200  200
2029  200  200
\end{sphinxVerbatim}

\end{sphinxuseclass}\end{sphinxVerbatimOutput}

\end{sphinxuseclass}

\subsection{Adding a column to a dataframe}
\label{\detokenize{content/04_PythonEssentials/PythonPandasDataframes:adding-a-column-to-a-dataframe}}
\sphinxAtStartPar
If a value is assigned to a column that does not exist, pandas will add a column with that name and fill it with values resulting from  the calculation.

\begin{sphinxadmonition}{note}{Note:}
\sphinxAtStartPar
The size of the object assigned to the new column must match the size (number of rows) of the pre\sphinxhyphen{}existing \sphinxcode{\sphinxupquote{DataFarame}}.
\end{sphinxadmonition}

\begin{sphinxuseclass}{cell}\begin{sphinxVerbatimInput}

\begin{sphinxuseclass}{cell_input}
\begin{sphinxVerbatim}[commandchars=\\\{\}]
\PYG{n}{df}\PYG{p}{[}\PYG{l+s+s2}{\PYGZdq{}}\PYG{l+s+s2}{NEW}\PYG{l+s+s2}{\PYGZdq{}}\PYG{p}{]}\PYG{o}{=}\PYG{p}{[}\PYG{l+m+mi}{10}\PYG{p}{,}\PYG{l+m+mi}{12}\PYG{p}{,}\PYG{l+m+mi}{10}\PYG{p}{,}\PYG{l+m+mi}{13}\PYG{p}{,}\PYG{l+m+mi}{14}\PYG{p}{,}\PYG{l+m+mi}{15}\PYG{p}{]}  \PYG{c+c1}{\PYGZsh{}df origiall has 6 rows so we must suup,y 6 data points for this command to run error free}
\PYG{n}{df}
\end{sphinxVerbatim}

\end{sphinxuseclass}\end{sphinxVerbatimInput}
\begin{sphinxVerbatimOutput}

\begin{sphinxuseclass}{cell_output}
\begin{sphinxVerbatim}[commandchars=\\\{\}]
        A    B  NEW
2020  100  100   10
2021  100  100   12
2022  100  100   10
2023  100  100   13
2024  100  100   14
2025  100  100   15
\end{sphinxVerbatim}

\end{sphinxuseclass}\end{sphinxVerbatimOutput}

\end{sphinxuseclass}

\subsection{Revising values}
\label{\detokenize{content/04_PythonEssentials/PythonPandasDataframes:revising-values}}
\sphinxAtStartPar
If the column exists than the = method will revise the values of the rows with the values assigned in the statement.

\begin{sphinxadmonition}{warning}{Warning:}
\sphinxAtStartPar
The dimensions of the list assigned via the \sphinxcode{\sphinxupquote{=}} method must be the same as the \sphinxcode{\sphinxupquote{DataFrame}} (i.e. there must be exactly as many values as there are rows).  Alternatively if only one value is provided, then that value will replace all of the values in the specified column (be broadcast to the other rows in the column).
\end{sphinxadmonition}

\begin{sphinxuseclass}{cell}\begin{sphinxVerbatimInput}

\begin{sphinxuseclass}{cell_input}
\begin{sphinxVerbatim}[commandchars=\\\{\}]
\PYG{n}{df}\PYG{p}{[}\PYG{l+s+s2}{\PYGZdq{}}\PYG{l+s+s2}{NEW}\PYG{l+s+s2}{\PYGZdq{}}\PYG{p}{]}\PYG{o}{=}\PYG{p}{[}\PYG{l+m+mi}{11}\PYG{p}{,}\PYG{l+m+mi}{12}\PYG{p}{,}\PYG{l+m+mi}{10}\PYG{p}{,}\PYG{l+m+mi}{14}\PYG{p}{,}\PYG{l+m+mi}{2}\PYG{p}{,}\PYG{l+m+mi}{1}\PYG{p}{]}

\PYG{n}{df}
\end{sphinxVerbatim}

\end{sphinxuseclass}\end{sphinxVerbatimInput}
\begin{sphinxVerbatimOutput}

\begin{sphinxuseclass}{cell_output}
\begin{sphinxVerbatim}[commandchars=\\\{\}]
        A    B  NEW
2020  100  100   11
2021  100  100   12
2022  100  100   10
2023  100  100   14
2024  100  100    2
2025  100  100    1
\end{sphinxVerbatim}

\end{sphinxuseclass}\end{sphinxVerbatimOutput}

\end{sphinxuseclass}
\begin{sphinxuseclass}{cell}\begin{sphinxVerbatimInput}

\begin{sphinxuseclass}{cell_input}
\begin{sphinxVerbatim}[commandchars=\\\{\}]
\PYG{c+c1}{\PYGZsh{} replace all of the rows of column B with the same value}
\PYG{n}{df}\PYG{p}{[}\PYG{l+s+s1}{\PYGZsq{}}\PYG{l+s+s1}{B}\PYG{l+s+s1}{\PYGZsq{}}\PYG{p}{]}\PYG{o}{=}\PYG{l+m+mi}{17}
\PYG{n}{df}
\end{sphinxVerbatim}

\end{sphinxuseclass}\end{sphinxVerbatimInput}
\begin{sphinxVerbatimOutput}

\begin{sphinxuseclass}{cell_output}
\begin{sphinxVerbatim}[commandchars=\\\{\}]
        A   B  NEW
2020  100  17   11
2021  100  17   12
2022  100  17   10
2023  100  17   14
2024  100  17    2
2025  100  17    1
\end{sphinxVerbatim}

\end{sphinxuseclass}\end{sphinxVerbatimOutput}

\end{sphinxuseclass}

\subsection{.columns lists the column names of a dataframe}
\label{\detokenize{content/04_PythonEssentials/PythonPandasDataframes:columns-lists-the-column-names-of-a-dataframe}}
\sphinxAtStartPar
The method \sphinxcode{\sphinxupquote{.columns}} returns the names of the columns in the dataframe.

\begin{sphinxuseclass}{cell}\begin{sphinxVerbatimInput}

\begin{sphinxuseclass}{cell_input}
\begin{sphinxVerbatim}[commandchars=\\\{\}]
\PYG{n}{df}\PYG{o}{.}\PYG{n}{columns}
\end{sphinxVerbatim}

\end{sphinxuseclass}\end{sphinxVerbatimInput}
\begin{sphinxVerbatimOutput}

\begin{sphinxuseclass}{cell_output}
\begin{sphinxVerbatim}[commandchars=\\\{\}]
Index([\PYGZsq{}A\PYGZsq{}, \PYGZsq{}B\PYGZsq{}, \PYGZsq{}NEW\PYGZsq{}], dtype=\PYGZsq{}object\PYGZsq{})
\end{sphinxVerbatim}

\end{sphinxuseclass}\end{sphinxVerbatimOutput}

\end{sphinxuseclass}

\subsection{.size indicates the dimension of a list}
\label{\detokenize{content/04_PythonEssentials/PythonPandasDataframes:size-indicates-the-dimension-of-a-list}}
\sphinxAtStartPar
so \sphinxcode{\sphinxupquote{df.columns.size}} returns the number of columns in a dataframe.

\begin{sphinxuseclass}{cell}\begin{sphinxVerbatimInput}

\begin{sphinxuseclass}{cell_input}
\begin{sphinxVerbatim}[commandchars=\\\{\}]
\PYG{n}{df}\PYG{o}{.}\PYG{n}{columns}\PYG{o}{.}\PYG{n}{size}
\end{sphinxVerbatim}

\end{sphinxuseclass}\end{sphinxVerbatimInput}
\begin{sphinxVerbatimOutput}

\begin{sphinxuseclass}{cell_output}
\begin{sphinxVerbatim}[commandchars=\\\{\}]
3
\end{sphinxVerbatim}

\end{sphinxuseclass}\end{sphinxVerbatimOutput}

\end{sphinxuseclass}
\sphinxAtStartPar
The dataframe df has 4 columns.


\subsection{.eval() evaluates calculates an expression on the data of a dataframe}
\label{\detokenize{content/04_PythonEssentials/PythonPandasDataframes:eval-evaluates-calculates-an-expression-on-the-data-of-a-dataframe}}
\sphinxAtStartPar
\sphinxcode{\sphinxupquote{.eval}} is a native dataframe method, which does calculations on a \sphinxcode{\sphinxupquote{dataframe}} and returns a revised \sphinxcode{\sphinxupquote{dataframe}}. With this method expressions can be evaluated and new columns created.

\begin{sphinxuseclass}{cell}\begin{sphinxVerbatimInput}

\begin{sphinxuseclass}{cell_input}
\begin{sphinxVerbatim}[commandchars=\\\{\}]
\PYG{n}{df}\PYG{o}{.}\PYG{n}{eval}\PYG{p}{(}\PYG{l+s+s1}{\PYGZsq{}\PYGZsq{}\PYGZsq{}}\PYG{l+s+s1}{X = B*NEW}
\PYG{l+s+s1}{           THE\PYGZus{}ANSWER = 42}\PYG{l+s+s1}{\PYGZsq{}\PYGZsq{}\PYGZsq{}}\PYG{p}{)}
\end{sphinxVerbatim}

\end{sphinxuseclass}\end{sphinxVerbatimInput}
\begin{sphinxVerbatimOutput}

\begin{sphinxuseclass}{cell_output}
\begin{sphinxVerbatim}[commandchars=\\\{\}]
        A   B  NEW    X  THE\PYGZus{}ANSWER
2020  100  17   11  187          42
2021  100  17   12  204          42
2022  100  17   10  170          42
2023  100  17   14  238          42
2024  100  17    2   34          42
2025  100  17    1   17          42
\end{sphinxVerbatim}

\end{sphinxuseclass}\end{sphinxVerbatimOutput}

\end{sphinxuseclass}
\sphinxAtStartPar
Because the result of the \sphinxcode{\sphinxupquote{.df.eval()}} call was not assigned to anything, least of all the dataframe df, the value of df is unchanged.

\begin{sphinxuseclass}{cell}\begin{sphinxVerbatimInput}

\begin{sphinxuseclass}{cell_input}
\begin{sphinxVerbatim}[commandchars=\\\{\}]
\PYG{n}{df}
\end{sphinxVerbatim}

\end{sphinxuseclass}\end{sphinxVerbatimInput}
\begin{sphinxVerbatimOutput}

\begin{sphinxuseclass}{cell_output}
\begin{sphinxVerbatim}[commandchars=\\\{\}]
        A   B  NEW
2020  100  17   11
2021  100  17   12
2022  100  17   10
2023  100  17   14
2024  100  17    2
2025  100  17    1
\end{sphinxVerbatim}

\end{sphinxuseclass}\end{sphinxVerbatimOutput}

\end{sphinxuseclass}
\sphinxAtStartPar
To store the results of the calculation must be assigned to a variable.  The pre\sphinxhyphen{}existing dataframe can be overwritten by assigning it the result of the eval statement.

\begin{sphinxuseclass}{cell}\begin{sphinxVerbatimInput}

\begin{sphinxuseclass}{cell_input}
\begin{sphinxVerbatim}[commandchars=\\\{\}]
\PYG{n}{df}\PYG{o}{=}\PYG{n}{df}\PYG{o}{.}\PYG{n}{eval}\PYG{p}{(}\PYG{l+s+s1}{\PYGZsq{}\PYGZsq{}\PYGZsq{}}\PYG{l+s+s1}{X = B*NEW}
\PYG{l+s+s1}{           THE\PYGZus{}ANSWER = 42}\PYG{l+s+s1}{\PYGZsq{}\PYGZsq{}\PYGZsq{}}\PYG{p}{)}
\PYG{n}{df}
\end{sphinxVerbatim}

\end{sphinxuseclass}\end{sphinxVerbatimInput}
\begin{sphinxVerbatimOutput}

\begin{sphinxuseclass}{cell_output}
\begin{sphinxVerbatim}[commandchars=\\\{\}]
        A   B  NEW    X  THE\PYGZus{}ANSWER
2020  100  17   11  187          42
2021  100  17   12  204          42
2022  100  17   10  170          42
2023  100  17   14  238          42
2024  100  17    2   34          42
2025  100  17    1   17          42
\end{sphinxVerbatim}

\end{sphinxuseclass}\end{sphinxVerbatimOutput}

\end{sphinxuseclass}
\sphinxAtStartPar
With this operation the new columns, x and THE\_ANSWER have been appended to the dataframe df.

\begin{sphinxadmonition}{note}{Note:}
\sphinxAtStartPar
The \sphinxcode{\sphinxupquote{.eval()}} method is a native pandas method.  As such it cannot handle lagged variables (because pandas do not support the idea of a lagged variable.

\sphinxAtStartPar
The \sphinxcode{\sphinxupquote{.mfcalc()}} and the \sphinxcode{\sphinxupquote{.upd()}} methods discussed in the next chapter are \sphinxcode{\sphinxupquote{modelflow}} features that extend the functionalities native to \sphinxcode{\sphinxupquote{dataframe}} that allows such calculations to be performed.
\end{sphinxadmonition}


\subsection{.loc{[}{]} selects a portion (slice) of a dataframe}
\label{\detokenize{content/04_PythonEssentials/PythonPandasDataframes:loc-selects-a-portion-slice-of-a-dataframe}}
\sphinxAtStartPar
The \sphinxcode{\sphinxupquote{.loc{[}{]}}} method allows you to display and/or revise specific sub\sphinxhyphen{}sections of a column or row in a dataframe.


\subsubsection{.loc{[}row,column{]} A single element}
\label{\detokenize{content/04_PythonEssentials/PythonPandasDataframes:loc-row-column-a-single-element}}
\sphinxAtStartPar
\sphinxcode{\sphinxupquote{.loc{[}row,column{]}}} operates on a single cell in the dataframe.  Thus the below displays the value of the cell with index=2023 observation from the column NEW.

\begin{sphinxuseclass}{cell}\begin{sphinxVerbatimInput}

\begin{sphinxuseclass}{cell_input}
\begin{sphinxVerbatim}[commandchars=\\\{\}]
\PYG{n}{df}\PYG{o}{.}\PYG{n}{loc}\PYG{p}{[}\PYG{l+m+mi}{2023}\PYG{p}{,}\PYG{l+s+s1}{\PYGZsq{}}\PYG{l+s+s1}{NEW}\PYG{l+s+s1}{\PYGZsq{}}\PYG{p}{]}
\end{sphinxVerbatim}

\end{sphinxuseclass}\end{sphinxVerbatimInput}
\begin{sphinxVerbatimOutput}

\begin{sphinxuseclass}{cell_output}
\begin{sphinxVerbatim}[commandchars=\\\{\}]
14
\end{sphinxVerbatim}

\end{sphinxuseclass}\end{sphinxVerbatimOutput}

\end{sphinxuseclass}

\subsubsection{.loc{[}:,column{]} A single column}
\label{\detokenize{content/04_PythonEssentials/PythonPandasDataframes:loc-column-a-single-column}}
\sphinxAtStartPar
The lone colon in a loc statement indicates all the rows or columns.  Here all of the rows.

\begin{sphinxuseclass}{cell}\begin{sphinxVerbatimInput}

\begin{sphinxuseclass}{cell_input}
\begin{sphinxVerbatim}[commandchars=\\\{\}]
\PYG{n}{df}\PYG{o}{.}\PYG{n}{loc}\PYG{p}{[}\PYG{p}{:}\PYG{p}{,}\PYG{l+s+s1}{\PYGZsq{}}\PYG{l+s+s1}{NEW}\PYG{l+s+s1}{\PYGZsq{}}\PYG{p}{]}
\end{sphinxVerbatim}

\end{sphinxuseclass}\end{sphinxVerbatimInput}
\begin{sphinxVerbatimOutput}

\begin{sphinxuseclass}{cell_output}
\begin{sphinxVerbatim}[commandchars=\\\{\}]
2020    11
2021    12
2022    10
2023    14
2024     2
2025     1
Name: NEW, dtype: int64
\end{sphinxVerbatim}

\end{sphinxuseclass}\end{sphinxVerbatimOutput}

\end{sphinxuseclass}

\subsubsection{.loc{[}row,:{]} A single row}
\label{\detokenize{content/04_PythonEssentials/PythonPandasDataframes:loc-row-a-single-row}}
\sphinxAtStartPar
Here all of the columns, for the selected row.

\begin{sphinxuseclass}{cell}\begin{sphinxVerbatimInput}

\begin{sphinxuseclass}{cell_input}
\begin{sphinxVerbatim}[commandchars=\\\{\}]
\PYG{n}{df}\PYG{o}{.}\PYG{n}{loc}\PYG{p}{[}\PYG{l+m+mi}{2023}\PYG{p}{,}\PYG{p}{:}\PYG{p}{]}
\end{sphinxVerbatim}

\end{sphinxuseclass}\end{sphinxVerbatimInput}
\begin{sphinxVerbatimOutput}

\begin{sphinxuseclass}{cell_output}
\begin{sphinxVerbatim}[commandchars=\\\{\}]
A             100
B              17
NEW            14
X             238
THE\PYGZus{}ANSWER     42
Name: 2023, dtype: int64
\end{sphinxVerbatim}

\end{sphinxuseclass}\end{sphinxVerbatimOutput}

\end{sphinxuseclass}

\subsubsection{.loc{[}:,{[}names…{]}{]} Several columns}
\label{\detokenize{content/04_PythonEssentials/PythonPandasDataframes:loc-names-several-columns}}
\sphinxAtStartPar
Passing a list in either the rows or columns portion of the loc statement will allow multiple rows or columns to be displayed.

\begin{sphinxuseclass}{cell}\begin{sphinxVerbatimInput}

\begin{sphinxuseclass}{cell_input}
\begin{sphinxVerbatim}[commandchars=\\\{\}]
\PYG{n}{df}\PYG{o}{.}\PYG{n}{loc}\PYG{p}{[}\PYG{p}{[}\PYG{l+m+mi}{2021}\PYG{p}{,}\PYG{l+m+mi}{2024}\PYG{p}{]}\PYG{p}{,}\PYG{p}{[}\PYG{l+s+s1}{\PYGZsq{}}\PYG{l+s+s1}{B}\PYG{l+s+s1}{\PYGZsq{}}\PYG{p}{,}\PYG{l+s+s1}{\PYGZsq{}}\PYG{l+s+s1}{NEW}\PYG{l+s+s1}{\PYGZsq{}}\PYG{p}{]}\PYG{p}{]}
\end{sphinxVerbatim}

\end{sphinxuseclass}\end{sphinxVerbatimInput}
\begin{sphinxVerbatimOutput}

\begin{sphinxuseclass}{cell_output}
\begin{sphinxVerbatim}[commandchars=\\\{\}]
       B  NEW
2021  17   12
2024  17    2
\end{sphinxVerbatim}

\end{sphinxuseclass}\end{sphinxVerbatimOutput}

\end{sphinxuseclass}

\subsubsection{.loc using the colon to select a range}
\label{\detokenize{content/04_PythonEssentials/PythonPandasDataframes:loc-using-the-colon-to-select-a-range}}
\sphinxAtStartPar
with the colon operator we can also select a range of results.

\sphinxAtStartPar
Here from 2018 to 2019.

\begin{sphinxuseclass}{cell}\begin{sphinxVerbatimInput}

\begin{sphinxuseclass}{cell_input}
\begin{sphinxVerbatim}[commandchars=\\\{\}]
\PYG{n}{df}\PYG{o}{.}\PYG{n}{loc}\PYG{p}{[}\PYG{l+m+mi}{2021}\PYG{p}{:}\PYG{l+m+mi}{2023}\PYG{p}{,}\PYG{p}{[}\PYG{l+s+s1}{\PYGZsq{}}\PYG{l+s+s1}{B}\PYG{l+s+s1}{\PYGZsq{}}\PYG{p}{,}\PYG{l+s+s1}{\PYGZsq{}}\PYG{l+s+s1}{NEW}\PYG{l+s+s1}{\PYGZsq{}}\PYG{p}{]}\PYG{p}{]}
\end{sphinxVerbatim}

\end{sphinxuseclass}\end{sphinxVerbatimInput}
\begin{sphinxVerbatimOutput}

\begin{sphinxuseclass}{cell_output}
\begin{sphinxVerbatim}[commandchars=\\\{\}]
       B  NEW
2021  17   12
2022  17   10
2023  17   14
\end{sphinxVerbatim}

\end{sphinxuseclass}\end{sphinxVerbatimOutput}

\end{sphinxuseclass}

\subsubsection{.loc{[}{]} can also be used on the left hand side to assign values to specific cells}
\label{\detokenize{content/04_PythonEssentials/PythonPandasDataframes:loc-can-also-be-used-on-the-left-hand-side-to-assign-values-to-specific-cells}}
\sphinxAtStartPar
This can be very handy when updating scenarios.

\begin{sphinxuseclass}{cell}\begin{sphinxVerbatimInput}

\begin{sphinxuseclass}{cell_input}
\begin{sphinxVerbatim}[commandchars=\\\{\}]
\PYG{n}{df}\PYG{o}{.}\PYG{n}{loc}\PYG{p}{[}\PYG{l+m+mi}{2022}\PYG{p}{:}\PYG{l+m+mi}{2024}\PYG{p}{,}\PYG{l+s+s1}{\PYGZsq{}}\PYG{l+s+s1}{NEW}\PYG{l+s+s1}{\PYGZsq{}}\PYG{p}{]} \PYG{o}{=} \PYG{l+m+mi}{17}
\PYG{n}{df}
\end{sphinxVerbatim}

\end{sphinxuseclass}\end{sphinxVerbatimInput}
\begin{sphinxVerbatimOutput}

\begin{sphinxuseclass}{cell_output}
\begin{sphinxVerbatim}[commandchars=\\\{\}]
        A   B  NEW    X  THE\PYGZus{}ANSWER
2020  100  17   11  187          42
2021  100  17   12  204          42
2022  100  17   17  170          42
2023  100  17   17  238          42
2024  100  17   17   34          42
2025  100  17    1   17          42
\end{sphinxVerbatim}

\end{sphinxuseclass}\end{sphinxVerbatimOutput}

\end{sphinxuseclass}
\begin{sphinxadmonition}{warning}{Warning:}
\sphinxAtStartPar
The dimensions on the right hand side of = and the left hand side should match. That is: either the dimensions should be the same, or the right hand side should be \sphinxcode{\sphinxupquote{broadcasted}} into the left hand slice.

\sphinxAtStartPar
For more on broadcasting \sphinxhref{https://jakevdp.github.io/PythonDataScienceHandbook/02.05-computation-on-arrays-broadcasting.html}{see here}
\end{sphinxadmonition}

\sphinxAtStartPar
\sphinxstylestrong{For more info on the .loc{[}{]} method}
\begin{itemize}
\item {} 
\sphinxAtStartPar
\sphinxhref{https://pandas.pydata.org/docs/reference/api/pandas.DataFrame.loc.html}{Description}

\item {} 
\sphinxAtStartPar
\sphinxhref{https://www.google.com/search?q=pandas+dataframe+loc\&newwindow=1}{Search}

\end{itemize}

\sphinxAtStartPar
\sphinxstylestrong{For more info on pandas:}
\begin{itemize}
\item {} 
\sphinxAtStartPar
\sphinxhref{https://pandas.pydata.org/}{Pandas homepage}

\item {} 
\sphinxAtStartPar
\sphinxhref{https://pandas.pydata.org/pandas-docs/stable/getting\_started/tutorials.html}{Pandas community tutorials}

\end{itemize}

\sphinxstepscope


\chapter{Modelflow extensions to pandas}
\label{\detokenize{content/04_PythonEssentials/UpdateCommand:modelflow-extensions-to-pandas}}\label{\detokenize{content/04_PythonEssentials/UpdateCommand::doc}}
\sphinxAtStartPar
Any class can have both properties (data) and methods (functions that operate on the data of the particular instance of the class). With object\sphinxhyphen{}oriented programming languages like python, classes can be built as supersets of existing classes. The \sphinxcode{\sphinxupquote{modelflow}} class \sphinxcode{\sphinxupquote{model}} inherits or encapsulates all of the features of the pandas dataframe and extends it in many important ways.  Some of the methods below are standard pandas methods, others have been added to it by \sphinxcode{\sphinxupquote{modelflow}} features

\sphinxAtStartPar
Data in a dataframe can be modified directly with built\sphinxhyphen{}in pandas functionalities like \sphinxcode{\sphinxupquote{.loc{[}{]}}} and \sphinxcode{\sphinxupquote{eval()}}, but \sphinxcode{\sphinxupquote{modelflow}} extends these capabilities with in important ways with the \sphinxcode{\sphinxupquote{.upd()}} and \sphinxcode{\sphinxupquote{.mfcalc()}} methods.


\section{Setting up the python environment}
\label{\detokenize{content/04_PythonEssentials/UpdateCommand:setting-up-the-python-environment}}
\sphinxAtStartPar
In order to use \sphinxcode{\sphinxupquote{.upd()}} all of the necessary libraries must be \sphinxstylestrong{imported} into the python session.

\begin{sphinxuseclass}{cell}\begin{sphinxVerbatimInput}

\begin{sphinxuseclass}{cell_input}
\begin{sphinxVerbatim}[commandchars=\\\{\}]
\PYG{o}{\PYGZpc{}}\PYG{k}{load\PYGZus{}ext} autoreload
\PYG{o}{\PYGZpc{}}\PYG{k}{autoreload} 2

\PYG{c+c1}{\PYGZsh{} First import pandas and the model into the  workspace}
\PYG{c+c1}{\PYGZsh{} There is no problem importing multiple times, though it is not very efficient.}
\PYG{k+kn}{import} \PYG{n+nn}{pandas} \PYG{k}{as} \PYG{n+nn}{pd}

\PYG{k+kn}{from} \PYG{n+nn}{modelclass} \PYG{k+kn}{import} \PYG{n}{model} 
\PYG{c+c1}{\PYGZsh{} functions that improve rendering of modelflow outputs under Jupyter Notebook}
\PYG{n}{model}\PYG{o}{.}\PYG{n}{widescreen}\PYG{p}{(}\PYG{p}{)}
\PYG{n}{model}\PYG{o}{.}\PYG{n}{scroll\PYGZus{}off}\PYG{p}{(}\PYG{p}{)}
\end{sphinxVerbatim}

\end{sphinxuseclass}\end{sphinxVerbatimInput}
\begin{sphinxVerbatimOutput}

\begin{sphinxuseclass}{cell_output}
\begin{sphinxVerbatim}[commandchars=\\\{\}]
\PYGZlt{}IPython.core.display.HTML object\PYGZgt{}
\end{sphinxVerbatim}

\end{sphinxuseclass}\end{sphinxVerbatimOutput}

\end{sphinxuseclass}

\section{Column names in  Modelflow}
\label{\detokenize{content/04_PythonEssentials/UpdateCommand:column-names-in-modelflow}}
\begin{sphinxShadowBox}
\sphinxstylesidebartitle{Modelflow variable names}

\sphinxAtStartPar
Modelflow places more restrictions on column names than do pandas \sphinxstyleemphasis{per se}.
\end{sphinxShadowBox}

\sphinxAtStartPar
While pandas dataframes are very liberal in what names can be given to columns, \sphinxcode{\sphinxupquote{modelflow}} is more restrictive.

\sphinxAtStartPar
Specifically, in modelflow a variable name must:
\begin{itemize}
\item {} 
\sphinxAtStartPar
start with a letter

\item {} 
\sphinxAtStartPar
be upper case

\end{itemize}

\sphinxAtStartPar
Thus while all these are legal column names in pandas, some are illegal in modelflow.


\begin{savenotes}\sphinxattablestart
\centering
\begin{tabulary}{\linewidth}[t]{|T|T|T|}
\hline
\sphinxstyletheadfamily 
\sphinxAtStartPar
Variable Name
&\sphinxstyletheadfamily 
\sphinxAtStartPar
Legal in modelfow?
&\sphinxstyletheadfamily 
\sphinxAtStartPar
Reason
\\
\hline
\sphinxAtStartPar
IB
&
\sphinxAtStartPar
yes
&
\sphinxAtStartPar
Starts with a letter and is uppercase
\\
\hline
\sphinxAtStartPar
ib
&
\sphinxAtStartPar
no
&
\sphinxAtStartPar
 lowercase letters are not allowed
\\
\hline
\sphinxAtStartPar
42ANSWER
&
\sphinxAtStartPar
No
&
\sphinxAtStartPar
 does not start with a letter 
\\
\hline
\sphinxAtStartPar
\_HORSE1
&
\sphinxAtStartPar
No
&
\sphinxAtStartPar
does not start with a letter 
\\
\hline
\sphinxAtStartPar
A\_VERY\_LONG\_NAME\_THAT\_IS\_LEGAL\_3
&
\sphinxAtStartPar
Yes
&
\sphinxAtStartPar
 Starts with a letter and is uppercase 
\\
\hline
\end{tabulary}
\par
\sphinxattableend\end{savenotes}


\section{.index and time dimensions in Modelflow}
\label{\detokenize{content/04_PythonEssentials/UpdateCommand:index-and-time-dimensions-in-modelflow}}
\sphinxAtStartPar
As we saw above, series have indices.  Dataframes also have indices, which are the row names of the dataframe.

\sphinxAtStartPar
In \sphinxcode{\sphinxupquote{modelflow}} the index series is typically understood to represent a date.

\sphinxAtStartPar
For yearly models a list of integers like in the above example works fine.

\sphinxAtStartPar
For higher frequency models (quarterly, monthly, weekly,daily, etc.) the index can be one of pandas datatypes.

\begin{sphinxadmonition}{warning}{Warning:}
\sphinxAtStartPar
Not all datetypes work well with the graphics routines of modelflow.  Users are advised to use the \sphinxcode{\sphinxupquote{pd.period\_range()}} method to generate date indexes.

\sphinxAtStartPar
For example:

\begin{sphinxVerbatim}[commandchars=\\\{\}]
    \PYG{n}{dates} \PYG{o}{=} \PYG{n}{pd}\PYG{o}{.}\PYG{n}{period\PYGZus{}range}\PYG{p}{(}\PYG{n}{start}\PYG{o}{=}\PYG{l+s+s1}{\PYGZsq{}}\PYG{l+s+s1}{1975q1}\PYG{l+s+s1}{\PYGZsq{}}\PYG{p}{,}\PYG{n}{end}\PYG{o}{=}\PYG{l+s+s1}{\PYGZsq{}}\PYG{l+s+s1}{2125q4}\PYG{l+s+s1}{\PYGZsq{}}\PYG{p}{,}\PYG{n}{freq}\PYG{o}{=}\PYG{l+s+s1}{\PYGZsq{}}\PYG{l+s+s1}{Q}\PYG{l+s+s1}{\PYGZsq{}}\PYG{p}{)}
    \PYG{n}{df}\PYG{o}{.}\PYG{n}{index}\PYG{o}{=}\PYG{n}{dates}
\end{sphinxVerbatim}
\end{sphinxadmonition}


\section{Leads and lags}
\label{\detokenize{content/04_PythonEssentials/UpdateCommand:leads-and-lags}}
\sphinxAtStartPar
In modelflow leads and lags can be indicated by following the variable with a parenthesis and either \sphinxhyphen{}1 or \sphinxhyphen{}2 two for one or two period lags (where the number following the negative sign indicates the number of time periods that are lagged). Positive numbers are used for forward leads (no +sign required).

\sphinxAtStartPar
When a method defined by the \sphinxcode{\sphinxupquote{modelflow}} class encounters something like \sphinxcode{\sphinxupquote{A(\sphinxhyphen{}1)}}, it will take the value from the row above the current row. No matter if the index is an integer, a year, quarter or a millisecond. The same goes for leads, \sphinxcode{\sphinxupquote{A(+1)}} will return the value of \sphinxcode{\sphinxupquote{A}} in the next row.

\sphinxAtStartPar
As a result in a quarterly model \sphinxcode{\sphinxupquote{B=A(\sphinxhyphen{}4)}} would assign B the value of A from the same quarter in the previous year.


\section{.upd() method of modelflow}
\label{\detokenize{content/04_PythonEssentials/UpdateCommand:upd-method-of-modelflow}}
\sphinxAtStartPar
The \sphinxcode{\sphinxupquote{.upd()}} method extends pandas by giving the user a concise and expressive way to modify data in a \sphinxcode{\sphinxupquote{dataframe}} using a syntax that a database\sphinxhyphen{}manager or macroeconomic modeler might find more natural.

\sphinxAtStartPar
Notably it allows the user to employ formula’s to do updates, and supports both lags and leads on variables.

\sphinxAtStartPar
\sphinxcode{\sphinxupquote{.upd()}} can be used to:
\begin{itemize}
\item {} 
\sphinxAtStartPar
Perform different types of  updates

\item {} 
\sphinxAtStartPar
Perform multiple updates each on a new line

\item {} 
\sphinxAtStartPar
Perform changes over specific periods

\item {} 
\sphinxAtStartPar
Use one input which is used for all time frames, or a separate input for each time

\item {} 
\sphinxAtStartPar
Preserve pre\sphinxhyphen{}shock growth rates for out of sample time\sphinxhyphen{}periods

\item {} 
\sphinxAtStartPar
Display results

\end{itemize}


\subsection{\sphinxstyleliteralintitle{\sphinxupquote{.upd()}} method operators}
\label{\detokenize{content/04_PythonEssentials/UpdateCommand:upd-method-operators}}
\sphinxAtStartPar
Below are some of the operators that can be used in the \sphinxcode{\sphinxupquote{.upd()}} method

\sphinxAtStartPar
\sphinxstylestrong{Types of update:}


\begin{savenotes}\sphinxattablestart
\centering
\begin{tabulary}{\linewidth}[t]{|T|T|}
\hline
\sphinxstyletheadfamily 
\sphinxAtStartPar
Update to perform
&\sphinxstyletheadfamily 
\sphinxAtStartPar
Use this operator
\\
\hline
\sphinxAtStartPar
Set a variable equal to the input
&
\sphinxAtStartPar
=
\\
\hline
\sphinxAtStartPar
Add the input to the LHS variable
&
\sphinxAtStartPar
+
\\
\hline
\sphinxAtStartPar
Set the variable to itself multiplied by the input
&
\sphinxAtStartPar
*
\\
\hline
\sphinxAtStartPar
Increase/Decrease the variable by a percent of itself (1+input/100)
&
\sphinxAtStartPar
\%
\\
\hline
\sphinxAtStartPar
Set the growth rate of the variable to the input
&
\sphinxAtStartPar
=growth
\\
\hline
\sphinxAtStartPar
Change the growth rate of the variable to its current growth rate plus the input value in percentage points
&
\sphinxAtStartPar
+growth
\\
\hline
\sphinxAtStartPar
Specify the amount by which the variable should increase from its previous period level (\(\Delta = var_t - var_{t-1}\))
&
\sphinxAtStartPar
=diff
\\
\hline
\end{tabulary}
\par
\sphinxattableend\end{savenotes}

\begin{sphinxadmonition}{note}{Note:}
\sphinxAtStartPar
The syntax of an update command requires that there be a space between variable names and the operators.

\sphinxAtStartPar
Thus \sphinxcode{\sphinxupquote{df.upd("A = 7")}} is fine, but \sphinxcode{\sphinxupquote{df.upd("A =7")}} will generate an error.

\sphinxAtStartPar
Similarly  \sphinxcode{\sphinxupquote{df.upd("A * 1.1")}} is fine, but \sphinxcode{\sphinxupquote{df.upd("A* 1.1")}} will generate an error.
\end{sphinxadmonition}


\subsection{\sphinxstyleliteralintitle{\sphinxupquote{.upd()}} some examples}
\label{\detokenize{content/04_PythonEssentials/UpdateCommand:upd-some-examples}}
\sphinxAtStartPar
First, create a dataframe using standard pandas syntax.  In this instance with years as the index and a dictionary defining the variables and their data.

\begin{sphinxuseclass}{cell}\begin{sphinxVerbatimInput}

\begin{sphinxuseclass}{cell_input}
\begin{sphinxVerbatim}[commandchars=\\\{\}]
\PYG{c+c1}{\PYGZsh{} Create a dataframe using standard pandas}

\PYG{n}{df} \PYG{o}{=} \PYG{n}{pd}\PYG{o}{.}\PYG{n}{DataFrame}\PYG{p}{(}\PYG{p}{\PYGZob{}}\PYG{l+s+s1}{\PYGZsq{}}\PYG{l+s+s1}{B}\PYG{l+s+s1}{\PYGZsq{}}\PYG{p}{:} \PYG{p}{[}\PYG{l+m+mi}{1}\PYG{p}{,}\PYG{l+m+mi}{1}\PYG{p}{,}\PYG{l+m+mi}{1}\PYG{p}{,}\PYG{l+m+mi}{1}\PYG{p}{,}\PYG{l+m+mi}{1}\PYG{p}{,}\PYG{l+m+mi}{1}\PYG{p}{]}\PYG{p}{,}\PYG{l+s+s1}{\PYGZsq{}}\PYG{l+s+s1}{C}\PYG{l+s+s1}{\PYGZsq{}}\PYG{p}{:}\PYG{p}{[}\PYG{l+m+mi}{1}\PYG{p}{,}\PYG{l+m+mi}{2}\PYG{p}{,}\PYG{l+m+mi}{3}\PYG{p}{,}\PYG{l+m+mi}{6}\PYG{p}{,}\PYG{l+m+mi}{8}\PYG{p}{,}\PYG{l+m+mi}{9}\PYG{p}{]}\PYG{p}{,}\PYG{l+s+s1}{\PYGZsq{}}\PYG{l+s+s1}{E}\PYG{l+s+s1}{\PYGZsq{}}\PYG{p}{:}\PYG{p}{[}\PYG{l+m+mi}{4}\PYG{p}{,}\PYG{l+m+mi}{4}\PYG{p}{,}\PYG{l+m+mi}{4}\PYG{p}{,}\PYG{l+m+mi}{4}\PYG{p}{,}\PYG{l+m+mi}{4}\PYG{p}{,}\PYG{l+m+mi}{4}\PYG{p}{]}\PYG{p}{\PYGZcb{}}\PYG{p}{,}\PYG{n}{index}\PYG{o}{=}\PYG{p}{[}\PYG{n}{v} \PYG{k}{for} \PYG{n}{v} \PYG{o+ow}{in} \PYG{n+nb}{range}\PYG{p}{(}\PYG{l+m+mi}{2020}\PYG{p}{,}\PYG{l+m+mi}{2026}\PYG{p}{)}\PYG{p}{]}\PYG{p}{)}
                  
\PYG{n}{df}
\end{sphinxVerbatim}

\end{sphinxuseclass}\end{sphinxVerbatimInput}
\begin{sphinxVerbatimOutput}

\begin{sphinxuseclass}{cell_output}
\begin{sphinxVerbatim}[commandchars=\\\{\}]
      B  C  E
2020  1  1  4
2021  1  2  4
2022  1  3  4
2023  1  6  4
2024  1  8  4
2025  1  9  4
\end{sphinxVerbatim}

\end{sphinxuseclass}\end{sphinxVerbatimOutput}

\end{sphinxuseclass}

\subsubsection{Use .upd to create a new variable (= operator)}
\label{\detokenize{content/04_PythonEssentials/UpdateCommand:use-upd-to-create-a-new-variable-operator}}
\sphinxAtStartPar
With standard pandas a user can add a column (series) to a dataframe simply by assigning a adding to a dataframe.  For example:

\sphinxAtStartPar
\sphinxcode{\sphinxupquote{df{[}'NEW2'{]}={[}17,12,14,15{]}}}

\sphinxAtStartPar
\sphinxcode{\sphinxupquote{.upd()}} provides this functionality as well.

\begin{sphinxuseclass}{cell}\begin{sphinxVerbatimInput}

\begin{sphinxuseclass}{cell_input}
\begin{sphinxVerbatim}[commandchars=\\\{\}]
\PYG{n}{df2}\PYG{o}{=}\PYG{n}{df}\PYG{o}{.}\PYG{n}{upd}\PYG{p}{(}\PYG{l+s+s1}{\PYGZsq{}}\PYG{l+s+s1}{c = 142}\PYG{l+s+s1}{\PYGZsq{}}\PYG{p}{)} 
\PYG{n}{df2}
\end{sphinxVerbatim}

\end{sphinxuseclass}\end{sphinxVerbatimInput}
\begin{sphinxVerbatimOutput}

\begin{sphinxuseclass}{cell_output}
\begin{sphinxVerbatim}[commandchars=\\\{\}]
      B    C  E
2020  1  142  4
2021  1  142  4
2022  1  142  4
2023  1  142  4
2024  1  142  4
2025  1  142  4
\end{sphinxVerbatim}

\end{sphinxuseclass}\end{sphinxVerbatimOutput}

\end{sphinxuseclass}
\begin{sphinxadmonition}{note}{Note:}
\sphinxAtStartPar
The new variable name was entered as a lower case ‘c’ here.  Lowercase letters are not legal \sphinxcode{\sphinxupquote{modelflow}} variable names.  The \sphinxcode{\sphinxupquote{.upd()}} method knows that it is part of modelflow and knows this rule. As a result, it automatically translates lowercase entries into upper case so that the statement works.
\end{sphinxadmonition}


\subsubsection{Multiple updates and specific time periods}
\label{\detokenize{content/04_PythonEssentials/UpdateCommand:multiple-updates-and-specific-time-periods}}
\sphinxAtStartPar
The modelflow method \sphinxcode{\sphinxupquote{.upd()}} takes a string as an argument.  That string can contain a single update command or can contain multiple commands.

\sphinxAtStartPar
Moreover by including a <Begin End> date clause in a given update command, the update will be restricted to the associated time period.

\sphinxAtStartPar
The below illustrates this, modifying two existing variables A, B over different time periods and creating two new variables, \sphinxcode{\sphinxupquote{C}} and \sphinxcode{\sphinxupquote{D}}.

\begin{sphinxadmonition}{note}{Note:}
\sphinxAtStartPar
The third line inherits the time period of the previous line.

\sphinxAtStartPar
Note also, the submitted string can include comments as well (denoted with the standard python \# indicator).
\end{sphinxadmonition}

\begin{sphinxuseclass}{cell}\begin{sphinxVerbatimInput}

\begin{sphinxuseclass}{cell_input}
\begin{sphinxVerbatim}[commandchars=\\\{\}]
\PYG{n}{df}\PYG{o}{.}\PYG{n}{upd}\PYG{p}{(}\PYG{l+s+s2}{\PYGZdq{}\PYGZdq{}\PYGZdq{}}
\PYG{l+s+s2}{\PYGZsh{} Same number of values as years}
\PYG{l+s+s2}{\PYGZlt{}2021 2024\PYGZgt{} A = 42 44 45 46    \PYGZsh{} 4 years}
\PYG{l+s+s2}{\PYGZlt{}2020     \PYGZgt{} B = 200            \PYGZsh{} 1 year }
\PYG{l+s+s2}{c = 500                        \PYGZsh{} Same period as previous line}
\PYG{l+s+s2}{\PYGZlt{}\PYGZhy{}0 \PYGZhy{}1\PYGZgt{} D = 33                   \PYGZsh{} All years }
\PYG{l+s+s2}{\PYGZdq{}\PYGZdq{}\PYGZdq{}}\PYG{p}{)}
\end{sphinxVerbatim}

\end{sphinxuseclass}\end{sphinxVerbatimInput}
\begin{sphinxVerbatimOutput}

\begin{sphinxuseclass}{cell_output}
\begin{sphinxVerbatim}[commandchars=\\\{\}]
        B    C  E     A     D
2020  200  500  4   0.0  33.0
2021    1    2  4  42.0  33.0
2022    1    3  4  44.0  33.0
2023    1    6  4  45.0  33.0
2024    1    8  4  46.0  33.0
2025    1    9  4   0.0  33.0
\end{sphinxVerbatim}

\end{sphinxuseclass}\end{sphinxVerbatimOutput}

\end{sphinxuseclass}
\begin{sphinxadmonition}{note}{\sphinxstylestrong{Box  Time scope of .upd() commands}}

\sphinxAtStartPar
Made this a margin just to see

\sphinxAtStartPar
The update command takes a variety of mathematical operators \sphinxcode{\sphinxupquote{=, +, *, \% =GROWTH, +GROWTH, =DIFF}} and applies them to data for the period set in the leading <>.

\sphinxAtStartPar
If the user wants to modify a series or group of series for only a specific point in time or a period of time, she can indicate the period in the command line.
\begin{itemize}
\item {} 
\sphinxAtStartPar
If \sphinxstylestrong{one date} is specified the operation is applied to a single point in time

\item {} 
\sphinxAtStartPar
If \sphinxstylestrong{two dates}  are specifies the operation is applied over a period of time.

\end{itemize}

\sphinxAtStartPar
\sphinxstylestrong{The selected time period will persist} until re\sphinxhyphen{}set with a new time specification. This is useful to if several variables are going to be updated for the same time period, but must be kept in mind if subsequent commands are to operate over a different time period.

\sphinxAtStartPar
The time period can be rest to the full time\sphinxhyphen{}period by using the special <\sphinxhyphen{}0 \sphinxhyphen{}1> time period.  More generally:
\begin{itemize}
\item {} 
\sphinxAtStartPar
\sphinxhyphen{}0 indicates the start of the dataframe

\item {} 
\sphinxAtStartPar
\sphinxhyphen{}1 indicates the end of the dataframe

\end{itemize}

\sphinxAtStartPar
If no time is provided the dataframe start and end period will be used.
\end{sphinxadmonition}


\subsubsection{Setting specific datapoints to specific values}
\label{\detokenize{content/04_PythonEssentials/UpdateCommand:setting-specific-datapoints-to-specific-values}}
\sphinxAtStartPar
This example, demonstrates the equals operator.  The \sphinxcode{\sphinxupquote{=}} operator indicates that the variable a should be set equal to the indicated values following the \sphinxcode{\sphinxupquote{=}} operator (42 44 45 46 in the first line, 200 in the second and 500 inthe third). The dates enclosed in <> indicate the period over which the change should be applied.

\sphinxAtStartPar
Either:
\begin{itemize}
\item {} 
\sphinxAtStartPar
The number of data points provided must match the number of dates in the period, Or

\item {} 
\sphinxAtStartPar
Only one data point is provided, it is applied to all dates in the period.

\end{itemize}

\sphinxAtStartPar
If only one period is to be modified then it can be followed by just one date.

\sphinxAtStartPar
Note that the final line inherited the time period set in the second line.

\begin{sphinxuseclass}{cell}\begin{sphinxVerbatimInput}

\begin{sphinxuseclass}{cell_input}
\begin{sphinxVerbatim}[commandchars=\\\{\}]
\PYG{n}{df}\PYG{o}{.}\PYG{n}{upd}\PYG{p}{(}\PYG{l+s+s2}{\PYGZdq{}\PYGZdq{}\PYGZdq{}}
\PYG{l+s+s2}{\PYGZsh{} Same number of values as years}
\PYG{l+s+s2}{\PYGZlt{}2021 2024\PYGZgt{} A = 42 44 45 46    \PYGZsh{} 4 years}
\PYG{l+s+s2}{\PYGZlt{}2023     \PYGZgt{} B = 200            \PYGZsh{} 1 year }
\PYG{l+s+s2}{c = 500}
\PYG{l+s+s2}{\PYGZdq{}\PYGZdq{}\PYGZdq{}}\PYG{p}{)}
\end{sphinxVerbatim}

\end{sphinxuseclass}\end{sphinxVerbatimInput}
\begin{sphinxVerbatimOutput}

\begin{sphinxuseclass}{cell_output}
\begin{sphinxVerbatim}[commandchars=\\\{\}]
        B    C  E     A
2020    1    1  4   0.0
2021    1    2  4  42.0
2022    1    3  4  44.0
2023  200  500  4  45.0
2024    1    8  4  46.0
2025    1    9  4   0.0
\end{sphinxVerbatim}

\end{sphinxuseclass}\end{sphinxVerbatimOutput}

\end{sphinxuseclass}

\subsubsection{Adding  the specified  values to all values in a range (the + operator)}
\label{\detokenize{content/04_PythonEssentials/UpdateCommand:adding-the-specified-values-to-all-values-in-a-range-the-operator}}
\sphinxAtStartPar
NB: Here upd with the  + operator indicates that we are adding 42.

\begin{sphinxuseclass}{cell}\begin{sphinxVerbatimInput}

\begin{sphinxuseclass}{cell_input}
\begin{sphinxVerbatim}[commandchars=\\\{\}]
\PYG{n}{df}\PYG{o}{.}\PYG{n}{upd}\PYG{p}{(}\PYG{l+s+s1}{\PYGZsq{}\PYGZsq{}\PYGZsq{}}
\PYG{l+s+s1}{\PYGZsh{} Or one number to all years in between start and end }
\PYG{l+s+s1}{\PYGZlt{}2022 2024\PYGZgt{} B  +  42    \PYGZsh{} one value broadcast to 3 years }
\PYG{l+s+s1}{\PYGZsq{}\PYGZsq{}\PYGZsq{}}\PYG{p}{)}
\end{sphinxVerbatim}

\end{sphinxuseclass}\end{sphinxVerbatimInput}
\begin{sphinxVerbatimOutput}

\begin{sphinxuseclass}{cell_output}
\begin{sphinxVerbatim}[commandchars=\\\{\}]
       B  C  E
2020   1  1  4
2021   1  2  4
2022  43  3  4
2023  43  6  4
2024  43  8  4
2025   1  9  4
\end{sphinxVerbatim}

\end{sphinxuseclass}\end{sphinxVerbatimOutput}

\end{sphinxuseclass}

\subsubsection{Multiplying all values in a range by the specified values (the * operator)}
\label{\detokenize{content/04_PythonEssentials/UpdateCommand:multiplying-all-values-in-a-range-by-the-specified-values-the-operator}}
\begin{sphinxuseclass}{cell}\begin{sphinxVerbatimInput}

\begin{sphinxuseclass}{cell_input}
\begin{sphinxVerbatim}[commandchars=\\\{\}]
\PYG{n}{df}\PYG{o}{.}\PYG{n}{upd}\PYG{p}{(}\PYG{l+s+s1}{\PYGZsq{}\PYGZsq{}\PYGZsq{}}
\PYG{l+s+s1}{\PYGZsh{} Same number of values as years}
\PYG{l+s+s1}{\PYGZlt{}2021 2023\PYGZgt{} A *  42 44 55}
\PYG{l+s+s1}{\PYGZsq{}\PYGZsq{}\PYGZsq{}}\PYG{p}{)}
\end{sphinxVerbatim}

\end{sphinxuseclass}\end{sphinxVerbatimInput}
\begin{sphinxVerbatimOutput}

\begin{sphinxuseclass}{cell_output}
\begin{sphinxVerbatim}[commandchars=\\\{\}]
      B  C  E    A
2020  1  1  4  0.0
2021  1  2  4  0.0
2022  1  3  4  0.0
2023  1  6  4  0.0
2024  1  8  4  0.0
2025  1  9  4  0.0
\end{sphinxVerbatim}

\end{sphinxuseclass}\end{sphinxVerbatimOutput}

\end{sphinxuseclass}

\subsubsection{Increasing all  values in a range by a  specified percent amount (the \% operator)}
\label{\detokenize{content/04_PythonEssentials/UpdateCommand:increasing-all-values-in-a-range-by-a-specified-percent-amount-the-operator}}
\sphinxAtStartPar
In this example:
\begin{itemize}
\item {} 
\sphinxAtStartPar
A is increased by 42 and 44\% over the range 2021 through 2022.

\item {} 
\sphinxAtStartPar
B is increased by 10 percent in all years

\item {} 
\sphinxAtStartPar
C is a new variable, is created and set to 100 for the whole range (because the previous line set the active range to  <\sphinxhyphen{}0 \sphinxhyphen{}1>)

\item {} 
\sphinxAtStartPar
C is decreased by 12 percent over the range 2023 through 2025.

\end{itemize}

\begin{sphinxuseclass}{cell}\begin{sphinxVerbatimInput}

\begin{sphinxuseclass}{cell_input}
\begin{sphinxVerbatim}[commandchars=\\\{\}]
\PYG{n}{df}\PYG{o}{.}\PYG{n}{upd}\PYG{p}{(}\PYG{l+s+s1}{\PYGZsq{}\PYGZsq{}\PYGZsq{}}
\PYG{l+s+s1}{\PYGZlt{}2021 2022 \PYGZgt{} A }\PYG{l+s+s1}{\PYGZpc{}}\PYG{l+s+s1}{  42 44   }
\PYG{l+s+s1}{\PYGZlt{}\PYGZhy{}0 \PYGZhy{}1\PYGZgt{} B }\PYG{l+s+s1}{\PYGZpc{}}\PYG{l+s+s1}{ 10            \PYGZsh{} all rows }
\PYG{l+s+s1}{C = 100                   \PYGZsh{} all rows persist }
\PYG{l+s+s1}{\PYGZlt{}2023 2025\PYGZgt{} C }\PYG{l+s+s1}{\PYGZpc{}}\PYG{l+s+s1}{ \PYGZhy{}12       \PYGZsh{} now only for 3 years }
\PYG{l+s+s1}{\PYGZsq{}\PYGZsq{}\PYGZsq{}}\PYG{p}{)}
\end{sphinxVerbatim}

\end{sphinxuseclass}\end{sphinxVerbatimInput}
\begin{sphinxVerbatimOutput}

\begin{sphinxuseclass}{cell_output}
\begin{sphinxVerbatim}[commandchars=\\\{\}]
        B    C  E    A
2020  1.1  100  4  0.0
2021  1.1  100  4  0.0
2022  1.1  100  4  0.0
2023  1.1   88  4  0.0
2024  1.1   88  4  0.0
2025  1.1   88  4  0.0
\end{sphinxVerbatim}

\end{sphinxuseclass}\end{sphinxVerbatimOutput}

\end{sphinxuseclass}

\subsubsection{Set the percent growth rate to specified values (=GROWTH)}
\label{\detokenize{content/04_PythonEssentials/UpdateCommand:set-the-percent-growth-rate-to-specified-values-growth}}
\begin{sphinxuseclass}{cell}\begin{sphinxVerbatimInput}

\begin{sphinxuseclass}{cell_input}
\begin{sphinxVerbatim}[commandchars=\\\{\}]
\PYG{n}{res} \PYG{o}{=} \PYG{n}{df}\PYG{o}{.}\PYG{n}{upd}\PYG{p}{(}\PYG{l+s+s1}{\PYGZsq{}\PYGZsq{}\PYGZsq{}}
\PYG{l+s+s1}{\PYGZsh{} Same number of values as years}
\PYG{l+s+s1}{\PYGZlt{}2021 2022\PYGZgt{} A =GROWTH  1 5  }
\PYG{l+s+s1}{\PYGZlt{}2020\PYGZgt{} c = 100 }
\PYG{l+s+s1}{\PYGZlt{}2021 2025\PYGZgt{} c =GROWTH 2 }
\PYG{l+s+s1}{\PYGZsq{}\PYGZsq{}\PYGZsq{}}\PYG{p}{)}
\PYG{n+nb}{print}\PYG{p}{(}\PYG{l+s+sa}{f}\PYG{l+s+s1}{\PYGZsq{}}\PYG{l+s+s1}{Dataframe:}\PYG{l+s+se}{\PYGZbs{}n}\PYG{l+s+si}{\PYGZob{}}\PYG{n}{res}\PYG{l+s+si}{\PYGZcb{}}\PYG{l+s+se}{\PYGZbs{}n}\PYG{l+s+se}{\PYGZbs{}n}\PYG{l+s+s1}{Growth:}\PYG{l+s+se}{\PYGZbs{}n}\PYG{l+s+si}{\PYGZob{}}\PYG{n}{res}\PYG{o}{.}\PYG{n}{pct\PYGZus{}change}\PYG{p}{(}\PYG{p}{)}\PYG{o}{*}\PYG{l+m+mi}{100}\PYG{l+s+si}{\PYGZcb{}}\PYG{l+s+se}{\PYGZbs{}n}\PYG{l+s+s1}{\PYGZsq{}}\PYG{p}{)} \PYG{c+c1}{\PYGZsh{} Explained b}
\end{sphinxVerbatim}

\end{sphinxuseclass}\end{sphinxVerbatimInput}
\begin{sphinxVerbatimOutput}

\begin{sphinxuseclass}{cell_output}
\begin{sphinxVerbatim}[commandchars=\\\{\}]
Dataframe:
      B           C  E    A
2020  1  100.000000  4  0.0
2021  1  102.000000  4  0.0
2022  1  104.040000  4  0.0
2023  1  106.120800  4  0.0
2024  1  108.243216  4  0.0
2025  1  110.408080  4  0.0

Growth:
        B    C    E   A
2020  NaN  NaN  NaN NaN
2021  0.0  2.0  0.0 NaN
2022  0.0  2.0  0.0 NaN
2023  0.0  2.0  0.0 NaN
2024  0.0  2.0  0.0 NaN
2025  0.0  2.0  0.0 NaN
\end{sphinxVerbatim}

\end{sphinxuseclass}\end{sphinxVerbatimOutput}

\end{sphinxuseclass}

\subsubsection{Add or subtract from the existing percent growth rate (+GROWTH operator)}
\label{\detokenize{content/04_PythonEssentials/UpdateCommand:add-or-subtract-from-the-existing-percent-growth-rate-growth-operator}}
\sphinxAtStartPar
The below example is a bit more complicated.

\sphinxAtStartPar
The first line sets the growth rate of A to 1\% in all periods beginning in  2021

\sphinxAtStartPar
The second command adds 2 3 4 5 6 to the growth rates in each period after 2021, resulting in growth rates of 3,4,5,6,7.

\begin{sphinxuseclass}{cell}\begin{sphinxVerbatimInput}

\begin{sphinxuseclass}{cell_input}
\begin{sphinxVerbatim}[commandchars=\\\{\}]
\PYG{n}{res} \PYG{o}{=}\PYG{n}{df}\PYG{o}{.}\PYG{n}{upd}\PYG{p}{(}\PYG{l+s+s1}{\PYGZsq{}\PYGZsq{}\PYGZsq{}}
\PYG{l+s+s1}{\PYGZlt{}2021 \PYGZgt{} A =GROWTH  1  \PYGZsh{} All selected years set to the same growth rate}
\PYG{l+s+s1}{a +growth   2  \PYGZsh{} Add to the existing growth rate these numbers  }
\PYG{l+s+s1}{\PYGZsq{}\PYGZsq{}\PYGZsq{}}\PYG{p}{)}
\PYG{n+nb}{print}\PYG{p}{(}\PYG{l+s+sa}{f}\PYG{l+s+s1}{\PYGZsq{}}\PYG{l+s+s1}{Dataframe:}\PYG{l+s+se}{\PYGZbs{}n}\PYG{l+s+si}{\PYGZob{}}\PYG{n}{res}\PYG{l+s+si}{\PYGZcb{}}\PYG{l+s+se}{\PYGZbs{}n}\PYG{l+s+se}{\PYGZbs{}n}\PYG{l+s+s1}{Growth:}\PYG{l+s+se}{\PYGZbs{}n}\PYG{l+s+si}{\PYGZob{}}\PYG{n}{res}\PYG{o}{.}\PYG{n}{pct\PYGZus{}change}\PYG{p}{(}\PYG{p}{)}\PYG{o}{*}\PYG{l+m+mi}{100}\PYG{l+s+si}{\PYGZcb{}}\PYG{l+s+se}{\PYGZbs{}n}\PYG{l+s+s1}{\PYGZsq{}}\PYG{p}{)}
\end{sphinxVerbatim}

\end{sphinxuseclass}\end{sphinxVerbatimInput}
\begin{sphinxVerbatimOutput}

\begin{sphinxuseclass}{cell_output}
\begin{sphinxVerbatim}[commandchars=\\\{\}]
Dataframe:
      B  C  E    A
2020  1  1  4  0.0
2021  1  2  4  NaN
2022  1  3  4  0.0
2023  1  6  4  0.0
2024  1  8  4  0.0
2025  1  9  4  0.0

Growth:
        B           C    E   A
2020  NaN         NaN  NaN NaN
2021  0.0  100.000000  0.0 NaN
2022  0.0   50.000000  0.0 NaN
2023  0.0  100.000000  0.0 NaN
2024  0.0   33.333333  0.0 NaN
2025  0.0   12.500000  0.0 NaN
\end{sphinxVerbatim}

\end{sphinxuseclass}\end{sphinxVerbatimOutput}

\end{sphinxuseclass}

\subsubsection{Set the change in a variable to specific values (=diff operator)}
\label{\detokenize{content/04_PythonEssentials/UpdateCommand:set-the-change-in-a-variable-to-specific-values-diff-operator}}
\sphinxAtStartPar
\(\Delta = var_t - var_{t-1} = some number\)

\sphinxAtStartPar
Here sets the value of A in 2021 to 2 more than the value of 2020, and the 2022 value as 4 more than the \sphinxstylestrong{revised} value of 2021.

\sphinxAtStartPar
The second line creates a new variable “UPBY2” to the data frame and sets it equal to 100 for all periods,

\sphinxAtStartPar
The third line adds 2 to the previous periods value UPBY2.

\begin{sphinxuseclass}{cell}\begin{sphinxVerbatimInput}

\begin{sphinxuseclass}{cell_input}
\begin{sphinxVerbatim}[commandchars=\\\{\}]
\PYG{n}{df}\PYG{o}{.}\PYG{n}{upd}\PYG{p}{(}\PYG{l+s+s1}{\PYGZsq{}\PYGZsq{}\PYGZsq{}}
\PYG{l+s+s1}{\PYGZlt{} 2021 2022\PYGZgt{} A =diff  2 4   \PYGZsh{} Same number of values as years}
\PYG{l+s+s1}{\PYGZlt{}2020 \PYGZgt{} UpBy2 = 100 \PYGZsh{} sets rows equal to the same  number for all years in between start and end }
\PYG{l+s+s1}{\PYGZlt{}2021 2025\PYGZgt{} UpBy2 =diff  2  }

\PYG{l+s+s1}{\PYGZsq{}\PYGZsq{}\PYGZsq{}}\PYG{p}{)}
\end{sphinxVerbatim}

\end{sphinxuseclass}\end{sphinxVerbatimInput}
\begin{sphinxVerbatimOutput}

\begin{sphinxuseclass}{cell_output}
\begin{sphinxVerbatim}[commandchars=\\\{\}]
      B  C  E    A  UPBY2
2020  1  1  4  0.0  100.0
2021  1  2  4  2.0  102.0
2022  1  3  4  6.0  104.0
2023  1  6  4  0.0  106.0
2024  1  8  4  0.0  108.0
2025  1  9  4  0.0  110.0
\end{sphinxVerbatim}

\end{sphinxuseclass}\end{sphinxVerbatimOutput}

\end{sphinxuseclass}

\subsubsection{Update several variable in one line}
\label{\detokenize{content/04_PythonEssentials/UpdateCommand:update-several-variable-in-one-line}}
\sphinxAtStartPar
Sometime there is a need to update several variable with the same value over the same time frame. To ease this case .update can accept several variables in one line

\begin{sphinxuseclass}{cell}\begin{sphinxVerbatimInput}

\begin{sphinxuseclass}{cell_input}
\begin{sphinxVerbatim}[commandchars=\\\{\}]
\PYG{n}{df}\PYG{o}{.}\PYG{n}{upd}\PYG{p}{(}\PYG{l+s+s1}{\PYGZsq{}\PYGZsq{}\PYGZsq{}}
\PYG{l+s+s1}{\PYGZlt{}2022 2024\PYGZgt{} h i j k =      40      \PYGZsh{} earlier values are set to zero by default}
\PYG{l+s+s1}{\PYGZlt{}2020\PYGZgt{}      p q r s =       1000   \PYGZsh{} All values beginning in 2020 set to 1000}
\PYG{l+s+s1}{\PYGZlt{}2021 \PYGZhy{}1\PYGZgt{}   p q r s =growth 2      \PYGZsh{} \PYGZhy{}1 indicates the last year of dataframe}
\PYG{l+s+s1}{\PYGZsq{}\PYGZsq{}\PYGZsq{}}\PYG{p}{)}
\end{sphinxVerbatim}

\end{sphinxuseclass}\end{sphinxVerbatimInput}
\begin{sphinxVerbatimOutput}

\begin{sphinxuseclass}{cell_output}
\begin{sphinxVerbatim}[commandchars=\\\{\}]
      B  C  E     H     I     J     K            P            Q            R   
2020  1  1  4   0.0   0.0   0.0   0.0  1000.000000  1000.000000  1000.000000  \PYGZbs{}
2021  1  2  4   0.0   0.0   0.0   0.0  1020.000000  1020.000000  1020.000000   
2022  1  3  4  40.0  40.0  40.0  40.0  1040.400000  1040.400000  1040.400000   
2023  1  6  4  40.0  40.0  40.0  40.0  1061.208000  1061.208000  1061.208000   
2024  1  8  4  40.0  40.0  40.0  40.0  1082.432160  1082.432160  1082.432160   
2025  1  9  4   0.0   0.0   0.0   0.0  1104.080803  1104.080803  1104.080803   

                S  
2020  1000.000000  
2021  1020.000000  
2022  1040.400000  
2023  1061.208000  
2024  1082.432160  
2025  1104.080803  
\end{sphinxVerbatim}

\end{sphinxuseclass}\end{sphinxVerbatimOutput}

\end{sphinxuseclass}
\sphinxAtStartPar
\sphinxstylestrong{Recall we have not overwritten df, so the df dataframe is unchanged.}

\begin{sphinxuseclass}{cell}\begin{sphinxVerbatimInput}

\begin{sphinxuseclass}{cell_input}
\begin{sphinxVerbatim}[commandchars=\\\{\}]
\PYG{n}{df}
\end{sphinxVerbatim}

\end{sphinxuseclass}\end{sphinxVerbatimInput}
\begin{sphinxVerbatimOutput}

\begin{sphinxuseclass}{cell_output}
\begin{sphinxVerbatim}[commandchars=\\\{\}]
      B  C  E
2020  1  1  4
2021  1  2  4
2022  1  3  4
2023  1  6  4
2024  1  8  4
2025  1  9  4
\end{sphinxVerbatim}

\end{sphinxuseclass}\end{sphinxVerbatimOutput}

\end{sphinxuseclass}
\begin{sphinxadmonition}{note}{Note:}
\sphinxAtStartPar
The method \sphinxcode{\sphinxupquote{.upd()}} only operates on one variable.  A command like \sphinxcode{\sphinxupquote{.upd('A = B')}} would not work. For these kind of functions, use \sphinxcode{\sphinxupquote{.mfcalc()}} (see next section).
\end{sphinxadmonition}


\subsection{Selected \sphinxstyleliteralintitle{\sphinxupquote{.upd()}} options}
\label{\detokenize{content/04_PythonEssentials/UpdateCommand:selected-upd-options}}

\subsubsection{The keep\_growth option (–kg and –nkg)}
\label{\detokenize{content/04_PythonEssentials/UpdateCommand:the-keep-growth-option-kg-and-nkg}}
\sphinxAtStartPar
When changing data and for certain kinds of simulations, it can sometime be useful to be able to update variables but keep the growth rate in subsequent periods unchanged. In database management this is frequently done when two time\sphinxhyphen{}series with different levels are spliced together. When forecasting this is useful if you have updated historical data but your views on future growth rates are unchanged.

\sphinxAtStartPar
The \sphinxhyphen{}kg or –keep\_growth option instructs modelflow to calculate the growth rate of the existing pre\sphinxhyphen{}change series, and then use it to preserve the pre\sphinxhyphen{}change growth rates of the series for the periods that were \sphinxstylestrong{not} changed.


\paragraph{The default keep\_growth behaviour}
\label{\detokenize{content/04_PythonEssentials/UpdateCommand:the-default-keep-growth-behaviour}}
\sphinxAtStartPar
The \sphinxcode{\sphinxupquote{keep\_growth}} option determines how data in  the time periods after those where an update is executed are treated.

\sphinxAtStartPar
If \sphinxcode{\sphinxupquote{keep\_growth}} is \sphinxcode{\sphinxupquote{False}} then data in the sub\sphinxhyphen{}period after a change is left unchanged.

\sphinxAtStartPar
if \sphinxcode{\sphinxupquote{keep\_growth}} is set to “\sphinxcode{\sphinxupquote{True}}” then the system will preserve the pre\sphinxhyphen{}change growth rate of the affected variable in the time period \sphinxstyleemphasis{after the change}.

\sphinxAtStartPar
By default \sphinxcode{\sphinxupquote{keep\_growth}} is set to \sphinxcode{\sphinxupquote{False}}.

\begin{sphinxadmonition}{note}{Note:}
\sphinxAtStartPar
At the line level:
\begin{itemize}
\item {} 
\sphinxAtStartPar
\sphinxcode{\sphinxupquote{keep\_growth=True}} can be expressed as –kg

\item {} 
\sphinxAtStartPar
\sphinxcode{\sphinxupquote{keep\_growth=False}} can be expressed as –nkg

\end{itemize}
\end{sphinxadmonition}

\sphinxAtStartPar
Consider the followingconcrete example. A \sphinxcode{\sphinxupquote{dataframe}} df has two variables A and B, that each grow by 2\% per period, with A initialized at a level of 100 and B at a level of 110 so that we can see each separately on a graph.

\begin{sphinxuseclass}{cell}\begin{sphinxVerbatimInput}

\begin{sphinxuseclass}{cell_input}
\begin{sphinxVerbatim}[commandchars=\\\{\}]
\PYG{n}{df} \PYG{o}{=} \PYG{n}{pd}\PYG{o}{.}\PYG{n}{DataFrame}\PYG{p}{(}\PYG{l+m+mi}{100}\PYG{p}{,}
     \PYG{n}{index}\PYG{o}{=}\PYG{p}{[}\PYG{n}{v} \PYG{k}{for} \PYG{n}{v} \PYG{o+ow}{in} \PYG{n+nb}{range}\PYG{p}{(}\PYG{l+m+mi}{2020}\PYG{p}{,}\PYG{l+m+mi}{2025}\PYG{p}{)}\PYG{p}{]}\PYG{p}{,}
       \PYG{n}{columns}\PYG{o}{=}\PYG{p}{[}\PYG{l+s+s1}{\PYGZsq{}}\PYG{l+s+s1}{A}\PYG{l+s+s1}{\PYGZsq{}}\PYG{p}{,}\PYG{l+s+s1}{\PYGZsq{}}\PYG{l+s+s1}{B}\PYG{l+s+s1}{\PYGZsq{}}\PYG{p}{]}\PYG{p}{)} 

\PYG{n}{df}\PYG{o}{=}\PYG{n}{df}\PYG{o}{.}\PYG{n}{upd}\PYG{p}{(}\PYG{l+s+s2}{\PYGZdq{}\PYGZdq{}\PYGZdq{}}\PYG{l+s+s2}{\PYGZlt{}2021 \PYGZhy{}1\PYGZgt{} A =growth 2}
\PYG{l+s+s2}{           \PYGZlt{}2020 \PYGZhy{}1\PYGZgt{}   B = 110}
\PYG{l+s+s2}{          \PYGZlt{}2021 \PYGZhy{}1\PYGZgt{}    B =growth 2}
\PYG{l+s+s2}{          }\PYG{l+s+s2}{\PYGZdq{}\PYGZdq{}\PYGZdq{}}\PYG{p}{)}
\PYG{c+c1}{\PYGZsh{} Store these variables for later use in comparisons}
\PYG{n}{df}\PYG{p}{[}\PYG{l+s+s1}{\PYGZsq{}}\PYG{l+s+s1}{A\PYGZus{}ORIG}\PYG{l+s+s1}{\PYGZsq{}}\PYG{p}{]}\PYG{o}{=}\PYG{n}{df}\PYG{p}{[}\PYG{l+s+s1}{\PYGZsq{}}\PYG{l+s+s1}{A}\PYG{l+s+s1}{\PYGZsq{}}\PYG{p}{]}
\PYG{n}{df}\PYG{p}{[}\PYG{l+s+s1}{\PYGZsq{}}\PYG{l+s+s1}{B\PYGZus{}ORIG}\PYG{l+s+s1}{\PYGZsq{}}\PYG{p}{]}\PYG{o}{=}\PYG{n}{df}\PYG{p}{[}\PYG{l+s+s1}{\PYGZsq{}}\PYG{l+s+s1}{B}\PYG{l+s+s1}{\PYGZsq{}}\PYG{p}{]}
\PYG{n}{df}
\end{sphinxVerbatim}

\end{sphinxuseclass}\end{sphinxVerbatimInput}
\begin{sphinxVerbatimOutput}

\begin{sphinxuseclass}{cell_output}
\begin{sphinxVerbatim}[commandchars=\\\{\}]
               A           B      A\PYGZus{}ORIG      B\PYGZus{}ORIG
2020  100.000000  110.000000  100.000000  110.000000
2021  102.000000  112.200000  102.000000  112.200000
2022  104.040000  114.444000  104.040000  114.444000
2023  106.120800  116.732880  106.120800  116.732880
2024  108.243216  119.067538  108.243216  119.067538
\end{sphinxVerbatim}

\end{sphinxuseclass}\end{sphinxVerbatimOutput}

\end{sphinxuseclass}
\begin{sphinxuseclass}{cell}\begin{sphinxVerbatimInput}

\begin{sphinxuseclass}{cell_input}
\begin{sphinxVerbatim}[commandchars=\\\{\}]
\PYG{n}{df}\PYG{p}{[}\PYG{p}{[}\PYG{l+s+s1}{\PYGZsq{}}\PYG{l+s+s1}{A}\PYG{l+s+s1}{\PYGZsq{}}\PYG{p}{,}\PYG{l+s+s1}{\PYGZsq{}}\PYG{l+s+s1}{B}\PYG{l+s+s1}{\PYGZsq{}}\PYG{p}{]}\PYG{p}{]}\PYG{o}{.}\PYG{n}{plot}\PYG{p}{(}\PYG{p}{)}
\end{sphinxVerbatim}

\end{sphinxuseclass}\end{sphinxVerbatimInput}
\begin{sphinxVerbatimOutput}

\begin{sphinxuseclass}{cell_output}
\begin{sphinxVerbatim}[commandchars=\\\{\}]
\PYGZlt{}Axes: \PYGZgt{}
\end{sphinxVerbatim}

\noindent\sphinxincludegraphics{{374953e85d0c2e2763f3012854dd4d5f6efeca46538348600dc4a9e85d654583}.png}

\end{sphinxuseclass}\end{sphinxVerbatimOutput}

\end{sphinxuseclass}
\sphinxAtStartPar
The \sphinxcode{\sphinxupquote{.upd()}} command below modifies both A and B by adding 5 to their levels in each of 2022 and 2023.

\sphinxAtStartPar
For A, this is done with the \sphinxcode{\sphinxupquote{keep\_growth}} option set to \sphinxcode{\sphinxupquote{True}} – the \sphinxcode{\sphinxupquote{\sphinxhyphen{}\sphinxhyphen{}kg}} option in the code below.  This means that for A the growth rate after the shock period 2022\sphinxhyphen{}23 will be unchanged at 2 percent.

\sphinxAtStartPar
For series B the same shock is applied but with keep\_growth set to \sphinxcode{\sphinxupquote{False}} using the –nkg option.

\sphinxAtStartPar
The keep\_growth global variable is ignored in this instance as each line in the update is overriding it using the –kg option (\sphinxcode{\sphinxupquote{keep\_growth=True}}) and –nkg option (\sphinxcode{\sphinxupquote{keep\_growth=False}}).

\begin{sphinxuseclass}{cell}\begin{sphinxVerbatimInput}

\begin{sphinxuseclass}{cell_input}
\begin{sphinxVerbatim}[commandchars=\\\{\}]
\PYG{n}{df}\PYG{o}{=}\PYG{n}{df}\PYG{o}{.}\PYG{n}{upd}\PYG{p}{(}\PYG{l+s+s2}{\PYGZdq{}\PYGZdq{}\PYGZdq{}}
\PYG{l+s+s2}{            \PYGZlt{}2022 2023\PYGZgt{} A + 5 \PYGZhy{}\PYGZhy{}kg}
\PYG{l+s+s2}{            \PYGZlt{}2022 2023\PYGZgt{} B + 5 \PYGZhy{}\PYGZhy{}nkg}
\PYG{l+s+s2}{            }\PYG{l+s+s2}{\PYGZdq{}\PYGZdq{}\PYGZdq{}}\PYG{p}{)}

\PYG{n}{df}\PYG{p}{[}\PYG{p}{[}\PYG{l+s+s1}{\PYGZsq{}}\PYG{l+s+s1}{A}\PYG{l+s+s1}{\PYGZsq{}}\PYG{p}{,}\PYG{l+s+s1}{\PYGZsq{}}\PYG{l+s+s1}{B}\PYG{l+s+s1}{\PYGZsq{}}\PYG{p}{,}\PYG{l+s+s1}{\PYGZsq{}}\PYG{l+s+s1}{A\PYGZus{}ORIG}\PYG{l+s+s1}{\PYGZsq{}}\PYG{p}{,}\PYG{l+s+s1}{\PYGZsq{}}\PYG{l+s+s1}{B\PYGZus{}ORIG}\PYG{l+s+s1}{\PYGZsq{}}\PYG{p}{]}\PYG{p}{]}\PYG{o}{.}\PYG{n}{plot}\PYG{p}{(}\PYG{p}{)}
    
\end{sphinxVerbatim}

\end{sphinxuseclass}\end{sphinxVerbatimInput}
\begin{sphinxVerbatimOutput}

\begin{sphinxuseclass}{cell_output}
\begin{sphinxVerbatim}[commandchars=\\\{\}]
\PYGZlt{}Axes: \PYGZgt{}
\end{sphinxVerbatim}

\noindent\sphinxincludegraphics{{2254b41aab36e7081f298729b7d3db88157168e2e725110a0c601997174132ad}.png}

\end{sphinxuseclass}\end{sphinxVerbatimOutput}

\end{sphinxuseclass}
\sphinxAtStartPar
In the first example ‘A’ (the green and blue lines) the level of A is increased by 5 for two periods (2021\sphinxhyphen{}2022). The levels of the subsequent values are also increased because the previous growth rate (2\%) is now applied to the new higher level of the data in 2022.

\sphinxAtStartPar
For the ‘B’ variable the same level change was input but because of the \sphinxcode{\sphinxupquote{\sphinxhyphen{}\sphinxhyphen{}nkg}} (equivalent to \sphinxcode{\sphinxupquote{keep\_growth=False}}) the periods after the change were unaffected. the shocked variable returns to its pre\sphinxhyphen{}shock level immediately in 2023.

\sphinxAtStartPar
Below are plots the growth rates of the two transformed series.

\sphinxAtStartPar
Here the growth in both series accelerates in 2022, by slightly less than 5 percentage points because a) the base of each is more than 100 in 2021 (because of the 2 percent growth in 2021). Substantially more in the case of  B, which was initialized at 110. In 2023 the growth rate of A returns to 2 percent, while the growth rate of B is actually negative because the level (see earlier graph) has fallen back to its original level.

\begin{sphinxuseclass}{cell}\begin{sphinxVerbatimInput}

\begin{sphinxuseclass}{cell_input}
\begin{sphinxVerbatim}[commandchars=\\\{\}]
\PYG{n}{dfg}\PYG{o}{=}\PYG{n}{df}\PYG{p}{[}\PYG{p}{[}\PYG{l+s+s1}{\PYGZsq{}}\PYG{l+s+s1}{A}\PYG{l+s+s1}{\PYGZsq{}}\PYG{p}{,}\PYG{l+s+s1}{\PYGZsq{}}\PYG{l+s+s1}{B}\PYG{l+s+s1}{\PYGZsq{}}\PYG{p}{]}\PYG{p}{]}\PYG{o}{.}\PYG{n}{pct\PYGZus{}change}\PYG{p}{(}\PYG{p}{)}\PYG{o}{*}\PYG{l+m+mi}{100}
\PYG{n}{dfg}\PYG{o}{.}\PYG{n}{plot}\PYG{p}{(}\PYG{p}{)}
\end{sphinxVerbatim}

\end{sphinxuseclass}\end{sphinxVerbatimInput}
\begin{sphinxVerbatimOutput}

\begin{sphinxuseclass}{cell_output}
\begin{sphinxVerbatim}[commandchars=\\\{\}]
\PYGZlt{}Axes: \PYGZgt{}
\end{sphinxVerbatim}

\noindent\sphinxincludegraphics{{9622d660dd9cc89d3f08f398ab116fb02810a34acc1e66b3cd362a663b1f017d}.png}

\end{sphinxuseclass}\end{sphinxVerbatimOutput}

\end{sphinxuseclass}

\subsubsection{.upd(,,,keep\_growth) some more examples}
\label{\detokenize{content/04_PythonEssentials/UpdateCommand:upd-keep-growth-some-more-examples}}
\sphinxAtStartPar
Initialize a new dataframe, with some growth rate

\begin{sphinxuseclass}{cell}\begin{sphinxVerbatimInput}

\begin{sphinxuseclass}{cell_input}
\begin{sphinxVerbatim}[commandchars=\\\{\}]
\PYG{c+c1}{\PYGZsh{} instantiate a new dataframe with one column \PYGZsq{}A\PYGZsq{} with avlue 100 everywhere and index 2020\PYGZhy{}2025}
\PYG{n}{dftest} \PYG{o}{=} \PYG{n}{pd}\PYG{o}{.}\PYG{n}{DataFrame}\PYG{p}{(}\PYG{l+m+mi}{100}\PYG{p}{,}
       \PYG{n}{index}\PYG{o}{=}\PYG{p}{[}\PYG{n}{v} \PYG{k}{for} \PYG{n}{v} \PYG{o+ow}{in} \PYG{n+nb}{range}\PYG{p}{(}\PYG{l+m+mi}{2020}\PYG{p}{,}\PYG{l+m+mi}{2026}\PYG{p}{)}\PYG{p}{]}\PYG{p}{,} \PYG{c+c1}{\PYGZsh{} create row index}
       \PYG{c+c1}{\PYGZsh{} equivalent to index=[2020,2021,2022,2023,2024,2025] }
       \PYG{n}{columns}\PYG{o}{=}\PYG{p}{[}\PYG{l+s+s1}{\PYGZsq{}}\PYG{l+s+s1}{A}\PYG{l+s+s1}{\PYGZsq{}}\PYG{p}{]}\PYG{p}{)}                                 \PYG{c+c1}{\PYGZsh{} create column name}
\PYG{n}{dftest}
\end{sphinxVerbatim}

\end{sphinxuseclass}\end{sphinxVerbatimInput}
\begin{sphinxVerbatimOutput}

\begin{sphinxuseclass}{cell_output}
\begin{sphinxVerbatim}[commandchars=\\\{\}]
        A
2020  100
2021  100
2022  100
2023  100
2024  100
2025  100
\end{sphinxVerbatim}

\end{sphinxuseclass}\end{sphinxVerbatimOutput}

\end{sphinxuseclass}
\begin{sphinxuseclass}{cell}\begin{sphinxVerbatimInput}

\begin{sphinxuseclass}{cell_input}
\begin{sphinxVerbatim}[commandchars=\\\{\}]
\PYG{c+c1}{\PYGZsh{} Update a to have growth rate accelerationg linearly by 1 from 1 oercent to 5 percent}
\PYG{n}{original} \PYG{o}{=} \PYG{n}{dftest}\PYG{o}{.}\PYG{n}{upd}\PYG{p}{(}\PYG{l+s+s1}{\PYGZsq{}}\PYG{l+s+s1}{\PYGZlt{}2021 2025\PYGZgt{} a =growth 1 2 3 4 5}\PYG{l+s+s1}{\PYGZsq{}}\PYG{p}{)}  
\PYG{n+nb}{print}\PYG{p}{(}\PYG{l+s+sa}{f}\PYG{l+s+s1}{\PYGZsq{}}\PYG{l+s+s1}{Levels:}\PYG{l+s+se}{\PYGZbs{}n}\PYG{l+s+si}{\PYGZob{}}\PYG{n}{original}\PYG{l+s+si}{\PYGZcb{}}\PYG{l+s+se}{\PYGZbs{}n}\PYG{l+s+se}{\PYGZbs{}n}\PYG{l+s+s1}{Growth:}\PYG{l+s+se}{\PYGZbs{}n}\PYG{l+s+si}{\PYGZob{}}\PYG{n}{original}\PYG{o}{.}\PYG{n}{pct\PYGZus{}change}\PYG{p}{(}\PYG{p}{)}\PYG{o}{*}\PYG{l+m+mi}{100}\PYG{l+s+si}{\PYGZcb{}}\PYG{l+s+se}{\PYGZbs{}n}\PYG{l+s+s1}{\PYGZsq{}}\PYG{p}{)}
\end{sphinxVerbatim}

\end{sphinxuseclass}\end{sphinxVerbatimInput}
\begin{sphinxVerbatimOutput}

\begin{sphinxuseclass}{cell_output}
\begin{sphinxVerbatim}[commandchars=\\\{\}]
Levels:
               A
2020  100.000000
2021  101.000000
2022  103.020000
2023  106.110600
2024  110.355024
2025  115.872775

Growth:
        A
2020  NaN
2021  1.0
2022  2.0
2023  3.0
2024  4.0
2025  5.0
\end{sphinxVerbatim}

\end{sphinxuseclass}\end{sphinxVerbatimOutput}

\end{sphinxuseclass}
\sphinxAtStartPar
\sphinxstylestrong{Now update A in 2021 to 2023 to a new value}

\sphinxAtStartPar
Below performs the same operation twice, the first time the updated value is assigned to the \sphinxcode{\sphinxupquote{dataframe}} \sphinxcode{\sphinxupquote{nkg}} and the default behaviour of \sphinxcode{\sphinxupquote{keep\_growth}} is \sphinxcode{\sphinxupquote{False}}

\sphinxAtStartPar
In the second example the \sphinxcode{\sphinxupquote{\sphinxhyphen{}kg}} line option is specified, telling modelflow to maintain the growth rates of the dependent variable in the periods after the update is executed.

\begin{sphinxuseclass}{cell}\begin{sphinxVerbatimInput}

\begin{sphinxuseclass}{cell_input}
\begin{sphinxVerbatim}[commandchars=\\\{\}]
\PYG{n}{nokg} \PYG{o}{=} \PYG{n}{original}\PYG{o}{.}\PYG{n}{upd}\PYG{p}{(}\PYG{l+s+s1}{\PYGZsq{}\PYGZsq{}\PYGZsq{}}
\PYG{l+s+s1}{\PYGZlt{}2021 2025\PYGZgt{}  a =growth 1 2 3 4 5 }
\PYG{l+s+s1}{\PYGZlt{}2021 2023\PYGZgt{}  a = 120  }
\PYG{l+s+s1}{\PYGZsq{}\PYGZsq{}\PYGZsq{}}\PYG{p}{,}\PYG{n}{lprint}\PYG{o}{=}\PYG{l+m+mi}{0}\PYG{p}{)}

\PYG{n}{kg} \PYG{o}{=} \PYG{n}{original}\PYG{o}{.}\PYG{n}{upd}\PYG{p}{(}\PYG{l+s+s1}{\PYGZsq{}\PYGZsq{}\PYGZsq{}}
\PYG{l+s+s1}{\PYGZlt{}2021 2025\PYGZgt{}  a =growth 1 2 3 4 5 }
\PYG{l+s+s1}{\PYGZlt{}2021 2023\PYGZgt{}  a = 120  \PYGZhy{}\PYGZhy{}kg}
\PYG{l+s+s1}{\PYGZsq{}\PYGZsq{}\PYGZsq{}}\PYG{p}{,}\PYG{n}{lprint}\PYG{o}{=}\PYG{l+m+mi}{0}\PYG{p}{)}


\PYG{n}{kg}\PYG{o}{=}\PYG{n}{kg}\PYG{o}{.}\PYG{n}{rename}\PYG{p}{(}\PYG{n}{columns}\PYG{o}{=}\PYG{p}{\PYGZob{}}\PYG{l+s+s2}{\PYGZdq{}}\PYG{l+s+s2}{A}\PYG{l+s+s2}{\PYGZdq{}}\PYG{p}{:}\PYG{l+s+s2}{\PYGZdq{}}\PYG{l+s+s2}{KG}\PYG{l+s+s2}{\PYGZdq{}}\PYG{p}{\PYGZcb{}}\PYG{p}{)}       \PYG{c+c1}{\PYGZsh{}rename cols to facilitate display}
\PYG{n}{nokg}\PYG{o}{=}\PYG{n}{nokg}\PYG{o}{.}\PYG{n}{rename}\PYG{p}{(}\PYG{n}{columns}\PYG{o}{=}\PYG{p}{\PYGZob{}}\PYG{l+s+s2}{\PYGZdq{}}\PYG{l+s+s2}{A}\PYG{l+s+s2}{\PYGZdq{}}\PYG{p}{:}\PYG{l+s+s2}{\PYGZdq{}}\PYG{l+s+s2}{NOKG}\PYG{l+s+s2}{\PYGZdq{}}\PYG{p}{\PYGZcb{}}\PYG{p}{)} \PYG{c+c1}{\PYGZsh{}rename cols to facilitate display}
\PYG{n}{df}\PYG{o}{=}\PYG{n}{original}\PYG{o}{.}\PYG{n}{rename}\PYG{p}{(}\PYG{n}{columns}\PYG{o}{=}\PYG{p}{\PYGZob{}}\PYG{l+s+s2}{\PYGZdq{}}\PYG{l+s+s2}{A}\PYG{l+s+s2}{\PYGZdq{}}\PYG{p}{:}\PYG{l+s+s2}{\PYGZdq{}}\PYG{l+s+s2}{Orig}\PYG{l+s+s2}{\PYGZdq{}}\PYG{p}{\PYGZcb{}}\PYG{p}{)} \PYG{c+c1}{\PYGZsh{}rename cols to facilitate display}

\PYG{n}{combo}\PYG{o}{=}\PYG{n}{pd}\PYG{o}{.}\PYG{n}{concat}\PYG{p}{(}\PYG{p}{[}\PYG{n}{kg}\PYG{p}{,}\PYG{n}{nokg}\PYG{p}{,}\PYG{n}{df}\PYG{p}{]}\PYG{p}{,} \PYG{n}{axis}\PYG{o}{=}\PYG{l+m+mi}{1}\PYG{p}{)}
\PYG{n}{combo}


\PYG{n+nb}{print}\PYG{p}{(}\PYG{l+s+sa}{f}\PYG{l+s+s1}{\PYGZsq{}}\PYG{l+s+s1}{Levels}\PYG{l+s+se}{\PYGZbs{}n}\PYG{l+s+si}{\PYGZob{}}\PYG{n}{combo}\PYG{l+s+si}{\PYGZcb{}}\PYG{l+s+se}{\PYGZbs{}n}\PYG{l+s+se}{\PYGZbs{}n}\PYG{l+s+s1}{Growth}\PYG{l+s+se}{\PYGZbs{}n}\PYG{l+s+si}{\PYGZob{}}\PYG{n}{combo}\PYG{o}{.}\PYG{n}{pct\PYGZus{}change}\PYG{p}{(}\PYG{p}{)}\PYG{o}{*}\PYG{l+m+mi}{100}\PYG{l+s+si}{\PYGZcb{}}\PYG{l+s+s1}{\PYGZsq{}}\PYG{p}{)}
\end{sphinxVerbatim}

\end{sphinxuseclass}\end{sphinxVerbatimInput}
\begin{sphinxVerbatimOutput}

\begin{sphinxuseclass}{cell_output}
\begin{sphinxVerbatim}[commandchars=\\\{\}]
Levels
          KG        NOKG        Orig
2020  100.00  100.000000  100.000000
2021  120.00  120.000000  101.000000
2022  120.00  120.000000  103.020000
2023  120.00  120.000000  106.110600
2024  124.80  110.355024  110.355024
2025  131.04  115.872775  115.872775

Growth
        KG      NOKG  Orig
2020   NaN       NaN   NaN
2021  20.0  20.00000   1.0
2022   0.0   0.00000   2.0
2023   0.0   0.00000   3.0
2024   4.0  \PYGZhy{}8.03748   4.0
2025   5.0   5.00000   5.0
\end{sphinxVerbatim}

\end{sphinxuseclass}\end{sphinxVerbatimOutput}

\end{sphinxuseclass}
\sphinxAtStartPar
\sphinxstylestrong{Understanding the results}

\sphinxAtStartPar
In the first example where KG (keep\_growth) \sphinxstylestrong{was set}, the level was set constant for three periods at 120 the rate of growth was 0 for the final two years of the set period.  But following this update, the level of A in 2023 is 120. With \sphinxcode{\sphinxupquote{keep\_Growth=True}} the KG variable growth at 2 percent per year in 2024 and 2025.

\sphinxAtStartPar
In the \sphinxstylestrong{–nkg} example, the levels of NOKG are the same as KG for 2020 through 2023, but because \sphinxcode{\sphinxupquote{\sphinxhyphen{}\sphinxhyphen{}nkg}} was selected the levels revert to their pre\sphinxhyphen{}shock values, which are lower than the 120 in 2023.  As a result the growth rate for NOKG is negative in 2024. The growth rate for 2024 remains 5 because neither the 2024 or 2025 data changed and therefore the 2025 the growth rate does not change.


\subsubsection{.upd() with the option keep\_growth set globally}
\label{\detokenize{content/04_PythonEssentials/UpdateCommand:upd-with-the-option-keep-growth-set-globally}}
\sphinxAtStartPar
Above the line level option \sphinxcode{\sphinxupquote{\sphinxhyphen{}\sphinxhyphen{}keep\_growth}} or \sphinxcode{\sphinxupquote{\sphinxhyphen{}\sphinxhyphen{}kg}} was used to keep the growth rate (or not) for a given operation.

\sphinxAtStartPar
This works because by default the option \sphinxcode{\sphinxupquote{Keep\_growth}} is set to false, implementing \sphinxcode{\sphinxupquote{\sphinxhyphen{}\sphinxhyphen{}kg}} at the line level temporarily set the keep\_growth flag to  true for the specific line (and those following).

\sphinxAtStartPar
The \sphinxcode{\sphinxupquote{keep\_growth}} flag can also be set globally for all the lines by setting the option in the command line.

\sphinxAtStartPar
\sphinxcode{\sphinxupquote{keep\_growth=True}}.

\sphinxAtStartPar
Now as default, all lines will keep the growth rate (unless overridden at the line level with \sphinxcode{\sphinxupquote{\sphinxhyphen{}\sphinxhyphen{}nkg}} or \sphinxcode{\sphinxupquote{\sphinxhyphen{}\sphinxhyphen{}no\_keep\_growth}}.
\begin{itemize}
\item {} 
\sphinxAtStartPar
c,d are updated in 2022 and 2023 and keep the growth rates afterwards

\item {} 
\sphinxAtStartPar
e the \sphinxcode{\sphinxupquote{\sphinxhyphen{}\sphinxhyphen{}no\_keep\_growth}} in this line prevents the updating 2024\sphinxhyphen{}2025

\end{itemize}

\begin{sphinxuseclass}{cell}\begin{sphinxVerbatimInput}

\begin{sphinxuseclass}{cell_input}
\begin{sphinxVerbatim}[commandchars=\\\{\}]
\PYG{c+c1}{\PYGZsh{} Create a data frame}
\PYG{n}{dftest} \PYG{o}{=} \PYG{n}{pd}\PYG{o}{.}\PYG{n}{DataFrame}\PYG{p}{(}\PYG{l+m+mi}{100}\PYG{p}{,}
       \PYG{n}{index}\PYG{o}{=}\PYG{p}{[}\PYG{n}{v} \PYG{k}{for} \PYG{n}{v} \PYG{o+ow}{in} \PYG{n+nb}{range}\PYG{p}{(}\PYG{l+m+mi}{2020}\PYG{p}{,}\PYG{l+m+mi}{2025}\PYG{p}{)}\PYG{p}{]}\PYG{p}{,} \PYG{c+c1}{\PYGZsh{} create row index}
       \PYG{c+c1}{\PYGZsh{} equivalent to index=[2020,2021,2022,2023,2024,2025] }
       \PYG{n}{columns}\PYG{o}{=}\PYG{p}{[}\PYG{l+s+s1}{\PYGZsq{}}\PYG{l+s+s1}{A}\PYG{l+s+s1}{\PYGZsq{}}\PYG{p}{,}\PYG{l+s+s1}{\PYGZsq{}}\PYG{l+s+s1}{B}\PYG{l+s+s1}{\PYGZsq{}}\PYG{p}{,}\PYG{l+s+s1}{\PYGZsq{}}\PYG{l+s+s1}{C}\PYG{l+s+s1}{\PYGZsq{}}\PYG{p}{,}\PYG{l+s+s1}{\PYGZsq{}}\PYG{l+s+s1}{D}\PYG{l+s+s1}{\PYGZsq{}}\PYG{p}{,}\PYG{l+s+s1}{\PYGZsq{}}\PYG{l+s+s1}{E}\PYG{l+s+s1}{\PYGZsq{}}\PYG{p}{]}\PYG{p}{)}                                 \PYG{c+c1}{\PYGZsh{} create column name }
\PYG{n}{dftest}
\end{sphinxVerbatim}

\end{sphinxuseclass}\end{sphinxVerbatimInput}
\begin{sphinxVerbatimOutput}

\begin{sphinxuseclass}{cell_output}
\begin{sphinxVerbatim}[commandchars=\\\{\}]
        A    B    C    D    E
2020  100  100  100  100  100
2021  100  100  100  100  100
2022  100  100  100  100  100
2023  100  100  100  100  100
2024  100  100  100  100  100
\end{sphinxVerbatim}

\end{sphinxuseclass}\end{sphinxVerbatimOutput}

\end{sphinxuseclass}
\begin{sphinxuseclass}{cell}\begin{sphinxVerbatimInput}

\begin{sphinxuseclass}{cell_input}
\begin{sphinxVerbatim}[commandchars=\\\{\}]
\PYG{n}{dfres} \PYG{o}{=} \PYG{n}{dftest}\PYG{o}{.}\PYG{n}{upd}\PYG{p}{(}\PYG{l+s+s1}{\PYGZsq{}\PYGZsq{}\PYGZsq{}}
\PYG{l+s+s1}{\PYGZlt{}2022 2023\PYGZgt{} c = 200 }
\PYG{l+s+s1}{\PYGZlt{}2022 2023\PYGZgt{} d = 300  }
\PYG{l+s+s1}{\PYGZlt{}2022 2023\PYGZgt{} e = 400  \PYGZhy{}\PYGZhy{}no\PYGZus{}keep\PYGZus{}growth }
\PYG{l+s+s1}{\PYGZsq{}\PYGZsq{}\PYGZsq{}}\PYG{p}{,}\PYG{n}{keep\PYGZus{}growth}\PYG{o}{=}\PYG{k+kc}{True}\PYG{p}{)}  \PYG{c+c1}{\PYGZsh{} \PYGZlt{}=  Set keep\PYGZus{}growth to True for the entirety of the command, }
                       \PYG{c+c1}{\PYGZsh{} except for e where it is overridden by the \PYGZhy{}\PYGZhy{}no\PYGZus{}keep\PYGZus{}growth flag}
\PYG{n+nb}{print}\PYG{p}{(}\PYG{l+s+sa}{f}\PYG{l+s+s1}{\PYGZsq{}}\PYG{l+s+s1}{Dataframe:}\PYG{l+s+se}{\PYGZbs{}n}\PYG{l+s+si}{\PYGZob{}}\PYG{n}{dfres}\PYG{l+s+si}{\PYGZcb{}}\PYG{l+s+se}{\PYGZbs{}n}\PYG{l+s+se}{\PYGZbs{}n}\PYG{l+s+s1}{Growth:}\PYG{l+s+se}{\PYGZbs{}n}\PYG{l+s+si}{\PYGZob{}}\PYG{n}{dfres}\PYG{o}{.}\PYG{n}{pct\PYGZus{}change}\PYG{p}{(}\PYG{p}{)}\PYG{o}{*}\PYG{l+m+mi}{100}\PYG{l+s+si}{\PYGZcb{}}\PYG{l+s+se}{\PYGZbs{}n}\PYG{l+s+s1}{\PYGZsq{}}\PYG{p}{)}
\end{sphinxVerbatim}

\end{sphinxuseclass}\end{sphinxVerbatimInput}
\begin{sphinxVerbatimOutput}

\begin{sphinxuseclass}{cell_output}
\begin{sphinxVerbatim}[commandchars=\\\{\}]
Dataframe:
        A    B      C      D    E
2020  100  100  100.0  100.0  100
2021  100  100  100.0  100.0  100
2022  100  100  200.0  300.0  400
2023  100  100  200.0  300.0  400
2024  100  100  200.0  300.0  100

Growth:
        A    B      C      D      E
2020  NaN  NaN    NaN    NaN    NaN
2021  0.0  0.0    0.0    0.0    0.0
2022  0.0  0.0  100.0  200.0  300.0
2023  0.0  0.0    0.0    0.0    0.0
2024  0.0  0.0    0.0    0.0  \PYGZhy{}75.0
\end{sphinxVerbatim}

\end{sphinxuseclass}\end{sphinxVerbatimOutput}

\end{sphinxuseclass}

\subsubsection{.upd(,,scale=<number, default=1>) Scale the updates}
\label{\detokenize{content/04_PythonEssentials/UpdateCommand:upd-scale-number-default-1-scale-the-updates}}
\sphinxAtStartPar
When running scenarios it can be useful to sensitivity analyses of model results, to better understand how the model responds when varying the intensity of a shock.

\sphinxAtStartPar
The scale option provides a mechanism for calculating a range shocks as a proportion of the initially indicated one.

\sphinxAtStartPar
When using the scale option, scale=0 implies no change (effectively the baseline) while scale=0.5 is a scenario half of the full severity.

\sphinxAtStartPar
In the example below \sphinxcode{\sphinxupquote{.upd()}} is executed three times for severity equals 0. 0.5 and 1.  If the list passed to scale (named severity in this case) had five items in it, the update would be run five times – one time for each item in the list.

\sphinxAtStartPar
This example just prints outputs, a more interesting example would involve the solving a model using different levels of a given shock.

\begin{sphinxuseclass}{cell}\begin{sphinxVerbatimInput}

\begin{sphinxuseclass}{cell_input}
\begin{sphinxVerbatim}[commandchars=\\\{\}]
\PYG{n+nb}{print}\PYG{p}{(}\PYG{l+s+sa}{f}\PYG{l+s+s1}{\PYGZsq{}}\PYG{l+s+s1}{input dataframe: }\PYG{l+s+se}{\PYGZbs{}n}\PYG{l+s+si}{\PYGZob{}}\PYG{n}{df}\PYG{l+s+si}{\PYGZcb{}}\PYG{l+s+se}{\PYGZbs{}n}\PYG{l+s+se}{\PYGZbs{}n}\PYG{l+s+s1}{\PYGZsq{}}\PYG{p}{)}
\PYG{k}{for} \PYG{n}{severity} \PYG{o+ow}{in} \PYG{p}{[}\PYG{l+m+mi}{0}\PYG{p}{,}\PYG{l+m+mf}{0.5}\PYG{p}{,}\PYG{l+m+mi}{1}\PYG{p}{]}\PYG{p}{:} 
    \PYG{c+c1}{\PYGZsh{} First make a dataframe with some growth rate }
    \PYG{n}{res} \PYG{o}{=} \PYG{n}{df}\PYG{o}{.}\PYG{n}{upd}\PYG{p}{(}\PYG{l+s+s1}{\PYGZsq{}\PYGZsq{}\PYGZsq{}}
\PYG{l+s+s1}{    \PYGZlt{}2021 2025\PYGZgt{}}
\PYG{l+s+s1}{    a =growth 1 2 3 4 5 }
\PYG{l+s+s1}{    b + 10}
\PYG{l+s+s1}{    }\PYG{l+s+s1}{\PYGZsq{}\PYGZsq{}\PYGZsq{}}\PYG{p}{,}\PYG{n}{scale}\PYG{o}{=}\PYG{n}{severity}\PYG{p}{)}
    \PYG{n+nb}{print}\PYG{p}{(}\PYG{l+s+sa}{f}\PYG{l+s+s1}{\PYGZsq{}}\PYG{l+s+si}{\PYGZob{}}\PYG{n}{severity}\PYG{l+s+si}{=\PYGZcb{}}\PYG{l+s+se}{\PYGZbs{}n}\PYG{l+s+s1}{Dataframe:}\PYG{l+s+se}{\PYGZbs{}n}\PYG{l+s+si}{\PYGZob{}}\PYG{n}{res}\PYG{l+s+si}{\PYGZcb{}}\PYG{l+s+se}{\PYGZbs{}n}\PYG{l+s+se}{\PYGZbs{}n}\PYG{l+s+s1}{Growth:}\PYG{l+s+se}{\PYGZbs{}n}\PYG{l+s+si}{\PYGZob{}}\PYG{n}{res}\PYG{o}{.}\PYG{n}{pct\PYGZus{}change}\PYG{p}{(}\PYG{p}{)}\PYG{o}{*}\PYG{l+m+mi}{100}\PYG{l+s+si}{\PYGZcb{}}\PYG{l+s+se}{\PYGZbs{}n}\PYG{l+s+se}{\PYGZbs{}n}\PYG{l+s+s1}{\PYGZsq{}}\PYG{p}{)}
    \PYG{c+c1}{\PYGZsh{}  }
    \PYG{c+c1}{\PYGZsh{} Here the updated dataframe is only printed. }
    \PYG{c+c1}{\PYGZsh{} A more realistic use case is to simulate a model like this: }
    \PYG{c+c1}{\PYGZsh{} dummy\PYGZus{} = mpak(res,keep=\PYGZsq{}Severity \PYGZob{}serverity\PYGZcb{}\PYGZsq{})    \PYGZsh{} more realistic }
\end{sphinxVerbatim}

\end{sphinxuseclass}\end{sphinxVerbatimInput}
\begin{sphinxVerbatimOutput}

\begin{sphinxuseclass}{cell_output}
\begin{sphinxVerbatim}[commandchars=\\\{\}]
input dataframe: 
            Orig
2020  100.000000
2021  101.000000
2022  103.020000
2023  106.110600
2024  110.355024
2025  115.872775


severity=0
Dataframe:
            Orig    A    B
2020  100.000000  0.0  0.0
2021  101.000000  0.0  0.0
2022  103.020000  0.0  0.0
2023  106.110600  0.0  0.0
2024  110.355024  0.0  0.0
2025  115.872775  0.0  0.0

Growth:
      Orig   A   B
2020   NaN NaN NaN
2021   1.0 NaN NaN
2022   2.0 NaN NaN
2023   3.0 NaN NaN
2024   4.0 NaN NaN
2025   5.0 NaN NaN


severity=0.5
Dataframe:
            Orig    A    B
2020  100.000000  0.0  0.0
2021  101.000000  0.0  5.0
2022  103.020000  0.0  5.0
2023  106.110600  0.0  5.0
2024  110.355024  0.0  5.0
2025  115.872775  0.0  5.0

Growth:
      Orig   A    B
2020   NaN NaN  NaN
2021   1.0 NaN  inf
2022   2.0 NaN  0.0
2023   3.0 NaN  0.0
2024   4.0 NaN  0.0
2025   5.0 NaN  0.0


severity=1
Dataframe:
            Orig    A     B
2020  100.000000  0.0   0.0
2021  101.000000  0.0  10.0
2022  103.020000  0.0  10.0
2023  106.110600  0.0  10.0
2024  110.355024  0.0  10.0
2025  115.872775  0.0  10.0

Growth:
      Orig   A    B
2020   NaN NaN  NaN
2021   1.0 NaN  inf
2022   2.0 NaN  0.0
2023   3.0 NaN  0.0
2024   4.0 NaN  0.0
2025   5.0 NaN  0.0
\end{sphinxVerbatim}

\end{sphinxuseclass}\end{sphinxVerbatimOutput}

\end{sphinxuseclass}

\subsubsection{.upd(,,lprint=True ) prints values the before and after update}
\label{\detokenize{content/04_PythonEssentials/UpdateCommand:upd-lprint-true-prints-values-the-before-and-after-update}}
\sphinxAtStartPar
The \sphinxcode{\sphinxupquote{lPrint}} option of the method \sphinxcode{\sphinxupquote{upd()}} is set to \sphinxcode{\sphinxupquote{= False}} by default.  By setting it true, an update command will output the results of the calculation comparing the values of the dataframe (over the impacted period) before, after and the difference between the two.

\begin{sphinxuseclass}{cell}\begin{sphinxVerbatimInput}

\begin{sphinxuseclass}{cell_input}
\begin{sphinxVerbatim}[commandchars=\\\{\}]
\PYG{n}{df}\PYG{o}{.}\PYG{n}{upd}\PYG{p}{(}\PYG{l+s+s1}{\PYGZsq{}\PYGZsq{}\PYGZsq{}}
\PYG{l+s+s1}{\PYGZsh{} Same number of values as years}
\PYG{l+s+s1}{\PYGZlt{}2021 2022\PYGZgt{} A *  42 44}
\PYG{l+s+s1}{\PYGZsq{}\PYGZsq{}\PYGZsq{}}\PYG{p}{,}\PYG{n}{lprint}\PYG{o}{=}\PYG{l+m+mi}{1}\PYG{p}{)}\PYG{p}{;}
\end{sphinxVerbatim}

\end{sphinxuseclass}\end{sphinxVerbatimInput}
\begin{sphinxVerbatimOutput}

\begin{sphinxuseclass}{cell_output}
\begin{sphinxVerbatim}[commandchars=\\\{\}]
Update * [42.0, 44.0] 2021 2022
A                    Before                After                 Diff
2021                 0.0000               0.0000               0.0000
2022                 0.0000               0.0000               0.0000
\end{sphinxVerbatim}

\end{sphinxuseclass}\end{sphinxVerbatimOutput}

\end{sphinxuseclass}

\subsubsection{.upd(,,create=True ) Requires the variable to exist}
\label{\detokenize{content/04_PythonEssentials/UpdateCommand:upd-create-true-requires-the-variable-to-exist}}
\sphinxAtStartPar
Until now .upd has created variables if they did not exist in the input dataframe.

\sphinxAtStartPar
To catch misspellings the parameter \sphinxcode{\sphinxupquote{create}} can be set to False.
New variables will not be created, and an exception will be raised.

\sphinxAtStartPar
Here Pythons exception handling is used, so the notebook will continue to run the cells below.

\begin{sphinxuseclass}{cell}\begin{sphinxVerbatimInput}

\begin{sphinxuseclass}{cell_input}
\begin{sphinxVerbatim}[commandchars=\\\{\}]
\PYG{k}{try}\PYG{p}{:}
    \PYG{n}{xx} \PYG{o}{=} \PYG{n}{df}\PYG{o}{.}\PYG{n}{upd}\PYG{p}{(}\PYG{l+s+s1}{\PYGZsq{}\PYGZsq{}\PYGZsq{}}
\PYG{l+s+s1}{    \PYGZsh{} Same number of values as years}
\PYG{l+s+s1}{    \PYGZlt{}2021 2022\PYGZgt{} Aa *  42 44}
\PYG{l+s+s1}{    }\PYG{l+s+s1}{\PYGZsq{}\PYGZsq{}\PYGZsq{}}\PYG{p}{,}\PYG{n}{create}\PYG{o}{=}\PYG{k+kc}{False}\PYG{p}{)}
    \PYG{n+nb}{print}\PYG{p}{(}\PYG{n}{xx}\PYG{p}{)}
\PYG{k}{except} \PYG{n+ne}{Exception} \PYG{k}{as} \PYG{n}{inst}\PYG{p}{:}
    \PYG{n}{xx} \PYG{o}{=} \PYG{k+kc}{None}
    \PYG{n+nb}{print}\PYG{p}{(}\PYG{n}{inst}\PYG{p}{)} 
\end{sphinxVerbatim}

\end{sphinxuseclass}\end{sphinxVerbatimInput}
\begin{sphinxVerbatimOutput}

\begin{sphinxuseclass}{cell_output}
\begin{sphinxVerbatim}[commandchars=\\\{\}]
Variable to update not found:AA, timespan = [2021 2022] 
Set create=True if you want the variable created: 
\end{sphinxVerbatim}

\end{sphinxuseclass}\end{sphinxVerbatimOutput}

\end{sphinxuseclass}
\sphinxstepscope


\section{\sphinxstyleliteralintitle{\sphinxupquote{.mfcalc()}} an extension of standard Pandas}
\label{\detokenize{content/04_PythonEssentials/mfcalc:mfcalc-an-extension-of-standard-pandas}}\label{\detokenize{content/04_PythonEssentials/mfcalc::doc}}
\sphinxAtStartPar
Like\sphinxcode{\sphinxupquote{.upd()}}, the \sphinxcode{\sphinxupquote{.mfcalc()}} method of \sphinxcode{\sphinxupquote{modelflow}} extends the functionality of standard pandas.  It is actually a much more powerful method that can be used to solve models or mini\sphinxhyphen{}models or see how modelflow normalizes equations.  It can be particularly useful when creating scenarios – uses that are presented elsewhere.

\sphinxAtStartPar
Here, the focus is on using \sphinxcode{\sphinxupquote{mfcalc()}}to perform quick and dirty calculations and modify datafames.


\subsection{Workspace initialization}
\label{\detokenize{content/04_PythonEssentials/mfcalc:workspace-initialization}}
\sphinxAtStartPar
Setting up our python session to use pandas and modelflow by importing their packages.  \sphinxcode{\sphinxupquote{modelmf}} is an extension of dataframes that is part of the modelflow installation package (and also used by modelflow itself).

\begin{sphinxuseclass}{cell}\begin{sphinxVerbatimInput}

\begin{sphinxuseclass}{cell_input}
\begin{sphinxVerbatim}[commandchars=\\\{\}]
\PYG{k+kn}{import} \PYG{n+nn}{pandas} \PYG{k}{as} \PYG{n+nn}{pd}  \PYG{c+c1}{\PYGZsh{} Python data science library}
\PYG{k+kn}{import} \PYG{n+nn}{modelmf}       \PYG{c+c1}{\PYGZsh{} Add useful features to pandas dataframes }
                     \PYG{c+c1}{\PYGZsh{} using utlities initially developed for modelflow}
\end{sphinxVerbatim}

\end{sphinxuseclass}\end{sphinxVerbatimInput}

\end{sphinxuseclass}
\sphinxAtStartPar
\sphinxstylestrong{Create a  simple dataframe}

\sphinxAtStartPar
Create a Pandas dataframe with one column with the name A and 6 rows.

\sphinxAtStartPar
Set set the index to 2020 through 2026 and set the values of all the cells to 100.
\begin{itemize}
\item {} 
\sphinxAtStartPar
\sphinxcode{\sphinxupquote{pd.DataFrame}} creates a dataframe  \sphinxhref{https://pandas.pydata.org/docs/reference/api/pandas.DataFrame.html\#pandas.DataFrame}{Description}

\item {} 
\sphinxAtStartPar
The expression \sphinxcode{\sphinxupquote{{[}v for v in range(2020,2026){]}}} dynamically creates a  python list, and fills it with  integers beginning with 2020 and ending 2025

\end{itemize}

\begin{sphinxuseclass}{cell}\begin{sphinxVerbatimInput}

\begin{sphinxuseclass}{cell_input}
\begin{sphinxVerbatim}[commandchars=\\\{\}]
\PYG{n}{df} \PYG{o}{=} \PYG{n}{pd}\PYG{o}{.}\PYG{n}{DataFrame}\PYG{p}{(}                                 \PYG{c+c1}{\PYGZsh{} call the dataframe constructure }
    \PYG{l+m+mf}{100.000}\PYG{p}{,}                                           \PYG{c+c1}{\PYGZsh{} the values }
    \PYG{n}{index}\PYG{o}{=}\PYG{p}{[}\PYG{n}{v} \PYG{k}{for} \PYG{n}{v} \PYG{o+ow}{in} \PYG{n+nb}{range}\PYG{p}{(}\PYG{l+m+mi}{2020}\PYG{p}{,}\PYG{l+m+mi}{2026}\PYG{p}{)}\PYG{p}{]}\PYG{p}{,}           \PYG{c+c1}{\PYGZsh{}index}
    \PYG{n}{columns}\PYG{o}{=}\PYG{p}{[}\PYG{l+s+s1}{\PYGZsq{}}\PYG{l+s+s1}{A}\PYG{l+s+s1}{\PYGZsq{}}\PYG{p}{]}                                  \PYG{c+c1}{\PYGZsh{} the column name }
                 \PYG{p}{)}
\PYG{n}{df}   \PYG{c+c1}{\PYGZsh{} the result of the last statement is displayed in the output cell }
\end{sphinxVerbatim}

\end{sphinxuseclass}\end{sphinxVerbatimInput}
\begin{sphinxVerbatimOutput}

\begin{sphinxuseclass}{cell_output}
\begin{sphinxVerbatim}[commandchars=\\\{\}]
          A
2020  100.0
2021  100.0
2022  100.0
2023  100.0
2024  100.0
2025  100.0
\end{sphinxVerbatim}

\end{sphinxuseclass}\end{sphinxVerbatimOutput}

\end{sphinxuseclass}

\subsection{Create a new series from an existing series}
\label{\detokenize{content/04_PythonEssentials/mfcalc:create-a-new-series-from-an-existing-series}}
\sphinxAtStartPar
Use  mfcalc to calculate a new column (series) as a function of the existing A column series

\sphinxAtStartPar
The below call creates a new column x.

\begin{sphinxuseclass}{cell}\begin{sphinxVerbatimInput}

\begin{sphinxuseclass}{cell_input}
\begin{sphinxVerbatim}[commandchars=\\\{\}]
\PYG{n}{df}\PYG{o}{.}\PYG{n}{mfcalc}\PYG{p}{(}\PYG{l+s+s1}{\PYGZsq{}}\PYG{l+s+s1}{x = x(\PYGZhy{}1) + a}\PYG{l+s+s1}{\PYGZsq{}}\PYG{p}{)}
\end{sphinxVerbatim}

\end{sphinxuseclass}\end{sphinxVerbatimInput}
\begin{sphinxVerbatimOutput}

\begin{sphinxuseclass}{cell_output}
\begin{sphinxVerbatim}[commandchars=\\\{\}]
* Take care. Lags or leads in the equations, mfcalc run for 2021 to 2022
\end{sphinxVerbatim}

\begin{sphinxVerbatim}[commandchars=\\\{\}]
          A      X
2020  100.0    0.0
2021  100.0  100.0
2022  100.0  200.0
2023  100.0  300.0
2024  100.0  400.0
2025  100.0  500.0
\end{sphinxVerbatim}

\end{sphinxuseclass}\end{sphinxVerbatimOutput}

\end{sphinxuseclass}
\begin{sphinxadmonition}{warning}{Warning:}
\sphinxAtStartPar
By default \sphinxcode{\sphinxupquote{.mfcalc}} will initialize a new variable with zeroes.

\sphinxAtStartPar
Moreover, if a formula passed to \sphinxcode{\sphinxupquote{.mfcalc}} contains a lag a value will be calculated for the a row only if there is data in the series for the preceding row.

\sphinxAtStartPar
These two behaviors affects how calculations generated with \sphinxcode{\sphinxupquote{.mfcalc}} are executed and can generate results that may sometimes by unexpected.
\end{sphinxadmonition}

\sphinxAtStartPar
The initialization of new variables with zero and the treatment of lags combined means that when the command \sphinxcode{\sphinxupquote{df.mfcalc('x = x(\sphinxhyphen{}1) + a')}} is executed, the value for X in 2020 will be zero (not n/a). This results because there was no X variable defined for 2019 (no such row exists). \sphinxcode{\sphinxupquote{modelflow}} first initializes all values of X with zero.  It then goes to calculate X in 2020.  There is no X value for 2019 so it skips ahead to 2021 and calculates X as equal to 0 (the value of x in 2020) + the value for a in 2021 – etc.

\sphinxAtStartPar
As with \sphinxcode{\sphinxupquote{.upd()}} unless we assign the result of \sphinxcode{\sphinxupquote{.mfcalc()}} to a variable the resulting dataframe is lost.  The above did not change \sphinxcode{\sphinxupquote{df}}.

\begin{sphinxuseclass}{cell}\begin{sphinxVerbatimInput}

\begin{sphinxuseclass}{cell_input}
\begin{sphinxVerbatim}[commandchars=\\\{\}]
\PYG{n}{df}
\end{sphinxVerbatim}

\end{sphinxuseclass}\end{sphinxVerbatimInput}
\begin{sphinxVerbatimOutput}

\begin{sphinxuseclass}{cell_output}
\begin{sphinxVerbatim}[commandchars=\\\{\}]
          A
2020  100.0
2021  100.0
2022  100.0
2023  100.0
2024  100.0
2025  100.0
\end{sphinxVerbatim}

\end{sphinxuseclass}\end{sphinxVerbatimOutput}

\end{sphinxuseclass}

\subsection{Storing the result of an \sphinxstyleliteralintitle{\sphinxupquote{.mfcalc()}} call}
\label{\detokenize{content/04_PythonEssentials/mfcalc:storing-the-result-of-an-mfcalc-call}}
\sphinxAtStartPar
Above the results of the \sphinxcode{\sphinxupquote{.mfcalc()}} operation was not assigned to an object – the \sphinxcode{\sphinxupquote{DataFrame}} object df itself was not changed.

\sphinxAtStartPar
Below the results of the same operation are assigned to the variable df2 and therefore stored.

\begin{sphinxuseclass}{cell}\begin{sphinxVerbatimInput}

\begin{sphinxuseclass}{cell_input}
\begin{sphinxVerbatim}[commandchars=\\\{\}]
\PYG{n}{df2}\PYG{o}{=}\PYG{n}{df}\PYG{o}{.}\PYG{n}{mfcalc}\PYG{p}{(}\PYG{l+s+s1}{\PYGZsq{}}\PYG{l+s+s1}{x = x(\PYGZhy{}1) + a}\PYG{l+s+s1}{\PYGZsq{}}\PYG{p}{)} \PYG{c+c1}{\PYGZsh{} Assign the result to df2}
\PYG{n}{df2}
\end{sphinxVerbatim}

\end{sphinxuseclass}\end{sphinxVerbatimInput}
\begin{sphinxVerbatimOutput}

\begin{sphinxuseclass}{cell_output}
\begin{sphinxVerbatim}[commandchars=\\\{\}]
* Take care. Lags or leads in the equations, mfcalc run for 2021 to 2022
\end{sphinxVerbatim}

\begin{sphinxVerbatim}[commandchars=\\\{\}]
          A      X
2020  100.0    0.0
2021  100.0  100.0
2022  100.0  200.0
2023  100.0  300.0
2024  100.0  400.0
2025  100.0  500.0
\end{sphinxVerbatim}

\end{sphinxuseclass}\end{sphinxVerbatimOutput}

\end{sphinxuseclass}

\subsection{Recalculate A so  it grows by 2 percent}
\label{\detokenize{content/04_PythonEssentials/mfcalc:recalculate-a-so-it-grows-by-2-percent}}
\sphinxAtStartPar
\sphinxcode{\sphinxupquote{mfcalc()}} understand lagged variables and can do recursive calculations.

\begin{sphinxuseclass}{cell}\begin{sphinxVerbatimInput}

\begin{sphinxuseclass}{cell_input}
\begin{sphinxVerbatim}[commandchars=\\\{\}]
\PYG{n}{res} \PYG{o}{=} \PYG{n}{df}\PYG{o}{.}\PYG{n}{mfcalc}\PYG{p}{(}\PYG{l+s+s1}{\PYGZsq{}}\PYG{l+s+s1}{a =  1.02 *  a(\PYGZhy{}1)}\PYG{l+s+s1}{\PYGZsq{}}\PYG{p}{)}
\PYG{n}{res}
\end{sphinxVerbatim}

\end{sphinxuseclass}\end{sphinxVerbatimInput}
\begin{sphinxVerbatimOutput}

\begin{sphinxuseclass}{cell_output}
\begin{sphinxVerbatim}[commandchars=\\\{\}]
* Take care. Lags or leads in the equations, mfcalc run for 2021 to 2022
\end{sphinxVerbatim}

\begin{sphinxVerbatim}[commandchars=\\\{\}]
               A
2020  100.000000
2021  102.000000
2022  104.040000
2023  106.120800
2024  108.243216
2025  110.408080
\end{sphinxVerbatim}

\end{sphinxuseclass}\end{sphinxVerbatimOutput}

\end{sphinxuseclass}
\begin{sphinxuseclass}{cell}\begin{sphinxVerbatimInput}

\begin{sphinxuseclass}{cell_input}
\begin{sphinxVerbatim}[commandchars=\\\{\}]
\PYG{n}{res}\PYG{o}{.}\PYG{n}{pct\PYGZus{}change}\PYG{p}{(}\PYG{p}{)}\PYG{o}{*}\PYG{l+m+mi}{100} \PYG{c+c1}{\PYGZsh{} to display the percent changes}
\end{sphinxVerbatim}

\end{sphinxuseclass}\end{sphinxVerbatimInput}
\begin{sphinxVerbatimOutput}

\begin{sphinxuseclass}{cell_output}
\begin{sphinxVerbatim}[commandchars=\\\{\}]
        A
2020  NaN
2021  2.0
2022  2.0
2023  2.0
2024  2.0
2025  2.0
\end{sphinxVerbatim}

\end{sphinxuseclass}\end{sphinxVerbatimOutput}

\end{sphinxuseclass}
\sphinxAtStartPar
In this example, \sphinxcode{\sphinxupquote{mfcalc()}}knows that it can not start to calculate in 2020, because A (the lagged variable) has no value in 2019.

\sphinxAtStartPar
\sphinxcode{\sphinxupquote{.mfcalc()}} therefore begins its calculation in 2021. Note, the existing value for 2020 is preserved.  This behaviour differs from other programs that might return a n/a value for the 2020.


\subsection{\sphinxstyleliteralintitle{\sphinxupquote{.mfcalc()}} \sphinxhyphen{} the showeq option}
\label{\detokenize{content/04_PythonEssentials/mfcalc:mfcalc-the-showeq-option}}
\sphinxAtStartPar
The \sphinxcode{\sphinxupquote{showeq}} option is by default \sphinxcode{\sphinxupquote{= False}}.

\sphinxAtStartPar
By setting equal to \sphinxcode{\sphinxupquote{True}}, mfcalc can be used to express the normalization of an entered equation.

\begin{sphinxuseclass}{cell}\begin{sphinxVerbatimInput}

\begin{sphinxuseclass}{cell_input}
\begin{sphinxVerbatim}[commandchars=\\\{\}]
\PYG{n}{df}\PYG{o}{.}\PYG{n}{mfcalc}\PYG{p}{(}\PYG{l+s+s1}{\PYGZsq{}}\PYG{l+s+s1}{dlog( a) =  0.02}\PYG{l+s+s1}{\PYGZsq{}}\PYG{p}{,}\PYG{n}{showeq}\PYG{o}{=}\PYG{l+m+mi}{1}\PYG{p}{)}\PYG{p}{;}
\end{sphinxVerbatim}

\end{sphinxuseclass}\end{sphinxVerbatimInput}
\begin{sphinxVerbatimOutput}

\begin{sphinxuseclass}{cell_output}
\begin{sphinxVerbatim}[commandchars=\\\{\}]
* Take care. Lags or leads in the equations, mfcalc run for 2021 to 2022
FRML \PYGZlt{}\PYGZgt{} A=EXP(LOG(A(\PYGZhy{}1))+0.02)\PYGZdl{}
\end{sphinxVerbatim}

\end{sphinxuseclass}\end{sphinxVerbatimOutput}

\end{sphinxuseclass}
\sphinxAtStartPar
In \sphinxcode{\sphinxupquote{modelflow}} the expression \sphinxcode{\sphinxupquote{dlog(a)}} refers to the difference in the natural logarithm \(dlog(x_t) \equiv ln(x_t)-ln(x_{t-1})\) and is equal to the growth rate for the variable.

\sphinxAtStartPar
\sphinxcode{\sphinxupquote{.mfcalc()}} normalizes the equation such that the systems solves for a as follows:
\begin{align*}
dlog(a) &= 0.02\\
log(a)-log(a_{t-1}) &= .02\\
log(a) &=log(a_{t-1})+.02\\
a &= e^{log(a_{t-1})+0.02}\\
a &=a_{t-1}*e^{0.02}\\
\end{align*}
\sphinxAtStartPar
which expressed in the business logic language of \sphinxcode{\sphinxupquote{modelflow}} is:

\sphinxAtStartPar
A=EXP(LOG(A(\sphinxhyphen{}1))+0.02)


\subsection{Using the diff() operator with mfcalc}
\label{\detokenize{content/04_PythonEssentials/mfcalc:using-the-diff-operator-with-mfcalc}}
\sphinxAtStartPar
The diff() operator, effectively normalizes to an equation that will add the value to the right of the equals sign to the lagged variable inserted in the diff operator.  Thus,  diff(a)=x normalizes to a=a(\sphinxhyphen{}1)+x

\begin{sphinxuseclass}{cell}\begin{sphinxVerbatimInput}

\begin{sphinxuseclass}{cell_input}
\begin{sphinxVerbatim}[commandchars=\\\{\}]
\PYG{n}{df}\PYG{o}{.}\PYG{n}{mfcalc}\PYG{p}{(}\PYG{l+s+s1}{\PYGZsq{}}\PYG{l+s+s1}{diff(a) =  2}\PYG{l+s+s1}{\PYGZsq{}}\PYG{p}{,}\PYG{n}{showeq}\PYG{o}{=}\PYG{l+m+mi}{1}\PYG{p}{)}
\end{sphinxVerbatim}

\end{sphinxuseclass}\end{sphinxVerbatimInput}
\begin{sphinxVerbatimOutput}

\begin{sphinxuseclass}{cell_output}
\begin{sphinxVerbatim}[commandchars=\\\{\}]
* Take care. Lags or leads in the equations, mfcalc run for 2021 to 2022
FRML \PYGZlt{}\PYGZgt{} A=A(\PYGZhy{}1)+(2)\PYGZdl{}
\end{sphinxVerbatim}

\begin{sphinxVerbatim}[commandchars=\\\{\}]
          A
2020  100.0
2021  102.0
2022  104.0
2023  106.0
2024  108.0
2025  110.0
\end{sphinxVerbatim}

\end{sphinxuseclass}\end{sphinxVerbatimOutput}

\end{sphinxuseclass}

\subsection{mfcalc with several equations and arguments}
\label{\detokenize{content/04_PythonEssentials/mfcalc:mfcalc-with-several-equations-and-arguments}}
\sphinxAtStartPar
In addition to a single equation multiple commands can be executed with one command.

\sphinxAtStartPar
However, \sphinxstylestrong{be careful} because the equation commands are executed simultaneously, which, combined with the treatments of lags, means that results may differ from what they would be if the commands were run sequentially.

\sphinxAtStartPar
For example:

\begin{sphinxuseclass}{cell}\begin{sphinxVerbatimInput}

\begin{sphinxuseclass}{cell_input}
\begin{sphinxVerbatim}[commandchars=\\\{\}]
\PYG{n}{res} \PYG{o}{=} \PYG{n}{df}\PYG{o}{.}\PYG{n}{mfcalc}\PYG{p}{(}\PYG{l+s+s1}{\PYGZsq{}\PYGZsq{}\PYGZsq{}}
\PYG{l+s+s1}{diff(a) =  2}
\PYG{l+s+s1}{x = a + 42 }
\PYG{l+s+s1}{\PYGZsq{}\PYGZsq{}\PYGZsq{}}\PYG{p}{)}

\PYG{n}{res}

\PYG{c+c1}{\PYGZsh{} use res.diff() to see the difference}
\end{sphinxVerbatim}

\end{sphinxuseclass}\end{sphinxVerbatimInput}
\begin{sphinxVerbatimOutput}

\begin{sphinxuseclass}{cell_output}
\begin{sphinxVerbatim}[commandchars=\\\{\}]
* Take care. Lags or leads in the equations, mfcalc run for 2021 to 2022
\end{sphinxVerbatim}

\begin{sphinxVerbatim}[commandchars=\\\{\}]
          A      X
2020  100.0    0.0
2021  102.0  144.0
2022  104.0  146.0
2023  106.0  148.0
2024  108.0  150.0
2025  110.0  152.0
\end{sphinxVerbatim}

\end{sphinxuseclass}\end{sphinxVerbatimOutput}

\end{sphinxuseclass}
\sphinxAtStartPar
In this example the variable \sphinxcode{\sphinxupquote{A}} in the \sphinxcode{\sphinxupquote{DataFarame}} df was initialized to 100 for the period 2020 through 2025.

\sphinxAtStartPar
The first line of the \sphinxcode{\sphinxupquote{.mfcalc()}} routine produces results only for the period 2021 \sphinxhyphen{} 2025 because there is no value for \sphinxcode{\sphinxupquote{A}} in 2019.  The value of a in 2020 is unchanged, and the following values rise by 2 in each period.

\sphinxAtStartPar
When calculating X however, \sphinxcode{\sphinxupquote{.mfcalc}} does not use the final result of the calculation of \sphinxcode{\sphinxupquote{A}}, but the intermediate result (the values for 2021 through 2025).

\sphinxAtStartPar
As a result, it is this series that is passed to the second question which adds 42 to that result.

\sphinxAtStartPar
\sphinxstylestrong{X in 2020 is not 142 as one might have expected but zero, the value to which the newly created variable defaults.}

\sphinxAtStartPar
Compare the results above with the results (below) when the same calculations are undertaken in two separate calls to \sphinxcode{\sphinxupquote{.mfcalc()}}.

\begin{sphinxuseclass}{cell}\begin{sphinxVerbatimInput}

\begin{sphinxuseclass}{cell_input}
\begin{sphinxVerbatim}[commandchars=\\\{\}]
\PYG{n}{res1} \PYG{o}{=} \PYG{n}{df}\PYG{o}{.}\PYG{n}{mfcalc}\PYG{p}{(}\PYG{l+s+s1}{\PYGZsq{}\PYGZsq{}\PYGZsq{}}
\PYG{l+s+s1}{diff(a) =  2}
\PYG{l+s+s1}{\PYGZsq{}\PYGZsq{}\PYGZsq{}}\PYG{p}{)}

\PYG{n}{res2} \PYG{o}{=} \PYG{n}{res1}\PYG{o}{.}\PYG{n}{mfcalc}\PYG{p}{(}\PYG{l+s+s1}{\PYGZsq{}\PYGZsq{}\PYGZsq{}}
\PYG{l+s+s1}{x = a + 42 }
\PYG{l+s+s1}{\PYGZsq{}\PYGZsq{}\PYGZsq{}}\PYG{p}{)}
\PYG{n}{res2}
\end{sphinxVerbatim}

\end{sphinxuseclass}\end{sphinxVerbatimInput}
\begin{sphinxVerbatimOutput}

\begin{sphinxuseclass}{cell_output}
\begin{sphinxVerbatim}[commandchars=\\\{\}]
* Take care. Lags or leads in the equations, mfcalc run for 2021 to 2022
\end{sphinxVerbatim}

\begin{sphinxVerbatim}[commandchars=\\\{\}]
          A      X
2020  100.0  142.0
2021  102.0  144.0
2022  104.0  146.0
2023  106.0  148.0
2024  108.0  150.0
2025  110.0  152.0
\end{sphinxVerbatim}

\end{sphinxuseclass}\end{sphinxVerbatimOutput}

\end{sphinxuseclass}
\begin{sphinxadmonition}{danger}{Danger:}
\sphinxAtStartPar
In \sphinxcode{\sphinxupquote{.mfcalc()}}, when there are multiple equation commands in a single call, they are executed simultaneously. This, combined with \sphinxcode{\sphinxupquote{mfcalc}}’s  treatments of lags, means only the results of the lagged calculation will be passed to other commands equations defined in the \sphinxcode{\sphinxupquote{.mfcalc}} command. As a consequence, results may differ from what would be expected and what would be seen if the two or more commands were run sequentially.
\end{sphinxadmonition}


\subsection{Setting a time frame with mfcalc.}
\label{\detokenize{content/04_PythonEssentials/mfcalc:setting-a-time-frame-with-mfcalc}}
\sphinxAtStartPar
It can useful in some circumstances to limit the time frame for which the calculations are performed. Specifying a start date and end date enclosed in <> in a line restricts the time period over which subsequent calculations are performed.

\sphinxAtStartPar
In the example below zeroes are generated for x prior to 2023 when the expressions are executed.

\begin{sphinxadmonition}{note}{Note:}
\sphinxAtStartPar
like \sphinxcode{\sphinxupquote{.upd()}} time frames set in one line are inherited by subsequent lines unless reset explicitly.
\end{sphinxadmonition}

\begin{sphinxuseclass}{cell}\begin{sphinxVerbatimInput}

\begin{sphinxuseclass}{cell_input}
\begin{sphinxVerbatim}[commandchars=\\\{\}]
\PYG{n}{res} \PYG{o}{=} \PYG{n}{df}\PYG{o}{.}\PYG{n}{mfcalc}\PYG{p}{(}\PYG{l+s+s1}{\PYGZsq{}\PYGZsq{}\PYGZsq{}}
\PYG{l+s+s1}{\PYGZlt{}2023 2025\PYGZgt{}}
\PYG{l+s+s1}{diff(a) =  2}
\PYG{l+s+s1}{x = a + 42 }
\PYG{l+s+s1}{\PYGZsq{}\PYGZsq{}\PYGZsq{}}\PYG{p}{)}

\PYG{n}{res}\PYG{o}{.}\PYG{n}{diff}\PYG{p}{(}\PYG{p}{)}

\PYG{n}{res}
\end{sphinxVerbatim}

\end{sphinxuseclass}\end{sphinxVerbatimInput}
\begin{sphinxVerbatimOutput}

\begin{sphinxuseclass}{cell_output}
\begin{sphinxVerbatim}[commandchars=\\\{\}]
          A      X
2020  100.0    0.0
2021  100.0    0.0
2022  100.0    0.0
2023  102.0  144.0
2024  104.0  146.0
2025  106.0  148.0
\end{sphinxVerbatim}

\end{sphinxuseclass}\end{sphinxVerbatimOutput}

\end{sphinxuseclass}
\sphinxstepscope


\part{Using modelflow with World Bank models}

\sphinxstepscope


\chapter{Using \sphinxstyleliteralintitle{\sphinxupquote{modelflow}} with World Bank models}
\label{\detokenize{content/05_WBModels/LoadingWBModel:using-modelflow-with-world-bank-models}}\label{\detokenize{content/05_WBModels/LoadingWBModel::doc}}
\sphinxAtStartPar
The \sphinxcode{\sphinxupquote{Modelflow}} python package has been developed to solve a wide range of models, see the modelflow \sphinxhref{https://gibhub.com}{github} web site for working examples of the Solow Model, the FR/USB model and others.

\sphinxAtStartPar
The package has been substantially expanded to include special features that enable it to work with World Bank models originally developed in EViews and designed to use the EViews Model Object for simuation.

\sphinxAtStartPar
This chapter illustrates how to access these models, how to load them into a \sphinxcode{\sphinxupquote{modelflow}} anaconda environment on your computer, and how to perform a variety of simulations.


\section{Accessing a world bank model}
\label{\detokenize{content/05_WBModels/LoadingWBModel:accessing-a-world-bank-model}}
\sphinxAtStartPar
At this time several World bank macrostructural models are available to download and use with \sphinxcode{\sphinxupquote{modelflow}}.  These include a macrostructural model for:
\begin{itemize}
\item {} 
\sphinxAtStartPar
Indonesia

\item {} 
\sphinxAtStartPar
Nepal

\item {} 
\sphinxAtStartPar
Croatia

\item {} 
\sphinxAtStartPar
Iraq

\item {} 
\sphinxAtStartPar
Kenya

\item {} 
\sphinxAtStartPar
Bolivia

\end{itemize}

\sphinxAtStartPar
Each of these models has been developed as part of the outreach work of the World Bank.  The basic modelling framework of each of these models is outlined in Burns \sphinxstyleemphasis{et al.} {[}\hyperlink{cite.content/99_BackMatter/References:id27}{2019}{]} with specific extensions reflecting features of the individual country modelled.

\sphinxAtStartPar
This book uses as an example a climate aware model for Pakistan developed in 2020 and described in Burns \sphinxstyleemphasis{et al.} {[}\hyperlink{cite.content/99_BackMatter/References:id14}{2021}{]}.

\sphinxAtStartPar
The World Bank models are distributed in the \sphinxcode{\sphinxupquote{pcim}} file format of the \sphinxcode{\sphinxupquote{modelflow}} package and can be downloaded by right clicking on the links above.  The Pakistan model can be downloaded \DUrole{xref,download,myst}{here} by right clicking on the above link and selecting Save Link as and placing the file on a directory accessible by your \sphinxcode{\sphinxupquote{modelflow}} installation.

\begin{sphinxadmonition}{note}{Comment}

\sphinxAtStartPar
\(\color{green}{\text{Ib can't we have a package WorldBankMFModModels that one could import?}}\)
\(\color{green}{\text{I seem to see this for other packages that have geographic data on countries or their population.}}\)

\sphinxAtStartPar
\(\color{green}{\text{I imagining something like:}}\)
\(\color{red}{\text{from worldbankMFModModels import pak}}\)
\end{sphinxadmonition}


\section{Preparing your python environment}
\label{\detokenize{content/05_WBModels/LoadingWBModel:preparing-your-python-environment}}
\sphinxAtStartPar
As always, \sphinxcode{\sphinxupquote{modelflow}}, and other python packages that will be used, need to be imported into your python session.  The examples here and this book were written and solved in a \sphinxstyleemphasis{Jupyter Notebook}. There are some Jupyter specific commands included in these examples and these are annotated. However, the bulk of the content of the programs can be run in other environments, including Interactive Development Environments (IDE) like \sphinxcode{\sphinxupquote{Spyder}}or \sphinxcode{\sphinxupquote{MS Visual Code}}.  All the programs have been tested under \sphinxcode{\sphinxupquote{spyder}} as well as Jupyter Notebook.

\sphinxAtStartPar
It is assumed that you:
\begin{enumerate}
\sphinxsetlistlabels{\arabic}{enumi}{enumii}{}{.}%
\item {} 
\sphinxAtStartPar
have already installed \sphinxcode{\sphinxupquote{modelflow}} and its various support packages following the instructions in Chapter 2.

\item {} 
\sphinxAtStartPar
are using Anaconda

\item {} 
\sphinxAtStartPar
have activated your \sphinxcode{\sphinxupquote{modelflow}} environment by executing the following command from your python command line:

\end{enumerate}

\begin{sphinxVerbatim}[commandchars=\\\{\}]
\PYG{n}{conda} \PYG{n}{activate} \PYG{n}{modelflow}
\end{sphinxVerbatim}

\sphinxAtStartPar
where \sphinxcode{\sphinxupquote{modelflow}} is the name you have given to the \sphinxcode{\sphinxupquote{conda}} environment into which you installed \sphinxcode{\sphinxupquote{modelflow}}.


\chapter{Working with PakMod under modelflow}
\label{\detokenize{content/05_WBModels/LoadingWBModel:working-with-pakmod-under-modelflow}}
\sphinxAtStartPar
The basic method for working with any model is the same. Indeed the initial steps followed here are the same as were followed during the simple model discussion.

\sphinxAtStartPar
Process:
\begin{enumerate}
\sphinxsetlistlabels{\arabic}{enumi}{enumii}{}{.}%
\item {} 
\sphinxAtStartPar
Prepare the workspace

\item {} 
\sphinxAtStartPar
Load the model Modelflow

\item {} 
\sphinxAtStartPar
Design some scenarios

\item {} 
\sphinxAtStartPar
Simulate the model

\item {} 
\sphinxAtStartPar
Visualize the results

\end{enumerate}


\section{Prepare the workspace}
\label{\detokenize{content/05_WBModels/LoadingWBModel:prepare-the-workspace}}
\begin{sphinxuseclass}{cell}\begin{sphinxVerbatimInput}

\begin{sphinxuseclass}{cell_input}
\begin{sphinxVerbatim}[commandchars=\\\{\}]
\PYG{c+c1}{\PYGZsh{} import the model class from modelflow package}
\PYG{k+kn}{from} \PYG{n+nn}{modelclass} \PYG{k+kn}{import} \PYG{n}{model} 
\PYG{k+kn}{import} \PYG{n+nn}{modelmf}       \PYG{c+c1}{\PYGZsh{} Add useful features to pandas dataframes }
                     \PYG{c+c1}{\PYGZsh{} using utlities initially developed for modelflow}

   

\PYG{n}{model}\PYG{o}{.}\PYG{n}{widescreen}\PYG{p}{(}\PYG{p}{)}   \PYG{c+c1}{\PYGZsh{} These modelflow commands ensure that outputs from modelflow play well with Jupyter Notebook}
\PYG{n}{model}\PYG{o}{.}\PYG{n}{scroll\PYGZus{}off}\PYG{p}{(}\PYG{p}{)}

\PYG{o}{\PYGZpc{}}\PYG{k}{load\PYGZus{}ext} autoreload   
\PYG{o}{\PYGZpc{}}\PYG{k}{autoreload} 2
\end{sphinxVerbatim}

\end{sphinxuseclass}\end{sphinxVerbatimInput}
\begin{sphinxVerbatimOutput}

\begin{sphinxuseclass}{cell_output}
\begin{sphinxVerbatim}[commandchars=\\\{\}]
\PYGZlt{}IPython.core.display.HTML object\PYGZgt{}
\end{sphinxVerbatim}

\end{sphinxuseclass}\end{sphinxVerbatimOutput}

\end{sphinxuseclass}

\section{Load the model: Load a pre\sphinxhyphen{}existing model, data and descriptions}
\label{\detokenize{content/05_WBModels/LoadingWBModel:load-the-model-load-a-pre-existing-model-data-and-descriptions}}
\sphinxAtStartPar
To load a model use the \sphinxcode{\sphinxupquote{model.modelload()}} method of \sphinxcode{\sphinxupquote{modelflow}}.


\subsection{The \sphinxstyleliteralintitle{\sphinxupquote{.modelload()}} method}
\label{\detokenize{content/05_WBModels/LoadingWBModel:the-modelload-method}}
\sphinxAtStartPar
The command below

\begin{sphinxVerbatim}[commandchars=\\\{\}]
\PYG{n}{mpak}\PYG{p}{,}\PYG{n}{bline} \PYG{o}{=} \PYG{n}{model}\PYG{o}{.}\PYG{n}{modelload}\PYG{p}{(}\PYG{l+s+s1}{\PYGZsq{}}\PYG{l+s+s1}{..}\PYG{l+s+s1}{\PYGZbs{}}\PYG{l+s+s1}{models}\PYG{l+s+s1}{\PYGZbs{}}\PYG{l+s+s1}{pak.pcim}\PYG{l+s+s1}{\PYGZsq{}}\PYG{p}{,} \PYG{n}{alfa}\PYG{o}{=}\PYG{l+m+mf}{0.7}\PYG{p}{,}\PYG{n}{run}\PYG{o}{=}\PYG{l+m+mi}{1}\PYG{p}{,}\PYG{n}{keep}\PYG{o}{=} \PYG{l+s+s1}{\PYGZsq{}}\PYG{l+s+s1}{Baseline}\PYG{l+s+s1}{\PYGZsq{}}\PYG{p}{)}
\end{sphinxVerbatim}

\sphinxAtStartPar
instantiates (creates an instance of) a \sphinxcode{\sphinxupquote{modelflow model}} object and assigns it to the variable name mpak.

\sphinxAtStartPar
The \sphinxcode{\sphinxupquote{run=1}} option executes the model and assigns the result of the model execution to the dataframe \sphinxcode{\sphinxupquote{bline}}.

\sphinxAtStartPar
The model is solved with the parameter alfa set to 0.7.  The \(alfa \in (0,1)\) parameter determines the step size of the solution engine. The larger alfa the larger the step size. Larger step sizes may solve faster, but may have trouble finding a unique solution.  Smaller step sizes take longer to solve but are more likely to find a unique solution.  Values of alfa=.7 work well for World Bank models.

\sphinxAtStartPar
The \sphinxcode{\sphinxupquote{keep}} option instructs \sphinxcode{\sphinxupquote{modelflow}} to maintain in the model object (\sphinxcode{\sphinxupquote{mpak}}) the results of the initial scenario, assigning it the text name \sphinxcode{\sphinxupquote{Baseline}}.   The variable \sphinxcode{\sphinxupquote{bline}} also contains the dataframe with the results of the simulation.  This is distinct from the data that is stored by the \sphinxcode{\sphinxupquote{keep=}} command. The data associated with each, while stored separately, have the same numerical values. The \sphinxcode{\sphinxupquote{keep}} option is described in more detail toward the end of this section.

\begin{sphinxuseclass}{cell}\begin{sphinxVerbatimInput}

\begin{sphinxuseclass}{cell_input}
\begin{sphinxVerbatim}[commandchars=\\\{\}]
\PYG{c+c1}{\PYGZsh{}Replace the path below with the location of the pak.pcim file on your computer}
\PYG{n}{mpak}\PYG{p}{,}\PYG{n}{bline} \PYG{o}{=} \PYG{n}{model}\PYG{o}{.}\PYG{n}{modelload}\PYG{p}{(}\PYG{l+s+s1}{\PYGZsq{}}\PYG{l+s+s1}{..}\PYG{l+s+s1}{\PYGZbs{}}\PYG{l+s+s1}{models}\PYG{l+s+s1}{\PYGZbs{}}\PYG{l+s+s1}{pak.pcim}\PYG{l+s+s1}{\PYGZsq{}}\PYG{p}{,} \PYGZbs{}
                                \PYG{n}{alfa}\PYG{o}{=}\PYG{l+m+mf}{0.7}\PYG{p}{,}\PYG{n}{run}\PYG{o}{=}\PYG{l+m+mi}{1}\PYG{p}{,}\PYG{n}{keep}\PYG{o}{=} \PYG{l+s+s1}{\PYGZsq{}}\PYG{l+s+s1}{Baseline}\PYG{l+s+s1}{\PYGZsq{}}\PYG{p}{)}
\end{sphinxVerbatim}

\end{sphinxuseclass}\end{sphinxVerbatimInput}
\begin{sphinxVerbatimOutput}

\begin{sphinxuseclass}{cell_output}
\begin{sphinxVerbatim}[commandchars=\\\{\}]
file read:  C:\PYGZbs{}modelflow manual\PYGZbs{}papers\PYGZbs{}mfbook\PYGZbs{}content\PYGZbs{}models\PYGZbs{}pak.pcim
\end{sphinxVerbatim}

\end{sphinxuseclass}\end{sphinxVerbatimOutput}

\end{sphinxuseclass}

\subsection{Extracting information about the model}
\label{\detokenize{content/05_WBModels/LoadingWBModel:extracting-information-about-the-model}}
\sphinxAtStartPar
The newly loaded python object  \sphinxcode{\sphinxupquote{mpak}} is an instance of the model class and as such inherits the \sphinxcode{\sphinxupquote{methods}} (functions) and \sphinxcode{\sphinxupquote{properties}} (data) of that class. To learn about the model there are a variety of methods that can be used to extract information about the model and its data.

\sphinxAtStartPar
A World Bank model in \sphinxcode{\sphinxupquote{modelflow}} will contain a wide range of objects.
\begin{itemize}
\item {} 
\sphinxAtStartPar
variables  – time series variables comprised of mnemonics and data

\item {} 
\sphinxAtStartPar
dataframes – data for each variable generated in  different simulations

\item {} 
\sphinxAtStartPar
groups     – lists of variables

\item {} 
\sphinxAtStartPar
equations  – identities and behaviourals

\item {} 
\sphinxAtStartPar
model      – the model object itself

\end{itemize}

\sphinxAtStartPar
Extracting information about each of these objects is central to working with WBG models in \sphinxcode{\sphinxupquote{modelflow}}.

\sphinxAtStartPar
The model object contains information about the model itself, its name, its structure (does it contain simultaneous equations or is it recursive), the number of variables it contains and the number that are exogenous and endogenous (have associated equations).

\begin{sphinxuseclass}{cell}\begin{sphinxVerbatimInput}

\begin{sphinxuseclass}{cell_input}
\begin{sphinxVerbatim}[commandchars=\\\{\}]
\PYG{n}{mpak}
\end{sphinxVerbatim}

\end{sphinxuseclass}\end{sphinxVerbatimInput}
\begin{sphinxVerbatimOutput}

\begin{sphinxuseclass}{cell_output}
\begin{sphinxVerbatim}[commandchars=\\\{\}]
\PYGZlt{}
Model name                              :                  PAK 
Model structure                         :         Simultaneous 
Number of variables                     :                  839 
Number of exogeneous  variables         :                  461 
Number of endogeneous variables         :                  378 
\PYGZgt{}
\end{sphinxVerbatim}

\end{sphinxuseclass}\end{sphinxVerbatimOutput}

\end{sphinxuseclass}
\sphinxAtStartPar
The model work space also has a time dimension, its sample period.  This can be retrieved and changed.

\begin{sphinxuseclass}{cell}\begin{sphinxVerbatimInput}

\begin{sphinxuseclass}{cell_input}
\begin{sphinxVerbatim}[commandchars=\\\{\}]
\PYG{n}{mpak}\PYG{o}{.}\PYG{n}{current\PYGZus{}per}
\end{sphinxVerbatim}

\end{sphinxuseclass}\end{sphinxVerbatimInput}
\begin{sphinxVerbatimOutput}

\begin{sphinxuseclass}{cell_output}
\begin{sphinxVerbatim}[commandchars=\\\{\}]
Index([2016, 2017, 2018, 2019, 2020, 2021, 2022, 2023, 2024, 2025, 2026, 2027,
       2028, 2029, 2030],
      dtype=\PYGZsq{}int64\PYGZsq{})
\end{sphinxVerbatim}

\end{sphinxuseclass}\end{sphinxVerbatimOutput}

\end{sphinxuseclass}

\subsection{Information about variables}
\label{\detokenize{content/05_WBModels/LoadingWBModel:information-about-variables}}
\sphinxAtStartPar
The model object \sphinxcode{\sphinxupquote{mpak}} contains lists of all the variables that form part of the model, and these lists can be interrogated to garner information about the model.  The Table below indicates some of the most important of these.  The variables for which information is sought can be specified directly or through a wildcard specification (see note).


\begin{savenotes}\sphinxattablestart
\centering
\begin{tabulary}{\linewidth}[t]{|T|T|T|}
\hline
\sphinxstyletheadfamily 
\sphinxAtStartPar
Method
&\sphinxstyletheadfamily 
\sphinxAtStartPar
Example
&\sphinxstyletheadfamily 
\sphinxAtStartPar
Information returned
\\
\hline
\sphinxAtStartPar
\sphinxcode{\sphinxupquote{.names}}
&
\sphinxAtStartPar
\sphinxcode{\sphinxupquote{modelname{[}'PAKNECON*XN{]}.name}}
&
\sphinxAtStartPar
returns a python list of the mnemnics of all the variables defined and contained in the model object that match the search paremers in the \sphinxcode{\sphinxupquote{{[}{]}}}
\\
\hline
\sphinxAtStartPar
\sphinxcode{\sphinxupquote{.des}}
&
\sphinxAtStartPar
\sphinxcode{\sphinxupquote{modelname{[}'PAKNECONPRVT?N'{]}.des}}
&
\sphinxAtStartPar
Dictionary of mnemonics and their variable descriptions
\\
\hline
\sphinxAtStartPar
\sphinxcode{\sphinxupquote{.desc}}
&
\sphinxAtStartPar
\sphinxcode{\sphinxupquote{modelname{[}'PAKNECONPRVTXN'{]}.desc}}
&
\sphinxAtStartPar
List of variable description alone
\\
\hline
\sphinxAtStartPar
\sphinxcode{\sphinxupquote{.<var name>.show}}
&
\sphinxAtStartPar
\sphinxcode{\sphinxupquote{modelname.PAKNECONPRVTXN.show}}
&
\sphinxAtStartPar
Lists the equation (formula), variable descriptions and variable values
\\
\hline
\end{tabulary}
\par
\sphinxattableend\end{savenotes}

\begin{sphinxadmonition}{note}{Note:}
\sphinxAtStartPar
\sphinxstylestrong{Wildcards}

\sphinxAtStartPar
Most of the information commands accept wildcard specifications in the search parameter.

\sphinxAtStartPar
The \sphinxcode{\sphinxupquote{*}} character in the command \sphinxcode{\sphinxupquote{mpak{[}'PAKNECON*XN'{]}.names}} example is a \sphinxcode{\sphinxupquote{wildcard}} character and the expression will return all variables that begin PAKNECON and end XN.

\sphinxAtStartPar
The \sphinxcode{\sphinxupquote{?}} in the \sphinxcode{\sphinxupquote{.des}} example is another wildcard expression. It will match only single characters.  Thus \sphinxcode{\sphinxupquote{mpak{[}'PAKNECONPRVT?N'{]}.names}}  would return three variables: \sphinxcode{\sphinxupquote{PAKNECONPRVTKN}}, \sphinxcode{\sphinxupquote{PAKNECONPRVTXN}}, and \sphinxcode{\sphinxupquote{PAKNECONPRVTXN}}.  The real, current value, and deflators for household consumption expenditure.

\sphinxAtStartPar
Note the final show example uses a slightly different syntax where the variable to be operated upon is specified directly: \sphinxcode{\sphinxupquote{modelname.PAKNECONPRVTXN.show}}.
\end{sphinxadmonition}

\sphinxAtStartPar
The example below returns the mnemonics and descriptions of all variables matching the pattern \sphinxcode{\sphinxupquote{PAKNYGDP*KN}}, i.e. Pakistani variables from the National Income Accounts from the main sub\sphinxhyphen{}category GDP that are also real variables.

\begin{sphinxuseclass}{cell}\begin{sphinxVerbatimInput}

\begin{sphinxuseclass}{cell_input}
\begin{sphinxVerbatim}[commandchars=\\\{\}]
\PYG{n}{mpak}\PYG{p}{[}\PYG{l+s+s1}{\PYGZsq{}}\PYG{l+s+s1}{PAKNYGDP*KN}\PYG{l+s+s1}{\PYGZsq{}}\PYG{p}{]}\PYG{o}{.}\PYG{n}{des}
\end{sphinxVerbatim}

\end{sphinxuseclass}\end{sphinxVerbatimInput}
\begin{sphinxVerbatimOutput}

\begin{sphinxuseclass}{cell_output}
\begin{sphinxVerbatim}[commandchars=\\\{\}]
PAKNYGDPDISCKN : GDP Disc., 2000 LCU mn
PAKNYGDPFCSTKN : GDP Factor Cost Local Currency units Volumes National base year
PAKNYGDPMKTPKN : Real GDP
PAKNYGDPPOTLKN : Potential Output, constant LCU
\end{sphinxVerbatim}

\end{sphinxuseclass}\end{sphinxVerbatimOutput}

\end{sphinxuseclass}
\begin{sphinxadmonition}{note}{Box {[}\textasciicircum{}BoxWBMnemonics{]}: World Bank Mnemonics}

\sphinxAtStartPar
A typical World Bank model will have in excess of 300 variables.  Each has a mnemonic that is structured in a specific way, The root for almost all are 14 characters long (some special variables have additional characters appended to this root) (see discussion in section).
\begin{equation*}
\begin{split}\texttt{12345678901234}\end{split}
\end{equation*}\begin{equation*}
\begin{split}\color{green}{\texttt{CCC}}\color{red}{\texttt{AA}}\color{lime}{\texttt{MMM}}\color{blue}{\texttt{NNNN}}\color{magenta}{\texttt{U}}\color{black}{\texttt{C}}\end{split}
\end{equation*}
\sphinxAtStartPar
where:


\begin{savenotes}\sphinxattablestart
\centering
\begin{tabulary}{\linewidth}[t]{|T|T|}
\hline
\sphinxstyletheadfamily 
\sphinxAtStartPar
Letters
&\sphinxstyletheadfamily 
\sphinxAtStartPar
Meaning
\\
\hline
\sphinxAtStartPar
\(\color{green}{\texttt{CCC}}\)
&
\sphinxAtStartPar
The three\sphinxhyphen{}leter ISO code for a country – i.e. IDN for Indonesia, RUS for Russia
\\
\hline
\sphinxAtStartPar
\(\color{red}{\texttt{AA}}\)
&
\sphinxAtStartPar
The two\sphinxhyphen{}letter major accounting system to which the variable attaches,
\\
\hline
\sphinxAtStartPar
\(\color{lime}{\texttt{MMM}}\)
&
\sphinxAtStartPar
The three\sphinxhyphen{}letter major sub\sphinxhyphen{}category of the data \sphinxhyphen{}  i.e. GDP, EXP \sphinxhyphen{} expenditure
\\
\hline
\sphinxAtStartPar
\(\color{blue}{\texttt{NNNN}}\)
&
\sphinxAtStartPar
The four\sphinxhyphen{}letter minor sub\sphinxhyphen{}category  MKTP for market prices
\\
\hline
\sphinxAtStartPar
\(\color{magenta}{\texttt{U}}\)
&
\sphinxAtStartPar
The measure  (K: real variable;C: Current Values; X: Prices)
\\
\hline
\sphinxAtStartPar
\(\color{black}{\texttt{C}}\)
&
\sphinxAtStartPar
denotes the Currency (N: National currency; D: USD; P: PPP)
\\
\hline
\end{tabulary}
\par
\sphinxattableend\end{savenotes}

\sphinxAtStartPar
Common major accounting systems mnemonics: the, \(\color{red}{\texttt{AA}}\)s from above:


\begin{savenotes}\sphinxattablestart
\centering
\begin{tabulary}{\linewidth}[t]{|T|T|}
\hline
\sphinxstyletheadfamily 
\sphinxAtStartPar
Code
&\sphinxstyletheadfamily 
\sphinxAtStartPar
Meaning
\\
\hline
\sphinxAtStartPar
NY
&
\sphinxAtStartPar
National income
\\
\hline
\sphinxAtStartPar
NE
&
\sphinxAtStartPar
National expenditure Accounts
\\
\hline
\sphinxAtStartPar
NV
&
\sphinxAtStartPar
Value added accounts
\\
\hline
\sphinxAtStartPar
GG
&
\sphinxAtStartPar
General Government Accounts
\\
\hline
\sphinxAtStartPar
BX
&
\sphinxAtStartPar
Balance of Payments: Exports
\\
\hline
\sphinxAtStartPar
BM
&
\sphinxAtStartPar
Balance of Payments: Imports
\\
\hline
\sphinxAtStartPar
BN
&
\sphinxAtStartPar
Balance of Payments: Net
\\
\hline
\sphinxAtStartPar
BF
&
\sphinxAtStartPar
Balance of Payments: Financial Account
\\
\hline
\end{tabulary}
\par
\sphinxattableend\end{savenotes}

\sphinxAtStartPar
Thus


\begin{savenotes}\sphinxattablestart
\centering
\begin{tabulary}{\linewidth}[t]{|T|T|}
\hline
\sphinxstyletheadfamily 
\sphinxAtStartPar
Mnemonic
&\sphinxstyletheadfamily 
\sphinxAtStartPar
Meaning
\\
\hline
\sphinxAtStartPar
IDNNYGDPMKTPKN
&
\sphinxAtStartPar
Indonesia GDP at market prices, real in Indonesian Rupiah
\\
\hline
\sphinxAtStartPar
KENNECPNPRVTXN
&
\sphinxAtStartPar
Kenya Private (household) consumption expenditure schillings deflator
\\
\hline
\sphinxAtStartPar
BOLGGEXPGNFSCN
&
\sphinxAtStartPar
Bolivia Government Expenditure on Goods and services (GNFS) in current Bolivars
\\
\hline
\sphinxAtStartPar
HRVGGREVDCITCN
&
\sphinxAtStartPar
Croatia Government Revenues Direct Corporate Income Taxes in current Euros
\\
\hline
\sphinxAtStartPar
NPLBXGSRNFSVCD
&
\sphinxAtStartPar
Nepal BOP Exports of non\sphinxhyphen{}factor services (goods and services) in current USD
\\
\hline
\end{tabulary}
\par
\sphinxattableend\end{savenotes}
\end{sphinxadmonition}

\sphinxAtStartPar
If executed, the command \sphinxcode{\sphinxupquote{mpak{[}'*'{]}.des}} would return a dictionary of all the mnemonics and descriptions of all the variables in the \sphinxcode{\sphinxupquote{mpak}} model object.


\subsubsection{The \sphinxstyleliteralintitle{\sphinxupquote{!}} operator – searching on the variable description}
\label{\detokenize{content/05_WBModels/LoadingWBModel:the-operator-searching-on-the-variable-description}}
\sphinxAtStartPar
The same methods can be used to retrieve information about variables, based on their descriptions (vs mnemonic), by pre\sphinxhyphen{}pending the search string with the  \sphinxcode{\sphinxupquote{!}} operator.

\begin{sphinxadmonition}{note}{Note:}
\sphinxAtStartPar
\sphinxstylestrong{The ! operator}
If a wildcard is preceded by an exclamation mark \sphinxstylestrong{!} the search will be done over the description of variables instead of the mnemonic
\end{sphinxadmonition}

\sphinxAtStartPar
The below expression returns all variables whose description includes the word Carbon.

\begin{sphinxuseclass}{cell}\begin{sphinxVerbatimInput}

\begin{sphinxuseclass}{cell_input}
\begin{sphinxVerbatim}[commandchars=\\\{\}]
\PYG{n}{mpak}\PYG{p}{[}\PYG{l+s+s1}{\PYGZsq{}}\PYG{l+s+s1}{!*Carbon*}\PYG{l+s+s1}{\PYGZsq{}}\PYG{p}{]}\PYG{o}{.}\PYG{n}{des}
\end{sphinxVerbatim}

\end{sphinxuseclass}\end{sphinxVerbatimInput}
\begin{sphinxVerbatimOutput}

\begin{sphinxuseclass}{cell_output}
\begin{sphinxVerbatim}[commandchars=\\\{\}]
PAKGGREVCO2CER : Carbon tax on coal (USD/t)
PAKGGREVCO2GER : Carbon tax on gas (USD/t)
PAKGGREVCO2OER : Carbon tax on oil (USD/t)
\end{sphinxVerbatim}

\end{sphinxuseclass}\end{sphinxVerbatimOutput}

\end{sphinxuseclass}

\section{Groups}
\label{\detokenize{content/05_WBModels/LoadingWBModel:groups}}
\sphinxAtStartPar
Modelflow inherits a variant of the idea of groups from \sphinxcode{\sphinxupquote{EViews}}.  In \sphinxcode{\sphinxupquote{modelflow}} the groups defined in an imported EViews workfile are converted into entries in a dictionary called \sphinxcode{\sphinxupquote{var\_groups}} which can be interrogated, added to and amended like any dictionary in python.

\sphinxAtStartPar
The command
\sphinxcode{\sphinxupquote{mpak.var\_groups}} will return all of the groups already defined in mpak.

\begin{sphinxuseclass}{cell}\begin{sphinxVerbatimInput}

\begin{sphinxuseclass}{cell_input}
\begin{sphinxVerbatim}[commandchars=\\\{\}]
\PYG{n}{mpak}\PYG{o}{.}\PYG{n}{var\PYGZus{}groups}
\end{sphinxVerbatim}

\end{sphinxuseclass}\end{sphinxVerbatimInput}
\begin{sphinxVerbatimOutput}

\begin{sphinxuseclass}{cell_output}
\begin{sphinxVerbatim}[commandchars=\\\{\}]
\PYGZob{}\PYGZsq{}Headline\PYGZsq{}: \PYGZsq{}???GDPpckn ???NRTOTLCN ???LMEMPTOTL ???BFFINCABDCD  ???BFBOPTOTLCD ???GGBALEXGRCN ???BNCABLOCLCD\PYGZus{} ???FPCPITOTLXN\PYGZsq{},
 \PYGZsq{}National income accounts\PYGZsq{}: \PYGZsq{}???NY*\PYGZsq{},
 \PYGZsq{}National expenditure accounts\PYGZsq{}: \PYGZsq{}???NE*\PYGZsq{},
 \PYGZsq{}Value added accounts\PYGZsq{}: \PYGZsq{}???NV*\PYGZsq{},
 \PYGZsq{}Balance of payments exports\PYGZsq{}: \PYGZsq{}???BX*\PYGZsq{},
 \PYGZsq{}Balance of payments exports and value added \PYGZsq{}: \PYGZsq{}???BX* ???NV*\PYGZsq{},
 \PYGZsq{}Balance of Payments Financial Account\PYGZsq{}: \PYGZsq{}???BF*\PYGZsq{},
 \PYGZsq{}General government fiscal accounts\PYGZsq{}: \PYGZsq{}???GG*\PYGZsq{},
 \PYGZsq{}World all\PYGZsq{}: \PYGZsq{}WLD*\PYGZsq{},
 \PYGZsq{}PAK all\PYGZsq{}: \PYGZsq{}PAK*\PYGZsq{}\PYGZcb{}
\end{sphinxVerbatim}

\end{sphinxuseclass}\end{sphinxVerbatimOutput}

\end{sphinxuseclass}
\sphinxAtStartPar
A group can be added to the dictionary by giving it a unique identifier (key) and associating with it a string defining the group, using a wildcard specification or just a space de\sphinxhyphen{}limited list of mnemonics.

\sphinxAtStartPar
Thus the command

\sphinxAtStartPar
\sphinxcode{\sphinxupquote{mpak.var\_groups{[}'Mygroup{]}='PAKGGREV*CN PAKGGEBALOVRLCN'}}

\begin{sphinxuseclass}{cell}\begin{sphinxVerbatimInput}

\begin{sphinxuseclass}{cell_input}
\begin{sphinxVerbatim}[commandchars=\\\{\}]
\PYG{n}{mpak}\PYG{o}{.}\PYG{n}{var\PYGZus{}groups}\PYG{p}{[}\PYG{l+s+s1}{\PYGZsq{}}\PYG{l+s+s1}{Mygroup}\PYG{l+s+s1}{\PYGZsq{}}\PYG{p}{]}\PYG{o}{=}\PYG{l+s+s1}{\PYGZsq{}}\PYG{l+s+s1}{PAKGGREV*CN PAKGGEBALOVRLCN}\PYG{l+s+s1}{\PYGZsq{}}
                
\end{sphinxVerbatim}

\end{sphinxuseclass}\end{sphinxVerbatimInput}

\end{sphinxuseclass}
\begin{sphinxuseclass}{cell}\begin{sphinxVerbatimInput}

\begin{sphinxuseclass}{cell_input}
\begin{sphinxVerbatim}[commandchars=\\\{\}]
\PYG{n}{mpak}\PYG{p}{[}\PYG{l+s+s1}{\PYGZsq{}}\PYG{l+s+s1}{\PYGZsh{}Mygroup}\PYG{l+s+s1}{\PYGZsq{}}\PYG{p}{]}\PYG{o}{.}\PYG{n}{names}
\end{sphinxVerbatim}

\end{sphinxuseclass}\end{sphinxVerbatimInput}
\begin{sphinxVerbatimOutput}

\begin{sphinxuseclass}{cell_output}
\begin{sphinxVerbatim}[commandchars=\\\{\}]
[\PYGZsq{}PAKGGREVDRCTCN\PYGZsq{},
 \PYGZsq{}PAKGGREVEMISCN\PYGZsq{},
 \PYGZsq{}PAKGGREVGNFSCN\PYGZsq{},
 \PYGZsq{}PAKGGREVGRNTCN\PYGZsq{},
 \PYGZsq{}PAKGGREVOTHRCN\PYGZsq{},
 \PYGZsq{}PAKGGREVTOTLCN\PYGZsq{},
 \PYGZsq{}PAKGGREVTRDECN\PYGZsq{}]
\end{sphinxVerbatim}

\end{sphinxuseclass}\end{sphinxVerbatimOutput}

\end{sphinxuseclass}

\section{Information about data}
\label{\detokenize{content/05_WBModels/LoadingWBModel:information-about-data}}
\sphinxAtStartPar
Note the same search functions can be used to display the data associated with the returned variables.

\sphinxAtStartPar
Thus to see the data for the \sphinxcode{\sphinxupquote{Mygroup}} group of variables, one could use the \sphinxcode{\sphinxupquote{.df}} method or \sphinxcode{\sphinxupquote{.plot()}} methods – here modified by pct (to show growth rates) and mul100 to multiply them by 100 to display them as percent change.

\begin{sphinxuseclass}{cell}\begin{sphinxVerbatimInput}

\begin{sphinxuseclass}{cell_input}
\begin{sphinxVerbatim}[commandchars=\\\{\}]
\PYG{n}{mpak}\PYG{p}{[}\PYG{l+s+s1}{\PYGZsq{}}\PYG{l+s+s1}{\PYGZsh{}Mygroup}\PYG{l+s+s1}{\PYGZsq{}}\PYG{p}{]}\PYG{o}{.}\PYG{n}{pct}\PYG{o}{.}\PYG{n}{mul100}\PYG{o}{.}\PYG{n}{plot}\PYG{p}{(}\PYG{p}{)}
\end{sphinxVerbatim}

\end{sphinxuseclass}\end{sphinxVerbatimInput}
\begin{sphinxVerbatimOutput}

\begin{sphinxuseclass}{cell_output}
\noindent\sphinxincludegraphics{{1d2f04bb581f8d29beb1f0e3b2e49b09c51c3e2c0d137a42ee517a0517b6affd}.png}

\end{sphinxuseclass}\end{sphinxVerbatimOutput}

\end{sphinxuseclass}
\sphinxAtStartPar
Below the same logic is used to display the data from variables matching a mnemonic search.  The results have been placed inside a \sphinxcode{\sphinxupquote{with m{[}pak.set\_smpl()}} clause to restrict the output to a shorter period.  If it was not used the output would cover the whole time period of the \sphinxcode{\sphinxupquote{.lastdf}} DataFrame from which all of this data is drawm.

\begin{sphinxuseclass}{cell}\begin{sphinxVerbatimInput}

\begin{sphinxuseclass}{cell_input}
\begin{sphinxVerbatim}[commandchars=\\\{\}]
\PYG{k}{with} \PYG{n}{mpak}\PYG{o}{.}\PYG{n}{set\PYGZus{}smpl}\PYG{p}{(}\PYG{l+m+mi}{2020}\PYG{p}{,}\PYG{l+m+mi}{2030}\PYG{p}{)}\PYG{p}{:}
    \PYG{n+nb}{print}\PYG{p}{(}\PYG{n+nb}{round}\PYG{p}{(}\PYG{n}{mpak}\PYG{p}{[}\PYG{l+s+s1}{\PYGZsq{}}\PYG{l+s+s1}{\PYGZsh{}Mygroup}\PYG{l+s+s1}{\PYGZsq{}}\PYG{p}{]}\PYG{o}{.}\PYG{n}{pct}\PYG{o}{.}\PYG{n}{mul100}\PYG{o}{.}\PYG{n}{df}\PYG{p}{,}\PYG{l+m+mi}{2}\PYG{p}{)}\PYG{p}{)}
\end{sphinxVerbatim}

\end{sphinxuseclass}\end{sphinxVerbatimInput}
\begin{sphinxVerbatimOutput}

\begin{sphinxuseclass}{cell_output}
\begin{sphinxVerbatim}[commandchars=\\\{\}]
      PAKGGREVDRCTCN  PAKGGREVEMISCN  PAKGGREVGNFSCN  PAKGGREVGRNTCN   
2020           13.30            1.10           13.25           39.48  \PYGZbs{}
2021           11.69            0.21           11.33           29.52   
2022           10.48            0.28           10.11           23.40   
2023            9.84            0.82            9.60           19.62   
2024            9.48            1.42            9.36           17.09   
2025            9.21            1.88            9.18           15.24   
2026            8.94            2.14            8.95           13.79   
2027            8.67            2.27            8.69           12.61   
2028            8.43            2.31            8.43           11.65   
2029            8.24            2.34            8.22           10.89   
2030            8.11            2.35            8.08           10.31   

      PAKGGREVOTHRCN  PAKGGREVTOTLCN  PAKGGREVTRDECN  
2020           17.83           16.77           18.25  
2021           15.34           14.39           15.28  
2022           13.45           12.69           13.71  
2023           12.29           11.72           12.89  
2024           11.51           11.10           12.35  
2025           10.90           10.61           11.87  
2026           10.35           10.15           11.38  
2027            9.85            9.72           10.90  
2028            9.42            9.34           10.46  
2029            9.07            9.03           10.08  
2030            8.82            8.80            9.76  
\end{sphinxVerbatim}

\end{sphinxuseclass}\end{sphinxVerbatimOutput}

\end{sphinxuseclass}
\sphinxAtStartPar
Jupyter truncates the output by showing the first and last five observations of the active sample period when the same call is  made without the with clause.

\begin{sphinxuseclass}{cell}\begin{sphinxVerbatimInput}

\begin{sphinxuseclass}{cell_input}
\begin{sphinxVerbatim}[commandchars=\\\{\}]
\PYG{n}{mpak}\PYG{o}{.}\PYG{n}{smpl}\PYG{p}{(}\PYG{l+m+mi}{2000}\PYG{p}{,}\PYG{l+m+mi}{2100}\PYG{p}{)}
\PYG{n}{mpak}\PYG{p}{[}\PYG{l+s+s1}{\PYGZsq{}}\PYG{l+s+s1}{\PYGZsh{}Mygroup}\PYG{l+s+s1}{\PYGZsq{}}\PYG{p}{]}\PYG{o}{.}\PYG{n}{pct}\PYG{o}{.}\PYG{n}{mul100}\PYG{o}{.}\PYG{n}{df}
\end{sphinxVerbatim}

\end{sphinxuseclass}\end{sphinxVerbatimInput}
\begin{sphinxVerbatimOutput}

\begin{sphinxuseclass}{cell_output}
\begin{sphinxVerbatim}[commandchars=\\\{\}]
      PAKGGREVDRCTCN  PAKGGREVEMISCN  PAKGGREVGNFSCN  PAKGGREVGRNTCN   
2000        9.550328      101.829915       70.016016             NaN  \PYGZbs{}
2001       11.138391       15.374786       31.458374             inf   
2002       14.660537      \PYGZhy{}13.232471        8.545928       94.822463   
2003        7.111796       35.469443       17.116998      \PYGZhy{}36.962342   
2004        8.425066       21.635646       13.051789      \PYGZhy{}39.977372   
...              ...             ...             ...             ...   
2096        9.025284        2.844796        9.066803        9.025327   
2097        9.021221        2.842709        9.063564        9.021258   
2098        9.017159        2.840601        9.060286        9.017190   
2099        9.013108        2.838480        9.056979        9.013134   
2100        9.009075        2.836350        9.053653        9.009098   

      PAKGGREVOTHRCN  PAKGGREVTOTLCN  PAKGGREVTRDECN  
2000             NaN        7.298335      \PYGZhy{}21.682305  
2001             inf       16.344272        5.519481  
2002       17.589007       22.839281      \PYGZhy{}26.435385  
2003       15.198571        6.038960       43.955079  
2004       26.297036       15.683372       32.113024  
...              ...             ...             ...  
2096        9.025299        9.027988        8.957897  
2097        9.021234        9.024111        8.955230  
2098        9.017170        9.020235        8.952564  
2099        9.013117        9.016371        8.949905  
2100        9.009083        9.012525        8.947260  

[101 rows x 7 columns]
\end{sphinxVerbatim}

\end{sphinxuseclass}\end{sphinxVerbatimOutput}

\end{sphinxuseclass}

\subsection{Some examples}
\label{\detokenize{content/05_WBModels/LoadingWBModel:some-examples}}

\subsubsection{\sphinxstyleliteralintitle{\sphinxupquote{.names}} property}
\label{\detokenize{content/05_WBModels/LoadingWBModel:names-property}}
\sphinxAtStartPar
\sphinxcode{\sphinxupquote{mpak{[}'PAKNECON*XN'{]}.names}}

\sphinxAtStartPar
Return the names (mnemonmics) of all variables that begin \sphinxcode{\sphinxupquote{PAKNECON}} and end \sphinxcode{\sphinxupquote{XN}} – i.e. Price deflators for various types of consumption demand.

\begin{sphinxuseclass}{cell}\begin{sphinxVerbatimInput}

\begin{sphinxuseclass}{cell_input}
\begin{sphinxVerbatim}[commandchars=\\\{\}]
\PYG{n}{mpak}\PYG{p}{[}\PYG{l+s+s1}{\PYGZsq{}}\PYG{l+s+s1}{PAKNECON*XN}\PYG{l+s+s1}{\PYGZsq{}}\PYG{p}{]}\PYG{o}{.}\PYG{n}{names}
\end{sphinxVerbatim}

\end{sphinxuseclass}\end{sphinxVerbatimInput}
\begin{sphinxVerbatimOutput}

\begin{sphinxuseclass}{cell_output}
\begin{sphinxVerbatim}[commandchars=\\\{\}]
[\PYGZsq{}PAKNECONENGYXN\PYGZsq{}, \PYGZsq{}PAKNECONGOVTXN\PYGZsq{}, \PYGZsq{}PAKNECONOTHRXN\PYGZsq{}, \PYGZsq{}PAKNECONPRVTXN\PYGZsq{}]
\end{sphinxVerbatim}

\end{sphinxuseclass}\end{sphinxVerbatimOutput}

\end{sphinxuseclass}

\subsubsection{The \sphinxstyleliteralintitle{\sphinxupquote{.des}} property}
\label{\detokenize{content/05_WBModels/LoadingWBModel:the-des-property}}
\sphinxAtStartPar
\sphinxcode{\sphinxupquote{mpak{[}'PAKNECONPRVT?N'{]}.des}}

\sphinxAtStartPar
Returns a dictionary comprised of the mnemonics and the descriptions of all the variables that begin \sphinxcode{\sphinxupquote{PAKNECONPRVT}} and end \sphinxcode{\sphinxupquote{N}}, but have only one character between the T and the N.

\begin{sphinxuseclass}{cell}\begin{sphinxVerbatimInput}

\begin{sphinxuseclass}{cell_input}
\begin{sphinxVerbatim}[commandchars=\\\{\}]
\PYG{n}{mpak}\PYG{p}{[}\PYG{l+s+s1}{\PYGZsq{}}\PYG{l+s+s1}{PAKNECONPRVT?N}\PYG{l+s+s1}{\PYGZsq{}}\PYG{p}{]}\PYG{o}{.}\PYG{n}{des}
\end{sphinxVerbatim}

\end{sphinxuseclass}\end{sphinxVerbatimInput}
\begin{sphinxVerbatimOutput}

\begin{sphinxuseclass}{cell_output}
\begin{sphinxVerbatim}[commandchars=\\\{\}]
PAKNECONPRVTCN : Pvt. Cons., LCU mn
PAKNECONPRVTKN : HH. Cons Real
PAKNECONPRVTXN : Implicit LCU defl., Pvt. Cons., 2000 = 1
\end{sphinxVerbatim}

\end{sphinxuseclass}\end{sphinxVerbatimOutput}

\end{sphinxuseclass}

\subsection{\sphinxstyleliteralintitle{\sphinxupquote{.var\_description}} method}
\label{\detokenize{content/05_WBModels/LoadingWBModel:var-description-method}}
\sphinxAtStartPar
The property \sphinxcode{\sphinxupquote{.var\_description}}returns the descriptor of all variables.  Modified to a psecifc variable it returns the description of that one variable.  \sphinxstylestrong{This method does not accept wildcards}.

\begin{sphinxuseclass}{cell}\begin{sphinxVerbatimInput}

\begin{sphinxuseclass}{cell_input}
\begin{sphinxVerbatim}[commandchars=\\\{\}]
\PYG{c+c1}{\PYGZsh{}mpak.var\PYGZus{}description \PYGZsh{} returns the descirptions for all variables}
\PYG{n}{mpak}\PYG{o}{.}\PYG{n}{var\PYGZus{}description}\PYG{p}{[}\PYG{l+s+s1}{\PYGZsq{}}\PYG{l+s+s1}{PAKNYGDPMKTPCN}\PYG{l+s+s1}{\PYGZsq{}}\PYG{p}{]} \PYG{c+c1}{\PYGZsh{} returns the description of a specific variable}
\end{sphinxVerbatim}

\end{sphinxuseclass}\end{sphinxVerbatimInput}
\begin{sphinxVerbatimOutput}

\begin{sphinxuseclass}{cell_output}
\begin{sphinxVerbatim}[commandchars=\\\{\}]
\PYGZsq{}GDP, Market Prices, LCU mn\PYGZsq{}
\end{sphinxVerbatim}

\end{sphinxuseclass}\end{sphinxVerbatimOutput}

\end{sphinxuseclass}
\begin{sphinxuseclass}{cell}\begin{sphinxVerbatimInput}

\begin{sphinxuseclass}{cell_input}
\begin{sphinxVerbatim}[commandchars=\\\{\}]
\PYG{n}{mpak}\PYG{o}{.}\PYG{n}{PAKNECONPRVTKN}\PYG{o}{.}\PYG{n}{frml}
\end{sphinxVerbatim}

\end{sphinxuseclass}\end{sphinxVerbatimInput}
\begin{sphinxVerbatimOutput}

\begin{sphinxuseclass}{cell_output}
\begin{sphinxVerbatim}[commandchars=\\\{\}]
Endogeneous: PAKNECONPRVTKN: HH. Cons Real
Formular: FRML \PYGZlt{}DAMP,STOC\PYGZgt{} PAKNECONPRVTKN = (PAKNECONPRVTKN(\PYGZhy{}1)*EXP(PAKNECONPRVTKN\PYGZus{}A+ (\PYGZhy{}0.2*(LOG(PAKNECONPRVTKN(\PYGZhy{}1))\PYGZhy{}LOG(1.21203101101442)\PYGZhy{}LOG((((PAKBXFSTREMTCD(\PYGZhy{}1)\PYGZhy{}PAKBMFSTREMTCD(\PYGZhy{}1))*PAKPANUSATLS(\PYGZhy{}1))+PAKGGEXPTRNSCN(\PYGZhy{}1)+PAKNYYWBTOTLCN(\PYGZhy{}1)*(1\PYGZhy{}PAKGGREVDRCTXN(\PYGZhy{}1)/100))/PAKNECONPRVTXN(\PYGZhy{}1)))+0.763938860758873*((LOG((((PAKBXFSTREMTCD\PYGZhy{}PAKBMFSTREMTCD)*PAKPANUSATLS)+PAKGGEXPTRNSCN+PAKNYYWBTOTLCN*(1\PYGZhy{}PAKGGREVDRCTXN/100))/PAKNECONPRVTXN))\PYGZhy{}(LOG((((PAKBXFSTREMTCD(\PYGZhy{}1)\PYGZhy{}PAKBMFSTREMTCD(\PYGZhy{}1))*PAKPANUSATLS(\PYGZhy{}1))+PAKGGEXPTRNSCN(\PYGZhy{}1)+PAKNYYWBTOTLCN(\PYGZhy{}1)*(1\PYGZhy{}PAKGGREVDRCTXN(\PYGZhy{}1)/100))/PAKNECONPRVTXN(\PYGZhy{}1))))\PYGZhy{}0.0634474791568939*DURING\PYGZus{}2009\PYGZhy{}0.3*(PAKFMLBLPOLYXN/100\PYGZhy{}((LOG(PAKNECONPRVTXN))\PYGZhy{}(LOG(PAKNECONPRVTXN(\PYGZhy{}1)))))) )) * (1\PYGZhy{}PAKNECONPRVTKN\PYGZus{}D)+ PAKNECONPRVTKN\PYGZus{}X*PAKNECONPRVTKN\PYGZus{}D  \PYGZdl{}

PAKNECONPRVTKN  : HH. Cons Real
DURING\PYGZus{}2009     : 
PAKBMFSTREMTCD  : Imp., Remittances (BOP), US\PYGZdl{} mn
PAKBXFSTREMTCD  : Exp., Remittances (BOP), US\PYGZdl{} mn
PAKFMLBLPOLYXN  : Key Policy Interest Rate
PAKGGEXPTRNSCN  : Current Transfers
PAKGGREVDRCTXN  : Direct Revenue Tax Rate
PAKNECONPRVTKN\PYGZus{}A: Add factor:HH. Cons Real
PAKNECONPRVTKN\PYGZus{}D: Fix dummy:HH. Cons Real
PAKNECONPRVTKN\PYGZus{}X: Fix value:HH. Cons Real
PAKNECONPRVTXN  : Implicit LCU defl., Pvt. Cons., 2000 = 1
PAKNYYWBTOTLCN  : Total Wage Bill
PAKPANUSATLS    : Exchange rate LCU / US\PYGZdl{} \PYGZhy{} Pakistan
\end{sphinxVerbatim}

\end{sphinxuseclass}\end{sphinxVerbatimOutput}

\end{sphinxuseclass}

\subsection{Information about equations}
\label{\detokenize{content/05_WBModels/LoadingWBModel:information-about-equations}}

\subsubsection{The \sphinxstyleliteralintitle{\sphinxupquote{endogene}} property}
\label{\detokenize{content/05_WBModels/LoadingWBModel:the-endogene-property}}
\sphinxAtStartPar
The  \sphinxcode{\sphinxupquote{endogene}} property either returns a list of all variables in the model that are endogenous (have an equation). It can also be used to test whether a a specific mnemonic has an equation associated with it.

\sphinxAtStartPar
The expression \sphinxcode{\sphinxupquote{'PAKNECONPRVTKN' in mpak.endogene}} returns True if the passed mnemonic is in the list returned by \sphinxcode{\sphinxupquote{mpak.endogene}}.

\begin{sphinxuseclass}{cell}\begin{sphinxVerbatimInput}

\begin{sphinxuseclass}{cell_input}
\begin{sphinxVerbatim}[commandchars=\\\{\}]
\PYG{l+s+s1}{\PYGZsq{}}\PYG{l+s+s1}{PAKNECONPRVTKN}\PYG{l+s+s1}{\PYGZsq{}} \PYG{o+ow}{in} \PYG{n}{mpak}\PYG{o}{.}\PYG{n}{endogene}
\end{sphinxVerbatim}

\end{sphinxuseclass}\end{sphinxVerbatimInput}
\begin{sphinxVerbatimOutput}

\begin{sphinxuseclass}{cell_output}
\begin{sphinxVerbatim}[commandchars=\\\{\}]
True
\end{sphinxVerbatim}

\end{sphinxuseclass}\end{sphinxVerbatimOutput}

\end{sphinxuseclass}

\subsubsection{Retrieving info on equations}
\label{\detokenize{content/05_WBModels/LoadingWBModel:retrieving-info-on-equations}}
\sphinxAtStartPar
There are three functions to extract the equations from a model.


\begin{savenotes}\sphinxattablestart
\centering
\begin{tabulary}{\linewidth}[t]{|T|T|}
\hline
\sphinxstyletheadfamily 
\sphinxAtStartPar
Command
&\sphinxstyletheadfamily 
\sphinxAtStartPar
Effect
\\
\hline
\sphinxAtStartPar
\sphinxcode{\sphinxupquote{mpak{[}'PAKNECONPRVTKN'{]}.frml}}
&
\sphinxAtStartPar
Returns a \sphinxstylestrong{normalized} version of the equation (the one actually used in modelflow)
\\
\hline
\sphinxAtStartPar
\sphinxcode{\sphinxupquote{mpak{[}'PAKNECONPRVTKN'{]}.eviews}}
&
\sphinxAtStartPar
In models imported from Eviews, reports the original eviews specification
\\
\hline
\sphinxAtStartPar
\sphinxcode{\sphinxupquote{mpak.PAKNECONPRVTXN.show}}
&
\sphinxAtStartPar
The equation (formula), variable descriptions variable values
\\
\hline
\end{tabulary}
\par
\sphinxattableend\end{savenotes}

\sphinxAtStartPar
The equation for consumption in \sphinxcode{\sphinxupquote{mpak}} we see that it follows something very close to this formulation.


\paragraph{The \sphinxstyleliteralintitle{\sphinxupquote{.eviews}} method}
\label{\detokenize{content/05_WBModels/LoadingWBModel:the-eviews-method}}
\sphinxAtStartPar
The \sphinxcode{\sphinxupquote{mpak{[}'PAKNECONPRVTKN'{]}.eviews}} command returns the equations before they were normalized. In most cases this is a slightly more legible form. Here following the EViews syntax, \(\Delta ln()\) is written as dlog().

\begin{sphinxuseclass}{cell}\begin{sphinxVerbatimInput}

\begin{sphinxuseclass}{cell_input}
\begin{sphinxVerbatim}[commandchars=\\\{\}]
\PYG{n}{mpak}\PYG{p}{[}\PYG{l+s+s1}{\PYGZsq{}}\PYG{l+s+s1}{PAKNECONPRVTKN}\PYG{l+s+s1}{\PYGZsq{}}\PYG{p}{]}\PYG{o}{.}\PYG{n}{eviews}
\end{sphinxVerbatim}

\end{sphinxuseclass}\end{sphinxVerbatimInput}
\begin{sphinxVerbatimOutput}

\begin{sphinxuseclass}{cell_output}
\begin{sphinxVerbatim}[commandchars=\\\{\}]
PAKNECONPRVTKN : DLOG(PAKNECONPRVTKN) =\PYGZhy{} 0.2*(LOG(PAKNECONPRVTKN( \PYGZhy{} 1)) \PYGZhy{} LOG(1.21203101101442) \PYGZhy{} LOG((((PAKBXFSTREMTCD( \PYGZhy{} 1) \PYGZhy{} PAKBMFSTREMTCD( \PYGZhy{} 1))*PAKPANUSATLS( \PYGZhy{} 1)) + PAKGGEXPTRNSCN( \PYGZhy{} 1) + PAKNYYWBTOTLCN( \PYGZhy{} 1)*(1 \PYGZhy{} PAKGGREVDRCTXN( \PYGZhy{} 1)/100))/PAKNECONPRVTXN( \PYGZhy{} 1))) + 0.763938860758873*DLOG((((PAKBXFSTREMTCD \PYGZhy{} PAKBMFSTREMTCD)*PAKPANUSATLS) + PAKGGEXPTRNSCN + PAKNYYWBTOTLCN*(1 \PYGZhy{} PAKGGREVDRCTXN/100))/PAKNECONPRVTXN) \PYGZhy{} 0.0634474791568939*@DURING(\PYGZdq{}2009\PYGZdq{}) \PYGZhy{} 0.3*(PAKFMLBLPOLYXN/100 \PYGZhy{} DLOG(PAKNECONPRVTXN))
\end{sphinxVerbatim}

\end{sphinxuseclass}\end{sphinxVerbatimOutput}

\end{sphinxuseclass}

\paragraph{The \sphinxstyleliteralintitle{\sphinxupquote{.frml}} method}
\label{\detokenize{content/05_WBModels/LoadingWBModel:the-frml-method}}
\sphinxAtStartPar
The \sphinxcode{\sphinxupquote{.frml}} method returns the normalized equation that is actually used in modelflow.  In this instance it is not called for the results of a search operation but by referencing directly the equation (which is itself a property of the mpak model).

\sphinxAtStartPar
Note that following the normalized equation is a listing of all the dependent variables of the equation.

\begin{sphinxuseclass}{cell}\begin{sphinxVerbatimInput}

\begin{sphinxuseclass}{cell_input}
\begin{sphinxVerbatim}[commandchars=\\\{\}]
\PYG{n}{mpak}\PYG{o}{.}\PYG{n}{PAKNECONPRVTKN}\PYG{o}{.}\PYG{n}{frml}
\end{sphinxVerbatim}

\end{sphinxuseclass}\end{sphinxVerbatimInput}
\begin{sphinxVerbatimOutput}

\begin{sphinxuseclass}{cell_output}
\begin{sphinxVerbatim}[commandchars=\\\{\}]
Endogeneous: PAKNECONPRVTKN: HH. Cons Real
Formular: FRML \PYGZlt{}DAMP,STOC\PYGZgt{} PAKNECONPRVTKN = (PAKNECONPRVTKN(\PYGZhy{}1)*EXP(PAKNECONPRVTKN\PYGZus{}A+ (\PYGZhy{}0.2*(LOG(PAKNECONPRVTKN(\PYGZhy{}1))\PYGZhy{}LOG(1.21203101101442)\PYGZhy{}LOG((((PAKBXFSTREMTCD(\PYGZhy{}1)\PYGZhy{}PAKBMFSTREMTCD(\PYGZhy{}1))*PAKPANUSATLS(\PYGZhy{}1))+PAKGGEXPTRNSCN(\PYGZhy{}1)+PAKNYYWBTOTLCN(\PYGZhy{}1)*(1\PYGZhy{}PAKGGREVDRCTXN(\PYGZhy{}1)/100))/PAKNECONPRVTXN(\PYGZhy{}1)))+0.763938860758873*((LOG((((PAKBXFSTREMTCD\PYGZhy{}PAKBMFSTREMTCD)*PAKPANUSATLS)+PAKGGEXPTRNSCN+PAKNYYWBTOTLCN*(1\PYGZhy{}PAKGGREVDRCTXN/100))/PAKNECONPRVTXN))\PYGZhy{}(LOG((((PAKBXFSTREMTCD(\PYGZhy{}1)\PYGZhy{}PAKBMFSTREMTCD(\PYGZhy{}1))*PAKPANUSATLS(\PYGZhy{}1))+PAKGGEXPTRNSCN(\PYGZhy{}1)+PAKNYYWBTOTLCN(\PYGZhy{}1)*(1\PYGZhy{}PAKGGREVDRCTXN(\PYGZhy{}1)/100))/PAKNECONPRVTXN(\PYGZhy{}1))))\PYGZhy{}0.0634474791568939*DURING\PYGZus{}2009\PYGZhy{}0.3*(PAKFMLBLPOLYXN/100\PYGZhy{}((LOG(PAKNECONPRVTXN))\PYGZhy{}(LOG(PAKNECONPRVTXN(\PYGZhy{}1)))))) )) * (1\PYGZhy{}PAKNECONPRVTKN\PYGZus{}D)+ PAKNECONPRVTKN\PYGZus{}X*PAKNECONPRVTKN\PYGZus{}D  \PYGZdl{}

PAKNECONPRVTKN  : HH. Cons Real
DURING\PYGZus{}2009     : 
PAKBMFSTREMTCD  : Imp., Remittances (BOP), US\PYGZdl{} mn
PAKBXFSTREMTCD  : Exp., Remittances (BOP), US\PYGZdl{} mn
PAKFMLBLPOLYXN  : Key Policy Interest Rate
PAKGGEXPTRNSCN  : Current Transfers
PAKGGREVDRCTXN  : Direct Revenue Tax Rate
PAKNECONPRVTKN\PYGZus{}A: Add factor:HH. Cons Real
PAKNECONPRVTKN\PYGZus{}D: Fix dummy:HH. Cons Real
PAKNECONPRVTKN\PYGZus{}X: Fix value:HH. Cons Real
PAKNECONPRVTXN  : Implicit LCU defl., Pvt. Cons., 2000 = 1
PAKNYYWBTOTLCN  : Total Wage Bill
PAKPANUSATLS    : Exchange rate LCU / US\PYGZdl{} \PYGZhy{} Pakistan
\end{sphinxVerbatim}

\end{sphinxuseclass}\end{sphinxVerbatimOutput}

\end{sphinxuseclass}

\paragraph{The \sphinxstyleliteralintitle{\sphinxupquote{.show}} method}
\label{\detokenize{content/05_WBModels/LoadingWBModel:the-show-method}}
\sphinxAtStartPar
The \sphinxcode{\sphinxupquote{.show}} method returns:
\begin{enumerate}
\sphinxsetlistlabels{\arabic}{enumi}{enumii}{}{.}%
\item {} 
\sphinxAtStartPar
The description of the variable

\item {} 
\sphinxAtStartPar
The normalized equation that is actually used in modelflow.

\item {} 
\sphinxAtStartPar
A listing of the mnemonics and descriptions of the RHS variables

\item {} 
\sphinxAtStartPar
The data of that variable (drawn from the \sphinxcode{\sphinxupquote{basedf}} and \sphinxcode{\sphinxupquote{.lastdf}} DataFrames in the model object as well as the data of the RHS variables of the equation from both the \sphinxcode{\sphinxupquote{basedf}} and \sphinxcode{\sphinxupquote{.lastdf}} DataFrames.

\end{enumerate}

\begin{sphinxuseclass}{cell}\begin{sphinxVerbatimInput}

\begin{sphinxuseclass}{cell_input}
\begin{sphinxVerbatim}[commandchars=\\\{\}]
\PYG{n}{mpak}\PYG{o}{.}\PYG{n}{smpl}\PYG{p}{(}\PYG{l+m+mi}{2020}\PYG{p}{,}\PYG{l+m+mi}{2025}\PYG{p}{)} \PYG{c+c1}{\PYGZsh{}change the actual sample range to limit the number of columns displayed}
\PYG{n}{mpak}\PYG{o}{.}\PYG{n}{PAKNECONPRVTKN}\PYG{o}{.}\PYG{n}{show}
\end{sphinxVerbatim}

\end{sphinxuseclass}\end{sphinxVerbatimInput}
\begin{sphinxVerbatimOutput}

\begin{sphinxuseclass}{cell_output}
\begin{sphinxVerbatim}[commandchars=\\\{\}]
Endogeneous: PAKNECONPRVTKN: HH. Cons Real
Formular: FRML \PYGZlt{}DAMP,STOC\PYGZgt{} PAKNECONPRVTKN = (PAKNECONPRVTKN(\PYGZhy{}1)*EXP(PAKNECONPRVTKN\PYGZus{}A+ (\PYGZhy{}0.2*(LOG(PAKNECONPRVTKN(\PYGZhy{}1))\PYGZhy{}LOG(1.21203101101442)\PYGZhy{}LOG((((PAKBXFSTREMTCD(\PYGZhy{}1)\PYGZhy{}PAKBMFSTREMTCD(\PYGZhy{}1))*PAKPANUSATLS(\PYGZhy{}1))+PAKGGEXPTRNSCN(\PYGZhy{}1)+PAKNYYWBTOTLCN(\PYGZhy{}1)*(1\PYGZhy{}PAKGGREVDRCTXN(\PYGZhy{}1)/100))/PAKNECONPRVTXN(\PYGZhy{}1)))+0.763938860758873*((LOG((((PAKBXFSTREMTCD\PYGZhy{}PAKBMFSTREMTCD)*PAKPANUSATLS)+PAKGGEXPTRNSCN+PAKNYYWBTOTLCN*(1\PYGZhy{}PAKGGREVDRCTXN/100))/PAKNECONPRVTXN))\PYGZhy{}(LOG((((PAKBXFSTREMTCD(\PYGZhy{}1)\PYGZhy{}PAKBMFSTREMTCD(\PYGZhy{}1))*PAKPANUSATLS(\PYGZhy{}1))+PAKGGEXPTRNSCN(\PYGZhy{}1)+PAKNYYWBTOTLCN(\PYGZhy{}1)*(1\PYGZhy{}PAKGGREVDRCTXN(\PYGZhy{}1)/100))/PAKNECONPRVTXN(\PYGZhy{}1))))\PYGZhy{}0.0634474791568939*DURING\PYGZus{}2009\PYGZhy{}0.3*(PAKFMLBLPOLYXN/100\PYGZhy{}((LOG(PAKNECONPRVTXN))\PYGZhy{}(LOG(PAKNECONPRVTXN(\PYGZhy{}1)))))) )) * (1\PYGZhy{}PAKNECONPRVTKN\PYGZus{}D)+ PAKNECONPRVTKN\PYGZus{}X*PAKNECONPRVTKN\PYGZus{}D  \PYGZdl{}

PAKNECONPRVTKN  : HH. Cons Real
DURING\PYGZus{}2009     : 
PAKBMFSTREMTCD  : Imp., Remittances (BOP), US\PYGZdl{} mn
PAKBXFSTREMTCD  : Exp., Remittances (BOP), US\PYGZdl{} mn
PAKFMLBLPOLYXN  : Key Policy Interest Rate
PAKGGEXPTRNSCN  : Current Transfers
PAKGGREVDRCTXN  : Direct Revenue Tax Rate
PAKNECONPRVTKN\PYGZus{}A: Add factor:HH. Cons Real
PAKNECONPRVTKN\PYGZus{}D: Fix dummy:HH. Cons Real
PAKNECONPRVTKN\PYGZus{}X: Fix value:HH. Cons Real
PAKNECONPRVTXN  : Implicit LCU defl., Pvt. Cons., 2000 = 1
PAKNYYWBTOTLCN  : Total Wage Bill
PAKPANUSATLS    : Exchange rate LCU / US\PYGZdl{} \PYGZhy{} Pakistan

Values :
\end{sphinxVerbatim}

\begin{sphinxVerbatim}[commandchars=\\\{\}]
\PYGZlt{}IPython.core.display.HTML object\PYGZgt{}
\end{sphinxVerbatim}

\begin{sphinxVerbatim}[commandchars=\\\{\}]
Input last run:
\end{sphinxVerbatim}

\begin{sphinxVerbatim}[commandchars=\\\{\}]
\PYGZlt{}pandas.io.formats.style.Styler at 0x1803d328f10\PYGZgt{}
\end{sphinxVerbatim}

\begin{sphinxVerbatim}[commandchars=\\\{\}]
Input base run:
\end{sphinxVerbatim}

\begin{sphinxVerbatim}[commandchars=\\\{\}]
\PYGZlt{}pandas.io.formats.style.Styler at 0x1803d329f00\PYGZgt{}
\end{sphinxVerbatim}

\begin{sphinxVerbatim}[commandchars=\\\{\}]
Difference for input variables
\end{sphinxVerbatim}

\begin{sphinxVerbatim}[commandchars=\\\{\}]
\PYGZlt{}pandas.io.formats.style.Styler at 0x1803db6c730\PYGZgt{}
\end{sphinxVerbatim}

\begin{sphinxVerbatim}[commandchars=\\\{\}]

\end{sphinxVerbatim}

\end{sphinxuseclass}\end{sphinxVerbatimOutput}

\end{sphinxuseclass}

\section{Behavioural equations in the MFMod framework}
\label{\detokenize{content/05_WBModels/LoadingWBModel:behavioural-equations-in-the-mfmod-framework}}
\sphinxAtStartPar
Recall a behavioural equation determines the value of an endogenous variable. For many of the variables in Wold Bank models, behavioural functions are estimated using an Error Correction Framework that splits the equation into a theoretically determined long run component and a more idiosyncratic short\sphinxhyphen{}run component.

\sphinxAtStartPar
Looking at the eviews representation of the consumption function:

\sphinxAtStartPar
\sphinxcode{\sphinxupquote{DLOG(PAKNECONPRVTKN) =\sphinxhyphen{} 0.2*(LOG(PAKNECONPRVTKN( \sphinxhyphen{} 1)) \sphinxhyphen{} LOG(1.21203101101442) \sphinxhyphen{} LOG((((PAKBXFSTREMTCD( \sphinxhyphen{} 1) \sphinxhyphen{} PAKBMFSTREMTCD( \sphinxhyphen{} 1))*PAKPANUSATLS( \sphinxhyphen{} 1)) + PAKGGEXPTRNSCN( \sphinxhyphen{} 1) + PAKNYYWBTOTLCN( \sphinxhyphen{} 1)*(1 \sphinxhyphen{} PAKGGREVDRCTXN( \sphinxhyphen{} 1)/100))/PAKNECONPRVTXN( \sphinxhyphen{} 1))) + 0.763938860758873*DLOG((((PAKBXFSTREMTCD \sphinxhyphen{} PAKBMFSTREMTCD)*PAKPANUSATLS) + PAKGGEXPTRNSCN + PAKNYYWBTOTLCN*(1 \sphinxhyphen{} PAKGGREVDRCTXN/100))/PAKNECONPRVTXN) \sphinxhyphen{} 0.0634474791568939*@DURING("2009") \sphinxhyphen{} 0.3*(PAKFMLBLPOLYXN/100 \sphinxhyphen{} DLOG(PAKNECONPRVTXN))}}

\sphinxAtStartPar
Below the mnemonics are simplified to ease reading of the equation using:


\begin{savenotes}\sphinxattablestart
\centering
\begin{tabulary}{\linewidth}[t]{|T|T|T|}
\hline
\sphinxstyletheadfamily 
\sphinxAtStartPar
Model Mnemonic
&\sphinxstyletheadfamily 
\sphinxAtStartPar
Simplified
&\sphinxstyletheadfamily 
\sphinxAtStartPar
Meaning
\\
\hline
\sphinxAtStartPar
PAKNECONPRVTKN
&
\sphinxAtStartPar
\(CON^{KN}_t\)
&
\sphinxAtStartPar
Household Consumption
\\
\hline
\sphinxAtStartPar
(PAKBXFSTREMTCD \sphinxhyphen{} PAKBMFSTREMTCD)*PAKPANUSATLS
&
\sphinxAtStartPar
\(Remit^{net}_t\)
&
\sphinxAtStartPar
Net remittances inflows in LCU
\\
\hline
\sphinxAtStartPar
PAKGGEXPTRNSCN
&
\sphinxAtStartPar
\(TRANSF^{hhld}_t\)
&
\sphinxAtStartPar
Government transfers to households
\\
\hline
\sphinxAtStartPar
DURING\_2010
&
\sphinxAtStartPar
\(D^{2010}_t\)
&
\sphinxAtStartPar
A dummy
\\
\hline
\sphinxAtStartPar
PAKFMLBLPOLYXN
&
\sphinxAtStartPar
\(r^{policy}_t\)
&
\sphinxAtStartPar
Policy Rate
\\
\hline
\sphinxAtStartPar
PAKGGREVDRCTXN
&
\sphinxAtStartPar
\(DirectTxR_t\)
&
\sphinxAtStartPar
Direct Taxes: Effective rate
\\
\hline
\sphinxAtStartPar
PAKNECONPRVTKN\_A
&
\sphinxAtStartPar
\(CON^{KN_AF}_t\)
&
\sphinxAtStartPar
Add factor:Household Consumption
\\
\hline
\sphinxAtStartPar
PAKNECONPRVTXN
&
\sphinxAtStartPar
\(CON^{XN}_t\)
&
\sphinxAtStartPar
Household Consumption Deflator
\\
\hline
\sphinxAtStartPar
PAKNYYWBTOTLCN
&
\sphinxAtStartPar
\(WAGEBILL^{CN}_t\)
&
\sphinxAtStartPar
Economy\sphinxhyphen{}wide wage bill
\\
\hline
\end{tabulary}
\par
\sphinxattableend\end{savenotes}

\sphinxAtStartPar
With those substitutions the equation can be rewritten as:
\begin{align*}
\Delta log(CON^{KN}_t) = &-0.2*\bigg[LOG(CON^{KN}_{t-1})-LOG\bigg({\frac{(Remit^{net}_{t-1}+WAGEBILL^{CN}_{t-1}+TRANSF^{hhld}_{t-1})*(1-DirectTxR_{t-1}/100)}{CON^{XN}_{t-1}}}\bigg)\bigg]  \\
&+0.76*\Delta log \bigg({\frac{(Remit^{net}_{t}+WAGEBILL^{CN}_{t}+TRANSF^{hhld}_{t})*(1-DirectTxR_{t}/100)}{CON^{XN}_{t}}}\bigg)  \\
&+0.030 + 0.016*D^{2010}_t-0.3*\bigg(r^{policy}_t/100-\Delta log(CON^{XN}_{t})\bigg) -CON^{KN_AF}_t
\end{align*}
\sphinxAtStartPar
Where in this instance the short\sphinxhyphen{}run elasticity of consumption to disposable income is .76 , and the short run elasticity of consumption to the real interest rate is 0.3.


\subsection{The ECM specification}
\label{\detokenize{content/05_WBModels/LoadingWBModel:the-ecm-specification}}
\sphinxAtStartPar
 Pretty sure this repeats and earlier section.  Delete one 

\sphinxAtStartPar
The ECM approach used in World Bank models is described in {[}\hyperlink{cite.content/99_BackMatter/References:id16}{Wickens and Breusch, 1988}{]}, and addresses the above challenge by modelling both the long run relationship and the short run short run behaviour and brings them together into one equation.

\sphinxAtStartPar
The ECM specification is therefore a single equation comprised of two parts (the long run relationship, and the short\sphinxhyphen{}run relationship).

\sphinxAtStartPar
Consider as an example two variables say consumption and disposable income.  Both have an underlying trend or in the parlance are co\sphinxhyphen{}integrated to degree 1.  For simplicity we call them y an x.


\subsubsection{The short run relationship}
\label{\detokenize{content/05_WBModels/LoadingWBModel:the-short-run-relationship}}
\sphinxAtStartPar
In its simplest form we might have a short run relationship between the growth rates of our two variables such that:
\begin{equation*}
\begin{split}\Delta ln(Y_t) = \alpha + \beta \Delta ln(X_t) +\epsilon_t\end{split}
\end{equation*}
\sphinxAtStartPar
or substituting lower case letters for the logged values.
\begin{equation*}
\begin{split}\Delta y_t = \alpha + \beta \Delta x_t +\epsilon_t\end{split}
\end{equation*}

\subsubsection{The long run equation}
\label{\detokenize{content/05_WBModels/LoadingWBModel:the-long-run-equation}}
\sphinxAtStartPar
The long run relates the level of the two (or more) variables.  A simplified version of that equation can be written as:
\begin{equation*}
\begin{split}Y_t=αX_t^β+ \eta_t\end{split}
\end{equation*}
\sphinxAtStartPar
Rewriting this (in logarithms) it can be expressed as:
\begin{equation*}
\begin{split}y_t = ln⁡(α) + βx_t + \eta_t\end{split}
\end{equation*}

\subsection{The long run equation in the steady state}
\label{\detokenize{content/05_WBModels/LoadingWBModel:the-long-run-equation-in-the-steady-state}}
\sphinxAtStartPar
Note that in the steady state the expected value of the error term in the long run equation is zero (\(\eta_t=0 \)) so in those conditions the long run relationship can be simplified to:
\begin{equation*}
\begin{split}y_t=ln⁡(α)+\beta x_t\end{split}
\end{equation*}
\sphinxAtStartPar
or equivalently (substituting A for the log of \(\alpha\)).
\begin{equation*}
\begin{split}y_t-A-βx_t=0\end{split}
\end{equation*}
\sphinxAtStartPar
Moreover if this expression is multiplied by some arbitrary constant, say \(-\lambda\), it would still equal zero.
\begin{equation*}
\begin{split}-\lambda(y_t -A-βx_t)\end{split}
\end{equation*}
\sphinxAtStartPar
and in the steady state this will also be true for the lagged variables
\begin{equation*}
\begin{split}-\lambda(y_{t-1}- A - βx_{t-1})\end{split}
\end{equation*}

\section{Putting it together}
\label{\detokenize{content/05_WBModels/LoadingWBModel:putting-it-together}}
\sphinxAtStartPar
From before we have the short run equation:
\begin{equation*}
\begin{split}\Delta y_t = \alpha + \beta \Delta x_t +\epsilon_t\end{split}
\end{equation*}
\sphinxAtStartPar
Inserting the steady state expression for the long\sphinxhyphen{}run into the short run equation makes no difference (in the long run) because in the long run it is equal to zero.
\begin{equation*}
\begin{split}\Delta y_t = -\lambda(y_{t-1}-A-βx_{t-1})  + \alpha + \beta \Delta x_t +\epsilon_t\end{split}
\end{equation*}
\sphinxAtStartPar
When the model is not in the steady state the expression \(y_{t-1}-A-βx_{t-1}\) is of course the error term from the long run equation (a measure of how far the dependent variable is from equilibrium).


\subsection{Lambda, the speed of adjustment}
\label{\detokenize{content/05_WBModels/LoadingWBModel:lambda-the-speed-of-adjustment}}
\sphinxAtStartPar
The parameter \(\lambda\) can be interpreted as the speed of adjustment.  As long as \(\lambda\) is greater than zero and less or equal to one if there are no further disturbances ( \(\epsilon_t=0\)) the expression multiplied by lambda will slowly decline toward zero. How fast depends on how large or small is \(\lambda\).

\sphinxAtStartPar
To be convergent \(\lambda\) must be between 0 and 2, if its is negative or greater than one, then the long run portion of the equation will cause the disequilibrium to grow each period (\(\lambda\) >1) not diminish or if  (\(\lambda\) >1<2) output will oscillate from positive to negative (\(\lambda <0\)) but will slowly converge.

\sphinxAtStartPar
Intuitively, the long\sphinxhyphen{}run error\sphinxhyphen{}term measures how far the model was from equilibrium one period earlier (at t\sphinxhyphen{}1). The ECM term (multiplied by \(\lambda\) ensures the model will slowly converge to equilibrium – the point at which the long run equation holds exactly. If \(\lambda\) is greater than zero but less than one (or equal to one) some portion of the previous period year’s disequilibrium will be absorbed each year. How much is absorbed depends on the size of estimated speed of the adjustment coefficient \(\lambda\). 

\sphinxAtStartPar
An ECM equation can, therefore be broken into its component parts.  For the consumption function it will look something like this:
\begin{equation*}
\begin{split}\Delta c_t = -\lambda (\underbrace{
        log(C_{t-1})-log(Wages_{t-1}-Taxes_{t-1}+Transfers_{t-1}) -log(\alpha))}  _\text{Long run}
+\beta \underbrace{\Delta x_t}_\text{short run}\end{split}
\end{equation*}
\sphinxstepscope

\begin{sphinxuseclass}{cell}\begin{sphinxVerbatimInput}

\begin{sphinxuseclass}{cell_input}
\begin{sphinxVerbatim}[commandchars=\\\{\}]
\PYG{o}{\PYGZpc{}}\PYG{k}{matplotlib} inline
\end{sphinxVerbatim}

\end{sphinxuseclass}\end{sphinxVerbatimInput}

\end{sphinxuseclass}

\chapter{Scenario analysis}
\label{\detokenize{content/05_WBModels/ScenarioAnalysis:scenario-analysis}}\label{\detokenize{content/05_WBModels/ScenarioAnalysis::doc}}
\sphinxAtStartPar
An essential feature of a model is that when given a specific set of inputs (the exogenous variables to the model) it will always return the same results.

\sphinxAtStartPar
Below a new \sphinxcode{\sphinxupquote{ModelFlow}} sessionis prepared, initializing a pandas session and importing and solving a saved WBG model (NB: these are precisely the same commands they used to start the previous chapter) and would form the essential initialization commands of any python session using \sphinxcode{\sphinxupquote{ModelFlow}}.

\begin{sphinxuseclass}{cell}\begin{sphinxVerbatimInput}

\begin{sphinxuseclass}{cell_input}
\begin{sphinxVerbatim}[commandchars=\\\{\}]
\PYG{c+c1}{\PYGZsh{} import the model class from modelflow package}
\PYG{k+kn}{from} \PYG{n+nn}{modelclass} \PYG{k+kn}{import} \PYG{n}{model} 
\PYG{k+kn}{import} \PYG{n+nn}{modelmf}       \PYG{c+c1}{\PYGZsh{} Add useful features to pandas dataframes }
                     \PYG{c+c1}{\PYGZsh{} using utlities initially developed for modelflow}

\PYG{n}{model}\PYG{o}{.}\PYG{n}{widescreen}\PYG{p}{(}\PYG{p}{)}   \PYG{c+c1}{\PYGZsh{} These modelflow commands ensure that outputs from modelflow play well with Jupyter Notebook}
\PYG{n}{model}\PYG{o}{.}\PYG{n}{scroll\PYGZus{}off}\PYG{p}{(}\PYG{p}{)}

\PYG{o}{\PYGZpc{}}\PYG{k}{load\PYGZus{}ext} autoreload   
\PYG{o}{\PYGZpc{}}\PYG{k}{autoreload} 2


\PYG{c+c1}{\PYGZsh{}Load a saved version of the Pakistan model and solve it, }
\PYG{c+c1}{\PYGZsh{}saving the results in the model object mpak, and the resulting dataframe in bline}

\PYG{c+c1}{\PYGZsh{}Replace the path below with the location of the pak.pcim file on your computer}
\PYG{n}{mpak}\PYG{p}{,}\PYG{n}{bline} \PYG{o}{=} \PYG{n}{model}\PYG{o}{.}\PYG{n}{modelload}\PYG{p}{(}\PYG{l+s+s1}{\PYGZsq{}}\PYG{l+s+s1}{..}\PYG{l+s+s1}{\PYGZbs{}}\PYG{l+s+s1}{models}\PYG{l+s+s1}{\PYGZbs{}}\PYG{l+s+s1}{pak.pcim}\PYG{l+s+s1}{\PYGZsq{}}\PYG{p}{,} \PYGZbs{}
                                \PYG{n}{alfa}\PYG{o}{=}\PYG{l+m+mf}{0.7}\PYG{p}{,}\PYG{n}{run}\PYG{o}{=}\PYG{l+m+mi}{1}\PYG{p}{,}\PYG{n}{keep}\PYG{o}{=} \PYG{l+s+s1}{\PYGZsq{}}\PYG{l+s+s1}{Baseline}\PYG{l+s+s1}{\PYGZsq{}}\PYG{p}{)}
\end{sphinxVerbatim}

\end{sphinxuseclass}\end{sphinxVerbatimInput}
\begin{sphinxVerbatimOutput}

\begin{sphinxuseclass}{cell_output}
\begin{sphinxVerbatim}[commandchars=\\\{\}]
\PYGZlt{}IPython.core.display.HTML object\PYGZgt{}
\end{sphinxVerbatim}

\begin{sphinxVerbatim}[commandchars=\\\{\}]
file read:  C:\PYGZbs{}modelflow manual\PYGZbs{}papers\PYGZbs{}mfbook\PYGZbs{}content\PYGZbs{}models\PYGZbs{}pak.pcim
\end{sphinxVerbatim}

\end{sphinxuseclass}\end{sphinxVerbatimOutput}

\end{sphinxuseclass}
\sphinxAtStartPar
As noted, when the model is solved without changing any inputs (as was the case of the load) the model should return (reproduce) exactly the same data as before{[}\textasciicircum{}fn2{]}.  To test this for \sphinxcode{\sphinxupquote{mpak}} the results from the simulation can be compared by using the \sphinxcode{\sphinxupquote{basedf}} and \sphinxcode{\sphinxupquote{lastdf}} DataFrames.

\sphinxAtStartPar
{[}\textasciicircum{}fn2:{]} If it does not, the model has violated he principle of reproducibility and there is something wrong (usually one of the identities does not hold).

\sphinxAtStartPar
Below, the percent difference between the values of the variables for real GDP and Consumer demand in the two \sphinxcode{\sphinxupquote{dataframes}} \sphinxcode{\sphinxupquote{.basedf}} and \sphinxcode{\sphinxupquote{lastdf}} is zero following a simulation where the inputs were not changed – confirming the reproduction of results.

\begin{sphinxuseclass}{cell}\begin{sphinxVerbatimInput}

\begin{sphinxuseclass}{cell_input}
\begin{sphinxVerbatim}[commandchars=\\\{\}]
\PYG{c+c1}{\PYGZsh{} Need statement to change the default format}
\PYG{n}{mpak}\PYG{o}{.}\PYG{n}{smpl}\PYG{p}{(}\PYG{l+m+mi}{2020}\PYG{p}{,}\PYG{l+m+mi}{2030}\PYG{p}{)}
\PYG{n}{mpak}\PYG{p}{[}\PYG{l+s+s1}{\PYGZsq{}}\PYG{l+s+s1}{PAKNYGDPMKTPKN PAKNECONPRVTKN}\PYG{l+s+s1}{\PYGZsq{}}\PYG{p}{]}\PYG{o}{.}\PYG{n}{difpctlevel}\PYG{o}{.}\PYG{n}{mul100}\PYG{o}{.}\PYG{n}{df}
\end{sphinxVerbatim}

\end{sphinxuseclass}\end{sphinxVerbatimInput}
\begin{sphinxVerbatimOutput}

\begin{sphinxuseclass}{cell_output}
\begin{sphinxVerbatim}[commandchars=\\\{\}]
      PAKNYGDPMKTPKN  PAKNECONPRVTKN
2020             0.0             0.0
2021             0.0             0.0
2022             0.0             0.0
2023             0.0             0.0
2024             0.0             0.0
2025             0.0             0.0
2026             0.0             0.0
2027             0.0             0.0
2028             0.0             0.0
2029             0.0             0.0
2030             0.0             0.0
\end{sphinxVerbatim}

\end{sphinxuseclass}\end{sphinxVerbatimOutput}

\end{sphinxuseclass}

\section{Different kinds of simulations}
\label{\detokenize{content/05_WBModels/ScenarioAnalysis:different-kinds-of-simulations}}
\sphinxAtStartPar
The \sphinxcode{\sphinxupquote{modelflow}} package performs 4 different kinds of simulation:
\begin{enumerate}
\sphinxsetlistlabels{\arabic}{enumi}{enumii}{}{.}%
\item {} 
\sphinxAtStartPar
A shock to an exogenous variable in the model

\item {} 
\sphinxAtStartPar
An exogenous shock of a behavioural variable, executed by exogenizing the variable

\item {} 
\sphinxAtStartPar
An endogenous shock of a behavioural variable, executed by shocking the add factor of the variable.

\item {} 
\sphinxAtStartPar
A mixed shock of a behavioural variable, achieved by temporarily exogenixing the variable.

\end{enumerate}

\sphinxAtStartPar
Although technically modelflow would allow us to shock identities, that would violate their nature as accounting rules. \sphinxstylestrong{Effectively such a shock would break the economic sense of the model.}

\sphinxAtStartPar
As a result, this we possibility is not discussed.


\subsection{A shock to an exogenous variable}
\label{\detokenize{content/05_WBModels/ScenarioAnalysis:a-shock-to-an-exogenous-variable}}
\sphinxAtStartPar
A World Bank model will reproduce the same values if inputs (exogenous variables) are not changed.  In the simulation below, the oil price is changed – increasing by \$25 for the three years between 2025 and 2027 inclusive.

\sphinxAtStartPar
As a first step a new input dataframe is created as a copy of the original and then the oil price in that data frame is modified using the \sphinxcode{\sphinxupquote{mfcalc}} method to change the value for the three years in question.

\sphinxAtStartPar
Finally pandas math is used to display the initial value, the changed value and the difference between the two, confirming that the mfcalc statement revised the oil price data.

\begin{sphinxuseclass}{cell}\begin{sphinxVerbatimInput}

\begin{sphinxuseclass}{cell_input}
\begin{sphinxVerbatim}[commandchars=\\\{\}]
\PYG{c+c1}{\PYGZsh{}Make a copy of the baseline dataframe}
\PYG{n}{oilshockdf}\PYG{o}{=}\PYG{n}{mpak}\PYG{o}{.}\PYG{n}{basedf}
\PYG{n}{oilshockdf}\PYG{o}{=}\PYG{n}{oilshockdf}\PYG{o}{.}\PYG{n}{mfcalc}\PYG{p}{(}\PYG{l+s+s2}{\PYGZdq{}}\PYG{l+s+s2}{\PYGZlt{}2025 2027\PYGZgt{} WLDFCRUDE\PYGZus{}PETRO = WLDFCRUDE\PYGZus{}PETRO +25}\PYG{l+s+s2}{\PYGZdq{}}\PYG{p}{)}

\PYG{n}{compdf}\PYG{o}{=}\PYG{n}{mpak}\PYG{o}{.}\PYG{n}{basedf}\PYG{o}{.}\PYG{n}{loc}\PYG{p}{[}\PYG{l+m+mi}{2000}\PYG{p}{:}\PYG{l+m+mi}{2030}\PYG{p}{,}\PYG{p}{[}\PYG{l+s+s1}{\PYGZsq{}}\PYG{l+s+s1}{WLDFCRUDE\PYGZus{}PETRO}\PYG{l+s+s1}{\PYGZsq{}}\PYG{p}{]}\PYG{p}{]}
\PYG{n}{compdf}\PYG{p}{[}\PYG{l+s+s1}{\PYGZsq{}}\PYG{l+s+s1}{LASTDF}\PYG{l+s+s1}{\PYGZsq{}}\PYG{p}{]}\PYG{o}{=}\PYG{n}{oilshockdf}\PYG{o}{.}\PYG{n}{loc}\PYG{p}{[}\PYG{l+m+mi}{2000}\PYG{p}{:}\PYG{l+m+mi}{2030}\PYG{p}{,}\PYG{p}{[}\PYG{l+s+s1}{\PYGZsq{}}\PYG{l+s+s1}{WLDFCRUDE\PYGZus{}PETRO}\PYG{l+s+s1}{\PYGZsq{}}\PYG{p}{]}\PYG{p}{]}
\PYG{n}{compdf}\PYG{p}{[}\PYG{l+s+s1}{\PYGZsq{}}\PYG{l+s+s1}{Dif}\PYG{l+s+s1}{\PYGZsq{}}\PYG{p}{]}\PYG{o}{=}\PYG{n}{compdf}\PYG{p}{[}\PYG{l+s+s1}{\PYGZsq{}}\PYG{l+s+s1}{LASTDF}\PYG{l+s+s1}{\PYGZsq{}}\PYG{p}{]}\PYG{o}{\PYGZhy{}}\PYG{n}{compdf}\PYG{p}{[}\PYG{l+s+s1}{\PYGZsq{}}\PYG{l+s+s1}{WLDFCRUDE\PYGZus{}PETRO}\PYG{l+s+s1}{\PYGZsq{}}\PYG{p}{]}

\PYG{n}{compdf}\PYG{o}{.}\PYG{n}{loc}\PYG{p}{[}\PYG{l+m+mi}{2024}\PYG{p}{:}\PYG{l+m+mi}{2030}\PYG{p}{]}
\end{sphinxVerbatim}

\end{sphinxuseclass}\end{sphinxVerbatimInput}
\begin{sphinxVerbatimOutput}

\begin{sphinxuseclass}{cell_output}
\begin{sphinxVerbatim}[commandchars=\\\{\}]
      WLDFCRUDE\PYGZus{}PETRO      LASTDF   Dif
2024        80.367180   80.367180   0.0
2025        85.336809  110.336809  25.0
2026        90.613742  115.613742  25.0
2027        96.216983  121.216983  25.0
2028       102.166709  102.166709   0.0
2029       108.484346  108.484346   0.0
2030       115.192643  115.192643   0.0
\end{sphinxVerbatim}

\end{sphinxuseclass}\end{sphinxVerbatimOutput}

\end{sphinxuseclass}

\subsubsection{Running the simulation}
\label{\detokenize{content/05_WBModels/ScenarioAnalysis:running-the-simulation}}
\sphinxAtStartPar
Having created a new dataframe comprised of all the old data plus the changed data for the oil price, a simulation can now be run.

\sphinxAtStartPar
In the command below, the simulation is run from 2020 to 2040, using the \sphinxcode{\sphinxupquote{oilshockdf}} as the input \sphinxcode{\sphinxupquote{DataFrame}}.  The results of the simulation are assigned to a new \sphinxcode{\sphinxupquote{DataFrame}}  named \sphinxcode{\sphinxupquote{ExogOilSimul}}.  The \sphinxcode{\sphinxupquote{Keep}} command ensures that the mpak model object stores (keeps) a copy of the results identified by the text name \sphinxstyleemphasis{‘\$25 increase in oil prices 2025\sphinxhyphen{}27’}.

\begin{sphinxuseclass}{cell}\begin{sphinxVerbatimInput}

\begin{sphinxuseclass}{cell_input}
\begin{sphinxVerbatim}[commandchars=\\\{\}]
\PYG{c+c1}{\PYGZsh{}Simulate the model }
\PYG{n}{ExogOilSimul} \PYG{o}{=} \PYG{n}{mpak}\PYG{p}{(}\PYG{n}{oilshockdf}\PYG{p}{,}\PYG{l+m+mi}{2020}\PYG{p}{,}\PYG{l+m+mi}{2040}\PYG{p}{,}\PYG{n}{keep}\PYG{o}{=}\PYG{l+s+s1}{\PYGZsq{}}\PYG{l+s+s1}{\PYGZdl{}25 increase in oil prices 2025\PYGZhy{}27}\PYG{l+s+s1}{\PYGZsq{}}\PYG{p}{)} 
\end{sphinxVerbatim}

\end{sphinxuseclass}\end{sphinxVerbatimInput}

\end{sphinxuseclass}

\paragraph{Results}
\label{\detokenize{content/05_WBModels/ScenarioAnalysis:results}}
\sphinxAtStartPar
\sphinxcode{\sphinxupquote{ModelFlow}} tools can be used to visualize the impacts of the shock; as a print out; as charts and within Jupyter notebook as an interactive widget.

\sphinxAtStartPar
The display below confirms that the shock was executed as desired. The \sphinxcode{\sphinxupquote{dif.df}} method returns the difference between the \sphinxcode{\sphinxupquote{.lastdf}} and \sphinxcode{\sphinxupquote{.basedf}} values of the selected variable(s) as a \sphinxcode{\sphinxupquote{DataFrame}}. The \sphinxcode{\sphinxupquote{with mpak.set\_smpl(2020,2030):}} clause temporarily restricts the sample period over which the following \sphinxstylestrong{indented} commands are executed.

\sphinxAtStartPar
Alternatively the \sphinxcode{\sphinxupquote{mpak.smpl(2020,2030)}}could be used. This would restricts the time period of over which \sphinxstylestrong{all} subsequent commands are executed.

\begin{sphinxuseclass}{cell}\begin{sphinxVerbatimInput}

\begin{sphinxuseclass}{cell_input}
\begin{sphinxVerbatim}[commandchars=\\\{\}]
\PYG{k}{with} \PYG{n}{mpak}\PYG{o}{.}\PYG{n}{set\PYGZus{}smpl}\PYG{p}{(}\PYG{l+m+mi}{2020}\PYG{p}{,}\PYG{l+m+mi}{2030}\PYG{p}{)}\PYG{p}{:}
    \PYG{n+nb}{print}\PYG{p}{(}\PYG{n}{mpak}\PYG{p}{[}\PYG{l+s+s1}{\PYGZsq{}}\PYG{l+s+s1}{WLDFCRUDE\PYGZus{}PETRO}\PYG{l+s+s1}{\PYGZsq{}}\PYG{p}{]}\PYG{o}{.}\PYG{n}{dif}\PYG{o}{.}\PYG{n}{df}\PYG{p}{)}\PYG{p}{;}
\end{sphinxVerbatim}

\end{sphinxuseclass}\end{sphinxVerbatimInput}
\begin{sphinxVerbatimOutput}

\begin{sphinxuseclass}{cell_output}
\begin{sphinxVerbatim}[commandchars=\\\{\}]
      WLDFCRUDE\PYGZus{}PETRO
2020              0.0
2021              0.0
2022              0.0
2023              0.0
2024              0.0
2025             25.0
2026             25.0
2027             25.0
2028              0.0
2029              0.0
2030              0.0
\end{sphinxVerbatim}

\end{sphinxuseclass}\end{sphinxVerbatimOutput}

\end{sphinxuseclass}
\sphinxAtStartPar
Below the impact of this change on a few variables are expressed graphically and in a table.

\sphinxAtStartPar
The first variable \sphinxcode{\sphinxupquote{PAKNYGDPMKTPKN}} is Pakistan’s real GDP, the second \sphinxcode{\sphinxupquote{PAKNECONPRVTKN}} is real consumption and the third is the Consumer price deflator \sphinxcode{\sphinxupquote{PAKNECONPRVTXN}}.

\begin{sphinxuseclass}{cell}\begin{sphinxVerbatimInput}

\begin{sphinxuseclass}{cell_input}
\begin{sphinxVerbatim}[commandchars=\\\{\}]
\PYG{n}{mpak}\PYG{p}{[}\PYG{l+s+s1}{\PYGZsq{}}\PYG{l+s+s1}{PAKNYGDPMKTPKN PAKNECONPRVTKN PAKNEIMPGNFSKN PAKNECONPRVTXN}\PYG{l+s+s1}{\PYGZsq{}}\PYG{p}{]}\PYG{o}{.}\PYG{n}{difpctlevel}\PYG{o}{.}\PYG{n}{mul100}\PYG{o}{.}\PYG{n}{plot}\PYG{p}{(}\PYG{n}{title}\PYG{o}{=}\PYG{l+s+s2}{\PYGZdq{}}\PYG{l+s+s2}{Impact of temporary \PYGZdl{}25 hike in oil prices}\PYG{l+s+s2}{\PYGZdq{}}\PYG{p}{)}
\end{sphinxVerbatim}

\end{sphinxuseclass}\end{sphinxVerbatimInput}
\begin{sphinxVerbatimOutput}

\begin{sphinxuseclass}{cell_output}
\noindent\sphinxincludegraphics{{0181516e63188095bdee3b6da520c3e1d00b512a013d07fb09504ec1fb7b23ac}.png}

\end{sphinxuseclass}\end{sphinxVerbatimOutput}

\end{sphinxuseclass}
\begin{sphinxuseclass}{cell}\begin{sphinxVerbatimInput}

\begin{sphinxuseclass}{cell_input}
\begin{sphinxVerbatim}[commandchars=\\\{\}]
\PYG{n+nb}{print}\PYG{p}{(}\PYG{n+nb}{round}\PYG{p}{(}\PYG{n}{mpak}\PYG{p}{[}\PYG{l+s+s1}{\PYGZsq{}}\PYG{l+s+s1}{PAKNYGDPMKTPKN PAKNECONPRVTKN PAKNEIMPGNFSKN PAKNECONPRVTXN}\PYG{l+s+s1}{\PYGZsq{}}\PYG{p}{]}\PYG{o}{.}\PYG{n}{difpctlevel}\PYG{o}{.}\PYG{n}{mul100}\PYG{o}{.}\PYG{n}{df}\PYG{p}{,}\PYG{l+m+mi}{2}\PYG{p}{)}\PYG{p}{)}
\end{sphinxVerbatim}

\end{sphinxuseclass}\end{sphinxVerbatimInput}
\begin{sphinxVerbatimOutput}

\begin{sphinxuseclass}{cell_output}
\begin{sphinxVerbatim}[commandchars=\\\{\}]
      PAKNYGDPMKTPKN  PAKNECONPRVTKN  PAKNEIMPGNFSKN  PAKNECONPRVTXN
2020            0.00            0.00            0.00            0.00
2021            0.00            0.00            0.00            0.00
2022            0.00            0.00            0.00            0.00
2023            0.00            0.00            0.00            0.00
2024            0.00            0.00            0.00            0.00
2025           \PYGZhy{}0.89           \PYGZhy{}1.32           \PYGZhy{}1.49            1.64
2026           \PYGZhy{}0.85           \PYGZhy{}1.48           \PYGZhy{}2.65            1.35
2027           \PYGZhy{}0.64           \PYGZhy{}1.37           \PYGZhy{}3.19            1.08
2028            0.34           \PYGZhy{}0.08           \PYGZhy{}2.17           \PYGZhy{}0.51
2029            0.50            0.20           \PYGZhy{}1.25           \PYGZhy{}0.43
2030            0.45            0.19           \PYGZhy{}0.80           \PYGZhy{}0.31
2031           \PYGZhy{}0.04            0.02           \PYGZhy{}0.10            0.26
2032           \PYGZhy{}0.06            0.01           \PYGZhy{}0.01            0.20
2033           \PYGZhy{}0.08            0.00            0.03            0.15
2034           \PYGZhy{}0.08            0.01            0.04            0.11
2035           \PYGZhy{}0.07            0.03            0.05            0.08
2036           \PYGZhy{}0.06            0.04            0.05            0.06
2037           \PYGZhy{}0.04            0.05            0.05            0.06
2038           \PYGZhy{}0.03            0.05            0.05            0.08
2039           \PYGZhy{}0.03            0.04            0.04            0.09
2040           \PYGZhy{}0.04            0.03            0.02            0.11
\end{sphinxVerbatim}

\end{sphinxuseclass}\end{sphinxVerbatimOutput}

\end{sphinxuseclass}
\sphinxAtStartPar
The graphs show the change in the level as a percent of the previous level. They suggest that a temporary \$25 oil price hike would reduce GDP in the first year by about 0.9 percent, that the impact would diminish by the third year to \sphinxhyphen{}.64 percent, and then turn positive in the fourth year when the price effect was eliminated.

\sphinxAtStartPar
The impacts on household consumption are stronger but follow a similar pattern.

\sphinxAtStartPar
The GDP impact is smaller partly because the decline in domestic demand reduces imports.  Because imports enter into the GDP identity with a negative sign. Therefore a reduction in imports actually increase aggregate GDP – or in this case partially offsets the declines coming from reduced consumption (and investment).

\sphinxAtStartPar
Finally as could be expected, initially prices rise sharply with higher oil prices. However, as the slow down in growth is felt, inflationary pressures turn negative and the overall impact on the price level turns negative.  The graph and table above shows what is happening to the \sphinxstylestrong{price level}. To see the impact on inflation (the rate of growth of prices), a separate graph can be generated using \sphinxcode{\sphinxupquote{difpct.mul100}}, which shows the change in the rate of growth of variables where the growth rate is expressed as a per cent \(\bigg[\bigg(\frac{x^{shock}_t}{x^{shock}_{t-1}}-1\bigg)\) \( - \bigg(\frac{x^{baseline}_t}{x^{baseline}_{t-1}}-1\bigg)\Bigg]*100\).

\begin{sphinxuseclass}{cell}\begin{sphinxVerbatimInput}

\begin{sphinxuseclass}{cell_input}
\begin{sphinxVerbatim}[commandchars=\\\{\}]
\PYG{n}{mpak}\PYG{p}{[}\PYG{l+s+s1}{\PYGZsq{}}\PYG{l+s+s1}{PAKNECONPRVTXN}\PYG{l+s+s1}{\PYGZsq{}}\PYG{p}{]}\PYG{o}{.}\PYG{n}{difpct}\PYG{o}{.}\PYG{n}{mul100}\PYG{o}{.}\PYG{n}{plot}\PYG{p}{(}\PYG{n}{title}\PYG{o}{=}\PYG{l+s+s2}{\PYGZdq{}}\PYG{l+s+s2}{Change in inflation from a temporary \PYGZdl{}25 hike in oil prices}\PYG{l+s+s2}{\PYGZdq{}}\PYG{p}{)}
\end{sphinxVerbatim}

\end{sphinxuseclass}\end{sphinxVerbatimInput}
\begin{sphinxVerbatimOutput}

\begin{sphinxuseclass}{cell_output}
\noindent\sphinxincludegraphics{{a16a1be534dd02da46933a3b12194e55750ad34807496e98e34c3941715360ba}.png}

\end{sphinxuseclass}\end{sphinxVerbatimOutput}

\end{sphinxuseclass}
\sphinxAtStartPar
 Ib how come this graph shows up so small.  How can we affect its size?

\sphinxAtStartPar
This view, gives a more nuanced result.  The inflation rate increases initially by about 1.2 percentage points, but falls compared with the baseline below in the 2026\sphinxhyphen{}2027 period as the as the influence of the slowdown in GDP more than offsets the continued inflationary impetus from the lagged increase in oil prices. In 2028, when oil prices drop back to their previous level, there is an additional dis\sphinxhyphen{}inflationary force and sharp drop in inflation as compared with the baseline. Overtime, the boost to demand from lower prices translates into an acceleration in growth and a return of inflation back to its trend rate.


\subsection{An exogenous shock to a Behavioural variable}
\label{\detokenize{content/05_WBModels/ScenarioAnalysis:an-exogenous-shock-to-a-behavioural-variable}}
\sphinxAtStartPar
Behavioural equations can be de\sphinxhyphen{}activated by exogenizing them, either for the entire simulation period, or for a selected sub period.  In this example, consumption is exogenized for the entire simulation period.

\sphinxAtStartPar
To motivate the simulation, it is assumed that a change in weather patterns has increased the number of sunny days by 10 percent. This increases households happiness and causes them to permanently increase their spending by 2.5\% beginning in 2025.

\sphinxAtStartPar
Such a shock can be specified either manually or by using the\sphinxcode{\sphinxupquote{.fix()}} method. Below the simpler \sphinxcode{\sphinxupquote{.fix()}} method is used, but the equivalent manual steps performed by \sphinxcode{\sphinxupquote{.fix()}} are also explained.

\sphinxAtStartPar
To exogenize \sphinxcode{\sphinxupquote{PAKNECONPRVTKN}} for the entire simulation period, initially a new \sphinxcode{\sphinxupquote{DataFrame}} \sphinxcode{\sphinxupquote{Cfixed}} is created as a slightly modified version of  \sphinxcode{\sphinxupquote{mpak.basedf}} using the \sphinxcode{\sphinxupquote{.fix()}} command.

\sphinxAtStartPar
\sphinxcode{\sphinxupquote{Cfixed=mpak.fix(mpak.basedf,PAKNECONPRVTKN)}}

\sphinxAtStartPar
This does two things, that could have been done manually.  First it sets the dummy variable \sphinxcode{\sphinxupquote{PAKNECONPRVTKN\_D=1}} for the entire simulation period. Recall the consumption equation like all behavioural equations of World Banjk models implemented in \sphinxcode{\sphinxupquote{ModelFlow}}is expressed in tow parts.
\begin{equation*}
\begin{split} cons= (1-cons_D)*\bigg[C'(X)\bigg] + cons_d*cons_x\end{split}
\end{equation*}
\sphinxAtStartPar
When \(cons_D=1\) the first part (as it does in this scenario) the equation evaluate to zero and consumption is equal to (1)* \(cons_x\).  If instead (which would be the normal case \(cons_d\) were set to zero. the the equation would simplify to \( cons= C'(X) \)

\sphinxAtStartPar
Then \sphinxcode{\sphinxupquote{.fix()}} method then sets the variable \sphinxcode{\sphinxupquote{PAKNECONPRVTKN\_X}} in the \sphinxcode{\sphinxupquote{Cfixed}} dataframe equal to the value of \sphinxcode{\sphinxupquote{PAKNECONPRVTKN}} in the \sphinxcode{\sphinxupquote{basedf}} \sphinxcode{\sphinxupquote{.DataFrame}}. All the other variables are  just copies of their values in \sphinxcode{\sphinxupquote{.basedf}}.

\sphinxAtStartPar
With \sphinxcode{\sphinxupquote{PAKNECONPRVTKN\_D=1}} throughout the normal behavioral equation is effectively de\sphinxhyphen{}activated or exogenized … \(PAKNECONPRVTKN=PAKNECONPRVTKN_X\).

\begin{sphinxuseclass}{cell}\begin{sphinxVerbatimInput}

\begin{sphinxuseclass}{cell_input}
\begin{sphinxVerbatim}[commandchars=\\\{\}]
\PYG{n}{mpak}\PYG{o}{.}\PYG{n}{smpl}\PYG{p}{(}\PYG{p}{)} \PYG{c+c1}{\PYGZsh{} reset the active sample period to the full model.}
\PYG{n}{Cfixed}\PYG{o}{=}\PYG{n}{mpak}\PYG{o}{.}\PYG{n}{fix}\PYG{p}{(}\PYG{n}{bline}\PYG{p}{,}\PYG{l+s+s1}{\PYGZsq{}}\PYG{l+s+s1}{PAKNECONPRVTKN}\PYG{l+s+s1}{\PYGZsq{}}\PYG{p}{)}
\end{sphinxVerbatim}

\end{sphinxuseclass}\end{sphinxVerbatimInput}
\begin{sphinxVerbatimOutput}

\begin{sphinxuseclass}{cell_output}
\begin{sphinxVerbatim}[commandchars=\\\{\}]
The folowing variables are fixed
PAKNECONPRVTKN
\end{sphinxVerbatim}

\end{sphinxuseclass}\end{sphinxVerbatimOutput}

\end{sphinxuseclass}
\sphinxAtStartPar
For the moment, the equation is exogenized but the values have been set to the same values as the \sphinxcode{\sphinxupquote{.basedf}} dataframe, so solving the model will not change anything.

\sphinxAtStartPar
The \sphinxcode{\sphinxupquote{.upd()}} method can be used to implement the assumption that Real consumption (\sphinxcode{\sphinxupquote{ PAKNECONPRVTYKN}}) would be 2.5\% stronger.

\begin{sphinxuseclass}{cell}\begin{sphinxVerbatimInput}

\begin{sphinxuseclass}{cell_input}
\begin{sphinxVerbatim}[commandchars=\\\{\}]
\PYG{n}{Cfixed}\PYG{o}{=}\PYG{n}{Cfixed}\PYG{o}{.}\PYG{n}{upd}\PYG{p}{(}\PYG{l+s+s2}{\PYGZdq{}}\PYG{l+s+s2}{\PYGZlt{}2025 2040\PYGZgt{} PAKNECONPRVTKN\PYGZus{}X  * 1.025}\PYG{l+s+s2}{\PYGZdq{}}\PYG{p}{)}
\end{sphinxVerbatim}

\end{sphinxuseclass}\end{sphinxVerbatimInput}

\end{sphinxuseclass}
\sphinxAtStartPar
To perform the simulation, the revised \sphinxcode{\sphinxupquote{CFixed}} DataFrame is input to the \sphinxcode{\sphinxupquote{mpak}} model solve routine.

\sphinxAtStartPar
\sphinxcode{\sphinxupquote{CFixedRes = mpak(Cfixed,2020,2040,keep='2.5\% increase in C 2025\sphinxhyphen{}40 (fix)')}}

\sphinxAtStartPar
And then the results can be examined graphically as before.

\begin{sphinxuseclass}{cell}\begin{sphinxVerbatimInput}

\begin{sphinxuseclass}{cell_input}
\begin{sphinxVerbatim}[commandchars=\\\{\}]
\PYG{n}{CFixedRes} \PYG{o}{=} \PYG{n}{mpak}\PYG{p}{(}\PYG{n}{Cfixed}\PYG{p}{,}\PYG{l+m+mi}{2020}\PYG{p}{,}\PYG{l+m+mi}{2040}\PYG{p}{,}\PYG{n}{keep}\PYG{o}{=}\PYG{l+s+s1}{\PYGZsq{}}\PYG{l+s+s1}{2.5}\PYG{l+s+si}{\PYGZpc{} i}\PYG{l+s+s1}{ncrease in C 2025\PYGZhy{}40}\PYG{l+s+s1}{\PYGZsq{}}\PYG{p}{)} \PYG{c+c1}{\PYGZsh{} simulates the model }
\PYG{n}{mpak}\PYG{p}{[}\PYG{l+s+s1}{\PYGZsq{}}\PYG{l+s+s1}{PAKNYGDPMKTPKN PAKNECONPRVTKN PAKNEIMPGNFSKN PAKNECONPRVTXN}\PYG{l+s+s1}{\PYGZsq{}}\PYG{p}{]}\PYG{o}{.}\PYG{n}{difpctlevel}\PYG{o}{.}\PYG{n}{mul100}\PYG{o}{.}\PYG{n}{plot}\PYG{p}{(}\PYG{n}{title}\PYG{o}{=}\PYG{l+s+s2}{\PYGZdq{}}\PYG{l+s+s2}{Impact of a permanent 2.5}\PYG{l+s+si}{\PYGZpc{} i}\PYG{l+s+s2}{ncrease in Consumption}\PYG{l+s+s2}{\PYGZdq{}}\PYG{p}{)}
\end{sphinxVerbatim}

\end{sphinxuseclass}\end{sphinxVerbatimInput}
\begin{sphinxVerbatimOutput}

\begin{sphinxuseclass}{cell_output}
\noindent\sphinxincludegraphics{{52ab02496390062cca08770ab63a94a25a9e1a0a89c22d72952c17bdd43961f8}.png}

\end{sphinxuseclass}\end{sphinxVerbatimOutput}

\end{sphinxuseclass}
\begin{sphinxuseclass}{cell}\begin{sphinxVerbatimInput}

\begin{sphinxuseclass}{cell_input}
\begin{sphinxVerbatim}[commandchars=\\\{\}]
\PYG{k+kn}{import} \PYG{n+nn}{pandas} \PYG{k}{as} \PYG{n+nn}{pd}
\PYG{k}{with} \PYG{n}{pd}\PYG{o}{.}\PYG{n}{option\PYGZus{}context}\PYG{p}{(}\PYG{l+s+s1}{\PYGZsq{}}\PYG{l+s+s1}{display.float\PYGZus{}format}\PYG{l+s+s1}{\PYGZsq{}}\PYG{p}{,} \PYG{l+s+s1}{\PYGZsq{}}\PYG{l+s+si}{\PYGZob{}:,.2f\PYGZcb{}}\PYG{l+s+s1}{\PYGZsq{}}\PYG{o}{.}\PYG{n}{format}\PYG{p}{)}\PYG{p}{:}
    \PYG{k}{with} \PYG{n}{mpak}\PYG{o}{.}\PYG{n}{set\PYGZus{}smpl}\PYG{p}{(}\PYG{l+m+mi}{2020}\PYG{p}{,}\PYG{l+m+mi}{2040}\PYG{p}{)}\PYG{p}{:}
        \PYG{n+nb}{print}\PYG{p}{(}\PYG{n}{mpak}\PYG{p}{[}\PYG{l+s+s1}{\PYGZsq{}}\PYG{l+s+s1}{PAKNYGDPMKTPKN PAKNECONPRVTKN PAKNEIMPGNFSKN PAKNECONPRVTXN}\PYG{l+s+s1}{\PYGZsq{}}\PYG{p}{]}\PYG{o}{.}\PYG{n}{difpctlevel}\PYG{o}{.}\PYG{n}{mul100}\PYG{o}{.}\PYG{n}{df}\PYG{p}{)}
\end{sphinxVerbatim}

\end{sphinxuseclass}\end{sphinxVerbatimInput}
\begin{sphinxVerbatimOutput}

\begin{sphinxuseclass}{cell_output}
\begin{sphinxVerbatim}[commandchars=\\\{\}]
      PAKNYGDPMKTPKN  PAKNECONPRVTKN  PAKNEIMPGNFSKN  PAKNECONPRVTXN
2020            0.00            0.00            0.00            0.00
2021            0.00            0.00            0.00            0.00
2022            0.00            0.00            0.00            0.00
2023            0.00            0.00            0.00            0.00
2024            0.00            0.00            0.00            0.00
2025            2.01            2.50            2.27            0.44
2026            2.07            2.50            2.43            1.06
2027            2.05            2.50            2.59            1.69
2028            1.99            2.50            2.78            2.31
2029            1.92            2.50            2.99            2.90
2030            1.83            2.50            3.22            3.47
2031            1.43            2.50            4.03            4.53
2032            1.37            2.50            4.18            4.92
2033            1.30            2.50            4.34            5.29
2034            1.23            2.50            4.50            5.64
2035            1.16            2.50            4.66            5.97
2036            1.09            2.50            4.81            6.28
2037            1.03            2.50            4.96            6.56
2038            0.96            2.50            5.10            6.82
2039            0.90            2.50            5.24            7.06
2040            0.84            2.50            5.36            7.28
\end{sphinxVerbatim}

\end{sphinxuseclass}\end{sphinxVerbatimOutput}

\end{sphinxuseclass}
\sphinxAtStartPar
The permanent rise in consumption by 2.5 percent causes a temporary increase in GDP of close to 2\% (1.86). Higher imports tend to diminish the effect on GDP. Over time higher prices due to the inflationary pressures caused by the additional demand cause the GDP impact to diminish to close to less than 1 percent by 2040.


\subsection{Temporarily exogenize a behavioural variable}
\label{\detokenize{content/05_WBModels/ScenarioAnalysis:temporarily-exogenize-a-behavioural-variable}}
\sphinxAtStartPar
The third method of formulating a scenario involves temporarily exogenizing a variable. The methodology is the same except the period for which the variable is exogenized is different.

\sphinxAtStartPar
Here the set up is basically the same as before.

\begin{sphinxuseclass}{cell}\begin{sphinxVerbatimInput}

\begin{sphinxuseclass}{cell_input}
\begin{sphinxVerbatim}[commandchars=\\\{\}]
\PYG{c+c1}{\PYGZsh{}reset the active sample period to the full period}
\PYG{n}{mpak}\PYG{o}{.}\PYG{n}{smpl}\PYG{p}{(}\PYG{p}{)}                                  
\PYG{c+c1}{\PYGZsh{} create a copy of the bline DataFrame, but setting the PAKNECONPRVTKN\PYGZus{}D variable to 1 for the period 2025 through 2027}
\PYG{n}{CTempExogAll}\PYG{o}{=}\PYG{n}{mpak}\PYG{o}{.}\PYG{n}{fix}\PYG{p}{(}\PYG{n}{bline}\PYG{p}{,}\PYG{l+s+s1}{\PYGZsq{}}\PYG{l+s+s1}{PAKNECONPRVTKN}\PYG{l+s+s1}{\PYGZsq{}}\PYG{p}{)} 
\PYG{c+c1}{\PYGZsh{} multiply the exogenized value of consumption by 2.5\PYGZpc{} for 2025 through 2027}
\PYG{n}{CTempExogAll}\PYG{o}{=}\PYG{n}{CTempExogAll}\PYG{o}{.}\PYG{n}{upd}\PYG{p}{(}\PYG{l+s+s2}{\PYGZdq{}}\PYG{l+s+s2}{\PYGZlt{}2025 2027\PYGZgt{} PAKNECONPRVTKN\PYGZus{}X * 1.025}\PYG{l+s+s2}{\PYGZdq{}}\PYG{p}{)}  

\PYG{c+c1}{\PYGZsh{}Solve the model}
\PYG{n}{CTempXAllRes} \PYG{o}{=} \PYG{n}{mpak}\PYG{p}{(}\PYG{n}{CTempExogAll}\PYG{p}{,}\PYG{l+m+mi}{2020}\PYG{p}{,}\PYG{l+m+mi}{2040}\PYG{p}{,}\PYG{n}{keep}\PYG{o}{=}\PYG{l+s+s1}{\PYGZsq{}}\PYG{l+s+s1}{2.5}\PYG{l+s+si}{\PYGZpc{} i}\PYG{l+s+s1}{ncrease in C 2025\PYGZhy{}27 \PYGZhy{}\PYGZhy{} exog whole period}\PYG{l+s+s1}{\PYGZsq{}}\PYG{p}{)} \PYG{c+c1}{\PYGZsh{} simulates the model }
\PYG{n}{mpak}\PYG{p}{[}\PYG{l+s+s1}{\PYGZsq{}}\PYG{l+s+s1}{PAKNYGDPMKTPKN PAKNECONPRVTKN PAKNEIMPGNFSKN PAKNECONPRVTXN}\PYG{l+s+s1}{\PYGZsq{}}\PYG{p}{]}\PYG{o}{.}\PYG{n}{difpctlevel}\PYG{o}{.}\PYG{n}{mul100}\PYG{o}{.}\PYG{n}{plot}\PYG{p}{(}\PYG{n}{title}\PYG{o}{=}\PYG{l+s+s2}{\PYGZdq{}}\PYG{l+s+s2}{Temporary hike in Consumption 2025\PYGZhy{}2027}\PYG{l+s+s2}{\PYGZdq{}}\PYG{p}{)}
\end{sphinxVerbatim}

\end{sphinxuseclass}\end{sphinxVerbatimInput}
\begin{sphinxVerbatimOutput}

\begin{sphinxuseclass}{cell_output}
\begin{sphinxVerbatim}[commandchars=\\\{\}]
The folowing variables are fixed
PAKNECONPRVTKN
\end{sphinxVerbatim}

\noindent\sphinxincludegraphics{{73f3b76200d255bb4afa4cda476ea2e927d7dfe2c628bc6783f8fb5acbe45c91}.png}

\end{sphinxuseclass}\end{sphinxVerbatimOutput}

\end{sphinxuseclass}
\sphinxAtStartPar
The results are quite different.  GDP is boosted initially as before but when consumption drops back to its pre\sphinxhyphen{}shock level, GDP and imports decline sharply.

\sphinxAtStartPar
Prices (and inflation) are higher initially but when the economy starts to slow after 2025 prices actually fall (deflation). While prices are falling, the level of prices remains higher at the end of the simulation.


\subsubsection{Temporary shock exogenized for the whole period}
\label{\detokenize{content/05_WBModels/ScenarioAnalysis:temporary-shock-exogenized-for-the-whole-period}}
\sphinxAtStartPar
This scenario is the same as the previous, but this time the \sphinxcode{\sphinxupquote{\sphinxhyphen{}\sphinxhyphen{}KG}} (keep\_growth) option is used to maintain the pre\sphinxhyphen{}shock growth rates of consumption in the post\sphinxhyphen{}shock period.  Effectively this is the same as a permanent increase in the level of consumption by 2.5\% because the final shocked value of consumption (which was 2.5\% higher then its pre\sphinxhyphen{}shock level) is grown at the same pre\sphinxhyphen{}shock rate – ensuring that all post\sphinxhyphen{}shock variables are also up by 2.5\%.

\begin{sphinxuseclass}{cell}\begin{sphinxVerbatimInput}

\begin{sphinxuseclass}{cell_input}
\begin{sphinxVerbatim}[commandchars=\\\{\}]
\PYG{n}{mpak}\PYG{o}{.}\PYG{n}{smpl}\PYG{p}{(}\PYG{p}{)} \PYG{c+c1}{\PYGZsh{} reset the active sample period to the full model.}
\PYG{n}{CTempExogAllKG}\PYG{o}{=}\PYG{n}{mpak}\PYG{o}{.}\PYG{n}{fix}\PYG{p}{(}\PYG{n}{bline}\PYG{p}{,}\PYG{l+s+s1}{\PYGZsq{}}\PYG{l+s+s1}{PAKNECONPRVTKN}\PYG{l+s+s1}{\PYGZsq{}}\PYG{p}{)}
\PYG{n}{CTempExogAllKG} \PYG{o}{=} \PYG{n}{CTempExogAllKG}\PYG{o}{.}\PYG{n}{upd}\PYG{p}{(}\PYG{l+s+s1}{\PYGZsq{}\PYGZsq{}\PYGZsq{}}
\PYG{l+s+s1}{\PYGZlt{}2025 2027\PYGZgt{} PAKNECONPRVTKN\PYGZus{}X * 1.025 \PYGZhy{}\PYGZhy{}kg}
\PYG{l+s+s1}{\PYGZsq{}\PYGZsq{}\PYGZsq{}}\PYG{p}{,}\PYG{n}{lprint}\PYG{o}{=}\PYG{l+m+mi}{0}\PYG{p}{)}

\PYG{c+c1}{\PYGZsh{}Now we solve the model}
\PYG{n}{CTempXAllResKG} \PYG{o}{=} \PYG{n}{mpak}\PYG{p}{(}\PYG{n}{CTempExogAllKG}\PYG{p}{,}\PYG{l+m+mi}{2020}\PYG{p}{,}\PYG{l+m+mi}{2040}\PYG{p}{,}\PYG{n}{keep}\PYG{o}{=}\PYG{l+s+s1}{\PYGZsq{}}\PYG{l+s+s1}{2.5}\PYG{l+s+si}{\PYGZpc{} i}\PYG{l+s+s1}{ncrease in C 2025\PYGZhy{}27 \PYGZhy{}\PYGZhy{} exog whole period \PYGZhy{}\PYGZhy{}KG=True}\PYG{l+s+s1}{\PYGZsq{}}\PYG{p}{)} \PYG{c+c1}{\PYGZsh{} simulates the model }
\PYG{n}{mpak}\PYG{p}{[}\PYG{l+s+s1}{\PYGZsq{}}\PYG{l+s+s1}{PAKNYGDPMKTPKN PAKNECONPRVTKN PAKNEIMPGNFSKN PAKNECONPRVTXN}\PYG{l+s+s1}{\PYGZsq{}}\PYG{p}{]}\PYG{o}{.}\PYG{n}{difpctlevel}\PYG{o}{.}\PYG{n}{mul100}\PYG{o}{.}\PYG{n}{plot}\PYG{p}{(}\PYG{n}{title}\PYG{o}{=}\PYG{l+s+s2}{\PYGZdq{}}\PYG{l+s+s2}{2.5}\PYG{l+s+s2}{\PYGZpc{}}\PYG{l+s+s2}{ boost to cons 2025\PYGZhy{}27 \PYGZhy{}\PYGZhy{}kg=True}\PYG{l+s+s2}{\PYGZdq{}}\PYG{p}{)}
\end{sphinxVerbatim}

\end{sphinxuseclass}\end{sphinxVerbatimInput}
\begin{sphinxVerbatimOutput}

\begin{sphinxuseclass}{cell_output}
\begin{sphinxVerbatim}[commandchars=\\\{\}]
The folowing variables are fixed
PAKNECONPRVTKN
\end{sphinxVerbatim}

\noindent\sphinxincludegraphics{{c2bd6e8006091883026a88290f9cc086d04d122e42c8bf6aa157d94595874cb0}.png}

\end{sphinxuseclass}\end{sphinxVerbatimOutput}

\end{sphinxuseclass}

\subsection{Exogenize the variable only for the period during which it is shocked}
\label{\detokenize{content/05_WBModels/ScenarioAnalysis:exogenize-the-variable-only-for-the-period-during-which-it-is-shocked}}
\sphinxAtStartPar
This scenario introduces a subtle but import difference.  Here we the variable is again exogenized using the fix syntax. But this time it is exogonized only for the period where the variable is shocked.

\sphinxAtStartPar
This means that the consumption function will be de\sphinxhyphen{}activated for only three years (instead of the whole period as in the previous examples).  As a result, the values that consumption takes in 2028, 2029, … 2040 depend on the model, not the level it was set to when exogenized (which was the case in the 3 previous versions).

\sphinxAtStartPar
Looking at the maths of the model the consumption equation is effectively split into two.

\sphinxAtStartPar
for the period before 2025 \(cons_d=0\) and the consumption equation simplifies to:

\sphinxAtStartPar
\(cons=C(X)\)

\sphinxAtStartPar
for the period 2025\sphinxhyphen{}2028 it is exogenized (\(cons_d=1\)) so it simplifies to:

\sphinxAtStartPar
\(cons=cons_x\)

\sphinxAtStartPar
but in the final period 2028\sphinxhyphen{}2040 (\(cons_d=0\)) and the equation reverts to:

\sphinxAtStartPar
\(cons=C(X)\)

\begin{sphinxuseclass}{cell}\begin{sphinxVerbatimInput}

\begin{sphinxuseclass}{cell_input}
\begin{sphinxVerbatim}[commandchars=\\\{\}]
\PYG{n}{mpak}\PYG{o}{.}\PYG{n}{smpl}\PYG{p}{(}\PYG{p}{)} \PYG{c+c1}{\PYGZsh{} reset the active sample period to the full model.}
\PYG{n}{CExogTemp}\PYG{o}{=}\PYG{n}{mpak}\PYG{o}{.}\PYG{n}{fix}\PYG{p}{(}\PYG{n}{bline}\PYG{p}{,}\PYG{l+s+s1}{\PYGZsq{}}\PYG{l+s+s1}{PAKNECONPRVTKN}\PYG{l+s+s1}{\PYGZsq{}}\PYG{p}{,}\PYG{l+m+mi}{2025}\PYG{p}{,}\PYG{l+m+mi}{2027}\PYG{p}{)}                             \PYG{c+c1}{\PYGZsh{}Consumption is exogenized only for three years 2025 2026 and 2027 PAKNECONPRVTKN\PYGZus{}D=1 for 2025,2026, 2027 0 elsewhere.}
\PYG{n}{CExogTemp} \PYG{o}{=} \PYG{n}{CExogTemp}\PYG{o}{.}\PYG{n}{upd}\PYG{p}{(}\PYG{l+s+s1}{\PYGZsq{}}\PYG{l+s+s1}{\PYGZlt{}2025 2027\PYGZgt{} PAKNECONPRVTKN\PYGZus{}X * 1.025}\PYG{l+s+s1}{\PYGZsq{}}\PYG{p}{,}\PYG{n}{lprint}\PYG{o}{=}\PYG{l+m+mi}{0}\PYG{p}{)}       \PYG{c+c1}{\PYGZsh{}In subsequent years it\PYGZsq{}s level will be determined by the equation }

\PYG{c+c1}{\PYGZsh{}Solve the model}
\PYG{n}{CExogTempRes} \PYG{o}{=} \PYG{n}{mpak}\PYG{p}{(}\PYG{n}{CExogTemp}\PYG{p}{,}\PYG{l+m+mi}{2020}\PYG{p}{,}\PYG{l+m+mi}{2040}\PYG{p}{,}\PYG{n}{keep}\PYG{o}{=}\PYG{l+s+s1}{\PYGZsq{}}\PYG{l+s+s1}{2.5}\PYG{l+s+si}{\PYGZpc{} i}\PYG{l+s+s1}{ncrease in C 2025\PYGZhy{}27 \PYGZhy{}\PYGZhy{} temporarily exogenized}\PYG{l+s+s1}{\PYGZsq{}}\PYG{p}{)} \PYG{c+c1}{\PYGZsh{} simulates the model }
\PYG{n}{mpak}\PYG{p}{[}\PYG{l+s+s1}{\PYGZsq{}}\PYG{l+s+s1}{PAKNYGDPMKTPKN PAKNECONPRVTKN PAKNEIMPGNFSKN PAKNECONPRVTXN}\PYG{l+s+s1}{\PYGZsq{}}\PYG{p}{]}\PYG{o}{.}\PYG{n}{difpctlevel}\PYG{o}{.}\PYG{n}{mul100}\PYG{o}{.}\PYG{n}{plot}\PYG{p}{(}\PYG{n}{title}\PYG{o}{=}\PYG{l+s+s2}{\PYGZdq{}}\PYG{l+s+s2}{Temporary 2.5}\PYG{l+s+s2}{\PYGZpc{}}\PYG{l+s+s2}{ boost to cons 2025\PYGZhy{}27 \PYGZhy{} equation active}\PYG{l+s+s2}{\PYGZdq{}}\PYG{p}{)}
\end{sphinxVerbatim}

\end{sphinxuseclass}\end{sphinxVerbatimInput}
\begin{sphinxVerbatimOutput}

\begin{sphinxuseclass}{cell_output}
\begin{sphinxVerbatim}[commandchars=\\\{\}]
The folowing variables are fixed
PAKNECONPRVTKN
\end{sphinxVerbatim}

\noindent\sphinxincludegraphics{{9ddab60686063ef2d66f0e207c24c525b79a14cc345de2d3e7fb9e96f9ce290d}.png}

\end{sphinxuseclass}\end{sphinxVerbatimOutput}

\end{sphinxuseclass}
\sphinxAtStartPar
These results have subtle differences compared with the previous.  The most obvious is visible in looking at the graph for Consumption.  Rather than reverting immediately to its earlier pre\sphinxhyphen{}shock level, it falls more gradually and actually overshoots (falls below its earlier level), before returning slowly to its pre\sphinxhyphen{}shock level.  That is because unlike in the previous shocks, its path is being determined endogenously and reacting to changes elsewhere in the model, notably changes to prices, wages and government spending as well as the lagged level of consumption.

\begin{sphinxuseclass}{cell}\begin{sphinxVerbatimInput}

\begin{sphinxuseclass}{cell_input}
\begin{sphinxVerbatim}[commandchars=\\\{\}]
\PYG{n+nb}{print}\PYG{p}{(}\PYG{l+s+s1}{\PYGZsq{}}\PYG{l+s+s1}{Consumption base and shock levels}\PYG{l+s+se}{\PYGZbs{}r}\PYG{l+s+se}{\PYGZbs{}n}\PYG{l+s+s1}{\PYGZsq{}}\PYG{p}{)}\PYG{p}{;}

\PYG{n+nb}{print}\PYG{p}{(}\PYG{l+s+s1}{\PYGZsq{}}\PYG{l+s+s1}{Real values in 2030}\PYG{l+s+s1}{\PYGZsq{}}\PYG{p}{)}\PYG{p}{;}
\PYG{n+nb}{print}\PYG{p}{(}\PYG{l+s+sa}{f}\PYG{l+s+s1}{\PYGZsq{}}\PYG{l+s+s1}{Base value:  }\PYG{l+s+si}{\PYGZob{}}\PYG{n}{bline}\PYG{o}{.}\PYG{n}{loc}\PYG{p}{[}\PYG{l+m+mi}{2028}\PYG{p}{,}\PYG{l+s+s2}{\PYGZdq{}}\PYG{l+s+s2}{PAKNECONPRVTKN}\PYG{l+s+s2}{\PYGZdq{}}\PYG{p}{]}\PYG{l+s+si}{:}\PYG{l+s+s1}{,.0f}\PYG{l+s+si}{\PYGZcb{}}\PYG{l+s+s1}{.}\PYG{l+s+se}{\PYGZbs{}t}\PYG{l+s+s1}{Shocked value: }\PYG{l+s+si}{\PYGZob{}}\PYG{n}{CExogTempRes}\PYG{o}{.}\PYG{n}{loc}\PYG{p}{[}\PYG{l+m+mi}{2028}\PYG{p}{,}\PYG{l+s+s2}{\PYGZdq{}}\PYG{l+s+s2}{PAKNECONPRVTKN}\PYG{l+s+s2}{\PYGZdq{}}\PYG{p}{]}\PYG{l+s+si}{:}\PYG{l+s+s1}{,.0f}\PYG{l+s+si}{\PYGZcb{}}\PYG{l+s+s1}{.}\PYG{l+s+se}{\PYGZbs{}r}\PYG{l+s+se}{\PYGZbs{}n}\PYG{l+s+s1}{\PYGZsq{}}
    \PYG{l+s+sa}{f}\PYG{l+s+s1}{\PYGZsq{}}\PYG{l+s+s1}{Percent difference: }\PYG{l+s+si}{\PYGZob{}}\PYG{n+nb}{round}\PYG{p}{(}\PYG{l+m+mi}{100}\PYG{o}{*}\PYG{p}{(}\PYG{p}{(}\PYG{n}{CExogTempRes}\PYG{o}{.}\PYG{n}{loc}\PYG{p}{[}\PYG{l+m+mi}{2030}\PYG{p}{,}\PYG{l+s+s2}{\PYGZdq{}}\PYG{l+s+s2}{PAKNECONPRVTKN}\PYG{l+s+s2}{\PYGZdq{}}\PYG{p}{]}\PYG{o}{\PYGZhy{}}\PYG{n}{bline}\PYG{o}{.}\PYG{n}{loc}\PYG{p}{[}\PYG{l+m+mi}{2028}\PYG{p}{,}\PYG{l+s+s2}{\PYGZdq{}}\PYG{l+s+s2}{PAKNECONPRVTKN}\PYG{l+s+s2}{\PYGZdq{}}\PYG{p}{]}\PYG{p}{)}\PYG{o}{/}\PYG{n}{bline}\PYG{o}{.}\PYG{n}{loc}\PYG{p}{[}\PYG{l+m+mi}{2028}\PYG{p}{,}\PYG{l+s+s2}{\PYGZdq{}}\PYG{l+s+s2}{PAKNECONPRVTKN}\PYG{l+s+s2}{\PYGZdq{}}\PYG{p}{]}\PYG{p}{)}\PYG{p}{,}\PYG{l+m+mi}{2}\PYG{p}{)}\PYG{l+s+si}{\PYGZcb{}}\PYG{l+s+s1}{\PYGZsq{}}\PYG{p}{)}
\PYG{n+nb}{print}\PYG{p}{(}\PYG{l+s+s1}{\PYGZsq{}}\PYG{l+s+se}{\PYGZbs{}r}\PYG{l+s+se}{\PYGZbs{}n}\PYG{l+s+s1}{Real values in 2040}\PYG{l+s+s1}{\PYGZsq{}}\PYG{p}{)}\PYG{p}{;}
\PYG{n+nb}{print}\PYG{p}{(}\PYG{l+s+sa}{f}\PYG{l+s+s1}{\PYGZsq{}}\PYG{l+s+s1}{Base value:  }\PYG{l+s+si}{\PYGZob{}}\PYG{n}{bline}\PYG{o}{.}\PYG{n}{loc}\PYG{p}{[}\PYG{l+m+mi}{2040}\PYG{p}{,}\PYG{l+s+s2}{\PYGZdq{}}\PYG{l+s+s2}{PAKNECONPRVTKN}\PYG{l+s+s2}{\PYGZdq{}}\PYG{p}{]}\PYG{l+s+si}{:}\PYG{l+s+s1}{,.0f}\PYG{l+s+si}{\PYGZcb{}}\PYG{l+s+s1}{.}\PYG{l+s+se}{\PYGZbs{}t}\PYG{l+s+s1}{Shocked value: }\PYG{l+s+si}{\PYGZob{}}\PYG{n}{CExogTempRes}\PYG{o}{.}\PYG{n}{loc}\PYG{p}{[}\PYG{l+m+mi}{2040}\PYG{p}{,}\PYG{l+s+s2}{\PYGZdq{}}\PYG{l+s+s2}{PAKNECONPRVTKN}\PYG{l+s+s2}{\PYGZdq{}}\PYG{p}{]}\PYG{l+s+si}{:}\PYG{l+s+s1}{,.0f}\PYG{l+s+si}{\PYGZcb{}}\PYG{l+s+s1}{.}\PYG{l+s+se}{\PYGZbs{}r}\PYG{l+s+se}{\PYGZbs{}n}\PYG{l+s+s1}{\PYGZsq{}}
    \PYG{l+s+sa}{f}\PYG{l+s+s1}{\PYGZsq{}}\PYG{l+s+s1}{Percent difference: }\PYG{l+s+si}{\PYGZob{}}\PYG{n+nb}{round}\PYG{p}{(}\PYG{l+m+mi}{100}\PYG{o}{*}\PYG{p}{(}\PYG{p}{(}\PYG{n}{CExogTempRes}\PYG{o}{.}\PYG{n}{loc}\PYG{p}{[}\PYG{l+m+mi}{2040}\PYG{p}{,}\PYG{l+s+s2}{\PYGZdq{}}\PYG{l+s+s2}{PAKNECONPRVTKN}\PYG{l+s+s2}{\PYGZdq{}}\PYG{p}{]}\PYG{o}{\PYGZhy{}}\PYG{n}{bline}\PYG{o}{.}\PYG{n}{loc}\PYG{p}{[}\PYG{l+m+mi}{2040}\PYG{p}{,}\PYG{l+s+s2}{\PYGZdq{}}\PYG{l+s+s2}{PAKNECONPRVTKN}\PYG{l+s+s2}{\PYGZdq{}}\PYG{p}{]}\PYG{p}{)}\PYG{o}{/}\PYG{n}{bline}\PYG{o}{.}\PYG{n}{loc}\PYG{p}{[}\PYG{l+m+mi}{2040}\PYG{p}{,}\PYG{l+s+s2}{\PYGZdq{}}\PYG{l+s+s2}{PAKNECONPRVTKN}\PYG{l+s+s2}{\PYGZdq{}}\PYG{p}{]}\PYG{p}{)}\PYG{p}{,}\PYG{l+m+mi}{2}\PYG{p}{)}\PYG{l+s+si}{\PYGZcb{}}\PYG{l+s+s1}{\PYGZsq{}}\PYG{p}{)}
\end{sphinxVerbatim}

\end{sphinxuseclass}\end{sphinxVerbatimInput}
\begin{sphinxVerbatimOutput}

\begin{sphinxuseclass}{cell_output}
\begin{sphinxVerbatim}[commandchars=\\\{\}]
Consumption base and shock levels

Real values in 2030
Base value:  27,241,278.	Shocked value: 27,616,949.
Percent difference: 5.36

Real values in 2040
Base value:  38,676,995.	Shocked value: 38,693,167.
Percent difference: 0.04
\end{sphinxVerbatim}

\end{sphinxuseclass}\end{sphinxVerbatimOutput}

\end{sphinxuseclass}

\subsection{Simulation with Add factors}
\label{\detokenize{content/05_WBModels/ScenarioAnalysis:simulation-with-add-factors}}
\sphinxAtStartPar
Add factors are a crucial element of the macromodels of the World Bank and serve multiple purposes.

\sphinxAtStartPar
In simulation, add\sphinxhyphen{}factors allow simulations to be conducted \sphinxstylestrong{without} de\sphinxhyphen{}activating behavioural equations.  Such shocks are often referred to as \sphinxstylestrong{endogenous} shocks because the equation of the behavioural variable that is shocked remains  active throughout.

\sphinxAtStartPar
In some ways they are very similar to a temporary exogenous shock. Both ways of producing the shock allow the shocked variable to respond endogenously in the period after the shock.  The main difference between the two approaches is that:
\begin{itemize}
\item {} 
\sphinxAtStartPar
\sphinxstylestrong{Endogenous} shocks (Add\sphinxhyphen{}Factor shocks) allow the shocked variable to respond to changed circumstances that occur during the period of the shock.
\begin{itemize}
\item {} 
\sphinxAtStartPar
This approach makes most sense for “animal spirits”, shocks where the underlying behaviour is expected to change.

\item {} 
\sphinxAtStartPar
It also makes sense when actions of one part of an aggregate is likely to impact behaviour of other sectors within an aggregate

\item {} 
\sphinxAtStartPar
increased investment by a particular sector would be an example here as the associated increase in activity is likely to increase investment incentives in other sectors, while increased demand for savings will increase interest rates and the cost of capital operating in the opposite direction.

\item {} 
\sphinxAtStartPar
Sustained changes in behaviour, for example increased propensity to invest because of improved recognition

\end{itemize}

\item {} 
\sphinxAtStartPar
\sphinxstylestrong{Exogenous} shocks to endogenous variables fix the level of the shocked variable during the shock period.
\begin{itemize}
\item {} 
\sphinxAtStartPar
Changes in government spending policy, something that is often largely an economically exogenous decision.

\end{itemize}

\end{itemize}


\subsubsection{Simulating the impact of a planned investment}
\label{\detokenize{content/05_WBModels/ScenarioAnalysis:simulating-the-impact-of-a-planned-investment}}
\sphinxAtStartPar
The below simulation uses the add\sphinxhyphen{}factor to simulate the impact of a 3 year investment  program beginning in 2025 of 1 percent of GDP per year, that is financed through an increase in foreign direct investment. This might reflect a specific large scale plant that is being constructed due to a deal reached by the government with a foreign manufacturer.  The add\sphinxhyphen{}factor approach is chosen because the additional investment is likely to increase demand for the products of other firms, which is likely to incite them to add to their investments as well.


\paragraph{How to translate the economic shock into a model shock}
\label{\detokenize{content/05_WBModels/ScenarioAnalysis:how-to-translate-the-economic-shock-into-a-model-shock}}
\sphinxAtStartPar
Add\sphinxhyphen{}factors in the \sphinxcode{\sphinxupquote{MFMod}} framework are applied to the intercept of an equation (not the level of the dependent variable).  This preserves the estimated elasticities of the equation, but makes introduction of an add\sphinxhyphen{}factor shock somewhat more complicated than the exogenous approach.  Below a step\sphinxhyphen{}by\sphinxhyphen{}step how\sphinxhyphen{}to guide:
\begin{enumerate}
\sphinxsetlistlabels{\arabic}{enumi}{enumii}{}{.}%
\item {} 
\sphinxAtStartPar
Identify numerical size of the shock

\item {} 
\sphinxAtStartPar
Examine the functional form of the equation, to determine the nature of the add factor.  If the equation is expressed as a:
\begin{itemize}
\item {} 
\sphinxAtStartPar
\sphinxstylestrong{growth rate} then the add\sphinxhyphen{}factor will be an addition or subtraction to the growth rate

\item {} 
\sphinxAtStartPar
\sphinxstylestrong{percent of GDP (or some other level)} then the add\sphinxhyphen{}factor will be an addition or subtraction to the share of growth.

\item {} 
\sphinxAtStartPar
\sphinxstylestrong{Level} then the add\sphinxhyphen{}factor will be a direct addition to the level of the dependent variable

\end{itemize}

\item {} 
\sphinxAtStartPar
Convert the economic shock into the units of the add\sphinxhyphen{}factor

\item {} 
\sphinxAtStartPar
Shock the add\sphinxhyphen{}factor by the above amount and run the model
\begin{itemize}
\item {} 
\sphinxAtStartPar
Note the add\sphinxhyphen{}factor is an exogenous variable in the model, so shocking it follows the well established process for shocking an exogenous variable.

\end{itemize}

\end{enumerate}


\paragraph{Determine the size of shock}
\label{\detokenize{content/05_WBModels/ScenarioAnalysis:determine-the-size-of-shock}}
\sphinxAtStartPar
Above we identified the shock as to be a 1 percent of GDP increase in FDI that flows directly into private\sphinxhyphen{}sector investment.  A first step would be to determine the variables that need to be shocked (FDI) and private investment. To do this we can query the variable dictionary.

\begin{sphinxuseclass}{cell}\begin{sphinxVerbatimInput}

\begin{sphinxuseclass}{cell_input}
\begin{sphinxVerbatim}[commandchars=\\\{\}]
\PYG{n}{mpak}\PYG{p}{[}\PYG{l+s+s1}{\PYGZsq{}}\PYG{l+s+s1}{*NY*}\PYG{l+s+s1}{\PYGZsq{}}\PYG{p}{]}\PYG{o}{.}\PYG{n}{des}
\end{sphinxVerbatim}

\end{sphinxuseclass}\end{sphinxVerbatimInput}
\begin{sphinxVerbatimOutput}

\begin{sphinxuseclass}{cell_output}
\begin{sphinxVerbatim}[commandchars=\\\{\}]
PAKNYGDPDISCCN            : GDP Disc., LCU mn
PAKNYGDPDISCKN            : GDP Disc., 2000 LCU mn
PAKNYGDPFCSTCN            : GDP Factor Cost Local Currency units Volumes National base year
PAKNYGDPFCSTKN            : GDP Factor Cost Local Currency units Volumes National base year
PAKNYGDPFCSTXN            : GDP Factor Cost Local Currency units Implicit Price deflator
PAKNYGDPFCSTXN\PYGZus{}A          : Add factor:GDP Factor Cost Local Currency units Implicit Price deflator
PAKNYGDPFCSTXN\PYGZus{}D          : Fix dummy:GDP Factor Cost Local Currency units Implicit Price deflator
PAKNYGDPFCSTXN\PYGZus{}FITTED     : Fitted  value:GDP Factor Cost Local Currency units Implicit Price deflator
PAKNYGDPFCSTXN\PYGZus{}X          : Fix value:GDP Factor Cost Local Currency units Implicit Price deflator
PAKNYGDPGAP\PYGZus{}              : Output Gap (\PYGZpc{} of Potential GDP)
PAKNYGDPMKTPCD            : GDP, Market Prices, US\PYGZdl{} mn
PAKNYGDPMKTPCN            : GDP, Market Prices, LCU mn
PAKNYGDPMKTPCN\PYGZus{}VALUE\PYGZus{}2010 : PAKNYGDPMKTPCN\PYGZus{}VALUE\PYGZus{}2010
PAKNYGDPMKTPKD            : GDP, Market Prices, 2000 US\PYGZdl{} mn
PAKNYGDPMKTPKN            : Real GDP
PAKNYGDPMKTPKN\PYGZus{}VALUE\PYGZus{}2010 : PAKNYGDPMKTPKN\PYGZus{}VALUE\PYGZus{}2010
PAKNYGDPMKTPXN            : GDP, Marker Prices, LCU Price defl., 2000 = 1
PAKNYGDPPOTLKN            : Potential Output, constant LCU
PAKNYGDPTFP               : Total factor productivity
PAKNYTAXNINDCN            : Net Indirect Taxes Local Currency units Values
PAKNYTAXNINDKN            : Net Indirect Taxes Local Currency units Volumes National base year
PAKNYWBFORMSH             : PAKNYWBFORMSH
PAKNYWBINFMSH             : PAKNYWBINFMSH
PAKNYWRTFORMCN            : PAKNYWRTFORMCN
PAKNYWRTFORMCN\PYGZus{}A          : Add factor:PAKNYWRTFORMCN
PAKNYWRTFORMCN\PYGZus{}D          : Fix dummy:PAKNYWRTFORMCN
PAKNYWRTFORMCN\PYGZus{}FITTED     : Fitted  value:PAKNYWRTFORMCN
PAKNYWRTFORMCN\PYGZus{}X          : Fix value:PAKNYWRTFORMCN
PAKNYWRTINFMCN            : PAKNYWRTINFMCN
PAKNYWRTINFMCN\PYGZus{}A          : Add factor:PAKNYWRTINFMCN
PAKNYWRTINFMCN\PYGZus{}D          : Fix dummy:PAKNYWRTINFMCN
PAKNYWRTINFMCN\PYGZus{}FITTED     : Fitted  value:PAKNYWRTINFMCN
PAKNYWRTINFMCN\PYGZus{}X          : Fix value:PAKNYWRTINFMCN
PAKNYWRTTOTLCN            : PAKNYWRTTOTLCN
PAKNYYGOSOTLCN            : PAKNYYGOSOTLCN
PAKNYYWBFORMCN            : PAKNYYWBFORMCN
PAKNYYWBINFMCN            : PAKNYYWBINFMCN
PAKNYYWBINFMCN\PYGZus{}           : PAKNYYWBINFMCN\PYGZus{}
PAKNYYWBTOTLCN            : Total Wage Bill
PAKNYYWBTOTLCN\PYGZus{}           : Labor Share of Income
\end{sphinxVerbatim}

\end{sphinxuseclass}\end{sphinxVerbatimOutput}

\end{sphinxuseclass}

\paragraph{Identify the functional form(s)}
\label{\detokenize{content/05_WBModels/ScenarioAnalysis:identify-the-functional-form-s}}
\sphinxAtStartPar
To understand how to shock using the add factor, it is essential to understand how the add\sphinxhyphen{}factor enters into the equation.


\begin{savenotes}\sphinxattablestart
\centering
\begin{tabulary}{\linewidth}[t]{|T|T|}
\hline
\sphinxstyletheadfamily 
\sphinxAtStartPar
Addfactor is on the intercept of
&\sphinxstyletheadfamily 
\sphinxAtStartPar
Shock needs to be calculated as
\\
\hline
\sphinxAtStartPar
a growth equation
&
\sphinxAtStartPar
a change in th growth rate
\\
\hline
\sphinxAtStartPar
Share of GDP
&
\sphinxAtStartPar
a percent of GDP
\\
\hline
\sphinxAtStartPar
Level
&
\sphinxAtStartPar
as change in the level
\\
\hline
\end{tabulary}
\par
\sphinxattableend\end{savenotes}

\sphinxAtStartPar
Use the .eviews command or .original command to identify the functional forms if the equation to be shocked.

\begin{sphinxuseclass}{cell}\begin{sphinxVerbatimInput}

\begin{sphinxuseclass}{cell_input}
\begin{sphinxVerbatim}[commandchars=\\\{\}]
\PYG{c+c1}{\PYGZsh{} This needs to be rewritten to use the eviews expression when published}
\PYG{n}{mpak}\PYG{p}{[}\PYG{l+s+s1}{\PYGZsq{}}\PYG{l+s+s1}{PAKNEGDIFPRVKN}\PYG{l+s+s1}{\PYGZsq{}}\PYG{p}{]}\PYG{o}{.}\PYG{n}{frml}
\end{sphinxVerbatim}

\end{sphinxuseclass}\end{sphinxVerbatimInput}
\begin{sphinxVerbatimOutput}

\begin{sphinxuseclass}{cell_output}
\begin{sphinxVerbatim}[commandchars=\\\{\}]
PAKNEGDIFPRVKN : FRML \PYGZlt{}DAMP,STOC\PYGZgt{} PAKNEGDIFPRVKN = (PAKNEGDIFPRVKN\PYGZus{}A*PAKNEGDIKSTKKN(\PYGZhy{}1)+ (0.00212272413966296+0.970234989019907*(PAKNEGDIFPRVKN(\PYGZhy{}1)/PAKNEGDIKSTKKN(\PYGZhy{}2))+(1\PYGZhy{}0.970234989019907)*(((LOG(PAKNYGDPPOTLKN))\PYGZhy{}(LOG(PAKNYGDPPOTLKN(\PYGZhy{}1))))+PAKDEPR)+0.0525240494260597*((LOG(PAKNEKRTTOTLCN/PAKNYGDPFCSTXN))\PYGZhy{}(LOG(PAKNEKRTTOTLCN(\PYGZhy{}1)/PAKNYGDPFCSTXN(\PYGZhy{}1))))) *PAKNEGDIKSTKKN(\PYGZhy{}1)) * (1\PYGZhy{}PAKNEGDIFPRVKN\PYGZus{}D)+ PAKNEGDIFPRVKN\PYGZus{}X*PAKNEGDIFPRVKN\PYGZus{}D \PYGZdl{}
\end{sphinxVerbatim}

\end{sphinxuseclass}\end{sphinxVerbatimOutput}

\end{sphinxuseclass}

\paragraph{Calculate the size of the required add factor shock}
\label{\detokenize{content/05_WBModels/ScenarioAnalysis:calculate-the-size-of-the-required-add-factor-shock}}
\sphinxAtStartPar
The shock to be executed is 0.5 percent of GDP.

\sphinxAtStartPar
It is assumed that the financing will come from FDI and that all the money will be spent in one year on private investment.

\sphinxAtStartPar
The private investment equation is written as a share of the capital stock.  Therefore, the add\sphinxhyphen{}factor needs to be shocked by adding 1 percent of GDP to private investment in 2028 divided by the capital stock in 2028.

\begin{sphinxuseclass}{cell}\begin{sphinxVerbatimInput}

\begin{sphinxuseclass}{cell_input}
\begin{sphinxVerbatim}[commandchars=\\\{\}]
\PYG{c+c1}{\PYGZsh{}Create a DataFrame AFShock that is equal tothe baseline}
\PYG{n}{AFShock}\PYG{o}{=}\PYG{n}{bline}

\PYG{c+c1}{\PYGZsh{}Display the level of the AF}
\PYG{n+nb}{print}\PYG{p}{(}\PYG{l+s+s2}{\PYGZdq{}}\PYG{l+s+s2}{Pre shock levels}\PYG{l+s+s2}{\PYGZdq{}}\PYG{p}{)}
\PYG{n}{AFShock}\PYG{o}{.}\PYG{n}{loc}\PYG{p}{[}\PYG{l+m+mi}{2025}\PYG{p}{:}\PYG{l+m+mi}{2030}\PYG{p}{,}\PYG{p}{[}\PYG{l+s+s1}{\PYGZsq{}}\PYG{l+s+s1}{PAKNEGDIFPRVKN\PYGZus{}A}\PYG{l+s+s1}{\PYGZsq{}}\PYG{p}{,}\PYG{l+s+s1}{\PYGZsq{}}\PYG{l+s+s1}{PAKNEGDIFPRVKN}\PYG{l+s+s1}{\PYGZsq{}}\PYG{p}{,}\PYG{l+s+s1}{\PYGZsq{}}\PYG{l+s+s1}{PAKNEGDIKSTKKN}\PYG{l+s+s1}{\PYGZsq{}}\PYG{p}{]}\PYG{p}{]}

\PYG{c+c1}{\PYGZsh{}print(AFShock.loc[2025:2030,\PYGZsq{}PAKNEGDIFPRVKN\PYGZsq{}]/AFShock.loc[2025:2030,\PYGZsq{}PAKNYGDPMKTPKN\PYGZsq{}]*100)}
\end{sphinxVerbatim}

\end{sphinxuseclass}\end{sphinxVerbatimInput}
\begin{sphinxVerbatimOutput}

\begin{sphinxuseclass}{cell_output}
\begin{sphinxVerbatim}[commandchars=\\\{\}]
Pre shock levels
\end{sphinxVerbatim}

\begin{sphinxVerbatim}[commandchars=\\\{\}]
      PAKNEGDIFPRVKN\PYGZus{}A  PAKNEGDIFPRVKN  PAKNEGDIKSTKKN
2025         \PYGZhy{}0.000458    1.602854e+06    4.730392e+07
2026         \PYGZhy{}0.000389    1.581104e+06    4.814879e+07
2027         \PYGZhy{}0.000331    1.569541e+06    4.900980e+07
2028         \PYGZhy{}0.000281    1.569141e+06    4.989869e+07
2029         \PYGZhy{}0.000239    1.580577e+06    5.082694e+07
2030         \PYGZhy{}0.000203    1.604394e+06    5.180590e+07
\end{sphinxVerbatim}

\end{sphinxuseclass}\end{sphinxVerbatimOutput}

\end{sphinxuseclass}
\sphinxAtStartPar
Below the mfcalc routine is used to set the addfactor variable equal to its previous value plus the equivalent of 1 percent of GDP when expressed as a percent of the previous period’s level of private investment.

\begin{sphinxuseclass}{cell}\begin{sphinxVerbatimInput}

\begin{sphinxuseclass}{cell_input}
\begin{sphinxVerbatim}[commandchars=\\\{\}]
\PYG{n}{AFShock}\PYG{o}{=}\PYG{n}{AFShock}\PYG{o}{.}\PYG{n}{mfcalc}\PYG{p}{(}\PYG{l+s+s2}{\PYGZdq{}}\PYG{l+s+s2}{\PYGZlt{}2028 2028\PYGZgt{} PAKNEGDIFPRVKN\PYGZus{}A = PAKNEGDIFPRVKN\PYGZus{}A + (.01*PAKNYGDPMKTPKN/PAKNEGDIKSTKKN)}\PYG{l+s+s2}{\PYGZdq{}}\PYG{p}{)}\PYG{p}{;}

\PYG{n+nb}{print}\PYG{p}{(}\PYG{l+s+s2}{\PYGZdq{}}\PYG{l+s+s2}{Post shock levels}\PYG{l+s+s2}{\PYGZdq{}}\PYG{p}{)}
\PYG{n}{AFShock}\PYG{o}{.}\PYG{n}{loc}\PYG{p}{[}\PYG{l+m+mi}{2025}\PYG{p}{:}\PYG{l+m+mi}{2030}\PYG{p}{,}\PYG{l+s+s1}{\PYGZsq{}}\PYG{l+s+s1}{PAKNEGDIFPRVKN\PYGZus{}A}\PYG{l+s+s1}{\PYGZsq{}}\PYG{p}{]}
\end{sphinxVerbatim}

\end{sphinxuseclass}\end{sphinxVerbatimInput}
\begin{sphinxVerbatimOutput}

\begin{sphinxuseclass}{cell_output}
\begin{sphinxVerbatim}[commandchars=\\\{\}]
Post shock levels
\end{sphinxVerbatim}

\begin{sphinxVerbatim}[commandchars=\\\{\}]
2025   \PYGZhy{}0.000458
2026   \PYGZhy{}0.000389
2027   \PYGZhy{}0.000331
2028    0.005774
2029   \PYGZhy{}0.000239
2030   \PYGZhy{}0.000203
Name: PAKNEGDIFPRVKN\PYGZus{}A, dtype: float64
\end{sphinxVerbatim}

\end{sphinxuseclass}\end{sphinxVerbatimOutput}

\end{sphinxuseclass}

\paragraph{Run the shock}
\label{\detokenize{content/05_WBModels/ScenarioAnalysis:run-the-shock}}
\begin{sphinxuseclass}{cell}\begin{sphinxVerbatimInput}

\begin{sphinxuseclass}{cell_input}
\begin{sphinxVerbatim}[commandchars=\\\{\}]
\PYG{n}{AFShockRes} \PYG{o}{=} \PYG{n}{mpak}\PYG{p}{(}\PYG{n}{AFShock}\PYG{p}{,}\PYG{l+m+mi}{2020}\PYG{p}{,}\PYG{l+m+mi}{2040}\PYG{p}{,}\PYG{n}{keep}\PYG{o}{=}\PYG{l+s+s1}{\PYGZsq{}}\PYG{l+s+s1}{1}\PYG{l+s+si}{\PYGZpc{} o}\PYG{l+s+s1}{f GDP increase in FDI and private investment (AF shock)}\PYG{l+s+s1}{\PYGZsq{}}\PYG{p}{)}
\PYG{n}{mpak}\PYG{p}{[}\PYG{l+s+s1}{\PYGZsq{}}\PYG{l+s+s1}{PAKNYGDPMKTPKN PAKNEGDIFPRVKN PAKNECONPRVTKN PAKNEIMPGNFSKN PAKNEGDIFTOTKN PAKNECONPRVTXN}\PYG{l+s+s1}{\PYGZsq{}}\PYG{p}{]}\PYG{o}{.}\PYG{n}{difpctlevel}\PYG{o}{.}\PYG{n}{mul100}\PYG{o}{.}\PYG{n}{plot}\PYG{p}{(}\PYG{n}{title}\PYG{o}{=}\PYG{l+s+s2}{\PYGZdq{}}\PYG{l+s+s2}{Add factor shock on private investment 1}\PYG{l+s+si}{\PYGZpc{} o}\PYG{l+s+s2}{f GDP}\PYG{l+s+s2}{\PYGZdq{}}\PYG{p}{)}
\end{sphinxVerbatim}

\end{sphinxuseclass}\end{sphinxVerbatimInput}
\begin{sphinxVerbatimOutput}

\begin{sphinxuseclass}{cell_output}
\noindent\sphinxincludegraphics{{d4950e081adfaf6a0176c276331778f597ded3e906e8f266c37ed929b53f4b9d}.png}

\end{sphinxuseclass}\end{sphinxVerbatimOutput}

\end{sphinxuseclass}

\chapter{Report writing and scenario results}
\label{\detokenize{content/05_WBModels/ScenarioAnalysis:report-writing-and-scenario-results}}
\sphinxAtStartPar
\sphinxcode{\sphinxupquote{ModelFlow}}, standard pandas routines and other python libraries like \sphinxcode{\sphinxupquote{Matplotlib}} and \sphinxcode{\sphinxupquote{Plotly}} can be used to visualize and compare dataframes and therefore the results, from scenarios – as indeed has been done in the preceding paragraphs.

\sphinxAtStartPar
In addition, \sphinxcode{\sphinxupquote{ModelFlow}} also provides several specific routines that make such comparisons easier.


\section{The Keep option}
\label{\detokenize{content/05_WBModels/ScenarioAnalysis:the-keep-option}}
\sphinxAtStartPar
The \sphinxstylestrong{Keep} option facilitates the comparison of results from different scenarios run on a give model object.  In each of the simulations executed above, the \sphinxcode{\sphinxupquote{keep}} option was activated. This causes the results from each simulation in a unique \sphinxcode{\sphinxupquote{DataFrame}} that can be identified by the descriptor given to it.


\section{The \sphinxstyleliteralintitle{\sphinxupquote{.keep\_plot()}} method}
\label{\detokenize{content/05_WBModels/ScenarioAnalysis:the-keep-plot-method}}
\sphinxAtStartPar
The keep\_plot method can be used to plot and compare results from the various scenarios that had been run earlier using the \sphinxcode{\sphinxupquote{keep=}} option.

\sphinxAtStartPar
By default the results across all scenarios for each selected variables will be shown on one chart at a time.

\begin{sphinxuseclass}{cell}\begin{sphinxVerbatimInput}

\begin{sphinxuseclass}{cell_input}
\begin{sphinxVerbatim}[commandchars=\\\{\}]
\PYG{n}{mpak}\PYG{o}{.}\PYG{n}{keep\PYGZus{}plot}\PYG{p}{(}\PYG{l+s+s1}{\PYGZsq{}}\PYG{l+s+s1}{PAKNYGDPMKTPCN PAKNECONPRVTKN PAKNEIMPGNFSKN}\PYG{l+s+s1}{\PYGZsq{}}\PYG{p}{,}  \PYG{n}{legend}\PYG{o}{=}\PYG{k+kc}{True}\PYG{p}{)}
\PYG{c+c1}{\PYGZsh{}show for each variable on a separate chart the results from each kept scenario}
\end{sphinxVerbatim}

\end{sphinxuseclass}\end{sphinxVerbatimInput}
\begin{sphinxVerbatimOutput}

\begin{sphinxuseclass}{cell_output}
\noindent\sphinxincludegraphics{{a80779fa393ddbd79cc8cdcfbb692a5a2f8f66ce26c19f461502efe8437620d8}.png}

\noindent\sphinxincludegraphics{{b3467fb82209da66debb03d797dea4955ccf452d3188b4ad5f9ae52c54ce4147}.png}

\noindent\sphinxincludegraphics{{eff7ebb23c31bb7fc082355adbdeb28ff091cd52c9e3a7596051d60822da9a9f}.png}

\begin{sphinxVerbatim}[commandchars=\\\{\}]
\PYGZob{}\PYGZsq{}PAKNYGDPMKTPCN\PYGZsq{}: \PYGZlt{}Figure size 1000x600 with 1 Axes\PYGZgt{},
 \PYGZsq{}PAKNECONPRVTKN\PYGZsq{}: \PYGZlt{}Figure size 1000x600 with 1 Axes\PYGZgt{},
 \PYGZsq{}PAKNEIMPGNFSKN\PYGZsq{}: \PYGZlt{}Figure size 1000x600 with 1 Axes\PYGZgt{}\PYGZcb{}
\end{sphinxVerbatim}

\end{sphinxuseclass}\end{sphinxVerbatimOutput}

\end{sphinxuseclass}

\subsection{\sphinxstyleliteralintitle{\sphinxupquote{keep\_plot()}} options}
\label{\detokenize{content/05_WBModels/ScenarioAnalysis:keep-plot-options}}
\sphinxAtStartPar
The \sphinxstylestrong{variables} to be displayed are listed as first argument. Variable names can include
wildcards (using * for any string and ? for any character).

\sphinxAtStartPar
\sphinxstylestrong{Transformation of data displayed:}


\begin{savenotes}\sphinxattablestart
\centering
\begin{tabulary}{\linewidth}[t]{|T|T|}
\hline
\sphinxstyletheadfamily 
\sphinxAtStartPar
showtype=
&\sphinxstyletheadfamily 
\sphinxAtStartPar
Use this operator
\\
\hline
\sphinxAtStartPar
‘level’ (default)
&
\sphinxAtStartPar
No transformation
\\
\hline
\sphinxAtStartPar
‘growth’
&
\sphinxAtStartPar
The growth rate  in percent
\\
\hline
\sphinxAtStartPar
‘change’
&
\sphinxAtStartPar
The yearly change (\(\Delta\))
\\
\hline
\end{tabulary}
\par
\sphinxattableend\end{savenotes}

\sphinxAtStartPar
\sphinxstylestrong{legend placement}


\begin{savenotes}\sphinxattablestart
\centering
\begin{tabulary}{\linewidth}[t]{|T|T|}
\hline
\sphinxstyletheadfamily 
\sphinxAtStartPar
legend=
&\sphinxstyletheadfamily 
\sphinxAtStartPar
Use this operator
\\
\hline
\sphinxAtStartPar
False (default)
&
\sphinxAtStartPar
The legends are  placed at the end of the corresponding line
\\
\hline
\sphinxAtStartPar
True
&
\sphinxAtStartPar
The legends are places in a legend box
\\
\hline
\end{tabulary}
\par
\sphinxattableend\end{savenotes}

\sphinxAtStartPar
Often it is useful to compare the scenario results with the baseline result. This is done with the diff argument.


\begin{savenotes}\sphinxattablestart
\centering
\begin{tabulary}{\linewidth}[t]{|T|T|}
\hline
\sphinxstyletheadfamily 
\sphinxAtStartPar
diff=
&\sphinxstyletheadfamily 
\sphinxAtStartPar
Use this operator
\\
\hline
\sphinxAtStartPar
False (default)
&
\sphinxAtStartPar
All entries in the keep\_solution dictionary are displayed
\\
\hline
\sphinxAtStartPar
True
&
\sphinxAtStartPar
The difference to the first entry is shown.
\\
\hline
\end{tabulary}
\par
\sphinxattableend\end{savenotes}

\sphinxAtStartPar
It can also be useful to compare the scenario results with the baseline result \sphinxstylestrong{measured in percent}. This is done with the diffpct argument.


\begin{savenotes}\sphinxattablestart
\centering
\begin{tabulary}{\linewidth}[t]{|T|T|}
\hline
\sphinxstyletheadfamily 
\sphinxAtStartPar
diffpct=
&\sphinxstyletheadfamily 
\sphinxAtStartPar
Use this operator
\\
\hline
\sphinxAtStartPar
False (default)
&
\sphinxAtStartPar
All entries in the keep\_solution dictionary is displayed
\\
\hline
\sphinxAtStartPar
True
&
\sphinxAtStartPar
The difference in percent to the first entry is shown
\\
\hline
\end{tabulary}
\par
\sphinxattableend\end{savenotes}

\begin{sphinxadmonition}{note}{Note:}
\sphinxAtStartPar
\sphinxcode{\sphinxupquote{'keep\_plot()}} and \sphinxcode{\sphinxupquote{.keep\_plot\_multi()}} return a python object that points to the in memory version of the rendered figure(s).  This object can be used to modify the graph (see examples towards the end of this chapter.
\end{sphinxadmonition}

\sphinxAtStartPar
\sphinxcode{\sphinxupquote{savefig='{[}path/{]}<prefix>.<extension>'}}
Will create a number of files with the charts.
The files will be saved location with name \sphinxcode{\sphinxupquote{<path>/<prefix><variable name>.<extension>}} The extension determines the
format of the saved file: \sphinxcode{\sphinxupquote{pdf}}, \sphinxcode{\sphinxupquote{svg}} and \sphinxcode{\sphinxupquote{png}} are the most common extensions.


\subsection{An example using the diff=TRUE option}
\label{\detokenize{content/05_WBModels/ScenarioAnalysis:an-example-using-the-diff-true-option}}
\sphinxAtStartPar
When \sphinxcode{\sphinxupquote{diff=True}} (or 1) results will be shown all of the selected scenarios presented as the change in selected variables with respect to the first scenario – in this instance the scenario saves with the name \sphinxcode{\sphinxupquote{baseline}}.

\begin{sphinxVerbatim}[commandchars=\\\{\}]
Note in this instance `baseline` and `basedf` are the same because they were defined that way.  However, there is nothing in the system that guarantees that the first `keep` scenario will be the baseline or the `basedf` scenario.
\end{sphinxVerbatim}

\begin{sphinxuseclass}{cell}\begin{sphinxVerbatimInput}

\begin{sphinxuseclass}{cell_input}
\begin{sphinxVerbatim}[commandchars=\\\{\}]
\PYG{n}{mpak}\PYG{o}{.}\PYG{n}{keep\PYGZus{}plot}\PYG{p}{(}\PYG{l+s+s1}{\PYGZsq{}}\PYG{l+s+s1}{PAKNYGDPMKTPCN PAKNECONPRVTKN PAKNEIMPGNFSKN}\PYG{l+s+s1}{\PYGZsq{}}\PYG{p}{,} \PYG{n}{diff}\PYG{o}{=}\PYG{l+m+mi}{1}\PYG{p}{,} \PYG{n}{legend}\PYG{o}{=}\PYG{k+kc}{True}\PYG{p}{)}
\end{sphinxVerbatim}

\end{sphinxuseclass}\end{sphinxVerbatimInput}
\begin{sphinxVerbatimOutput}

\begin{sphinxuseclass}{cell_output}
\noindent\sphinxincludegraphics{{4ed3a6faba74255578850aa87591a3ef66edfb86d94f60abf444c9784c98ddcb}.png}

\noindent\sphinxincludegraphics{{6068f7cfc39d5183906a46c9b317b7b30387deeef6e1f3b9f64f632ab44b955d}.png}

\noindent\sphinxincludegraphics{{1df642e2b515fb53cb88dfe69da3ca7c6ba66dea3e36f58cd65910bd96e4b7d1}.png}

\begin{sphinxVerbatim}[commandchars=\\\{\}]
\PYGZob{}\PYGZsq{}PAKNYGDPMKTPCN\PYGZsq{}: \PYGZlt{}Figure size 1000x600 with 1 Axes\PYGZgt{},
 \PYGZsq{}PAKNECONPRVTKN\PYGZsq{}: \PYGZlt{}Figure size 1000x600 with 1 Axes\PYGZgt{},
 \PYGZsq{}PAKNEIMPGNFSKN\PYGZsq{}: \PYGZlt{}Figure size 1000x600 with 1 Axes\PYGZgt{}\PYGZcb{}
\end{sphinxVerbatim}

\end{sphinxuseclass}\end{sphinxVerbatimOutput}

\end{sphinxuseclass}

\subsection{The \sphinxstyleliteralintitle{\sphinxupquote{showtype}} option}
\label{\detokenize{content/05_WBModels/ScenarioAnalysis:the-showtype-option}}
\sphinxAtStartPar
In this example the difference with respect first \sphinxcode{\sphinxupquote{'keep}} scenario \sphinxcode{\sphinxupquote{baseline}} values are once again shown. This time the \sphinxcode{\sphinxupquote{showtype}} option has been set to \sphinxcode{\sphinxupquote{growth}}.  As a result the data is displayed as the  difference in the growth rate.

\begin{sphinxuseclass}{cell}\begin{sphinxVerbatimInput}

\begin{sphinxuseclass}{cell_input}
\begin{sphinxVerbatim}[commandchars=\\\{\}]
\PYG{n}{mpak}\PYG{o}{.}\PYG{n}{keep\PYGZus{}plot}\PYG{p}{(}\PYG{l+s+s1}{\PYGZsq{}}\PYG{l+s+s1}{PAKNYGDPMKTPCN PAKNEIMPGNFSKN}\PYG{l+s+s1}{\PYGZsq{}}\PYG{p}{,} \PYG{n}{diff}\PYG{o}{=}\PYG{l+m+mi}{1}\PYG{p}{,}\PYG{n}{showtype}\PYG{o}{=}\PYG{l+s+s1}{\PYGZsq{}}\PYG{l+s+s1}{growth}\PYG{l+s+s1}{\PYGZsq{}}\PYG{p}{,} \PYG{n}{legend}\PYG{o}{=}\PYG{k+kc}{True}\PYG{p}{)}
\end{sphinxVerbatim}

\end{sphinxuseclass}\end{sphinxVerbatimInput}
\begin{sphinxVerbatimOutput}

\begin{sphinxuseclass}{cell_output}
\noindent\sphinxincludegraphics{{90bbf1d36ebb87ef3938c818ea87f9e7d4a087c25792b6c4ce335d9dd7ddf4e3}.png}

\noindent\sphinxincludegraphics{{193460901e5b43bf8e7c979cb8b4ebd41896b733bd18742b47e277ba8c4aec72}.png}

\begin{sphinxVerbatim}[commandchars=\\\{\}]
\PYGZob{}\PYGZsq{}PAKNYGDPMKTPCN\PYGZsq{}: \PYGZlt{}Figure size 1000x600 with 1 Axes\PYGZgt{},
 \PYGZsq{}PAKNEIMPGNFSKN\PYGZsq{}: \PYGZlt{}Figure size 1000x600 with 1 Axes\PYGZgt{}\PYGZcb{}
\end{sphinxVerbatim}

\end{sphinxuseclass}\end{sphinxVerbatimOutput}

\end{sphinxuseclass}

\subsection{The diffpct option}
\label{\detokenize{content/05_WBModels/ScenarioAnalysis:the-diffpct-option}}
\sphinxAtStartPar
Setting \sphinxcode{\sphinxupquote{diffpct=True}} instructs \sphinxcode{\sphinxupquote{.keep\_plot()}} to display the data as a percent deviation from the first \sphinxcode{\sphinxupquote{kep}} scenario.

\begin{sphinxuseclass}{cell}\begin{sphinxVerbatimInput}

\begin{sphinxuseclass}{cell_input}
\begin{sphinxVerbatim}[commandchars=\\\{\}]
\PYG{n}{mpak}\PYG{o}{.}\PYG{n}{keep\PYGZus{}plot}\PYG{p}{(}\PYG{l+s+s1}{\PYGZsq{}}\PYG{l+s+s1}{PAKNYGDPMKTPCN  PAKNEIMPGNFSKN}\PYG{l+s+s1}{\PYGZsq{}}\PYG{p}{,} \PYG{n}{diffpct}\PYG{o}{=}\PYG{l+m+mi}{1}\PYG{p}{,}\PYG{n}{legend}\PYG{o}{=}\PYG{l+s+s2}{\PYGZdq{}}\PYG{l+s+s2}{Change in level as a }\PYG{l+s+si}{\PYGZpc{} o}\PYG{l+s+s2}{f first keep scenario}\PYG{l+s+s2}{\PYGZdq{}}\PYG{p}{)}
\end{sphinxVerbatim}

\end{sphinxuseclass}\end{sphinxVerbatimInput}
\begin{sphinxVerbatimOutput}

\begin{sphinxuseclass}{cell_output}
\noindent\sphinxincludegraphics{{4b7a0f485c3bda9b03cd6ef8415efe8594fce756039adbbc3eba4b1a904fccb9}.png}

\noindent\sphinxincludegraphics{{6278f82621bf975487c10a155f909c73cf0dfeb85d9725aa26c5989599943638}.png}

\begin{sphinxVerbatim}[commandchars=\\\{\}]
\PYGZob{}\PYGZsq{}PAKNYGDPMKTPCN\PYGZsq{}: \PYGZlt{}Figure size 1000x600 with 1 Axes\PYGZgt{},
 \PYGZsq{}PAKNEIMPGNFSKN\PYGZsq{}: \PYGZlt{}Figure size 1000x600 with 1 Axes\PYGZgt{}\PYGZcb{}
\end{sphinxVerbatim}

\end{sphinxuseclass}\end{sphinxVerbatimOutput}

\end{sphinxuseclass}

\subsubsection{Differences in percent of baseline values}
\label{\detokenize{content/05_WBModels/ScenarioAnalysis:differences-in-percent-of-baseline-values}}
\sphinxAtStartPar
In this plot, the same results are presented, but as percent deviations from the baseline values of the displayed data.

\begin{sphinxuseclass}{cell}\begin{sphinxVerbatimInput}

\begin{sphinxuseclass}{cell_input}
\begin{sphinxVerbatim}[commandchars=\\\{\}]
\PYG{n}{mpak}\PYG{o}{.}\PYG{n}{keep\PYGZus{}plot}\PYG{p}{(}\PYG{l+s+s1}{\PYGZsq{}}\PYG{l+s+s1}{PAKNYGDPMKTPCN PAKNECONPRVTKN }\PYG{l+s+s1}{\PYGZsq{}}\PYG{p}{,} \PYG{n}{diffpct}\PYG{o}{=}\PYG{l+m+mi}{1}\PYG{p}{,}\PYG{n}{showtype}\PYG{o}{=}\PYG{l+s+s1}{\PYGZsq{}}\PYG{l+s+s1}{level}\PYG{l+s+s1}{\PYGZsq{}}\PYG{p}{,} \PYG{n}{legend}\PYG{o}{=}\PYG{k+kc}{True}\PYG{p}{)}
\end{sphinxVerbatim}

\end{sphinxuseclass}\end{sphinxVerbatimInput}
\begin{sphinxVerbatimOutput}

\begin{sphinxuseclass}{cell_output}
\noindent\sphinxincludegraphics{{4b7a0f485c3bda9b03cd6ef8415efe8594fce756039adbbc3eba4b1a904fccb9}.png}

\noindent\sphinxincludegraphics{{1be61a7c5fe15786c3eab55cc284214194a1eed7ec84ff124b3ef644f10f4215}.png}

\begin{sphinxVerbatim}[commandchars=\\\{\}]
\PYGZob{}\PYGZsq{}PAKNYGDPMKTPCN\PYGZsq{}: \PYGZlt{}Figure size 1000x600 with 1 Axes\PYGZgt{},
 \PYGZsq{}PAKNECONPRVTKN\PYGZsq{}: \PYGZlt{}Figure size 1000x600 with 1 Axes\PYGZgt{}\PYGZcb{}
\end{sphinxVerbatim}

\end{sphinxuseclass}\end{sphinxVerbatimOutput}

\end{sphinxuseclass}

\subsection{The \sphinxstyleliteralintitle{\sphinxupquote{.keep\_switch()}} method}
\label{\detokenize{content/05_WBModels/ScenarioAnalysis:the-keep-switch-method}}
\sphinxAtStartPar
The \sphinxcode{\sphinxupquote{.keep\_switch()}} method restricts the number of scenarios on which subsequent calls to \sphinxcode{\sphinxupquote{.keep\_plot()}} (and \sphinxcode{\sphinxupquote{.keep\_plot\_multi()}}) are executed on.  \sphinxcode{\sphinxupquote{.keep\_switch()}} can be passed a list of scenarios or using a wildcard selector.


\subsubsection{The \sphinxstyleliteralintitle{\sphinxupquote{.keep\_solutions.keys()}} method}
\label{\detokenize{content/05_WBModels/ScenarioAnalysis:the-keep-solutions-keys-method}}
\sphinxAtStartPar
The \sphinxcode{\sphinxupquote{.keep\_solutions.keys()}} method generates a list of the solutions that have been kept previously.

\begin{sphinxuseclass}{cell}\begin{sphinxVerbatimInput}

\begin{sphinxuseclass}{cell_input}
\begin{sphinxVerbatim}[commandchars=\\\{\}]
\PYG{n}{mpak}\PYG{o}{.}\PYG{n}{keep\PYGZus{}solutions}\PYG{o}{.}\PYG{n}{keys}\PYG{p}{(}\PYG{p}{)}
\end{sphinxVerbatim}

\end{sphinxuseclass}\end{sphinxVerbatimInput}
\begin{sphinxVerbatimOutput}

\begin{sphinxuseclass}{cell_output}
\begin{sphinxVerbatim}[commandchars=\\\{\}]
dict\PYGZus{}keys([\PYGZsq{}Baseline\PYGZsq{}, \PYGZsq{}\PYGZdl{}25 increase in oil prices 2025\PYGZhy{}27\PYGZsq{}, \PYGZsq{}2.5\PYGZpc{} increase in C 2025\PYGZhy{}40\PYGZsq{}, \PYGZsq{}2.5\PYGZpc{} increase in C 2025\PYGZhy{}27 \PYGZhy{}\PYGZhy{} exog whole period\PYGZsq{}, \PYGZsq{}2.5\PYGZpc{} increase in C 2025\PYGZhy{}27 \PYGZhy{}\PYGZhy{} exog whole period \PYGZhy{}\PYGZhy{}KG=True\PYGZsq{}, \PYGZsq{}2.5\PYGZpc{} increase in C 2025\PYGZhy{}27 \PYGZhy{}\PYGZhy{} temporarily exogenized\PYGZsq{}, \PYGZsq{}1\PYGZpc{} of GDP increase in FDI and private investment (AF shock)\PYGZsq{}])
\end{sphinxVerbatim}

\end{sphinxuseclass}\end{sphinxVerbatimOutput}

\end{sphinxuseclass}
\sphinxAtStartPar
To specify exactly which scenarios to show in a keep\_plot, the \sphinxcode{\sphinxupquote{scenarios=}} option of \sphinxcode{\sphinxupquote{.keepswitch()}} must be initialized with a “|” delimited string of the names of the scenarios (retrieved above) that are to be displayed.

\sphinxAtStartPar
By placing the \sphinxcode{\sphinxupquote{.keepswitch()}} in a \sphinxcode{\sphinxupquote{with}} clause the scenario restriction will only apply to indented lines that follow the with construct.

\begin{sphinxuseclass}{cell}\begin{sphinxVerbatimInput}

\begin{sphinxuseclass}{cell_input}
\begin{sphinxVerbatim}[commandchars=\\\{\}]
\PYG{k}{with} \PYG{n}{mpak}\PYG{o}{.}\PYG{n}{keepswitch}\PYG{p}{(}\PYG{n}{scenarios}\PYG{o}{=}\PYG{l+s+s1}{\PYGZsq{}}\PYG{l+s+s1}{2.5}\PYG{l+s+si}{\PYGZpc{} i}\PYG{l+s+s1}{ncrease in C 2025\PYGZhy{}40|2.5}\PYG{l+s+si}{\PYGZpc{} i}\PYG{l+s+s1}{ncrease in C 2025\PYGZhy{}27 \PYGZhy{}\PYGZhy{} exog whole period|2.5}\PYG{l+s+si}{\PYGZpc{} i}\PYG{l+s+s1}{ncrease in C 2025\PYGZhy{}27 \PYGZhy{}\PYGZhy{} exog whole period \PYGZhy{}\PYGZhy{}KG=True|2.5}\PYG{l+s+si}{\PYGZpc{} i}\PYG{l+s+s1}{ncrease in C 2025\PYGZhy{}27 \PYGZhy{}\PYGZhy{} temporarily exogenized}\PYG{l+s+s1}{\PYGZsq{}}\PYG{p}{)}\PYG{p}{:}
    \PYG{n}{mpak}\PYG{o}{.}\PYG{n}{keep\PYGZus{}plot}\PYG{p}{(}\PYG{l+s+s1}{\PYGZsq{}}\PYG{l+s+s1}{PAKNYGDPMKTPKN PAKGGBALOVRLCN PAKGGDEBTTOTLCN}\PYG{l+s+s1}{\PYGZsq{}}\PYG{p}{,}\PYG{n}{diff}\PYG{o}{=}\PYG{k+kc}{False}\PYG{p}{,}\PYG{n}{showtype}\PYG{o}{=}\PYG{l+s+s1}{\PYGZsq{}}\PYG{l+s+s1}{growth}\PYG{l+s+s1}{\PYGZsq{}}\PYG{p}{,}\PYG{n}{legend}\PYG{o}{=}\PYG{k+kc}{True}\PYG{p}{)}\PYG{p}{;}
\end{sphinxVerbatim}

\end{sphinxuseclass}\end{sphinxVerbatimInput}
\begin{sphinxVerbatimOutput}

\begin{sphinxuseclass}{cell_output}
\noindent\sphinxincludegraphics{{4d4082f1eb35f31728b16dfc3ce40cde3588dab0fdb1f28afb94397881f0437f}.png}

\noindent\sphinxincludegraphics{{79c4faa9eebd093cebdf32aa5f4137a7d51775132845aedc327b97c04275222a}.png}

\end{sphinxuseclass}\end{sphinxVerbatimOutput}

\end{sphinxuseclass}

\subsubsection{Keepswitch with wildcard selection}
\label{\detokenize{content/05_WBModels/ScenarioAnalysis:keepswitch-with-wildcard-selection}}
\sphinxAtStartPar
Below we generate a series of plots using

\begin{sphinxuseclass}{cell}\begin{sphinxVerbatimInput}

\begin{sphinxuseclass}{cell_input}
\begin{sphinxVerbatim}[commandchars=\\\{\}]
\PYG{k}{with} \PYG{n}{mpak}\PYG{o}{.}\PYG{n}{keepswitch}\PYG{p}{(}\PYG{n}{scenarios}\PYG{o}{=}\PYG{l+s+s1}{\PYGZsq{}}\PYG{l+s+s1}{*2025*}\PYG{l+s+s1}{\PYGZsq{}}\PYG{p}{)}\PYG{p}{:}
    \PYG{n}{mpak}\PYG{o}{.}\PYG{n}{keep\PYGZus{}plot}\PYG{p}{(}\PYG{l+s+s1}{\PYGZsq{}}\PYG{l+s+s1}{PAKNYGDPMKTPKN PAKNECONPRVTKN}\PYG{l+s+s1}{\PYGZsq{}}\PYG{p}{,}\PYG{n}{showtype}\PYG{o}{=}\PYG{l+s+s1}{\PYGZsq{}}\PYG{l+s+s1}{growth}\PYG{l+s+s1}{\PYGZsq{}}\PYG{p}{,}\PYG{n}{legend}\PYG{o}{=}\PYG{k+kc}{True}\PYG{p}{)}\PYG{p}{;}
\end{sphinxVerbatim}

\end{sphinxuseclass}\end{sphinxVerbatimInput}
\begin{sphinxVerbatimOutput}

\begin{sphinxuseclass}{cell_output}
\noindent\sphinxincludegraphics{{7d1aa1ca7b80bc840d25f12a2679f4c190d82eecf72cf6f12a4d247df9830ea2}.png}

\noindent\sphinxincludegraphics{{53ecd99caf80c691a952bb6131692e98a3a3fe68aac682da277baf00fe32b45e}.png}

\end{sphinxuseclass}\end{sphinxVerbatimOutput}

\end{sphinxuseclass}

\section{The \sphinxstyleliteralintitle{\sphinxupquote{.keep\_plot\_multi()}} method}
\label{\detokenize{content/05_WBModels/ScenarioAnalysis:the-keep-plot-multi-method}}
\sphinxAtStartPar
The \sphinxcode{\sphinxupquote{.keep\_plot\_multi()}} method allows several charts to be displayed in a grid.  The size of each chart can be set with the \sphinxcode{\sphinxupquote{size=(w,h)}} option, where the units of with and height are in centimetres.

\begin{sphinxuseclass}{cell}\begin{sphinxVerbatimInput}

\begin{sphinxuseclass}{cell_input}
\begin{sphinxVerbatim}[commandchars=\\\{\}]
\PYG{k}{with} \PYG{n}{mpak}\PYG{o}{.}\PYG{n}{set\PYGZus{}smpl}\PYG{p}{(}\PYG{l+m+mi}{2000}\PYG{p}{,}\PYG{l+m+mi}{2040}\PYG{p}{)}\PYG{p}{:}
    \PYG{k}{with} \PYG{n}{mpak}\PYG{o}{.}\PYG{n}{keepswitch}\PYG{p}{(}\PYG{n}{scenarios}\PYG{o}{=}\PYG{l+s+s2}{\PYGZdq{}}\PYG{l+s+s2}{baseline *exog*}\PYG{l+s+s2}{\PYGZdq{}}\PYG{p}{)}\PYG{p}{:}
        \PYG{n}{var\PYGZus{}figs} \PYG{o}{=} \PYG{n}{mpak}\PYG{o}{.}\PYG{n}{keep\PYGZus{}plot\PYGZus{}multi}\PYG{p}{(}\PYG{l+s+s1}{\PYGZsq{}}\PYG{l+s+s1}{PAKNYGDPMKTPKN PAKNECONPRVTKN PAKNEGDIFTOTKN PAKNEIMPGNFSKN PAKNEEXPGNFSKN}\PYG{l+s+s1}{\PYGZsq{}}\PYG{p}{,}\PYG{l+m+mi}{2010}\PYG{p}{,}\PYG{l+m+mi}{2100}\PYG{p}{,}\PYG{n}{keep\PYGZus{}dim}\PYG{o}{=}\PYG{l+m+mi}{0}\PYG{p}{,}\PYG{n}{legend}\PYG{o}{=}\PYG{l+m+mi}{1}
                                \PYG{p}{,}\PYG{n}{size}\PYG{o}{=}\PYG{p}{(}\PYG{l+m+mi}{20}\PYG{p}{,}\PYG{l+m+mi}{20}\PYG{p}{)} \PYG{p}{,}\PYG{n}{diffpct}\PYG{o}{=}\PYG{k+kc}{True}\PYG{p}{,}\PYG{n}{title}\PYG{o}{=}\PYG{l+s+s1}{\PYGZsq{}}\PYG{l+s+s1}{\PYGZsq{}}  \PYG{p}{)}\PYG{p}{;}
\end{sphinxVerbatim}

\end{sphinxuseclass}\end{sphinxVerbatimInput}
\begin{sphinxVerbatimOutput}

\begin{sphinxuseclass}{cell_output}
\noindent\sphinxincludegraphics{{110b073013bb48ae907f8e787aab3de274c42258c568c7e059e8bf4220299939}.png}

\noindent\sphinxincludegraphics{{d0207b592c68e6eba3c886c8bf3e4c0e913066366b28e2733a47ee824b534250}.png}

\noindent\sphinxincludegraphics{{2aa184207903a01c4f7797817edf3b2310faa67a91efd0eba85f80edc6e9dad5}.png}

\end{sphinxuseclass}\end{sphinxVerbatimOutput}

\end{sphinxuseclass}
\sphinxAtStartPar
As indicated earlier both \sphinxcode{\sphinxupquote{keep\_plot()}} and \sphinxcode{\sphinxupquote{keep\_plot\_multi()}} return a variable that can be used to embellish or modify the figures produced by the automatic routines.

\sphinxAtStartPar
For example the charts can be resized.

\begin{sphinxuseclass}{cell}\begin{sphinxVerbatimInput}

\begin{sphinxuseclass}{cell_input}
\begin{sphinxVerbatim}[commandchars=\\\{\}]
\PYG{n}{var\PYGZus{}figs}\PYG{o}{.}\PYG{n}{set\PYGZus{}size\PYGZus{}inches}\PYG{p}{(}\PYG{l+m+mi}{15}\PYG{p}{,}\PYG{l+m+mi}{10}\PYG{p}{)}
\PYG{n}{var\PYGZus{}figs}
\end{sphinxVerbatim}

\end{sphinxuseclass}\end{sphinxVerbatimInput}
\begin{sphinxVerbatimOutput}

\begin{sphinxuseclass}{cell_output}
\noindent\sphinxincludegraphics{{0fde1c76130e7f66f8852126d948739ff4016752f6fcb6d6abe4b255a3d4e1b8}.png}

\end{sphinxuseclass}\end{sphinxVerbatimOutput}

\end{sphinxuseclass}
\sphinxAtStartPar
Individual charts can be deleted from the grid.

\begin{sphinxadmonition}{note}{Note:}
\sphinxAtStartPar
The grid representation of the individual charts is returned as a 0\sphinxhyphen{}based vector of charts.  Thus the first figure is the zeroeth and the second is the first.
\end{sphinxadmonition}


\subsection{Delete a chart from th grid}
\label{\detokenize{content/05_WBModels/ScenarioAnalysis:delete-a-chart-from-th-grid}}
\sphinxAtStartPar
A chart can be deleted from the grid by referencing it and calling the \sphinxcode{\sphinxupquote{.remove()}} method.

\begin{sphinxuseclass}{cell}\begin{sphinxVerbatimInput}

\begin{sphinxuseclass}{cell_input}
\begin{sphinxVerbatim}[commandchars=\\\{\}]
\PYG{n}{var\PYGZus{}figs}\PYG{o}{.}\PYG{n}{axes}\PYG{p}{[}\PYG{l+m+mi}{1}\PYG{p}{]}\PYG{o}{.}\PYG{n}{remove}\PYG{p}{(}\PYG{p}{)}
\PYG{n}{var\PYGZus{}figs}
\end{sphinxVerbatim}

\end{sphinxuseclass}\end{sphinxVerbatimInput}
\begin{sphinxVerbatimOutput}

\begin{sphinxuseclass}{cell_output}
\noindent\sphinxincludegraphics{{cecac040212a456116c95e7cd162e606f5d337a4f87674b97c1c1c58922d1234}.png}

\end{sphinxuseclass}\end{sphinxVerbatimOutput}

\end{sphinxuseclass}
\sphinxAtStartPar
The same mechanism can be used to revise the titles of the indidividual charts and annotate them.

\begin{sphinxuseclass}{cell}\begin{sphinxVerbatimInput}

\begin{sphinxuseclass}{cell_input}
\begin{sphinxVerbatim}[commandchars=\\\{\}]
\PYG{n}{var\PYGZus{}figs}\PYG{o}{.}\PYG{n}{axes}\PYG{p}{[}\PYG{l+m+mi}{0}\PYG{p}{]}\PYG{o}{.}\PYG{n}{set\PYGZus{}title}\PYG{p}{(}\PYG{l+s+s1}{\PYGZsq{}}\PYG{l+s+s1}{Impact of a 2.5}\PYG{l+s+si}{\PYGZpc{} i}\PYG{l+s+s1}{ncrease in C using exog method}\PYG{l+s+s1}{\PYGZsq{}}\PYG{p}{)}\PYG{p}{;}    \PYG{c+c1}{\PYGZsh{} many properties can be set afterward }
\PYG{n}{var\PYGZus{}figs}\PYG{o}{.}\PYG{n}{axes}\PYG{p}{[}\PYG{l+m+mi}{1}\PYG{p}{]}\PYG{o}{.}\PYG{n}{set\PYGZus{}title}\PYG{p}{(}\PYG{l+s+s1}{\PYGZsq{}}\PYG{l+s+s1}{Impact of a 2.5}\PYG{l+s+si}{\PYGZpc{} i}\PYG{l+s+s1}{ncrease in C using temporary exog method}\PYG{l+s+s1}{\PYGZsq{}}\PYG{p}{)}\PYG{p}{;}
\PYG{n}{var\PYGZus{}figs}
\end{sphinxVerbatim}

\end{sphinxuseclass}\end{sphinxVerbatimInput}
\begin{sphinxVerbatimOutput}

\begin{sphinxuseclass}{cell_output}
\noindent\sphinxincludegraphics{{da77be21155cd4f2f0166647e86314721d8a52769a63a2c2239a87fe8fcee618}.png}

\end{sphinxuseclass}\end{sphinxVerbatimOutput}

\end{sphinxuseclass}
\begin{sphinxuseclass}{cell}\begin{sphinxVerbatimInput}

\begin{sphinxuseclass}{cell_input}
\begin{sphinxVerbatim}[commandchars=\\\{\}]
\PYG{n}{var\PYGZus{}figs}\PYG{o}{.}\PYG{n}{axes}\PYG{p}{[}\PYG{l+m+mi}{0}\PYG{p}{]}\PYG{o}{.}\PYG{n}{set\PYGZus{}xlabel}\PYG{p}{(}\PYG{l+s+s1}{\PYGZsq{}}\PYG{l+s+s1}{Year}\PYG{l+s+s1}{\PYGZsq{}}\PYG{p}{)}
\PYG{n}{var\PYGZus{}figs}\PYG{o}{.}\PYG{n}{axes}\PYG{p}{[}\PYG{l+m+mi}{0}\PYG{p}{]}\PYG{o}{.}\PYG{n}{set\PYGZus{}ylabel}\PYG{p}{(}\PYG{l+s+s1}{\PYGZsq{}}\PYG{l+s+s1}{Change percent}\PYG{l+s+se}{\PYGZbs{}n}\PYG{l+s+s1}{change in level}\PYG{l+s+s1}{\PYGZsq{}}\PYG{p}{,}\PYG{n}{fontsize}\PYG{o}{=}\PYG{l+m+mi}{10}\PYG{p}{)}
\PYG{n}{var\PYGZus{}figs}\PYG{o}{.}\PYG{n}{axes}\PYG{p}{[}\PYG{l+m+mi}{0}\PYG{p}{]}\PYG{o}{.}\PYG{n}{yaxis}\PYG{o}{.}\PYG{n}{set\PYGZus{}label\PYGZus{}coords}\PYG{p}{(}\PYG{o}{\PYGZhy{}}\PYG{l+m+mf}{0.1}\PYG{p}{,}\PYG{l+m+mf}{1.02}\PYG{p}{)}
\PYG{n}{var\PYGZus{}figs}\PYG{o}{.}\PYG{n}{axes}\PYG{p}{[}\PYG{l+m+mi}{0}\PYG{p}{]}\PYG{o}{.}\PYG{n}{xaxis}\PYG{o}{.}\PYG{n}{set\PYGZus{}label\PYGZus{}coords}\PYG{p}{(}\PYG{l+m+mf}{.95}\PYG{p}{,}\PYG{o}{\PYGZhy{}}\PYG{l+m+mf}{.06}\PYG{p}{)}
\PYG{n}{var\PYGZus{}figs}
\end{sphinxVerbatim}

\end{sphinxuseclass}\end{sphinxVerbatimInput}
\begin{sphinxVerbatimOutput}

\begin{sphinxuseclass}{cell_output}
\noindent\sphinxincludegraphics{{8854e8c856d491acf263a49b3187e73e76e83c41f1c7a15802bec25c688ffb45}.png}

\end{sphinxuseclass}\end{sphinxVerbatimOutput}

\end{sphinxuseclass}
\sphinxAtStartPar
Variables pointing to the individual charts can be defined and used to make modifications to individual charts within the overall figure.

\begin{sphinxuseclass}{cell}\begin{sphinxVerbatimInput}

\begin{sphinxuseclass}{cell_input}
\begin{sphinxVerbatim}[commandchars=\\\{\}]
\PYG{n}{fig1}\PYG{o}{=}\PYG{n}{var\PYGZus{}figs}\PYG{o}{.}\PYG{n}{axes}\PYG{p}{[}\PYG{l+m+mi}{0}\PYG{p}{]}
\PYG{n}{fig2}\PYG{o}{=}\PYG{n}{var\PYGZus{}figs}\PYG{o}{.}\PYG{n}{axes}\PYG{p}{[}\PYG{l+m+mi}{1}\PYG{p}{]}
\end{sphinxVerbatim}

\end{sphinxuseclass}\end{sphinxVerbatimInput}

\end{sphinxuseclass}
\begin{sphinxuseclass}{cell}\begin{sphinxVerbatimInput}

\begin{sphinxuseclass}{cell_input}
\begin{sphinxVerbatim}[commandchars=\\\{\}]
\PYG{n}{fig2}\PYG{o}{.}\PYG{n}{set\PYGZus{}ylabel}\PYG{p}{(}\PYG{l+s+s1}{\PYGZsq{}}\PYG{l+s+s1}{Percent change}\PYG{l+s+se}{\PYGZbs{}n}\PYG{l+s+s1}{in level}\PYG{l+s+s1}{\PYGZsq{}}\PYG{p}{,}\PYG{n}{fontsize}\PYG{o}{=}\PYG{l+m+mi}{10}\PYG{p}{)}
\PYG{n}{fig2}\PYG{o}{.}\PYG{n}{yaxis}\PYG{o}{.}\PYG{n}{set\PYGZus{}label\PYGZus{}coords}\PYG{p}{(}\PYG{o}{\PYGZhy{}}\PYG{l+m+mf}{0.1}\PYG{p}{,}\PYG{l+m+mf}{1.02}\PYG{p}{)} \PYG{c+c1}{\PYGZsh{}place axes labels}
\PYG{n}{fig2}\PYG{o}{.}\PYG{n}{xaxis}\PYG{o}{.}\PYG{n}{set\PYGZus{}label\PYGZus{}coords}\PYG{p}{(}\PYG{l+m+mf}{.95}\PYG{p}{,}\PYG{o}{\PYGZhy{}}\PYG{l+m+mf}{.06}\PYG{p}{)}



\PYG{n}{fig2}\PYG{o}{.}\PYG{n}{text}\PYG{p}{(}\PYG{l+m+mf}{2040.}\PYG{p}{,}\PYG{l+m+mf}{0.4}\PYG{p}{,} \PYG{l+s+s1}{\PYGZsq{}}\PYG{l+s+s1}{Some text in a box}\PYG{l+s+s1}{\PYGZsq{}}\PYG{p}{,} 
          \PYG{n}{color}\PYG{o}{=}\PYG{l+s+s1}{\PYGZsq{}}\PYG{l+s+s1}{yellow}\PYG{l+s+s1}{\PYGZsq{}}\PYG{p}{,}\PYG{n}{bbox}\PYG{o}{=}\PYG{n+nb}{dict}\PYG{p}{(}\PYG{n}{facecolor}\PYG{o}{=}\PYG{l+s+s1}{\PYGZsq{}}\PYG{l+s+s1}{red}\PYG{l+s+s1}{\PYGZsq{}}\PYG{p}{,} \PYG{n}{alpha}\PYG{o}{=}\PYG{l+m+mf}{0.5}\PYG{p}{)}\PYG{p}{)}\PYG{p}{;}
\PYG{n}{fig2}\PYG{o}{.}\PYG{n}{text}\PYG{p}{(}\PYG{l+m+mf}{2040.}\PYG{p}{,}\PYG{l+m+mf}{0.1}\PYG{p}{,} \PYG{l+s+s1}{\PYGZsq{}}\PYG{l+s+s1}{Some nice text}\PYG{l+s+s1}{\PYGZsq{}}\PYG{p}{,} 
          \PYG{n}{style}\PYG{o}{=}\PYG{l+s+s1}{\PYGZsq{}}\PYG{l+s+s1}{italic}\PYG{l+s+s1}{\PYGZsq{}}\PYG{p}{,}\PYG{n}{color}\PYG{o}{=}\PYG{l+s+s1}{\PYGZsq{}}\PYG{l+s+s1}{green}\PYG{l+s+s1}{\PYGZsq{}}\PYG{p}{)}\PYG{p}{;}
          
\PYG{n}{var\PYGZus{}figs}
          
\end{sphinxVerbatim}

\end{sphinxuseclass}\end{sphinxVerbatimInput}
\begin{sphinxVerbatimOutput}

\begin{sphinxuseclass}{cell_output}
\noindent\sphinxincludegraphics{{6731336bfbd25841772227371b0c5d2df598a1407e399fd694bdf65d4e024f82}.png}

\end{sphinxuseclass}\end{sphinxVerbatimOutput}

\end{sphinxuseclass}

\subsection{The results visualization widget view}
\label{\detokenize{content/05_WBModels/ScenarioAnalysis:the-results-visualization-widget-view}}
\sphinxAtStartPar
When working in Jupyter Notebook, referencing a selection of series will cause a data visualization widget to be generated that allows you to look at results (\sphinxcode{\sphinxupquote{basesdf}} vs \sphinxcode{\sphinxupquote{latestdf}}) for the selected variables as tables or charts, as levels, as growth rates and as percent differences from baseline.

\begin{sphinxuseclass}{cell}\begin{sphinxVerbatimInput}

\begin{sphinxuseclass}{cell_input}
\begin{sphinxVerbatim}[commandchars=\\\{\}]
\PYG{n}{mpak}\PYG{p}{[}\PYG{l+s+s1}{\PYGZsq{}}\PYG{l+s+s1}{PAKNYGDPMKTPCN PAKNYGDPMKTPKN PAKGGEXPTOTLCN PAKGGREVTOTLCN PAKNECONGOVTKN}\PYG{l+s+s1}{\PYGZsq{}}\PYG{p}{]}
\end{sphinxVerbatim}

\end{sphinxuseclass}\end{sphinxVerbatimInput}
\begin{sphinxVerbatimOutput}

\begin{sphinxuseclass}{cell_output}
\begin{sphinxVerbatim}[commandchars=\\\{\}]
Tab(children=(Tab(children=(HTML(value=\PYGZsq{}\PYGZlt{}?xml version=\PYGZdq{}1.0\PYGZdq{} encoding=\PYGZdq{}utf\PYGZhy{}8\PYGZdq{} standalone=\PYGZdq{}no\PYGZdq{}?\PYGZgt{}\PYGZbs{}n\PYGZlt{}!DOCTYPE svg …
\end{sphinxVerbatim}

\begin{sphinxVerbatim}[commandchars=\\\{\}]

\end{sphinxVerbatim}

\end{sphinxuseclass}\end{sphinxVerbatimOutput}

\end{sphinxuseclass}
\sphinxstepscope

\begin{sphinxuseclass}{cell}\begin{sphinxVerbatimInput}

\begin{sphinxuseclass}{cell_input}
\begin{sphinxVerbatim}[commandchars=\\\{\}]
\PYG{o}{\PYGZpc{}}\PYG{k}{matplotlib} inline
\end{sphinxVerbatim}

\end{sphinxuseclass}\end{sphinxVerbatimInput}

\end{sphinxuseclass}

\chapter{More complex scenarios}
\label{\detokenize{content/05_WBModels/MoreComplexScenarios:more-complex-scenarios}}\label{\detokenize{content/05_WBModels/MoreComplexScenarios::doc}}
\sphinxAtStartPar
The preceding chapter introduced four different ways of preparing a solution and the forms the backbone of running simulations on World Bank models in \sphinxcode{\sphinxupquote{modelflow}}. This chapter builds on those examples and delves into some of the challenges involved in translating a real\sphinxhyphen{}world policy challenge into the model\sphinxhyphen{}world and then back again.

\sphinxAtStartPar
The particular problem to be examined is in the introduction of a Carbon Tax. The model used and the example presented are both taken from the model of Pakistan presented here: \hyperlink{cite.content/99_BackMatter/References:id14}{Burns \sphinxstyleemphasis{et al.}}.


\section{Setting up the environment}
\label{\detokenize{content/05_WBModels/MoreComplexScenarios:setting-up-the-environment}}
\sphinxAtStartPar
As always the modelflow and other python libraries that are to be used must be imported into the current session.

\begin{sphinxuseclass}{cell}\begin{sphinxVerbatimInput}

\begin{sphinxuseclass}{cell_input}
\begin{sphinxVerbatim}[commandchars=\\\{\}]
\PYG{k+kn}{from} \PYG{n+nn}{modelclass} \PYG{k+kn}{import} \PYG{n}{model} 
\PYG{n}{model}\PYG{o}{.}\PYG{n}{widescreen}\PYG{p}{(}\PYG{p}{)}
\PYG{n}{model}\PYG{o}{.}\PYG{n}{scroll\PYGZus{}off}\PYG{p}{(}\PYG{p}{)}
\PYG{o}{\PYGZpc{}}\PYG{k}{load\PYGZus{}ext} autoreload
\PYG{o}{\PYGZpc{}}\PYG{k}{autoreload} 2
\end{sphinxVerbatim}

\end{sphinxuseclass}\end{sphinxVerbatimInput}
\begin{sphinxVerbatimOutput}

\begin{sphinxuseclass}{cell_output}
\begin{sphinxVerbatim}[commandchars=\\\{\}]
\PYGZlt{}IPython.core.display.HTML object\PYGZgt{}
\end{sphinxVerbatim}

\end{sphinxuseclass}\end{sphinxVerbatimOutput}

\end{sphinxuseclass}

\section{Load a pre\sphinxhyphen{}existing model, data and descriptions}
\label{\detokenize{content/05_WBModels/MoreComplexScenarios:load-a-pre-existing-model-data-and-descriptions}}
\sphinxAtStartPar
Load the Pakistan model, which is comprised of the model object, its estimated equations and the data.  The \sphinxcode{\sphinxupquote{pcim}} file was created by the World Bank from the original EViews model used in the paper \hyperlink{cite.content/99_BackMatter/References:id14}{Burns \sphinxstyleemphasis{et al.}}.

\begin{sphinxuseclass}{cell}\begin{sphinxVerbatimInput}

\begin{sphinxuseclass}{cell_input}
\begin{sphinxVerbatim}[commandchars=\\\{\}]
\PYG{n}{mpak}\PYG{p}{,}\PYG{n}{baseline} \PYG{o}{=} \PYG{n}{model}\PYG{o}{.}\PYG{n}{modelload}\PYG{p}{(}\PYG{l+s+s1}{\PYGZsq{}}\PYG{l+s+s1}{../models/pak.pcim}\PYG{l+s+s1}{\PYGZsq{}}\PYG{p}{,}\PYG{n}{alfa}\PYG{o}{=}\PYG{l+m+mf}{0.7}\PYG{p}{,}\PYG{n}{run}\PYG{o}{=}\PYG{l+m+mi}{1}\PYG{p}{,}\PYG{n}{keep}\PYG{o}{=}\PYG{l+s+s2}{\PYGZdq{}}\PYG{l+s+s2}{Baseline}\PYG{l+s+s2}{\PYGZdq{}}\PYG{p}{)}
\end{sphinxVerbatim}

\end{sphinxuseclass}\end{sphinxVerbatimInput}
\begin{sphinxVerbatimOutput}

\begin{sphinxuseclass}{cell_output}
\begin{sphinxVerbatim}[commandchars=\\\{\}]
file read:  C:\PYGZbs{}modelflow manual\PYGZbs{}papers\PYGZbs{}mfbook\PYGZbs{}content\PYGZbs{}models\PYGZbs{}pak.pcim
\end{sphinxVerbatim}

\end{sphinxuseclass}\end{sphinxVerbatimOutput}

\end{sphinxuseclass}

\section{The policy problem}
\label{\detokenize{content/05_WBModels/MoreComplexScenarios:the-policy-problem}}
\sphinxAtStartPar
The objective of this chapter is to produce a simulation of the economic and climate effects of the introduction of a carbon tax in Pakistan.

\sphinxAtStartPar
The variable \sphinxcode{\sphinxupquote{mpak}} loaded above contains the model instance, the variables, equations and the data for the model.  On load the model was solved, and the results of that initiaql solve was assigned to the \sphinxcode{\sphinxupquote{DataFrame}} \sphinxcode{\sphinxupquote{baseline}}.

\sphinxAtStartPar
The Pakistan model contains three carbon tax variables:


\begin{savenotes}\sphinxattablestart
\centering
\begin{tabulary}{\linewidth}[t]{|T|T|}
\hline
\sphinxstyletheadfamily 
\sphinxAtStartPar
Mnemonic
&\sphinxstyletheadfamily 
\sphinxAtStartPar
Meaning
\\
\hline
\sphinxAtStartPar
PAKGGREVCO2CER
&
\sphinxAtStartPar
The effective carbon tax rate on Coal
\\
\hline
\sphinxAtStartPar
PAKGGREVCO2GER
&
\sphinxAtStartPar
The effective carbon tax rate on Gas
\\
\hline
\sphinxAtStartPar
PAKGGREVCO2OER
&
\sphinxAtStartPar
The effective carbon tax rate on Crude Oil
\\
\hline
\end{tabulary}
\par
\sphinxattableend\end{savenotes}

\sphinxAtStartPar
As discussed in earlier chapters the meaning of the mnemonics can be retrieved from the model:

\begin{sphinxuseclass}{cell}\begin{sphinxVerbatimInput}

\begin{sphinxuseclass}{cell_input}
\begin{sphinxVerbatim}[commandchars=\\\{\}]
\PYG{n}{mpak}\PYG{p}{[}\PYG{l+s+s1}{\PYGZsq{}}\PYG{l+s+s1}{PAKGGREVCO2*ER}\PYG{l+s+s1}{\PYGZsq{}}\PYG{p}{]}\PYG{o}{.}\PYG{n}{des}
    
\end{sphinxVerbatim}

\end{sphinxuseclass}\end{sphinxVerbatimInput}
\begin{sphinxVerbatimOutput}

\begin{sphinxuseclass}{cell_output}
\begin{sphinxVerbatim}[commandchars=\\\{\}]
PAKGGREVCO2CER : Carbon tax on coal (USD/t)
PAKGGREVCO2GER : Carbon tax on gas (USD/t)
PAKGGREVCO2OER : Carbon tax on oil (USD/t)
\end{sphinxVerbatim}

\end{sphinxuseclass}\end{sphinxVerbatimOutput}

\end{sphinxuseclass}
\sphinxAtStartPar
Alternatively one can search on the variable descriptions to retrieve the mnemonics of variables. Below the exclamation sign at the beginning of the string notifies the matching algorithm to search the variables descriptions (not he mnemonics) and return all variables that match.

\begin{sphinxuseclass}{cell}\begin{sphinxVerbatimInput}

\begin{sphinxuseclass}{cell_input}
\begin{sphinxVerbatim}[commandchars=\\\{\}]
\PYG{n}{mpak}\PYG{p}{[}\PYG{l+s+s1}{\PYGZsq{}}\PYG{l+s+s1}{!*Carbon*}\PYG{l+s+s1}{\PYGZsq{}}\PYG{p}{]}\PYG{o}{.}\PYG{n}{des}
\end{sphinxVerbatim}

\end{sphinxuseclass}\end{sphinxVerbatimInput}
\begin{sphinxVerbatimOutput}

\begin{sphinxuseclass}{cell_output}
\begin{sphinxVerbatim}[commandchars=\\\{\}]
PAKGGREVCO2CER : Carbon tax on coal (USD/t)
PAKGGREVCO2GER : Carbon tax on gas (USD/t)
PAKGGREVCO2OER : Carbon tax on oil (USD/t)
\end{sphinxVerbatim}

\end{sphinxuseclass}\end{sphinxVerbatimOutput}

\end{sphinxuseclass}

\section{Add variable descriptions}
\label{\detokenize{content/05_WBModels/MoreComplexScenarios:add-variable-descriptions}}
\sphinxAtStartPar
A modelflow model imported from EViews will inherit the variable descriptors coming from Eviews.  Not all EViews variables will necessarily have a description so such descriptions can be added to the existing using the \sphinxcode{\sphinxupquote{.set\_var\_descri{[}tion()}} method as below.

\sphinxAtStartPar
As coded, the call

\sphinxAtStartPar
\sphinxcode{\sphinxupquote{mpak.set\_var\_description(\{**mpak.var\_description,**extra\_description\})}}

\sphinxAtStartPar
adds the extra\_description dictionary to the pre\sphinxhyphen{}existing mpak,var\_description dictionary.

\sphinxAtStartPar
Several \sphinxcode{\sphinxupquote{modelflow}} methods include a \sphinxcode{\sphinxupquote{rename}} option, which if set to True will substitute the description for the variable name in any outputs.  Variables can also be selected for by using the \sphinxcode{\sphinxupquote{mpak{[}'!*subtext*'{]}}} syntax, where subtext is some text that appears in the variable descriptor.

\begin{sphinxuseclass}{cell}\begin{sphinxVerbatimInput}

\begin{sphinxuseclass}{cell_input}
\begin{sphinxVerbatim}[commandchars=\\\{\}]
\PYG{n}{extra\PYGZus{}description} \PYG{o}{=} \PYG{p}{\PYGZob{}}\PYG{l+s+s1}{\PYGZsq{}}\PYG{l+s+s1}{PAKNYGDPMKTPKN}\PYG{l+s+s1}{\PYGZsq{}}\PYG{p}{:} \PYG{l+s+s1}{\PYGZsq{}}\PYG{l+s+s1}{GDP}\PYG{l+s+s1}{\PYGZsq{}}\PYG{p}{,}
\PYG{l+s+s1}{\PYGZsq{}}\PYG{l+s+s1}{EMISCOAL}\PYG{l+s+s1}{\PYGZsq{}}        \PYG{p}{:} \PYG{l+s+s1}{\PYGZsq{}}\PYG{l+s+s1}{Coal emissions}\PYG{l+s+s1}{\PYGZsq{}}\PYG{p}{,}
\PYG{l+s+s1}{\PYGZsq{}}\PYG{l+s+s1}{EMISGAS}\PYG{l+s+s1}{\PYGZsq{}}         \PYG{p}{:} \PYG{l+s+s1}{\PYGZsq{}}\PYG{l+s+s1}{Gas Emissions}\PYG{l+s+s1}{\PYGZsq{}}\PYG{p}{,}
\PYG{l+s+s1}{\PYGZsq{}}\PYG{l+s+s1}{EMISOIL}\PYG{l+s+s1}{\PYGZsq{}}         \PYG{p}{:} \PYG{l+s+s1}{\PYGZsq{}}\PYG{l+s+s1}{Gas Emissions}\PYG{l+s+s1}{\PYGZsq{}}\PYG{p}{,}
\PYG{l+s+s1}{\PYGZsq{}}\PYG{l+s+s1}{PAKCCEMISCO2CKN}\PYG{l+s+s1}{\PYGZsq{}} \PYG{p}{:} \PYG{l+s+s1}{\PYGZsq{}}\PYG{l+s+s1}{Coal emissions, tCO2e}\PYG{l+s+s1}{\PYGZsq{}}\PYG{p}{,}
\PYG{l+s+s1}{\PYGZsq{}}\PYG{l+s+s1}{PAKCCEMISCO2GKN}\PYG{l+s+s1}{\PYGZsq{}} \PYG{p}{:} \PYG{l+s+s1}{\PYGZsq{}}\PYG{l+s+s1}{Natural Gas emissions, tCO2e}\PYG{l+s+s1}{\PYGZsq{}}\PYG{p}{,}
\PYG{l+s+s1}{\PYGZsq{}}\PYG{l+s+s1}{PAKCCEMISCO2OKN}\PYG{l+s+s1}{\PYGZsq{}} \PYG{p}{:} \PYG{l+s+s1}{\PYGZsq{}}\PYG{l+s+s1}{Crude Oil emissions, tCO2e}\PYG{l+s+s1}{\PYGZsq{}}\PYG{p}{,}
\PYG{l+s+s1}{\PYGZsq{}}\PYG{l+s+s1}{PAKCCEMISCO2TKN}\PYG{l+s+s1}{\PYGZsq{}} \PYG{p}{:} \PYG{l+s+s1}{\PYGZsq{}}\PYG{l+s+s1}{Total emissions, tCO2e}\PYG{l+s+s1}{\PYGZsq{}}\PYG{p}{,}
\PYG{l+s+s1}{\PYGZsq{}}\PYG{l+s+s1}{PAKGGREVEMISCN}\PYG{l+s+s1}{\PYGZsq{}}  \PYG{p}{:} \PYG{l+s+s1}{\PYGZsq{}}\PYG{l+s+s1}{Revenue from emissions taxes}\PYG{l+s+s1}{\PYGZsq{}}\PYG{p}{,}
 \PYG{l+s+s1}{\PYGZsq{}}\PYG{l+s+s1}{PAKLMUNRTOTLCN}\PYG{l+s+s1}{\PYGZsq{}}\PYG{p}{:} \PYG{l+s+s1}{\PYGZsq{}}\PYG{l+s+s1}{Unemployment rate}\PYG{l+s+s1}{\PYGZsq{}}\PYG{p}{,}
 \PYG{l+s+s1}{\PYGZsq{}}\PYG{l+s+s1}{PAKGGDBTTOTLCN\PYGZus{}}\PYG{l+s+s1}{\PYGZsq{}}\PYG{p}{:} \PYG{l+s+s1}{\PYGZsq{}}\PYG{l+s+s1}{Debt (}\PYG{l+s+si}{\PYGZpc{}G}\PYG{l+s+s1}{DP)}\PYG{l+s+s1}{\PYGZsq{}}\PYG{p}{,}
 \PYG{l+s+s1}{\PYGZsq{}}\PYG{l+s+s1}{PAKGGREVTOTLCN}\PYG{l+s+s1}{\PYGZsq{}}\PYG{p}{:} \PYG{l+s+s1}{\PYGZsq{}}\PYG{l+s+s1}{Fiscal revenues}\PYG{l+s+s1}{\PYGZsq{}}\PYG{p}{,}
 \PYG{l+s+s1}{\PYGZsq{}}\PYG{l+s+s1}{PAKWDL}\PYG{l+s+s1}{\PYGZsq{}}\PYG{p}{:} \PYG{l+s+s1}{\PYGZsq{}}\PYG{l+s+s1}{Working days lost due to pollution}\PYG{l+s+s1}{\PYGZsq{}}\PYG{p}{\PYGZcb{}}
\PYG{n}{mpak}\PYG{o}{.}\PYG{n}{set\PYGZus{}var\PYGZus{}description}\PYG{p}{(}\PYG{p}{\PYGZob{}}\PYG{o}{*}\PYG{o}{*}\PYG{n}{mpak}\PYG{o}{.}\PYG{n}{var\PYGZus{}description}\PYG{p}{,}\PYG{o}{*}\PYG{o}{*}\PYG{n}{extra\PYGZus{}description}\PYG{p}{\PYGZcb{}}\PYG{p}{)}
\end{sphinxVerbatim}

\end{sphinxuseclass}\end{sphinxVerbatimInput}

\end{sphinxuseclass}

\section{Simulating the impact of a imposing a carbon price}
\label{\detokenize{content/05_WBModels/MoreComplexScenarios:simulating-the-impact-of-a-imposing-a-carbon-price}}
\sphinxAtStartPar
To run a simulation, the following steps must invariably be followed.
\begin{enumerate}
\sphinxsetlistlabels{\arabic}{enumi}{enumii}{}{.}%
\item {} 
\sphinxAtStartPar
Create a new DataFrame, typically a copy of an existing one.

\item {} 
\sphinxAtStartPar
Change the value  in the new df of the variable(s) to be shocked.

\item {} 
\sphinxAtStartPar
Solve the model using the newly altered df as the input df.

\end{enumerate}

\begin{sphinxuseclass}{cell}\begin{sphinxVerbatimInput}

\begin{sphinxuseclass}{cell_input}
\begin{sphinxVerbatim}[commandchars=\\\{\}]
\PYG{c+c1}{\PYGZsh{} Create copy of the baseline df}
\PYG{n}{alternative\PYGZus{}df} \PYG{o}{=} \PYG{n}{baseline}\PYG{o}{.}\PYG{n}{copy}\PYG{p}{(}\PYG{p}{)}
\PYG{c+c1}{\PYGZsh{}set the effective carbon tax of all three carbon tax variables equal to 30 USD}
\PYG{n}{alternative\PYGZus{}df}\PYG{o}{.}\PYG{n}{loc}\PYG{p}{[}\PYG{l+m+mi}{2025}\PYG{p}{:}\PYG{l+m+mi}{2100}\PYG{p}{,}\PYG{p}{[}\PYG{l+s+s1}{\PYGZsq{}}\PYG{l+s+s1}{PAKGGREVCO2CER}\PYG{l+s+s1}{\PYGZsq{}}\PYG{p}{,}\PYG{l+s+s1}{\PYGZsq{}}\PYG{l+s+s1}{PAKGGREVCO2GER}\PYG{l+s+s1}{\PYGZsq{}}\PYG{p}{,} \PYG{l+s+s1}{\PYGZsq{}}\PYG{l+s+s1}{PAKGGREVCO2OER}\PYG{l+s+s1}{\PYGZsq{}}\PYG{p}{]}\PYG{p}{]} \PYG{o}{=} \PYG{l+m+mi}{30} 
\end{sphinxVerbatim}

\end{sphinxuseclass}\end{sphinxVerbatimInput}

\end{sphinxuseclass}
\sphinxAtStartPar
The above used the \sphinxcode{\sphinxupquote{pandas}} function \sphinxcode{\sphinxupquote{.loc{[}{]}}} to change the Carbon Tax rate variables.

\sphinxAtStartPar
The \sphinxcode{\sphinxupquote{modelflow}} method \sphinxcode{\sphinxupquote{.upd()}} could be used to perform the same change.

\begin{sphinxuseclass}{cell}\begin{sphinxVerbatimInput}

\begin{sphinxuseclass}{cell_input}
\begin{sphinxVerbatim}[commandchars=\\\{\}]
\PYG{c+c1}{\PYGZsh{} This modelflow command is equivalent to the previous standard pandas command abive that used the .loc[] syntax}
\PYG{n}{CT30df}  \PYG{o}{=}  \PYG{n}{baseline}\PYG{o}{.}\PYG{n}{upd}\PYG{p}{(}\PYG{l+s+s2}{\PYGZdq{}}\PYG{l+s+s2}{\PYGZlt{}2025 2100\PYGZgt{} PAKGGREVCO2CER PAKGGREVCO2GER PAKGGREVCO2OER = 30}\PYG{l+s+s2}{\PYGZdq{}}\PYG{p}{)}
\end{sphinxVerbatim}

\end{sphinxuseclass}\end{sphinxVerbatimInput}

\end{sphinxuseclass}

\subsection{Solve the model}
\label{\detokenize{content/05_WBModels/MoreComplexScenarios:solve-the-model}}
\sphinxAtStartPar
Solving the model is as simple as calling the mpak function with the altered \sphinxcode{\sphinxupquote{DataFrame}} and assigning the results to dataframe (\sphinxcode{\sphinxupquote{resultsdf}} in this instance).  The \sphinxcode{\sphinxupquote{keep}} option causes a copy of the dataframe to be stores within the mpak model obkect.

\begin{sphinxuseclass}{cell}\begin{sphinxVerbatimInput}

\begin{sphinxuseclass}{cell_input}
\begin{sphinxVerbatim}[commandchars=\\\{\}]
\PYG{n}{resultsdf} \PYG{o}{=} \PYG{n}{mpak}\PYG{p}{(}\PYG{n}{CT30df}\PYG{p}{,}\PYG{l+m+mi}{2020}\PYG{p}{,}\PYG{l+m+mi}{2100}\PYG{p}{,}\PYG{n}{keep}\PYG{o}{=}\PYG{l+s+s2}{\PYGZdq{}}\PYG{l+s+s2}{Nominal \PYGZdl{}30USD Carbon tax}\PYG{l+s+s2}{\PYGZdq{}}\PYG{p}{)} \PYG{c+c1}{\PYGZsh{} simulates the model }
\end{sphinxVerbatim}

\end{sphinxuseclass}\end{sphinxVerbatimInput}

\end{sphinxuseclass}

\subsubsection{Examining the results}
\label{\detokenize{content/05_WBModels/MoreComplexScenarios:examining-the-results}}
\sphinxAtStartPar
Every time the model is solved the results of the simulation are assigned to a variable on the left hands side of the solve call (\sphinxcode{\sphinxupquote{resultdf}} in the example above.  The results of the most recent scenario are also always stored in the \sphinxcode{\sphinxupquote{.lastdf}} \sphinxcode{\sphinxupquote{DataFrame}} that is part of the properties of any \sphinxcode{\sphinxupquote{modelflow}} model. the \sphinxcode{\sphinxupquote{basedf}} property of \sphinxcode{\sphinxupquote{mpak}} (an instantiation of a modelflow model object) contains a copy of the initial DataFrame from which the model was built.

\sphinxAtStartPar
The \sphinxcode{\sphinxupquote{DataFrames}} \sphinxcode{\sphinxupquote{Baseline}} and \sphinxcode{\sphinxupquote{Resultsdf}} were created by us when we solved the model (initially on load) and now with the simulation.  Currently their contents are the same as, but separate from the contents of \sphinxcode{\sphinxupquote{basedf}} and \sphinxcode{\sphinxupquote{lastdf}}.

\begin{sphinxadmonition}{note}{Note:}
\sphinxAtStartPar
The standard dataframes are part of the \sphinxcode{\sphinxupquote{modelflow}} object and managed by it.
\begin{itemize}
\item {} 
\sphinxAtStartPar
\sphinxstylestrong{mpak.basedf}: Dataframe with the values for baseline

\item {} 
\sphinxAtStartPar
\sphinxstylestrong{mpak.lastdf}: Dataframe with the values from the most recent simulation

\end{itemize}
\end{sphinxadmonition}

\sphinxAtStartPar
The impact of the imposition of the carbon tax in the model is relatively quick, resulting in an overall decline in emissions of 21.8\% in the first year, with coal emissions recording the biggest hit at \sphinxhyphen{}40.5 percent.

\sphinxAtStartPar
Abstracting from the fact that the impact is occurring too quickly (it would take time for the substitution towards alternative sources of power to occur), the fact that impacts are fading with time suggests an error in the specification of the shock.  High domestic inflation means that the relative price change of a given Carbon price is declining over time.

\begin{sphinxuseclass}{cell}\begin{sphinxVerbatimInput}

\begin{sphinxuseclass}{cell_input}
\begin{sphinxVerbatim}[commandchars=\\\{\}]
\PYG{k}{with} \PYG{n}{mpak}\PYG{o}{.}\PYG{n}{set\PYGZus{}smpl}\PYG{p}{(}\PYG{l+m+mi}{2023}\PYG{p}{,}\PYG{l+m+mi}{2030}\PYG{p}{)}\PYG{p}{:}
    \PYG{n+nb}{print}\PYG{p}{(}\PYG{n+nb}{round}\PYG{p}{(}\PYG{n}{mpak}\PYG{p}{[}\PYG{l+s+s1}{\PYGZsq{}}\PYG{l+s+s1}{PAKCCEMISCO2*}\PYG{l+s+s1}{\PYGZsq{}}\PYG{p}{]}\PYG{o}{.}\PYG{n}{difpctlevel}\PYG{o}{.}\PYG{n}{mul100}\PYG{o}{.}\PYG{n}{df}\PYG{p}{,}\PYG{l+m+mi}{2}\PYG{p}{)}\PYG{p}{)}\PYG{p}{;}
\end{sphinxVerbatim}

\end{sphinxuseclass}\end{sphinxVerbatimInput}
\begin{sphinxVerbatimOutput}

\begin{sphinxuseclass}{cell_output}
\begin{sphinxVerbatim}[commandchars=\\\{\}]
      PAKCCEMISCO2CKN  PAKCCEMISCO2GKN  PAKCCEMISCO2OKN  PAKCCEMISCO2TKN
2023             0.00             0.00             0.00             0.00
2024             0.00             0.00             0.00             0.00
2025           \PYGZhy{}41.19           \PYGZhy{}26.99           \PYGZhy{}10.93           \PYGZhy{}22.17
2026           \PYGZhy{}40.06           \PYGZhy{}25.72           \PYGZhy{}10.98           \PYGZhy{}21.56
2027           \PYGZhy{}38.85           \PYGZhy{}24.48           \PYGZhy{}10.89           \PYGZhy{}20.89
2028           \PYGZhy{}37.59           \PYGZhy{}23.30           \PYGZhy{}10.68           \PYGZhy{}20.17
2029           \PYGZhy{}36.26           \PYGZhy{}22.14           \PYGZhy{}10.35           \PYGZhy{}19.38
2030           \PYGZhy{}34.89           \PYGZhy{}21.01            \PYGZhy{}9.95           \PYGZhy{}18.55
\end{sphinxVerbatim}

\end{sphinxuseclass}\end{sphinxVerbatimOutput}

\end{sphinxuseclass}
\begin{sphinxuseclass}{cell}\begin{sphinxVerbatimInput}

\begin{sphinxuseclass}{cell_input}
\begin{sphinxVerbatim}[commandchars=\\\{\}]
\PYG{n}{mpak}\PYG{p}{[}\PYG{l+s+s1}{\PYGZsq{}}\PYG{l+s+s1}{PAKCCEMISCO2?KN}\PYG{l+s+s1}{\PYGZsq{}}\PYG{p}{]}\PYG{o}{.}\PYG{n}{difpctlevel}\PYG{o}{.}\PYG{n}{mul100}\PYG{o}{.}\PYG{n}{plot}\PYG{p}{(}\PYG{n}{title}\PYG{o}{=}\PYG{l+s+s2}{\PYGZdq{}}\PYG{l+s+s2}{Emissions impact of a \PYGZdl{}30 USD Carbon tax}\PYG{l+s+s2}{\PYGZdq{}}\PYG{p}{,}\PYG{n}{showfig}\PYG{o}{=}\PYG{k+kc}{True}\PYG{p}{)}
\end{sphinxVerbatim}

\end{sphinxuseclass}\end{sphinxVerbatimInput}
\begin{sphinxVerbatimOutput}

\begin{sphinxuseclass}{cell_output}
\noindent\sphinxincludegraphics{{34e14977fb5369f8e881ba18f47e2b8853f07c037e8f0b61abb6e8ecf41ce3b6}.png}

\end{sphinxuseclass}\end{sphinxVerbatimOutput}

\end{sphinxuseclass}
\sphinxAtStartPar
Abstracting from the fact that the impact is occurring too quickly (it would take time for the substitution towards alternative sources of power to occur), the fact that impacts are fading with time suggests an error in the specification of the shock.  High domestic inflation means that the relative price change of a given Carbon price is declining over time.


\section{Re\sphinxhyphen{}thinking the shock as an ex\sphinxhyphen{}ante real shock}
\label{\detokenize{content/05_WBModels/MoreComplexScenarios:re-thinking-the-shock-as-an-ex-ante-real-shock}}
\sphinxAtStartPar
Inflation in Pakistan is relatively high so a \$30 shock quickly loses its relative price effect. Increasing the nominal value of the Carbon Tax by the amount of domestic inflation (converted into USD each year) would resolve the problem.

\sphinxAtStartPar
Below a new dataframe is created as acopy of the baseline and the three Carbon taxes are first set to \$30 in 2025 and then grown at the rate of domestic inflation to keep the relative price of the Carbon Tax constant.

\sphinxAtStartPar
Finally the model is re\sphinxhyphen{}solved.

\begin{sphinxuseclass}{cell}\begin{sphinxVerbatimInput}

\begin{sphinxuseclass}{cell_input}
\begin{sphinxVerbatim}[commandchars=\\\{\}]
\PYG{k+kn}{import} \PYG{n+nn}{modelmf}  \PYG{c+c1}{\PYGZsh{} import the mfcalc functionality and append it to standard pandas}
\PYG{n}{CT30realdf}  \PYG{o}{=}  \PYG{n}{baseline}\PYG{o}{.}\PYG{n}{copy}\PYG{p}{(}\PYG{p}{)}
\PYG{n}{CT30realdf}\PYG{o}{=}\PYG{n}{CT30realdf}\PYG{o}{.}\PYG{n}{upd}\PYG{p}{(}\PYG{l+s+s2}{\PYGZdq{}}\PYG{l+s+s2}{\PYGZlt{}2025 2025\PYGZgt{} PAKGGREVCO2CER PAKGGREVCO2OER PAKGGREVCO2GER = 30}\PYG{l+s+s2}{\PYGZdq{}}\PYG{p}{)}

\PYG{n}{CT30realdf}\PYG{o}{=}\PYG{n}{CT30realdf}\PYG{o}{.}\PYG{n}{mfcalc}\PYG{p}{(}\PYG{l+s+s1}{\PYGZsq{}\PYGZsq{}\PYGZsq{}}
\PYG{l+s+s1}{                              \PYGZlt{}2026 2100\PYGZgt{} PAKGGREVCO2CER = PAKGGREVCO2CER(\PYGZhy{}1)*(PAKNECONPRVTXN*PAKPANUSATLS)/(PAKNECONPRVTXN(\PYGZhy{}1)*PAKPANUSATLS(\PYGZhy{}1))}
\PYG{l+s+s1}{                                      PAKGGREVCO2OER = PAKGGREVCO2OER(\PYGZhy{}1)*(PAKNECONPRVTXN*PAKPANUSATLS)/(PAKNECONPRVTXN(\PYGZhy{}1)*PAKPANUSATLS(\PYGZhy{}1))}
\PYG{l+s+s1}{                                      PAKGGREVCO2GER = PAKGGREVCO2CER(\PYGZhy{}1)*(PAKNECONPRVTXN*PAKPANUSATLS)/(PAKNECONPRVTXN(\PYGZhy{}1)*PAKPANUSATLS(\PYGZhy{}1))}
\PYG{l+s+s1}{                          }\PYG{l+s+s1}{\PYGZsq{}\PYGZsq{}\PYGZsq{}}\PYG{p}{)}                         

\PYG{n}{CT30realdf}\PYG{o}{.}\PYG{n}{loc}\PYG{p}{[}\PYG{l+m+mi}{2023}\PYG{p}{:}\PYG{l+m+mi}{2030}\PYG{p}{,}\PYG{l+s+s1}{\PYGZsq{}}\PYG{l+s+s1}{PAKGGREVCO2CER}\PYG{l+s+s1}{\PYGZsq{}}\PYG{p}{]}


\PYG{n}{resultsdf} \PYG{o}{=} \PYG{n}{mpak}\PYG{p}{(}\PYG{n}{CT30realdf}\PYG{p}{,}\PYG{l+m+mi}{2020}\PYG{p}{,}\PYG{l+m+mi}{2100}\PYG{p}{,}\PYG{n}{keep}\PYG{o}{=}\PYG{l+s+s2}{\PYGZdq{}}\PYG{l+s+s2}{Real \PYGZdl{}30USD Carbon tax}\PYG{l+s+s2}{\PYGZdq{}}\PYG{p}{)} \PYG{c+c1}{\PYGZsh{} simulates the model }
\end{sphinxVerbatim}

\end{sphinxuseclass}\end{sphinxVerbatimInput}

\end{sphinxuseclass}
\begin{sphinxuseclass}{cell}\begin{sphinxVerbatimInput}

\begin{sphinxuseclass}{cell_input}
\begin{sphinxVerbatim}[commandchars=\\\{\}]
\PYG{k}{with} \PYG{n}{mpak}\PYG{o}{.}\PYG{n}{set\PYGZus{}smpl}\PYG{p}{(}\PYG{l+m+mi}{2023}\PYG{p}{,}\PYG{l+m+mi}{2030}\PYG{p}{)}\PYG{p}{:}
    \PYG{n+nb}{print}\PYG{p}{(}\PYG{n}{mpak}\PYG{p}{[}\PYG{l+s+s1}{\PYGZsq{}}\PYG{l+s+s1}{PAKGG*ER}\PYG{l+s+s1}{\PYGZsq{}}\PYG{p}{]}\PYG{o}{.}\PYG{n}{df}\PYG{p}{)}
\end{sphinxVerbatim}

\end{sphinxuseclass}\end{sphinxVerbatimInput}
\begin{sphinxVerbatimOutput}

\begin{sphinxuseclass}{cell_output}
\begin{sphinxVerbatim}[commandchars=\\\{\}]
      PAKGGREVCO2CER  PAKGGREVCO2GER  PAKGGREVCO2OER
2023       \PYGZhy{}5.549839      \PYGZhy{}41.000884       \PYGZhy{}8.710650
2024       \PYGZhy{}5.549839      \PYGZhy{}41.000884       \PYGZhy{}8.710650
2025       30.000000       30.000000       30.000000
2026       31.827691       31.827691       31.827691
2027       33.642459       33.642459       33.642459
2028       35.451255       35.451255       35.451255
2029       37.263353       37.263353       37.263353
2030       39.091481       39.091481       39.091481
\end{sphinxVerbatim}

\end{sphinxuseclass}\end{sphinxVerbatimOutput}

\end{sphinxuseclass}
\begin{sphinxuseclass}{cell}\begin{sphinxVerbatimInput}

\begin{sphinxuseclass}{cell_input}
\begin{sphinxVerbatim}[commandchars=\\\{\}]
\PYG{n}{mpak}\PYG{p}{[}\PYG{l+s+s1}{\PYGZsq{}}\PYG{l+s+s1}{PAKCCEMISCO2?KN}\PYG{l+s+s1}{\PYGZsq{}}\PYG{p}{]}\PYG{o}{.}\PYG{n}{difpctlevel}\PYG{o}{.}\PYG{n}{mul100}\PYG{o}{.}\PYG{n}{plot}\PYG{p}{(}\PYG{n}{title}\PYG{o}{=}\PYG{l+s+s2}{\PYGZdq{}}\PYG{l+s+s2}{Emissions impact of a \PYGZdl{}30 USD Carbon tax}\PYG{l+s+s2}{\PYGZdq{}}\PYG{p}{,}\PYG{n}{showfig}\PYG{o}{=}\PYG{k+kc}{True}\PYG{p}{)}
\end{sphinxVerbatim}

\end{sphinxuseclass}\end{sphinxVerbatimInput}
\begin{sphinxVerbatimOutput}

\begin{sphinxuseclass}{cell_output}
\noindent\sphinxincludegraphics{{c206757ff83f15977d28f5270afb33eae08b2d6c477cee847f9368913bd0546e}.png}

\end{sphinxuseclass}\end{sphinxVerbatimOutput}

\end{sphinxuseclass}
\sphinxAtStartPar
These results are better, but still there is an erosion of the effect of the tax.

\sphinxAtStartPar
On introspection, this is likely due to the fact that the carbon tax itself is inflationary.  As a result, prices probably rose to a higher level than supposed by the ex ante calculation.

\sphinxAtStartPar
To deal with this, a different approach is needed.  Rather than maintaining the carbon price as an exogenous variable, instead it should be made an endogenous variable by changing the model and adding equations for all of the carbon tax variables.

\sphinxAtStartPar
Before doing so lets save the current version of the model for further work later.

\begin{sphinxuseclass}{cell}\begin{sphinxVerbatimInput}

\begin{sphinxuseclass}{cell_input}
\begin{sphinxVerbatim}[commandchars=\\\{\}]
\PYG{n}{mpak}\PYG{o}{.}\PYG{n}{modeldump}\PYG{p}{(}\PYG{l+s+s1}{\PYGZsq{}}\PYG{l+s+s1}{..}\PYG{l+s+s1}{\PYGZbs{}}\PYG{l+s+s1}{models}\PYG{l+s+s1}{\PYGZbs{}}\PYG{l+s+s1}{pakCarbonTaxScenarios.pcim}\PYG{l+s+s1}{\PYGZsq{}}\PYG{p}{)}
\end{sphinxVerbatim}

\end{sphinxuseclass}\end{sphinxVerbatimInput}

\end{sphinxuseclass}

\section{Changing the model – modifying and or adding equations}
\label{\detokenize{content/05_WBModels/MoreComplexScenarios:changing-the-model-modifying-and-or-adding-equations}}
\sphinxAtStartPar
To endogenize the carbon price, an equation for each carbon price has to be added to the model.   This cane be done with the \sphinxcode{\sphinxupquote{.equpdate()}} method.

\begin{sphinxuseclass}{cell}\begin{sphinxVerbatimInput}

\begin{sphinxuseclass}{cell_input}
\begin{sphinxVerbatim}[commandchars=\\\{\}]
\PYG{n}{mpak1}\PYG{p}{,}\PYG{n}{baseline} \PYG{o}{=} \PYG{n}{model}\PYG{o}{.}\PYG{n}{modelload}\PYG{p}{(}\PYG{l+s+s1}{\PYGZsq{}}\PYG{l+s+s1}{../models/pak.pcim}\PYG{l+s+s1}{\PYGZsq{}}\PYG{p}{,}\PYG{n}{alfa}\PYG{o}{=}\PYG{l+m+mf}{0.7}\PYG{p}{,}\PYG{n}{run}\PYG{o}{=}\PYG{l+m+mi}{1}\PYG{p}{,}\PYG{n}{keep}\PYG{o}{=}\PYG{l+s+s2}{\PYGZdq{}}\PYG{l+s+s2}{Baseline}\PYG{l+s+s2}{\PYGZdq{}}\PYG{p}{)}
\PYG{n}{mpakreal}\PYG{p}{,}\PYG{n}{baselinereal} \PYG{o}{=} \PYG{n}{mpak1}\PYG{o}{.}\PYG{n}{equpdate}\PYG{p}{(}\PYG{l+s+s1}{\PYGZsq{}\PYGZsq{}\PYGZsq{}}
\PYG{l+s+s1}{\PYGZlt{}fixable\PYGZgt{} PAKGGREVCO2CER = PAKGGREVCO2CER(\PYGZhy{}1) * (PAKNYGDPMKTPXN*PAKPANUSATLS) / (PAKNYGDPMKTPXN(\PYGZhy{}1)*PAKPANUSATLS(\PYGZhy{}1))}
\PYG{l+s+s1}{\PYGZlt{}fixable\PYGZgt{} PAKGGREVCO2OER = PAKGGREVCO2OER(\PYGZhy{}1) * (PAKNYGDPMKTPXN*PAKPANUSATLS) / (PAKNYGDPMKTPXN(\PYGZhy{}1)*PAKPANUSATLS(\PYGZhy{}1))}
\PYG{l+s+s1}{\PYGZlt{}fixable\PYGZgt{} PAKGGREVCO2GER = PAKGGREVCO2GER(\PYGZhy{}1) * (PAKNYGDPMKTPXN*PAKPANUSATLS) / (PAKNYGDPMKTPXN(\PYGZhy{}1)*PAKPANUSATLS(\PYGZhy{}1))}
\PYG{l+s+s1}{\PYGZsq{}\PYGZsq{}\PYGZsq{}}\PYG{p}{,}\PYG{n}{add\PYGZus{}add\PYGZus{}factor}\PYG{o}{=}\PYG{k+kc}{False}\PYG{p}{,} \PYG{n}{calc\PYGZus{}add}\PYG{o}{=}\PYG{k+kc}{False}\PYG{p}{,}\PYG{n}{newname}\PYG{o}{=}\PYG{l+s+s1}{\PYGZsq{}}\PYG{l+s+s1}{Pak model, with real Carbon price equations}\PYG{l+s+s1}{\PYGZsq{}}\PYG{p}{)}
\end{sphinxVerbatim}

\end{sphinxuseclass}\end{sphinxVerbatimInput}
\begin{sphinxVerbatimOutput}

\begin{sphinxuseclass}{cell_output}
\begin{sphinxVerbatim}[commandchars=\\\{\}]
file read:  C:\PYGZbs{}modelflow manual\PYGZbs{}papers\PYGZbs{}mfbook\PYGZbs{}content\PYGZbs{}models\PYGZbs{}pak.pcim
\end{sphinxVerbatim}

\begin{sphinxVerbatim}[commandchars=\\\{\}]
The model:\PYGZdq{}PAK\PYGZdq{} got new equations, new model name is:\PYGZdq{}Pak model, with real Carbon price equations\PYGZdq{}
New equation for For PAKGGREVCO2CER
Old frml   :new endogeneous variable 
New frml   :FRML \PYGZlt{}fixable\PYGZgt{} PAKGGREVCO2CER = (PAKGGREVCO2CER(\PYGZhy{}1)*(PAKNYGDPMKTPXN*PAKPANUSATLS)/(PAKNYGDPMKTPXN(\PYGZhy{}1)*PAKPANUSATLS(\PYGZhy{}1)))* (1\PYGZhy{}PAKGGREVCO2CER\PYGZus{}D)+ PAKGGREVCO2CER\PYGZus{}X*PAKGGREVCO2CER\PYGZus{}D\PYGZdl{}
Adjust calc:No frml for adjustment calc  

New equation for For PAKGGREVCO2OER
Old frml   :new endogeneous variable 
New frml   :FRML \PYGZlt{}fixable\PYGZgt{} PAKGGREVCO2OER = (PAKGGREVCO2OER(\PYGZhy{}1)*(PAKNYGDPMKTPXN*PAKPANUSATLS)/(PAKNYGDPMKTPXN(\PYGZhy{}1)*PAKPANUSATLS(\PYGZhy{}1)))* (1\PYGZhy{}PAKGGREVCO2OER\PYGZus{}D)+ PAKGGREVCO2OER\PYGZus{}X*PAKGGREVCO2OER\PYGZus{}D\PYGZdl{}
Adjust calc:No frml for adjustment calc  

New equation for For PAKGGREVCO2GER
Old frml   :new endogeneous variable 
New frml   :FRML \PYGZlt{}fixable\PYGZgt{} PAKGGREVCO2GER = (PAKGGREVCO2GER(\PYGZhy{}1)*(PAKNYGDPMKTPXN*PAKPANUSATLS)/(PAKNYGDPMKTPXN(\PYGZhy{}1)*PAKPANUSATLS(\PYGZhy{}1)))* (1\PYGZhy{}PAKGGREVCO2GER\PYGZus{}D)+ PAKGGREVCO2GER\PYGZus{}X*PAKGGREVCO2GER\PYGZus{}D\PYGZdl{}
Adjust calc:No frml for adjustment calc  
\end{sphinxVerbatim}

\end{sphinxuseclass}\end{sphinxVerbatimOutput}

\end{sphinxuseclass}
\sphinxAtStartPar
As written, the \sphinxcode{\sphinxupquote{.equpdate()}} command creates a new model, which is a copy of the existing model with three new equations.

\sphinxAtStartPar
Each equation grows the nominal rate of the carbon tax at the same rate as inflation (\sphinxcode{\sphinxupquote{PAKNECONPRVTXN}}) converted into USD via the exchange rate \sphinxcode{\sphinxupquote{PAKPANUSATLS}}. The equations are introduced as exogenizable equations (as distinct from an identity which must always hold), by adding the <fixable> prefix to each equation. The equations are not estimated, so no add\sphinxhyphen{}factors are included in the equations.

\sphinxAtStartPar
The output for the \sphinxcode{\sphinxupquote{.equpdate()}} reports the actual formulae included in the model.

\begin{sphinxVerbatim}[commandchars=\\\{\}]
New equation for For PAKGGREVCO2CER
Old frml   :new endogeneous variable 
New frml   :FRML \PYGZlt{}fixable\PYGZgt{} PAKGGREVCO2CER = (PAKGGREVCO2CER(\PYGZhy{}1)*(PAKNECONPRVTXN*PAKPANUSATLS)/(PAKNECONPRVTXN(\PYGZhy{}1)*PAKPANUSATLS(\PYGZhy{}1)))* (1\PYGZhy{}PAKGGREVCO2CER\PYGZus{}D)+ PAKGGREVCO2CER\PYGZus{}X*PAKGGREVCO2CER\PYGZus{}D\PYGZdl{}
Adjust calc:No frml for adjustment calc 
\end{sphinxVerbatim}

\sphinxAtStartPar
Note that because the equations are to be fixable, an \_X and \_D variable are added to the specified equations. Combined they effectively split each equation into  two:
\begin{enumerate}
\sphinxsetlistlabels{\arabic}{enumi}{enumii}{}{.}%
\item {} 
\sphinxAtStartPar
the specified equations when \_D equals zero

\item {} 
\sphinxAtStartPar
equal to \_X when the \_D equals one.

\end{enumerate}

\sphinxAtStartPar
The newly created model is given the name mpakreal and is given a text description.

\sphinxAtStartPar
Following the addition of the equations, the new variables (\_D and \_X) must be initialized. The \_X variables are made equal to the current values of the various tax rates, while the \_D is set to 1 everywhere – effectively turning the equation off and re\sphinxhyphen{}creating the same situation as the initial model where the tax rates are fully exogenous.

\begin{sphinxuseclass}{cell}\begin{sphinxVerbatimInput}

\begin{sphinxuseclass}{cell_input}
\begin{sphinxVerbatim}[commandchars=\\\{\}]
\PYG{c+c1}{\PYGZsh{}Exogenizes the newly added equations and sets the dummy =1 amd the \PYGZus{}x to the current value of the dependent variable  }
\PYG{n}{baseline\PYGZus{}real}\PYG{o}{=}\PYG{n}{mpakreal}\PYG{o}{.}\PYG{n}{fix}\PYG{p}{(}\PYG{n}{baselinereal}\PYG{p}{,}\PYG{l+s+s1}{\PYGZsq{}}\PYG{l+s+s1}{PAKGGREVCO2CER PAKGGREVCO2GER PAKGGREVCO2OER}\PYG{l+s+s1}{\PYGZsq{}}\PYG{p}{)}
\end{sphinxVerbatim}

\end{sphinxuseclass}\end{sphinxVerbatimInput}
\begin{sphinxVerbatimOutput}

\begin{sphinxuseclass}{cell_output}
\begin{sphinxVerbatim}[commandchars=\\\{\}]
The folowing variables are fixed
PAKGGREVCO2CER
PAKGGREVCO2GER
PAKGGREVCO2OER
\end{sphinxVerbatim}

\end{sphinxuseclass}\end{sphinxVerbatimOutput}

\end{sphinxuseclass}
\sphinxAtStartPar
The fix command above effectively does in one line all of the following code.

\begin{sphinxVerbatim}[commandchars=\\\{\}]
\PYG{n}{baseline\PYGZus{}real}\PYG{o}{=}\PYG{n}{baselinereal}\PYG{o}{.}\PYG{n}{copy}\PYG{p}{(}\PYG{p}{)}


\PYG{c+c1}{\PYGZsh{}Create the \PYGZus{}X variables if we exogenize the equation}
\PYG{n}{baseline\PYGZus{}real} \PYG{o}{=} \PYG{n}{baseline\PYGZus{}real}\PYG{o}{.}\PYG{n}{mfcalc}\PYG{p}{(}\PYG{l+s+s1}{\PYGZsq{}\PYGZsq{}\PYGZsq{}}
\PYG{l+s+s1}{PAKGGREVCO2CER\PYGZus{}X = PAKGGREVCO2CER}
\PYG{l+s+s1}{PAKGGREVCO2GER\PYGZus{}X = PAKGGREVCO2GER}
\PYG{l+s+s1}{PAKGGREVCO2OER\PYGZus{}X = PAKGGREVCO2OER}
\PYG{l+s+s1}{\PYGZsq{}\PYGZsq{}\PYGZsq{}}\PYG{p}{)}

\PYG{c+c1}{\PYGZsh{}create the \PYGZus{}D varibale so we can exogenize the equations (set \PYGZus{}D=1)\PYGZhy{}\PYGZhy{} currently it is exogenized }
\PYG{n}{baseline\PYGZus{}real} \PYG{o}{=} \PYG{n}{baseline\PYGZus{}real}\PYG{o}{.}\PYG{n}{upd}\PYG{p}{(}\PYG{l+s+s1}{\PYGZsq{}\PYGZsq{}\PYGZsq{}}
\PYG{l+s+s1}{\PYGZlt{}\PYGZhy{}0 \PYGZhy{}1\PYGZgt{} }
\PYG{l+s+s1}{PAKGGREVCO2CER\PYGZus{}D PAKGGREVCO2GER\PYGZus{}D PAKGGREVCO2OER\PYGZus{}D = 1}
\PYG{l+s+s1}{\PYGZsq{}\PYGZsq{}\PYGZsq{}}\PYG{p}{)}
\end{sphinxVerbatim}

\sphinxAtStartPar
Finally the new model is solved, the result is kept in a new baseline and a quick check ensures that the model did indeed reproduce the data that it was originally fed, including the initial Carbon Tax levels.

\begin{sphinxuseclass}{cell}\begin{sphinxVerbatimInput}

\begin{sphinxuseclass}{cell_input}
\begin{sphinxVerbatim}[commandchars=\\\{\}]
\PYG{c+c1}{\PYGZsh{}Solve the model for the new baseline}
\PYG{n}{res} \PYG{o}{=} \PYG{n}{mpakreal}\PYG{p}{(}\PYG{n}{baseline\PYGZus{}real}\PYG{p}{,}\PYG{l+m+mi}{2021}\PYG{p}{,}\PYG{l+m+mi}{2100}\PYG{p}{,}\PYG{n}{alfa}\PYG{o}{=}\PYG{l+m+mf}{0.5}\PYG{p}{,}\PYG{n}{keep}\PYG{o}{=}\PYG{l+s+s1}{\PYGZsq{}}\PYG{l+s+s1}{Baseline \PYGZhy{} adjusted model}\PYG{l+s+s1}{\PYGZsq{}}\PYG{p}{)} 

\PYG{n}{mpakreal}\PYG{p}{[}\PYG{l+s+s1}{\PYGZsq{}}\PYG{l+s+s1}{PAKNYGDPMKTPKN PAKNECONPRVTXN PAKGGBALOVRL PAKGGREVCO2CER PAKCCEMISCO2TKN}\PYG{l+s+s1}{\PYGZsq{}}\PYG{p}{]}\PYG{o}{.}\PYG{n}{difpctlevel}\PYG{o}{.}\PYG{n}{mul100}\PYG{o}{.}\PYG{n}{df}
\end{sphinxVerbatim}

\end{sphinxuseclass}\end{sphinxVerbatimInput}
\begin{sphinxVerbatimOutput}

\begin{sphinxuseclass}{cell_output}
\begin{sphinxVerbatim}[commandchars=\\\{\}]
      PAKNYGDPMKTPKN  PAKNECONPRVTXN  PAKGGREVCO2CER  PAKCCEMISCO2TKN
2021             0.0             0.0            \PYGZhy{}0.0              0.0
2022             0.0             0.0            \PYGZhy{}0.0              0.0
2023             0.0             0.0            \PYGZhy{}0.0              0.0
2024             0.0             0.0            \PYGZhy{}0.0              0.0
2025             0.0             0.0            \PYGZhy{}0.0              0.0
...              ...             ...             ...              ...
2096             0.0             0.0            \PYGZhy{}0.0              0.0
2097             0.0             0.0            \PYGZhy{}0.0              0.0
2098             0.0             0.0            \PYGZhy{}0.0              0.0
2099             0.0             0.0            \PYGZhy{}0.0              0.0
2100             0.0             0.0            \PYGZhy{}0.0              0.0

[80 rows x 4 columns]
\end{sphinxVerbatim}

\end{sphinxuseclass}\end{sphinxVerbatimOutput}

\end{sphinxuseclass}

\subsection{Solving the revised model}
\label{\detokenize{content/05_WBModels/MoreComplexScenarios:solving-the-revised-model}}
\sphinxAtStartPar
With the new model generated, it can now be solves with the real tax rate endogenized in the forecast period.  This involves three steps.
\begin{enumerate}
\sphinxsetlistlabels{\arabic}{enumi}{enumii}{}{.}%
\item {} 
\sphinxAtStartPar
Set the nominal tax rate to 30 in 2024

\item {} 
\sphinxAtStartPar
Now Endogenize the equation for the rest of the period

\item {} 
\sphinxAtStartPar
Solve the model.

\end{enumerate}

\begin{sphinxuseclass}{cell}\begin{sphinxVerbatimInput}

\begin{sphinxuseclass}{cell_input}
\begin{sphinxVerbatim}[commandchars=\\\{\}]
\PYG{n}{scenario\PYGZus{}real\PYGZus{}CTax} \PYG{o}{=} \PYG{n}{baseline\PYGZus{}real}\PYG{o}{.}\PYG{n}{upd}\PYG{p}{(}\PYG{l+s+s1}{\PYGZsq{}\PYGZsq{}\PYGZsq{}}
\PYG{l+s+s1}{\PYGZlt{}2024 2024\PYGZgt{} }
\PYG{l+s+s1}{PAKGGREVCO2CER\PYGZus{}x PAKGGREVCO2GER\PYGZus{}x PAKGGREVCO2OER\PYGZus{}x = 30 \PYGZsh{} Sets the exogenous value to 29 in 2024}
\PYG{l+s+s1}{\PYGZlt{}2025 2100 \PYGZgt{} }
\PYG{l+s+s1}{PAKGGREVCO2CER\PYGZus{}D PAKGGREVCO2GER\PYGZus{}D PAKGGREVCO2OER\PYGZus{}d = 0   \PYGZsh{} Endogenizes the new equations for the rest of time so that the real\PYGZhy{}rate stays at 30USD}
\PYG{l+s+s1}{\PYGZsq{}\PYGZsq{}\PYGZsq{}}\PYG{p}{)}


\PYG{n}{\PYGZus{}} \PYG{o}{=} \PYG{n}{mpakreal}\PYG{p}{(}\PYG{n}{scenario\PYGZus{}real\PYGZus{}CTax}\PYG{p}{,}\PYG{l+m+mi}{2021}\PYG{p}{,}\PYG{l+m+mi}{2100}\PYG{p}{,}\PYG{n}{alfa}\PYG{o}{=}\PYG{l+m+mf}{0.5}\PYG{p}{,}\PYG{n}{keep}\PYG{o}{=}\PYG{l+s+s1}{\PYGZsq{}}\PYG{l+s+s1}{Real model real tax = 30 in 2022 currency units}\PYG{l+s+s1}{\PYGZsq{}}\PYG{p}{)}
\end{sphinxVerbatim}

\end{sphinxuseclass}\end{sphinxVerbatimInput}

\end{sphinxuseclass}
\sphinxAtStartPar
Initially the Carbon tax comes in at 30 but gradually its rate in USD rises in line with inflation such that it reaches \$1500 by 2100.

\begin{sphinxuseclass}{cell}\begin{sphinxVerbatimInput}

\begin{sphinxuseclass}{cell_input}
\begin{sphinxVerbatim}[commandchars=\\\{\}]
\PYG{n+nb}{round}\PYG{p}{(}\PYG{n}{mpakreal}\PYG{p}{[}\PYG{l+s+s1}{\PYGZsq{}}\PYG{l+s+s1}{PAKGGREVCO2?ER PAKNECONPRVTXN PAKPANUSATLS}\PYG{l+s+s1}{\PYGZsq{}}\PYG{p}{]}\PYG{o}{.}\PYG{n}{df}\PYG{p}{,}\PYG{l+m+mi}{1}\PYG{p}{)}
\end{sphinxVerbatim}

\end{sphinxuseclass}\end{sphinxVerbatimInput}
\begin{sphinxVerbatimOutput}

\begin{sphinxuseclass}{cell_output}
\begin{sphinxVerbatim}[commandchars=\\\{\}]
      PAKGGREVCO2CER  PAKGGREVCO2GER  PAKGGREVCO2OER  PAKNECONPRVTXN   
2021            \PYGZhy{}5.5           \PYGZhy{}41.0            \PYGZhy{}8.7             1.8  \PYGZbs{}
2022            \PYGZhy{}5.5           \PYGZhy{}41.0            \PYGZhy{}8.7             2.0   
2023            \PYGZhy{}5.5           \PYGZhy{}41.0            \PYGZhy{}8.7             2.1   
2024            30.0            30.0            30.0             2.4   
2025            32.2            32.2            32.2             2.5   
...              ...             ...             ...             ...   
2096          1174.0          1174.0          1174.0           106.7   
2097          1237.9          1237.9          1237.9           112.8   
2098          1305.4          1305.4          1305.4           119.2   
2099          1376.5          1376.5          1376.5           126.0   
2100          1451.4          1451.4          1451.4           133.2   

      PAKPANUSATLS  
2021         107.0  
2022         106.8  
2023         106.7  
2024         106.3  
2025         106.2  
...            ...  
2096          94.3  
2097          94.1  
2098          93.9  
2099          93.7  
2100          93.5  

[80 rows x 5 columns]
\end{sphinxVerbatim}

\end{sphinxuseclass}\end{sphinxVerbatimOutput}

\end{sphinxuseclass}
\sphinxAtStartPar
This seemingly very high level is just a reflection of the 75 years of inflation that compounded require a much higher nominal Carbon tax rate to have the same relative price effect. The cumulative effect of inflation in the range of 5.5 percent per annum causes the price level to increase 74 times (7400 percent increase  133/1.8 from fourth data column in the above table).

\sphinxAtStartPar
The table below shows the same data but in growth rate terms – indicating that the Carbon tax effective rate is gradually rising each year in line domestic inflation adjusted for the exchange rate.

\begin{sphinxuseclass}{cell}\begin{sphinxVerbatimInput}

\begin{sphinxuseclass}{cell_input}
\begin{sphinxVerbatim}[commandchars=\\\{\}]
\PYG{n}{mpakreal}\PYG{p}{[}\PYG{l+s+s1}{\PYGZsq{}}\PYG{l+s+s1}{PAKGGREVCO2?ER PAKNECONPRVTXN PAKPANUSATLS}\PYG{l+s+s1}{\PYGZsq{}}\PYG{p}{]}\PYG{o}{.}\PYG{n}{pct}\PYG{o}{.}\PYG{n}{mul100}\PYG{o}{.}\PYG{n}{df}
\end{sphinxVerbatim}

\end{sphinxuseclass}\end{sphinxVerbatimInput}
\begin{sphinxVerbatimOutput}

\begin{sphinxuseclass}{cell_output}
\begin{sphinxVerbatim}[commandchars=\\\{\}]
      PAKGGREVCO2CER  PAKGGREVCO2GER  PAKGGREVCO2OER  PAKNECONPRVTXN   
2021        0.000000        0.000000        0.000000        9.476828  \PYGZbs{}
2022        0.000000        0.000000        0.000000        8.776453   
2023        0.000000        0.000000        0.000000        7.978008   
2024     \PYGZhy{}640.556268     \PYGZhy{}173.169155     \PYGZhy{}444.405991       10.081669   
2025        7.388261        7.388261        7.388261        7.073469   
...              ...             ...             ...             ...   
2096        5.448370        5.448370        5.448370        5.695440   
2097        5.447880        5.447880        5.447880        5.695184   
2098        5.447324        5.447324        5.447324        5.694856   
2099        5.446709        5.446709        5.446709        5.694466   
2100        5.446042        5.446042        5.446042        5.694020   

      PAKPANUSATLS  
2021     \PYGZhy{}0.158806  
2022     \PYGZhy{}0.155332  
2023     \PYGZhy{}0.137261  
2024     \PYGZhy{}0.328810  
2025     \PYGZhy{}0.134969  
...            ...  
2096     \PYGZhy{}0.201701  
2097     \PYGZhy{}0.201682  
2098     \PYGZhy{}0.201658  
2099     \PYGZhy{}0.201629  
2100     \PYGZhy{}0.201596  

[80 rows x 5 columns]
\end{sphinxVerbatim}

\end{sphinxuseclass}\end{sphinxVerbatimOutput}

\end{sphinxuseclass}

\subsection{Results}
\label{\detokenize{content/05_WBModels/MoreComplexScenarios:results}}
\sphinxAtStartPar
The results from the simulation with the Carbon Tax rate endogenized so as to maintain its real value over time, are broadly consistent with the results from the ex ante real scenario performed above.

\begin{sphinxuseclass}{cell}\begin{sphinxVerbatimInput}

\begin{sphinxuseclass}{cell_input}
\begin{sphinxVerbatim}[commandchars=\\\{\}]
\PYG{n}{mpakreal}\PYG{p}{[}\PYG{l+s+s1}{\PYGZsq{}}\PYG{l+s+s1}{PAKCCEMISCO2?KN}\PYG{l+s+s1}{\PYGZsq{}}\PYG{p}{]}\PYG{o}{.}\PYG{n}{difpctlevel}\PYG{o}{.}\PYG{n}{mul100}\PYG{o}{.}\PYG{n}{df}
\end{sphinxVerbatim}

\end{sphinxuseclass}\end{sphinxVerbatimInput}
\begin{sphinxVerbatimOutput}

\begin{sphinxuseclass}{cell_output}
\begin{sphinxVerbatim}[commandchars=\\\{\}]
      PAKCCEMISCO2CKN  PAKCCEMISCO2GKN  PAKCCEMISCO2OKN  PAKCCEMISCO2TKN
2021         0.000000         0.000000         0.000000         0.000000
2022         0.000000         0.000000         0.000000         0.000000
2023         0.000000         0.000000         0.000000         0.000000
2024       \PYGZhy{}42.758320       \PYGZhy{}28.536509       \PYGZhy{}11.513706       \PYGZhy{}23.275423
2025       \PYGZhy{}43.091703       \PYGZhy{}27.682273       \PYGZhy{}12.205914       \PYGZhy{}23.439333
...               ...              ...              ...              ...
2096       \PYGZhy{}25.808144       \PYGZhy{}11.978411        \PYGZhy{}5.652806       \PYGZhy{}11.846356
2097       \PYGZhy{}25.660625       \PYGZhy{}11.894713        \PYGZhy{}5.601072       \PYGZhy{}11.764000
2098       \PYGZhy{}25.513633       \PYGZhy{}11.811483        \PYGZhy{}5.549747       \PYGZhy{}11.682105
2099       \PYGZhy{}25.367155       \PYGZhy{}11.728713        \PYGZhy{}5.498819       \PYGZhy{}11.600661
2100       \PYGZhy{}25.221178       \PYGZhy{}11.646393        \PYGZhy{}5.448280       \PYGZhy{}11.519657

[80 rows x 4 columns]
\end{sphinxVerbatim}

\end{sphinxuseclass}\end{sphinxVerbatimOutput}

\end{sphinxuseclass}
\begin{sphinxuseclass}{cell}\begin{sphinxVerbatimInput}

\begin{sphinxuseclass}{cell_input}
\begin{sphinxVerbatim}[commandchars=\\\{\}]
\PYG{n}{mpakreal}\PYG{p}{[}\PYG{l+s+s1}{\PYGZsq{}}\PYG{l+s+s1}{PAKCCEMISCO2?KN PAKNYGDPMKTPKN}\PYG{l+s+s1}{\PYGZsq{}}\PYG{p}{]}\PYG{o}{.}\PYG{n}{difpctlevel}\PYG{o}{.}\PYG{n}{mul100}\PYG{o}{.}\PYG{n}{plot}\PYG{p}{(}\PYG{n}{title}\PYG{o}{=}\PYG{l+s+s2}{\PYGZdq{}}\PYG{l+s+s2}{Emissions impact of a \PYGZdl{}30 USD Carbon tax}\PYG{l+s+s2}{\PYGZdq{}}\PYG{p}{,}\PYG{n}{showfig}\PYG{o}{=}\PYG{k+kc}{True}\PYG{p}{)}
\end{sphinxVerbatim}

\end{sphinxuseclass}\end{sphinxVerbatimInput}
\begin{sphinxVerbatimOutput}

\begin{sphinxuseclass}{cell_output}
\noindent\sphinxincludegraphics{{6ebfa5844b3cf308dfd9056e1a7c48e6635b095a03cf55ab4dc55f38a78222ee}.png}

\end{sphinxuseclass}\end{sphinxVerbatimOutput}

\end{sphinxuseclass}
\sphinxAtStartPar
The modified model, which preserves the real value of the carbon tax has a permanent and substantive negative effect on emissions. The impact is not at an unchanging level, reflecting in part adaptation within the economy. Of particular import is what is being done with the revenues from the Carbon tax, the structure of GDP (shifts to more carbon intensive activities) and in this particular scenario the fact that the level of activity is higher and therefore emission higher than they would have been had GDP remained unchanged.

\sphinxstepscope


\part{Model Analytics}

\sphinxstepscope


\chapter{Model analytics}
\label{\detokenize{content/06_ModelAnalytics/ModelStructure:model-analytics}}\label{\detokenize{content/06_ModelAnalytics/ModelStructure::doc}}
\sphinxAtStartPar
A model has a well defined logical and causal structure.\hyperlink{cite.content/99_BackMatter/References:id4}{Kogiku}provides an introduction to causal analysis of models, while \hyperlink{cite.content/99_BackMatter/References:id2}{Berndsen} provides a more elaborate discussion.

\sphinxAtStartPar
At the simplest level, the equations of a model can be organized into blocks.
\begin{itemize}
\item {} 
\sphinxAtStartPar
\sphinxstylestrong{Simultaneous block} include equations that have are co\sphinxhyphen{}determined simultaneously. They contain feedback loops that mean they may require several iterations before a solution that satisfies them all is found. A classic simultaneous block would include GDP and Consumption. Consumption depends on income. Income depends on GDP, but COnsumption determines GDP.

\item {} 
\sphinxAtStartPar
\sphinxstylestrong{Recursive blocks} include equations that are a simple function of other variables. For example, the current account balance is just the difference between Export Revenues and Import Revenues.  These can be solved with just one pass once the values of the simultaneous blocks have been resolved.

\end{itemize}

\sphinxAtStartPar
At the equation level, each endogenous variable is a function of one or more variables, but because these variables are also dependent on other variables in the model, those right hand side variables that are endogenous can have their equations substituted into the first level equation to get an extended set of dependencies and he endogenous right hand side variables of these second level variables can also have their right hand sides substituted into the equation etc.

\sphinxAtStartPar
\sphinxcode{\sphinxupquote{Modelflow}} uses the \sphinxhref{https://networkx.org/}{networkx} python package to analyze the interrelation ships within the model and between equations and includes a number of methods and properties to present these interrelationships both in tabular and graphical form {[}\textasciicircum{}Graphvix{]}, a subset of which is exposed in this chapter.

\sphinxAtStartPar
Setting up the python environment and loading a pre\sphinxhyphen{}exisitng model

\begin{sphinxuseclass}{cell}\begin{sphinxVerbatimInput}

\begin{sphinxuseclass}{cell_input}
\begin{sphinxVerbatim}[commandchars=\\\{\}]
\PYG{k+kn}{from} \PYG{n+nn}{modelclass} \PYG{k+kn}{import} \PYG{n}{model}
\PYG{o}{\PYGZpc{}}\PYG{k}{load\PYGZus{}ext} autoreload
\PYG{o}{\PYGZpc{}}\PYG{k}{autoreload} 2

\PYG{n}{mpak}\PYG{p}{,}\PYG{n}{baseline} \PYG{o}{=} \PYG{n}{model}\PYG{o}{.}\PYG{n}{modelload}\PYG{p}{(}\PYG{l+s+s1}{\PYGZsq{}}\PYG{l+s+s1}{../models/pak.pcim}\PYG{l+s+s1}{\PYGZsq{}}\PYG{p}{,}\PYG{n}{alfa}\PYG{o}{=}\PYG{l+m+mf}{0.7}\PYG{p}{,}\PYG{n}{run}\PYG{o}{=}\PYG{l+m+mi}{1}\PYG{p}{)}

\PYG{n}{mpak}\PYG{o}{.}\PYG{n}{model\PYGZus{}description}\PYG{o}{=}\PYG{l+s+s2}{\PYGZdq{}}\PYG{l+s+s2}{World Bank climate aware model of Pakistan as described in Burns et al. (2019)}\PYG{l+s+s2}{\PYGZdq{}}
\PYG{n}{mpak}\PYG{o}{.}\PYG{n}{model\PYGZus{}description}
\PYG{n}{mpak}\PYG{o}{.}\PYG{n}{periode}\PYG{o}{=}\PYG{l+m+mi}{2100}
\end{sphinxVerbatim}

\end{sphinxuseclass}\end{sphinxVerbatimInput}
\begin{sphinxVerbatimOutput}

\begin{sphinxuseclass}{cell_output}
\begin{sphinxVerbatim}[commandchars=\\\{\}]
file read:  C:\PYGZbs{}modelflow manual\PYGZbs{}papers\PYGZbs{}mfbook\PYGZbs{}content\PYGZbs{}models\PYGZbs{}pak.pcim
\end{sphinxVerbatim}

\end{sphinxuseclass}\end{sphinxVerbatimOutput}

\end{sphinxuseclass}

\section{Model information}
\label{\detokenize{content/06_ModelAnalytics/ModelStructure:model-information}}
\sphinxAtStartPar
The model object contains information about the model itself, its name, its structure (does it contain simultaneous equations or is it recursive), the number of variables it contains and the number that are exogenous and endogenous (have associated equations).

\begin{sphinxuseclass}{cell}\begin{sphinxVerbatimInput}

\begin{sphinxuseclass}{cell_input}
\begin{sphinxVerbatim}[commandchars=\\\{\}]
\PYG{n}{mpak}
\end{sphinxVerbatim}

\end{sphinxuseclass}\end{sphinxVerbatimInput}
\begin{sphinxVerbatimOutput}

\begin{sphinxuseclass}{cell_output}
\begin{sphinxVerbatim}[commandchars=\\\{\}]
\PYGZlt{}
Model name                              :                  PAK 
Model structure                         :         Simultaneous 
Number of variables                     :                  839 
Number of exogeneous  variables         :                  461 
Number of endogeneous variables         :                  378 
\PYGZgt{}
\end{sphinxVerbatim}

\end{sphinxuseclass}\end{sphinxVerbatimOutput}

\end{sphinxuseclass}
\sphinxAtStartPar
The model work space also has a time dimension, its sample period. This can be retrieved and changed.

\sphinxAtStartPar
`mpak.per\_current’

\begin{sphinxuseclass}{cell}\begin{sphinxVerbatimInput}

\begin{sphinxuseclass}{cell_input}
\begin{sphinxVerbatim}[commandchars=\\\{\}]
\PYG{n}{mpak}\PYG{o}{.}\PYG{n}{current\PYGZus{}per}
\end{sphinxVerbatim}

\end{sphinxuseclass}\end{sphinxVerbatimInput}
\begin{sphinxVerbatimOutput}

\begin{sphinxuseclass}{cell_output}
\begin{sphinxVerbatim}[commandchars=\\\{\}]
Index([2016, 2017, 2018, 2019, 2020, 2021, 2022, 2023, 2024, 2025, 2026, 2027,
       2028, 2029, 2030],
      dtype=\PYGZsq{}int64\PYGZsq{})
\end{sphinxVerbatim}

\end{sphinxuseclass}\end{sphinxVerbatimOutput}

\end{sphinxuseclass}

\section{Model structure}
\label{\detokenize{content/06_ModelAnalytics/ModelStructure:model-structure}}
\sphinxAtStartPar
A quick way to visualize the structure of a model is to plot its \sphinxhref{https://en.wikipedia.org/wiki/Adjacency\_matrix}{adjacency matrix}.

\sphinxAtStartPar
The adjacency matrix plots the relationships between endogenous variables in the model, dividing them into one or more simultaneous blocks and one or more recursive blocks.

\sphinxAtStartPar
Below is the adjacency matrix for the Pakistan model. Variables in the red square block depend on one or more variables that in turn depends upon them, requiring the mode to solve for their values simultaneously.  The variables in the green triangles do not enter directly or indirectly as an argument in the variables that determine them and therefore can be solved in one iteration once the values for the simultaneous variables are determined.

\begin{sphinxuseclass}{cell}\begin{sphinxVerbatimInput}

\begin{sphinxuseclass}{cell_input}
\begin{sphinxVerbatim}[commandchars=\\\{\}]
\PYG{n}{mpak}\PYG{o}{.}\PYG{n}{plotadjacency}\PYG{p}{(}\PYG{n}{size}\PYG{o}{=}\PYG{p}{(}\PYG{l+m+mi}{20}\PYG{p}{,}\PYG{l+m+mi}{20}\PYG{p}{)}\PYG{p}{,}\PYG{n}{nolag}\PYG{o}{=}\PYG{l+m+mi}{0}\PYG{p}{)}\PYG{p}{;}
\end{sphinxVerbatim}

\end{sphinxuseclass}\end{sphinxVerbatimInput}
\begin{sphinxVerbatimOutput}

\begin{sphinxuseclass}{cell_output}
\noindent\sphinxincludegraphics{{5646e014d8646c3c87ed72c8da4a2b5c56017e4c4d4f666277f9815ab1ff3d33}.png}

\end{sphinxuseclass}\end{sphinxVerbatimOutput}

\end{sphinxuseclass}

\chapter{The dependencies of individual endogenous variables (the \sphinxstyleliteralintitle{\sphinxupquote{.tracepre()}} method)}
\label{\detokenize{content/06_ModelAnalytics/ModelStructure:the-dependencies-of-individual-endogenous-variables-the-tracepre-method}}
\sphinxAtStartPar
As noted above, every endogenous variables is directly depenednt on the variables that occur onits right hand side, but us also indirectly dependent on the variables that determine its RHS variables and in turn those that determine the varibales to the right of them \sphinxstyleemphasis{ad infinitum}.

\sphinxAtStartPar
\sphinxcode{\sphinxupquote{Modelflow}} includes several methods and properties that allow these dependencies to be explored.

\sphinxAtStartPar
The \sphinxcode{\sphinxupquote{.frml}} property returns the normalized formula of an equation, from which the right hand variables for the equation can be discerned.

\begin{sphinxuseclass}{cell}\begin{sphinxVerbatimInput}

\begin{sphinxuseclass}{cell_input}
\begin{sphinxVerbatim}[commandchars=\\\{\}]
\PYG{n}{mpak}\PYG{o}{.}\PYG{n}{PAKNECONPRVTKN}\PYG{o}{.}\PYG{n}{frml}
\end{sphinxVerbatim}

\end{sphinxuseclass}\end{sphinxVerbatimInput}
\begin{sphinxVerbatimOutput}

\begin{sphinxuseclass}{cell_output}
\begin{sphinxVerbatim}[commandchars=\\\{\}]
Endogeneous: PAKNECONPRVTKN: HH. Cons Real
Formular: FRML \PYGZlt{}DAMP,STOC\PYGZgt{} PAKNECONPRVTKN = (PAKNECONPRVTKN(\PYGZhy{}1)*EXP(PAKNECONPRVTKN\PYGZus{}A+ (\PYGZhy{}0.2*(LOG(PAKNECONPRVTKN(\PYGZhy{}1))\PYGZhy{}LOG(1.21203101101442)\PYGZhy{}LOG((((PAKBXFSTREMTCD(\PYGZhy{}1)\PYGZhy{}PAKBMFSTREMTCD(\PYGZhy{}1))*PAKPANUSATLS(\PYGZhy{}1))+PAKGGEXPTRNSCN(\PYGZhy{}1)+PAKNYYWBTOTLCN(\PYGZhy{}1)*(1\PYGZhy{}PAKGGREVDRCTXN(\PYGZhy{}1)/100))/PAKNECONPRVTXN(\PYGZhy{}1)))+0.763938860758873*((LOG((((PAKBXFSTREMTCD\PYGZhy{}PAKBMFSTREMTCD)*PAKPANUSATLS)+PAKGGEXPTRNSCN+PAKNYYWBTOTLCN*(1\PYGZhy{}PAKGGREVDRCTXN/100))/PAKNECONPRVTXN))\PYGZhy{}(LOG((((PAKBXFSTREMTCD(\PYGZhy{}1)\PYGZhy{}PAKBMFSTREMTCD(\PYGZhy{}1))*PAKPANUSATLS(\PYGZhy{}1))+PAKGGEXPTRNSCN(\PYGZhy{}1)+PAKNYYWBTOTLCN(\PYGZhy{}1)*(1\PYGZhy{}PAKGGREVDRCTXN(\PYGZhy{}1)/100))/PAKNECONPRVTXN(\PYGZhy{}1))))\PYGZhy{}0.0634474791568939*DURING\PYGZus{}2009\PYGZhy{}0.3*(PAKFMLBLPOLYXN/100\PYGZhy{}((LOG(PAKNECONPRVTXN))\PYGZhy{}(LOG(PAKNECONPRVTXN(\PYGZhy{}1)))))) )) * (1\PYGZhy{}PAKNECONPRVTKN\PYGZus{}D)+ PAKNECONPRVTKN\PYGZus{}X*PAKNECONPRVTKN\PYGZus{}D  \PYGZdl{}

PAKNECONPRVTKN  : HH. Cons Real
DURING\PYGZus{}2009     : 
PAKBMFSTREMTCD  : Imp., Remittances (BOP), US\PYGZdl{} mn
PAKBXFSTREMTCD  : Exp., Remittances (BOP), US\PYGZdl{} mn
PAKFMLBLPOLYXN  : Key Policy Interest Rate
PAKGGEXPTRNSCN  : Current Transfers
PAKGGREVDRCTXN  : Direct Revenue Tax Rate
PAKNECONPRVTKN\PYGZus{}A: Add factor:HH. Cons Real
PAKNECONPRVTKN\PYGZus{}D: Fix dummy:HH. Cons Real
PAKNECONPRVTKN\PYGZus{}X: Fix value:HH. Cons Real
PAKNECONPRVTXN  : Implicit LCU defl., Pvt. Cons., 2000 = 1
PAKNYYWBTOTLCN  : Total Wage Bill
PAKPANUSATLS    : Exchange rate LCU / US\PYGZdl{} \PYGZhy{} Pakistan
\end{sphinxVerbatim}

\end{sphinxuseclass}\end{sphinxVerbatimOutput}

\end{sphinxuseclass}
\sphinxAtStartPar
The method \sphinxcode{\sphinxupquote{.tracepre()}} provides a graphical representation of this relationship, showing all the variables that directly determine an endogenous variable (in this example real GDP), distinguishing between RHS variables that are endogenous (in blue) and those that are exogenous (yellow).

\begin{sphinxuseclass}{cell}\begin{sphinxVerbatimInput}

\begin{sphinxuseclass}{cell_input}
\begin{sphinxVerbatim}[commandchars=\\\{\}]
\PYG{n}{latex}\PYG{o}{=}\PYG{l+m+mi}{1}
\PYG{n}{mpak}\PYG{o}{.}\PYG{n}{PAKNYGDPMKTPKN}\PYG{o}{.}\PYG{n}{tracepre}\PYG{p}{(}\PYG{n}{png}\PYG{o}{=}\PYG{n}{latex}\PYG{p}{)}
\end{sphinxVerbatim}

\end{sphinxuseclass}\end{sphinxVerbatimInput}
\begin{sphinxVerbatimOutput}

\begin{sphinxuseclass}{cell_output}
\noindent\sphinxincludegraphics{{aea07c2b96a64cd1dd5a77015ef977d3158efdea97e6cee10fec3d6e1d984bf0}.png}

\end{sphinxuseclass}\end{sphinxVerbatimOutput}

\end{sphinxuseclass}
\sphinxAtStartPar
If the model has been solved, \sphinxcode{\sphinxupquote{.pretrc()}} goes one step further and reveals the relative imortance of each variable in the change of the dependent variable.


\section{Shock the model}
\label{\detokenize{content/06_ModelAnalytics/ModelStructure:shock-the-model}}
\sphinxAtStartPar
Below a \$30 nominal Carbon tax is applied beginning in 2025.

\begin{sphinxuseclass}{cell}\begin{sphinxVerbatimInput}

\begin{sphinxuseclass}{cell_input}
\begin{sphinxVerbatim}[commandchars=\\\{\}]
\PYG{n}{alternative}  \PYG{o}{=}  \PYG{n}{baseline}\PYG{o}{.}\PYG{n}{upd}\PYG{p}{(}\PYG{l+s+s2}{\PYGZdq{}}\PYG{l+s+s2}{\PYGZlt{}2025 2100\PYGZgt{} PAKGGREVCO2CER PAKGGREVCO2GER PAKGGREVCO2OER = 29}\PYG{l+s+s2}{\PYGZdq{}}\PYG{p}{)}
\PYG{n}{result} \PYG{o}{=} \PYG{n}{mpak}\PYG{p}{(}\PYG{n}{alternative}\PYG{p}{,}\PYG{l+m+mi}{2020}\PYG{p}{,}\PYG{l+m+mi}{2100}\PYG{p}{)} \PYG{c+c1}{\PYGZsh{} simulates the model }
\end{sphinxVerbatim}

\end{sphinxuseclass}\end{sphinxVerbatimInput}

\end{sphinxuseclass}
\sphinxAtStartPar
As a result GDP, consumption investment and most all variables in the model change, as illustrated in the below graphs that show the percent deviation of the main components of GDP from their baseline values.

\begin{sphinxuseclass}{cell}\begin{sphinxVerbatimInput}

\begin{sphinxuseclass}{cell_input}
\begin{sphinxVerbatim}[commandchars=\\\{\}]
\PYG{n}{mpak}\PYG{p}{[}\PYG{l+s+s1}{\PYGZsq{}}\PYG{l+s+s1}{PAKNYGDPMKTPKN PAKNECONPRVTKN PAKNEGDIFTOTKN PAKNEEXPGNFSKN PAKNEIMPGNFSKN}\PYG{l+s+s1}{\PYGZsq{}}\PYG{p}{]}\PYG{o}{.}\PYG{n}{difpctlevel}\PYG{o}{.}\PYG{n}{plot}\PYG{p}{(}\PYG{p}{)}\PYG{p}{;}
\end{sphinxVerbatim}

\end{sphinxuseclass}\end{sphinxVerbatimInput}

\end{sphinxuseclass}

\section{Post shock \sphinxstyleliteralintitle{\sphinxupquote{.tracepre()}} indicates the relative importance of different variables in explaining the change in the dependent variable}
\label{\detokenize{content/06_ModelAnalytics/ModelStructure:post-shock-tracepre-indicates-the-relative-importance-of-different-variables-in-explaining-the-change-in-the-dependent-variable}}
\sphinxAtStartPar
Below the same command is executed, but because of the shock, the width of the lines indicating representing the causal links between variables is ticker the more important a given variable was in the previous simulation in explaining the change in the level of the dependent variable (GDP).

\begin{sphinxuseclass}{cell}\begin{sphinxVerbatimInput}

\begin{sphinxuseclass}{cell_input}
\begin{sphinxVerbatim}[commandchars=\\\{\}]
\PYG{n}{latex}\PYG{o}{=}\PYG{l+m+mi}{1}
\PYG{n}{mpak}\PYG{o}{.}\PYG{n}{PAKNYGDPMKTPKN}\PYG{o}{.}\PYG{n}{tracepre}\PYG{p}{(}\PYG{n}{png}\PYG{o}{=}\PYG{n}{latex}\PYG{p}{)}
\end{sphinxVerbatim}

\end{sphinxuseclass}\end{sphinxVerbatimInput}
\begin{sphinxVerbatimOutput}

\begin{sphinxuseclass}{cell_output}
\noindent\sphinxincludegraphics{{a1cdecfa944385b27516afe764eb3bcafe63b7d940930122f0d502aa33026f9f}.png}

\end{sphinxuseclass}\end{sphinxVerbatimOutput}

\end{sphinxuseclass}
\begin{sphinxadmonition}{note}{Note:}
\sphinxAtStartPar
\sphinxstylestrong{png=latex}

\sphinxAtStartPar
The default behavior when displaying graphs in a \sphinxstyleemphasis{jupyter notebook} is to produce images in .svg format.
These images scale well and the mouseover feature can be used. That is: On mouseover of a node, the variable and the equation are displayed.  On mouseover on an joining line, the extent to which the variable contributed to the change in the dependent variable is displayed.

\sphinxAtStartPar
Unfortunately this \sphinxstyleemphasis{jupyter book} (that is not a notebook) requires images be in he jpg or PNG format so this functionality has been disabled, by specifying that the png format be used instead of svg.

\sphinxAtStartPar
For other purposes, the variable latex could be set equal to False in which case the same code will generate SVGs.
\end{sphinxadmonition}


\section{The filter option, restricting the output of \sphinxstyleliteralintitle{\sphinxupquote{.tracepre()}}}
\label{\detokenize{content/06_ModelAnalytics/ModelStructure:the-filter-option-restricting-the-output-of-tracepre}}
\sphinxAtStartPar
Using the filter option, the output of \sphinxcode{\sphinxupquote{.tracepre()}} can be restricted to RHS variables that have had large impacts on the dependent variable.v

\begin{sphinxuseclass}{cell}\begin{sphinxVerbatimInput}

\begin{sphinxuseclass}{cell_input}
\begin{sphinxVerbatim}[commandchars=\\\{\}]
\PYG{n}{gg}\PYG{o}{=}\PYG{n}{mpak}\PYG{o}{.}\PYG{n}{PAKNYGDPMKTPKN}\PYG{o}{.}\PYG{n}{tracepre}\PYG{p}{(}\PYG{n+nb}{filter}\PYG{o}{=}\PYG{l+m+mi}{20}\PYG{p}{,}\PYG{n}{png}\PYG{o}{=}\PYG{n}{latex}\PYG{p}{)}
\PYG{n}{gg}
\end{sphinxVerbatim}

\end{sphinxuseclass}\end{sphinxVerbatimInput}
\begin{sphinxVerbatimOutput}

\begin{sphinxuseclass}{cell_output}
\noindent\sphinxincludegraphics{{280d786b7b4bae1381e7db7c000320b66e19fe8e554dddc0ac2ad992882cbb2b}.png}

\end{sphinxuseclass}\end{sphinxVerbatimOutput}

\end{sphinxuseclass}

\section{The up option, extending the \sphinxstyleliteralintitle{\sphinxupquote{.tracepre}} beyond the first level causal variables}
\label{\detokenize{content/06_ModelAnalytics/ModelStructure:the-up-option-extending-the-tracepre-beyond-the-first-level-causal-variables}}
\sphinxAtStartPar
The up option allows \sphinxcode{\sphinxupquote{.tracepre}} dependencies to be followed through beyond the first level of causal variables.  Below it is extended to variables as much as three levels back, and restricted to those whose variation explains at least 20 percent of the change in GDP.

\begin{sphinxuseclass}{cell}\begin{sphinxVerbatimInput}

\begin{sphinxuseclass}{cell_input}
\begin{sphinxVerbatim}[commandchars=\\\{\}]
\PYG{n}{mpak}\PYG{o}{.}\PYG{n}{PAKNYGDPMKTPKN}\PYG{o}{.}\PYG{n}{tracepre}\PYG{p}{(}\PYG{n+nb}{filter} \PYG{o}{=} \PYG{l+m+mi}{20}\PYG{p}{,}\PYG{n}{up}\PYG{o}{=}\PYG{l+m+mi}{3}\PYG{p}{,}\PYG{n}{png}\PYG{o}{=}\PYG{n}{latex}\PYG{p}{,}\PYG{p}{)}
\end{sphinxVerbatim}

\end{sphinxuseclass}\end{sphinxVerbatimInput}
\begin{sphinxVerbatimOutput}

\begin{sphinxuseclass}{cell_output}
\noindent\sphinxincludegraphics{{227e371e3743739f860d5ab9c46637df27afbeff803285ee69fc3fc304ae2a3f}.png}

\end{sphinxuseclass}\end{sphinxVerbatimOutput}

\end{sphinxuseclass}

\subsection{The Fokus2 option with the showtable option adds a table to the output}
\label{\detokenize{content/06_ModelAnalytics/ModelStructure:the-fokus2-option-with-the-showtable-option-adds-a-table-to-the-output}}
\begin{sphinxuseclass}{cell}\begin{sphinxVerbatimInput}

\begin{sphinxuseclass}{cell_input}
\begin{sphinxVerbatim}[commandchars=\\\{\}]
\PYG{k}{with} \PYG{n}{mpak}\PYG{o}{.}\PYG{n}{set\PYGZus{}smpl}\PYG{p}{(}\PYG{l+m+mi}{2025}\PYG{p}{,}\PYG{l+m+mi}{2027}\PYG{p}{)}\PYG{p}{:}
    \PYG{n}{mpak}\PYG{o}{.}\PYG{n}{PAKNYGDPMKTPKN}\PYG{o}{.}\PYG{n}{tracepre}\PYG{p}{(}\PYG{n+nb}{filter} \PYG{o}{=} \PYG{l+m+mi}{20}\PYG{p}{,}\PYG{n}{fokus2}\PYG{o}{=}\PYG{l+s+s1}{\PYGZsq{}}\PYG{l+s+s1}{PAKNEGDIFTOTKN PAKNECONPRVTKN PAKNECONGOVTKN PAKNYGDPMKTPKN PAKNEIMPGNFSKN}\PYG{l+s+s1}{\PYGZsq{}}\PYG{p}{,}\PYG{n}{growthshow}\PYG{o}{=}\PYG{k+kc}{True}\PYG{p}{)}
\end{sphinxVerbatim}

\end{sphinxuseclass}\end{sphinxVerbatimInput}
\begin{sphinxVerbatimOutput}

\begin{sphinxuseclass}{cell_output}
\begin{sphinxVerbatim}[commandchars=\\\{\}]
\PYGZlt{}IPython.core.display.SVG object\PYGZgt{}
\end{sphinxVerbatim}

\end{sphinxuseclass}\end{sphinxVerbatimOutput}

\end{sphinxuseclass}

\chapter{\sphinxstyleliteralintitle{\sphinxupquote{.tracedep}} traces the impact of a variable on other variables}
\label{\detokenize{content/06_ModelAnalytics/ModelStructure:tracedep-traces-the-impact-of-a-variable-on-other-variables}}
\begin{sphinxuseclass}{cell}\begin{sphinxVerbatimInput}

\begin{sphinxuseclass}{cell_input}
\begin{sphinxVerbatim}[commandchars=\\\{\}]
\PYG{n}{mpak}\PYG{o}{.}\PYG{n}{PAKNECONPRVTKN}\PYG{o}{.}\PYG{n}{tracedep}\PYG{p}{(}\PYG{n}{down}\PYG{o}{=}\PYG{l+m+mi}{3}\PYG{p}{,}\PYG{n+nb}{filter}\PYG{o}{=}\PYG{l+m+mi}{20}\PYG{p}{)}
\end{sphinxVerbatim}

\end{sphinxuseclass}\end{sphinxVerbatimInput}
\begin{sphinxVerbatimOutput}

\begin{sphinxuseclass}{cell_output}
\begin{sphinxVerbatim}[commandchars=\\\{\}]
\PYGZlt{}IPython.core.display.SVG object\PYGZgt{}
\end{sphinxVerbatim}

\end{sphinxuseclass}\end{sphinxVerbatimOutput}

\end{sphinxuseclass}

\chapter{\sphinxstyleliteralintitle{\sphinxupquote{.modeldash()}} An interactive way to explore dependencies}
\label{\detokenize{content/06_ModelAnalytics/ModelStructure:modeldash-an-interactive-way-to-explore-dependencies}}
\sphinxAtStartPar
The \sphinxcode{\sphinxupquote{.modeldash()}} method (when executed in a Jupyter Notebook) generates a widget that allows you to dynamicaly adjust the arguments to the \sphinxcode{\sphinxupquote{tracepre()}} and \sphinxcode{\sphinxupquote{tracedep}} functions.

\begin{sphinxVerbatim}[commandchars=\\\{\}]
 \PYG{k}{with} \PYG{n}{mpak}\PYG{o}{.}\PYG{n}{set\PYGZus{}smpl}\PYG{p}{(}\PYG{l+m+mi}{2022}\PYG{p}{,}\PYG{l+m+mi}{2026}\PYG{p}{)}\PYG{p}{:}
        \PYG{n}{mpak}\PYG{o}{.}\PYG{n}{modeldash}\PYG{p}{(}\PYG{l+s+s1}{\PYGZsq{}}\PYG{l+s+s1}{PAKNYGDPMKTPKN}\PYG{l+s+s1}{\PYGZsq{}}\PYG{p}{,}\PYG{n}{jupyter}\PYG{o}{=}\PYG{k+kc}{True}\PYG{p}{,}\PYG{n}{inline}\PYG{o}{=}\PYG{k+kc}{False}\PYG{p}{)} 
\end{sphinxVerbatim}

\sphinxAtStartPar
The above commands generate a dashboard that looks a like the below, where the panel to the left allows the user to change options including the filter, the depth of the trace among other things.

\sphinxAtStartPar
\sphinxincludegraphics{{dash}.png}

\begin{sphinxuseclass}{cell}\begin{sphinxVerbatimInput}

\begin{sphinxuseclass}{cell_input}
\begin{sphinxVerbatim}[commandchars=\\\{\}]
\PYG{c+c1}{\PYGZsh{} with mpak.set\PYGZus{}smpl(2022,2026):}
\PYG{c+c1}{\PYGZsh{}        mpak.modeldash(\PYGZsq{}PAKNYGDPMKTPKN\PYGZsq{},jupyter=True,inline=False) }
\end{sphinxVerbatim}

\end{sphinxuseclass}\end{sphinxVerbatimInput}

\end{sphinxuseclass}
\sphinxstepscope


\chapter{Analyzing the impact of a shock}
\label{\detokenize{content/06_ModelAnalytics/AttributionSomeFeatures:analyzing-the-impact-of-a-shock}}\label{\detokenize{content/06_ModelAnalytics/AttributionSomeFeatures::doc}}
\sphinxAtStartPar
When working with a model it is often useful to have a better sense of the contribution of different channels to a final result.  For example, an increase in interest rates will tend to reduce investment and consumer demand – contributing to a reduction in GDP. At the same time, lower inflation as the higher interest rate takes effect will tend to work in the opposite direction.

\sphinxAtStartPar
The \sphinxcode{\sphinxupquote{tracedep()}} and \sphinxcode{\sphinxupquote{tracepre()}} methods introduced in the previous section give a sense of impacts, but the \sphinxcode{\sphinxupquote{modelflow}} methods \sphinxcode{\sphinxupquote{.dekomp()}} and \sphinxcode{\sphinxupquote{.totdif()}} take that one step further by calculating the  contribution of each channel to the overall result.


\section{Prepare the workspace}
\label{\detokenize{content/06_ModelAnalytics/AttributionSomeFeatures:prepare-the-workspace}}
\sphinxAtStartPar
As always before running \sphinxcode{\sphinxupquote{modelflow}} the python environment needs to be initialized and libraries to be used imported.

\begin{sphinxuseclass}{cell}\begin{sphinxVerbatimInput}

\begin{sphinxuseclass}{cell_input}
\begin{sphinxVerbatim}[commandchars=\\\{\}]
\PYG{k+kn}{import} \PYG{n+nn}{pandas} \PYG{k}{as} \PYG{n+nn}{pd}

\PYG{c+c1}{\PYGZsh{} Modules from Modelflow }
\PYG{k+kn}{from} \PYG{n+nn}{modelclass} \PYG{k+kn}{import} \PYG{n}{model} 

\PYG{c+c1}{\PYGZsh{} optional functionalities }
\PYG{n}{model}\PYG{o}{.}\PYG{n}{widescreen}\PYG{p}{(}\PYG{p}{)}
\PYG{n}{model}\PYG{o}{.}\PYG{n}{scroll\PYGZus{}off}\PYG{p}{(}\PYG{p}{)}

\PYG{c+c1}{\PYGZsh{} Output compatabiltity with LaTeX }
\PYG{n}{latex}\PYG{o}{=}\PYG{k+kc}{True}
\end{sphinxVerbatim}

\end{sphinxuseclass}\end{sphinxVerbatimInput}
\begin{sphinxVerbatimOutput}

\begin{sphinxuseclass}{cell_output}
\begin{sphinxVerbatim}[commandchars=\\\{\}]
\PYGZlt{}IPython.core.display.HTML object\PYGZgt{}
\end{sphinxVerbatim}

\end{sphinxuseclass}\end{sphinxVerbatimOutput}

\end{sphinxuseclass}

\section{Load the existing model, data and descriptions}
\label{\detokenize{content/06_ModelAnalytics/AttributionSomeFeatures:load-the-existing-model-data-and-descriptions}}
\sphinxAtStartPar
The file \sphinxcode{\sphinxupquote{pak.pcim}} contains a dump of model equations, dataframe, simulation options and variable descriptions for the World Bank climate aware model for Pakistan described in Burns \sphinxstyleemphasis{et al.} {[}\hyperlink{cite.content/99_BackMatter/References:id14}{2021}{]}. The code below:
\begin{itemize}
\item {} 
\sphinxAtStartPar
Loads the model and simulates it using the dataframe stored in the pcim file to establish a baseline.

\item {} 
\sphinxAtStartPar
Creates a dataframe that is a copy of the active solution in \sphinxcode{\sphinxupquote{mpak}} and changes the tax rate to 30 USD/Ton for carbon emissions from coal, oil and natural gas.

\item {} 
\sphinxAtStartPar
Runs a simulations with these new carbon taxes.

\end{itemize}

\sphinxAtStartPar
The results from this simulation will be used below to explore the attribution functionality of modelflow.

\begin{sphinxuseclass}{cell}\begin{sphinxVerbatimInput}

\begin{sphinxuseclass}{cell_input}
\begin{sphinxVerbatim}[commandchars=\\\{\}]
\PYG{n}{mpak}\PYG{p}{,}\PYG{n}{baseline} \PYG{o}{=} \PYG{n}{model}\PYG{o}{.}\PYG{n}{modelload}\PYG{p}{(}\PYG{l+s+s1}{\PYGZsq{}}\PYG{l+s+s1}{../models/pak.pcim}\PYG{l+s+s1}{\PYGZsq{}}\PYG{p}{,}\PYG{n}{alfa}\PYG{o}{=}\PYG{l+m+mf}{0.7}\PYG{p}{,}\PYG{n}{run}\PYG{o}{=}\PYG{l+m+mi}{1}\PYG{p}{,}\PYG{n}{keep}\PYG{o}{=}\PYG{l+s+s1}{\PYGZsq{}}\PYG{l+s+s1}{Business as Usual}\PYG{l+s+s1}{\PYGZsq{}}\PYG{p}{)}
\PYG{n}{alternative}  \PYG{o}{=}  \PYG{n}{baseline}\PYG{o}{.}\PYG{n}{upd}\PYG{p}{(}\PYG{l+s+s2}{\PYGZdq{}}\PYG{l+s+s2}{\PYGZlt{}2020 2100\PYGZgt{} PAKGGREVCO2CER PAKGGREVCO2GER PAKGGREVCO2OER = 30}\PYG{l+s+s2}{\PYGZdq{}}\PYG{p}{)}
\PYG{n}{result} \PYG{o}{=} \PYG{n}{mpak}\PYG{p}{(}\PYG{n}{alternative}\PYG{p}{,}\PYG{l+m+mi}{2020}\PYG{p}{,}\PYG{l+m+mi}{2100}\PYG{p}{,}\PYG{n}{keep}\PYG{o}{=}\PYG{l+s+s1}{\PYGZsq{}}\PYG{l+s+s1}{Carbon tax nominal 30}\PYG{l+s+s1}{\PYGZsq{}}\PYG{p}{,}\PYG{n}{ljit}\PYG{o}{=}\PYG{k+kc}{False}\PYG{p}{,}\PYG{n}{nfirst}\PYG{o}{=}\PYG{l+m+mi}{800}\PYG{p}{,}\PYG{n}{maxiteration}\PYG{o}{=}\PYG{l+m+mi}{1000}\PYG{p}{)} \PYG{c+c1}{\PYGZsh{} simulates the model }
\end{sphinxVerbatim}

\end{sphinxuseclass}\end{sphinxVerbatimInput}
\begin{sphinxVerbatimOutput}

\begin{sphinxuseclass}{cell_output}
\begin{sphinxVerbatim}[commandchars=\\\{\}]
file read:  C:\PYGZbs{}modelflow manual\PYGZbs{}papers\PYGZbs{}mfbook\PYGZbs{}content\PYGZbs{}models\PYGZbs{}pak.pcim
\end{sphinxVerbatim}

\end{sphinxuseclass}\end{sphinxVerbatimOutput}

\end{sphinxuseclass}

\section{The mathematics of attribution}
\label{\detokenize{content/06_ModelAnalytics/AttributionSomeFeatures:the-mathematics-of-attribution}}
\sphinxAtStartPar
At its root the idea of attribution is simply taking the total derivative of the model to identify the sensitivity of the equation we are interested in to changes elsewhere in the model and then combine that with the changes in other variables.

\sphinxAtStartPar
Take a variable y that is a function of two other variables a and b.  In the model the relationship might be written as:

\sphinxAtStartPar
\(y = f(a,b)\)

\sphinxAtStartPar
If there are two sets of results designated with a subscript 0 and 1, these can be written as:
\label{equation:content/06_ModelAnalytics/AttributionSomeFeatures:6b774984-3ffb-4b9a-85d9-ca7109aa07b8}\begin{eqnarray}
y_0 = f(a_0,b_0)\\
y_1 = f(a_1,b_1)
\end{eqnarray}
\sphinxAtStartPar
then we also have the change in all three variables \(\Delta y, \Delta a, \Delta b\) and the total derivative of y can be written as:

\sphinxAtStartPar
\(\Delta y = \underbrace{\Delta a \dfrac{\partial {f}}{\partial{a}}(a,b)}_{\Omega a} + 
\underbrace{\Delta b \dfrac{\partial {f}}{\partial{b}}(a,b)}_{\Omega b}+Residual\)

\sphinxAtStartPar
The first expresion can be called \(\Omega_a\) or the contribution of changes in a to changes in y, and the second \(\Omega_b\),  or the contribution of changes in b to changes in y.

\sphinxAtStartPar
\sphinxcode{\sphinxupquote{Modelflow}} performs a numerical approximation of \(\Omega_a\) and \(\Omega_b\) by performing two runs of the \(f()\):
\label{equation:content/06_ModelAnalytics/AttributionSomeFeatures:b45ef03c-1918-452a-bdac-9fba5088a072}\begin{eqnarray}  
y_0&=&f(a_{0},b_{0}) \\
y_1&=&f(a_0+\Delta a,b_{0}+ \Delta b)
\end{eqnarray}
\sphinxAtStartPar
and calculates \(\Omega_a\) and \(\Omega_b\) as:
\label{equation:content/06_ModelAnalytics/AttributionSomeFeatures:0b70351f-5a4e-46af-aab3-ca6802e9eab7}\begin{eqnarray}  
\Omega a&=&f(a_1,b_1 )-f(a_1-\Delta a,b_1) \\
\Omega b&=&f(a_1,b_1 )-f(a_1,b_1-\Delta  b)
\end{eqnarray}
\sphinxAtStartPar
And:
\label{equation:content/06_ModelAnalytics/AttributionSomeFeatures:9e31a31e-cb3b-456d-ba68-0b86e11ddeb6}\begin{eqnarray}
residual = \Omega a + \Omega b -(y_1 - y_0) 
\end{eqnarray}
\sphinxAtStartPar
If the model is fairly linear, the residual will be small.


\section{Model attribution or  single equation attribution?}
\label{\detokenize{content/06_ModelAnalytics/AttributionSomeFeatures:model-attribution-or-single-equation-attribution}}
\sphinxAtStartPar
Above the relationship between y, a, and b was summarized by the function f().

\sphinxAtStartPar
\(f(a,b)\) could represent \sphinxstylestrong{a single equation} in the model or it could represent \sphinxstylestrong{the entire model}.

\sphinxAtStartPar
In the \sphinxstylestrong{single equation} instance, \(\Delta a\) and \(\Delta b\) would be treated as exogenous variables in the attribution calculation as they are both on the right hand side of the equation. It does not matter if \(a\) and \(b\) are endogenous of exogenous variables in the complete model.

\sphinxAtStartPar
In the the \sphinxstylestrong{entire model} instance \(a\) and \(b\) are exogenous variables in the model. In this case we are looking to identify the impact of the changes we made to exogenous (or exogenized variables) on different endogenous variables in the model

\sphinxAtStartPar
Assume the simple equation example such that  \(a\) and \(b\) are simple variables. When \(\Delta y\), \(\Delta a\) and \(\Delta b\) reflect the difference across scenarios (say the value of the three variables in \sphinxcode{\sphinxupquote{.lastdf}} less the value in \sphinxcode{\sphinxupquote{.basedf}} then;

\sphinxAtStartPar
\(\Omega_a\), \(\Omega_b\) are the absolute contribution of a and b to the change in y, and
\(100*\bigg[\cfrac{\Omega_a}{\Delta y}\bigg]\)  \(100*\bigg[\cfrac{\Omega_b}{\Delta y}\bigg]\) are the share of the change in y explained by a and b respectively.

\sphinxAtStartPar
If \(\Delta y\), \(\Delta a\) and \(\Delta b\) are the changes over time (\(\Delta y_t=y_t-y_{t-1}\)), then \(\Omega_a\), \(\Omega_b\) are the contributions of a and b to the rate of growth of y, while \(100*\bigg[\cfrac{\Omega_a}{\Delta y_{t-1}}\bigg]\)  \(100*\bigg[\cfrac{\Omega_b}{\Delta y_{t-1}}\bigg]\) are are the contributions of a and b to the rate of growth of y.


\section{Decomposing the source of changes to a single endogenous variable}
\label{\detokenize{content/06_ModelAnalytics/AttributionSomeFeatures:decomposing-the-source-of-changes-to-a-single-endogenous-variable}}
\sphinxAtStartPar
The \sphinxcode{\sphinxupquote{modelflow}} method \sphinxcode{\sphinxupquote{.dekomp()}} is used to calculate the contribution of RHS variables to the change in an endogenous (LHS) variable.

\sphinxAtStartPar
This method takes advantage of the fact that the model object stores the initial and most recent simulation result in two dataframes called \sphinxcode{\sphinxupquote{.basedf}} and \sphinxcode{\sphinxupquote{.lastdf}}, as well as all of the equations of the model.

\sphinxAtStartPar
The \sphinxcode{\sphinxupquote{dekomp()}} method calculates the contribution to changes in the level of the dependent variable in a given equation. It does not calculate what caused the changes to the RHS variables.

\sphinxAtStartPar
In the example below, the contribution to the change in Total emissions is decomposed into the contribution from each of three sources in the model, the consumption of Crude Oil, Natural Gas and Coal.  As the equation for total emissions is just the sum of the three this is a fairly trivial decomposition, but it provides an easily understood illustration of the process at work.

\sphinxAtStartPar
Note that initially some carbon taxes were negative because the products benefited from some sort of subsidy.  As a result, although each carbon tax is set to 30 the change in the levels is different across carbon taxes, with the increase in the net taxation on the carbon emissions from natural gas being particularly large.

\begin{sphinxuseclass}{cell}\begin{sphinxVerbatimInput}

\begin{sphinxuseclass}{cell_input}
\begin{sphinxVerbatim}[commandchars=\\\{\}]
\PYG{k}{with} \PYG{n}{mpak}\PYG{o}{.}\PYG{n}{set\PYGZus{}smpl}\PYG{p}{(}\PYG{l+m+mi}{2020}\PYG{p}{,}\PYG{l+m+mi}{2030}\PYG{p}{)}\PYG{p}{:}
    \PYG{n+nb}{print}\PYG{p}{(}\PYG{n}{mpak}\PYG{p}{[}\PYG{l+s+s1}{\PYGZsq{}}\PYG{l+s+s1}{PAKGGREVCO2CER PAKGGREVCO2GER PAKGGREVCO2OER}\PYG{l+s+s1}{\PYGZsq{}}\PYG{p}{]}\PYG{o}{.}\PYG{n}{dif}\PYG{o}{.}\PYG{n}{df}\PYG{p}{)}
\end{sphinxVerbatim}

\end{sphinxuseclass}\end{sphinxVerbatimInput}
\begin{sphinxVerbatimOutput}

\begin{sphinxuseclass}{cell_output}
\begin{sphinxVerbatim}[commandchars=\\\{\}]
      PAKGGREVCO2CER  PAKGGREVCO2GER  PAKGGREVCO2OER
2020       35.549839       71.000884        38.71065
2021       35.549839       71.000884        38.71065
2022       35.549839       71.000884        38.71065
2023       35.549839       71.000884        38.71065
2024       35.549839       71.000884        38.71065
2025       35.549839       71.000884        38.71065
2026       35.549839       71.000884        38.71065
2027       35.549839       71.000884        38.71065
2028       35.549839       71.000884        38.71065
2029       35.549839       71.000884        38.71065
2030       35.549839       71.000884        38.71065
\end{sphinxVerbatim}

\end{sphinxuseclass}\end{sphinxVerbatimOutput}

\end{sphinxuseclass}
\sphinxAtStartPar
Moreover, because each product has a different carbon intensity (and different shares in total energy use) the contribution of the tax rate change to the change in emissions varies with that intensity. All told, around 40 percent of the decline is attributable to the increase in carbon taxes on coal, 30\sphinxhyphen{}36\% from the carbon tax on oil, and 25\sphinxhyphen{}27\% is due to the increase in the carbon tax on natural gas.

\begin{sphinxuseclass}{cell}\begin{sphinxVerbatimInput}

\begin{sphinxuseclass}{cell_input}
\begin{sphinxVerbatim}[commandchars=\\\{\}]
\PYG{n}{dekomp\PYGZus{}result} \PYG{o}{=} \PYG{n}{mpak}\PYG{o}{.}\PYG{n}{PAKCCEMISCO2TKN}\PYG{o}{.}\PYG{n}{dekomp}\PYG{p}{(}\PYG{n}{start}\PYG{o}{=}\PYG{l+m+mi}{2020}\PYG{p}{,}\PYG{n}{end}\PYG{o}{=}\PYG{l+m+mi}{2024}\PYG{p}{)}\PYG{p}{;}
\end{sphinxVerbatim}

\end{sphinxuseclass}\end{sphinxVerbatimInput}
\begin{sphinxVerbatimOutput}

\begin{sphinxuseclass}{cell_output}
\begin{sphinxVerbatim}[commandchars=\\\{\}]
Formula        : FRML \PYGZlt{}IDENT\PYGZgt{} PAKCCEMISCO2TKN = PAKCCEMISCO2CKN+PAKCCEMISCO2OKN+PAKCCEMISCO2GKN \PYGZdl{} 

                        2020         2021         2022         2023         2024
Variable    lag                                                                 
Base        0   213515545.24 217548186.56 221072469.97 225253519.79 230370294.27
Alternative 0   153790043.82 158136496.56 162262040.05 167116569.38 173010036.73
Difference  0   \PYGZhy{}59725501.42 \PYGZhy{}59411690.00 \PYGZhy{}58810429.92 \PYGZhy{}58136950.40 \PYGZhy{}57360257.55
Percent     0         \PYGZhy{}27.97       \PYGZhy{}27.31       \PYGZhy{}26.60       \PYGZhy{}25.81       \PYGZhy{}24.90

 Contributions to differende for  PAKCCEMISCO2TKN
                            2020         2021         2022         2023         2024
Variable        lag                                                                 
PAKCCEMISCO2CKN 0   \PYGZhy{}24185661.58 \PYGZhy{}24251583.97 \PYGZhy{}24147394.08 \PYGZhy{}23999340.06 \PYGZhy{}23829476.98
PAKCCEMISCO2OKN 0   \PYGZhy{}14253885.95 \PYGZhy{}14968093.58 \PYGZhy{}15456795.05 \PYGZhy{}15753667.05 \PYGZhy{}15847991.86
PAKCCEMISCO2GKN 0   \PYGZhy{}21285953.90 \PYGZhy{}20192012.46 \PYGZhy{}19206240.79 \PYGZhy{}18383943.30 \PYGZhy{}17682788.70

 Share of contributions to differende for  PAKCCEMISCO2TKN
                           2020        2021        2022        2023        2024
Variable        lag                                                            
PAKCCEMISCO2CKN 0           40\PYGZpc{}         41\PYGZpc{}         41\PYGZpc{}         41\PYGZpc{}         42\PYGZpc{}
PAKCCEMISCO2GKN 0           36\PYGZpc{}         34\PYGZpc{}         33\PYGZpc{}         32\PYGZpc{}         31\PYGZpc{}
PAKCCEMISCO2OKN 0           24\PYGZpc{}         25\PYGZpc{}         26\PYGZpc{}         27\PYGZpc{}         28\PYGZpc{}
Total           0          100\PYGZpc{}        100\PYGZpc{}        100\PYGZpc{}        100\PYGZpc{}        100\PYGZpc{}
Residual        0           \PYGZhy{}0\PYGZpc{}          0\PYGZpc{}          0\PYGZpc{}          0\PYGZpc{}          0\PYGZpc{}

 Difference in growth rate PAKCCEMISCO2TKN
                       2020        2021        2022        2023        2024
Variable    lag                                                            
Base        0          2.8\PYGZpc{}        1.9\PYGZpc{}        1.6\PYGZpc{}        1.9\PYGZpc{}        2.3\PYGZpc{}
Alternative 0        \PYGZhy{}25.9\PYGZpc{}        2.8\PYGZpc{}        2.6\PYGZpc{}        3.0\PYGZpc{}        3.5\PYGZpc{}
Difference  0        \PYGZhy{}28.8\PYGZpc{}        0.9\PYGZpc{}        1.0\PYGZpc{}        1.1\PYGZpc{}        1.3\PYGZpc{}
None

 Contribution to growth rate PAKCCEMISCO2TKN
                           2020        2021        2022        2023        2024
Variable        lag                                                            
PAKCCEMISCO2CKN 0        \PYGZhy{}11.6\PYGZpc{}      \PYGZhy{}15.8\PYGZpc{}      \PYGZhy{}15.3\PYGZpc{}      \PYGZhy{}14.8\PYGZpc{}      \PYGZhy{}14.3\PYGZpc{}
PAKCCEMISCO2OKN 0         \PYGZhy{}6.9\PYGZpc{}       \PYGZhy{}9.7\PYGZpc{}       \PYGZhy{}9.8\PYGZpc{}       \PYGZhy{}9.7\PYGZpc{}       \PYGZhy{}9.5\PYGZpc{}
PAKCCEMISCO2GKN 0        \PYGZhy{}10.3\PYGZpc{}      \PYGZhy{}13.1\PYGZpc{}      \PYGZhy{}12.1\PYGZpc{}      \PYGZhy{}11.3\PYGZpc{}      \PYGZhy{}10.6\PYGZpc{}
Total           0        \PYGZhy{}28.8\PYGZpc{}      \PYGZhy{}38.6\PYGZpc{}      \PYGZhy{}37.2\PYGZpc{}      \PYGZhy{}35.8\PYGZpc{}      \PYGZhy{}34.3\PYGZpc{}
Residual        0          0.0\PYGZpc{}      \PYGZhy{}39.6\PYGZpc{}      \PYGZhy{}38.2\PYGZpc{}      \PYGZhy{}36.9\PYGZpc{}      \PYGZhy{}35.6\PYGZpc{}
\end{sphinxVerbatim}

\end{sphinxuseclass}\end{sphinxVerbatimOutput}

\end{sphinxuseclass}
\sphinxAtStartPar
The above results from the call to \sphinxcode{\sphinxupquote{.dekomp()}} are presented in several sections.


\begin{savenotes}\sphinxattablestart
\centering
\begin{tabulary}{\linewidth}[t]{|T|T|T|}
\hline
\sphinxstyletheadfamily 
\sphinxAtStartPar
Section
&\sphinxstyletheadfamily 
\sphinxAtStartPar
Table
&\sphinxstyletheadfamily 
\sphinxAtStartPar
Contents
\\
\hline
\sphinxAtStartPar
\sphinxstylestrong{The first section}
&
\sphinxAtStartPar

&
\sphinxAtStartPar
the normalized formula of the RHS variable \sphinxcode{\sphinxupquote{PAKCCEMISCO2TKN}}
\\
\hline
\sphinxAtStartPar
\sphinxstylestrong{The second section}
&
\sphinxAtStartPar

&
\sphinxAtStartPar
Shows the changes in level terms.
\\
\hline
\sphinxAtStartPar

&
\sphinxAtStartPar
diff\_level
&
\sphinxAtStartPar
First by showing the results of the simulation \sphinxstylestrong{base}, then the previous level \sphinxstylestrong{last}, then the difference and then the difference expressed as a percent
\\
\hline
\sphinxAtStartPar

&
\sphinxAtStartPar
att\_level
&
\sphinxAtStartPar
This is followed by a table showing the contribution of the changes in every LHS variable to the observed change in the dependent variable.
\\
\hline
\sphinxAtStartPar
\sphinxstylestrong{The third section}
&
\sphinxAtStartPar
att\_pct
&
\sphinxAtStartPar
Shows the same results for the RHS variables, but expressed as a percent of the total change in the dependent variable.
\\
\hline
\sphinxAtStartPar
\sphinxstylestrong{The fourth section}
&
\sphinxAtStartPar

&
\sphinxAtStartPar
Shows the same results but for the change in the growth rate of the dependent variable.  !
\\
\hline
\sphinxAtStartPar

&
\sphinxAtStartPar
diff\_growth
&
\sphinxAtStartPar
The first table shows the post\sphinxhyphen{}shock growth rate of the dependent variable from the \sphinxcode{\sphinxupquote{.lastdf}} dataframe, followed by the pre\sphinxhyphen{}shock growth rate and the difference in the growth rates.
\\
\hline
\sphinxAtStartPar

&
\sphinxAtStartPar
att\_growth
&
\sphinxAtStartPar
The second table of this section shows the contribution to the change in the growth rate from each RHS variable.
\\
\hline
\end{tabulary}
\par
\sphinxattableend\end{savenotes}

\sphinxAtStartPar
The object returned by \sphinxcode{\sphinxupquote{.dekomp()}} is a \sphinxhref{https://realpython.com/python-namedtuple/}{namedtuple} that contains each of these tables which can then be referred to later.

\begin{sphinxuseclass}{cell}\begin{sphinxVerbatimInput}

\begin{sphinxuseclass}{cell_input}
\begin{sphinxVerbatim}[commandchars=\\\{\}]
\PYG{c+c1}{\PYGZsh{} Loop over the elements in the result of dekomp. }
\PYG{c+c1}{\PYGZsh{} a named tuple can be used both as a straight tuple and the elements}
\PYG{c+c1}{\PYGZsh{} can be accessed through the field name. }

\PYG{k}{for} \PYG{n}{f}\PYG{p}{,}\PYG{n}{df} \PYG{o+ow}{in} \PYG{n+nb}{zip}\PYG{p}{(}\PYG{n}{dekomp\PYGZus{}result}\PYG{o}{.}\PYG{n}{\PYGZus{}fields}\PYG{p}{,}\PYG{n}{dekomp\PYGZus{}result}\PYG{p}{)}\PYG{p}{:}
    \PYG{n}{display}\PYG{p}{(}\PYG{n}{f}\PYG{p}{)}
    \PYG{n}{display}\PYG{p}{(}\PYG{n}{df}\PYG{p}{)}
    
\end{sphinxVerbatim}

\end{sphinxuseclass}\end{sphinxVerbatimInput}
\begin{sphinxVerbatimOutput}

\begin{sphinxuseclass}{cell_output}
\begin{sphinxVerbatim}[commandchars=\\\{\}]
\PYGZsq{}diff\PYGZus{}level\PYGZsq{}
\end{sphinxVerbatim}

\begin{sphinxVerbatim}[commandchars=\\\{\}]
                             2020              2021              2022   
Variable    lag                                                         
Base        0    213515545.236496  217548186.562568  221072469.966585  \PYGZbs{}
Alternative 0    153790043.815063  158136496.557737  162262040.051477   
Difference  0    \PYGZhy{}59725501.421434  \PYGZhy{}59411690.004831  \PYGZhy{}58810429.915109   
Percent     0          \PYGZhy{}27.972437        \PYGZhy{}27.309669        \PYGZhy{}26.602331   

                             2023              2024  
Variable    lag                                      
Base        0    225253519.786842  230370294.274522  
Alternative 0    167116569.384125  173010036.728357  
Difference  0    \PYGZhy{}58136950.402717  \PYGZhy{}57360257.546165  
Percent     0          \PYGZhy{}25.809564        \PYGZhy{}24.899155  
\end{sphinxVerbatim}

\begin{sphinxVerbatim}[commandchars=\\\{\}]
\PYGZsq{}att\PYGZus{}level\PYGZsq{}
\end{sphinxVerbatim}

\begin{sphinxVerbatim}[commandchars=\\\{\}]
                                2020             2021             2022   
Variable        lag                                                      
PAKCCEMISCO2CKN 0   \PYGZhy{}24185661.576144  \PYGZhy{}24251583.96689 \PYGZhy{}24147394.079948  \PYGZbs{}
PAKCCEMISCO2OKN 0   \PYGZhy{}14253885.945374 \PYGZhy{}14968093.581607 \PYGZhy{}15456795.048495   
PAKCCEMISCO2GKN 0   \PYGZhy{}21285953.899916 \PYGZhy{}20192012.456334 \PYGZhy{}19206240.786665   

                                2023             2024  
Variable        lag                                    
PAKCCEMISCO2CKN 0   \PYGZhy{}23999340.058995 \PYGZhy{}23829476.980619  
PAKCCEMISCO2OKN 0   \PYGZhy{}15753667.045575 \PYGZhy{}15847991.864322  
PAKCCEMISCO2GKN 0   \PYGZhy{}18383943.298148 \PYGZhy{}17682788.701224  
\end{sphinxVerbatim}

\begin{sphinxVerbatim}[commandchars=\\\{\}]
\PYGZsq{}att\PYGZus{}pct\PYGZsq{}
\end{sphinxVerbatim}

\begin{sphinxVerbatim}[commandchars=\\\{\}]
                             2020          2021        2022        2023   
Variable        lag                                                       
PAKCCEMISCO2CKN 0    4.049470e+01  4.081955e+01   41.059714   41.280700  \PYGZbs{}
PAKCCEMISCO2GKN 0    3.563964e+01  3.398660e+01   32.657882   31.621788   
PAKCCEMISCO2OKN 0    2.386566e+01  2.519385e+01   26.282404   27.097512   
Total           0    1.000000e+02  1.000000e+02  100.000000  100.000000   
Residual        0   \PYGZhy{}9.947598e\PYGZhy{}14  4.263256e\PYGZhy{}14    0.000000    0.000000   

                             2024  
Variable        lag                
PAKCCEMISCO2CKN 0    4.154353e+01  
PAKCCEMISCO2GKN 0    3.082760e+01  
PAKCCEMISCO2OKN 0    2.762887e+01  
Total           0    1.000000e+02  
Residual        0    4.263256e\PYGZhy{}14  
\end{sphinxVerbatim}

\begin{sphinxVerbatim}[commandchars=\\\{\}]
\PYGZsq{}diff\PYGZus{}growth\PYGZsq{}
\end{sphinxVerbatim}

\begin{sphinxVerbatim}[commandchars=\\\{\}]
                      2020      2021      2022      2023      2024
Variable    lag                                                   
Base        0      2.81828  1.888687  1.620001  1.891258  2.271563
Alternative 0   \PYGZhy{}25.942499  2.826225   2.60885  2.991784  3.526561
Difference  0   \PYGZhy{}28.760779  0.937538  0.988848  1.100526  1.254998
\end{sphinxVerbatim}

\begin{sphinxVerbatim}[commandchars=\\\{\}]
\PYGZsq{}att\PYGZus{}growth\PYGZsq{}
\end{sphinxVerbatim}

\begin{sphinxVerbatim}[commandchars=\\\{\}]
                          2020       2021       2022       2023       2024
Variable        lag                                                       
PAKCCEMISCO2CKN 0   \PYGZhy{}11.646591 \PYGZhy{}15.769281 \PYGZhy{}15.269969 \PYGZhy{}14.790483 \PYGZhy{}14.259195
PAKCCEMISCO2OKN 0     \PYGZhy{}6.86395  \PYGZhy{}9.732811  \PYGZhy{}9.774338  \PYGZhy{}9.708782  \PYGZhy{}9.483196
PAKCCEMISCO2GKN 0   \PYGZhy{}10.250238 \PYGZhy{}13.129597 \PYGZhy{}12.145356 \PYGZhy{}11.329787  \PYGZhy{}10.58111
Total           0   \PYGZhy{}28.760779 \PYGZhy{}38.631688 \PYGZhy{}37.189663 \PYGZhy{}35.829052 \PYGZhy{}34.323501
Residual        0          0.0 \PYGZhy{}39.569226 \PYGZhy{}38.178511 \PYGZhy{}36.929578 \PYGZhy{}35.578499
\end{sphinxVerbatim}

\end{sphinxuseclass}\end{sphinxVerbatimOutput}

\end{sphinxuseclass}

\subsection{A more complex example}
\label{\detokenize{content/06_ModelAnalytics/AttributionSomeFeatures:a-more-complex-example}}
\sphinxAtStartPar
The above decomposition is fairly straight forward because the equation that we are decomposing is a simple identity, where Total Emissions are just the sum of its three component parts: Total Carbon emissions = Emissions from Oil+  Emissions from Coal + Emissions from Natural Gas.

\sphinxAtStartPar
The following single\sphinxhyphen{}equation decomposition looks to the impact of the same shock (introduction of a carbon tax) on inflation.  The inflation equation is more complex and has more direct causal variables, so here the decomposition is more interesting.

\sphinxAtStartPar
Recall the inflation equation is given by the \sphinxcode{\sphinxupquote{.frml}} method for its normalized version and \sphinxcode{\sphinxupquote{.eviews}} for its original specification.  The equation for the consumer price level (PAKNECONPRVTXN) was originally specified in eviews as:

\begin{sphinxuseclass}{cell}\begin{sphinxVerbatimInput}

\begin{sphinxuseclass}{cell_input}
\begin{sphinxVerbatim}[commandchars=\\\{\}]
\PYG{n}{mpak}\PYG{p}{[}\PYG{l+s+s1}{\PYGZsq{}}\PYG{l+s+s1}{PAKNECONPRVTXN}\PYG{l+s+s1}{\PYGZsq{}}\PYG{p}{]}\PYG{o}{.}\PYG{n}{eviews}
\end{sphinxVerbatim}

\end{sphinxuseclass}\end{sphinxVerbatimInput}
\begin{sphinxVerbatimOutput}

\begin{sphinxuseclass}{cell_output}
\begin{sphinxVerbatim}[commandchars=\\\{\}]
PAKNECONPRVTXN : @IDENTITY PAKNECONPRVTXN  = ((PAKNECONENGYSH\PYGZca{}PAKCESENGYCON)  * PAKNECONENGYXN\PYGZca{}(1  \PYGZhy{} PAKCESENGYCON)  + (PAKNECONOTHRSH\PYGZca{}PAKCESENGYCON)  * PAKNECONOTHRXN\PYGZca{}(1  \PYGZhy{} PAKCESENGYCON))\PYGZca{}(1  / (1  \PYGZhy{} PAKCESENGYCON))
\end{sphinxVerbatim}

\end{sphinxuseclass}\end{sphinxVerbatimOutput}

\end{sphinxuseclass}
\sphinxAtStartPar
When normalized the equation solves for the \sphinxstylestrong{level} of the price deflator.  It is this normalized equation that is:

\begin{sphinxuseclass}{cell}\begin{sphinxVerbatimInput}

\begin{sphinxuseclass}{cell_input}
\begin{sphinxVerbatim}[commandchars=\\\{\}]
\PYG{n}{mpak}\PYG{p}{[}\PYG{l+s+s1}{\PYGZsq{}}\PYG{l+s+s1}{PAKNECONPRVTXN}\PYG{l+s+s1}{\PYGZsq{}}\PYG{p}{]}\PYG{o}{.}\PYG{n}{frml}
\end{sphinxVerbatim}

\end{sphinxuseclass}\end{sphinxVerbatimInput}
\begin{sphinxVerbatimOutput}

\begin{sphinxuseclass}{cell_output}
\begin{sphinxVerbatim}[commandchars=\\\{\}]
PAKNECONPRVTXN : FRML \PYGZlt{}IDENT\PYGZgt{} PAKNECONPRVTXN = ((PAKNECONENGYSH**PAKCESENGYCON)*PAKNECONENGYXN**(1\PYGZhy{}PAKCESENGYCON)+(PAKNECONOTHRSH**PAKCESENGYCON)*PAKNECONOTHRXN**(1\PYGZhy{}PAKCESENGYCON))**(1/(1\PYGZhy{}PAKCESENGYCON)) \PYGZdl{}
\end{sphinxVerbatim}

\end{sphinxuseclass}\end{sphinxVerbatimOutput}

\end{sphinxuseclass}
\sphinxAtStartPar
Because the normalized equation solves for the level of the price deflator, the decomposition will show the contributions of each explanatory variable to the increase in the price level (not that of the inflation rate).

\sphinxAtStartPar
Note in the Pakistan model, consumer inflation is derived as a CET aggregation of the price of energy goods(PAKNECONENGYXN) and non\sphinxhyphen{}energy goods (PAKNECONOTHRXN).

\begin{sphinxuseclass}{cell}\begin{sphinxVerbatimInput}

\begin{sphinxuseclass}{cell_input}
\begin{sphinxVerbatim}[commandchars=\\\{\}]
\PYG{n}{mpak}\PYG{p}{[}\PYG{l+s+s1}{\PYGZsq{}}\PYG{l+s+s1}{PAKNECONPRVTXN}\PYG{l+s+s1}{\PYGZsq{}}\PYG{p}{]}\PYG{o}{.}\PYG{n}{dekomp}\PYG{p}{(}\PYG{n}{start}\PYG{o}{=}\PYG{l+m+mi}{2020}\PYG{p}{,}\PYG{n}{end}\PYG{o}{=}\PYG{l+m+mi}{2024}\PYG{p}{)}\PYG{p}{;}
\end{sphinxVerbatim}

\end{sphinxuseclass}\end{sphinxVerbatimInput}
\begin{sphinxVerbatimOutput}

\begin{sphinxuseclass}{cell_output}
\begin{sphinxVerbatim}[commandchars=\\\{\}]
Formula        : FRML \PYGZlt{}IDENT\PYGZgt{} PAKNECONPRVTXN = ((PAKNECONENGYSH**PAKCESENGYCON)*PAKNECONENGYXN**(1\PYGZhy{}PAKCESENGYCON)+(PAKNECONOTHRSH**PAKCESENGYCON)*PAKNECONOTHRXN**(1\PYGZhy{}PAKCESENGYCON))**(1/(1\PYGZhy{}PAKCESENGYCON)) \PYGZdl{} 

                      2020       2021       2022       2023       2024
Variable    lag                                                       
Base        0         1.67       1.82       1.98       2.14       2.30
Alternative 0         1.72       1.89       2.07       2.23       2.39
Difference  0         0.06       0.07       0.08       0.09       0.10
Percent     0         3.40       3.84       4.11       4.21       4.18

 Contributions to differende for  PAKNECONPRVTXN
                         2020       2021       2022       2023       2024
Variable       lag                                                       
PAKNECONENGYSH 0        \PYGZhy{}0.00      \PYGZhy{}0.00      \PYGZhy{}0.00      \PYGZhy{}0.00      \PYGZhy{}0.00
PAKCESENGYCON  0        \PYGZhy{}0.00      \PYGZhy{}0.00      \PYGZhy{}0.00      \PYGZhy{}0.00      \PYGZhy{}0.00
PAKNECONENGYXN 0         0.01       0.01       0.01       0.01       0.01
PAKNECONOTHRSH 0        \PYGZhy{}0.00      \PYGZhy{}0.00      \PYGZhy{}0.00      \PYGZhy{}0.00      \PYGZhy{}0.00
PAKNECONOTHRXN 0         0.04       0.06       0.07       0.08       0.08

 Share of contributions to differende for  PAKNECONPRVTXN
                          2020        2021        2022        2023        2024
Variable       lag                                                            
PAKNECONOTHRXN 0           77\PYGZpc{}         81\PYGZpc{}         83\PYGZpc{}         85\PYGZpc{}         86\PYGZpc{}
PAKNECONENGYXN 0           23\PYGZpc{}         20\PYGZpc{}         17\PYGZpc{}         16\PYGZpc{}         15\PYGZpc{}
PAKNECONENGYSH 0           \PYGZhy{}0\PYGZpc{}         \PYGZhy{}0\PYGZpc{}         \PYGZhy{}0\PYGZpc{}         \PYGZhy{}0\PYGZpc{}         \PYGZhy{}0\PYGZpc{}
PAKCESENGYCON  0           \PYGZhy{}0\PYGZpc{}         \PYGZhy{}0\PYGZpc{}         \PYGZhy{}0\PYGZpc{}         \PYGZhy{}0\PYGZpc{}         \PYGZhy{}0\PYGZpc{}
PAKNECONOTHRSH 0           \PYGZhy{}0\PYGZpc{}         \PYGZhy{}0\PYGZpc{}         \PYGZhy{}0\PYGZpc{}         \PYGZhy{}0\PYGZpc{}         \PYGZhy{}0\PYGZpc{}
Total          0          101\PYGZpc{}        101\PYGZpc{}        101\PYGZpc{}        101\PYGZpc{}        101\PYGZpc{}
Residual       0            1\PYGZpc{}          1\PYGZpc{}          1\PYGZpc{}          1\PYGZpc{}          1\PYGZpc{}

 Difference in growth rate PAKNECONPRVTXN
                       2020        2021        2022        2023        2024
Variable    lag                                                            
Base        0          9.8\PYGZpc{}        9.5\PYGZpc{}        8.8\PYGZpc{}        8.0\PYGZpc{}        7.3\PYGZpc{}
Alternative 0         13.5\PYGZpc{}        9.9\PYGZpc{}        9.1\PYGZpc{}        8.1\PYGZpc{}        7.2\PYGZpc{}
Difference  0          3.7\PYGZpc{}        0.5\PYGZpc{}        0.3\PYGZpc{}        0.1\PYGZpc{}       \PYGZhy{}0.0\PYGZpc{}
None

 Contribution to growth rate PAKNECONPRVTXN
                          2020        2021        2022        2023        2024
Variable       lag                                                            
PAKNECONENGYSH 0         \PYGZhy{}0.0\PYGZpc{}       \PYGZhy{}0.0\PYGZpc{}       \PYGZhy{}0.0\PYGZpc{}       \PYGZhy{}0.0\PYGZpc{}       \PYGZhy{}0.0\PYGZpc{}
PAKCESENGYCON  0         \PYGZhy{}0.0\PYGZpc{}       \PYGZhy{}0.0\PYGZpc{}       \PYGZhy{}0.0\PYGZpc{}       \PYGZhy{}0.0\PYGZpc{}       \PYGZhy{}0.0\PYGZpc{}
PAKNECONENGYXN 0          0.9\PYGZpc{}        0.8\PYGZpc{}        0.7\PYGZpc{}        0.7\PYGZpc{}        0.6\PYGZpc{}
PAKNECONOTHRSH 0         \PYGZhy{}0.0\PYGZpc{}       \PYGZhy{}0.0\PYGZpc{}       \PYGZhy{}0.0\PYGZpc{}       \PYGZhy{}0.0\PYGZpc{}       \PYGZhy{}0.0\PYGZpc{}
PAKNECONOTHRXN 0          2.9\PYGZpc{}        3.3\PYGZpc{}        3.6\PYGZpc{}        3.7\PYGZpc{}        3.7\PYGZpc{}
Total          0          3.8\PYGZpc{}        4.1\PYGZpc{}        4.3\PYGZpc{}        4.4\PYGZpc{}        4.3\PYGZpc{}
Residual       0          0.0\PYGZpc{}        3.6\PYGZpc{}        4.0\PYGZpc{}        4.3\PYGZpc{}        4.4\PYGZpc{}
\end{sphinxVerbatim}

\end{sphinxuseclass}\end{sphinxVerbatimOutput}

\end{sphinxuseclass}
\sphinxAtStartPar
Interestingly only 25\% of the increase in the price level each period is due to the direct channel (the impact on the price of energy consumed by households), the bulk of the increase comes indirectly through other prices.  Indeed as time progresses this share rises from 77\% in the first year of the price change (2020) to 86\% by 2024.

\sphinxAtStartPar
Below is the formula for nonenergy consumer prices and its decomposition. This equation is written out as a more standard inflation equation reflecting changes in the cost of local goods production (PAKNYGDPFCSTXN), Government taxes on goods and services (PAKGGREVGNFSXN), the price of imports (PAKNEIMPGNGSXN) and the influence of the economic cycle (PAKNYGDPGAP\_).

\begin{sphinxuseclass}{cell}\begin{sphinxVerbatimInput}

\begin{sphinxuseclass}{cell_input}
\begin{sphinxVerbatim}[commandchars=\\\{\}]
\PYG{n}{mpak}\PYG{p}{[}\PYG{l+s+s1}{\PYGZsq{}}\PYG{l+s+s1}{PAKNECONOTHRXN}\PYG{l+s+s1}{\PYGZsq{}}\PYG{p}{]}\PYG{o}{.}\PYG{n}{eviews}
\end{sphinxVerbatim}

\end{sphinxuseclass}\end{sphinxVerbatimInput}
\begin{sphinxVerbatimOutput}

\begin{sphinxuseclass}{cell_output}
\begin{sphinxVerbatim}[commandchars=\\\{\}]
PAKNECONOTHRXN : DLOG(PAKNECONOTHRXN) = 0.590372627657176*DLOG(PAKNYGDPFCSTXN) + D(PAKGGREVGNFSXN/100) + (1 \PYGZhy{} 0.590372627657176)*DLOG(PAKNEIMPGNFSXN) + 0.2*PAKNYGDPGAP\PYGZus{}/100
\end{sphinxVerbatim}

\end{sphinxuseclass}\end{sphinxVerbatimOutput}

\end{sphinxuseclass}
\begin{sphinxuseclass}{cell}\begin{sphinxVerbatimInput}

\begin{sphinxuseclass}{cell_input}
\begin{sphinxVerbatim}[commandchars=\\\{\}]
\PYG{n}{mpak}\PYG{p}{[}\PYG{l+s+s1}{\PYGZsq{}}\PYG{l+s+s1}{PAKNECONOTHRXN}\PYG{l+s+s1}{\PYGZsq{}}\PYG{p}{]}\PYG{o}{.}\PYG{n}{dekomp}\PYG{p}{(}\PYG{n}{start}\PYG{o}{=}\PYG{l+m+mi}{2020}\PYG{p}{,}\PYG{n}{end}\PYG{o}{=}\PYG{l+m+mi}{2024}\PYG{p}{)}\PYG{p}{;}
\end{sphinxVerbatim}

\end{sphinxuseclass}\end{sphinxVerbatimInput}
\begin{sphinxVerbatimOutput}

\begin{sphinxuseclass}{cell_output}
\begin{sphinxVerbatim}[commandchars=\\\{\}]
Formula        :
\end{sphinxVerbatim}

\begin{sphinxVerbatim}[commandchars=\\\{\}]
 FRML \PYGZlt{}DAMP,STOC\PYGZgt{} PAKNECONOTHRXN = (PAKNECONOTHRXN(\PYGZhy{}1)*EXP(PAKNECONOTHRXN\PYGZus{}A+ (0.590372627657176*((LOG(PAKNYGDPFCSTXN))\PYGZhy{}(LOG(PAKNYGDPFCSTXN(\PYGZhy{}1))))+((PAKGGREVGNFSXN/100)\PYGZhy{}(PAKGGREVGNFSXN(\PYGZhy{}1)/100))+(1\PYGZhy{}0.590372627657176)*((LOG(PAKNEIMPGNFSXN))\PYGZhy{}(LOG(PAKNEIMPGNFSXN(\PYGZhy{}1))))+0.2*PAKNYGDPGAP\PYGZus{}/100) )) * (1\PYGZhy{}PAKNECONOTHRXN\PYGZus{}D)+ PAKNECONOTHRXN\PYGZus{}X*PAKNECONOTHRXN\PYGZus{}D  \PYGZdl{} 

                      2020       2021       2022       2023       2024
Variable    lag                                                       
Base        0         1.70       1.86       2.02       2.18       2.34
Alternative 0         1.74       1.92       2.09       2.26       2.43
Difference  0         0.05       0.06       0.07       0.08       0.09
Percent     0         2.68       3.18       3.50       3.65       3.67

 Contributions to differende for  PAKNECONOTHRXN
                           2020       2021       2022       2023       2024
Variable         lag                                                       
PAKNECONOTHRXN   \PYGZhy{}1       \PYGZhy{}0.00       0.05       0.06       0.08       0.09
PAKNECONOTHRXN\PYGZus{}A  0       \PYGZhy{}0.00      \PYGZhy{}0.00      \PYGZhy{}0.00      \PYGZhy{}0.00      \PYGZhy{}0.00
PAKNYGDPFCSTXN    0        0.00       0.01       0.01       0.02       0.02
                 \PYGZhy{}1       \PYGZhy{}0.00      \PYGZhy{}0.00      \PYGZhy{}0.01      \PYGZhy{}0.01      \PYGZhy{}0.02
PAKGGREVGNFSXN    0       \PYGZhy{}0.00      \PYGZhy{}0.00      \PYGZhy{}0.00      \PYGZhy{}0.00      \PYGZhy{}0.00
                 \PYGZhy{}1       \PYGZhy{}0.00      \PYGZhy{}0.00      \PYGZhy{}0.00      \PYGZhy{}0.00      \PYGZhy{}0.00
PAKNEIMPGNFSXN    0        0.04       0.04       0.05       0.05       0.05
                 \PYGZhy{}1       \PYGZhy{}0.00      \PYGZhy{}0.05      \PYGZhy{}0.05      \PYGZhy{}0.05      \PYGZhy{}0.05
PAKNYGDPGAP\PYGZus{}      0        0.00       0.00       0.00       0.00       0.00
PAKNECONOTHRXN\PYGZus{}D  0       \PYGZhy{}0.00      \PYGZhy{}0.00      \PYGZhy{}0.00      \PYGZhy{}0.00      \PYGZhy{}0.00
PAKNECONOTHRXN\PYGZus{}X  0       \PYGZhy{}0.00      \PYGZhy{}0.00      \PYGZhy{}0.00      \PYGZhy{}0.00      \PYGZhy{}0.00

 Share of contributions to differende for  PAKNECONOTHRXN
                            2020        2021        2022        2023        2024
Variable         lag                                                            
PAKNECONOTHRXN   \PYGZhy{}1          \PYGZhy{}0\PYGZpc{}         85\PYGZpc{}         91\PYGZpc{}         96\PYGZpc{}        100\PYGZpc{}
PAKNEIMPGNFSXN    0          92\PYGZpc{}         75\PYGZpc{}         66\PYGZpc{}         61\PYGZpc{}         58\PYGZpc{}
PAKNYGDPFCSTXN    0           2\PYGZpc{}         13\PYGZpc{}         19\PYGZpc{}         23\PYGZpc{}         26\PYGZpc{}
PAKNYGDPGAP\PYGZus{}      0           7\PYGZpc{}          7\PYGZpc{}          4\PYGZpc{}          2\PYGZpc{}          0\PYGZpc{}
PAKNECONOTHRXN\PYGZus{}A  0          \PYGZhy{}0\PYGZpc{}         \PYGZhy{}0\PYGZpc{}         \PYGZhy{}0\PYGZpc{}         \PYGZhy{}0\PYGZpc{}         \PYGZhy{}0\PYGZpc{}
PAKGGREVGNFSXN    0          \PYGZhy{}0\PYGZpc{}         \PYGZhy{}0\PYGZpc{}         \PYGZhy{}0\PYGZpc{}         \PYGZhy{}0\PYGZpc{}         \PYGZhy{}0\PYGZpc{}
                 \PYGZhy{}1          \PYGZhy{}0\PYGZpc{}         \PYGZhy{}0\PYGZpc{}         \PYGZhy{}0\PYGZpc{}         \PYGZhy{}0\PYGZpc{}         \PYGZhy{}0\PYGZpc{}
PAKNECONOTHRXN\PYGZus{}D  0          \PYGZhy{}0\PYGZpc{}         \PYGZhy{}0\PYGZpc{}         \PYGZhy{}0\PYGZpc{}         \PYGZhy{}0\PYGZpc{}         \PYGZhy{}0\PYGZpc{}
PAKNECONOTHRXN\PYGZus{}X  0          \PYGZhy{}0\PYGZpc{}         \PYGZhy{}0\PYGZpc{}         \PYGZhy{}0\PYGZpc{}         \PYGZhy{}0\PYGZpc{}         \PYGZhy{}0\PYGZpc{}
PAKNYGDPFCSTXN   \PYGZhy{}1          \PYGZhy{}0\PYGZpc{}         \PYGZhy{}2\PYGZpc{}        \PYGZhy{}12\PYGZpc{}        \PYGZhy{}19\PYGZpc{}        \PYGZhy{}23\PYGZpc{}
PAKNEIMPGNFSXN   \PYGZhy{}1          \PYGZhy{}0\PYGZpc{}        \PYGZhy{}80\PYGZpc{}        \PYGZhy{}70\PYGZpc{}        \PYGZhy{}65\PYGZpc{}        \PYGZhy{}62\PYGZpc{}
Total             0         100\PYGZpc{}         99\PYGZpc{}         99\PYGZpc{}         99\PYGZpc{}         99\PYGZpc{}
Residual          0           0\PYGZpc{}         \PYGZhy{}1\PYGZpc{}         \PYGZhy{}1\PYGZpc{}         \PYGZhy{}1\PYGZpc{}         \PYGZhy{}1\PYGZpc{}

 Difference in growth rate PAKNECONOTHRXN
                       2020        2021        2022        2023        2024
Variable    lag                                                            
Base        0          9.8\PYGZpc{}        9.5\PYGZpc{}        8.8\PYGZpc{}        8.0\PYGZpc{}        7.3\PYGZpc{}
Alternative 0         12.8\PYGZpc{}       10.1\PYGZpc{}        9.2\PYGZpc{}        8.2\PYGZpc{}        7.3\PYGZpc{}
Difference  0          2.9\PYGZpc{}        0.5\PYGZpc{}        0.3\PYGZpc{}        0.2\PYGZpc{}        0.0\PYGZpc{}
None

 Contribution to growth rate PAKNECONOTHRXN
                            2020        2021        2022        2023        2024
Variable         lag                                                            
PAKNECONOTHRXN   \PYGZhy{}1        \PYGZhy{}0.0\PYGZpc{}       \PYGZhy{}0.0\PYGZpc{}       \PYGZhy{}0.0\PYGZpc{}       \PYGZhy{}0.0\PYGZpc{}       \PYGZhy{}0.0\PYGZpc{}
PAKNECONOTHRXN\PYGZus{}A  0        \PYGZhy{}0.0\PYGZpc{}       \PYGZhy{}0.0\PYGZpc{}       \PYGZhy{}0.0\PYGZpc{}       \PYGZhy{}0.0\PYGZpc{}       \PYGZhy{}0.0\PYGZpc{}
PAKNYGDPFCSTXN    0         0.1\PYGZpc{}        0.4\PYGZpc{}        0.7\PYGZpc{}        0.9\PYGZpc{}        1.0\PYGZpc{}
                 \PYGZhy{}1        \PYGZhy{}0.0\PYGZpc{}       \PYGZhy{}0.1\PYGZpc{}       \PYGZhy{}0.4\PYGZpc{}       \PYGZhy{}0.7\PYGZpc{}       \PYGZhy{}0.9\PYGZpc{}
PAKGGREVGNFSXN    0        \PYGZhy{}0.0\PYGZpc{}       \PYGZhy{}0.0\PYGZpc{}       \PYGZhy{}0.0\PYGZpc{}       \PYGZhy{}0.0\PYGZpc{}       \PYGZhy{}0.0\PYGZpc{}
                 \PYGZhy{}1        \PYGZhy{}0.0\PYGZpc{}       \PYGZhy{}0.0\PYGZpc{}       \PYGZhy{}0.0\PYGZpc{}       \PYGZhy{}0.0\PYGZpc{}       \PYGZhy{}0.0\PYGZpc{}
PAKNEIMPGNFSXN    0         2.7\PYGZpc{}        2.6\PYGZpc{}        2.4\PYGZpc{}        2.3\PYGZpc{}        2.2\PYGZpc{}
                 \PYGZhy{}1        \PYGZhy{}0.0\PYGZpc{}       \PYGZhy{}2.7\PYGZpc{}       \PYGZhy{}2.6\PYGZpc{}       \PYGZhy{}2.5\PYGZpc{}       \PYGZhy{}2.4\PYGZpc{}
PAKNYGDPGAP\PYGZus{}      0         0.2\PYGZpc{}        0.2\PYGZpc{}        0.2\PYGZpc{}        0.1\PYGZpc{}        0.0\PYGZpc{}
PAKNECONOTHRXN\PYGZus{}D  0        \PYGZhy{}0.0\PYGZpc{}       \PYGZhy{}0.0\PYGZpc{}       \PYGZhy{}0.0\PYGZpc{}       \PYGZhy{}0.0\PYGZpc{}       \PYGZhy{}0.0\PYGZpc{}
PAKNECONOTHRXN\PYGZus{}X  0        \PYGZhy{}0.0\PYGZpc{}       \PYGZhy{}0.0\PYGZpc{}       \PYGZhy{}0.0\PYGZpc{}       \PYGZhy{}0.0\PYGZpc{}       \PYGZhy{}0.0\PYGZpc{}
Total             0         3.0\PYGZpc{}        0.5\PYGZpc{}        0.3\PYGZpc{}        0.1\PYGZpc{}       \PYGZhy{}0.0\PYGZpc{}
Residual          0         0.0\PYGZpc{}       \PYGZhy{}0.1\PYGZpc{}       \PYGZhy{}0.1\PYGZpc{}       \PYGZhy{}0.1\PYGZpc{}       \PYGZhy{}0.1\PYGZpc{}
\end{sphinxVerbatim}

\end{sphinxuseclass}\end{sphinxVerbatimOutput}

\end{sphinxuseclass}
\sphinxAtStartPar
These results indicate that much of the initial impact on prices is coming from the increase in the price of imported goods (which includes a large fuel component). As time progresses, the imported inflation component declines and the lagged consumption price dominates.  Other factors such as the cost of domestically produced goods play a larger role and the net impact of imported prices (the total of the contemporaneous and lagged value) approaches zero. Cyclical pressure are initially adding to inflation before declining and eventually turning negative.


\subsection{The \sphinxstyleliteralintitle{\sphinxupquote{get\_att()}} method provides more control over the outputs of \sphinxstyleliteralintitle{\sphinxupquote{.dekomp()}}}
\label{\detokenize{content/06_ModelAnalytics/AttributionSomeFeatures:the-get-att-method-provides-more-control-over-the-outputs-of-dekomp}}
\sphinxAtStartPar
Following a call to the \sphinxcode{\sphinxupquote{.dekomp()}} method, the \sphinxcode{\sphinxupquote{.get\_att()}} method provides a range of mechanisms that allow the results to be displayed in different ways.


\subsubsection{The default display of \sphinxstyleliteralintitle{\sphinxupquote{.get\_act()}}}
\label{\detokenize{content/06_ModelAnalytics/AttributionSomeFeatures:the-default-display-of-get-act}}
\sphinxAtStartPar
By default \sphinxcode{\sphinxupquote{.get\_att()}} displays the share contributions of RHS variables to the total change in the LHS variable.  The start= and end= options allow us to restrict the period for which results are displayed.

\begin{sphinxuseclass}{cell}\begin{sphinxVerbatimInput}

\begin{sphinxuseclass}{cell_input}
\begin{sphinxVerbatim}[commandchars=\\\{\}]
\PYG{n}{mpak}\PYG{o}{.}\PYG{n}{PAKNECONPRVTKN}\PYG{o}{.}\PYG{n}{get\PYGZus{}att}\PYG{p}{(}\PYG{n}{start}\PYG{o}{=}\PYG{l+m+mi}{2020}\PYG{p}{,}\PYG{n}{end}\PYG{o}{=}\PYG{l+m+mi}{2030}\PYG{p}{)}
\end{sphinxVerbatim}

\end{sphinxuseclass}\end{sphinxVerbatimInput}
\begin{sphinxVerbatimOutput}

\begin{sphinxuseclass}{cell_output}
\begin{sphinxVerbatim}[commandchars=\\\{\}]
\PYGZlt{}pandas.io.formats.style.Styler at 0x22439f3bc40\PYGZgt{}
\end{sphinxVerbatim}

\end{sphinxuseclass}\end{sphinxVerbatimOutput}

\end{sphinxuseclass}

\subsubsection{Options: Lag=True/False}
\label{\detokenize{content/06_ModelAnalytics/AttributionSomeFeatures:options-lag-true-false}}
\sphinxAtStartPar
Because the decomposition of the equation is based on the normalized (levelized) version of the equation, for equations initially written as growth rates or ECMs many variables will occur several times in the attribution table with both the contribution of the current value of the variable and those of any lagged versions that appear in the normalized equation.

\sphinxAtStartPar
The \sphinxcode{\sphinxupquote{Lag=False}} option changes the default behaviour of \sphinxcode{\sphinxupquote{.get\_att()}} and aggregates the contributions of different lags.

\sphinxAtStartPar
By aggregating the lags the net effect of changes in the variables can be more easily determined. Below it is clearer that the initial impact of higher import prices drove most of the inflation response.  IN subsequent periods, import prices were stable or even falling so most of the contribution to the change in the level of other goods inflation was from the lagged dependent variable and the changed state of the economic cycle (the Gap variable which initially was adding to inflationary pressures eventually subtracts from inflation as the economy slows).

\begin{sphinxuseclass}{cell}\begin{sphinxVerbatimInput}

\begin{sphinxuseclass}{cell_input}
\begin{sphinxVerbatim}[commandchars=\\\{\}]
\PYG{n}{mpak}\PYG{o}{.}\PYG{n}{PAKNECONOTHRXN}\PYG{o}{.}\PYG{n}{get\PYGZus{}att}\PYG{p}{(}\PYG{n}{lag}\PYG{o}{=}\PYG{k+kc}{False}\PYG{p}{)}
\end{sphinxVerbatim}

\end{sphinxuseclass}\end{sphinxVerbatimInput}
\begin{sphinxVerbatimOutput}

\begin{sphinxuseclass}{cell_output}
\begin{sphinxVerbatim}[commandchars=\\\{\}]
\PYGZlt{}pandas.io.formats.style.Styler at 0x2243c1f2fe0\PYGZgt{}
\end{sphinxVerbatim}

\end{sphinxuseclass}\end{sphinxVerbatimOutput}

\end{sphinxuseclass}

\subsubsection{Options: Type=”growth/pct/Level”}
\label{\detokenize{content/06_ModelAnalytics/AttributionSomeFeatures:options-type-growth-pct-level}}
\sphinxAtStartPar
The option \sphinxcode{\sphinxupquote{Type}} controls which of the tables generated by \sphinxcode{\sphinxupquote{dekomp()}} is displayed. The contributions of RHS variables to the level of the dependent variable (\sphinxcode{\sphinxupquote{=level}}), the share of the observed change in the RHS variable attributable to each dependent variable (\sphinxcode{\sphinxupquote{=pct}}), and the change in the growth rate of the RHS available attributable to the changes in the growth rate of each RHS variable.


\subsubsection{Options: threshold=xx”}
\label{\detokenize{content/06_ModelAnalytics/AttributionSomeFeatures:options-threshold-xx}}
\sphinxAtStartPar
The \sphinxcode{\sphinxupquote{threshold=}} option will suppress from the output those variables whose contribution is less than the stated threshold.

\sphinxAtStartPar
In the example below, lags are suppressed, and only contributions to the growth rate of variables whose contributions was more than 0.1 percent are displayed.  The dropped variables influence is aggregated and displayed in a row labeled \sphinxstylestrong{small}.


\subsubsection{Options: bare=True/False}
\label{\detokenize{content/06_ModelAnalytics/AttributionSomeFeatures:options-bare-true-false}}
\sphinxAtStartPar
If bare is set to \sphinxcode{\sphinxupquote{False}} then the values of the LHS variable in the \sphinxcode{\sphinxupquote{basedf}} and \sphinxcode{\sphinxupquote{lastdf}} dataframes and their difference are also displayed. By default this option is \sphinxcode{\sphinxupquote{True}} which suppresses the display.

\begin{sphinxuseclass}{cell}\begin{sphinxVerbatimInput}

\begin{sphinxuseclass}{cell_input}
\begin{sphinxVerbatim}[commandchars=\\\{\}]
\PYG{n}{mpak}\PYG{o}{.}\PYG{n}{PAKNECONPRVTKN}\PYG{o}{.}\PYG{n}{get\PYGZus{}att}\PYG{p}{(}\PYG{n}{lag}\PYG{o}{=}\PYG{k+kc}{False}\PYG{p}{,}\PYG{n+nb}{type}\PYG{o}{=}\PYG{l+s+s1}{\PYGZsq{}}\PYG{l+s+s1}{growth}\PYG{l+s+s1}{\PYGZsq{}}\PYG{p}{,}\PYG{n}{bare}\PYG{o}{=}\PYG{k+kc}{False}\PYG{p}{,}\PYG{n}{threshold}\PYG{o}{=}\PYG{l+m+mf}{0.1}\PYG{p}{)}
\end{sphinxVerbatim}

\end{sphinxuseclass}\end{sphinxVerbatimInput}
\begin{sphinxVerbatimOutput}

\begin{sphinxuseclass}{cell_output}
\begin{sphinxVerbatim}[commandchars=\\\{\}]
\PYGZlt{}pandas.io.formats.style.Styler at 0x22441661150\PYGZgt{}
\end{sphinxVerbatim}

\end{sphinxuseclass}\end{sphinxVerbatimOutput}

\end{sphinxuseclass}

\subsubsection{Several examples}
\label{\detokenize{content/06_ModelAnalytics/AttributionSomeFeatures:several-examples}}
\sphinxAtStartPar
Here the default type (pct) is displayed and the threshold is set to 10, so only variables whose aggregate impact was more than 10 percent of the total in one or more of the displayed years are shown.

\begin{sphinxuseclass}{cell}\begin{sphinxVerbatimInput}

\begin{sphinxuseclass}{cell_input}
\begin{sphinxVerbatim}[commandchars=\\\{\}]
\PYG{n}{mpak}\PYG{o}{.}\PYG{n}{PAKNECONPRVTKN}\PYG{o}{.}\PYG{n}{get\PYGZus{}att}\PYG{p}{(}\PYG{n}{lag}\PYG{o}{=}\PYG{k+kc}{False}\PYG{p}{,}\PYG{n}{threshold}\PYG{o}{=}\PYG{l+m+mi}{10}\PYG{p}{,}\PYG{n}{start}\PYG{o}{=}\PYG{l+m+mi}{2020}\PYG{p}{,}\PYG{n}{end}\PYG{o}{=}\PYG{l+m+mi}{2030}\PYG{p}{)}
\end{sphinxVerbatim}

\end{sphinxuseclass}\end{sphinxVerbatimInput}
\begin{sphinxVerbatimOutput}

\begin{sphinxuseclass}{cell_output}
\begin{sphinxVerbatim}[commandchars=\\\{\}]
\PYGZlt{}pandas.io.formats.style.Styler at 0x22441662cb0\PYGZgt{}
\end{sphinxVerbatim}

\end{sphinxuseclass}\end{sphinxVerbatimOutput}

\end{sphinxuseclass}
\sphinxAtStartPar
The three examples below show the impact on real consumption of the changes induced on its LHS variables as changes in percent level and growth, with the threshold set to focus only on the main channels.

\begin{sphinxuseclass}{cell}\begin{sphinxVerbatimInput}

\begin{sphinxuseclass}{cell_input}
\begin{sphinxVerbatim}[commandchars=\\\{\}]
\PYG{n}{mpak}\PYG{o}{.}\PYG{n}{PAKNECONPRVTKN}\PYG{o}{.}\PYG{n}{get\PYGZus{}att}\PYG{p}{(}\PYG{n}{lag}\PYG{o}{=}\PYG{k+kc}{False}\PYG{p}{,}\PYG{n}{threshold}\PYG{o}{=}\PYG{l+m+mi}{10}\PYG{p}{,}\PYG{n}{bare}\PYG{o}{=}\PYG{k+kc}{False}\PYG{p}{)}
\PYG{n}{mpak}\PYG{o}{.}\PYG{n}{PAKNECONPRVTKN}\PYG{o}{.}\PYG{n}{get\PYGZus{}att}\PYG{p}{(}\PYG{n}{lag}\PYG{o}{=}\PYG{k+kc}{False}\PYG{p}{,}\PYG{n}{threshold}\PYG{o}{=}\PYG{l+m+mi}{10}\PYG{p}{,}\PYG{n+nb}{type}\PYG{o}{=}\PYG{l+s+s1}{\PYGZsq{}}\PYG{l+s+s1}{level}\PYG{l+s+s1}{\PYGZsq{}}\PYG{p}{,}\PYG{n}{bare}\PYG{o}{=}\PYG{k+kc}{False}\PYG{p}{)}
\PYG{n}{mpak}\PYG{o}{.}\PYG{n}{PAKNECONPRVTKN}\PYG{o}{.}\PYG{n}{get\PYGZus{}att}\PYG{p}{(}\PYG{n}{lag}\PYG{o}{=}\PYG{k+kc}{False}\PYG{p}{,}\PYG{n}{threshold}\PYG{o}{=}\PYG{l+m+mf}{0.1}\PYG{p}{,}\PYG{n+nb}{type}\PYG{o}{=}\PYG{l+s+s1}{\PYGZsq{}}\PYG{l+s+s1}{growth}\PYG{l+s+s1}{\PYGZsq{}}\PYG{p}{,}\PYG{n}{bare}\PYG{o}{=}\PYG{k+kc}{False}\PYG{p}{)}
\end{sphinxVerbatim}

\end{sphinxuseclass}\end{sphinxVerbatimInput}
\begin{sphinxVerbatimOutput}

\begin{sphinxuseclass}{cell_output}
\begin{sphinxVerbatim}[commandchars=\\\{\}]
\PYGZlt{}pandas.io.formats.style.Styler at 0x224416628c0\PYGZgt{}
\end{sphinxVerbatim}

\begin{sphinxVerbatim}[commandchars=\\\{\}]
\PYGZlt{}pandas.io.formats.style.Styler at 0x224416613f0\PYGZgt{}
\end{sphinxVerbatim}

\begin{sphinxVerbatim}[commandchars=\\\{\}]
\PYGZlt{}pandas.io.formats.style.Styler at 0x22441660be0\PYGZgt{}
\end{sphinxVerbatim}

\end{sphinxuseclass}\end{sphinxVerbatimOutput}

\end{sphinxuseclass}

\subsection{Displaying \sphinxstyleliteralintitle{\sphinxupquote{.dekomp()}} results graphically}
\label{\detokenize{content/06_ModelAnalytics/AttributionSomeFeatures:displaying-dekomp-results-graphically}}
\sphinxAtStartPar
The \sphinxcode{\sphinxupquote{dekomp\_plot}} method allows singe\sphinxhyphen{}equation decompositions to be displayed graphically. Below in the initial periods, the import price, and cost\sphinxhyphen{}push factors dominate, but as the model equilibrates the lagged level of the price deflator explains virtually all of the movement in the level of the price.

\begin{sphinxuseclass}{cell}\begin{sphinxVerbatimInput}

\begin{sphinxuseclass}{cell_input}
\begin{sphinxVerbatim}[commandchars=\\\{\}]
\PYG{n}{fig}\PYG{o}{=}\PYG{n}{mpak}\PYG{o}{.}\PYG{n}{dekomp\PYGZus{}plot}\PYG{p}{(}\PYG{l+s+s1}{\PYGZsq{}}\PYG{l+s+s1}{PAKNECONOTHRXN}\PYG{l+s+s1}{\PYGZsq{}}\PYG{p}{,}\PYG{n}{pct}\PYG{o}{=}\PYG{k+kc}{True}\PYG{p}{,}\PYG{n}{rename}\PYG{o}{=}\PYG{k+kc}{True}\PYG{p}{,}\PYG{n}{threshold}\PYG{o}{=}\PYG{l+m+mf}{.01}\PYG{p}{,}\PYG{n}{lag}\PYG{o}{=}\PYG{k+kc}{False}\PYG{p}{)}\PYG{p}{;} \PYG{c+c1}{\PYGZsh{}decomp of teh change in the level}
\end{sphinxVerbatim}

\end{sphinxuseclass}\end{sphinxVerbatimInput}
\begin{sphinxVerbatimOutput}

\begin{sphinxuseclass}{cell_output}
\noindent\sphinxincludegraphics{{708a79e106893cb878d7a9a5afaef8fb5c70b9a796b226c6ad9ff17a4c82233e}.png}

\end{sphinxuseclass}\end{sphinxVerbatimOutput}

\end{sphinxuseclass}
\sphinxAtStartPar
In the following example the change in the level of the dependent variable is displayed for a restricted time period.  Here the distinction between the initial impulse (import prices) and the lagged effect of past prices is very evident.

\begin{sphinxuseclass}{cell}\begin{sphinxVerbatimInput}

\begin{sphinxuseclass}{cell_input}
\begin{sphinxVerbatim}[commandchars=\\\{\}]
\PYG{k}{with} \PYG{n}{mpak}\PYG{o}{.}\PYG{n}{set\PYGZus{}smpl}\PYG{p}{(}\PYG{l+m+mi}{2020}\PYG{p}{,}\PYG{l+m+mi}{2030}\PYG{p}{)}\PYG{p}{:}
    \PYG{n}{fig}\PYG{o}{=}\PYG{n}{mpak}\PYG{o}{.}\PYG{n}{dekomp\PYGZus{}plot}\PYG{p}{(}\PYG{l+s+s1}{\PYGZsq{}}\PYG{l+s+s1}{PAKNECONOTHRXN}\PYG{l+s+s1}{\PYGZsq{}}\PYG{p}{,}\PYG{n}{pct}\PYG{o}{=}\PYG{k+kc}{False}\PYG{p}{,}\PYG{n}{rename}\PYG{o}{=}\PYG{k+kc}{True}\PYG{p}{,}\PYG{n}{threshold}\PYG{o}{=}\PYG{l+m+mf}{.005}\PYG{p}{,}\PYG{n}{lag}\PYG{o}{=}\PYG{k+kc}{False}\PYG{p}{)}\PYG{p}{;} \PYG{c+c1}{\PYGZsh{}decomp of teh change in the level}
\end{sphinxVerbatim}

\end{sphinxuseclass}\end{sphinxVerbatimInput}
\begin{sphinxVerbatimOutput}

\begin{sphinxuseclass}{cell_output}
\noindent\sphinxincludegraphics{{d8aead5519e017059bfe3cc06e4a138ba8fa601f199c59991ceca8188882f9df}.png}

\end{sphinxuseclass}\end{sphinxVerbatimOutput}

\end{sphinxuseclass}

\subsection{the time\_att option}
\label{\detokenize{content/06_ModelAnalytics/AttributionSomeFeatures:the-time-att-option}}
\sphinxAtStartPar
The above displays focused on the difference between the values in the two dataframes \sphinxcode{\sphinxupquote{basedf}} and \sphinxcode{\sphinxupquote{lastdf}}.

\sphinxAtStartPar
By setting he time\_att option to True, \sphinxcode{\sphinxupquote{get\_att()}} displays the contribution of changes in the levels of the RHS variables between t and t\sphinxhyphen{}1,in explaining the changes in the LHS variable between t and t\sphinxhyphen{}1 with all data pulled from the same \sphinxcode{\sphinxupquote{.tasdf}} datafame.

\begin{sphinxVerbatim}[commandchars=\\\{\}]
With the `time\PYGZus{}att` option set **only the .lastdf dataframe** is used.   The comparison is not .basedf vs .lastdf but the influence of last year\PYGZsq{}s changes on the level of this year\PYGZsq{}s variable. The attribution is calculated by lagging each right hand side variable one year and recalculation the equation.
\end{sphinxVerbatim}

\begin{sphinxuseclass}{cell}\begin{sphinxVerbatimInput}

\begin{sphinxuseclass}{cell_input}
\begin{sphinxVerbatim}[commandchars=\\\{\}]
\PYG{n}{mpak}\PYG{o}{.}\PYG{n}{PAKCCEMISCO2TKN} \PYG{o}{.}\PYG{n}{get\PYGZus{}att}\PYG{p}{(}\PYG{n}{time\PYGZus{}att}\PYG{o}{=} \PYG{k+kc}{True}\PYG{p}{,}\PYG{n+nb}{type}\PYG{o}{=}\PYG{l+s+s1}{\PYGZsq{}}\PYG{l+s+s1}{level}\PYG{l+s+s1}{\PYGZsq{}}\PYG{p}{,}\PYG{n}{bare}\PYG{o}{=}\PYG{l+m+mi}{0}\PYG{p}{)}
\end{sphinxVerbatim}

\end{sphinxuseclass}\end{sphinxVerbatimInput}
\begin{sphinxVerbatimOutput}

\begin{sphinxuseclass}{cell_output}
\begin{sphinxVerbatim}[commandchars=\\\{\}]
\PYGZlt{}pandas.io.formats.style.Styler at 0x22441663b80\PYGZgt{}
\end{sphinxVerbatim}

\end{sphinxuseclass}\end{sphinxVerbatimOutput}

\end{sphinxuseclass}
\begin{sphinxuseclass}{cell}\begin{sphinxVerbatimInput}

\begin{sphinxuseclass}{cell_input}
\begin{sphinxVerbatim}[commandchars=\\\{\}]
\PYG{n}{help}\PYG{p}{(}\PYG{n}{mpak}\PYG{o}{.}\PYG{n}{get\PYGZus{}att}\PYG{p}{)}
\end{sphinxVerbatim}

\end{sphinxuseclass}\end{sphinxVerbatimInput}
\begin{sphinxVerbatimOutput}

\begin{sphinxuseclass}{cell_output}
\begin{sphinxVerbatim}[commandchars=\\\{\}]
Help on method get\PYGZus{}att in module modelclass:

get\PYGZus{}att(n, type=\PYGZsq{}pct\PYGZsq{}, filter=False, lag=True, start=\PYGZsq{}\PYGZsq{}, end=\PYGZsq{}\PYGZsq{}, time\PYGZus{}att=False, threshold=0.0) method of modelclass.model instance
    Calculate the attribution percentage for a variable.
    
    Parameters:
        n (str): Name of the variable to calculate attribution for.
        type (str): Type of attribution calculation. Options: \PYGZsq{}pct\PYGZsq{} (percentage), \PYGZsq{}level\PYGZsq{}, \PYGZsq{}growth\PYGZsq{}. Default: \PYGZsq{}pct\PYGZsq{}.
        filter (bool): [Deprecated] Use threshold instead of filter. Default: False.
        lag (bool): Flag to indicate whether to include lag information in the output. Default: True.
        start (str): Start period for calculation. If not provided, uses the first period in the model instance. Default: \PYGZsq{}\PYGZsq{}.
        end (str): End period for calculation. If not provided, uses the last period in the model instance. Default: \PYGZsq{}\PYGZsq{}.
        time\PYGZus{}att (bool): Flag to indicate time attribute calculation. Default: False.
        threshold (float): Threshold value for excluding rows with values close to zero. Default: 0.0.
    
    Returns:
        pandas.DataFrame: DataFrame containing the calculated attribution results.
    
    Raises:
        Exception: If an invalid type is provided.
\end{sphinxVerbatim}

\end{sphinxuseclass}\end{sphinxVerbatimOutput}

\end{sphinxuseclass}
\begin{sphinxuseclass}{cell}\begin{sphinxVerbatimInput}

\begin{sphinxuseclass}{cell_input}
\begin{sphinxVerbatim}[commandchars=\\\{\}]
\PYG{n}{help}\PYG{p}{(}\PYG{n}{mpak}\PYG{o}{.}\PYG{n}{dekomp\PYGZus{}plot}\PYG{p}{)}
\end{sphinxVerbatim}

\end{sphinxuseclass}\end{sphinxVerbatimInput}
\begin{sphinxVerbatimOutput}

\begin{sphinxuseclass}{cell_output}
\begin{sphinxVerbatim}[commandchars=\\\{\}]
Help on method dekomp\PYGZus{}plot in module modelclass:

dekomp\PYGZus{}plot(varnavn, sort=True, pct=True, per=\PYGZsq{}\PYGZsq{}, top=0.9, threshold=0.0, lag=True, rename=True, nametrans=\PYGZlt{}function Dekomp\PYGZus{}Mixin.\PYGZlt{}lambda\PYGZgt{} at 0x0000022437E6E170\PYGZgt{}, time\PYGZus{}att=False) method of modelclass.model instance
    Returns  a chart with attribution for a variable over the smpl  
    
    Parameters
    \PYGZhy{}\PYGZhy{}\PYGZhy{}\PYGZhy{}\PYGZhy{}\PYGZhy{}\PYGZhy{}\PYGZhy{}\PYGZhy{}\PYGZhy{}
    varnavn : TYPE
        variable name.
    sort : TYPE, optional
        . The default is False.
    pct : TYPE, optional
        display pct contribution . The default is True.
    per : TYPE, optional
        DESCRIPTION. The default is \PYGZsq{}\PYGZsq{}.
    threshold : TYPE, optional
        cutoff. The default is 0.0.
    rename : TYPE, optional
        Use descriptions instead of variable names. The default is True.
    time\PYGZus{}att : TYPE, optional
        Do time attribution . The default is False.
    lag : TYPE, optional
       separete by lags The default is True.           
    top : TYPE, optional
      where to place the title 
       
    
    Returns
    \PYGZhy{}\PYGZhy{}\PYGZhy{}\PYGZhy{}\PYGZhy{}\PYGZhy{}\PYGZhy{}
    a matplotlib figure instance .
\end{sphinxVerbatim}

\end{sphinxuseclass}\end{sphinxVerbatimOutput}

\end{sphinxuseclass}

\section{Trace and attribution combined}
\label{\detokenize{content/06_ModelAnalytics/AttributionSomeFeatures:trace-and-attribution-combined}}
\sphinxAtStartPar
The \sphinxcode{\sphinxupquote{.tracepre()}} method can combine the graphical representation of the \sphinxcode{\sphinxupquote{tracepre()}} method described in the previous chapter and the tabular results from \sphinxcode{\sphinxupquote{dekomp()}}.

\sphinxAtStartPar
This is implicit in the standard call to \sphinxcode{\sphinxupquote{.tracepre()}}where the thickness of the lines is derived from the empirical importance of the changes in each LHS variable in determining the change in the RHS variable.

\begin{sphinxuseclass}{cell}\begin{sphinxVerbatimInput}

\begin{sphinxuseclass}{cell_input}
\begin{sphinxVerbatim}[commandchars=\\\{\}]
\PYG{n}{mpak}\PYG{o}{.}\PYG{n}{PAKNECONPRVTKN}\PYG{o}{.}\PYG{n}{tracepre}\PYG{p}{(}\PYG{p}{)}
\end{sphinxVerbatim}

\end{sphinxuseclass}\end{sphinxVerbatimInput}
\begin{sphinxVerbatimOutput}

\begin{sphinxuseclass}{cell_output}
\begin{sphinxVerbatim}[commandchars=\\\{\}]
\PYGZlt{}IPython.core.display.SVG object\PYGZgt{}
\end{sphinxVerbatim}

\end{sphinxuseclass}\end{sphinxVerbatimOutput}

\end{sphinxuseclass}

\subsection{Tabular output from \sphinxstyleliteralintitle{\sphinxupquote{tracepre()}}}
\label{\detokenize{content/06_ModelAnalytics/AttributionSomeFeatures:tabular-output-from-tracepre}}
\sphinxAtStartPar
The following call to \sphinxcode{\sphinxupquote{.tracepre}} combines several features of Get\_attr.


\begin{savenotes}\sphinxattablestart
\centering
\begin{tabulary}{\linewidth}[t]{|T|T|}
\hline

\sphinxAtStartPar

&
\sphinxAtStartPar

\\
\hline
\sphinxAtStartPar
\sphinxstylestrong{fokus2all=True}
&
\sphinxAtStartPar
Causes the tables of attribution for each displayed variable to be displayed
\\
\hline
\sphinxAtStartPar
\sphinxstylestrong{attshow = True}
&
\sphinxAtStartPar
adds in the contribution of each to the total change
\\
\hline
\sphinxAtStartPar
\sphinxstylestrong{growthshow = True}
&
\sphinxAtStartPar
will include a table of growth for each variable
\\
\hline
\sphinxAtStartPar
\sphinxstylestrong{HR = True}
&
\sphinxAtStartPar
will reorient the dependency graph
\\
\hline
\sphinxAtStartPar
\sphinxstylestrong{‘up = xx}
&
\sphinxAtStartPar
determines how many levels of parents to include
\\
\hline
\sphinxAtStartPar
\sphinxstylestrong{filter=xx}
&
\sphinxAtStartPar
(a synonym for threshold) restrict outputs to variables that explain at least xx\% of the change in the level of dependent variable
\\
\hline
\sphinxAtStartPar
\sphinxstylestrong{browser = True}
&
\sphinxAtStartPar
Opens a browser with the resulting dependency graph \sphinxhyphen{} useful for zooming on a big graph or table
\\
\hline
\end{tabulary}
\par
\sphinxattableend\end{savenotes}

\sphinxAtStartPar
\sphinxcode{\sphinxupquote{mpak.PAKNECONOTHRXN.tracepre(filter=0.20,HR=True,fokus2all= True,attshow=True)}}

\begin{sphinxuseclass}{cell}\begin{sphinxVerbatimInput}

\begin{sphinxuseclass}{cell_input}
\begin{sphinxVerbatim}[commandchars=\\\{\}]
\PYG{k}{with} \PYG{n}{mpak}\PYG{o}{.}\PYG{n}{set\PYGZus{}smpl}\PYG{p}{(}\PYG{l+m+mi}{2020}\PYG{p}{,}\PYG{l+m+mi}{2030}\PYG{p}{)}\PYG{p}{:}
    \PYG{n}{mpak}\PYG{o}{.}\PYG{n}{PAKNECONOTHRXN}\PYG{o}{.}\PYG{n}{tracepre}\PYG{p}{(}\PYG{n+nb}{filter}\PYG{o}{=}\PYG{l+m+mf}{5.0}\PYG{p}{,}\PYG{n}{HR}\PYG{o}{=}\PYG{k+kc}{False}\PYG{p}{,}\PYG{n}{fokus2all}\PYG{o}{=} \PYG{k+kc}{True}\PYG{p}{,}\PYG{n}{attshow}\PYG{o}{=}\PYG{k+kc}{True}\PYG{p}{,}\PYG{n}{per}\PYG{o}{=}\PYG{l+m+mi}{2020}\PYG{p}{)}
\end{sphinxVerbatim}

\end{sphinxuseclass}\end{sphinxVerbatimInput}
\begin{sphinxVerbatimOutput}

\begin{sphinxuseclass}{cell_output}
\begin{sphinxVerbatim}[commandchars=\\\{\}]
\PYGZlt{}IPython.core.display.SVG object\PYGZgt{}
\end{sphinxVerbatim}

\end{sphinxuseclass}\end{sphinxVerbatimOutput}

\end{sphinxuseclass}
\sphinxAtStartPar
The big difference with this representation is the contributions of both the direct variables that directly impact the LHS variable (as well as the influence of those variables one or two steps up the causal chain) can be traced.

\sphinxAtStartPar
Below the same command as above but we specify that we want to go up two levels in the causal chain.

\begin{sphinxuseclass}{cell}\begin{sphinxVerbatimInput}

\begin{sphinxuseclass}{cell_input}
\begin{sphinxVerbatim}[commandchars=\\\{\}]
\PYG{k}{with} \PYG{n}{mpak}\PYG{o}{.}\PYG{n}{set\PYGZus{}smpl}\PYG{p}{(}\PYG{l+m+mi}{2020}\PYG{p}{,}\PYG{l+m+mi}{2030}\PYG{p}{)}\PYG{p}{:}
    \PYG{n}{mpak}\PYG{o}{.}\PYG{n}{PAKNECONOTHRXN}\PYG{o}{.}\PYG{n}{tracepre}\PYG{p}{(}\PYG{n}{up}\PYG{o}{=}\PYG{l+m+mi}{2}\PYG{p}{,}\PYG{n+nb}{filter}\PYG{o}{=}\PYG{l+m+mi}{5}\PYG{p}{,}\PYG{n}{HR}\PYG{o}{=}\PYG{k+kc}{False}\PYG{p}{,}\PYG{n}{fokus2all}\PYG{o}{=} \PYG{k+kc}{True}\PYG{p}{,}\PYG{n}{attshow}\PYG{o}{=}\PYG{k+kc}{True}\PYG{p}{)}
\end{sphinxVerbatim}

\end{sphinxuseclass}\end{sphinxVerbatimInput}
\begin{sphinxVerbatimOutput}

\begin{sphinxuseclass}{cell_output}
\begin{sphinxVerbatim}[commandchars=\\\{\}]
\PYGZlt{}IPython.core.display.SVG object\PYGZgt{}
\end{sphinxVerbatim}

\end{sphinxuseclass}\end{sphinxVerbatimOutput}

\end{sphinxuseclass}
\begin{sphinxuseclass}{cell}\begin{sphinxVerbatimInput}

\begin{sphinxuseclass}{cell_input}
\begin{sphinxVerbatim}[commandchars=\\\{\}]
\PYG{n}{help}\PYG{p}{(}\PYG{n}{mpak}\PYG{o}{.}\PYG{n}{PAKNECONPRVTKN}\PYG{o}{.}\PYG{n}{tracepre}\PYG{p}{)}
\end{sphinxVerbatim}

\end{sphinxuseclass}\end{sphinxVerbatimInput}
\begin{sphinxVerbatimOutput}

\begin{sphinxuseclass}{cell_output}
\begin{sphinxVerbatim}[commandchars=\\\{\}]
Help on method tracepre in module modelvis:

tracepre(up=1, **kwargs) method of modelvis.varvis instance
    Trace dependensies of name down to level down
\end{sphinxVerbatim}

\end{sphinxuseclass}\end{sphinxVerbatimOutput}

\end{sphinxuseclass}

\subsection{Chart of the contributions over time}
\label{\detokenize{content/06_ModelAnalytics/AttributionSomeFeatures:chart-of-the-contributions-over-time}}
\begin{sphinxuseclass}{cell}\begin{sphinxVerbatimInput}

\begin{sphinxuseclass}{cell_input}
\begin{sphinxVerbatim}[commandchars=\\\{\}]
\PYG{k}{with} \PYG{n}{mpak}\PYG{o}{.}\PYG{n}{set\PYGZus{}smpl}\PYG{p}{(}\PYG{l+m+mi}{2020}\PYG{p}{,}\PYG{l+m+mi}{2025}\PYG{p}{)}\PYG{p}{:}
    \PYG{c+c1}{\PYGZsh{}mpak.PAKNECONOTHRXN.tracepre(filter=0.20,HR=True,fokus2all= True,attshow=True)}
    \PYG{n}{mpak}\PYG{o}{.}\PYG{n}{PAKNECONOTHRXN}\PYG{o}{.}\PYG{n}{tracepre}\PYG{p}{(}\PYG{n+nb}{filter}\PYG{o}{=}\PYG{l+m+mf}{0.20}\PYG{p}{,}\PYG{n}{HR}\PYG{o}{=}\PYG{k+kc}{True}\PYG{p}{,}\PYG{n}{fokus2all}\PYG{o}{=} \PYG{k+kc}{True}\PYG{p}{,}\PYG{n}{attshow}\PYG{o}{=}\PYG{k+kc}{True}\PYG{p}{)}
\end{sphinxVerbatim}

\end{sphinxuseclass}\end{sphinxVerbatimInput}
\begin{sphinxVerbatimOutput}

\begin{sphinxuseclass}{cell_output}
\begin{sphinxVerbatim}[commandchars=\\\{\}]
\PYGZlt{}IPython.core.display.SVG object\PYGZgt{}
\end{sphinxVerbatim}

\end{sphinxuseclass}\end{sphinxVerbatimOutput}

\end{sphinxuseclass}

\subsection{Chart of the contributions for one year}
\label{\detokenize{content/06_ModelAnalytics/AttributionSomeFeatures:chart-of-the-contributions-for-one-year}}
\sphinxAtStartPar
It can be useful to visualize the attribution as a waterfall chart for a single year

\begin{sphinxuseclass}{cell}\begin{sphinxVerbatimInput}

\begin{sphinxuseclass}{cell_input}
\begin{sphinxVerbatim}[commandchars=\\\{\}]
\PYG{n}{mpak}\PYG{o}{.}\PYG{n}{dekomp\PYGZus{}plot\PYGZus{}per}\PYG{p}{(}\PYG{l+s+s1}{\PYGZsq{}}\PYG{l+s+s1}{PAKNYGDPFCSTXN}\PYG{l+s+s1}{\PYGZsq{}}\PYG{p}{,}\PYG{n}{per}\PYG{o}{=}\PYG{l+m+mi}{2021}\PYG{p}{,}\PYG{n}{threshold}\PYG{o}{=}\PYG{l+m+mi}{5}\PYG{p}{)}  \PYG{c+c1}{\PYGZsh{} gives a waterfall of contributions}
\end{sphinxVerbatim}

\end{sphinxuseclass}\end{sphinxVerbatimInput}
\begin{sphinxVerbatimOutput}

\begin{sphinxuseclass}{cell_output}
\noindent\sphinxincludegraphics{{b3df29e7c1af281debca5c1d8ab226acb041fd31b5d6af69658876998ebfdf86}.png}

\end{sphinxuseclass}\end{sphinxVerbatimOutput}

\end{sphinxuseclass}

\subsubsection{The waterfall can be sorted}
\label{\detokenize{content/06_ModelAnalytics/AttributionSomeFeatures:the-waterfall-can-be-sorted}}
\begin{sphinxuseclass}{cell}\begin{sphinxVerbatimInput}

\begin{sphinxuseclass}{cell_input}
\begin{sphinxVerbatim}[commandchars=\\\{\}]
\PYG{n}{mpak}\PYG{o}{.}\PYG{n}{dekomp\PYGZus{}plot\PYGZus{}per}\PYG{p}{(}\PYG{l+s+s1}{\PYGZsq{}}\PYG{l+s+s1}{PAKNYGDPFCSTXN}\PYG{l+s+s1}{\PYGZsq{}}\PYG{p}{,}\PYG{n}{per}\PYG{o}{=}\PYG{l+m+mi}{2029}\PYG{p}{,}\PYG{n}{threshold}\PYG{o}{=}\PYG{l+m+mi}{5}\PYG{p}{,}\PYG{n}{sort}\PYG{o}{=}\PYG{k+kc}{True}\PYG{p}{)}  \PYG{c+c1}{\PYGZsh{} gives a waterfall of contributions}
\end{sphinxVerbatim}

\end{sphinxuseclass}\end{sphinxVerbatimInput}
\begin{sphinxVerbatimOutput}

\begin{sphinxuseclass}{cell_output}
\noindent\sphinxincludegraphics{{6bb394af30691737fe6e5681006dd423e96863248590c53ca76b6e8c6161b168}.png}

\end{sphinxuseclass}\end{sphinxVerbatimOutput}

\end{sphinxuseclass}

\section{Impacts at the model level: the \sphinxstyleliteralintitle{\sphinxupquote{.totdif()}} method}
\label{\detokenize{content/06_ModelAnalytics/AttributionSomeFeatures:impacts-at-the-model-level-the-totdif-method}}
\sphinxAtStartPar
The method \sphinxcode{\sphinxupquote{.totdif()}} returns an instance of the totdif class, which provides a number of methods and properties to explore decomposition at the model level.

\sphinxAtStartPar
It works by solving the model numerous time, each time changing one of the right hand side variables and calculating the impact on all dependent variables. By default it uses the values from the \sphinxcode{\sphinxupquote{.lastdf}} Dataframe as the shock values and the values in \sphinxcode{\sphinxupquote{.basedf}} as the initial values. Separate simulations are run for every exogenous (or exogenized) variables that have changed between the two dataframes.

\sphinxAtStartPar
For advanced users the RHS variables can be grouped into user defined blocks, which in cases where there are many changes can help identify the main causal pathways.


\subsection{The \sphinxstyleliteralintitle{\sphinxupquote{.exo\_dif()}} method}
\label{\detokenize{content/06_ModelAnalytics/AttributionSomeFeatures:the-exo-dif-method}}
\sphinxAtStartPar
The \sphinxcode{\sphinxupquote{.exodif()}} method displays the exogenous variables that have changed between the two dataframes (the shock). It determines which of the exogenous variables  have changed between \sphinxcode{\sphinxupquote{.lastdf}}and \sphinxcode{\sphinxupquote{.basedf}} and then returns a dataframe with the changes in the values.

\sphinxAtStartPar
In this case the dataframe contains the the effect of updating the \(CO^2\) tax to 30 for coal, gas and oil.

\begin{sphinxuseclass}{cell}\begin{sphinxVerbatimInput}

\begin{sphinxuseclass}{cell_input}
\begin{sphinxVerbatim}[commandchars=\\\{\}]
\PYG{n}{mpak}\PYG{o}{.}\PYG{n}{exodif}\PYG{p}{(}\PYG{p}{)}
\end{sphinxVerbatim}

\end{sphinxuseclass}\end{sphinxVerbatimInput}
\begin{sphinxVerbatimOutput}

\begin{sphinxuseclass}{cell_output}
\begin{sphinxVerbatim}[commandchars=\\\{\}]
      PAKGGREVCO2CER  PAKGGREVCO2GER  PAKGGREVCO2OER
2020       35.549839       71.000884        38.71065
2021       35.549839       71.000884        38.71065
2022       35.549839       71.000884        38.71065
2023       35.549839       71.000884        38.71065
2024       35.549839       71.000884        38.71065
...              ...             ...             ...
2096       35.549839       71.000884        38.71065
2097       35.549839       71.000884        38.71065
2098       35.549839       71.000884        38.71065
2099       35.549839       71.000884        38.71065
2100       35.549839       71.000884        38.71065

[81 rows x 3 columns]
\end{sphinxVerbatim}

\end{sphinxuseclass}\end{sphinxVerbatimOutput}

\end{sphinxuseclass}

\subsubsection{The \sphinxstyleliteralintitle{\sphinxupquote{.totdif()}} command calculates the contribution of each changed variable to the changes in a specified LHS variable}
\label{\detokenize{content/06_ModelAnalytics/AttributionSomeFeatures:the-totdif-command-calculates-the-contribution-of-each-changed-variable-to-the-changes-in-a-specified-lhs-variable}}
\sphinxAtStartPar
This involves solving the model a number of times, so can take some time. For this case the solve took 5 seconds.

\begin{sphinxuseclass}{cell}\begin{sphinxVerbatimInput}

\begin{sphinxuseclass}{cell_input}
\begin{sphinxVerbatim}[commandchars=\\\{\}]
\PYG{n}{totdekomp} \PYG{o}{=} \PYG{n}{mpak}\PYG{o}{.}\PYG{n}{totdif}\PYG{p}{(}\PYG{p}{)} \PYG{c+c1}{\PYGZsh{} Calculate the total derivative½s of all equations in the model.}
\end{sphinxVerbatim}

\end{sphinxuseclass}\end{sphinxVerbatimInput}
\begin{sphinxVerbatimOutput}

\begin{sphinxuseclass}{cell_output}
\begin{sphinxVerbatim}[commandchars=\\\{\}]
Total dekomp took       :         3.241 Seconds
\end{sphinxVerbatim}

\end{sphinxuseclass}\end{sphinxVerbatimOutput}

\end{sphinxuseclass}

\subsection{The method \sphinxstyleliteralintitle{\sphinxupquote{.explain\_all()}} presents the results graphically}
\label{\detokenize{content/06_ModelAnalytics/AttributionSomeFeatures:the-method-explain-all-presents-the-results-graphically}}
\sphinxAtStartPar
In the example below, the relative importance of the three shocked carbon taxes on the change in real GDP are presented.

\begin{sphinxuseclass}{cell}\begin{sphinxVerbatimInput}

\begin{sphinxuseclass}{cell_input}
\begin{sphinxVerbatim}[commandchars=\\\{\}]
\PYG{n}{showvar} \PYG{o}{=} \PYG{l+s+s1}{\PYGZsq{}}\PYG{l+s+s1}{PAKNYGDPMKTPKN}\PYG{l+s+s1}{\PYGZsq{}}
\PYG{n}{totdekomp}\PYG{o}{.}\PYG{n}{explain\PYGZus{}all}\PYG{p}{(}\PYG{n}{showvar}\PYG{p}{,}\PYG{n}{kind}\PYG{o}{=}\PYG{l+s+s1}{\PYGZsq{}}\PYG{l+s+s1}{area}\PYG{l+s+s1}{\PYGZsq{}}\PYG{p}{,}\PYG{n}{use}\PYG{o}{=}\PYG{l+s+s1}{\PYGZsq{}}\PYG{l+s+s1}{growth}\PYG{l+s+s1}{\PYGZsq{}}\PYG{p}{,}\PYG{n}{stacked}\PYG{o}{=}\PYG{k+kc}{True}\PYG{p}{,}
                      \PYG{n}{title}\PYG{o}{=}\PYG{l+s+s2}{\PYGZdq{}}\PYG{l+s+s2}{Contributions of different carbon taxes to Real GDP growth}\PYG{l+s+s2}{\PYGZdq{}}\PYG{p}{)} \PYG{p}{;}
\end{sphinxVerbatim}

\end{sphinxuseclass}\end{sphinxVerbatimInput}
\begin{sphinxVerbatimOutput}

\begin{sphinxuseclass}{cell_output}
\noindent\sphinxincludegraphics{{d2b9044aae28f05dff9f04cea2cb25f18f8540c2a4190fd6e4805949b5b249cc}.png}

\end{sphinxuseclass}\end{sphinxVerbatimOutput}

\end{sphinxuseclass}
\begin{sphinxuseclass}{cell}\begin{sphinxVerbatimInput}

\begin{sphinxuseclass}{cell_input}
\begin{sphinxVerbatim}[commandchars=\\\{\}]
\PYG{n}{help}\PYG{p}{(}\PYG{n}{totdekomp}\PYG{o}{.}\PYG{n}{explain\PYGZus{}all}\PYG{p}{)}
\end{sphinxVerbatim}

\end{sphinxuseclass}\end{sphinxVerbatimInput}
\begin{sphinxVerbatimOutput}

\begin{sphinxuseclass}{cell_output}
\begin{sphinxVerbatim}[commandchars=\\\{\}]
Help on method explain\PYGZus{}all in module modeldekom:

explain\PYGZus{}all(pat=\PYGZsq{}\PYGZsq{}, stacked=True, kind=\PYGZsq{}bar\PYGZsq{}, top=0.9, title=\PYGZsq{}\PYGZsq{}, use=\PYGZsq{}level\PYGZsq{}, threshold=0.0, resample=\PYGZsq{}\PYGZsq{}, axvline=None) method of modeldekom.totdif instance
    Explains all
    
    Args:
        pat (TYPE, optional): DESCRIPTION. Defaults to \PYGZsq{}\PYGZsq{}.
        stacked (TYPE, optional): DESCRIPTION. Defaults to True.
        kind (TYPE, optional): DESCRIPTION. Defaults to \PYGZsq{}bar\PYGZsq{}.
        top (TYPE, optional): DESCRIPTION. Defaults to 0.9.
        title (TYPE, optional): DESCRIPTION. Defaults to \PYGZsq{}\PYGZsq{}.
        use (TYPE, optional): DESCRIPTION. Defaults to \PYGZsq{}level\PYGZsq{}.
        threshold (TYPE, optional): DESCRIPTION. Defaults to 0.0.
        resample (TYPE, optional): DESCRIPTION. Defaults to \PYGZsq{}\PYGZsq{}.
        axvline (TYPE, optional): DESCRIPTION. Defaults to None.
    
    Returns:
        None.
\end{sphinxVerbatim}

\end{sphinxuseclass}\end{sphinxVerbatimOutput}

\end{sphinxuseclass}

\subsection{Many variables}
\label{\detokenize{content/06_ModelAnalytics/AttributionSomeFeatures:many-variables}}
\sphinxAtStartPar
If many variables are passed to explain\_all then separate graphs will be created for each.

\begin{sphinxuseclass}{cell}\begin{sphinxVerbatimInput}

\begin{sphinxuseclass}{cell_input}
\begin{sphinxVerbatim}[commandchars=\\\{\}]
\PYG{n}{showvar} \PYG{o}{=} \PYG{l+s+s1}{\PYGZsq{}}\PYG{l+s+s1}{PAKNYGDPMKTPKN PAKCCEMISCO2CKN PAKCCEMISCO2OKN PAKCCEMISCO2GKN PAKGGREVTOTLCN}\PYG{l+s+s1}{\PYGZsq{}}

\PYG{n}{totdekomp}\PYG{o}{.}\PYG{n}{explain\PYGZus{}all}\PYG{p}{(}\PYG{n}{showvar}\PYG{p}{,}\PYG{n}{kind}\PYG{o}{=}\PYG{l+s+s1}{\PYGZsq{}}\PYG{l+s+s1}{area}\PYG{l+s+s1}{\PYGZsq{}}\PYG{p}{,}\PYG{n}{stacked}\PYG{o}{=}\PYG{k+kc}{True}\PYG{p}{)}\PYG{p}{;}    
\end{sphinxVerbatim}

\end{sphinxuseclass}\end{sphinxVerbatimInput}
\begin{sphinxVerbatimOutput}

\begin{sphinxuseclass}{cell_output}
\noindent\sphinxincludegraphics{{ef59e7d20117c4502d187cc87f4388081a241d3d7fbddd4aa451c8d5de712bf4}.png}

\end{sphinxuseclass}\end{sphinxVerbatimOutput}

\end{sphinxuseclass}

\subsection{Similarly the impacts on different variables for one year can be shown}
\label{\detokenize{content/06_ModelAnalytics/AttributionSomeFeatures:similarly-the-impacts-on-different-variables-for-one-year-can-be-shown}}
\begin{sphinxuseclass}{cell}\begin{sphinxVerbatimInput}

\begin{sphinxuseclass}{cell_input}
\begin{sphinxVerbatim}[commandchars=\\\{\}]
\PYG{n}{showvar} \PYG{o}{=} \PYG{l+s+s1}{\PYGZsq{}}\PYG{l+s+s1}{PAKNYGDPMKTPKN PAKNECONPRVTXN}\PYG{l+s+s1}{\PYGZsq{}}

\PYG{n}{totdekomp}\PYG{o}{.}\PYG{n}{explain\PYGZus{}per}\PYG{p}{(}\PYG{n}{showvar}\PYG{p}{,}\PYG{n}{per}\PYG{o}{=}\PYG{l+m+mi}{2023}\PYG{p}{,}\PYG{n}{ysize}\PYG{o}{=}\PYG{l+m+mi}{8}\PYG{p}{)}
\end{sphinxVerbatim}

\end{sphinxuseclass}\end{sphinxVerbatimInput}
\begin{sphinxVerbatimOutput}

\begin{sphinxuseclass}{cell_output}
\noindent\sphinxincludegraphics{{494a1aba2f02e6d227af96083462a66369600f6270b7520eed034464b0afd72c}.png}

\end{sphinxuseclass}\end{sphinxVerbatimOutput}

\end{sphinxuseclass}

\subsection{Or an interactive widgets can be generated}
\label{\detokenize{content/06_ModelAnalytics/AttributionSomeFeatures:or-an-interactive-widgets-can-be-generated}}
\sphinxAtStartPar
This allows the user to select the specific variable of interest and what to display:

\begin{sphinxadmonition}{note}{Note:}
\sphinxAtStartPar
If this is read in a manual the widget is not live.

\sphinxAtStartPar
In a notebook the selection widgets are live.
\end{sphinxadmonition}

\begin{sphinxuseclass}{cell}\begin{sphinxVerbatimInput}

\begin{sphinxuseclass}{cell_input}
\begin{sphinxVerbatim}[commandchars=\\\{\}]
\PYG{n}{display}\PYG{p}{(}\PYG{n}{mpak}\PYG{o}{.}\PYG{n}{get\PYGZus{}att\PYGZus{}gui}\PYG{p}{(}\PYG{n}{var}\PYG{o}{=}\PYG{l+s+s1}{\PYGZsq{}}\PYG{l+s+s1}{PAKGGREVTOTLCN}\PYG{l+s+s1}{\PYGZsq{}}\PYG{p}{,}\PYG{n}{ysize}\PYG{o}{=}\PYG{l+m+mi}{7}\PYG{p}{)}\PYG{p}{)}\PYG{p}{;}
\end{sphinxVerbatim}

\end{sphinxuseclass}\end{sphinxVerbatimInput}
\begin{sphinxVerbatimOutput}

\begin{sphinxuseclass}{cell_output}
\begin{sphinxVerbatim}[commandchars=\\\{\}]
interactive(children=(Dropdown(description=\PYGZsq{}Variable\PYGZsq{}, index=108, options=(\PYGZsq{}CHNEXR05\PYGZsq{}, \PYGZsq{}CHNPCEXN05\PYGZsq{}, \PYGZsq{}DEUEXR05…
\end{sphinxVerbatim}

\begin{sphinxVerbatim}[commandchars=\\\{\}]
None
\end{sphinxVerbatim}

\end{sphinxuseclass}\end{sphinxVerbatimOutput}

\end{sphinxuseclass}

\subsection{Attribution of the last year}
\label{\detokenize{content/06_ModelAnalytics/AttributionSomeFeatures:attribution-of-the-last-year}}
\begin{sphinxuseclass}{cell}\begin{sphinxVerbatimInput}

\begin{sphinxuseclass}{cell_input}
\begin{sphinxVerbatim}[commandchars=\\\{\}]
\PYG{n}{totdekomp}\PYG{o}{.}\PYG{n}{explain\PYGZus{}last}\PYG{p}{(}\PYG{n}{showvar}\PYG{p}{,}\PYG{n}{ysize}\PYG{o}{=}\PYG{l+m+mi}{8}\PYG{p}{)}
\end{sphinxVerbatim}

\end{sphinxuseclass}\end{sphinxVerbatimInput}
\begin{sphinxVerbatimOutput}

\begin{sphinxuseclass}{cell_output}
\noindent\sphinxincludegraphics{{73baa0bf8549848cb35c6d547b07816b17c807a1dfc4cfde2cca5836831cdc37}.png}

\end{sphinxuseclass}\end{sphinxVerbatimOutput}

\end{sphinxuseclass}

\subsection{Attribution of accumulated effects}
\label{\detokenize{content/06_ModelAnalytics/AttributionSomeFeatures:attribution-of-accumulated-effects}}
\begin{sphinxuseclass}{cell}\begin{sphinxVerbatimInput}

\begin{sphinxuseclass}{cell_input}
\begin{sphinxVerbatim}[commandchars=\\\{\}]
\PYG{n}{totdekomp}\PYG{o}{.}\PYG{n}{explain\PYGZus{}sum}\PYG{p}{(}\PYG{n}{showvar}\PYG{p}{,}\PYG{n}{ysize}\PYG{o}{=}\PYG{l+m+mi}{8}\PYG{p}{)}
\end{sphinxVerbatim}

\end{sphinxuseclass}\end{sphinxVerbatimInput}
\begin{sphinxVerbatimOutput}

\begin{sphinxuseclass}{cell_output}
\noindent\sphinxincludegraphics{{47dd1f7a2f7647bcdd870e7aa24803f4696dbdacf44626d313488e8c3bfee3af}.png}

\end{sphinxuseclass}\end{sphinxVerbatimOutput}

\end{sphinxuseclass}

\section{More advanced model attribution}
\label{\detokenize{content/06_ModelAnalytics/AttributionSomeFeatures:more-advanced-model-attribution}}
\sphinxAtStartPar
For some  simulations the number of changed exogenous variables can be large. Using a dictionary to contain the experiments allows us to create experiments where all variables for each country are analyzed, or each macro variable for all countries are analyzed.

\sphinxAtStartPar
Also it is possible to use aggregated sums.

\sphinxAtStartPar
If there are many experiments, data can be filtered in order to look only at the variables with an impact above a certain threshold.

\sphinxAtStartPar
The is also the possibility to anonymize the row and column names and to randomize the order of rows and/or columns.


\subsection{Grouping variables}
\label{\detokenize{content/06_ModelAnalytics/AttributionSomeFeatures:grouping-variables}}
\sphinxAtStartPar
If two experiments differ by many exogenous variables it can make sense to group the variables into experiments. This allows the user to
slice and dice the impact along different dimensions.

\sphinxAtStartPar
To illustrate this below you will find a model attribution where the impact of gas and coal tax is grouped together and the oil tax is in its own group.

\begin{sphinxuseclass}{cell}\begin{sphinxVerbatimInput}

\begin{sphinxuseclass}{cell_input}
\begin{sphinxVerbatim}[commandchars=\\\{\}]
\PYG{n}{experiments} \PYG{o}{=} \PYG{p}{\PYGZob{}}\PYG{l+s+s1}{\PYGZsq{}}\PYG{l+s+s1}{gas and coal}\PYG{l+s+s1}{\PYGZsq{}}\PYG{p}{:}\PYG{p}{[}\PYG{l+s+s1}{\PYGZsq{}}\PYG{l+s+s1}{PAKGGREVCO2CER}\PYG{l+s+s1}{\PYGZsq{}}\PYG{p}{,} \PYG{l+s+s1}{\PYGZsq{}}\PYG{l+s+s1}{PAKGGREVCO2GER}\PYG{l+s+s1}{\PYGZsq{}}\PYG{p}{]}\PYG{p}{,}\PYG{l+s+s1}{\PYGZsq{}}\PYG{l+s+s1}{Oil}\PYG{l+s+s1}{\PYGZsq{}}\PYG{p}{:}\PYG{p}{[}\PYG{l+s+s1}{\PYGZsq{}}\PYG{l+s+s1}{PAKGGREVCO2OER}\PYG{l+s+s1}{\PYGZsq{}}\PYG{p}{]}\PYG{p}{\PYGZcb{}}
\PYG{n}{totdekomp\PYGZus{}group} \PYG{o}{=} \PYG{n}{mpak}\PYG{o}{.}\PYG{n}{totdif}\PYG{p}{(}\PYG{n}{experiments} \PYG{o}{=} \PYG{n}{experiments}\PYG{p}{)} \PYG{c+c1}{\PYGZsh{} Calculate the total derivative½s of all equations in the model.}
\end{sphinxVerbatim}

\end{sphinxuseclass}\end{sphinxVerbatimInput}
\begin{sphinxVerbatimOutput}

\begin{sphinxuseclass}{cell_output}
\begin{sphinxVerbatim}[commandchars=\\\{\}]
Total dekomp took       :         2.250 Seconds
\end{sphinxVerbatim}

\end{sphinxuseclass}\end{sphinxVerbatimOutput}

\end{sphinxuseclass}
\begin{sphinxuseclass}{cell}\begin{sphinxVerbatimInput}

\begin{sphinxuseclass}{cell_input}
\begin{sphinxVerbatim}[commandchars=\\\{\}]
\PYG{n}{showvar} \PYG{o}{=} \PYG{l+s+s1}{\PYGZsq{}}\PYG{l+s+s1}{PAKNYGDPMKTPKN}\PYG{l+s+s1}{\PYGZsq{}}
\PYG{n}{totdekomp\PYGZus{}group}\PYG{o}{.}\PYG{n}{explain\PYGZus{}all}\PYG{p}{(}\PYG{n}{showvar}\PYG{p}{,}\PYG{n}{kind}\PYG{o}{=}\PYG{l+s+s1}{\PYGZsq{}}\PYG{l+s+s1}{area}\PYG{l+s+s1}{\PYGZsq{}}\PYG{p}{,}\PYG{n}{stacked}\PYG{o}{=}\PYG{k+kc}{True}\PYG{p}{)}\PYG{p}{;}    
\end{sphinxVerbatim}

\end{sphinxuseclass}\end{sphinxVerbatimInput}
\begin{sphinxVerbatimOutput}

\begin{sphinxuseclass}{cell_output}
\noindent\sphinxincludegraphics{{1293a8fa1a10b1492cce34db04672d46f21a13f9a5000d06de1e896d161d95af}.png}

\end{sphinxuseclass}\end{sphinxVerbatimOutput}

\end{sphinxuseclass}

\subsection{Single equation attribution chart}
\label{\detokenize{content/06_ModelAnalytics/AttributionSomeFeatures:single-equation-attribution-chart}}
\sphinxAtStartPar
The results can be visualized in different ways.

\begin{sphinxuseclass}{cell}\begin{sphinxVerbatimInput}

\begin{sphinxuseclass}{cell_input}
\begin{sphinxVerbatim}[commandchars=\\\{\}]
\PYG{n}{mpak}\PYG{o}{.}\PYG{n}{dekomp\PYGZus{}plot\PYGZus{}per}\PYG{p}{(}\PYG{l+s+s1}{\PYGZsq{}}\PYG{l+s+s1}{PAKNYGDPMKTPKN}\PYG{l+s+s1}{\PYGZsq{}}\PYG{p}{,}\PYG{n}{per}\PYG{o}{=}\PYG{l+m+mi}{2023}\PYG{p}{,}\PYG{n}{pct}\PYG{o}{=}\PYG{l+m+mi}{0}\PYG{p}{,}\PYG{n}{rename}\PYG{o}{=}\PYG{l+m+mi}{1}\PYG{p}{,}\PYG{n}{sort}\PYG{o}{=}\PYG{l+m+mi}{1}\PYG{p}{,}\PYG{n}{threshold} \PYG{o}{=}\PYG{l+m+mi}{200000}\PYG{p}{,}\PYG{n}{ysize}\PYG{o}{=}\PYG{l+m+mi}{7}\PYG{p}{)}
\end{sphinxVerbatim}

\end{sphinxuseclass}\end{sphinxVerbatimInput}
\begin{sphinxVerbatimOutput}

\begin{sphinxuseclass}{cell_output}
\noindent\sphinxincludegraphics{{00b1d2ade7a49e99757d605b4efc1a6427c314a857560f01b22a48f4dc133899}.png}

\end{sphinxuseclass}\end{sphinxVerbatimOutput}

\end{sphinxuseclass}

\subsection{Attribution when comparing time frames}
\label{\detokenize{content/06_ModelAnalytics/AttributionSomeFeatures:attribution-when-comparing-time-frames}}
\sphinxAtStartPar
In this case we seek to find out which variables explains the development from year to year. This is done only for the .lastdf dateframe.

\begin{sphinxuseclass}{cell}\begin{sphinxVerbatimInput}

\begin{sphinxuseclass}{cell_input}
\begin{sphinxVerbatim}[commandchars=\\\{\}]
\PYG{k}{with} \PYG{n}{mpak}\PYG{o}{.}\PYG{n}{set\PYGZus{}smpl}\PYG{p}{(}\PYG{l+m+mi}{2020}\PYG{p}{,}\PYG{l+m+mi}{2024}\PYG{p}{)}\PYG{p}{:}
    \PYG{n}{mpak}\PYG{p}{[}\PYG{l+s+s1}{\PYGZsq{}}\PYG{l+s+s1}{PAKNYGDPMKTPKN}\PYG{l+s+s1}{\PYGZsq{}}\PYG{p}{]}\PYG{o}{.}\PYG{n}{dekomp}\PYG{p}{(}\PYG{n}{time\PYGZus{}att}\PYG{o}{=}\PYG{k+kc}{True}\PYG{p}{)}
\end{sphinxVerbatim}

\end{sphinxuseclass}\end{sphinxVerbatimInput}
\begin{sphinxVerbatimOutput}

\begin{sphinxuseclass}{cell_output}
\begin{sphinxVerbatim}[commandchars=\\\{\}]
Formula        : FRML \PYGZlt{}IDENT\PYGZgt{} PAKNYGDPMKTPKN = PAKNECONPRVTKN+PAKNECONGOVTKN+PAKNEGDIFTOTKN+PAKNEGDISTKBKN+PAKNEEXPGNFSKN\PYGZhy{}PAKNEIMPGNFSKN+PAKNYGDPDISCKN+PAKADAP*PAKDISPREPKN \PYGZdl{} 

                      2020        2021        2022        2023        2024
Variable   lag                                                            
t\PYGZhy{}1        0   25760579.27 26471666.00 26767616.83 26891916.75 27090476.06
t          0   26471666.00 26767616.83 26891916.75 27090476.06 27455212.23
Difference 0     711086.73   295950.83   124299.92   198559.31   364736.17
Percent    0          2.76        1.12        0.46        0.74        1.35

 Contributions to differende for  PAKNYGDPMKTPKN
                         2020       2021       2022       2023       2024
Variable       lag                                                       
PAKNECONPRVTKN 0    416507.93  218812.97  107677.34  182518.76  332666.98
PAKNECONGOVTKN 0    348503.70   56821.70   \PYGZhy{}8433.94   15765.70   61617.96
PAKNEGDIFTOTKN 0    224846.18   63575.21   21086.24    8250.65   10770.78
PAKNEGDISTKBKN 0      9896.74   10138.31   10385.77   10639.25   10898.93
PAKNEEXPGNFSKN 0     95590.65  108489.15  115927.86  120153.92  122600.63
PAKNEIMPGNFSKN 0   \PYGZhy{}385555.01 \PYGZhy{}163214.66 \PYGZhy{}123703.85 \PYGZhy{}140162.72 \PYGZhy{}175246.89
PAKNYGDPDISCKN 0      1296.36    1328.02    1360.44    1393.63    1427.65
PAKADAP        0        \PYGZhy{}0.03      \PYGZhy{}0.02      \PYGZhy{}0.01      \PYGZhy{}0.02      \PYGZhy{}0.02
PAKDISPREPKN   0        \PYGZhy{}0.03      \PYGZhy{}0.02      \PYGZhy{}0.01      \PYGZhy{}0.02      \PYGZhy{}0.02

 Share of contributions to differende for  PAKNYGDPMKTPKN
                          2020        2021        2022        2023        2024
Variable       lag                                                            
PAKNECONPRVTKN 0           59\PYGZpc{}         74\PYGZpc{}         87\PYGZpc{}         92\PYGZpc{}         91\PYGZpc{}
PAKNEEXPGNFSKN 0           13\PYGZpc{}         37\PYGZpc{}         93\PYGZpc{}         61\PYGZpc{}         34\PYGZpc{}
PAKNECONGOVTKN 0           49\PYGZpc{}         19\PYGZpc{}         \PYGZhy{}7\PYGZpc{}          8\PYGZpc{}         17\PYGZpc{}
PAKNEGDISTKBKN 0            1\PYGZpc{}          3\PYGZpc{}          8\PYGZpc{}          5\PYGZpc{}          3\PYGZpc{}
PAKNEGDIFTOTKN 0           32\PYGZpc{}         21\PYGZpc{}         17\PYGZpc{}          4\PYGZpc{}          3\PYGZpc{}
PAKNYGDPDISCKN 0            0\PYGZpc{}          0\PYGZpc{}          1\PYGZpc{}          1\PYGZpc{}          0\PYGZpc{}
PAKADAP        0           \PYGZhy{}0\PYGZpc{}         \PYGZhy{}0\PYGZpc{}         \PYGZhy{}0\PYGZpc{}         \PYGZhy{}0\PYGZpc{}         \PYGZhy{}0\PYGZpc{}
PAKDISPREPKN   0           \PYGZhy{}0\PYGZpc{}         \PYGZhy{}0\PYGZpc{}         \PYGZhy{}0\PYGZpc{}         \PYGZhy{}0\PYGZpc{}         \PYGZhy{}0\PYGZpc{}
PAKNEIMPGNFSKN 0          \PYGZhy{}54\PYGZpc{}        \PYGZhy{}55\PYGZpc{}       \PYGZhy{}100\PYGZpc{}        \PYGZhy{}71\PYGZpc{}        \PYGZhy{}48\PYGZpc{}
Total          0          100\PYGZpc{}        100\PYGZpc{}        100\PYGZpc{}        100\PYGZpc{}        100\PYGZpc{}
Residual       0           \PYGZhy{}0\PYGZpc{}         \PYGZhy{}0\PYGZpc{}         \PYGZhy{}0\PYGZpc{}         \PYGZhy{}0\PYGZpc{}         \PYGZhy{}0\PYGZpc{}

 Difference in growth rate PAKNYGDPMKTPKN
                      2020        2021        2022        2023        2024
Variable   lag                                                            
t\PYGZhy{}1        0          4.7\PYGZpc{}        2.8\PYGZpc{}        1.1\PYGZpc{}        0.5\PYGZpc{}        0.7\PYGZpc{}
t          0          2.8\PYGZpc{}        1.1\PYGZpc{}        0.5\PYGZpc{}        0.7\PYGZpc{}        1.3\PYGZpc{}
Difference 0         \PYGZhy{}1.9\PYGZpc{}       \PYGZhy{}1.6\PYGZpc{}       \PYGZhy{}0.7\PYGZpc{}        0.3\PYGZpc{}        0.6\PYGZpc{}
None

 Contribution to growth rate PAKNYGDPMKTPKN
                          2020        2021        2022        2023        2024
Variable       lag                                                            
PAKNECONPRVTKN 0          1.6\PYGZpc{}        0.8\PYGZpc{}        0.4\PYGZpc{}        0.7\PYGZpc{}        1.2\PYGZpc{}
PAKNECONGOVTKN 0          1.4\PYGZpc{}        0.2\PYGZpc{}       \PYGZhy{}0.0\PYGZpc{}        0.1\PYGZpc{}        0.2\PYGZpc{}
PAKNEGDIFTOTKN 0          0.9\PYGZpc{}        0.2\PYGZpc{}        0.1\PYGZpc{}        0.0\PYGZpc{}        0.0\PYGZpc{}
PAKNEGDISTKBKN 0          0.0\PYGZpc{}        0.0\PYGZpc{}        0.0\PYGZpc{}        0.0\PYGZpc{}        0.0\PYGZpc{}
PAKNEEXPGNFSKN 0          0.4\PYGZpc{}        0.4\PYGZpc{}        0.4\PYGZpc{}        0.4\PYGZpc{}        0.5\PYGZpc{}
PAKNEIMPGNFSKN 0         \PYGZhy{}1.5\PYGZpc{}       \PYGZhy{}0.6\PYGZpc{}       \PYGZhy{}0.5\PYGZpc{}       \PYGZhy{}0.5\PYGZpc{}       \PYGZhy{}0.6\PYGZpc{}
PAKNYGDPDISCKN 0          0.0\PYGZpc{}        0.0\PYGZpc{}        0.0\PYGZpc{}        0.0\PYGZpc{}        0.0\PYGZpc{}
PAKADAP        0         \PYGZhy{}0.0\PYGZpc{}       \PYGZhy{}0.0\PYGZpc{}       \PYGZhy{}0.0\PYGZpc{}       \PYGZhy{}0.0\PYGZpc{}       \PYGZhy{}0.0\PYGZpc{}
PAKDISPREPKN   0         \PYGZhy{}0.0\PYGZpc{}       \PYGZhy{}0.0\PYGZpc{}       \PYGZhy{}0.0\PYGZpc{}       \PYGZhy{}0.0\PYGZpc{}       \PYGZhy{}0.0\PYGZpc{}
Total          0          2.8\PYGZpc{}        1.1\PYGZpc{}        0.5\PYGZpc{}        0.7\PYGZpc{}        1.3\PYGZpc{}
Residual       0          4.7\PYGZpc{}        2.8\PYGZpc{}        1.1\PYGZpc{}        0.5\PYGZpc{}        0.7\PYGZpc{}
\end{sphinxVerbatim}

\end{sphinxuseclass}\end{sphinxVerbatimOutput}

\end{sphinxuseclass}
\begin{sphinxuseclass}{cell}\begin{sphinxVerbatimInput}

\begin{sphinxuseclass}{cell_input}
\begin{sphinxVerbatim}[commandchars=\\\{\}]
\PYG{n}{mpak}\PYG{o}{.}\PYG{n}{dekomp\PYGZus{}plot}\PYG{p}{(}\PYG{l+s+s1}{\PYGZsq{}}\PYG{l+s+s1}{PAKNYGDPMKTPKN}\PYG{l+s+s1}{\PYGZsq{}}\PYG{p}{,}\PYG{n}{pct}\PYG{o}{=}\PYG{l+m+mi}{0}\PYG{p}{,}\PYG{n}{rename}\PYG{o}{=}\PYG{l+m+mi}{1}\PYG{p}{,}\PYG{n}{sort}\PYG{o}{=}\PYG{l+m+mi}{1}\PYG{p}{,}\PYG{n}{threshold} \PYG{o}{=}\PYG{l+m+mi}{0}\PYG{p}{,}\PYG{n}{time\PYGZus{}att} \PYG{o}{=} \PYG{k+kc}{True}\PYG{p}{)}\PYG{p}{;}
\end{sphinxVerbatim}

\end{sphinxuseclass}\end{sphinxVerbatimInput}
\begin{sphinxVerbatimOutput}

\begin{sphinxuseclass}{cell_output}
\noindent\sphinxincludegraphics{{57a71b9cc93c8ce0f7108f4c0eec17b7bb017ffc28bda32c0a6445899795ac3c}.png}

\end{sphinxuseclass}\end{sphinxVerbatimOutput}

\end{sphinxuseclass}

\subsection{Visualizing attribution in dependency graphs}
\label{\detokenize{content/06_ModelAnalytics/AttributionSomeFeatures:visualizing-attribution-in-dependency-graphs}}
\sphinxAtStartPar
The logical graph of the model can be used to show the upstream and downstream variable for a specific variable. More on this \DUrole{xref,myst}{here}
When drawing the logical graph for a variable the model attribution will be used to guide the thickness of edges between nodes (variables). This enables a visual impression of which
variables drives the impact.

\begin{sphinxadmonition}{note}{Note:}
\sphinxAtStartPar
If png == 0 the graph below will be rendered in SVG format. This enables tooltips with additional information when the mouse hovers
over an edge or an node.

\sphinxAtStartPar
Unfortunately svg can’t be displayed in the manual, so png has to be True for the manual. In a live jupyter notebook set latex=False. This will
enable svg format.
\end{sphinxadmonition}

\begin{sphinxuseclass}{cell}\begin{sphinxVerbatimInput}

\begin{sphinxuseclass}{cell_input}
\begin{sphinxVerbatim}[commandchars=\\\{\}]
\PYG{c+c1}{\PYGZsh{}mpak[\PYGZsq{}PAKNYGDPMKTPKN PAKNECONPRVTKN\PYGZsq{}].draw(up=3,down=0,png=latex,filter=20) \PYGZsh{} For book}
\PYG{n}{mpak}\PYG{p}{[}\PYG{l+s+s1}{\PYGZsq{}}\PYG{l+s+s1}{PAKNYGDPMKTPKN}\PYG{l+s+s1}{\PYGZsq{}}\PYG{p}{]}\PYG{o}{.}\PYG{n}{draw}\PYG{p}{(}\PYG{n}{up}\PYG{o}{=}\PYG{l+m+mi}{3}\PYG{p}{,}\PYG{n}{down}\PYG{o}{=}\PYG{l+m+mi}{1}\PYG{p}{,}\PYG{n}{png}\PYG{o}{=}\PYG{n}{latex}\PYG{p}{,}\PYG{n+nb}{filter}\PYG{o}{=}\PYG{l+m+mi}{400}\PYG{p}{,}\PYG{n}{svg}\PYG{o}{=}\PYG{k+kc}{True}\PYG{p}{,}\PYG{n}{size}\PYG{o}{=}\PYG{p}{(}\PYG{l+m+mi}{8}\PYG{p}{,}\PYG{l+m+mi}{40}\PYG{p}{)}\PYG{p}{)}\PYG{c+c1}{\PYGZsh{}3 for interactice}
\end{sphinxVerbatim}

\end{sphinxuseclass}\end{sphinxVerbatimInput}
\begin{sphinxVerbatimOutput}

\begin{sphinxuseclass}{cell_output}
\noindent\sphinxincludegraphics{{09878dac1a0d94df35fce059390a9c3fa174b5b44a9ca1ef5343c041e6fb9638}.png}

\end{sphinxuseclass}\end{sphinxVerbatimOutput}

\end{sphinxuseclass}
\begin{sphinxuseclass}{cell}\begin{sphinxVerbatimInput}

\begin{sphinxuseclass}{cell_input}
\begin{sphinxVerbatim}[commandchars=\\\{\}]
\PYG{n}{mpak}\PYG{p}{[}\PYG{l+s+s1}{\PYGZsq{}}\PYG{l+s+s1}{PAKNYGDPMKTPKN PAKNECONPRVTKN}\PYG{l+s+s1}{\PYGZsq{}}\PYG{p}{]}\PYG{o}{.}\PYG{n}{draw}\PYG{p}{(}\PYG{n}{up}\PYG{o}{=}\PYG{l+m+mi}{1}\PYG{p}{,}\PYG{n}{down}\PYG{o}{=}\PYG{l+m+mi}{1}\PYG{p}{,}\PYG{n}{png}\PYG{o}{=}\PYG{n}{latex}\PYG{p}{)}  \PYG{c+c1}{\PYGZsh{} diagram all direct dependencies }
\end{sphinxVerbatim}

\end{sphinxuseclass}\end{sphinxVerbatimInput}
\begin{sphinxVerbatimOutput}

\begin{sphinxuseclass}{cell_output}
\noindent\sphinxincludegraphics{{a8a6a95915f8e5563c3990a77472a236c084d88309a8c3e13ee63bca9165669b}.png}

\noindent\sphinxincludegraphics{{e4c3ce54bbdf770cdbd4741c7765726b4701a1c02658982ac137e9dac2af342a}.png}

\end{sphinxuseclass}\end{sphinxVerbatimOutput}

\end{sphinxuseclass}

\subsection{The attribution can be filtered and more levels can be displayed.}
\label{\detokenize{content/06_ModelAnalytics/AttributionSomeFeatures:the-attribution-can-be-filtered-and-more-levels-can-be-displayed}}
\begin{sphinxuseclass}{cell}\begin{sphinxVerbatimInput}

\begin{sphinxuseclass}{cell_input}
\begin{sphinxVerbatim}[commandchars=\\\{\}]
\PYG{n}{mpak}\PYG{p}{[}\PYG{l+s+s1}{\PYGZsq{}}\PYG{l+s+s1}{PAKNYGDPMKTPKN}\PYG{l+s+s1}{\PYGZsq{}}\PYG{p}{]}\PYG{o}{.}\PYG{n}{draw}\PYG{p}{(}\PYG{n}{up}\PYG{o}{=}\PYG{l+m+mi}{2}\PYG{p}{,}\PYG{n}{down}\PYG{o}{=}\PYG{l+m+mi}{1}\PYG{p}{,}\PYG{n}{png}\PYG{o}{=}\PYG{n}{latex}\PYG{p}{,}\PYG{n+nb}{filter}\PYG{o}{=}\PYG{l+m+mi}{20}\PYG{p}{)} 
\end{sphinxVerbatim}

\end{sphinxuseclass}\end{sphinxVerbatimInput}
\begin{sphinxVerbatimOutput}

\begin{sphinxuseclass}{cell_output}
\noindent\sphinxincludegraphics{{161e5c6979a7870bcda024de402073c78146a3e7f2918e943e4f09dc0963aaa1}.png}

\end{sphinxuseclass}\end{sphinxVerbatimOutput}

\end{sphinxuseclass}

\subsection{Or it can be used in a dashboard (not avaiable in the offline manual)}
\label{\detokenize{content/06_ModelAnalytics/AttributionSomeFeatures:or-it-can-be-used-in-a-dashboard-not-avaiable-in-the-offline-manual}}
\begin{sphinxuseclass}{cell}\begin{sphinxVerbatimInput}

\begin{sphinxuseclass}{cell_input}
\begin{sphinxVerbatim}[commandchars=\\\{\}]
\PYG{k}{try}\PYG{p}{:}
    \PYG{n}{mpak}\PYG{o}{.}\PYG{n}{modeldash}\PYG{p}{(}\PYG{l+s+s1}{\PYGZsq{}}\PYG{l+s+s1}{PAKNYGDPMKTPKN}\PYG{l+s+s1}{\PYGZsq{}}\PYG{p}{,}\PYG{n}{jupyter}\PYG{o}{=}\PYG{l+m+mi}{1}\PYG{p}{,}\PYG{n}{inline}\PYG{o}{=}\PYG{k+kc}{False}\PYG{p}{)}
\PYG{k}{except}\PYG{p}{:} 
    \PYG{n+nb}{print}\PYG{p}{(}\PYG{l+s+s1}{\PYGZsq{}}\PYG{l+s+s1}{No Dashboard installed}\PYG{l+s+s1}{\PYGZsq{}}\PYG{p}{)}
\end{sphinxVerbatim}

\end{sphinxuseclass}\end{sphinxVerbatimInput}
\begin{sphinxVerbatimOutput}

\begin{sphinxuseclass}{cell_output}
\begin{sphinxVerbatim}[commandchars=\\\{\}]
apprun
Dash is running on http://127.0.0.1:5001/
\end{sphinxVerbatim}

\begin{sphinxVerbatim}[commandchars=\\\{\}]
Dash app running on http://127.0.0.1:5001/
\end{sphinxVerbatim}

\end{sphinxuseclass}\end{sphinxVerbatimOutput}

\end{sphinxuseclass}

\section{.tracepre() advanced}
\label{\detokenize{content/06_ModelAnalytics/AttributionSomeFeatures:tracepre-advanced}}\begin{itemize}
\item {} 
\sphinxAtStartPar
\sphinxcode{\sphinxupquote{fokus2all=True}} will include a table of values for each variable

\item {} 
\sphinxAtStartPar
\sphinxcode{\sphinxupquote{attshow = True}} will include a table of attributions for each variable

\item {} 
\sphinxAtStartPar
\sphinxcode{\sphinxupquote{growthshow = True}} will include a table of growth for each variable

\item {} 
\sphinxAtStartPar
\sphinxcode{\sphinxupquote{HR = True}} will reorient the dependency graph

\item {} 
\sphinxAtStartPar
‘up = ` will determine how many levels of parents to include

\item {} 
\sphinxAtStartPar
\sphinxcode{\sphinxupquote{browser = True}} will open a browser with the resulting dependency graph \sphinxhyphen{} useful for zooming on a big graph

\end{itemize}

\begin{sphinxuseclass}{cell}\begin{sphinxVerbatimInput}

\begin{sphinxuseclass}{cell_input}
\begin{sphinxVerbatim}[commandchars=\\\{\}]
\PYG{k}{with} \PYG{n}{mpak}\PYG{o}{.}\PYG{n}{set\PYGZus{}smpl}\PYG{p}{(}\PYG{l+m+mi}{2020}\PYG{p}{,}\PYG{l+m+mi}{2023}\PYG{p}{)}\PYG{p}{:}
    \PYG{n}{mpak}\PYG{o}{.}\PYG{n}{PAKNECONPRVTKN}\PYG{o}{.}\PYG{n}{tracepre}\PYG{p}{(}\PYG{p}{)}
\end{sphinxVerbatim}

\end{sphinxuseclass}\end{sphinxVerbatimInput}
\begin{sphinxVerbatimOutput}

\begin{sphinxuseclass}{cell_output}
\begin{sphinxVerbatim}[commandchars=\\\{\}]
\PYGZlt{}IPython.core.display.SVG object\PYGZgt{}
\end{sphinxVerbatim}

\end{sphinxuseclass}\end{sphinxVerbatimOutput}

\end{sphinxuseclass}
\begin{sphinxuseclass}{cell}\begin{sphinxVerbatimInput}

\begin{sphinxuseclass}{cell_input}
\begin{sphinxVerbatim}[commandchars=\\\{\}]
\PYG{k}{with} \PYG{n}{mpak}\PYG{o}{.}\PYG{n}{set\PYGZus{}smpl}\PYG{p}{(}\PYG{l+m+mi}{2020}\PYG{p}{,}\PYG{l+m+mi}{2023}\PYG{p}{)}\PYG{p}{:}
    \PYG{n}{mpak}\PYG{o}{.}\PYG{n}{PAKNECONPRVTKN}\PYG{o}{.}\PYG{n}{tracepre}\PYG{p}{(}\PYG{n}{fokus2all}\PYG{o}{=}\PYG{k+kc}{True}\PYG{p}{,}\PYG{n}{HR}\PYG{o}{=}\PYG{l+m+mi}{0}\PYG{p}{,}\PYG{n+nb}{filter}\PYG{o}{=}\PYG{l+m+mi}{20}\PYG{p}{,}
                                 \PYG{n}{attshow}\PYG{o}{=}\PYG{l+m+mi}{1}\PYG{p}{,}\PYG{n}{up}\PYG{o}{=}\PYG{l+m+mi}{2}\PYG{p}{,}\PYG{n}{browser}\PYG{o}{=}\PYG{l+m+mi}{1}\PYG{p}{,}\PYG{n}{growthshow}\PYG{o}{=}\PYG{l+m+mi}{0}\PYG{p}{)}
\end{sphinxVerbatim}

\end{sphinxuseclass}\end{sphinxVerbatimInput}
\begin{sphinxVerbatimOutput}

\begin{sphinxuseclass}{cell_output}
\begin{sphinxVerbatim}[commandchars=\\\{\}]
\PYGZlt{}IPython.core.display.SVG object\PYGZgt{}
\end{sphinxVerbatim}

\end{sphinxuseclass}\end{sphinxVerbatimOutput}

\end{sphinxuseclass}
\sphinxstepscope


\part{Technical how tos}

\sphinxstepscope

\begin{sphinxuseclass}{cell}\begin{sphinxVerbatimInput}

\begin{sphinxuseclass}{cell_input}
\begin{sphinxVerbatim}[commandchars=\\\{\}]
\PYG{o}{\PYGZpc{}}\PYG{k}{matplotlib} inline
\end{sphinxVerbatim}

\end{sphinxuseclass}\end{sphinxVerbatimInput}

\end{sphinxuseclass}

\chapter{Getting Help}
\label{\detokenize{content/07_MoreFeatures/GettingHelp:getting-help}}\label{\detokenize{content/07_MoreFeatures/GettingHelp::doc}}
\sphinxAtStartPar
mpak.fix?
mpak.fix??

\sphinxstepscope


\part{References}

\sphinxstepscope

\begin{sphinxthebibliography}{1}
\bibitem[1]{content/99_BackMatter/References:id2}
\sphinxAtStartPar
Ron Berndsen. Causal ordering in economic models. \sphinxstyleemphasis{Decision Support Systems}, 1995. ISBN: 0167\sphinxhyphen{}9236. \sphinxhref{https://doi.org/10.1016/0167-9236(94)00034-P}{doi:10.1016/0167\sphinxhyphen{}9236(94)00034\sphinxhyphen{}P}.
\bibitem[2]{content/99_BackMatter/References:id17}
\sphinxAtStartPar
Olivier Blanchard. On the future of Macroeconomic models. \sphinxstyleemphasis{Oxford Review of Economic Policy}, 34(1\sphinxhyphen{}2):43–54, 2018. URL: \sphinxurl{https://academic.oup.com/oxrep/article/34/1-2/43/4781808}, \sphinxhref{https://doi.org/https://doi.org/10.1093/oxrep/grx045}{doi:https://doi.org/10.1093/oxrep/grx045}.
\bibitem[3]{content/99_BackMatter/References:id27}
\sphinxAtStartPar
Andrew Burns, Benoit Campagne, Charl Jooste, David Stephan, and Thi Thanh Bui. \sphinxstyleemphasis{The World Bank Macro\sphinxhyphen{}Fiscal Model Technical Description}. Number 8965 in Policy Research Working Papers. World Bank, Washington DC., 2019. URL: \sphinxurl{https://openknowledge.worldbank.org/handle/10986/32217}.
\bibitem[4]{content/99_BackMatter/References:id14}
\sphinxAtStartPar
Andrew Burns, Charl Jooste, and Gregor Schwerhoff. \sphinxstyleemphasis{Climate Modeling for Macroeconomic Policy : A Case Study for Pakistan}. Number 9780 in Policy Research Working Papers. World Bank, Washington, DC, 2021. URL: \sphinxurl{https://openknowledge.worldbank.org/bitstream/handle/10986/36307/Climate-Modeling-for-Macroeconomic-Policy-A-Case-Study-for-Pakistan.pdf?sequence=1\&isAllowed=y}.
\bibitem[5]{content/99_BackMatter/References:id4}
\sphinxAtStartPar
K.C. Kogiku. \sphinxstyleemphasis{An Introduction to Macroeconomic Models}. McGwaw\sphinxhyphen{}Hill, 1968. URL: \sphinxurl{https://books.google.de/books?id=jp4LzQEACAAJ}.
\bibitem[6]{content/99_BackMatter/References:id16}
\sphinxAtStartPar
M. R. Wickens and T. S. Breusch. Dynamic Specification, the Long\sphinxhyphen{}Run and the Estimation of Transformed Regression Models. \sphinxstyleemphasis{The Economic Journal}, 98:189–205, April 1988.
\end{sphinxthebibliography}







\renewcommand{\indexname}{Index}
\printindex
\end{document}